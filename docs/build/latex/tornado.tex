%% Generated by Sphinx.
\def\sphinxdocclass{report}
\documentclass[letterpaper,10pt,english]{sphinxmanual}
\ifdefined\pdfpxdimen
   \let\sphinxpxdimen\pdfpxdimen\else\newdimen\sphinxpxdimen
\fi \sphinxpxdimen=.75bp\relax

\PassOptionsToPackage{warn}{textcomp}
\usepackage[utf8]{inputenc}
\ifdefined\DeclareUnicodeCharacter
% support both utf8 and utf8x syntaxes
  \ifdefined\DeclareUnicodeCharacterAsOptional
    \def\sphinxDUC#1{\DeclareUnicodeCharacter{"#1}}
  \else
    \let\sphinxDUC\DeclareUnicodeCharacter
  \fi
  \sphinxDUC{00A0}{\nobreakspace}
  \sphinxDUC{2500}{\sphinxunichar{2500}}
  \sphinxDUC{2502}{\sphinxunichar{2502}}
  \sphinxDUC{2514}{\sphinxunichar{2514}}
  \sphinxDUC{251C}{\sphinxunichar{251C}}
  \sphinxDUC{2572}{\textbackslash}
\fi
\usepackage{cmap}
\usepackage[T1]{fontenc}
\usepackage{amsmath,amssymb,amstext}
\usepackage{babel}



\usepackage{times}
\expandafter\ifx\csname T@LGR\endcsname\relax
\else
% LGR was declared as font encoding
  \substitutefont{LGR}{\rmdefault}{cmr}
  \substitutefont{LGR}{\sfdefault}{cmss}
  \substitutefont{LGR}{\ttdefault}{cmtt}
\fi
\expandafter\ifx\csname T@X2\endcsname\relax
  \expandafter\ifx\csname T@T2A\endcsname\relax
  \else
  % T2A was declared as font encoding
    \substitutefont{T2A}{\rmdefault}{cmr}
    \substitutefont{T2A}{\sfdefault}{cmss}
    \substitutefont{T2A}{\ttdefault}{cmtt}
  \fi
\else
% X2 was declared as font encoding
  \substitutefont{X2}{\rmdefault}{cmr}
  \substitutefont{X2}{\sfdefault}{cmss}
  \substitutefont{X2}{\ttdefault}{cmtt}
\fi


\usepackage[Bjarne]{fncychap}
\usepackage{sphinx}

\fvset{fontsize=\small}
\usepackage{geometry}

% Include hyperref last.
\usepackage{hyperref}
% Fix anchor placement for figures with captions.
\usepackage{hypcap}% it must be loaded after hyperref.
% Set up styles of URL: it should be placed after hyperref.
\urlstyle{same}

\usepackage{sphinxmessages}




\title{Tornado Documentation}
\date{Apr 02, 2019}
\release{6.1.dev1}
\author{The Tornado Authors}
\newcommand{\sphinxlogo}{\vbox{}}
\renewcommand{\releasename}{Release}
\makeindex
\begin{document}

\pagestyle{empty}
\sphinxmaketitle
\pagestyle{plain}
\sphinxtableofcontents
\pagestyle{normal}
\phantomsection\label{\detokenize{index::doc}}


\sphinxhref{http://www.tornadoweb.org}{Tornado} is a Python web framework and
asynchronous networking library, originally developed at \sphinxhref{http://friendfeed.com}{FriendFeed}.  By using non-blocking network I/O, Tornado
can scale to tens of thousands of open connections, making it ideal for
\sphinxhref{http://en.wikipedia.org/wiki/Push\_technology\#Long\_polling}{long polling},
\sphinxhref{http://en.wikipedia.org/wiki/WebSocket}{WebSockets}, and other
applications that require a long-lived connection to each user.


\chapter{Quick links}
\label{\detokenize{index:quick-links}}\begin{itemize}
\item {} 
Current version: 6.1.dev1 (\sphinxhref{https://pypi.python.org/pypi/tornado}{download from PyPI}, {\hyperref[\detokenize{releases::doc}]{\sphinxcrossref{\DUrole{doc}{release notes}}}})

\item {} 
\sphinxhref{https://github.com/tornadoweb/tornado}{Source (GitHub)}

\item {} 
Mailing lists: \sphinxhref{http://groups.google.com/group/python-tornado}{discussion} and \sphinxhref{http://groups.google.com/group/python-tornado-announce}{announcements}

\item {} 
\sphinxhref{http://stackoverflow.com/questions/tagged/tornado}{Stack Overflow}

\item {} 
\sphinxhref{https://github.com/tornadoweb/tornado/wiki/Links}{Wiki}

\end{itemize}


\chapter{Hello, world}
\label{\detokenize{index:hello-world}}
Here is a simple “Hello, world” example web app for Tornado:

\begin{sphinxVerbatim}[commandchars=\\\{\}]
\PYG{k+kn}{import} \PYG{n+nn}{tornado}\PYG{n+nn}{.}\PYG{n+nn}{ioloop}
\PYG{k+kn}{import} \PYG{n+nn}{tornado}\PYG{n+nn}{.}\PYG{n+nn}{web}

\PYG{k}{class} \PYG{n+nc}{MainHandler}\PYG{p}{(}\PYG{n}{tornado}\PYG{o}{.}\PYG{n}{web}\PYG{o}{.}\PYG{n}{RequestHandler}\PYG{p}{)}\PYG{p}{:}
    \PYG{k}{def} \PYG{n+nf}{get}\PYG{p}{(}\PYG{n+nb+bp}{self}\PYG{p}{)}\PYG{p}{:}
        \PYG{n+nb+bp}{self}\PYG{o}{.}\PYG{n}{write}\PYG{p}{(}\PYG{l+s+s2}{\PYGZdq{}}\PYG{l+s+s2}{Hello, world}\PYG{l+s+s2}{\PYGZdq{}}\PYG{p}{)}

\PYG{k}{def} \PYG{n+nf}{make\PYGZus{}app}\PYG{p}{(}\PYG{p}{)}\PYG{p}{:}
    \PYG{k}{return} \PYG{n}{tornado}\PYG{o}{.}\PYG{n}{web}\PYG{o}{.}\PYG{n}{Application}\PYG{p}{(}\PYG{p}{[}
        \PYG{p}{(}\PYG{l+s+sa}{r}\PYG{l+s+s2}{\PYGZdq{}}\PYG{l+s+s2}{/}\PYG{l+s+s2}{\PYGZdq{}}\PYG{p}{,} \PYG{n}{MainHandler}\PYG{p}{)}\PYG{p}{,}
    \PYG{p}{]}\PYG{p}{)}

\PYG{k}{if} \PYG{n+nv+vm}{\PYGZus{}\PYGZus{}name\PYGZus{}\PYGZus{}} \PYG{o}{==} \PYG{l+s+s2}{\PYGZdq{}}\PYG{l+s+s2}{\PYGZus{}\PYGZus{}main\PYGZus{}\PYGZus{}}\PYG{l+s+s2}{\PYGZdq{}}\PYG{p}{:}
    \PYG{n}{app} \PYG{o}{=} \PYG{n}{make\PYGZus{}app}\PYG{p}{(}\PYG{p}{)}
    \PYG{n}{app}\PYG{o}{.}\PYG{n}{listen}\PYG{p}{(}\PYG{l+m+mi}{8888}\PYG{p}{)}
    \PYG{n}{tornado}\PYG{o}{.}\PYG{n}{ioloop}\PYG{o}{.}\PYG{n}{IOLoop}\PYG{o}{.}\PYG{n}{current}\PYG{p}{(}\PYG{p}{)}\PYG{o}{.}\PYG{n}{start}\PYG{p}{(}\PYG{p}{)}
\end{sphinxVerbatim}

This example does not use any of Tornado’s asynchronous features; for
that see this \sphinxhref{https://github.com/tornadoweb/tornado/tree/stable/demos/chat}{simple chat room}.


\chapter{Threads and WSGI}
\label{\detokenize{index:threads-and-wsgi}}
Tornado is different from most Python web frameworks. It is not based
on \sphinxhref{https://wsgi.readthedocs.io/en/latest/}{WSGI}, and it is
typically run with only one thread per process. See the {\hyperref[\detokenize{guide::doc}]{\sphinxcrossref{\DUrole{doc}{User’s guide}}}}
for more on Tornado’s approach to asynchronous programming.

While some support of WSGI is available in the {\hyperref[\detokenize{wsgi:module-tornado.wsgi}]{\sphinxcrossref{\sphinxcode{\sphinxupquote{tornado.wsgi}}}}} module,
it is not a focus of development and most applications should be
written to use Tornado’s own interfaces (such as {\hyperref[\detokenize{web:module-tornado.web}]{\sphinxcrossref{\sphinxcode{\sphinxupquote{tornado.web}}}}})
directly instead of using WSGI.

In general, Tornado code is not thread-safe. The only method in
Tornado that is safe to call from other threads is
{\hyperref[\detokenize{ioloop:tornado.ioloop.IOLoop.add_callback}]{\sphinxcrossref{\sphinxcode{\sphinxupquote{IOLoop.add\_callback}}}}}. You can also use {\hyperref[\detokenize{ioloop:tornado.ioloop.IOLoop.run_in_executor}]{\sphinxcrossref{\sphinxcode{\sphinxupquote{IOLoop.run\_in\_executor}}}}} to
asynchronously run a blocking function on another thread, but note
that the function passed to \sphinxcode{\sphinxupquote{run\_in\_executor}} should avoid
referencing any Tornado objects. \sphinxcode{\sphinxupquote{run\_in\_executor}} is the
recommended way to interact with blocking code.


\chapter{\sphinxstyleliteralintitle{\sphinxupquote{asyncio}} Integration}
\label{\detokenize{index:asyncio-integration}}
Tornado is integrated with the standard library \sphinxhref{https://docs.python.org/3.6/library/asyncio.html\#module-asyncio}{\sphinxcode{\sphinxupquote{asyncio}}} module and
shares the same event loop (by default since Tornado 5.0). In general,
libraries designed for use with \sphinxhref{https://docs.python.org/3.6/library/asyncio.html\#module-asyncio}{\sphinxcode{\sphinxupquote{asyncio}}} can be mixed freely with
Tornado.


\chapter{Installation}
\label{\detokenize{index:installation}}
\begin{sphinxVerbatim}[commandchars=\\\{\}]
\PYG{n}{pip} \PYG{n}{install} \PYG{n}{tornado}
\end{sphinxVerbatim}

Tornado is listed in \sphinxhref{http://pypi.python.org/pypi/tornado}{PyPI} and
can be installed with \sphinxcode{\sphinxupquote{pip}}. Note that the source distribution
includes demo applications that are not present when Tornado is
installed in this way, so you may wish to download a copy of the
source tarball or clone the \sphinxhref{https://github.com/tornadoweb/tornado}{git repository} as well.

\sphinxstylestrong{Prerequisites}: Tornado 6.0 requires Python 3.5.2 or newer (See
\sphinxhref{https://www.tornadoweb.org/en/branch5.1/}{Tornado 5.1} if
compatibility with Python 2.7 is required). The following optional
packages may be useful:
\begin{itemize}
\item {} 
\sphinxhref{http://pycurl.sourceforge.net}{pycurl} is used by the optional
\sphinxcode{\sphinxupquote{tornado.curl\_httpclient}}.  Libcurl version 7.22 or higher is required.

\item {} 
\sphinxhref{http://www.twistedmatrix.com}{Twisted} may be used with the classes in
{\hyperref[\detokenize{twisted:module-tornado.platform.twisted}]{\sphinxcrossref{\sphinxcode{\sphinxupquote{tornado.platform.twisted}}}}}.

\item {} 
\sphinxhref{https://pypi.python.org/pypi/pycares}{pycares} is an alternative
non-blocking DNS resolver that can be used when threads are not
appropriate.

\end{itemize}

\sphinxstylestrong{Platforms}: Tornado should run on any Unix-like platform, although
for the best performance and scalability only Linux (with \sphinxcode{\sphinxupquote{epoll}})
and BSD (with \sphinxcode{\sphinxupquote{kqueue}}) are recommended for production deployment
(even though Mac OS X is derived from BSD and supports kqueue, its
networking performance is generally poor so it is recommended only for
development use).  Tornado will also run on Windows, although this
configuration is not officially supported and is recommended only for
development use. Without reworking Tornado IOLoop interface, it’s not
possible to add a native Tornado Windows IOLoop implementation or
leverage Windows’ IOCP support from frameworks like AsyncIO or Twisted.


\chapter{Documentation}
\label{\detokenize{index:documentation}}
This documentation is also available in \sphinxhref{https://readthedocs.org/projects/tornado/downloads/}{PDF and Epub formats}.


\section{User’s guide}
\label{\detokenize{guide:user-s-guide}}\label{\detokenize{guide::doc}}

\subsection{Introduction}
\label{\detokenize{guide/intro:introduction}}\label{\detokenize{guide/intro::doc}}
\sphinxhref{http://www.tornadoweb.org}{Tornado} is a Python web framework and
asynchronous networking library, originally developed at \sphinxhref{https://en.wikipedia.org/wiki/FriendFeed}{FriendFeed}.  By using non-blocking network I/O, Tornado
can scale to tens of thousands of open connections, making it ideal for
\sphinxhref{http://en.wikipedia.org/wiki/Push\_technology\#Long\_polling}{long polling},
\sphinxhref{http://en.wikipedia.org/wiki/WebSocket}{WebSockets}, and other
applications that require a long-lived connection to each user.

Tornado can be roughly divided into four major components:
\begin{itemize}
\item {} 
A web framework (including {\hyperref[\detokenize{web:tornado.web.RequestHandler}]{\sphinxcrossref{\sphinxcode{\sphinxupquote{RequestHandler}}}}} which is subclassed to
create web applications, and various supporting classes).

\item {} 
Client- and server-side implementions of HTTP ({\hyperref[\detokenize{httpserver:tornado.httpserver.HTTPServer}]{\sphinxcrossref{\sphinxcode{\sphinxupquote{HTTPServer}}}}} and
{\hyperref[\detokenize{httpclient:tornado.httpclient.AsyncHTTPClient}]{\sphinxcrossref{\sphinxcode{\sphinxupquote{AsyncHTTPClient}}}}}).

\item {} 
An asynchronous networking library including the classes {\hyperref[\detokenize{ioloop:tornado.ioloop.IOLoop}]{\sphinxcrossref{\sphinxcode{\sphinxupquote{IOLoop}}}}}
and {\hyperref[\detokenize{iostream:tornado.iostream.IOStream}]{\sphinxcrossref{\sphinxcode{\sphinxupquote{IOStream}}}}}, which serve as the building blocks for the HTTP
components and can also be used to implement other protocols.

\item {} 
A coroutine library ({\hyperref[\detokenize{gen:module-tornado.gen}]{\sphinxcrossref{\sphinxcode{\sphinxupquote{tornado.gen}}}}}) which allows asynchronous
code to be written in a more straightforward way than chaining
callbacks. This is similar to the native coroutine feature introduced
in Python 3.5 (\sphinxcode{\sphinxupquote{async def}}). Native coroutines are recommended
in place of the {\hyperref[\detokenize{gen:module-tornado.gen}]{\sphinxcrossref{\sphinxcode{\sphinxupquote{tornado.gen}}}}} module when available.

\end{itemize}

The Tornado web framework and HTTP server together offer a full-stack
alternative to \sphinxhref{http://www.python.org/dev/peps/pep-3333/}{WSGI}.
While it is possible to use the Tornado HTTP server as a container for
other WSGI frameworks ({\hyperref[\detokenize{wsgi:tornado.wsgi.WSGIContainer}]{\sphinxcrossref{\sphinxcode{\sphinxupquote{WSGIContainer}}}}}), this combination has
limitations and to take full advantage of Tornado you will need to use
Tornado’s web framework and HTTP server together.


\subsection{Asynchronous and non-Blocking I/O}
\label{\detokenize{guide/async:asynchronous-and-non-blocking-i-o}}\label{\detokenize{guide/async::doc}}
Real-time web features require a long-lived mostly-idle connection per
user.  In a traditional synchronous web server, this implies devoting
one thread to each user, which can be very expensive.

To minimize the cost of concurrent connections, Tornado uses a
single-threaded event loop.  This means that all application code
should aim to be asynchronous and non-blocking because only one
operation can be active at a time.

The terms asynchronous and non-blocking are closely related and are
often used interchangeably, but they are not quite the same thing.


\subsubsection{Blocking}
\label{\detokenize{guide/async:blocking}}
A function \sphinxstylestrong{blocks} when it waits for something to happen before
returning.  A function may block for many reasons: network I/O, disk
I/O, mutexes, etc.  In fact, \sphinxstyleemphasis{every} function blocks, at least a
little bit, while it is running and using the CPU (for an extreme
example that demonstrates why CPU blocking must be taken as seriously
as other kinds of blocking, consider password hashing functions like
\sphinxhref{http://bcrypt.sourceforge.net/}{bcrypt}, which by design use
hundreds of milliseconds of CPU time, far more than a typical network
or disk access).

A function can be blocking in some respects and non-blocking in
others.  In the context of Tornado we generally talk about
blocking in the context of network I/O, although all kinds of blocking
are to be minimized.


\subsubsection{Asynchronous}
\label{\detokenize{guide/async:asynchronous}}
An \sphinxstylestrong{asynchronous} function returns before it is finished, and
generally causes some work to happen in the background before
triggering some future action in the application (as opposed to normal
\sphinxstylestrong{synchronous} functions, which do everything they are going to do
before returning).  There are many styles of asynchronous interfaces:
\begin{itemize}
\item {} 
Callback argument

\item {} 
Return a placeholder ({\hyperref[\detokenize{concurrent:tornado.concurrent.Future}]{\sphinxcrossref{\sphinxcode{\sphinxupquote{Future}}}}}, \sphinxcode{\sphinxupquote{Promise}}, \sphinxcode{\sphinxupquote{Deferred}})

\item {} 
Deliver to a queue

\item {} 
Callback registry (e.g. POSIX signals)

\end{itemize}

Regardless of which type of interface is used, asynchronous functions
\sphinxstyleemphasis{by definition} interact differently with their callers; there is no
free way to make a synchronous function asynchronous in a way that is
transparent to its callers (systems like \sphinxhref{http://www.gevent.org}{gevent} use lightweight threads to offer performance
comparable to asynchronous systems, but they do not actually make
things asynchronous).

Asynchronous operations in Tornado generally return placeholder
objects (\sphinxcode{\sphinxupquote{Futures}}), with the exception of some low-level components
like the {\hyperref[\detokenize{ioloop:tornado.ioloop.IOLoop}]{\sphinxcrossref{\sphinxcode{\sphinxupquote{IOLoop}}}}} that use callbacks. \sphinxcode{\sphinxupquote{Futures}} are usually
transformed into their result with the \sphinxcode{\sphinxupquote{await}} or \sphinxcode{\sphinxupquote{yield}}
keywords.


\subsubsection{Examples}
\label{\detokenize{guide/async:examples}}
Here is a sample synchronous function:

\begin{sphinxVerbatim}[commandchars=\\\{\}]
\PYG{k+kn}{from} \PYG{n+nn}{tornado}\PYG{n+nn}{.}\PYG{n+nn}{httpclient} \PYG{k}{import} \PYG{n}{HTTPClient}

\PYG{k}{def} \PYG{n+nf}{synchronous\PYGZus{}fetch}\PYG{p}{(}\PYG{n}{url}\PYG{p}{)}\PYG{p}{:}
    \PYG{n}{http\PYGZus{}client} \PYG{o}{=} \PYG{n}{HTTPClient}\PYG{p}{(}\PYG{p}{)}
    \PYG{n}{response} \PYG{o}{=} \PYG{n}{http\PYGZus{}client}\PYG{o}{.}\PYG{n}{fetch}\PYG{p}{(}\PYG{n}{url}\PYG{p}{)}
    \PYG{k}{return} \PYG{n}{response}\PYG{o}{.}\PYG{n}{body}
\end{sphinxVerbatim}

And here is the same function rewritten asynchronously as a native coroutine:

\begin{sphinxVerbatim}[commandchars=\\\{\}]
\PYG{k+kn}{from} \PYG{n+nn}{tornado}\PYG{n+nn}{.}\PYG{n+nn}{httpclient} \PYG{k}{import} \PYG{n}{AsyncHTTPClient}

\PYG{k}{async} \PYG{k}{def} \PYG{n+nf}{asynchronous\PYGZus{}fetch}\PYG{p}{(}\PYG{n}{url}\PYG{p}{)}\PYG{p}{:}
    \PYG{n}{http\PYGZus{}client} \PYG{o}{=} \PYG{n}{AsyncHTTPClient}\PYG{p}{(}\PYG{p}{)}
    \PYG{n}{response} \PYG{o}{=} \PYG{k}{await} \PYG{n}{http\PYGZus{}client}\PYG{o}{.}\PYG{n}{fetch}\PYG{p}{(}\PYG{n}{url}\PYG{p}{)}
    \PYG{k}{return} \PYG{n}{response}\PYG{o}{.}\PYG{n}{body}
\end{sphinxVerbatim}

Or for compatibility with older versions of Python, using the {\hyperref[\detokenize{gen:module-tornado.gen}]{\sphinxcrossref{\sphinxcode{\sphinxupquote{tornado.gen}}}}} module:

\begin{sphinxVerbatim}[commandchars=\\\{\}]
\PYG{k+kn}{from} \PYG{n+nn}{tornado}\PYG{n+nn}{.}\PYG{n+nn}{httpclient} \PYG{k}{import} \PYG{n}{AsyncHTTPClient}
\PYG{k+kn}{from} \PYG{n+nn}{tornado} \PYG{k}{import} \PYG{n}{gen}

\PYG{n+nd}{@gen}\PYG{o}{.}\PYG{n}{coroutine}
\PYG{k}{def} \PYG{n+nf}{async\PYGZus{}fetch\PYGZus{}gen}\PYG{p}{(}\PYG{n}{url}\PYG{p}{)}\PYG{p}{:}
    \PYG{n}{http\PYGZus{}client} \PYG{o}{=} \PYG{n}{AsyncHTTPClient}\PYG{p}{(}\PYG{p}{)}
    \PYG{n}{response} \PYG{o}{=} \PYG{k}{yield} \PYG{n}{http\PYGZus{}client}\PYG{o}{.}\PYG{n}{fetch}\PYG{p}{(}\PYG{n}{url}\PYG{p}{)}
    \PYG{k}{raise} \PYG{n}{gen}\PYG{o}{.}\PYG{n}{Return}\PYG{p}{(}\PYG{n}{response}\PYG{o}{.}\PYG{n}{body}\PYG{p}{)}
\end{sphinxVerbatim}

Coroutines are a little magical, but what they do internally is something like this:

\begin{sphinxVerbatim}[commandchars=\\\{\}]
\PYG{k+kn}{from} \PYG{n+nn}{tornado}\PYG{n+nn}{.}\PYG{n+nn}{concurrent} \PYG{k}{import} \PYG{n}{Future}

\PYG{k}{def} \PYG{n+nf}{async\PYGZus{}fetch\PYGZus{}manual}\PYG{p}{(}\PYG{n}{url}\PYG{p}{)}\PYG{p}{:}
    \PYG{n}{http\PYGZus{}client} \PYG{o}{=} \PYG{n}{AsyncHTTPClient}\PYG{p}{(}\PYG{p}{)}
    \PYG{n}{my\PYGZus{}future} \PYG{o}{=} \PYG{n}{Future}\PYG{p}{(}\PYG{p}{)}
    \PYG{n}{fetch\PYGZus{}future} \PYG{o}{=} \PYG{n}{http\PYGZus{}client}\PYG{o}{.}\PYG{n}{fetch}\PYG{p}{(}\PYG{n}{url}\PYG{p}{)}
    \PYG{k}{def} \PYG{n+nf}{on\PYGZus{}fetch}\PYG{p}{(}\PYG{n}{f}\PYG{p}{)}\PYG{p}{:}
        \PYG{n}{my\PYGZus{}future}\PYG{o}{.}\PYG{n}{set\PYGZus{}result}\PYG{p}{(}\PYG{n}{f}\PYG{o}{.}\PYG{n}{result}\PYG{p}{(}\PYG{p}{)}\PYG{o}{.}\PYG{n}{body}\PYG{p}{)}
    \PYG{n}{fetch\PYGZus{}future}\PYG{o}{.}\PYG{n}{add\PYGZus{}done\PYGZus{}callback}\PYG{p}{(}\PYG{n}{on\PYGZus{}fetch}\PYG{p}{)}
    \PYG{k}{return} \PYG{n}{my\PYGZus{}future}
\end{sphinxVerbatim}

Notice that the coroutine returns its {\hyperref[\detokenize{concurrent:tornado.concurrent.Future}]{\sphinxcrossref{\sphinxcode{\sphinxupquote{Future}}}}} before the fetch is
done. This is what makes coroutines \sphinxstyleemphasis{asynchronous}.

Anything you can do with coroutines you can also do by passing
callback objects around, but coroutines provide an important
simplification by letting you organize your code in the same way you
would if it were synchronous. This is especially important for error
handling, since \sphinxcode{\sphinxupquote{try}}/\sphinxcode{\sphinxupquote{except}} blocks work as you would expect in
coroutines while this is difficult to achieve with callbacks.
Coroutines will be discussed in depth in the next section of this
guide.


\subsection{Coroutines}
\label{\detokenize{guide/coroutines:coroutines}}\label{\detokenize{guide/coroutines::doc}}
\sphinxstylestrong{Coroutines} are the recommended way to write asynchronous code in
Tornado. Coroutines use the Python \sphinxcode{\sphinxupquote{await}} or \sphinxcode{\sphinxupquote{yield}} keyword to
suspend and resume execution instead of a chain of callbacks
(cooperative lightweight threads as seen in frameworks like \sphinxhref{http://www.gevent.org}{gevent} are sometimes called coroutines as well, but
in Tornado all coroutines use explicit context switches and are called
as asynchronous functions).

Coroutines are almost as simple as synchronous code, but without the
expense of a thread.  They also \sphinxhref{https://glyph.twistedmatrix.com/2014/02/unyielding.html}{make concurrency easier} to reason
about by reducing the number of places where a context switch can
happen.

Example:

\begin{sphinxVerbatim}[commandchars=\\\{\}]
\PYG{k}{async} \PYG{k}{def} \PYG{n+nf}{fetch\PYGZus{}coroutine}\PYG{p}{(}\PYG{n}{url}\PYG{p}{)}\PYG{p}{:}
    \PYG{n}{http\PYGZus{}client} \PYG{o}{=} \PYG{n}{AsyncHTTPClient}\PYG{p}{(}\PYG{p}{)}
    \PYG{n}{response} \PYG{o}{=} \PYG{k}{await} \PYG{n}{http\PYGZus{}client}\PYG{o}{.}\PYG{n}{fetch}\PYG{p}{(}\PYG{n}{url}\PYG{p}{)}
    \PYG{k}{return} \PYG{n}{response}\PYG{o}{.}\PYG{n}{body}
\end{sphinxVerbatim}


\subsubsection{Native vs decorated coroutines}
\label{\detokenize{guide/coroutines:native-vs-decorated-coroutines}}\label{\detokenize{guide/coroutines:native-coroutines}}
Python 3.5 introduced the \sphinxcode{\sphinxupquote{async}} and \sphinxcode{\sphinxupquote{await}} keywords (functions
using these keywords are also called “native coroutines”). For
compatibility with older versions of Python, you can use “decorated”
or “yield-based” coroutines using the {\hyperref[\detokenize{gen:tornado.gen.coroutine}]{\sphinxcrossref{\sphinxcode{\sphinxupquote{tornado.gen.coroutine}}}}}
decorator.

Native coroutines are the recommended form whenever possible. Only use
decorated coroutines when compatibility with older versions of Python
is required. Examples in the Tornado documentation will generally use
the native form.

Translation between the two forms is generally straightforward:

\begin{sphinxVerbatim}[commandchars=\\\{\}]
\PYG{c+c1}{\PYGZsh{} Decorated:                    \PYGZsh{} Native:}

\PYG{c+c1}{\PYGZsh{} Normal function declaration}
\PYG{c+c1}{\PYGZsh{} with decorator                \PYGZsh{} \PYGZdq{}async def\PYGZdq{} keywords}
\PYG{n+nd}{@gen}\PYG{o}{.}\PYG{n}{coroutine}
\PYG{k}{def} \PYG{n+nf}{a}\PYG{p}{(}\PYG{p}{)}\PYG{p}{:}                        \PYG{k}{async} \PYG{k}{def} \PYG{n+nf}{a}\PYG{p}{(}\PYG{p}{)}\PYG{p}{:}
    \PYG{c+c1}{\PYGZsh{} \PYGZdq{}yield\PYGZdq{} all async funcs       \PYGZsh{} \PYGZdq{}await\PYGZdq{} all async funcs}
    \PYG{n}{b} \PYG{o}{=} \PYG{k}{yield} \PYG{n}{c}\PYG{p}{(}\PYG{p}{)}                   \PYG{n}{b} \PYG{o}{=} \PYG{k}{await} \PYG{n}{c}\PYG{p}{(}\PYG{p}{)}
    \PYG{c+c1}{\PYGZsh{} \PYGZdq{}return\PYGZdq{} and \PYGZdq{}yield\PYGZdq{}}
    \PYG{c+c1}{\PYGZsh{} cannot be mixed in}
    \PYG{c+c1}{\PYGZsh{} Python 2, so raise a}
    \PYG{c+c1}{\PYGZsh{} special exception.            \PYGZsh{} Return normally}
    \PYG{k}{raise} \PYG{n}{gen}\PYG{o}{.}\PYG{n}{Return}\PYG{p}{(}\PYG{n}{b}\PYG{p}{)}             \PYG{k}{return} \PYG{n}{b}
\end{sphinxVerbatim}

Other differences between the two forms of coroutine are outlined below.
\begin{itemize}
\item {} 
Native coroutines:
\begin{itemize}
\item {} 
are generally faster.

\item {} 
can use \sphinxcode{\sphinxupquote{async for}} and \sphinxcode{\sphinxupquote{async with}}
statements which make some patterns much simpler.

\item {} 
do not run at all unless you \sphinxcode{\sphinxupquote{await}} or
\sphinxcode{\sphinxupquote{yield}} them. Decorated coroutines can start running “in the
background” as soon as they are called. Note that for both kinds of
coroutines it is important to use \sphinxcode{\sphinxupquote{await}} or \sphinxcode{\sphinxupquote{yield}} so that
any exceptions have somewhere to go.

\end{itemize}

\item {} 
Decorated coroutines:
\begin{itemize}
\item {} 
have additional integration with the
\sphinxhref{https://docs.python.org/3.6/library/concurrent.futures.html\#module-concurrent.futures}{\sphinxcode{\sphinxupquote{concurrent.futures}}} package, allowing the result of
\sphinxcode{\sphinxupquote{executor.submit}} to be yielded directly. For native coroutines,
use {\hyperref[\detokenize{ioloop:tornado.ioloop.IOLoop.run_in_executor}]{\sphinxcrossref{\sphinxcode{\sphinxupquote{IOLoop.run\_in\_executor}}}}} instead.

\item {} 
support some shorthand for waiting on multiple
objects by yielding a list or dict. Use {\hyperref[\detokenize{gen:tornado.gen.multi}]{\sphinxcrossref{\sphinxcode{\sphinxupquote{tornado.gen.multi}}}}} to do
this in native coroutines.

\item {} 
can support integration with other packages
including Twisted via a registry of conversion functions.
To access this functionality in native coroutines, use
{\hyperref[\detokenize{gen:tornado.gen.convert_yielded}]{\sphinxcrossref{\sphinxcode{\sphinxupquote{tornado.gen.convert\_yielded}}}}}.

\item {} 
always return a {\hyperref[\detokenize{concurrent:tornado.concurrent.Future}]{\sphinxcrossref{\sphinxcode{\sphinxupquote{Future}}}}} object. Native
coroutines return an \sphinxstyleemphasis{awaitable} object that is not a {\hyperref[\detokenize{concurrent:tornado.concurrent.Future}]{\sphinxcrossref{\sphinxcode{\sphinxupquote{Future}}}}}. In
Tornado the two are mostly interchangeable.

\end{itemize}

\end{itemize}


\subsubsection{How it works}
\label{\detokenize{guide/coroutines:how-it-works}}
This section explains the operation of decorated coroutines. Native
coroutines are conceptually similar, but a little more complicated
because of the extra integration with the Python runtime.

A function containing \sphinxcode{\sphinxupquote{yield}} is a \sphinxstylestrong{generator}.  All generators
are asynchronous; when called they return a generator object instead
of running to completion.  The \sphinxcode{\sphinxupquote{@gen.coroutine}} decorator
communicates with the generator via the \sphinxcode{\sphinxupquote{yield}} expressions, and
with the coroutine’s caller by returning a {\hyperref[\detokenize{concurrent:tornado.concurrent.Future}]{\sphinxcrossref{\sphinxcode{\sphinxupquote{Future}}}}}.

Here is a simplified version of the coroutine decorator’s inner loop:

\begin{sphinxVerbatim}[commandchars=\\\{\}]
\PYG{c+c1}{\PYGZsh{} Simplified inner loop of tornado.gen.Runner}
\PYG{k}{def} \PYG{n+nf}{run}\PYG{p}{(}\PYG{n+nb+bp}{self}\PYG{p}{)}\PYG{p}{:}
    \PYG{c+c1}{\PYGZsh{} send(x) makes the current yield return x.}
    \PYG{c+c1}{\PYGZsh{} It returns when the next yield is reached}
    \PYG{n}{future} \PYG{o}{=} \PYG{n+nb+bp}{self}\PYG{o}{.}\PYG{n}{gen}\PYG{o}{.}\PYG{n}{send}\PYG{p}{(}\PYG{n+nb+bp}{self}\PYG{o}{.}\PYG{n}{next}\PYG{p}{)}
    \PYG{k}{def} \PYG{n+nf}{callback}\PYG{p}{(}\PYG{n}{f}\PYG{p}{)}\PYG{p}{:}
        \PYG{n+nb+bp}{self}\PYG{o}{.}\PYG{n}{next} \PYG{o}{=} \PYG{n}{f}\PYG{o}{.}\PYG{n}{result}\PYG{p}{(}\PYG{p}{)}
        \PYG{n+nb+bp}{self}\PYG{o}{.}\PYG{n}{run}\PYG{p}{(}\PYG{p}{)}
    \PYG{n}{future}\PYG{o}{.}\PYG{n}{add\PYGZus{}done\PYGZus{}callback}\PYG{p}{(}\PYG{n}{callback}\PYG{p}{)}
\end{sphinxVerbatim}

The decorator receives a {\hyperref[\detokenize{concurrent:tornado.concurrent.Future}]{\sphinxcrossref{\sphinxcode{\sphinxupquote{Future}}}}} from the generator, waits (without
blocking) for that {\hyperref[\detokenize{concurrent:tornado.concurrent.Future}]{\sphinxcrossref{\sphinxcode{\sphinxupquote{Future}}}}} to complete, then “unwraps” the {\hyperref[\detokenize{concurrent:tornado.concurrent.Future}]{\sphinxcrossref{\sphinxcode{\sphinxupquote{Future}}}}}
and sends the result back into the generator as the result of the
\sphinxcode{\sphinxupquote{yield}} expression.  Most asynchronous code never touches the {\hyperref[\detokenize{concurrent:tornado.concurrent.Future}]{\sphinxcrossref{\sphinxcode{\sphinxupquote{Future}}}}}
class directly except to immediately pass the {\hyperref[\detokenize{concurrent:tornado.concurrent.Future}]{\sphinxcrossref{\sphinxcode{\sphinxupquote{Future}}}}} returned by
an asynchronous function to a \sphinxcode{\sphinxupquote{yield}} expression.


\subsubsection{How to call a coroutine}
\label{\detokenize{guide/coroutines:how-to-call-a-coroutine}}
Coroutines do not raise exceptions in the normal way: any exception
they raise will be trapped in the awaitable object until it is
yielded. This means it is important to call coroutines in the right
way, or you may have errors that go unnoticed:

\begin{sphinxVerbatim}[commandchars=\\\{\}]
\PYG{k}{async} \PYG{k}{def} \PYG{n+nf}{divide}\PYG{p}{(}\PYG{n}{x}\PYG{p}{,} \PYG{n}{y}\PYG{p}{)}\PYG{p}{:}
    \PYG{k}{return} \PYG{n}{x} \PYG{o}{/} \PYG{n}{y}

\PYG{k}{def} \PYG{n+nf}{bad\PYGZus{}call}\PYG{p}{(}\PYG{p}{)}\PYG{p}{:}
    \PYG{c+c1}{\PYGZsh{} This should raise a ZeroDivisionError, but it won\PYGZsq{}t because}
    \PYG{c+c1}{\PYGZsh{} the coroutine is called incorrectly.}
    \PYG{n}{divide}\PYG{p}{(}\PYG{l+m+mi}{1}\PYG{p}{,} \PYG{l+m+mi}{0}\PYG{p}{)}
\end{sphinxVerbatim}

In nearly all cases, any function that calls a coroutine must be a
coroutine itself, and use the \sphinxcode{\sphinxupquote{await}} or \sphinxcode{\sphinxupquote{yield}} keyword in the
call. When you are overriding a method defined in a superclass,
consult the documentation to see if coroutines are allowed (the
documentation should say that the method “may be a coroutine” or “may
return a {\hyperref[\detokenize{concurrent:tornado.concurrent.Future}]{\sphinxcrossref{\sphinxcode{\sphinxupquote{Future}}}}}”):

\begin{sphinxVerbatim}[commandchars=\\\{\}]
\PYG{k}{async} \PYG{k}{def} \PYG{n+nf}{good\PYGZus{}call}\PYG{p}{(}\PYG{p}{)}\PYG{p}{:}
    \PYG{c+c1}{\PYGZsh{} await will unwrap the object returned by divide() and raise}
    \PYG{c+c1}{\PYGZsh{} the exception.}
    \PYG{k}{await} \PYG{n}{divide}\PYG{p}{(}\PYG{l+m+mi}{1}\PYG{p}{,} \PYG{l+m+mi}{0}\PYG{p}{)}
\end{sphinxVerbatim}

Sometimes you may want to “fire and forget” a coroutine without waiting
for its result. In this case it is recommended to use {\hyperref[\detokenize{ioloop:tornado.ioloop.IOLoop.spawn_callback}]{\sphinxcrossref{\sphinxcode{\sphinxupquote{IOLoop.spawn\_callback}}}}},
which makes the {\hyperref[\detokenize{ioloop:tornado.ioloop.IOLoop}]{\sphinxcrossref{\sphinxcode{\sphinxupquote{IOLoop}}}}} responsible for the call. If it fails,
the {\hyperref[\detokenize{ioloop:tornado.ioloop.IOLoop}]{\sphinxcrossref{\sphinxcode{\sphinxupquote{IOLoop}}}}} will log a stack trace:

\begin{sphinxVerbatim}[commandchars=\\\{\}]
\PYG{c+c1}{\PYGZsh{} The IOLoop will catch the exception and print a stack trace in}
\PYG{c+c1}{\PYGZsh{} the logs. Note that this doesn\PYGZsq{}t look like a normal call, since}
\PYG{c+c1}{\PYGZsh{} we pass the function object to be called by the IOLoop.}
\PYG{n}{IOLoop}\PYG{o}{.}\PYG{n}{current}\PYG{p}{(}\PYG{p}{)}\PYG{o}{.}\PYG{n}{spawn\PYGZus{}callback}\PYG{p}{(}\PYG{n}{divide}\PYG{p}{,} \PYG{l+m+mi}{1}\PYG{p}{,} \PYG{l+m+mi}{0}\PYG{p}{)}
\end{sphinxVerbatim}

Using {\hyperref[\detokenize{ioloop:tornado.ioloop.IOLoop.spawn_callback}]{\sphinxcrossref{\sphinxcode{\sphinxupquote{IOLoop.spawn\_callback}}}}} in this way is \sphinxstyleemphasis{recommended} for
functions using \sphinxcode{\sphinxupquote{@gen.coroutine}}, but it is \sphinxstyleemphasis{required} for functions
using \sphinxcode{\sphinxupquote{async def}} (otherwise the coroutine runner will not start).

Finally, at the top level of a program, \sphinxstyleemphasis{if the IOLoop is not yet
running,} you can start the {\hyperref[\detokenize{ioloop:tornado.ioloop.IOLoop}]{\sphinxcrossref{\sphinxcode{\sphinxupquote{IOLoop}}}}}, run the coroutine, and then
stop the {\hyperref[\detokenize{ioloop:tornado.ioloop.IOLoop}]{\sphinxcrossref{\sphinxcode{\sphinxupquote{IOLoop}}}}} with the {\hyperref[\detokenize{ioloop:tornado.ioloop.IOLoop.run_sync}]{\sphinxcrossref{\sphinxcode{\sphinxupquote{IOLoop.run\_sync}}}}} method. This is often
used to start the \sphinxcode{\sphinxupquote{main}} function of a batch-oriented program:

\begin{sphinxVerbatim}[commandchars=\\\{\}]
\PYG{c+c1}{\PYGZsh{} run\PYGZus{}sync() doesn\PYGZsq{}t take arguments, so we must wrap the}
\PYG{c+c1}{\PYGZsh{} call in a lambda.}
\PYG{n}{IOLoop}\PYG{o}{.}\PYG{n}{current}\PYG{p}{(}\PYG{p}{)}\PYG{o}{.}\PYG{n}{run\PYGZus{}sync}\PYG{p}{(}\PYG{k}{lambda}\PYG{p}{:} \PYG{n}{divide}\PYG{p}{(}\PYG{l+m+mi}{1}\PYG{p}{,} \PYG{l+m+mi}{0}\PYG{p}{)}\PYG{p}{)}
\end{sphinxVerbatim}


\subsubsection{Coroutine patterns}
\label{\detokenize{guide/coroutines:coroutine-patterns}}

\paragraph{Calling blocking functions}
\label{\detokenize{guide/coroutines:calling-blocking-functions}}
The simplest way to call a blocking function from a coroutine is to
use {\hyperref[\detokenize{ioloop:tornado.ioloop.IOLoop.run_in_executor}]{\sphinxcrossref{\sphinxcode{\sphinxupquote{IOLoop.run\_in\_executor}}}}}, which returns
\sphinxcode{\sphinxupquote{Futures}} that are compatible with coroutines:

\begin{sphinxVerbatim}[commandchars=\\\{\}]
\PYG{k}{async} \PYG{k}{def} \PYG{n+nf}{call\PYGZus{}blocking}\PYG{p}{(}\PYG{p}{)}\PYG{p}{:}
    \PYG{k}{await} \PYG{n}{IOLoop}\PYG{o}{.}\PYG{n}{current}\PYG{p}{(}\PYG{p}{)}\PYG{o}{.}\PYG{n}{run\PYGZus{}in\PYGZus{}executor}\PYG{p}{(}\PYG{k+kc}{None}\PYG{p}{,} \PYG{n}{blocking\PYGZus{}func}\PYG{p}{,} \PYG{n}{args}\PYG{p}{)}
\end{sphinxVerbatim}


\paragraph{Parallelism}
\label{\detokenize{guide/coroutines:parallelism}}
The {\hyperref[\detokenize{gen:tornado.gen.multi}]{\sphinxcrossref{\sphinxcode{\sphinxupquote{multi}}}}} function accepts lists and dicts whose values are
\sphinxcode{\sphinxupquote{Futures}}, and waits for all of those \sphinxcode{\sphinxupquote{Futures}} in parallel:

\begin{sphinxVerbatim}[commandchars=\\\{\}]
\PYG{k+kn}{from} \PYG{n+nn}{tornado}\PYG{n+nn}{.}\PYG{n+nn}{gen} \PYG{k}{import} \PYG{n}{multi}

\PYG{k}{async} \PYG{k}{def} \PYG{n+nf}{parallel\PYGZus{}fetch}\PYG{p}{(}\PYG{n}{url1}\PYG{p}{,} \PYG{n}{url2}\PYG{p}{)}\PYG{p}{:}
    \PYG{n}{resp1}\PYG{p}{,} \PYG{n}{resp2} \PYG{o}{=} \PYG{k}{await} \PYG{n}{multi}\PYG{p}{(}\PYG{p}{[}\PYG{n}{http\PYGZus{}client}\PYG{o}{.}\PYG{n}{fetch}\PYG{p}{(}\PYG{n}{url1}\PYG{p}{)}\PYG{p}{,}
                                \PYG{n}{http\PYGZus{}client}\PYG{o}{.}\PYG{n}{fetch}\PYG{p}{(}\PYG{n}{url2}\PYG{p}{)}\PYG{p}{]}\PYG{p}{)}

\PYG{k}{async} \PYG{k}{def} \PYG{n+nf}{parallel\PYGZus{}fetch\PYGZus{}many}\PYG{p}{(}\PYG{n}{urls}\PYG{p}{)}\PYG{p}{:}
    \PYG{n}{responses} \PYG{o}{=} \PYG{k}{await} \PYG{n}{multi} \PYG{p}{(}\PYG{p}{[}\PYG{n}{http\PYGZus{}client}\PYG{o}{.}\PYG{n}{fetch}\PYG{p}{(}\PYG{n}{url}\PYG{p}{)} \PYG{k}{for} \PYG{n}{url} \PYG{o+ow}{in} \PYG{n}{urls}\PYG{p}{]}\PYG{p}{)}
    \PYG{c+c1}{\PYGZsh{} responses is a list of HTTPResponses in the same order}

\PYG{k}{async} \PYG{k}{def} \PYG{n+nf}{parallel\PYGZus{}fetch\PYGZus{}dict}\PYG{p}{(}\PYG{n}{urls}\PYG{p}{)}\PYG{p}{:}
    \PYG{n}{responses} \PYG{o}{=} \PYG{k}{await} \PYG{n}{multi}\PYG{p}{(}\PYG{p}{\PYGZob{}}\PYG{n}{url}\PYG{p}{:} \PYG{n}{http\PYGZus{}client}\PYG{o}{.}\PYG{n}{fetch}\PYG{p}{(}\PYG{n}{url}\PYG{p}{)}
                             \PYG{k}{for} \PYG{n}{url} \PYG{o+ow}{in} \PYG{n}{urls}\PYG{p}{\PYGZcb{}}\PYG{p}{)}
    \PYG{c+c1}{\PYGZsh{} responses is a dict \PYGZob{}url: HTTPResponse\PYGZcb{}}
\end{sphinxVerbatim}

In decorated coroutines, it is possible to \sphinxcode{\sphinxupquote{yield}} the list or dict directly:

\begin{sphinxVerbatim}[commandchars=\\\{\}]
\PYG{n+nd}{@gen}\PYG{o}{.}\PYG{n}{coroutine}
\PYG{k}{def} \PYG{n+nf}{parallel\PYGZus{}fetch\PYGZus{}decorated}\PYG{p}{(}\PYG{n}{url1}\PYG{p}{,} \PYG{n}{url2}\PYG{p}{)}\PYG{p}{:}
    \PYG{n}{resp1}\PYG{p}{,} \PYG{n}{resp2} \PYG{o}{=} \PYG{k}{yield} \PYG{p}{[}\PYG{n}{http\PYGZus{}client}\PYG{o}{.}\PYG{n}{fetch}\PYG{p}{(}\PYG{n}{url1}\PYG{p}{)}\PYG{p}{,}
                          \PYG{n}{http\PYGZus{}client}\PYG{o}{.}\PYG{n}{fetch}\PYG{p}{(}\PYG{n}{url2}\PYG{p}{)}\PYG{p}{]}
\end{sphinxVerbatim}


\paragraph{Interleaving}
\label{\detokenize{guide/coroutines:interleaving}}
Sometimes it is useful to save a {\hyperref[\detokenize{concurrent:tornado.concurrent.Future}]{\sphinxcrossref{\sphinxcode{\sphinxupquote{Future}}}}} instead of yielding it
immediately, so you can start another operation before waiting.

\begin{sphinxVerbatim}[commandchars=\\\{\}]
\PYG{k+kn}{from} \PYG{n+nn}{tornado}\PYG{n+nn}{.}\PYG{n+nn}{gen} \PYG{k}{import} \PYG{n}{convert\PYGZus{}yielded}

\PYG{k}{async} \PYG{k}{def} \PYG{n+nf}{get}\PYG{p}{(}\PYG{n+nb+bp}{self}\PYG{p}{)}\PYG{p}{:}
    \PYG{c+c1}{\PYGZsh{} convert\PYGZus{}yielded() starts the native coroutine in the background.}
    \PYG{c+c1}{\PYGZsh{} This is equivalent to asyncio.ensure\PYGZus{}future() (both work in Tornado).}
    \PYG{n}{fetch\PYGZus{}future} \PYG{o}{=} \PYG{n}{convert\PYGZus{}yielded}\PYG{p}{(}\PYG{n+nb+bp}{self}\PYG{o}{.}\PYG{n}{fetch\PYGZus{}next\PYGZus{}chunk}\PYG{p}{(}\PYG{p}{)}\PYG{p}{)}
    \PYG{k}{while} \PYG{k+kc}{True}\PYG{p}{:}
        \PYG{n}{chunk} \PYG{o}{=} \PYG{k}{yield} \PYG{n}{fetch\PYGZus{}future}
        \PYG{k}{if} \PYG{n}{chunk} \PYG{o+ow}{is} \PYG{k+kc}{None}\PYG{p}{:} \PYG{k}{break}
        \PYG{n+nb+bp}{self}\PYG{o}{.}\PYG{n}{write}\PYG{p}{(}\PYG{n}{chunk}\PYG{p}{)}
        \PYG{n}{fetch\PYGZus{}future} \PYG{o}{=} \PYG{n}{convert\PYGZus{}yielded}\PYG{p}{(}\PYG{n+nb+bp}{self}\PYG{o}{.}\PYG{n}{fetch\PYGZus{}next\PYGZus{}chunk}\PYG{p}{(}\PYG{p}{)}\PYG{p}{)}
        \PYG{k}{yield} \PYG{n+nb+bp}{self}\PYG{o}{.}\PYG{n}{flush}\PYG{p}{(}\PYG{p}{)}
\end{sphinxVerbatim}

This is a little easier to do with decorated coroutines, because they
start immediately when called:

\begin{sphinxVerbatim}[commandchars=\\\{\}]
\PYG{n+nd}{@gen}\PYG{o}{.}\PYG{n}{coroutine}
\PYG{k}{def} \PYG{n+nf}{get}\PYG{p}{(}\PYG{n+nb+bp}{self}\PYG{p}{)}\PYG{p}{:}
    \PYG{n}{fetch\PYGZus{}future} \PYG{o}{=} \PYG{n+nb+bp}{self}\PYG{o}{.}\PYG{n}{fetch\PYGZus{}next\PYGZus{}chunk}\PYG{p}{(}\PYG{p}{)}
    \PYG{k}{while} \PYG{k+kc}{True}\PYG{p}{:}
        \PYG{n}{chunk} \PYG{o}{=} \PYG{k}{yield} \PYG{n}{fetch\PYGZus{}future}
        \PYG{k}{if} \PYG{n}{chunk} \PYG{o+ow}{is} \PYG{k+kc}{None}\PYG{p}{:} \PYG{k}{break}
        \PYG{n+nb+bp}{self}\PYG{o}{.}\PYG{n}{write}\PYG{p}{(}\PYG{n}{chunk}\PYG{p}{)}
        \PYG{n}{fetch\PYGZus{}future} \PYG{o}{=} \PYG{n+nb+bp}{self}\PYG{o}{.}\PYG{n}{fetch\PYGZus{}next\PYGZus{}chunk}\PYG{p}{(}\PYG{p}{)}
        \PYG{k}{yield} \PYG{n+nb+bp}{self}\PYG{o}{.}\PYG{n}{flush}\PYG{p}{(}\PYG{p}{)}
\end{sphinxVerbatim}


\paragraph{Looping}
\label{\detokenize{guide/coroutines:looping}}
In native coroutines, \sphinxcode{\sphinxupquote{async for}} can be used. In older versions of
Python, looping is tricky with coroutines since there is no way to
\sphinxcode{\sphinxupquote{yield}} on every iteration of a \sphinxcode{\sphinxupquote{for}} or \sphinxcode{\sphinxupquote{while}} loop and
capture the result of the yield. Instead, you’ll need to separate the
loop condition from accessing the results, as in this example from
\sphinxhref{https://motor.readthedocs.io/en/stable/}{Motor}:

\begin{sphinxVerbatim}[commandchars=\\\{\}]
\PYG{k+kn}{import} \PYG{n+nn}{motor}
\PYG{n}{db} \PYG{o}{=} \PYG{n}{motor}\PYG{o}{.}\PYG{n}{MotorClient}\PYG{p}{(}\PYG{p}{)}\PYG{o}{.}\PYG{n}{test}

\PYG{n+nd}{@gen}\PYG{o}{.}\PYG{n}{coroutine}
\PYG{k}{def} \PYG{n+nf}{loop\PYGZus{}example}\PYG{p}{(}\PYG{n}{collection}\PYG{p}{)}\PYG{p}{:}
    \PYG{n}{cursor} \PYG{o}{=} \PYG{n}{db}\PYG{o}{.}\PYG{n}{collection}\PYG{o}{.}\PYG{n}{find}\PYG{p}{(}\PYG{p}{)}
    \PYG{k}{while} \PYG{p}{(}\PYG{k}{yield} \PYG{n}{cursor}\PYG{o}{.}\PYG{n}{fetch\PYGZus{}next}\PYG{p}{)}\PYG{p}{:}
        \PYG{n}{doc} \PYG{o}{=} \PYG{n}{cursor}\PYG{o}{.}\PYG{n}{next\PYGZus{}object}\PYG{p}{(}\PYG{p}{)}
\end{sphinxVerbatim}


\paragraph{Running in the background}
\label{\detokenize{guide/coroutines:running-in-the-background}}
{\hyperref[\detokenize{ioloop:tornado.ioloop.PeriodicCallback}]{\sphinxcrossref{\sphinxcode{\sphinxupquote{PeriodicCallback}}}}} is not normally used with coroutines. Instead, a
coroutine can contain a \sphinxcode{\sphinxupquote{while True:}} loop and use
{\hyperref[\detokenize{gen:tornado.gen.sleep}]{\sphinxcrossref{\sphinxcode{\sphinxupquote{tornado.gen.sleep}}}}}:

\begin{sphinxVerbatim}[commandchars=\\\{\}]
\PYG{k}{async} \PYG{k}{def} \PYG{n+nf}{minute\PYGZus{}loop}\PYG{p}{(}\PYG{p}{)}\PYG{p}{:}
    \PYG{k}{while} \PYG{k+kc}{True}\PYG{p}{:}
        \PYG{k}{await} \PYG{n}{do\PYGZus{}something}\PYG{p}{(}\PYG{p}{)}
        \PYG{k}{await} \PYG{n}{gen}\PYG{o}{.}\PYG{n}{sleep}\PYG{p}{(}\PYG{l+m+mi}{60}\PYG{p}{)}

\PYG{c+c1}{\PYGZsh{} Coroutines that loop forever are generally started with}
\PYG{c+c1}{\PYGZsh{} spawn\PYGZus{}callback().}
\PYG{n}{IOLoop}\PYG{o}{.}\PYG{n}{current}\PYG{p}{(}\PYG{p}{)}\PYG{o}{.}\PYG{n}{spawn\PYGZus{}callback}\PYG{p}{(}\PYG{n}{minute\PYGZus{}loop}\PYG{p}{)}
\end{sphinxVerbatim}

Sometimes a more complicated loop may be desirable. For example, the
previous loop runs every \sphinxcode{\sphinxupquote{60+N}} seconds, where \sphinxcode{\sphinxupquote{N}} is the running
time of \sphinxcode{\sphinxupquote{do\_something()}}. To run exactly every 60 seconds, use the
interleaving pattern from above:

\begin{sphinxVerbatim}[commandchars=\\\{\}]
\PYG{k}{async} \PYG{k}{def} \PYG{n+nf}{minute\PYGZus{}loop2}\PYG{p}{(}\PYG{p}{)}\PYG{p}{:}
    \PYG{k}{while} \PYG{k+kc}{True}\PYG{p}{:}
        \PYG{n}{nxt} \PYG{o}{=} \PYG{n}{gen}\PYG{o}{.}\PYG{n}{sleep}\PYG{p}{(}\PYG{l+m+mi}{60}\PYG{p}{)}   \PYG{c+c1}{\PYGZsh{} Start the clock.}
        \PYG{k}{await} \PYG{n}{do\PYGZus{}something}\PYG{p}{(}\PYG{p}{)}  \PYG{c+c1}{\PYGZsh{} Run while the clock is ticking.}
        \PYG{k}{await} \PYG{n}{nxt}             \PYG{c+c1}{\PYGZsh{} Wait for the timer to run out.}
\end{sphinxVerbatim}


\subsection{\sphinxstyleliteralintitle{\sphinxupquote{Queue}} example - a concurrent web spider}
\label{\detokenize{guide/queues:queue-example-a-concurrent-web-spider}}\label{\detokenize{guide/queues::doc}}
Tornado’s {\hyperref[\detokenize{queues:module-tornado.queues}]{\sphinxcrossref{\sphinxcode{\sphinxupquote{tornado.queues}}}}} module implements an asynchronous producer /
consumer pattern for coroutines, analogous to the pattern implemented for
threads by the Python standard library’s \sphinxhref{https://docs.python.org/3.6/library/queue.html\#module-queue}{\sphinxcode{\sphinxupquote{queue}}} module.

A coroutine that yields {\hyperref[\detokenize{queues:tornado.queues.Queue.get}]{\sphinxcrossref{\sphinxcode{\sphinxupquote{Queue.get}}}}} pauses until there is an item in the queue.
If the queue has a maximum size set, a coroutine that yields {\hyperref[\detokenize{queues:tornado.queues.Queue.put}]{\sphinxcrossref{\sphinxcode{\sphinxupquote{Queue.put}}}}} pauses
until there is room for another item.

A {\hyperref[\detokenize{queues:tornado.queues.Queue}]{\sphinxcrossref{\sphinxcode{\sphinxupquote{Queue}}}}} maintains a count of unfinished tasks, which begins at zero.
{\hyperref[\detokenize{queues:tornado.queues.Queue.put}]{\sphinxcrossref{\sphinxcode{\sphinxupquote{put}}}}} increments the count; {\hyperref[\detokenize{queues:tornado.queues.Queue.task_done}]{\sphinxcrossref{\sphinxcode{\sphinxupquote{task\_done}}}}} decrements it.

In the web-spider example here, the queue begins containing only base\_url. When
a worker fetches a page it parses the links and puts new ones in the queue,
then calls {\hyperref[\detokenize{queues:tornado.queues.Queue.task_done}]{\sphinxcrossref{\sphinxcode{\sphinxupquote{task\_done}}}}} to decrement the counter once. Eventually, a
worker fetches a page whose URLs have all been seen before, and there is also
no work left in the queue. Thus that worker’s call to {\hyperref[\detokenize{queues:tornado.queues.Queue.task_done}]{\sphinxcrossref{\sphinxcode{\sphinxupquote{task\_done}}}}}
decrements the counter to zero. The main coroutine, which is waiting for
{\hyperref[\detokenize{queues:tornado.queues.Queue.join}]{\sphinxcrossref{\sphinxcode{\sphinxupquote{join}}}}}, is unpaused and finishes.

\begin{sphinxVerbatim}[commandchars=\\\{\}]
\PYG{c+ch}{\PYGZsh{}!/usr/bin/env python3}

\PYG{k+kn}{import} \PYG{n+nn}{time}
\PYG{k+kn}{from} \PYG{n+nn}{datetime} \PYG{k}{import} \PYG{n}{timedelta}

\PYG{k+kn}{from} \PYG{n+nn}{html}\PYG{n+nn}{.}\PYG{n+nn}{parser} \PYG{k}{import} \PYG{n}{HTMLParser}
\PYG{k+kn}{from} \PYG{n+nn}{urllib}\PYG{n+nn}{.}\PYG{n+nn}{parse} \PYG{k}{import} \PYG{n}{urljoin}\PYG{p}{,} \PYG{n}{urldefrag}

\PYG{k+kn}{from} \PYG{n+nn}{tornado} \PYG{k}{import} \PYG{n}{gen}\PYG{p}{,} \PYG{n}{httpclient}\PYG{p}{,} \PYG{n}{ioloop}\PYG{p}{,} \PYG{n}{queues}

\PYG{n}{base\PYGZus{}url} \PYG{o}{=} \PYG{l+s+s2}{\PYGZdq{}}\PYG{l+s+s2}{http://www.tornadoweb.org/en/stable/}\PYG{l+s+s2}{\PYGZdq{}}
\PYG{n}{concurrency} \PYG{o}{=} \PYG{l+m+mi}{10}


\PYG{k}{async} \PYG{k}{def} \PYG{n+nf}{get\PYGZus{}links\PYGZus{}from\PYGZus{}url}\PYG{p}{(}\PYG{n}{url}\PYG{p}{)}\PYG{p}{:}
    \PYG{l+s+sd}{\PYGZdq{}\PYGZdq{}\PYGZdq{}Download the page at {}`url{}` and parse it for links.}

\PYG{l+s+sd}{    Returned links have had the fragment after {}`\PYGZsh{}{}` removed, and have been made}
\PYG{l+s+sd}{    absolute so, e.g. the URL \PYGZsq{}gen.html\PYGZsh{}tornado.gen.coroutine\PYGZsq{} becomes}
\PYG{l+s+sd}{    \PYGZsq{}http://www.tornadoweb.org/en/stable/gen.html\PYGZsq{}.}
\PYG{l+s+sd}{    \PYGZdq{}\PYGZdq{}\PYGZdq{}}
    \PYG{n}{response} \PYG{o}{=} \PYG{k}{await} \PYG{n}{httpclient}\PYG{o}{.}\PYG{n}{AsyncHTTPClient}\PYG{p}{(}\PYG{p}{)}\PYG{o}{.}\PYG{n}{fetch}\PYG{p}{(}\PYG{n}{url}\PYG{p}{)}
    \PYG{n+nb}{print}\PYG{p}{(}\PYG{l+s+s2}{\PYGZdq{}}\PYG{l+s+s2}{fetched }\PYG{l+s+si}{\PYGZpc{}s}\PYG{l+s+s2}{\PYGZdq{}} \PYG{o}{\PYGZpc{}} \PYG{n}{url}\PYG{p}{)}

    \PYG{n}{html} \PYG{o}{=} \PYG{n}{response}\PYG{o}{.}\PYG{n}{body}\PYG{o}{.}\PYG{n}{decode}\PYG{p}{(}\PYG{n}{errors}\PYG{o}{=}\PYG{l+s+s2}{\PYGZdq{}}\PYG{l+s+s2}{ignore}\PYG{l+s+s2}{\PYGZdq{}}\PYG{p}{)}
    \PYG{k}{return} \PYG{p}{[}\PYG{n}{urljoin}\PYG{p}{(}\PYG{n}{url}\PYG{p}{,} \PYG{n}{remove\PYGZus{}fragment}\PYG{p}{(}\PYG{n}{new\PYGZus{}url}\PYG{p}{)}\PYG{p}{)} \PYG{k}{for} \PYG{n}{new\PYGZus{}url} \PYG{o+ow}{in} \PYG{n}{get\PYGZus{}links}\PYG{p}{(}\PYG{n}{html}\PYG{p}{)}\PYG{p}{]}


\PYG{k}{def} \PYG{n+nf}{remove\PYGZus{}fragment}\PYG{p}{(}\PYG{n}{url}\PYG{p}{)}\PYG{p}{:}
    \PYG{n}{pure\PYGZus{}url}\PYG{p}{,} \PYG{n}{frag} \PYG{o}{=} \PYG{n}{urldefrag}\PYG{p}{(}\PYG{n}{url}\PYG{p}{)}
    \PYG{k}{return} \PYG{n}{pure\PYGZus{}url}


\PYG{k}{def} \PYG{n+nf}{get\PYGZus{}links}\PYG{p}{(}\PYG{n}{html}\PYG{p}{)}\PYG{p}{:}
    \PYG{k}{class} \PYG{n+nc}{URLSeeker}\PYG{p}{(}\PYG{n}{HTMLParser}\PYG{p}{)}\PYG{p}{:}
        \PYG{k}{def} \PYG{n+nf}{\PYGZus{}\PYGZus{}init\PYGZus{}\PYGZus{}}\PYG{p}{(}\PYG{n+nb+bp}{self}\PYG{p}{)}\PYG{p}{:}
            \PYG{n}{HTMLParser}\PYG{o}{.}\PYG{n+nf+fm}{\PYGZus{}\PYGZus{}init\PYGZus{}\PYGZus{}}\PYG{p}{(}\PYG{n+nb+bp}{self}\PYG{p}{)}
            \PYG{n+nb+bp}{self}\PYG{o}{.}\PYG{n}{urls} \PYG{o}{=} \PYG{p}{[}\PYG{p}{]}

        \PYG{k}{def} \PYG{n+nf}{handle\PYGZus{}starttag}\PYG{p}{(}\PYG{n+nb+bp}{self}\PYG{p}{,} \PYG{n}{tag}\PYG{p}{,} \PYG{n}{attrs}\PYG{p}{)}\PYG{p}{:}
            \PYG{n}{href} \PYG{o}{=} \PYG{n+nb}{dict}\PYG{p}{(}\PYG{n}{attrs}\PYG{p}{)}\PYG{o}{.}\PYG{n}{get}\PYG{p}{(}\PYG{l+s+s2}{\PYGZdq{}}\PYG{l+s+s2}{href}\PYG{l+s+s2}{\PYGZdq{}}\PYG{p}{)}
            \PYG{k}{if} \PYG{n}{href} \PYG{o+ow}{and} \PYG{n}{tag} \PYG{o}{==} \PYG{l+s+s2}{\PYGZdq{}}\PYG{l+s+s2}{a}\PYG{l+s+s2}{\PYGZdq{}}\PYG{p}{:}
                \PYG{n+nb+bp}{self}\PYG{o}{.}\PYG{n}{urls}\PYG{o}{.}\PYG{n}{append}\PYG{p}{(}\PYG{n}{href}\PYG{p}{)}

    \PYG{n}{url\PYGZus{}seeker} \PYG{o}{=} \PYG{n}{URLSeeker}\PYG{p}{(}\PYG{p}{)}
    \PYG{n}{url\PYGZus{}seeker}\PYG{o}{.}\PYG{n}{feed}\PYG{p}{(}\PYG{n}{html}\PYG{p}{)}
    \PYG{k}{return} \PYG{n}{url\PYGZus{}seeker}\PYG{o}{.}\PYG{n}{urls}


\PYG{k}{async} \PYG{k}{def} \PYG{n+nf}{main}\PYG{p}{(}\PYG{p}{)}\PYG{p}{:}
    \PYG{n}{q} \PYG{o}{=} \PYG{n}{queues}\PYG{o}{.}\PYG{n}{Queue}\PYG{p}{(}\PYG{p}{)}
    \PYG{n}{start} \PYG{o}{=} \PYG{n}{time}\PYG{o}{.}\PYG{n}{time}\PYG{p}{(}\PYG{p}{)}
    \PYG{n}{fetching}\PYG{p}{,} \PYG{n}{fetched} \PYG{o}{=} \PYG{n+nb}{set}\PYG{p}{(}\PYG{p}{)}\PYG{p}{,} \PYG{n+nb}{set}\PYG{p}{(}\PYG{p}{)}

    \PYG{k}{async} \PYG{k}{def} \PYG{n+nf}{fetch\PYGZus{}url}\PYG{p}{(}\PYG{n}{current\PYGZus{}url}\PYG{p}{)}\PYG{p}{:}
        \PYG{k}{if} \PYG{n}{current\PYGZus{}url} \PYG{o+ow}{in} \PYG{n}{fetching}\PYG{p}{:}
            \PYG{k}{return}

        \PYG{n+nb}{print}\PYG{p}{(}\PYG{l+s+s2}{\PYGZdq{}}\PYG{l+s+s2}{fetching }\PYG{l+s+si}{\PYGZpc{}s}\PYG{l+s+s2}{\PYGZdq{}} \PYG{o}{\PYGZpc{}} \PYG{n}{current\PYGZus{}url}\PYG{p}{)}
        \PYG{n}{fetching}\PYG{o}{.}\PYG{n}{add}\PYG{p}{(}\PYG{n}{current\PYGZus{}url}\PYG{p}{)}
        \PYG{n}{urls} \PYG{o}{=} \PYG{k}{await} \PYG{n}{get\PYGZus{}links\PYGZus{}from\PYGZus{}url}\PYG{p}{(}\PYG{n}{current\PYGZus{}url}\PYG{p}{)}
        \PYG{n}{fetched}\PYG{o}{.}\PYG{n}{add}\PYG{p}{(}\PYG{n}{current\PYGZus{}url}\PYG{p}{)}

        \PYG{k}{for} \PYG{n}{new\PYGZus{}url} \PYG{o+ow}{in} \PYG{n}{urls}\PYG{p}{:}
            \PYG{c+c1}{\PYGZsh{} Only follow links beneath the base URL}
            \PYG{k}{if} \PYG{n}{new\PYGZus{}url}\PYG{o}{.}\PYG{n}{startswith}\PYG{p}{(}\PYG{n}{base\PYGZus{}url}\PYG{p}{)}\PYG{p}{:}
                \PYG{k}{await} \PYG{n}{q}\PYG{o}{.}\PYG{n}{put}\PYG{p}{(}\PYG{n}{new\PYGZus{}url}\PYG{p}{)}

    \PYG{k}{async} \PYG{k}{def} \PYG{n+nf}{worker}\PYG{p}{(}\PYG{p}{)}\PYG{p}{:}
        \PYG{k}{async} \PYG{k}{for} \PYG{n}{url} \PYG{o+ow}{in} \PYG{n}{q}\PYG{p}{:}
            \PYG{k}{if} \PYG{n}{url} \PYG{o+ow}{is} \PYG{k+kc}{None}\PYG{p}{:}
                \PYG{k}{return}
            \PYG{k}{try}\PYG{p}{:}
                \PYG{k}{await} \PYG{n}{fetch\PYGZus{}url}\PYG{p}{(}\PYG{n}{url}\PYG{p}{)}
            \PYG{k}{except} \PYG{n+ne}{Exception} \PYG{k}{as} \PYG{n}{e}\PYG{p}{:}
                \PYG{n+nb}{print}\PYG{p}{(}\PYG{l+s+s2}{\PYGZdq{}}\PYG{l+s+s2}{Exception: }\PYG{l+s+si}{\PYGZpc{}s}\PYG{l+s+s2}{ }\PYG{l+s+si}{\PYGZpc{}s}\PYG{l+s+s2}{\PYGZdq{}} \PYG{o}{\PYGZpc{}} \PYG{p}{(}\PYG{n}{e}\PYG{p}{,} \PYG{n}{url}\PYG{p}{)}\PYG{p}{)}
            \PYG{k}{finally}\PYG{p}{:}
                \PYG{n}{q}\PYG{o}{.}\PYG{n}{task\PYGZus{}done}\PYG{p}{(}\PYG{p}{)}

    \PYG{k}{await} \PYG{n}{q}\PYG{o}{.}\PYG{n}{put}\PYG{p}{(}\PYG{n}{base\PYGZus{}url}\PYG{p}{)}

    \PYG{c+c1}{\PYGZsh{} Start workers, then wait for the work queue to be empty.}
    \PYG{n}{workers} \PYG{o}{=} \PYG{n}{gen}\PYG{o}{.}\PYG{n}{multi}\PYG{p}{(}\PYG{p}{[}\PYG{n}{worker}\PYG{p}{(}\PYG{p}{)} \PYG{k}{for} \PYG{n}{\PYGZus{}} \PYG{o+ow}{in} \PYG{n+nb}{range}\PYG{p}{(}\PYG{n}{concurrency}\PYG{p}{)}\PYG{p}{]}\PYG{p}{)}
    \PYG{k}{await} \PYG{n}{q}\PYG{o}{.}\PYG{n}{join}\PYG{p}{(}\PYG{n}{timeout}\PYG{o}{=}\PYG{n}{timedelta}\PYG{p}{(}\PYG{n}{seconds}\PYG{o}{=}\PYG{l+m+mi}{300}\PYG{p}{)}\PYG{p}{)}
    \PYG{k}{assert} \PYG{n}{fetching} \PYG{o}{==} \PYG{n}{fetched}
    \PYG{n+nb}{print}\PYG{p}{(}\PYG{l+s+s2}{\PYGZdq{}}\PYG{l+s+s2}{Done in }\PYG{l+s+si}{\PYGZpc{}d}\PYG{l+s+s2}{ seconds, fetched }\PYG{l+s+si}{\PYGZpc{}s}\PYG{l+s+s2}{ URLs.}\PYG{l+s+s2}{\PYGZdq{}} \PYG{o}{\PYGZpc{}} \PYG{p}{(}\PYG{n}{time}\PYG{o}{.}\PYG{n}{time}\PYG{p}{(}\PYG{p}{)} \PYG{o}{\PYGZhy{}} \PYG{n}{start}\PYG{p}{,} \PYG{n+nb}{len}\PYG{p}{(}\PYG{n}{fetched}\PYG{p}{)}\PYG{p}{)}\PYG{p}{)}

    \PYG{c+c1}{\PYGZsh{} Signal all the workers to exit.}
    \PYG{k}{for} \PYG{n}{\PYGZus{}} \PYG{o+ow}{in} \PYG{n+nb}{range}\PYG{p}{(}\PYG{n}{concurrency}\PYG{p}{)}\PYG{p}{:}
        \PYG{k}{await} \PYG{n}{q}\PYG{o}{.}\PYG{n}{put}\PYG{p}{(}\PYG{k+kc}{None}\PYG{p}{)}
    \PYG{k}{await} \PYG{n}{workers}


\PYG{k}{if} \PYG{n+nv+vm}{\PYGZus{}\PYGZus{}name\PYGZus{}\PYGZus{}} \PYG{o}{==} \PYG{l+s+s2}{\PYGZdq{}}\PYG{l+s+s2}{\PYGZus{}\PYGZus{}main\PYGZus{}\PYGZus{}}\PYG{l+s+s2}{\PYGZdq{}}\PYG{p}{:}
    \PYG{n}{io\PYGZus{}loop} \PYG{o}{=} \PYG{n}{ioloop}\PYG{o}{.}\PYG{n}{IOLoop}\PYG{o}{.}\PYG{n}{current}\PYG{p}{(}\PYG{p}{)}
    \PYG{n}{io\PYGZus{}loop}\PYG{o}{.}\PYG{n}{run\PYGZus{}sync}\PYG{p}{(}\PYG{n}{main}\PYG{p}{)}
\end{sphinxVerbatim}


\subsection{Structure of a Tornado web application}
\label{\detokenize{guide/structure:structure-of-a-tornado-web-application}}\label{\detokenize{guide/structure::doc}}
A Tornado web application generally consists of one or more
{\hyperref[\detokenize{web:tornado.web.RequestHandler}]{\sphinxcrossref{\sphinxcode{\sphinxupquote{RequestHandler}}}}} subclasses, an {\hyperref[\detokenize{web:tornado.web.Application}]{\sphinxcrossref{\sphinxcode{\sphinxupquote{Application}}}}} object which
routes incoming requests to handlers, and a \sphinxcode{\sphinxupquote{main()}} function
to start the server.

A minimal “hello world” example looks something like this:

\begin{sphinxVerbatim}[commandchars=\\\{\}]
\PYG{k+kn}{import} \PYG{n+nn}{tornado}\PYG{n+nn}{.}\PYG{n+nn}{ioloop}
\PYG{k+kn}{import} \PYG{n+nn}{tornado}\PYG{n+nn}{.}\PYG{n+nn}{web}

\PYG{k}{class} \PYG{n+nc}{MainHandler}\PYG{p}{(}\PYG{n}{tornado}\PYG{o}{.}\PYG{n}{web}\PYG{o}{.}\PYG{n}{RequestHandler}\PYG{p}{)}\PYG{p}{:}
    \PYG{k}{def} \PYG{n+nf}{get}\PYG{p}{(}\PYG{n+nb+bp}{self}\PYG{p}{)}\PYG{p}{:}
        \PYG{n+nb+bp}{self}\PYG{o}{.}\PYG{n}{write}\PYG{p}{(}\PYG{l+s+s2}{\PYGZdq{}}\PYG{l+s+s2}{Hello, world}\PYG{l+s+s2}{\PYGZdq{}}\PYG{p}{)}

\PYG{k}{def} \PYG{n+nf}{make\PYGZus{}app}\PYG{p}{(}\PYG{p}{)}\PYG{p}{:}
    \PYG{k}{return} \PYG{n}{tornado}\PYG{o}{.}\PYG{n}{web}\PYG{o}{.}\PYG{n}{Application}\PYG{p}{(}\PYG{p}{[}
        \PYG{p}{(}\PYG{l+s+sa}{r}\PYG{l+s+s2}{\PYGZdq{}}\PYG{l+s+s2}{/}\PYG{l+s+s2}{\PYGZdq{}}\PYG{p}{,} \PYG{n}{MainHandler}\PYG{p}{)}\PYG{p}{,}
    \PYG{p}{]}\PYG{p}{)}

\PYG{k}{if} \PYG{n+nv+vm}{\PYGZus{}\PYGZus{}name\PYGZus{}\PYGZus{}} \PYG{o}{==} \PYG{l+s+s2}{\PYGZdq{}}\PYG{l+s+s2}{\PYGZus{}\PYGZus{}main\PYGZus{}\PYGZus{}}\PYG{l+s+s2}{\PYGZdq{}}\PYG{p}{:}
    \PYG{n}{app} \PYG{o}{=} \PYG{n}{make\PYGZus{}app}\PYG{p}{(}\PYG{p}{)}
    \PYG{n}{app}\PYG{o}{.}\PYG{n}{listen}\PYG{p}{(}\PYG{l+m+mi}{8888}\PYG{p}{)}
    \PYG{n}{tornado}\PYG{o}{.}\PYG{n}{ioloop}\PYG{o}{.}\PYG{n}{IOLoop}\PYG{o}{.}\PYG{n}{current}\PYG{p}{(}\PYG{p}{)}\PYG{o}{.}\PYG{n}{start}\PYG{p}{(}\PYG{p}{)}
\end{sphinxVerbatim}


\subsubsection{The \sphinxstyleliteralintitle{\sphinxupquote{Application}} object}
\label{\detokenize{guide/structure:the-application-object}}
The {\hyperref[\detokenize{web:tornado.web.Application}]{\sphinxcrossref{\sphinxcode{\sphinxupquote{Application}}}}} object is responsible for global configuration, including
the routing table that maps requests to handlers.

The routing table is a list of {\hyperref[\detokenize{web:tornado.web.URLSpec}]{\sphinxcrossref{\sphinxcode{\sphinxupquote{URLSpec}}}}} objects (or tuples), each of
which contains (at least) a regular expression and a handler class.
Order matters; the first matching rule is used.  If the regular
expression contains capturing groups, these groups are the \sphinxstyleemphasis{path
arguments} and will be passed to the handler’s HTTP method.  If a
dictionary is passed as the third element of the {\hyperref[\detokenize{web:tornado.web.URLSpec}]{\sphinxcrossref{\sphinxcode{\sphinxupquote{URLSpec}}}}}, it
supplies the \sphinxstyleemphasis{initialization arguments} which will be passed to
{\hyperref[\detokenize{web:tornado.web.RequestHandler.initialize}]{\sphinxcrossref{\sphinxcode{\sphinxupquote{RequestHandler.initialize}}}}}.  Finally, the {\hyperref[\detokenize{web:tornado.web.URLSpec}]{\sphinxcrossref{\sphinxcode{\sphinxupquote{URLSpec}}}}} may have a
name, which will allow it to be used with
{\hyperref[\detokenize{web:tornado.web.RequestHandler.reverse_url}]{\sphinxcrossref{\sphinxcode{\sphinxupquote{RequestHandler.reverse\_url}}}}}.

For example, in this fragment the root URL \sphinxcode{\sphinxupquote{/}} is mapped to
\sphinxcode{\sphinxupquote{MainHandler}} and URLs of the form \sphinxcode{\sphinxupquote{/story/}} followed by a number
are mapped to \sphinxcode{\sphinxupquote{StoryHandler}}.  That number is passed (as a string) to
\sphinxcode{\sphinxupquote{StoryHandler.get}}.

\begin{sphinxVerbatim}[commandchars=\\\{\}]
\PYG{k}{class} \PYG{n+nc}{MainHandler}\PYG{p}{(}\PYG{n}{RequestHandler}\PYG{p}{)}\PYG{p}{:}
    \PYG{k}{def} \PYG{n+nf}{get}\PYG{p}{(}\PYG{n+nb+bp}{self}\PYG{p}{)}\PYG{p}{:}
        \PYG{n+nb+bp}{self}\PYG{o}{.}\PYG{n}{write}\PYG{p}{(}\PYG{l+s+s1}{\PYGZsq{}}\PYG{l+s+s1}{\PYGZlt{}a href=}\PYG{l+s+s1}{\PYGZdq{}}\PYG{l+s+si}{\PYGZpc{}s}\PYG{l+s+s1}{\PYGZdq{}}\PYG{l+s+s1}{\PYGZgt{}link to story 1\PYGZlt{}/a\PYGZgt{}}\PYG{l+s+s1}{\PYGZsq{}} \PYG{o}{\PYGZpc{}}
                   \PYG{n+nb+bp}{self}\PYG{o}{.}\PYG{n}{reverse\PYGZus{}url}\PYG{p}{(}\PYG{l+s+s2}{\PYGZdq{}}\PYG{l+s+s2}{story}\PYG{l+s+s2}{\PYGZdq{}}\PYG{p}{,} \PYG{l+s+s2}{\PYGZdq{}}\PYG{l+s+s2}{1}\PYG{l+s+s2}{\PYGZdq{}}\PYG{p}{)}\PYG{p}{)}

\PYG{k}{class} \PYG{n+nc}{StoryHandler}\PYG{p}{(}\PYG{n}{RequestHandler}\PYG{p}{)}\PYG{p}{:}
    \PYG{k}{def} \PYG{n+nf}{initialize}\PYG{p}{(}\PYG{n+nb+bp}{self}\PYG{p}{,} \PYG{n}{db}\PYG{p}{)}\PYG{p}{:}
        \PYG{n+nb+bp}{self}\PYG{o}{.}\PYG{n}{db} \PYG{o}{=} \PYG{n}{db}

    \PYG{k}{def} \PYG{n+nf}{get}\PYG{p}{(}\PYG{n+nb+bp}{self}\PYG{p}{,} \PYG{n}{story\PYGZus{}id}\PYG{p}{)}\PYG{p}{:}
        \PYG{n+nb+bp}{self}\PYG{o}{.}\PYG{n}{write}\PYG{p}{(}\PYG{l+s+s2}{\PYGZdq{}}\PYG{l+s+s2}{this is story }\PYG{l+s+si}{\PYGZpc{}s}\PYG{l+s+s2}{\PYGZdq{}} \PYG{o}{\PYGZpc{}} \PYG{n}{story\PYGZus{}id}\PYG{p}{)}

\PYG{n}{app} \PYG{o}{=} \PYG{n}{Application}\PYG{p}{(}\PYG{p}{[}
    \PYG{n}{url}\PYG{p}{(}\PYG{l+s+sa}{r}\PYG{l+s+s2}{\PYGZdq{}}\PYG{l+s+s2}{/}\PYG{l+s+s2}{\PYGZdq{}}\PYG{p}{,} \PYG{n}{MainHandler}\PYG{p}{)}\PYG{p}{,}
    \PYG{n}{url}\PYG{p}{(}\PYG{l+s+sa}{r}\PYG{l+s+s2}{\PYGZdq{}}\PYG{l+s+s2}{/story/([0\PYGZhy{}9]+)}\PYG{l+s+s2}{\PYGZdq{}}\PYG{p}{,} \PYG{n}{StoryHandler}\PYG{p}{,} \PYG{n+nb}{dict}\PYG{p}{(}\PYG{n}{db}\PYG{o}{=}\PYG{n}{db}\PYG{p}{)}\PYG{p}{,} \PYG{n}{name}\PYG{o}{=}\PYG{l+s+s2}{\PYGZdq{}}\PYG{l+s+s2}{story}\PYG{l+s+s2}{\PYGZdq{}}\PYG{p}{)}
    \PYG{p}{]}\PYG{p}{)}
\end{sphinxVerbatim}

The {\hyperref[\detokenize{web:tornado.web.Application}]{\sphinxcrossref{\sphinxcode{\sphinxupquote{Application}}}}} constructor takes many keyword arguments that
can be used to customize the behavior of the application and enable
optional features; see {\hyperref[\detokenize{web:tornado.web.Application.settings}]{\sphinxcrossref{\sphinxcode{\sphinxupquote{Application.settings}}}}} for the complete list.


\subsubsection{Subclassing \sphinxstyleliteralintitle{\sphinxupquote{RequestHandler}}}
\label{\detokenize{guide/structure:subclassing-requesthandler}}
Most of the work of a Tornado web application is done in subclasses
of {\hyperref[\detokenize{web:tornado.web.RequestHandler}]{\sphinxcrossref{\sphinxcode{\sphinxupquote{RequestHandler}}}}}.  The main entry point for a handler subclass
is a method named after the HTTP method being handled: \sphinxcode{\sphinxupquote{get()}},
\sphinxcode{\sphinxupquote{post()}}, etc.  Each handler may define one or more of these methods
to handle different HTTP actions.  As described above, these methods
will be called with arguments corresponding to the capturing groups
of the routing rule that matched.

Within a handler, call methods such as {\hyperref[\detokenize{web:tornado.web.RequestHandler.render}]{\sphinxcrossref{\sphinxcode{\sphinxupquote{RequestHandler.render}}}}} or
{\hyperref[\detokenize{web:tornado.web.RequestHandler.write}]{\sphinxcrossref{\sphinxcode{\sphinxupquote{RequestHandler.write}}}}} to produce a response.  \sphinxcode{\sphinxupquote{render()}} loads a
{\hyperref[\detokenize{template:tornado.template.Template}]{\sphinxcrossref{\sphinxcode{\sphinxupquote{Template}}}}} by name and renders it with the given
arguments. \sphinxcode{\sphinxupquote{write()}} is used for non-template-based output; it
accepts strings, bytes, and dictionaries (dicts will be encoded as
JSON).

Many methods in {\hyperref[\detokenize{web:tornado.web.RequestHandler}]{\sphinxcrossref{\sphinxcode{\sphinxupquote{RequestHandler}}}}} are designed to be overridden in
subclasses and be used throughout the application.  It is common
to define a \sphinxcode{\sphinxupquote{BaseHandler}} class that overrides methods such as
{\hyperref[\detokenize{web:tornado.web.RequestHandler.write_error}]{\sphinxcrossref{\sphinxcode{\sphinxupquote{write\_error}}}}} and {\hyperref[\detokenize{web:tornado.web.RequestHandler.get_current_user}]{\sphinxcrossref{\sphinxcode{\sphinxupquote{get\_current\_user}}}}}
and then subclass your own \sphinxcode{\sphinxupquote{BaseHandler}} instead of {\hyperref[\detokenize{web:tornado.web.RequestHandler}]{\sphinxcrossref{\sphinxcode{\sphinxupquote{RequestHandler}}}}}
for all your specific handlers.


\subsubsection{Handling request input}
\label{\detokenize{guide/structure:handling-request-input}}
The request handler can access the object representing the current
request with \sphinxcode{\sphinxupquote{self.request}}.  See the class definition for
{\hyperref[\detokenize{httputil:tornado.httputil.HTTPServerRequest}]{\sphinxcrossref{\sphinxcode{\sphinxupquote{HTTPServerRequest}}}}} for a complete list of
attributes.

Request data in the formats used by HTML forms will be parsed for you
and is made available in methods like {\hyperref[\detokenize{web:tornado.web.RequestHandler.get_query_argument}]{\sphinxcrossref{\sphinxcode{\sphinxupquote{get\_query\_argument}}}}}
and {\hyperref[\detokenize{web:tornado.web.RequestHandler.get_body_argument}]{\sphinxcrossref{\sphinxcode{\sphinxupquote{get\_body\_argument}}}}}.

\begin{sphinxVerbatim}[commandchars=\\\{\}]
\PYG{k}{class} \PYG{n+nc}{MyFormHandler}\PYG{p}{(}\PYG{n}{tornado}\PYG{o}{.}\PYG{n}{web}\PYG{o}{.}\PYG{n}{RequestHandler}\PYG{p}{)}\PYG{p}{:}
    \PYG{k}{def} \PYG{n+nf}{get}\PYG{p}{(}\PYG{n+nb+bp}{self}\PYG{p}{)}\PYG{p}{:}
        \PYG{n+nb+bp}{self}\PYG{o}{.}\PYG{n}{write}\PYG{p}{(}\PYG{l+s+s1}{\PYGZsq{}}\PYG{l+s+s1}{\PYGZlt{}html\PYGZgt{}\PYGZlt{}body\PYGZgt{}\PYGZlt{}form action=}\PYG{l+s+s1}{\PYGZdq{}}\PYG{l+s+s1}{/myform}\PYG{l+s+s1}{\PYGZdq{}}\PYG{l+s+s1}{ method=}\PYG{l+s+s1}{\PYGZdq{}}\PYG{l+s+s1}{POST}\PYG{l+s+s1}{\PYGZdq{}}\PYG{l+s+s1}{\PYGZgt{}}\PYG{l+s+s1}{\PYGZsq{}}
                   \PYG{l+s+s1}{\PYGZsq{}}\PYG{l+s+s1}{\PYGZlt{}input type=}\PYG{l+s+s1}{\PYGZdq{}}\PYG{l+s+s1}{text}\PYG{l+s+s1}{\PYGZdq{}}\PYG{l+s+s1}{ name=}\PYG{l+s+s1}{\PYGZdq{}}\PYG{l+s+s1}{message}\PYG{l+s+s1}{\PYGZdq{}}\PYG{l+s+s1}{\PYGZgt{}}\PYG{l+s+s1}{\PYGZsq{}}
                   \PYG{l+s+s1}{\PYGZsq{}}\PYG{l+s+s1}{\PYGZlt{}input type=}\PYG{l+s+s1}{\PYGZdq{}}\PYG{l+s+s1}{submit}\PYG{l+s+s1}{\PYGZdq{}}\PYG{l+s+s1}{ value=}\PYG{l+s+s1}{\PYGZdq{}}\PYG{l+s+s1}{Submit}\PYG{l+s+s1}{\PYGZdq{}}\PYG{l+s+s1}{\PYGZgt{}}\PYG{l+s+s1}{\PYGZsq{}}
                   \PYG{l+s+s1}{\PYGZsq{}}\PYG{l+s+s1}{\PYGZlt{}/form\PYGZgt{}\PYGZlt{}/body\PYGZgt{}\PYGZlt{}/html\PYGZgt{}}\PYG{l+s+s1}{\PYGZsq{}}\PYG{p}{)}

    \PYG{k}{def} \PYG{n+nf}{post}\PYG{p}{(}\PYG{n+nb+bp}{self}\PYG{p}{)}\PYG{p}{:}
        \PYG{n+nb+bp}{self}\PYG{o}{.}\PYG{n}{set\PYGZus{}header}\PYG{p}{(}\PYG{l+s+s2}{\PYGZdq{}}\PYG{l+s+s2}{Content\PYGZhy{}Type}\PYG{l+s+s2}{\PYGZdq{}}\PYG{p}{,} \PYG{l+s+s2}{\PYGZdq{}}\PYG{l+s+s2}{text/plain}\PYG{l+s+s2}{\PYGZdq{}}\PYG{p}{)}
        \PYG{n+nb+bp}{self}\PYG{o}{.}\PYG{n}{write}\PYG{p}{(}\PYG{l+s+s2}{\PYGZdq{}}\PYG{l+s+s2}{You wrote }\PYG{l+s+s2}{\PYGZdq{}} \PYG{o}{+} \PYG{n+nb+bp}{self}\PYG{o}{.}\PYG{n}{get\PYGZus{}body\PYGZus{}argument}\PYG{p}{(}\PYG{l+s+s2}{\PYGZdq{}}\PYG{l+s+s2}{message}\PYG{l+s+s2}{\PYGZdq{}}\PYG{p}{)}\PYG{p}{)}
\end{sphinxVerbatim}

Since the HTML form encoding is ambiguous as to whether an argument is
a single value or a list with one element, {\hyperref[\detokenize{web:tornado.web.RequestHandler}]{\sphinxcrossref{\sphinxcode{\sphinxupquote{RequestHandler}}}}} has
distinct methods to allow the application to indicate whether or not
it expects a list.  For lists, use
{\hyperref[\detokenize{web:tornado.web.RequestHandler.get_query_arguments}]{\sphinxcrossref{\sphinxcode{\sphinxupquote{get\_query\_arguments}}}}} and
{\hyperref[\detokenize{web:tornado.web.RequestHandler.get_body_arguments}]{\sphinxcrossref{\sphinxcode{\sphinxupquote{get\_body\_arguments}}}}} instead of their singular
counterparts.

Files uploaded via a form are available in \sphinxcode{\sphinxupquote{self.request.files}},
which maps names (the name of the HTML \sphinxcode{\sphinxupquote{\textless{}input type="file"\textgreater{}}}
element) to a list of files. Each file is a dictionary of the form
\sphinxcode{\sphinxupquote{\{"filename":..., "content\_type":..., "body":...\}}}.  The \sphinxcode{\sphinxupquote{files}}
object is only present if the files were uploaded with a form wrapper
(i.e. a \sphinxcode{\sphinxupquote{multipart/form-data}} Content-Type); if this format was not used
the raw uploaded data is available in \sphinxcode{\sphinxupquote{self.request.body}}.
By default uploaded files are fully buffered in memory; if you need to
handle files that are too large to comfortably keep in memory see the
{\hyperref[\detokenize{web:tornado.web.stream_request_body}]{\sphinxcrossref{\sphinxcode{\sphinxupquote{stream\_request\_body}}}}} class decorator.

In the demos directory,
\sphinxhref{https://github.com/tornadoweb/tornado/tree/master/demos/file\_upload/}{file\_receiver.py}
shows both methods of receiving file uploads.

Due to the quirks of the HTML form encoding (e.g. the ambiguity around
singular versus plural arguments), Tornado does not attempt to unify
form arguments with other types of input.  In particular, we do not
parse JSON request bodies.  Applications that wish to use JSON instead
of form-encoding may override {\hyperref[\detokenize{web:tornado.web.RequestHandler.prepare}]{\sphinxcrossref{\sphinxcode{\sphinxupquote{prepare}}}}} to parse their
requests:

\begin{sphinxVerbatim}[commandchars=\\\{\}]
\PYG{k}{def} \PYG{n+nf}{prepare}\PYG{p}{(}\PYG{n+nb+bp}{self}\PYG{p}{)}\PYG{p}{:}
    \PYG{k}{if} \PYG{n+nb+bp}{self}\PYG{o}{.}\PYG{n}{request}\PYG{o}{.}\PYG{n}{headers}\PYG{o}{.}\PYG{n}{get}\PYG{p}{(}\PYG{l+s+s2}{\PYGZdq{}}\PYG{l+s+s2}{Content\PYGZhy{}Type}\PYG{l+s+s2}{\PYGZdq{}}\PYG{p}{,} \PYG{l+s+s2}{\PYGZdq{}}\PYG{l+s+s2}{\PYGZdq{}}\PYG{p}{)}\PYG{o}{.}\PYG{n}{startswith}\PYG{p}{(}\PYG{l+s+s2}{\PYGZdq{}}\PYG{l+s+s2}{application/json}\PYG{l+s+s2}{\PYGZdq{}}\PYG{p}{)}\PYG{p}{:}
        \PYG{n+nb+bp}{self}\PYG{o}{.}\PYG{n}{json\PYGZus{}args} \PYG{o}{=} \PYG{n}{json}\PYG{o}{.}\PYG{n}{loads}\PYG{p}{(}\PYG{n+nb+bp}{self}\PYG{o}{.}\PYG{n}{request}\PYG{o}{.}\PYG{n}{body}\PYG{p}{)}
    \PYG{k}{else}\PYG{p}{:}
        \PYG{n+nb+bp}{self}\PYG{o}{.}\PYG{n}{json\PYGZus{}args} \PYG{o}{=} \PYG{k+kc}{None}
\end{sphinxVerbatim}


\subsubsection{Overriding RequestHandler methods}
\label{\detokenize{guide/structure:overriding-requesthandler-methods}}
In addition to \sphinxcode{\sphinxupquote{get()}}/\sphinxcode{\sphinxupquote{post()}}/etc, certain other methods in
{\hyperref[\detokenize{web:tornado.web.RequestHandler}]{\sphinxcrossref{\sphinxcode{\sphinxupquote{RequestHandler}}}}} are designed to be overridden by subclasses when
necessary. On every request, the following sequence of calls takes
place:
\begin{enumerate}
\def\theenumi{\arabic{enumi}}
\def\labelenumi{\theenumi .}
\makeatletter\def\p@enumii{\p@enumi \theenumi .}\makeatother
\item {} 
A new {\hyperref[\detokenize{web:tornado.web.RequestHandler}]{\sphinxcrossref{\sphinxcode{\sphinxupquote{RequestHandler}}}}} object is created on each request.

\item {} 
{\hyperref[\detokenize{web:tornado.web.RequestHandler.initialize}]{\sphinxcrossref{\sphinxcode{\sphinxupquote{initialize()}}}}} is called with the initialization
arguments from the {\hyperref[\detokenize{web:tornado.web.Application}]{\sphinxcrossref{\sphinxcode{\sphinxupquote{Application}}}}} configuration. \sphinxcode{\sphinxupquote{initialize}}
should typically just save the arguments passed into member
variables; it may not produce any output or call methods like
{\hyperref[\detokenize{web:tornado.web.RequestHandler.send_error}]{\sphinxcrossref{\sphinxcode{\sphinxupquote{send\_error}}}}}.

\item {} 
{\hyperref[\detokenize{web:tornado.web.RequestHandler.prepare}]{\sphinxcrossref{\sphinxcode{\sphinxupquote{prepare()}}}}} is called. This is most useful in a
base class shared by all of your handler subclasses, as \sphinxcode{\sphinxupquote{prepare}}
is called no matter which HTTP method is used. \sphinxcode{\sphinxupquote{prepare}} may
produce output; if it calls {\hyperref[\detokenize{web:tornado.web.RequestHandler.finish}]{\sphinxcrossref{\sphinxcode{\sphinxupquote{finish}}}}} (or
\sphinxcode{\sphinxupquote{redirect}}, etc), processing stops here.

\item {} 
One of the HTTP methods is called: \sphinxcode{\sphinxupquote{get()}}, \sphinxcode{\sphinxupquote{post()}}, \sphinxcode{\sphinxupquote{put()}},
etc. If the URL regular expression contains capturing groups, they
are passed as arguments to this method.

\item {} 
When the request is finished, {\hyperref[\detokenize{web:tornado.web.RequestHandler.on_finish}]{\sphinxcrossref{\sphinxcode{\sphinxupquote{on\_finish()}}}}} is
called. This is generally after \sphinxcode{\sphinxupquote{get()}} or another HTTP method
returns.

\end{enumerate}

All methods designed to be overridden are noted as such in the
{\hyperref[\detokenize{web:tornado.web.RequestHandler}]{\sphinxcrossref{\sphinxcode{\sphinxupquote{RequestHandler}}}}} documentation.  Some of the most commonly
overridden methods include:
\begin{itemize}
\item {} 
{\hyperref[\detokenize{web:tornado.web.RequestHandler.write_error}]{\sphinxcrossref{\sphinxcode{\sphinxupquote{write\_error}}}}} -
outputs HTML for use on error pages.

\item {} 
{\hyperref[\detokenize{web:tornado.web.RequestHandler.on_connection_close}]{\sphinxcrossref{\sphinxcode{\sphinxupquote{on\_connection\_close}}}}} - called when the client
disconnects; applications may choose to detect this case and halt
further processing.  Note that there is no guarantee that a closed
connection can be detected promptly.

\item {} 
{\hyperref[\detokenize{web:tornado.web.RequestHandler.get_current_user}]{\sphinxcrossref{\sphinxcode{\sphinxupquote{get\_current\_user}}}}} - see {\hyperref[\detokenize{guide/security:user-authentication}]{\sphinxcrossref{\DUrole{std,std-ref}{User authentication}}}}.

\item {} 
{\hyperref[\detokenize{web:tornado.web.RequestHandler.get_user_locale}]{\sphinxcrossref{\sphinxcode{\sphinxupquote{get\_user\_locale}}}}} - returns {\hyperref[\detokenize{locale:tornado.locale.Locale}]{\sphinxcrossref{\sphinxcode{\sphinxupquote{Locale}}}}} object to use
for the current user.

\item {} 
{\hyperref[\detokenize{web:tornado.web.RequestHandler.set_default_headers}]{\sphinxcrossref{\sphinxcode{\sphinxupquote{set\_default\_headers}}}}} - may be used to set
additional headers on the response (such as a custom \sphinxcode{\sphinxupquote{Server}}
header).

\end{itemize}


\subsubsection{Error Handling}
\label{\detokenize{guide/structure:error-handling}}
If a handler raises an exception, Tornado will call
{\hyperref[\detokenize{web:tornado.web.RequestHandler.write_error}]{\sphinxcrossref{\sphinxcode{\sphinxupquote{RequestHandler.write\_error}}}}} to generate an error page.
{\hyperref[\detokenize{web:tornado.web.HTTPError}]{\sphinxcrossref{\sphinxcode{\sphinxupquote{tornado.web.HTTPError}}}}} can be used to generate a specified status
code; all other exceptions return a 500 status.

The default error page includes a stack trace in debug mode and a
one-line description of the error (e.g. “500: Internal Server Error”)
otherwise.  To produce a custom error page, override
{\hyperref[\detokenize{web:tornado.web.RequestHandler.write_error}]{\sphinxcrossref{\sphinxcode{\sphinxupquote{RequestHandler.write\_error}}}}} (probably in a base class shared by all
your handlers).  This method may produce output normally via
methods such as {\hyperref[\detokenize{web:tornado.web.RequestHandler.write}]{\sphinxcrossref{\sphinxcode{\sphinxupquote{write}}}}} and {\hyperref[\detokenize{web:tornado.web.RequestHandler.render}]{\sphinxcrossref{\sphinxcode{\sphinxupquote{render}}}}}.
If the error was caused by an exception, an \sphinxcode{\sphinxupquote{exc\_info}} triple will
be passed as a keyword argument (note that this exception is not
guaranteed to be the current exception in \sphinxhref{https://docs.python.org/3.6/library/sys.html\#sys.exc\_info}{\sphinxcode{\sphinxupquote{sys.exc\_info}}}, so
\sphinxcode{\sphinxupquote{write\_error}} must use e.g.  \sphinxhref{https://docs.python.org/3.6/library/traceback.html\#traceback.format\_exception}{\sphinxcode{\sphinxupquote{traceback.format\_exception}}} instead of
\sphinxhref{https://docs.python.org/3.6/library/traceback.html\#traceback.format\_exc}{\sphinxcode{\sphinxupquote{traceback.format\_exc}}}).

It is also possible to generate an error page from regular handler
methods instead of \sphinxcode{\sphinxupquote{write\_error}} by calling
{\hyperref[\detokenize{web:tornado.web.RequestHandler.set_status}]{\sphinxcrossref{\sphinxcode{\sphinxupquote{set\_status}}}}}, writing a response, and returning.
The special exception {\hyperref[\detokenize{web:tornado.web.Finish}]{\sphinxcrossref{\sphinxcode{\sphinxupquote{tornado.web.Finish}}}}} may be raised to terminate
the handler without calling \sphinxcode{\sphinxupquote{write\_error}} in situations where simply
returning is not convenient.

For 404 errors, use the \sphinxcode{\sphinxupquote{default\_handler\_class}} {\hyperref[\detokenize{web:tornado.web.Application.settings}]{\sphinxcrossref{\sphinxcode{\sphinxupquote{Application setting}}}}}.  This handler should override
{\hyperref[\detokenize{web:tornado.web.RequestHandler.prepare}]{\sphinxcrossref{\sphinxcode{\sphinxupquote{prepare}}}}} instead of a more specific method like
\sphinxcode{\sphinxupquote{get()}} so it works with any HTTP method.  It should produce its
error page as described above: either by raising a \sphinxcode{\sphinxupquote{HTTPError(404)}}
and overriding \sphinxcode{\sphinxupquote{write\_error}}, or calling \sphinxcode{\sphinxupquote{self.set\_status(404)}}
and producing the response directly in \sphinxcode{\sphinxupquote{prepare()}}.


\subsubsection{Redirection}
\label{\detokenize{guide/structure:redirection}}
There are two main ways you can redirect requests in Tornado:
{\hyperref[\detokenize{web:tornado.web.RequestHandler.redirect}]{\sphinxcrossref{\sphinxcode{\sphinxupquote{RequestHandler.redirect}}}}} and with the {\hyperref[\detokenize{web:tornado.web.RedirectHandler}]{\sphinxcrossref{\sphinxcode{\sphinxupquote{RedirectHandler}}}}}.

You can use \sphinxcode{\sphinxupquote{self.redirect()}} within a {\hyperref[\detokenize{web:tornado.web.RequestHandler}]{\sphinxcrossref{\sphinxcode{\sphinxupquote{RequestHandler}}}}} method to
redirect users elsewhere. There is also an optional parameter
\sphinxcode{\sphinxupquote{permanent}} which you can use to indicate that the redirection is
considered permanent.  The default value of \sphinxcode{\sphinxupquote{permanent}} is
\sphinxcode{\sphinxupquote{False}}, which generates a \sphinxcode{\sphinxupquote{302 Found}} HTTP response code and is
appropriate for things like redirecting users after successful
\sphinxcode{\sphinxupquote{POST}} requests.  If \sphinxcode{\sphinxupquote{permanent}} is \sphinxcode{\sphinxupquote{True}}, the \sphinxcode{\sphinxupquote{301 Moved
Permanently}} HTTP response code is used, which is useful for
e.g. redirecting to a canonical URL for a page in an SEO-friendly
manner.

{\hyperref[\detokenize{web:tornado.web.RedirectHandler}]{\sphinxcrossref{\sphinxcode{\sphinxupquote{RedirectHandler}}}}} lets you configure redirects directly in your
{\hyperref[\detokenize{web:tornado.web.Application}]{\sphinxcrossref{\sphinxcode{\sphinxupquote{Application}}}}} routing table.  For example, to configure a single
static redirect:

\begin{sphinxVerbatim}[commandchars=\\\{\}]
\PYG{n}{app} \PYG{o}{=} \PYG{n}{tornado}\PYG{o}{.}\PYG{n}{web}\PYG{o}{.}\PYG{n}{Application}\PYG{p}{(}\PYG{p}{[}
    \PYG{n}{url}\PYG{p}{(}\PYG{l+s+sa}{r}\PYG{l+s+s2}{\PYGZdq{}}\PYG{l+s+s2}{/app}\PYG{l+s+s2}{\PYGZdq{}}\PYG{p}{,} \PYG{n}{tornado}\PYG{o}{.}\PYG{n}{web}\PYG{o}{.}\PYG{n}{RedirectHandler}\PYG{p}{,}
        \PYG{n+nb}{dict}\PYG{p}{(}\PYG{n}{url}\PYG{o}{=}\PYG{l+s+s2}{\PYGZdq{}}\PYG{l+s+s2}{http://itunes.apple.com/my\PYGZhy{}app\PYGZhy{}id}\PYG{l+s+s2}{\PYGZdq{}}\PYG{p}{)}\PYG{p}{)}\PYG{p}{,}
    \PYG{p}{]}\PYG{p}{)}
\end{sphinxVerbatim}

{\hyperref[\detokenize{web:tornado.web.RedirectHandler}]{\sphinxcrossref{\sphinxcode{\sphinxupquote{RedirectHandler}}}}} also supports regular expression substitutions.
The following rule redirects all requests beginning with \sphinxcode{\sphinxupquote{/pictures/}}
to the prefix \sphinxcode{\sphinxupquote{/photos/}} instead:

\begin{sphinxVerbatim}[commandchars=\\\{\}]
\PYG{n}{app} \PYG{o}{=} \PYG{n}{tornado}\PYG{o}{.}\PYG{n}{web}\PYG{o}{.}\PYG{n}{Application}\PYG{p}{(}\PYG{p}{[}
    \PYG{n}{url}\PYG{p}{(}\PYG{l+s+sa}{r}\PYG{l+s+s2}{\PYGZdq{}}\PYG{l+s+s2}{/photos/(.*)}\PYG{l+s+s2}{\PYGZdq{}}\PYG{p}{,} \PYG{n}{MyPhotoHandler}\PYG{p}{)}\PYG{p}{,}
    \PYG{n}{url}\PYG{p}{(}\PYG{l+s+sa}{r}\PYG{l+s+s2}{\PYGZdq{}}\PYG{l+s+s2}{/pictures/(.*)}\PYG{l+s+s2}{\PYGZdq{}}\PYG{p}{,} \PYG{n}{tornado}\PYG{o}{.}\PYG{n}{web}\PYG{o}{.}\PYG{n}{RedirectHandler}\PYG{p}{,}
        \PYG{n+nb}{dict}\PYG{p}{(}\PYG{n}{url}\PYG{o}{=}\PYG{l+s+sa}{r}\PYG{l+s+s2}{\PYGZdq{}}\PYG{l+s+s2}{/photos/}\PYG{l+s+si}{\PYGZob{}0\PYGZcb{}}\PYG{l+s+s2}{\PYGZdq{}}\PYG{p}{)}\PYG{p}{)}\PYG{p}{,}
    \PYG{p}{]}\PYG{p}{)}
\end{sphinxVerbatim}

Unlike {\hyperref[\detokenize{web:tornado.web.RequestHandler.redirect}]{\sphinxcrossref{\sphinxcode{\sphinxupquote{RequestHandler.redirect}}}}}, {\hyperref[\detokenize{web:tornado.web.RedirectHandler}]{\sphinxcrossref{\sphinxcode{\sphinxupquote{RedirectHandler}}}}} uses permanent
redirects by default.  This is because the routing table does not change
at runtime and is presumed to be permanent, while redirects found in
handlers are likely to be the result of other logic that may change.
To send a temporary redirect with a {\hyperref[\detokenize{web:tornado.web.RedirectHandler}]{\sphinxcrossref{\sphinxcode{\sphinxupquote{RedirectHandler}}}}}, add
\sphinxcode{\sphinxupquote{permanent=False}} to the {\hyperref[\detokenize{web:tornado.web.RedirectHandler}]{\sphinxcrossref{\sphinxcode{\sphinxupquote{RedirectHandler}}}}} initialization arguments.


\subsubsection{Asynchronous handlers}
\label{\detokenize{guide/structure:asynchronous-handlers}}
Certain handler methods (including \sphinxcode{\sphinxupquote{prepare()}} and the HTTP verb
methods \sphinxcode{\sphinxupquote{get()}}/\sphinxcode{\sphinxupquote{post()}}/etc) may be overridden as coroutines to
make the handler asynchronous.

For example, here is a simple handler using a coroutine:

\begin{sphinxVerbatim}[commandchars=\\\{\}]
\PYG{k}{class} \PYG{n+nc}{MainHandler}\PYG{p}{(}\PYG{n}{tornado}\PYG{o}{.}\PYG{n}{web}\PYG{o}{.}\PYG{n}{RequestHandler}\PYG{p}{)}\PYG{p}{:}
    \PYG{k}{async} \PYG{k}{def} \PYG{n+nf}{get}\PYG{p}{(}\PYG{n+nb+bp}{self}\PYG{p}{)}\PYG{p}{:}
        \PYG{n}{http} \PYG{o}{=} \PYG{n}{tornado}\PYG{o}{.}\PYG{n}{httpclient}\PYG{o}{.}\PYG{n}{AsyncHTTPClient}\PYG{p}{(}\PYG{p}{)}
        \PYG{n}{response} \PYG{o}{=} \PYG{k}{await} \PYG{n}{http}\PYG{o}{.}\PYG{n}{fetch}\PYG{p}{(}\PYG{l+s+s2}{\PYGZdq{}}\PYG{l+s+s2}{http://friendfeed\PYGZhy{}api.com/v2/feed/bret}\PYG{l+s+s2}{\PYGZdq{}}\PYG{p}{)}
        \PYG{n}{json} \PYG{o}{=} \PYG{n}{tornado}\PYG{o}{.}\PYG{n}{escape}\PYG{o}{.}\PYG{n}{json\PYGZus{}decode}\PYG{p}{(}\PYG{n}{response}\PYG{o}{.}\PYG{n}{body}\PYG{p}{)}
        \PYG{n+nb+bp}{self}\PYG{o}{.}\PYG{n}{write}\PYG{p}{(}\PYG{l+s+s2}{\PYGZdq{}}\PYG{l+s+s2}{Fetched }\PYG{l+s+s2}{\PYGZdq{}} \PYG{o}{+} \PYG{n+nb}{str}\PYG{p}{(}\PYG{n+nb}{len}\PYG{p}{(}\PYG{n}{json}\PYG{p}{[}\PYG{l+s+s2}{\PYGZdq{}}\PYG{l+s+s2}{entries}\PYG{l+s+s2}{\PYGZdq{}}\PYG{p}{]}\PYG{p}{)}\PYG{p}{)} \PYG{o}{+} \PYG{l+s+s2}{\PYGZdq{}}\PYG{l+s+s2}{ entries }\PYG{l+s+s2}{\PYGZdq{}}
                   \PYG{l+s+s2}{\PYGZdq{}}\PYG{l+s+s2}{from the FriendFeed API}\PYG{l+s+s2}{\PYGZdq{}}\PYG{p}{)}
\end{sphinxVerbatim}

For a more advanced asynchronous example, take a look at the \sphinxhref{https://github.com/tornadoweb/tornado/tree/stable/demos/chat}{chat
example application}, which
implements an AJAX chat room using \sphinxhref{http://en.wikipedia.org/wiki/Push\_technology\#Long\_polling}{long polling}.  Users
of long polling may want to override \sphinxcode{\sphinxupquote{on\_connection\_close()}} to
clean up after the client closes the connection (but see that method’s
docstring for caveats).


\subsection{Templates and UI}
\label{\detokenize{guide/templates:templates-and-ui}}\label{\detokenize{guide/templates::doc}}
Tornado includes a simple, fast, and flexible templating language.
This section describes that language as well as related issues
such as internationalization.

Tornado can also be used with any other Python template language,
although there is no provision for integrating these systems into
{\hyperref[\detokenize{web:tornado.web.RequestHandler.render}]{\sphinxcrossref{\sphinxcode{\sphinxupquote{RequestHandler.render}}}}}.  Simply render the template to a string
and pass it to {\hyperref[\detokenize{web:tornado.web.RequestHandler.write}]{\sphinxcrossref{\sphinxcode{\sphinxupquote{RequestHandler.write}}}}}


\subsubsection{Configuring templates}
\label{\detokenize{guide/templates:configuring-templates}}
By default, Tornado looks for template files in the same directory as
the \sphinxcode{\sphinxupquote{.py}} files that refer to them.  To put your template files in a
different directory, use the \sphinxcode{\sphinxupquote{template\_path}} {\hyperref[\detokenize{web:tornado.web.Application.settings}]{\sphinxcrossref{\sphinxcode{\sphinxupquote{Application setting}}}}} (or override {\hyperref[\detokenize{web:tornado.web.RequestHandler.get_template_path}]{\sphinxcrossref{\sphinxcode{\sphinxupquote{RequestHandler.get\_template\_path}}}}}
if you have different template paths for different handlers).

To load templates from a non-filesystem location, subclass
{\hyperref[\detokenize{template:tornado.template.BaseLoader}]{\sphinxcrossref{\sphinxcode{\sphinxupquote{tornado.template.BaseLoader}}}}} and pass an instance as the
\sphinxcode{\sphinxupquote{template\_loader}} application setting.

Compiled templates are cached by default; to turn off this caching
and reload templates so changes to the underlying files are always
visible, use the application settings \sphinxcode{\sphinxupquote{compiled\_template\_cache=False}}
or \sphinxcode{\sphinxupquote{debug=True}}.


\subsubsection{Template syntax}
\label{\detokenize{guide/templates:template-syntax}}
A Tornado template is just HTML (or any other text-based format) with
Python control sequences and expressions embedded within the markup:

\begin{sphinxVerbatim}[commandchars=\\\{\}]
\PYG{o}{\PYGZlt{}}\PYG{n}{html}\PYG{o}{\PYGZgt{}}
   \PYG{o}{\PYGZlt{}}\PYG{n}{head}\PYG{o}{\PYGZgt{}}
      \PYG{o}{\PYGZlt{}}\PYG{n}{title}\PYG{o}{\PYGZgt{}}\PYG{p}{\PYGZob{}}\PYG{p}{\PYGZob{}} \PYG{n}{title} \PYG{p}{\PYGZcb{}}\PYG{p}{\PYGZcb{}}\PYG{o}{\PYGZlt{}}\PYG{o}{/}\PYG{n}{title}\PYG{o}{\PYGZgt{}}
   \PYG{o}{\PYGZlt{}}\PYG{o}{/}\PYG{n}{head}\PYG{o}{\PYGZgt{}}
   \PYG{o}{\PYGZlt{}}\PYG{n}{body}\PYG{o}{\PYGZgt{}}
     \PYG{o}{\PYGZlt{}}\PYG{n}{ul}\PYG{o}{\PYGZgt{}}
       \PYG{p}{\PYGZob{}}\PYG{o}{\PYGZpc{}} \PYG{k}{for} \PYG{n}{item} \PYG{o+ow}{in} \PYG{n}{items} \PYG{o}{\PYGZpc{}}\PYG{p}{\PYGZcb{}}
         \PYG{o}{\PYGZlt{}}\PYG{n}{li}\PYG{o}{\PYGZgt{}}\PYG{p}{\PYGZob{}}\PYG{p}{\PYGZob{}} \PYG{n}{escape}\PYG{p}{(}\PYG{n}{item}\PYG{p}{)} \PYG{p}{\PYGZcb{}}\PYG{p}{\PYGZcb{}}\PYG{o}{\PYGZlt{}}\PYG{o}{/}\PYG{n}{li}\PYG{o}{\PYGZgt{}}
       \PYG{p}{\PYGZob{}}\PYG{o}{\PYGZpc{}} \PYG{n}{end} \PYG{o}{\PYGZpc{}}\PYG{p}{\PYGZcb{}}
     \PYG{o}{\PYGZlt{}}\PYG{o}{/}\PYG{n}{ul}\PYG{o}{\PYGZgt{}}
   \PYG{o}{\PYGZlt{}}\PYG{o}{/}\PYG{n}{body}\PYG{o}{\PYGZgt{}}
 \PYG{o}{\PYGZlt{}}\PYG{o}{/}\PYG{n}{html}\PYG{o}{\PYGZgt{}}
\end{sphinxVerbatim}

If you saved this template as “template.html” and put it in the same
directory as your Python file, you could render this template with:

\begin{sphinxVerbatim}[commandchars=\\\{\}]
\PYG{k}{class} \PYG{n+nc}{MainHandler}\PYG{p}{(}\PYG{n}{tornado}\PYG{o}{.}\PYG{n}{web}\PYG{o}{.}\PYG{n}{RequestHandler}\PYG{p}{)}\PYG{p}{:}
    \PYG{k}{def} \PYG{n+nf}{get}\PYG{p}{(}\PYG{n+nb+bp}{self}\PYG{p}{)}\PYG{p}{:}
        \PYG{n}{items} \PYG{o}{=} \PYG{p}{[}\PYG{l+s+s2}{\PYGZdq{}}\PYG{l+s+s2}{Item 1}\PYG{l+s+s2}{\PYGZdq{}}\PYG{p}{,} \PYG{l+s+s2}{\PYGZdq{}}\PYG{l+s+s2}{Item 2}\PYG{l+s+s2}{\PYGZdq{}}\PYG{p}{,} \PYG{l+s+s2}{\PYGZdq{}}\PYG{l+s+s2}{Item 3}\PYG{l+s+s2}{\PYGZdq{}}\PYG{p}{]}
        \PYG{n+nb+bp}{self}\PYG{o}{.}\PYG{n}{render}\PYG{p}{(}\PYG{l+s+s2}{\PYGZdq{}}\PYG{l+s+s2}{template.html}\PYG{l+s+s2}{\PYGZdq{}}\PYG{p}{,} \PYG{n}{title}\PYG{o}{=}\PYG{l+s+s2}{\PYGZdq{}}\PYG{l+s+s2}{My title}\PYG{l+s+s2}{\PYGZdq{}}\PYG{p}{,} \PYG{n}{items}\PYG{o}{=}\PYG{n}{items}\PYG{p}{)}
\end{sphinxVerbatim}

Tornado templates support \sphinxstyleemphasis{control statements} and \sphinxstyleemphasis{expressions}.
Control statements are surrounded by \sphinxcode{\sphinxupquote{\{\%}} and \sphinxcode{\sphinxupquote{\%\}}}, e.g.
\sphinxcode{\sphinxupquote{\{\% if len(items) \textgreater{} 2 \%\}}}. Expressions are surrounded by \sphinxcode{\sphinxupquote{\{\{}} and
\sphinxcode{\sphinxupquote{\}\}}}, e.g. \sphinxcode{\sphinxupquote{\{\{ items{[}0{]} \}\}}}.

Control statements more or less map exactly to Python statements. We
support \sphinxcode{\sphinxupquote{if}}, \sphinxcode{\sphinxupquote{for}}, \sphinxcode{\sphinxupquote{while}}, and \sphinxcode{\sphinxupquote{try}}, all of which are
terminated with \sphinxcode{\sphinxupquote{\{\% end \%\}}}. We also support \sphinxstyleemphasis{template inheritance}
using the \sphinxcode{\sphinxupquote{extends}} and \sphinxcode{\sphinxupquote{block}} statements, which are described in
detail in the documentation for the {\hyperref[\detokenize{template:module-tornado.template}]{\sphinxcrossref{\sphinxcode{\sphinxupquote{tornado.template}}}}}.

Expressions can be any Python expression, including function calls.
Template code is executed in a namespace that includes the following
objects and functions. (Note that this list applies to templates
rendered using {\hyperref[\detokenize{web:tornado.web.RequestHandler.render}]{\sphinxcrossref{\sphinxcode{\sphinxupquote{RequestHandler.render}}}}} and
{\hyperref[\detokenize{web:tornado.web.RequestHandler.render_string}]{\sphinxcrossref{\sphinxcode{\sphinxupquote{render\_string}}}}}. If you’re using the
{\hyperref[\detokenize{template:module-tornado.template}]{\sphinxcrossref{\sphinxcode{\sphinxupquote{tornado.template}}}}} module directly outside of a {\hyperref[\detokenize{web:tornado.web.RequestHandler}]{\sphinxcrossref{\sphinxcode{\sphinxupquote{RequestHandler}}}}} many
of these entries are not present).
\begin{itemize}
\item {} 
\sphinxcode{\sphinxupquote{escape}}: alias for {\hyperref[\detokenize{escape:tornado.escape.xhtml_escape}]{\sphinxcrossref{\sphinxcode{\sphinxupquote{tornado.escape.xhtml\_escape}}}}}

\item {} 
\sphinxcode{\sphinxupquote{xhtml\_escape}}: alias for {\hyperref[\detokenize{escape:tornado.escape.xhtml_escape}]{\sphinxcrossref{\sphinxcode{\sphinxupquote{tornado.escape.xhtml\_escape}}}}}

\item {} 
\sphinxcode{\sphinxupquote{url\_escape}}: alias for {\hyperref[\detokenize{escape:tornado.escape.url_escape}]{\sphinxcrossref{\sphinxcode{\sphinxupquote{tornado.escape.url\_escape}}}}}

\item {} 
\sphinxcode{\sphinxupquote{json\_encode}}: alias for {\hyperref[\detokenize{escape:tornado.escape.json_encode}]{\sphinxcrossref{\sphinxcode{\sphinxupquote{tornado.escape.json\_encode}}}}}

\item {} 
\sphinxcode{\sphinxupquote{squeeze}}: alias for {\hyperref[\detokenize{escape:tornado.escape.squeeze}]{\sphinxcrossref{\sphinxcode{\sphinxupquote{tornado.escape.squeeze}}}}}

\item {} 
\sphinxcode{\sphinxupquote{linkify}}: alias for {\hyperref[\detokenize{escape:tornado.escape.linkify}]{\sphinxcrossref{\sphinxcode{\sphinxupquote{tornado.escape.linkify}}}}}

\item {} 
\sphinxcode{\sphinxupquote{datetime}}: the Python \sphinxhref{https://docs.python.org/3.6/library/datetime.html\#module-datetime}{\sphinxcode{\sphinxupquote{datetime}}} module

\item {} 
\sphinxcode{\sphinxupquote{handler}}: the current {\hyperref[\detokenize{web:tornado.web.RequestHandler}]{\sphinxcrossref{\sphinxcode{\sphinxupquote{RequestHandler}}}}} object

\item {} 
\sphinxcode{\sphinxupquote{request}}: alias for {\hyperref[\detokenize{httputil:tornado.httputil.HTTPServerRequest}]{\sphinxcrossref{\sphinxcode{\sphinxupquote{handler.request}}}}}

\item {} 
\sphinxcode{\sphinxupquote{current\_user}}: alias for {\hyperref[\detokenize{web:tornado.web.RequestHandler.current_user}]{\sphinxcrossref{\sphinxcode{\sphinxupquote{handler.current\_user}}}}}

\item {} 
\sphinxcode{\sphinxupquote{locale}}: alias for {\hyperref[\detokenize{locale:tornado.locale.Locale}]{\sphinxcrossref{\sphinxcode{\sphinxupquote{handler.locale}}}}}

\item {} 
\sphinxcode{\sphinxupquote{\_}}: alias for {\hyperref[\detokenize{locale:tornado.locale.Locale.translate}]{\sphinxcrossref{\sphinxcode{\sphinxupquote{handler.locale.translate}}}}}

\item {} 
\sphinxcode{\sphinxupquote{static\_url}}: alias for {\hyperref[\detokenize{web:tornado.web.RequestHandler.static_url}]{\sphinxcrossref{\sphinxcode{\sphinxupquote{handler.static\_url}}}}}

\item {} 
\sphinxcode{\sphinxupquote{xsrf\_form\_html}}: alias for {\hyperref[\detokenize{web:tornado.web.RequestHandler.xsrf_form_html}]{\sphinxcrossref{\sphinxcode{\sphinxupquote{handler.xsrf\_form\_html}}}}}

\item {} 
\sphinxcode{\sphinxupquote{reverse\_url}}: alias for {\hyperref[\detokenize{web:tornado.web.Application.reverse_url}]{\sphinxcrossref{\sphinxcode{\sphinxupquote{Application.reverse\_url}}}}}

\item {} 
All entries from the \sphinxcode{\sphinxupquote{ui\_methods}} and \sphinxcode{\sphinxupquote{ui\_modules}}
\sphinxcode{\sphinxupquote{Application}} settings

\item {} 
Any keyword arguments passed to {\hyperref[\detokenize{web:tornado.web.RequestHandler.render}]{\sphinxcrossref{\sphinxcode{\sphinxupquote{render}}}}} or
{\hyperref[\detokenize{web:tornado.web.RequestHandler.render_string}]{\sphinxcrossref{\sphinxcode{\sphinxupquote{render\_string}}}}}

\end{itemize}

When you are building a real application, you are going to want to use
all of the features of Tornado templates, especially template
inheritance. Read all about those features in the {\hyperref[\detokenize{template:module-tornado.template}]{\sphinxcrossref{\sphinxcode{\sphinxupquote{tornado.template}}}}}
section (some features, including \sphinxcode{\sphinxupquote{UIModules}} are implemented in the
{\hyperref[\detokenize{web:module-tornado.web}]{\sphinxcrossref{\sphinxcode{\sphinxupquote{tornado.web}}}}} module)

Under the hood, Tornado templates are translated directly to Python. The
expressions you include in your template are copied verbatim into a
Python function representing your template. We don’t try to prevent
anything in the template language; we created it explicitly to provide
the flexibility that other, stricter templating systems prevent.
Consequently, if you write random stuff inside of your template
expressions, you will get random Python errors when you execute the
template.

All template output is escaped by default, using the
{\hyperref[\detokenize{escape:tornado.escape.xhtml_escape}]{\sphinxcrossref{\sphinxcode{\sphinxupquote{tornado.escape.xhtml\_escape}}}}} function. This behavior can be changed
globally by passing \sphinxcode{\sphinxupquote{autoescape=None}} to the {\hyperref[\detokenize{web:tornado.web.Application}]{\sphinxcrossref{\sphinxcode{\sphinxupquote{Application}}}}} or
{\hyperref[\detokenize{template:tornado.template.Loader}]{\sphinxcrossref{\sphinxcode{\sphinxupquote{tornado.template.Loader}}}}} constructors, for a template file with the
\sphinxcode{\sphinxupquote{\{\% autoescape None \%\}}} directive, or for a single expression by
replacing \sphinxcode{\sphinxupquote{\{\{ ... \}\}}} with \sphinxcode{\sphinxupquote{\{\% raw ...\%\}}}. Additionally, in each of
these places the name of an alternative escaping function may be used
instead of \sphinxcode{\sphinxupquote{None}}.

Note that while Tornado’s automatic escaping is helpful in avoiding
XSS vulnerabilities, it is not sufficient in all cases.  Expressions
that appear in certain locations, such as in Javascript or CSS, may need
additional escaping.  Additionally, either care must be taken to always
use double quotes and {\hyperref[\detokenize{escape:tornado.escape.xhtml_escape}]{\sphinxcrossref{\sphinxcode{\sphinxupquote{xhtml\_escape}}}}} in HTML attributes that may contain
untrusted content, or a separate escaping function must be used for
attributes (see e.g.
\sphinxhref{http://wonko.com/post/html-escaping}{this blog post}).


\subsubsection{Internationalization}
\label{\detokenize{guide/templates:internationalization}}
The locale of the current user (whether they are logged in or not) is
always available as \sphinxcode{\sphinxupquote{self.locale}} in the request handler and as
\sphinxcode{\sphinxupquote{locale}} in templates. The name of the locale (e.g., \sphinxcode{\sphinxupquote{en\_US}}) is
available as \sphinxcode{\sphinxupquote{locale.name}}, and you can translate strings with the
{\hyperref[\detokenize{locale:tornado.locale.Locale.translate}]{\sphinxcrossref{\sphinxcode{\sphinxupquote{Locale.translate}}}}} method. Templates also have the global function
call \sphinxcode{\sphinxupquote{\_()}} available for string translation. The translate function
has two forms:

\begin{sphinxVerbatim}[commandchars=\\\{\}]
\PYG{n}{\PYGZus{}}\PYG{p}{(}\PYG{l+s+s2}{\PYGZdq{}}\PYG{l+s+s2}{Translate this string}\PYG{l+s+s2}{\PYGZdq{}}\PYG{p}{)}
\end{sphinxVerbatim}

which translates the string directly based on the current locale, and:

\begin{sphinxVerbatim}[commandchars=\\\{\}]
\PYG{n}{\PYGZus{}}\PYG{p}{(}\PYG{l+s+s2}{\PYGZdq{}}\PYG{l+s+s2}{A person liked this}\PYG{l+s+s2}{\PYGZdq{}}\PYG{p}{,} \PYG{l+s+s2}{\PYGZdq{}}\PYG{l+s+si}{\PYGZpc{}(num)d}\PYG{l+s+s2}{ people liked this}\PYG{l+s+s2}{\PYGZdq{}}\PYG{p}{,}
  \PYG{n+nb}{len}\PYG{p}{(}\PYG{n}{people}\PYG{p}{)}\PYG{p}{)} \PYG{o}{\PYGZpc{}} \PYG{p}{\PYGZob{}}\PYG{l+s+s2}{\PYGZdq{}}\PYG{l+s+s2}{num}\PYG{l+s+s2}{\PYGZdq{}}\PYG{p}{:} \PYG{n+nb}{len}\PYG{p}{(}\PYG{n}{people}\PYG{p}{)}\PYG{p}{\PYGZcb{}}
\end{sphinxVerbatim}

which translates a string that can be singular or plural based on the
value of the third argument. In the example above, a translation of the
first string will be returned if \sphinxcode{\sphinxupquote{len(people)}} is \sphinxcode{\sphinxupquote{1}}, or a
translation of the second string will be returned otherwise.

The most common pattern for translations is to use Python named
placeholders for variables (the \sphinxcode{\sphinxupquote{\%(num)d}} in the example above) since
placeholders can move around on translation.

Here is a properly internationalized template:

\begin{sphinxVerbatim}[commandchars=\\\{\}]
\PYG{o}{\PYGZlt{}}\PYG{n}{html}\PYG{o}{\PYGZgt{}}
   \PYG{o}{\PYGZlt{}}\PYG{n}{head}\PYG{o}{\PYGZgt{}}
      \PYG{o}{\PYGZlt{}}\PYG{n}{title}\PYG{o}{\PYGZgt{}}\PYG{n}{FriendFeed} \PYG{o}{\PYGZhy{}} \PYG{p}{\PYGZob{}}\PYG{p}{\PYGZob{}} \PYG{n}{\PYGZus{}}\PYG{p}{(}\PYG{l+s+s2}{\PYGZdq{}}\PYG{l+s+s2}{Sign in}\PYG{l+s+s2}{\PYGZdq{}}\PYG{p}{)} \PYG{p}{\PYGZcb{}}\PYG{p}{\PYGZcb{}}\PYG{o}{\PYGZlt{}}\PYG{o}{/}\PYG{n}{title}\PYG{o}{\PYGZgt{}}
   \PYG{o}{\PYGZlt{}}\PYG{o}{/}\PYG{n}{head}\PYG{o}{\PYGZgt{}}
   \PYG{o}{\PYGZlt{}}\PYG{n}{body}\PYG{o}{\PYGZgt{}}
     \PYG{o}{\PYGZlt{}}\PYG{n}{form} \PYG{n}{action}\PYG{o}{=}\PYG{l+s+s2}{\PYGZdq{}}\PYG{l+s+s2}{\PYGZob{}\PYGZob{}}\PYG{l+s+s2}{ request.path \PYGZcb{}\PYGZcb{}}\PYG{l+s+s2}{\PYGZdq{}} \PYG{n}{method}\PYG{o}{=}\PYG{l+s+s2}{\PYGZdq{}}\PYG{l+s+s2}{post}\PYG{l+s+s2}{\PYGZdq{}}\PYG{o}{\PYGZgt{}}
       \PYG{o}{\PYGZlt{}}\PYG{n}{div}\PYG{o}{\PYGZgt{}}\PYG{p}{\PYGZob{}}\PYG{p}{\PYGZob{}} \PYG{n}{\PYGZus{}}\PYG{p}{(}\PYG{l+s+s2}{\PYGZdq{}}\PYG{l+s+s2}{Username}\PYG{l+s+s2}{\PYGZdq{}}\PYG{p}{)} \PYG{p}{\PYGZcb{}}\PYG{p}{\PYGZcb{}} \PYG{o}{\PYGZlt{}}\PYG{n+nb}{input} \PYG{n+nb}{type}\PYG{o}{=}\PYG{l+s+s2}{\PYGZdq{}}\PYG{l+s+s2}{text}\PYG{l+s+s2}{\PYGZdq{}} \PYG{n}{name}\PYG{o}{=}\PYG{l+s+s2}{\PYGZdq{}}\PYG{l+s+s2}{username}\PYG{l+s+s2}{\PYGZdq{}}\PYG{o}{/}\PYG{o}{\PYGZgt{}}\PYG{o}{\PYGZlt{}}\PYG{o}{/}\PYG{n}{div}\PYG{o}{\PYGZgt{}}
       \PYG{o}{\PYGZlt{}}\PYG{n}{div}\PYG{o}{\PYGZgt{}}\PYG{p}{\PYGZob{}}\PYG{p}{\PYGZob{}} \PYG{n}{\PYGZus{}}\PYG{p}{(}\PYG{l+s+s2}{\PYGZdq{}}\PYG{l+s+s2}{Password}\PYG{l+s+s2}{\PYGZdq{}}\PYG{p}{)} \PYG{p}{\PYGZcb{}}\PYG{p}{\PYGZcb{}} \PYG{o}{\PYGZlt{}}\PYG{n+nb}{input} \PYG{n+nb}{type}\PYG{o}{=}\PYG{l+s+s2}{\PYGZdq{}}\PYG{l+s+s2}{password}\PYG{l+s+s2}{\PYGZdq{}} \PYG{n}{name}\PYG{o}{=}\PYG{l+s+s2}{\PYGZdq{}}\PYG{l+s+s2}{password}\PYG{l+s+s2}{\PYGZdq{}}\PYG{o}{/}\PYG{o}{\PYGZgt{}}\PYG{o}{\PYGZlt{}}\PYG{o}{/}\PYG{n}{div}\PYG{o}{\PYGZgt{}}
       \PYG{o}{\PYGZlt{}}\PYG{n}{div}\PYG{o}{\PYGZgt{}}\PYG{o}{\PYGZlt{}}\PYG{n+nb}{input} \PYG{n+nb}{type}\PYG{o}{=}\PYG{l+s+s2}{\PYGZdq{}}\PYG{l+s+s2}{submit}\PYG{l+s+s2}{\PYGZdq{}} \PYG{n}{value}\PYG{o}{=}\PYG{l+s+s2}{\PYGZdq{}}\PYG{l+s+s2}{\PYGZob{}\PYGZob{}}\PYG{l+s+s2}{ \PYGZus{}(}\PYG{l+s+s2}{\PYGZdq{}}\PYG{n}{Sign} \PYG{o+ow}{in}\PYG{l+s+s2}{\PYGZdq{}}\PYG{l+s+s2}{) \PYGZcb{}\PYGZcb{}}\PYG{l+s+s2}{\PYGZdq{}}\PYG{o}{/}\PYG{o}{\PYGZgt{}}\PYG{o}{\PYGZlt{}}\PYG{o}{/}\PYG{n}{div}\PYG{o}{\PYGZgt{}}
       \PYG{p}{\PYGZob{}}\PYG{o}{\PYGZpc{}} \PYG{n}{module} \PYG{n}{xsrf\PYGZus{}form\PYGZus{}html}\PYG{p}{(}\PYG{p}{)} \PYG{o}{\PYGZpc{}}\PYG{p}{\PYGZcb{}}
     \PYG{o}{\PYGZlt{}}\PYG{o}{/}\PYG{n}{form}\PYG{o}{\PYGZgt{}}
   \PYG{o}{\PYGZlt{}}\PYG{o}{/}\PYG{n}{body}\PYG{o}{\PYGZgt{}}
 \PYG{o}{\PYGZlt{}}\PYG{o}{/}\PYG{n}{html}\PYG{o}{\PYGZgt{}}
\end{sphinxVerbatim}

By default, we detect the user’s locale using the \sphinxcode{\sphinxupquote{Accept-Language}}
header sent by the user’s browser. We choose \sphinxcode{\sphinxupquote{en\_US}} if we can’t find
an appropriate \sphinxcode{\sphinxupquote{Accept-Language}} value. If you let user’s set their
locale as a preference, you can override this default locale selection
by overriding {\hyperref[\detokenize{web:tornado.web.RequestHandler.get_user_locale}]{\sphinxcrossref{\sphinxcode{\sphinxupquote{RequestHandler.get\_user\_locale}}}}}:

\begin{sphinxVerbatim}[commandchars=\\\{\}]
\PYG{k}{class} \PYG{n+nc}{BaseHandler}\PYG{p}{(}\PYG{n}{tornado}\PYG{o}{.}\PYG{n}{web}\PYG{o}{.}\PYG{n}{RequestHandler}\PYG{p}{)}\PYG{p}{:}
    \PYG{k}{def} \PYG{n+nf}{get\PYGZus{}current\PYGZus{}user}\PYG{p}{(}\PYG{n+nb+bp}{self}\PYG{p}{)}\PYG{p}{:}
        \PYG{n}{user\PYGZus{}id} \PYG{o}{=} \PYG{n+nb+bp}{self}\PYG{o}{.}\PYG{n}{get\PYGZus{}secure\PYGZus{}cookie}\PYG{p}{(}\PYG{l+s+s2}{\PYGZdq{}}\PYG{l+s+s2}{user}\PYG{l+s+s2}{\PYGZdq{}}\PYG{p}{)}
        \PYG{k}{if} \PYG{o+ow}{not} \PYG{n}{user\PYGZus{}id}\PYG{p}{:} \PYG{k}{return} \PYG{k+kc}{None}
        \PYG{k}{return} \PYG{n+nb+bp}{self}\PYG{o}{.}\PYG{n}{backend}\PYG{o}{.}\PYG{n}{get\PYGZus{}user\PYGZus{}by\PYGZus{}id}\PYG{p}{(}\PYG{n}{user\PYGZus{}id}\PYG{p}{)}

    \PYG{k}{def} \PYG{n+nf}{get\PYGZus{}user\PYGZus{}locale}\PYG{p}{(}\PYG{n+nb+bp}{self}\PYG{p}{)}\PYG{p}{:}
        \PYG{k}{if} \PYG{l+s+s2}{\PYGZdq{}}\PYG{l+s+s2}{locale}\PYG{l+s+s2}{\PYGZdq{}} \PYG{o+ow}{not} \PYG{o+ow}{in} \PYG{n+nb+bp}{self}\PYG{o}{.}\PYG{n}{current\PYGZus{}user}\PYG{o}{.}\PYG{n}{prefs}\PYG{p}{:}
            \PYG{c+c1}{\PYGZsh{} Use the Accept\PYGZhy{}Language header}
            \PYG{k}{return} \PYG{k+kc}{None}
        \PYG{k}{return} \PYG{n+nb+bp}{self}\PYG{o}{.}\PYG{n}{current\PYGZus{}user}\PYG{o}{.}\PYG{n}{prefs}\PYG{p}{[}\PYG{l+s+s2}{\PYGZdq{}}\PYG{l+s+s2}{locale}\PYG{l+s+s2}{\PYGZdq{}}\PYG{p}{]}
\end{sphinxVerbatim}

If \sphinxcode{\sphinxupquote{get\_user\_locale}} returns \sphinxcode{\sphinxupquote{None}}, we fall back on the
\sphinxcode{\sphinxupquote{Accept-Language}} header.

The {\hyperref[\detokenize{locale:module-tornado.locale}]{\sphinxcrossref{\sphinxcode{\sphinxupquote{tornado.locale}}}}} module supports loading translations in two
formats: the \sphinxcode{\sphinxupquote{.mo}} format used by \sphinxhref{https://docs.python.org/3.6/library/gettext.html\#module-gettext}{\sphinxcode{\sphinxupquote{gettext}}} and related tools, and a
simple \sphinxcode{\sphinxupquote{.csv}} format.  An application will generally call either
{\hyperref[\detokenize{locale:tornado.locale.load_translations}]{\sphinxcrossref{\sphinxcode{\sphinxupquote{tornado.locale.load\_translations}}}}} or
{\hyperref[\detokenize{locale:tornado.locale.load_gettext_translations}]{\sphinxcrossref{\sphinxcode{\sphinxupquote{tornado.locale.load\_gettext\_translations}}}}} once at startup; see those
methods for more details on the supported formats.

You can get the list of supported locales in your application with
{\hyperref[\detokenize{locale:tornado.locale.get_supported_locales}]{\sphinxcrossref{\sphinxcode{\sphinxupquote{tornado.locale.get\_supported\_locales()}}}}}. The user’s locale is chosen
to be the closest match based on the supported locales. For example, if
the user’s locale is \sphinxcode{\sphinxupquote{es\_GT}}, and the \sphinxcode{\sphinxupquote{es}} locale is supported,
\sphinxcode{\sphinxupquote{self.locale}} will be \sphinxcode{\sphinxupquote{es}} for that request. We fall back on
\sphinxcode{\sphinxupquote{en\_US}} if no close match can be found.


\subsubsection{UI modules}
\label{\detokenize{guide/templates:ui-modules}}\label{\detokenize{guide/templates:id1}}
Tornado supports \sphinxstyleemphasis{UI modules} to make it easy to support standard,
reusable UI widgets across your application. UI modules are like special
function calls to render components of your page, and they can come
packaged with their own CSS and JavaScript.

For example, if you are implementing a blog, and you want to have blog
entries appear on both the blog home page and on each blog entry page,
you can make an \sphinxcode{\sphinxupquote{Entry}} module to render them on both pages. First,
create a Python module for your UI modules, e.g. \sphinxcode{\sphinxupquote{uimodules.py}}:

\begin{sphinxVerbatim}[commandchars=\\\{\}]
\PYG{k}{class} \PYG{n+nc}{Entry}\PYG{p}{(}\PYG{n}{tornado}\PYG{o}{.}\PYG{n}{web}\PYG{o}{.}\PYG{n}{UIModule}\PYG{p}{)}\PYG{p}{:}
    \PYG{k}{def} \PYG{n+nf}{render}\PYG{p}{(}\PYG{n+nb+bp}{self}\PYG{p}{,} \PYG{n}{entry}\PYG{p}{,} \PYG{n}{show\PYGZus{}comments}\PYG{o}{=}\PYG{k+kc}{False}\PYG{p}{)}\PYG{p}{:}
        \PYG{k}{return} \PYG{n+nb+bp}{self}\PYG{o}{.}\PYG{n}{render\PYGZus{}string}\PYG{p}{(}
            \PYG{l+s+s2}{\PYGZdq{}}\PYG{l+s+s2}{module\PYGZhy{}entry.html}\PYG{l+s+s2}{\PYGZdq{}}\PYG{p}{,} \PYG{n}{entry}\PYG{o}{=}\PYG{n}{entry}\PYG{p}{,} \PYG{n}{show\PYGZus{}comments}\PYG{o}{=}\PYG{n}{show\PYGZus{}comments}\PYG{p}{)}
\end{sphinxVerbatim}

Tell Tornado to use \sphinxcode{\sphinxupquote{uimodules.py}} using the \sphinxcode{\sphinxupquote{ui\_modules}} setting in
your application:

\begin{sphinxVerbatim}[commandchars=\\\{\}]
\PYG{k+kn}{from} \PYG{n+nn}{.} \PYG{k}{import} \PYG{n}{uimodules}

\PYG{k}{class} \PYG{n+nc}{HomeHandler}\PYG{p}{(}\PYG{n}{tornado}\PYG{o}{.}\PYG{n}{web}\PYG{o}{.}\PYG{n}{RequestHandler}\PYG{p}{)}\PYG{p}{:}
    \PYG{k}{def} \PYG{n+nf}{get}\PYG{p}{(}\PYG{n+nb+bp}{self}\PYG{p}{)}\PYG{p}{:}
        \PYG{n}{entries} \PYG{o}{=} \PYG{n+nb+bp}{self}\PYG{o}{.}\PYG{n}{db}\PYG{o}{.}\PYG{n}{query}\PYG{p}{(}\PYG{l+s+s2}{\PYGZdq{}}\PYG{l+s+s2}{SELECT * FROM entries ORDER BY date DESC}\PYG{l+s+s2}{\PYGZdq{}}\PYG{p}{)}
        \PYG{n+nb+bp}{self}\PYG{o}{.}\PYG{n}{render}\PYG{p}{(}\PYG{l+s+s2}{\PYGZdq{}}\PYG{l+s+s2}{home.html}\PYG{l+s+s2}{\PYGZdq{}}\PYG{p}{,} \PYG{n}{entries}\PYG{o}{=}\PYG{n}{entries}\PYG{p}{)}

\PYG{k}{class} \PYG{n+nc}{EntryHandler}\PYG{p}{(}\PYG{n}{tornado}\PYG{o}{.}\PYG{n}{web}\PYG{o}{.}\PYG{n}{RequestHandler}\PYG{p}{)}\PYG{p}{:}
    \PYG{k}{def} \PYG{n+nf}{get}\PYG{p}{(}\PYG{n+nb+bp}{self}\PYG{p}{,} \PYG{n}{entry\PYGZus{}id}\PYG{p}{)}\PYG{p}{:}
        \PYG{n}{entry} \PYG{o}{=} \PYG{n+nb+bp}{self}\PYG{o}{.}\PYG{n}{db}\PYG{o}{.}\PYG{n}{get}\PYG{p}{(}\PYG{l+s+s2}{\PYGZdq{}}\PYG{l+s+s2}{SELECT * FROM entries WHERE id = }\PYG{l+s+si}{\PYGZpc{}s}\PYG{l+s+s2}{\PYGZdq{}}\PYG{p}{,} \PYG{n}{entry\PYGZus{}id}\PYG{p}{)}
        \PYG{k}{if} \PYG{o+ow}{not} \PYG{n}{entry}\PYG{p}{:} \PYG{k}{raise} \PYG{n}{tornado}\PYG{o}{.}\PYG{n}{web}\PYG{o}{.}\PYG{n}{HTTPError}\PYG{p}{(}\PYG{l+m+mi}{404}\PYG{p}{)}
        \PYG{n+nb+bp}{self}\PYG{o}{.}\PYG{n}{render}\PYG{p}{(}\PYG{l+s+s2}{\PYGZdq{}}\PYG{l+s+s2}{entry.html}\PYG{l+s+s2}{\PYGZdq{}}\PYG{p}{,} \PYG{n}{entry}\PYG{o}{=}\PYG{n}{entry}\PYG{p}{)}

\PYG{n}{settings} \PYG{o}{=} \PYG{p}{\PYGZob{}}
    \PYG{l+s+s2}{\PYGZdq{}}\PYG{l+s+s2}{ui\PYGZus{}modules}\PYG{l+s+s2}{\PYGZdq{}}\PYG{p}{:} \PYG{n}{uimodules}\PYG{p}{,}
\PYG{p}{\PYGZcb{}}
\PYG{n}{application} \PYG{o}{=} \PYG{n}{tornado}\PYG{o}{.}\PYG{n}{web}\PYG{o}{.}\PYG{n}{Application}\PYG{p}{(}\PYG{p}{[}
    \PYG{p}{(}\PYG{l+s+sa}{r}\PYG{l+s+s2}{\PYGZdq{}}\PYG{l+s+s2}{/}\PYG{l+s+s2}{\PYGZdq{}}\PYG{p}{,} \PYG{n}{HomeHandler}\PYG{p}{)}\PYG{p}{,}
    \PYG{p}{(}\PYG{l+s+sa}{r}\PYG{l+s+s2}{\PYGZdq{}}\PYG{l+s+s2}{/entry/([0\PYGZhy{}9]+)}\PYG{l+s+s2}{\PYGZdq{}}\PYG{p}{,} \PYG{n}{EntryHandler}\PYG{p}{)}\PYG{p}{,}
\PYG{p}{]}\PYG{p}{,} \PYG{o}{*}\PYG{o}{*}\PYG{n}{settings}\PYG{p}{)}
\end{sphinxVerbatim}

Within a template, you can call a module with the \sphinxcode{\sphinxupquote{\{\% module \%\}}}
statement.  For example, you could call the \sphinxcode{\sphinxupquote{Entry}} module from both
\sphinxcode{\sphinxupquote{home.html}}:

\begin{sphinxVerbatim}[commandchars=\\\{\}]
\PYG{p}{\PYGZob{}}\PYG{o}{\PYGZpc{}} \PYG{k}{for} \PYG{n}{entry} \PYG{o+ow}{in} \PYG{n}{entries} \PYG{o}{\PYGZpc{}}\PYG{p}{\PYGZcb{}}
  \PYG{p}{\PYGZob{}}\PYG{o}{\PYGZpc{}} \PYG{n}{module} \PYG{n}{Entry}\PYG{p}{(}\PYG{n}{entry}\PYG{p}{)} \PYG{o}{\PYGZpc{}}\PYG{p}{\PYGZcb{}}
\PYG{p}{\PYGZob{}}\PYG{o}{\PYGZpc{}} \PYG{n}{end} \PYG{o}{\PYGZpc{}}\PYG{p}{\PYGZcb{}}
\end{sphinxVerbatim}

and \sphinxcode{\sphinxupquote{entry.html}}:

\begin{sphinxVerbatim}[commandchars=\\\{\}]
\PYG{p}{\PYGZob{}}\PYG{o}{\PYGZpc{}} \PYG{n}{module} \PYG{n}{Entry}\PYG{p}{(}\PYG{n}{entry}\PYG{p}{,} \PYG{n}{show\PYGZus{}comments}\PYG{o}{=}\PYG{k+kc}{True}\PYG{p}{)} \PYG{o}{\PYGZpc{}}\PYG{p}{\PYGZcb{}}
\end{sphinxVerbatim}

Modules can include custom CSS and JavaScript functions by overriding
the \sphinxcode{\sphinxupquote{embedded\_css}}, \sphinxcode{\sphinxupquote{embedded\_javascript}}, \sphinxcode{\sphinxupquote{javascript\_files}}, or
\sphinxcode{\sphinxupquote{css\_files}} methods:

\begin{sphinxVerbatim}[commandchars=\\\{\}]
\PYG{k}{class} \PYG{n+nc}{Entry}\PYG{p}{(}\PYG{n}{tornado}\PYG{o}{.}\PYG{n}{web}\PYG{o}{.}\PYG{n}{UIModule}\PYG{p}{)}\PYG{p}{:}
    \PYG{k}{def} \PYG{n+nf}{embedded\PYGZus{}css}\PYG{p}{(}\PYG{n+nb+bp}{self}\PYG{p}{)}\PYG{p}{:}
        \PYG{k}{return} \PYG{l+s+s2}{\PYGZdq{}}\PYG{l+s+s2}{.entry }\PYG{l+s+s2}{\PYGZob{}}\PYG{l+s+s2}{ margin\PYGZhy{}bottom: 1em; \PYGZcb{}}\PYG{l+s+s2}{\PYGZdq{}}

    \PYG{k}{def} \PYG{n+nf}{render}\PYG{p}{(}\PYG{n+nb+bp}{self}\PYG{p}{,} \PYG{n}{entry}\PYG{p}{,} \PYG{n}{show\PYGZus{}comments}\PYG{o}{=}\PYG{k+kc}{False}\PYG{p}{)}\PYG{p}{:}
        \PYG{k}{return} \PYG{n+nb+bp}{self}\PYG{o}{.}\PYG{n}{render\PYGZus{}string}\PYG{p}{(}
            \PYG{l+s+s2}{\PYGZdq{}}\PYG{l+s+s2}{module\PYGZhy{}entry.html}\PYG{l+s+s2}{\PYGZdq{}}\PYG{p}{,} \PYG{n}{show\PYGZus{}comments}\PYG{o}{=}\PYG{n}{show\PYGZus{}comments}\PYG{p}{)}
\end{sphinxVerbatim}

Module CSS and JavaScript will be included once no matter how many times
a module is used on a page. CSS is always included in the \sphinxcode{\sphinxupquote{\textless{}head\textgreater{}}} of
the page, and JavaScript is always included just before the \sphinxcode{\sphinxupquote{\textless{}/body\textgreater{}}}
tag at the end of the page.

When additional Python code is not required, a template file itself may
be used as a module. For example, the preceding example could be
rewritten to put the following in \sphinxcode{\sphinxupquote{module-entry.html}}:

\begin{sphinxVerbatim}[commandchars=\\\{\}]
\PYGZob{}\PYGZob{} set\PYGZus{}resources(embedded\PYGZus{}css=\PYGZdq{}.entry \PYGZob{} margin\PYGZhy{}bottom: 1em; \PYGZcb{}\PYGZdq{}) \PYGZcb{}\PYGZcb{}
\PYGZlt{}!\PYGZhy{}\PYGZhy{} more template html... \PYGZhy{}\PYGZhy{}\PYGZgt{}
\end{sphinxVerbatim}

This revised template module would be invoked with:

\begin{sphinxVerbatim}[commandchars=\\\{\}]
\PYG{p}{\PYGZob{}}\PYG{o}{\PYGZpc{}} \PYG{n}{module} \PYG{n}{Template}\PYG{p}{(}\PYG{l+s+s2}{\PYGZdq{}}\PYG{l+s+s2}{module\PYGZhy{}entry.html}\PYG{l+s+s2}{\PYGZdq{}}\PYG{p}{,} \PYG{n}{show\PYGZus{}comments}\PYG{o}{=}\PYG{k+kc}{True}\PYG{p}{)} \PYG{o}{\PYGZpc{}}\PYG{p}{\PYGZcb{}}
\end{sphinxVerbatim}

The \sphinxcode{\sphinxupquote{set\_resources}} function is only available in templates invoked
via \sphinxcode{\sphinxupquote{\{\% module Template(...) \%\}}}. Unlike the \sphinxcode{\sphinxupquote{\{\% include ... \%\}}}
directive, template modules have a distinct namespace from their
containing template - they can only see the global template namespace
and their own keyword arguments.


\subsection{Authentication and security}
\label{\detokenize{guide/security:authentication-and-security}}\label{\detokenize{guide/security::doc}}

\subsubsection{Cookies and secure cookies}
\label{\detokenize{guide/security:cookies-and-secure-cookies}}
You can set cookies in the user’s browser with the \sphinxcode{\sphinxupquote{set\_cookie}}
method:

\begin{sphinxVerbatim}[commandchars=\\\{\}]
\PYG{k}{class} \PYG{n+nc}{MainHandler}\PYG{p}{(}\PYG{n}{tornado}\PYG{o}{.}\PYG{n}{web}\PYG{o}{.}\PYG{n}{RequestHandler}\PYG{p}{)}\PYG{p}{:}
    \PYG{k}{def} \PYG{n+nf}{get}\PYG{p}{(}\PYG{n+nb+bp}{self}\PYG{p}{)}\PYG{p}{:}
        \PYG{k}{if} \PYG{o+ow}{not} \PYG{n+nb+bp}{self}\PYG{o}{.}\PYG{n}{get\PYGZus{}cookie}\PYG{p}{(}\PYG{l+s+s2}{\PYGZdq{}}\PYG{l+s+s2}{mycookie}\PYG{l+s+s2}{\PYGZdq{}}\PYG{p}{)}\PYG{p}{:}
            \PYG{n+nb+bp}{self}\PYG{o}{.}\PYG{n}{set\PYGZus{}cookie}\PYG{p}{(}\PYG{l+s+s2}{\PYGZdq{}}\PYG{l+s+s2}{mycookie}\PYG{l+s+s2}{\PYGZdq{}}\PYG{p}{,} \PYG{l+s+s2}{\PYGZdq{}}\PYG{l+s+s2}{myvalue}\PYG{l+s+s2}{\PYGZdq{}}\PYG{p}{)}
            \PYG{n+nb+bp}{self}\PYG{o}{.}\PYG{n}{write}\PYG{p}{(}\PYG{l+s+s2}{\PYGZdq{}}\PYG{l+s+s2}{Your cookie was not set yet!}\PYG{l+s+s2}{\PYGZdq{}}\PYG{p}{)}
        \PYG{k}{else}\PYG{p}{:}
            \PYG{n+nb+bp}{self}\PYG{o}{.}\PYG{n}{write}\PYG{p}{(}\PYG{l+s+s2}{\PYGZdq{}}\PYG{l+s+s2}{Your cookie was set!}\PYG{l+s+s2}{\PYGZdq{}}\PYG{p}{)}
\end{sphinxVerbatim}

Cookies are not secure and can easily be modified by clients.  If you
need to set cookies to, e.g., identify the currently logged in user,
you need to sign your cookies to prevent forgery. Tornado supports
signed cookies with the {\hyperref[\detokenize{web:tornado.web.RequestHandler.set_secure_cookie}]{\sphinxcrossref{\sphinxcode{\sphinxupquote{set\_secure\_cookie}}}}} and
{\hyperref[\detokenize{web:tornado.web.RequestHandler.get_secure_cookie}]{\sphinxcrossref{\sphinxcode{\sphinxupquote{get\_secure\_cookie}}}}} methods. To use these methods,
you need to specify a secret key named \sphinxcode{\sphinxupquote{cookie\_secret}} when you
create your application. You can pass in application settings as
keyword arguments to your application:

\begin{sphinxVerbatim}[commandchars=\\\{\}]
\PYG{n}{application} \PYG{o}{=} \PYG{n}{tornado}\PYG{o}{.}\PYG{n}{web}\PYG{o}{.}\PYG{n}{Application}\PYG{p}{(}\PYG{p}{[}
    \PYG{p}{(}\PYG{l+s+sa}{r}\PYG{l+s+s2}{\PYGZdq{}}\PYG{l+s+s2}{/}\PYG{l+s+s2}{\PYGZdq{}}\PYG{p}{,} \PYG{n}{MainHandler}\PYG{p}{)}\PYG{p}{,}
\PYG{p}{]}\PYG{p}{,} \PYG{n}{cookie\PYGZus{}secret}\PYG{o}{=}\PYG{l+s+s2}{\PYGZdq{}}\PYG{l+s+s2}{\PYGZus{}\PYGZus{}TODO:\PYGZus{}GENERATE\PYGZus{}YOUR\PYGZus{}OWN\PYGZus{}RANDOM\PYGZus{}VALUE\PYGZus{}HERE\PYGZus{}\PYGZus{}}\PYG{l+s+s2}{\PYGZdq{}}\PYG{p}{)}
\end{sphinxVerbatim}

Signed cookies contain the encoded value of the cookie in addition to a
timestamp and an \sphinxhref{http://en.wikipedia.org/wiki/HMAC}{HMAC} signature.
If the cookie is old or if the signature doesn’t match,
\sphinxcode{\sphinxupquote{get\_secure\_cookie}} will return \sphinxcode{\sphinxupquote{None}} just as if the cookie isn’t
set. The secure version of the example above:

\begin{sphinxVerbatim}[commandchars=\\\{\}]
\PYG{k}{class} \PYG{n+nc}{MainHandler}\PYG{p}{(}\PYG{n}{tornado}\PYG{o}{.}\PYG{n}{web}\PYG{o}{.}\PYG{n}{RequestHandler}\PYG{p}{)}\PYG{p}{:}
    \PYG{k}{def} \PYG{n+nf}{get}\PYG{p}{(}\PYG{n+nb+bp}{self}\PYG{p}{)}\PYG{p}{:}
        \PYG{k}{if} \PYG{o+ow}{not} \PYG{n+nb+bp}{self}\PYG{o}{.}\PYG{n}{get\PYGZus{}secure\PYGZus{}cookie}\PYG{p}{(}\PYG{l+s+s2}{\PYGZdq{}}\PYG{l+s+s2}{mycookie}\PYG{l+s+s2}{\PYGZdq{}}\PYG{p}{)}\PYG{p}{:}
            \PYG{n+nb+bp}{self}\PYG{o}{.}\PYG{n}{set\PYGZus{}secure\PYGZus{}cookie}\PYG{p}{(}\PYG{l+s+s2}{\PYGZdq{}}\PYG{l+s+s2}{mycookie}\PYG{l+s+s2}{\PYGZdq{}}\PYG{p}{,} \PYG{l+s+s2}{\PYGZdq{}}\PYG{l+s+s2}{myvalue}\PYG{l+s+s2}{\PYGZdq{}}\PYG{p}{)}
            \PYG{n+nb+bp}{self}\PYG{o}{.}\PYG{n}{write}\PYG{p}{(}\PYG{l+s+s2}{\PYGZdq{}}\PYG{l+s+s2}{Your cookie was not set yet!}\PYG{l+s+s2}{\PYGZdq{}}\PYG{p}{)}
        \PYG{k}{else}\PYG{p}{:}
            \PYG{n+nb+bp}{self}\PYG{o}{.}\PYG{n}{write}\PYG{p}{(}\PYG{l+s+s2}{\PYGZdq{}}\PYG{l+s+s2}{Your cookie was set!}\PYG{l+s+s2}{\PYGZdq{}}\PYG{p}{)}
\end{sphinxVerbatim}

Tornado’s secure cookies guarantee integrity but not confidentiality.
That is, the cookie cannot be modified but its contents can be seen by the
user.  The \sphinxcode{\sphinxupquote{cookie\_secret}} is a symmetric key and must be kept secret \textendash{}
anyone who obtains the value of this key could produce their own signed
cookies.

By default, Tornado’s secure cookies expire after 30 days.  To change this,
use the \sphinxcode{\sphinxupquote{expires\_days}} keyword argument to \sphinxcode{\sphinxupquote{set\_secure\_cookie}} \sphinxstyleemphasis{and} the
\sphinxcode{\sphinxupquote{max\_age\_days}} argument to \sphinxcode{\sphinxupquote{get\_secure\_cookie}}.  These two values are
passed separately so that you may e.g. have a cookie that is valid for 30 days
for most purposes, but for certain sensitive actions (such as changing billing
information) you use a smaller \sphinxcode{\sphinxupquote{max\_age\_days}} when reading the cookie.

Tornado also supports multiple signing keys to enable signing key
rotation. \sphinxcode{\sphinxupquote{cookie\_secret}} then must be a dict with integer key versions
as keys and the corresponding secrets as values. The currently used
signing key must then be set as \sphinxcode{\sphinxupquote{key\_version}} application setting
but all other keys in the dict are allowed for cookie signature validation,
if the correct key version is set in the cookie.
To implement cookie updates, the current signing key version can be
queried via {\hyperref[\detokenize{web:tornado.web.RequestHandler.get_secure_cookie_key_version}]{\sphinxcrossref{\sphinxcode{\sphinxupquote{get\_secure\_cookie\_key\_version}}}}}.


\subsubsection{User authentication}
\label{\detokenize{guide/security:user-authentication}}\label{\detokenize{guide/security:id1}}
The currently authenticated user is available in every request handler
as {\hyperref[\detokenize{web:tornado.web.RequestHandler.current_user}]{\sphinxcrossref{\sphinxcode{\sphinxupquote{self.current\_user}}}}}, and in every
template as \sphinxcode{\sphinxupquote{current\_user}}. By default, \sphinxcode{\sphinxupquote{current\_user}} is
\sphinxcode{\sphinxupquote{None}}.

To implement user authentication in your application, you need to
override the \sphinxcode{\sphinxupquote{get\_current\_user()}} method in your request handlers to
determine the current user based on, e.g., the value of a cookie. Here
is an example that lets users log into the application simply by
specifying a nickname, which is then saved in a cookie:

\begin{sphinxVerbatim}[commandchars=\\\{\}]
\PYG{k}{class} \PYG{n+nc}{BaseHandler}\PYG{p}{(}\PYG{n}{tornado}\PYG{o}{.}\PYG{n}{web}\PYG{o}{.}\PYG{n}{RequestHandler}\PYG{p}{)}\PYG{p}{:}
    \PYG{k}{def} \PYG{n+nf}{get\PYGZus{}current\PYGZus{}user}\PYG{p}{(}\PYG{n+nb+bp}{self}\PYG{p}{)}\PYG{p}{:}
        \PYG{k}{return} \PYG{n+nb+bp}{self}\PYG{o}{.}\PYG{n}{get\PYGZus{}secure\PYGZus{}cookie}\PYG{p}{(}\PYG{l+s+s2}{\PYGZdq{}}\PYG{l+s+s2}{user}\PYG{l+s+s2}{\PYGZdq{}}\PYG{p}{)}

\PYG{k}{class} \PYG{n+nc}{MainHandler}\PYG{p}{(}\PYG{n}{BaseHandler}\PYG{p}{)}\PYG{p}{:}
    \PYG{k}{def} \PYG{n+nf}{get}\PYG{p}{(}\PYG{n+nb+bp}{self}\PYG{p}{)}\PYG{p}{:}
        \PYG{k}{if} \PYG{o+ow}{not} \PYG{n+nb+bp}{self}\PYG{o}{.}\PYG{n}{current\PYGZus{}user}\PYG{p}{:}
            \PYG{n+nb+bp}{self}\PYG{o}{.}\PYG{n}{redirect}\PYG{p}{(}\PYG{l+s+s2}{\PYGZdq{}}\PYG{l+s+s2}{/login}\PYG{l+s+s2}{\PYGZdq{}}\PYG{p}{)}
            \PYG{k}{return}
        \PYG{n}{name} \PYG{o}{=} \PYG{n}{tornado}\PYG{o}{.}\PYG{n}{escape}\PYG{o}{.}\PYG{n}{xhtml\PYGZus{}escape}\PYG{p}{(}\PYG{n+nb+bp}{self}\PYG{o}{.}\PYG{n}{current\PYGZus{}user}\PYG{p}{)}
        \PYG{n+nb+bp}{self}\PYG{o}{.}\PYG{n}{write}\PYG{p}{(}\PYG{l+s+s2}{\PYGZdq{}}\PYG{l+s+s2}{Hello, }\PYG{l+s+s2}{\PYGZdq{}} \PYG{o}{+} \PYG{n}{name}\PYG{p}{)}

\PYG{k}{class} \PYG{n+nc}{LoginHandler}\PYG{p}{(}\PYG{n}{BaseHandler}\PYG{p}{)}\PYG{p}{:}
    \PYG{k}{def} \PYG{n+nf}{get}\PYG{p}{(}\PYG{n+nb+bp}{self}\PYG{p}{)}\PYG{p}{:}
        \PYG{n+nb+bp}{self}\PYG{o}{.}\PYG{n}{write}\PYG{p}{(}\PYG{l+s+s1}{\PYGZsq{}}\PYG{l+s+s1}{\PYGZlt{}html\PYGZgt{}\PYGZlt{}body\PYGZgt{}\PYGZlt{}form action=}\PYG{l+s+s1}{\PYGZdq{}}\PYG{l+s+s1}{/login}\PYG{l+s+s1}{\PYGZdq{}}\PYG{l+s+s1}{ method=}\PYG{l+s+s1}{\PYGZdq{}}\PYG{l+s+s1}{post}\PYG{l+s+s1}{\PYGZdq{}}\PYG{l+s+s1}{\PYGZgt{}}\PYG{l+s+s1}{\PYGZsq{}}
                   \PYG{l+s+s1}{\PYGZsq{}}\PYG{l+s+s1}{Name: \PYGZlt{}input type=}\PYG{l+s+s1}{\PYGZdq{}}\PYG{l+s+s1}{text}\PYG{l+s+s1}{\PYGZdq{}}\PYG{l+s+s1}{ name=}\PYG{l+s+s1}{\PYGZdq{}}\PYG{l+s+s1}{name}\PYG{l+s+s1}{\PYGZdq{}}\PYG{l+s+s1}{\PYGZgt{}}\PYG{l+s+s1}{\PYGZsq{}}
                   \PYG{l+s+s1}{\PYGZsq{}}\PYG{l+s+s1}{\PYGZlt{}input type=}\PYG{l+s+s1}{\PYGZdq{}}\PYG{l+s+s1}{submit}\PYG{l+s+s1}{\PYGZdq{}}\PYG{l+s+s1}{ value=}\PYG{l+s+s1}{\PYGZdq{}}\PYG{l+s+s1}{Sign in}\PYG{l+s+s1}{\PYGZdq{}}\PYG{l+s+s1}{\PYGZgt{}}\PYG{l+s+s1}{\PYGZsq{}}
                   \PYG{l+s+s1}{\PYGZsq{}}\PYG{l+s+s1}{\PYGZlt{}/form\PYGZgt{}\PYGZlt{}/body\PYGZgt{}\PYGZlt{}/html\PYGZgt{}}\PYG{l+s+s1}{\PYGZsq{}}\PYG{p}{)}

    \PYG{k}{def} \PYG{n+nf}{post}\PYG{p}{(}\PYG{n+nb+bp}{self}\PYG{p}{)}\PYG{p}{:}
        \PYG{n+nb+bp}{self}\PYG{o}{.}\PYG{n}{set\PYGZus{}secure\PYGZus{}cookie}\PYG{p}{(}\PYG{l+s+s2}{\PYGZdq{}}\PYG{l+s+s2}{user}\PYG{l+s+s2}{\PYGZdq{}}\PYG{p}{,} \PYG{n+nb+bp}{self}\PYG{o}{.}\PYG{n}{get\PYGZus{}argument}\PYG{p}{(}\PYG{l+s+s2}{\PYGZdq{}}\PYG{l+s+s2}{name}\PYG{l+s+s2}{\PYGZdq{}}\PYG{p}{)}\PYG{p}{)}
        \PYG{n+nb+bp}{self}\PYG{o}{.}\PYG{n}{redirect}\PYG{p}{(}\PYG{l+s+s2}{\PYGZdq{}}\PYG{l+s+s2}{/}\PYG{l+s+s2}{\PYGZdq{}}\PYG{p}{)}

\PYG{n}{application} \PYG{o}{=} \PYG{n}{tornado}\PYG{o}{.}\PYG{n}{web}\PYG{o}{.}\PYG{n}{Application}\PYG{p}{(}\PYG{p}{[}
    \PYG{p}{(}\PYG{l+s+sa}{r}\PYG{l+s+s2}{\PYGZdq{}}\PYG{l+s+s2}{/}\PYG{l+s+s2}{\PYGZdq{}}\PYG{p}{,} \PYG{n}{MainHandler}\PYG{p}{)}\PYG{p}{,}
    \PYG{p}{(}\PYG{l+s+sa}{r}\PYG{l+s+s2}{\PYGZdq{}}\PYG{l+s+s2}{/login}\PYG{l+s+s2}{\PYGZdq{}}\PYG{p}{,} \PYG{n}{LoginHandler}\PYG{p}{)}\PYG{p}{,}
\PYG{p}{]}\PYG{p}{,} \PYG{n}{cookie\PYGZus{}secret}\PYG{o}{=}\PYG{l+s+s2}{\PYGZdq{}}\PYG{l+s+s2}{\PYGZus{}\PYGZus{}TODO:\PYGZus{}GENERATE\PYGZus{}YOUR\PYGZus{}OWN\PYGZus{}RANDOM\PYGZus{}VALUE\PYGZus{}HERE\PYGZus{}\PYGZus{}}\PYG{l+s+s2}{\PYGZdq{}}\PYG{p}{)}
\end{sphinxVerbatim}

You can require that the user be logged in using the \sphinxhref{http://www.python.org/dev/peps/pep-0318/}{Python
decorator}
{\hyperref[\detokenize{web:tornado.web.authenticated}]{\sphinxcrossref{\sphinxcode{\sphinxupquote{tornado.web.authenticated}}}}}. If a request goes to a method with this
decorator, and the user is not logged in, they will be redirected to
\sphinxcode{\sphinxupquote{login\_url}} (another application setting). The example above could be
rewritten:

\begin{sphinxVerbatim}[commandchars=\\\{\}]
\PYG{k}{class} \PYG{n+nc}{MainHandler}\PYG{p}{(}\PYG{n}{BaseHandler}\PYG{p}{)}\PYG{p}{:}
    \PYG{n+nd}{@tornado}\PYG{o}{.}\PYG{n}{web}\PYG{o}{.}\PYG{n}{authenticated}
    \PYG{k}{def} \PYG{n+nf}{get}\PYG{p}{(}\PYG{n+nb+bp}{self}\PYG{p}{)}\PYG{p}{:}
        \PYG{n}{name} \PYG{o}{=} \PYG{n}{tornado}\PYG{o}{.}\PYG{n}{escape}\PYG{o}{.}\PYG{n}{xhtml\PYGZus{}escape}\PYG{p}{(}\PYG{n+nb+bp}{self}\PYG{o}{.}\PYG{n}{current\PYGZus{}user}\PYG{p}{)}
        \PYG{n+nb+bp}{self}\PYG{o}{.}\PYG{n}{write}\PYG{p}{(}\PYG{l+s+s2}{\PYGZdq{}}\PYG{l+s+s2}{Hello, }\PYG{l+s+s2}{\PYGZdq{}} \PYG{o}{+} \PYG{n}{name}\PYG{p}{)}

\PYG{n}{settings} \PYG{o}{=} \PYG{p}{\PYGZob{}}
    \PYG{l+s+s2}{\PYGZdq{}}\PYG{l+s+s2}{cookie\PYGZus{}secret}\PYG{l+s+s2}{\PYGZdq{}}\PYG{p}{:} \PYG{l+s+s2}{\PYGZdq{}}\PYG{l+s+s2}{\PYGZus{}\PYGZus{}TODO:\PYGZus{}GENERATE\PYGZus{}YOUR\PYGZus{}OWN\PYGZus{}RANDOM\PYGZus{}VALUE\PYGZus{}HERE\PYGZus{}\PYGZus{}}\PYG{l+s+s2}{\PYGZdq{}}\PYG{p}{,}
    \PYG{l+s+s2}{\PYGZdq{}}\PYG{l+s+s2}{login\PYGZus{}url}\PYG{l+s+s2}{\PYGZdq{}}\PYG{p}{:} \PYG{l+s+s2}{\PYGZdq{}}\PYG{l+s+s2}{/login}\PYG{l+s+s2}{\PYGZdq{}}\PYG{p}{,}
\PYG{p}{\PYGZcb{}}
\PYG{n}{application} \PYG{o}{=} \PYG{n}{tornado}\PYG{o}{.}\PYG{n}{web}\PYG{o}{.}\PYG{n}{Application}\PYG{p}{(}\PYG{p}{[}
    \PYG{p}{(}\PYG{l+s+sa}{r}\PYG{l+s+s2}{\PYGZdq{}}\PYG{l+s+s2}{/}\PYG{l+s+s2}{\PYGZdq{}}\PYG{p}{,} \PYG{n}{MainHandler}\PYG{p}{)}\PYG{p}{,}
    \PYG{p}{(}\PYG{l+s+sa}{r}\PYG{l+s+s2}{\PYGZdq{}}\PYG{l+s+s2}{/login}\PYG{l+s+s2}{\PYGZdq{}}\PYG{p}{,} \PYG{n}{LoginHandler}\PYG{p}{)}\PYG{p}{,}
\PYG{p}{]}\PYG{p}{,} \PYG{o}{*}\PYG{o}{*}\PYG{n}{settings}\PYG{p}{)}
\end{sphinxVerbatim}

If you decorate \sphinxcode{\sphinxupquote{post()}} methods with the \sphinxcode{\sphinxupquote{authenticated}}
decorator, and the user is not logged in, the server will send a
\sphinxcode{\sphinxupquote{403}} response.  The \sphinxcode{\sphinxupquote{@authenticated}} decorator is simply
shorthand for \sphinxcode{\sphinxupquote{if not self.current\_user: self.redirect()}} and may
not be appropriate for non-browser-based login schemes.

Check out the \sphinxhref{https://github.com/tornadoweb/tornado/tree/stable/demos/blog}{Tornado Blog example application} for a
complete example that uses authentication (and stores user data in a
MySQL database).


\subsubsection{Third party authentication}
\label{\detokenize{guide/security:third-party-authentication}}
The {\hyperref[\detokenize{auth:module-tornado.auth}]{\sphinxcrossref{\sphinxcode{\sphinxupquote{tornado.auth}}}}} module implements the authentication and
authorization protocols for a number of the most popular sites on the
web, including Google/Gmail, Facebook, Twitter, and FriendFeed.
The module includes methods to log users in via these sites and, where
applicable, methods to authorize access to the service so you can, e.g.,
download a user’s address book or publish a Twitter message on their
behalf.

Here is an example handler that uses Google for authentication, saving
the Google credentials in a cookie for later access:

\begin{sphinxVerbatim}[commandchars=\\\{\}]
\PYG{k}{class} \PYG{n+nc}{GoogleOAuth2LoginHandler}\PYG{p}{(}\PYG{n}{tornado}\PYG{o}{.}\PYG{n}{web}\PYG{o}{.}\PYG{n}{RequestHandler}\PYG{p}{,}
                               \PYG{n}{tornado}\PYG{o}{.}\PYG{n}{auth}\PYG{o}{.}\PYG{n}{GoogleOAuth2Mixin}\PYG{p}{)}\PYG{p}{:}
    \PYG{k}{async} \PYG{k}{def} \PYG{n+nf}{get}\PYG{p}{(}\PYG{n+nb+bp}{self}\PYG{p}{)}\PYG{p}{:}
        \PYG{k}{if} \PYG{n+nb+bp}{self}\PYG{o}{.}\PYG{n}{get\PYGZus{}argument}\PYG{p}{(}\PYG{l+s+s1}{\PYGZsq{}}\PYG{l+s+s1}{code}\PYG{l+s+s1}{\PYGZsq{}}\PYG{p}{,} \PYG{k+kc}{False}\PYG{p}{)}\PYG{p}{:}
            \PYG{n}{user} \PYG{o}{=} \PYG{k}{await} \PYG{n+nb+bp}{self}\PYG{o}{.}\PYG{n}{get\PYGZus{}authenticated\PYGZus{}user}\PYG{p}{(}
                \PYG{n}{redirect\PYGZus{}uri}\PYG{o}{=}\PYG{l+s+s1}{\PYGZsq{}}\PYG{l+s+s1}{http://your.site.com/auth/google}\PYG{l+s+s1}{\PYGZsq{}}\PYG{p}{,}
                \PYG{n}{code}\PYG{o}{=}\PYG{n+nb+bp}{self}\PYG{o}{.}\PYG{n}{get\PYGZus{}argument}\PYG{p}{(}\PYG{l+s+s1}{\PYGZsq{}}\PYG{l+s+s1}{code}\PYG{l+s+s1}{\PYGZsq{}}\PYG{p}{)}\PYG{p}{)}
            \PYG{c+c1}{\PYGZsh{} Save the user with e.g. set\PYGZus{}secure\PYGZus{}cookie}
        \PYG{k}{else}\PYG{p}{:}
            \PYG{k}{await} \PYG{n+nb+bp}{self}\PYG{o}{.}\PYG{n}{authorize\PYGZus{}redirect}\PYG{p}{(}
                \PYG{n}{redirect\PYGZus{}uri}\PYG{o}{=}\PYG{l+s+s1}{\PYGZsq{}}\PYG{l+s+s1}{http://your.site.com/auth/google}\PYG{l+s+s1}{\PYGZsq{}}\PYG{p}{,}
                \PYG{n}{client\PYGZus{}id}\PYG{o}{=}\PYG{n+nb+bp}{self}\PYG{o}{.}\PYG{n}{settings}\PYG{p}{[}\PYG{l+s+s1}{\PYGZsq{}}\PYG{l+s+s1}{google\PYGZus{}oauth}\PYG{l+s+s1}{\PYGZsq{}}\PYG{p}{]}\PYG{p}{[}\PYG{l+s+s1}{\PYGZsq{}}\PYG{l+s+s1}{key}\PYG{l+s+s1}{\PYGZsq{}}\PYG{p}{]}\PYG{p}{,}
                \PYG{n}{scope}\PYG{o}{=}\PYG{p}{[}\PYG{l+s+s1}{\PYGZsq{}}\PYG{l+s+s1}{profile}\PYG{l+s+s1}{\PYGZsq{}}\PYG{p}{,} \PYG{l+s+s1}{\PYGZsq{}}\PYG{l+s+s1}{email}\PYG{l+s+s1}{\PYGZsq{}}\PYG{p}{]}\PYG{p}{,}
                \PYG{n}{response\PYGZus{}type}\PYG{o}{=}\PYG{l+s+s1}{\PYGZsq{}}\PYG{l+s+s1}{code}\PYG{l+s+s1}{\PYGZsq{}}\PYG{p}{,}
                \PYG{n}{extra\PYGZus{}params}\PYG{o}{=}\PYG{p}{\PYGZob{}}\PYG{l+s+s1}{\PYGZsq{}}\PYG{l+s+s1}{approval\PYGZus{}prompt}\PYG{l+s+s1}{\PYGZsq{}}\PYG{p}{:} \PYG{l+s+s1}{\PYGZsq{}}\PYG{l+s+s1}{auto}\PYG{l+s+s1}{\PYGZsq{}}\PYG{p}{\PYGZcb{}}\PYG{p}{)}
\end{sphinxVerbatim}

See the {\hyperref[\detokenize{auth:module-tornado.auth}]{\sphinxcrossref{\sphinxcode{\sphinxupquote{tornado.auth}}}}} module documentation for more details.


\subsubsection{Cross-site request forgery protection}
\label{\detokenize{guide/security:cross-site-request-forgery-protection}}\label{\detokenize{guide/security:xsrf}}
\sphinxhref{http://en.wikipedia.org/wiki/Cross-site\_request\_forgery}{Cross-site request
forgery}, or
XSRF, is a common problem for personalized web applications. See the
\sphinxhref{http://en.wikipedia.org/wiki/Cross-site\_request\_forgery}{Wikipedia
article} for
more information on how XSRF works.

The generally accepted solution to prevent XSRF is to cookie every user
with an unpredictable value and include that value as an additional
argument with every form submission on your site. If the cookie and the
value in the form submission do not match, then the request is likely
forged.

Tornado comes with built-in XSRF protection. To include it in your site,
include the application setting \sphinxcode{\sphinxupquote{xsrf\_cookies}}:

\begin{sphinxVerbatim}[commandchars=\\\{\}]
\PYG{n}{settings} \PYG{o}{=} \PYG{p}{\PYGZob{}}
    \PYG{l+s+s2}{\PYGZdq{}}\PYG{l+s+s2}{cookie\PYGZus{}secret}\PYG{l+s+s2}{\PYGZdq{}}\PYG{p}{:} \PYG{l+s+s2}{\PYGZdq{}}\PYG{l+s+s2}{\PYGZus{}\PYGZus{}TODO:\PYGZus{}GENERATE\PYGZus{}YOUR\PYGZus{}OWN\PYGZus{}RANDOM\PYGZus{}VALUE\PYGZus{}HERE\PYGZus{}\PYGZus{}}\PYG{l+s+s2}{\PYGZdq{}}\PYG{p}{,}
    \PYG{l+s+s2}{\PYGZdq{}}\PYG{l+s+s2}{login\PYGZus{}url}\PYG{l+s+s2}{\PYGZdq{}}\PYG{p}{:} \PYG{l+s+s2}{\PYGZdq{}}\PYG{l+s+s2}{/login}\PYG{l+s+s2}{\PYGZdq{}}\PYG{p}{,}
    \PYG{l+s+s2}{\PYGZdq{}}\PYG{l+s+s2}{xsrf\PYGZus{}cookies}\PYG{l+s+s2}{\PYGZdq{}}\PYG{p}{:} \PYG{k+kc}{True}\PYG{p}{,}
\PYG{p}{\PYGZcb{}}
\PYG{n}{application} \PYG{o}{=} \PYG{n}{tornado}\PYG{o}{.}\PYG{n}{web}\PYG{o}{.}\PYG{n}{Application}\PYG{p}{(}\PYG{p}{[}
    \PYG{p}{(}\PYG{l+s+sa}{r}\PYG{l+s+s2}{\PYGZdq{}}\PYG{l+s+s2}{/}\PYG{l+s+s2}{\PYGZdq{}}\PYG{p}{,} \PYG{n}{MainHandler}\PYG{p}{)}\PYG{p}{,}
    \PYG{p}{(}\PYG{l+s+sa}{r}\PYG{l+s+s2}{\PYGZdq{}}\PYG{l+s+s2}{/login}\PYG{l+s+s2}{\PYGZdq{}}\PYG{p}{,} \PYG{n}{LoginHandler}\PYG{p}{)}\PYG{p}{,}
\PYG{p}{]}\PYG{p}{,} \PYG{o}{*}\PYG{o}{*}\PYG{n}{settings}\PYG{p}{)}
\end{sphinxVerbatim}

If \sphinxcode{\sphinxupquote{xsrf\_cookies}} is set, the Tornado web application will set the
\sphinxcode{\sphinxupquote{\_xsrf}} cookie for all users and reject all \sphinxcode{\sphinxupquote{POST}}, \sphinxcode{\sphinxupquote{PUT}}, and
\sphinxcode{\sphinxupquote{DELETE}} requests that do not contain a correct \sphinxcode{\sphinxupquote{\_xsrf}} value. If
you turn this setting on, you need to instrument all forms that submit
via \sphinxcode{\sphinxupquote{POST}} to contain this field. You can do this with the special
{\hyperref[\detokenize{web:tornado.web.UIModule}]{\sphinxcrossref{\sphinxcode{\sphinxupquote{UIModule}}}}} \sphinxcode{\sphinxupquote{xsrf\_form\_html()}}, available in all templates:

\begin{sphinxVerbatim}[commandchars=\\\{\}]
\PYG{o}{\PYGZlt{}}\PYG{n}{form} \PYG{n}{action}\PYG{o}{=}\PYG{l+s+s2}{\PYGZdq{}}\PYG{l+s+s2}{/new\PYGZus{}message}\PYG{l+s+s2}{\PYGZdq{}} \PYG{n}{method}\PYG{o}{=}\PYG{l+s+s2}{\PYGZdq{}}\PYG{l+s+s2}{post}\PYG{l+s+s2}{\PYGZdq{}}\PYG{o}{\PYGZgt{}}
  \PYG{p}{\PYGZob{}}\PYG{o}{\PYGZpc{}} \PYG{n}{module} \PYG{n}{xsrf\PYGZus{}form\PYGZus{}html}\PYG{p}{(}\PYG{p}{)} \PYG{o}{\PYGZpc{}}\PYG{p}{\PYGZcb{}}
  \PYG{o}{\PYGZlt{}}\PYG{n+nb}{input} \PYG{n+nb}{type}\PYG{o}{=}\PYG{l+s+s2}{\PYGZdq{}}\PYG{l+s+s2}{text}\PYG{l+s+s2}{\PYGZdq{}} \PYG{n}{name}\PYG{o}{=}\PYG{l+s+s2}{\PYGZdq{}}\PYG{l+s+s2}{message}\PYG{l+s+s2}{\PYGZdq{}}\PYG{o}{/}\PYG{o}{\PYGZgt{}}
  \PYG{o}{\PYGZlt{}}\PYG{n+nb}{input} \PYG{n+nb}{type}\PYG{o}{=}\PYG{l+s+s2}{\PYGZdq{}}\PYG{l+s+s2}{submit}\PYG{l+s+s2}{\PYGZdq{}} \PYG{n}{value}\PYG{o}{=}\PYG{l+s+s2}{\PYGZdq{}}\PYG{l+s+s2}{Post}\PYG{l+s+s2}{\PYGZdq{}}\PYG{o}{/}\PYG{o}{\PYGZgt{}}
\PYG{o}{\PYGZlt{}}\PYG{o}{/}\PYG{n}{form}\PYG{o}{\PYGZgt{}}
\end{sphinxVerbatim}

If you submit AJAX \sphinxcode{\sphinxupquote{POST}} requests, you will also need to instrument
your JavaScript to include the \sphinxcode{\sphinxupquote{\_xsrf}} value with each request. This
is the \sphinxhref{http://jquery.com/}{jQuery} function we use at FriendFeed for
AJAX \sphinxcode{\sphinxupquote{POST}} requests that automatically adds the \sphinxcode{\sphinxupquote{\_xsrf}} value to
all requests:

\begin{sphinxVerbatim}[commandchars=\\\{\}]
function getCookie(name) \PYGZob{}
    var r = document.cookie.match(\PYGZdq{}\PYGZbs{}\PYGZbs{}b\PYGZdq{} + name + \PYGZdq{}=([\PYGZca{};]*)\PYGZbs{}\PYGZbs{}b\PYGZdq{});
    return r ? r[1] : undefined;
\PYGZcb{}

jQuery.postJSON = function(url, args, callback) \PYGZob{}
    args.\PYGZus{}xsrf = getCookie(\PYGZdq{}\PYGZus{}xsrf\PYGZdq{});
    \PYGZdl{}.ajax(\PYGZob{}url: url, data: \PYGZdl{}.param(args), dataType: \PYGZdq{}text\PYGZdq{}, type: \PYGZdq{}POST\PYGZdq{},
        success: function(response) \PYGZob{}
        callback(eval(\PYGZdq{}(\PYGZdq{} + response + \PYGZdq{})\PYGZdq{}));
    \PYGZcb{}\PYGZcb{});
\PYGZcb{};
\end{sphinxVerbatim}

For \sphinxcode{\sphinxupquote{PUT}} and \sphinxcode{\sphinxupquote{DELETE}} requests (as well as \sphinxcode{\sphinxupquote{POST}} requests that
do not use form-encoded arguments), the XSRF token may also be passed
via an HTTP header named \sphinxcode{\sphinxupquote{X-XSRFToken}}.  The XSRF cookie is normally
set when \sphinxcode{\sphinxupquote{xsrf\_form\_html}} is used, but in a pure-Javascript application
that does not use any regular forms you may need to access
\sphinxcode{\sphinxupquote{self.xsrf\_token}} manually (just reading the property is enough to
set the cookie as a side effect).

If you need to customize XSRF behavior on a per-handler basis, you can
override {\hyperref[\detokenize{web:tornado.web.RequestHandler.check_xsrf_cookie}]{\sphinxcrossref{\sphinxcode{\sphinxupquote{RequestHandler.check\_xsrf\_cookie()}}}}}. For example, if you
have an API whose authentication does not use cookies, you may want to
disable XSRF protection by making \sphinxcode{\sphinxupquote{check\_xsrf\_cookie()}} do nothing.
However, if you support both cookie and non-cookie-based authentication,
it is important that XSRF protection be used whenever the current
request is authenticated with a cookie.


\subsubsection{DNS Rebinding}
\label{\detokenize{guide/security:dns-rebinding}}\label{\detokenize{guide/security:dnsrebinding}}
\sphinxhref{https://en.wikipedia.org/wiki/DNS\_rebinding}{DNS rebinding} is an
attack that can bypass the same-origin policy and allow external sites
to access resources on private networks. This attack involves a DNS
name (with a short TTL) that alternates between returning an IP
address controlled by the attacker and one controlled by the victim
(often a guessable private IP address such as \sphinxcode{\sphinxupquote{127.0.0.1}} or
\sphinxcode{\sphinxupquote{192.168.1.1}}).

Applications that use TLS are \sphinxstyleemphasis{not} vulnerable to this attack (because
the browser will display certificate mismatch warnings that block
automated access to the target site).

Applications that cannot use TLS and rely on network-level access
controls (for example, assuming that a server on \sphinxcode{\sphinxupquote{127.0.0.1}} can
only be accessed by the local machine) should guard against DNS
rebinding by validating the \sphinxcode{\sphinxupquote{Host}} HTTP header. This means passing a
restrictive hostname pattern to either a {\hyperref[\detokenize{routing:tornado.routing.HostMatches}]{\sphinxcrossref{\sphinxcode{\sphinxupquote{HostMatches}}}}} router or the
first argument of {\hyperref[\detokenize{web:tornado.web.Application.add_handlers}]{\sphinxcrossref{\sphinxcode{\sphinxupquote{Application.add\_handlers}}}}}:

\begin{sphinxVerbatim}[commandchars=\\\{\}]
\PYG{c+c1}{\PYGZsh{} BAD: uses a default host pattern of r\PYGZsq{}.*\PYGZsq{}}
\PYG{n}{app} \PYG{o}{=} \PYG{n}{Application}\PYG{p}{(}\PYG{p}{[}\PYG{p}{(}\PYG{l+s+s1}{\PYGZsq{}}\PYG{l+s+s1}{/foo}\PYG{l+s+s1}{\PYGZsq{}}\PYG{p}{,} \PYG{n}{FooHandler}\PYG{p}{)}\PYG{p}{]}\PYG{p}{)}

\PYG{c+c1}{\PYGZsh{} GOOD: only matches localhost or its ip address.}
\PYG{n}{app} \PYG{o}{=} \PYG{n}{Application}\PYG{p}{(}\PYG{p}{)}
\PYG{n}{app}\PYG{o}{.}\PYG{n}{add\PYGZus{}handlers}\PYG{p}{(}\PYG{l+s+sa}{r}\PYG{l+s+s1}{\PYGZsq{}}\PYG{l+s+s1}{(localhost\textbar{}127}\PYG{l+s+s1}{\PYGZbs{}}\PYG{l+s+s1}{.0}\PYG{l+s+s1}{\PYGZbs{}}\PYG{l+s+s1}{.0}\PYG{l+s+s1}{\PYGZbs{}}\PYG{l+s+s1}{.1)}\PYG{l+s+s1}{\PYGZsq{}}\PYG{p}{,}
                 \PYG{p}{[}\PYG{p}{(}\PYG{l+s+s1}{\PYGZsq{}}\PYG{l+s+s1}{/foo}\PYG{l+s+s1}{\PYGZsq{}}\PYG{p}{,} \PYG{n}{FooHandler}\PYG{p}{)}\PYG{p}{]}\PYG{p}{)}

\PYG{c+c1}{\PYGZsh{} GOOD: same as previous example using tornado.routing.}
\PYG{n}{app} \PYG{o}{=} \PYG{n}{Application}\PYG{p}{(}\PYG{p}{[}
    \PYG{p}{(}\PYG{n}{HostMatches}\PYG{p}{(}\PYG{l+s+sa}{r}\PYG{l+s+s1}{\PYGZsq{}}\PYG{l+s+s1}{(localhost\textbar{}127}\PYG{l+s+s1}{\PYGZbs{}}\PYG{l+s+s1}{.0}\PYG{l+s+s1}{\PYGZbs{}}\PYG{l+s+s1}{.0}\PYG{l+s+s1}{\PYGZbs{}}\PYG{l+s+s1}{.1)}\PYG{l+s+s1}{\PYGZsq{}}\PYG{p}{)}\PYG{p}{,}
        \PYG{p}{[}\PYG{p}{(}\PYG{l+s+s1}{\PYGZsq{}}\PYG{l+s+s1}{/foo}\PYG{l+s+s1}{\PYGZsq{}}\PYG{p}{,} \PYG{n}{FooHandler}\PYG{p}{)}\PYG{p}{]}\PYG{p}{)}\PYG{p}{,}
    \PYG{p}{]}\PYG{p}{)}
\end{sphinxVerbatim}

In addition, the \sphinxcode{\sphinxupquote{default\_host}} argument to {\hyperref[\detokenize{web:tornado.web.Application}]{\sphinxcrossref{\sphinxcode{\sphinxupquote{Application}}}}} and the
{\hyperref[\detokenize{routing:tornado.routing.DefaultHostMatches}]{\sphinxcrossref{\sphinxcode{\sphinxupquote{DefaultHostMatches}}}}} router must not be used in applications that may
be vulnerable to DNS rebinding, because it has a similar effect to a
wildcard host pattern.


\subsection{Running and deploying}
\label{\detokenize{guide/running:running-and-deploying}}\label{\detokenize{guide/running::doc}}
Since Tornado supplies its own HTTPServer, running and deploying it is
a little different from other Python web frameworks.  Instead of
configuring a WSGI container to find your application, you write a
\sphinxcode{\sphinxupquote{main()}} function that starts the server:

\begin{sphinxVerbatim}[commandchars=\\\{\}]
\PYG{k}{def} \PYG{n+nf}{main}\PYG{p}{(}\PYG{p}{)}\PYG{p}{:}
    \PYG{n}{app} \PYG{o}{=} \PYG{n}{make\PYGZus{}app}\PYG{p}{(}\PYG{p}{)}
    \PYG{n}{app}\PYG{o}{.}\PYG{n}{listen}\PYG{p}{(}\PYG{l+m+mi}{8888}\PYG{p}{)}
    \PYG{n}{IOLoop}\PYG{o}{.}\PYG{n}{current}\PYG{p}{(}\PYG{p}{)}\PYG{o}{.}\PYG{n}{start}\PYG{p}{(}\PYG{p}{)}

\PYG{k}{if} \PYG{n+nv+vm}{\PYGZus{}\PYGZus{}name\PYGZus{}\PYGZus{}} \PYG{o}{==} \PYG{l+s+s1}{\PYGZsq{}}\PYG{l+s+s1}{\PYGZus{}\PYGZus{}main\PYGZus{}\PYGZus{}}\PYG{l+s+s1}{\PYGZsq{}}\PYG{p}{:}
    \PYG{n}{main}\PYG{p}{(}\PYG{p}{)}
\end{sphinxVerbatim}

Configure your operating system or process manager to run this program to
start the server. Please note that it may be necessary to increase the number
of open files per process (to avoid “Too many open files”-Error).
To raise this limit (setting it to 50000 for example)  you can use the
\sphinxcode{\sphinxupquote{ulimit}} command, modify \sphinxcode{\sphinxupquote{/etc/security/limits.conf}} or set
\sphinxcode{\sphinxupquote{minfds}} in your \sphinxhref{http://www.supervisord.org}{supervisord} config.


\subsubsection{Processes and ports}
\label{\detokenize{guide/running:processes-and-ports}}
Due to the Python GIL (Global Interpreter Lock), it is necessary to run
multiple Python processes to take full advantage of multi-CPU machines.
Typically it is best to run one process per CPU.

Tornado includes a built-in multi-process mode to start several
processes at once.  This requires a slight alteration to the standard
main function:

\begin{sphinxVerbatim}[commandchars=\\\{\}]
\PYG{k}{def} \PYG{n+nf}{main}\PYG{p}{(}\PYG{p}{)}\PYG{p}{:}
    \PYG{n}{app} \PYG{o}{=} \PYG{n}{make\PYGZus{}app}\PYG{p}{(}\PYG{p}{)}
    \PYG{n}{server} \PYG{o}{=} \PYG{n}{tornado}\PYG{o}{.}\PYG{n}{httpserver}\PYG{o}{.}\PYG{n}{HTTPServer}\PYG{p}{(}\PYG{n}{app}\PYG{p}{)}
    \PYG{n}{server}\PYG{o}{.}\PYG{n}{bind}\PYG{p}{(}\PYG{l+m+mi}{8888}\PYG{p}{)}
    \PYG{n}{server}\PYG{o}{.}\PYG{n}{start}\PYG{p}{(}\PYG{l+m+mi}{0}\PYG{p}{)}  \PYG{c+c1}{\PYGZsh{} forks one process per cpu}
    \PYG{n}{IOLoop}\PYG{o}{.}\PYG{n}{current}\PYG{p}{(}\PYG{p}{)}\PYG{o}{.}\PYG{n}{start}\PYG{p}{(}\PYG{p}{)}
\end{sphinxVerbatim}

This is the easiest way to start multiple processes and have them all
share the same port, although it has some limitations.  First, each
child process will have its own \sphinxcode{\sphinxupquote{IOLoop}}, so it is important that
nothing touches the global \sphinxcode{\sphinxupquote{IOLoop}} instance (even indirectly) before the
fork.  Second, it is difficult to do zero-downtime updates in this model.
Finally, since all the processes share the same port it is more difficult
to monitor them individually.

For more sophisticated deployments, it is recommended to start the processes
independently, and have each one listen on a different port.
The “process groups” feature of \sphinxhref{http://www.supervisord.org}{supervisord}
is one good way to arrange this.  When each process uses a different port,
an external load balancer such as HAProxy or nginx is usually needed
to present a single address to outside visitors.


\subsubsection{Running behind a load balancer}
\label{\detokenize{guide/running:running-behind-a-load-balancer}}
When running behind a load balancer like \sphinxhref{http://nginx.net/}{nginx},
it is recommended to pass \sphinxcode{\sphinxupquote{xheaders=True}} to the {\hyperref[\detokenize{httpserver:tornado.httpserver.HTTPServer}]{\sphinxcrossref{\sphinxcode{\sphinxupquote{HTTPServer}}}}} constructor.
This will tell Tornado to use headers like \sphinxcode{\sphinxupquote{X-Real-IP}} to get the user’s
IP address instead of attributing all traffic to the balancer’s IP address.

This is a barebones nginx config file that is structurally similar to
the one we use at FriendFeed. It assumes nginx and the Tornado servers
are running on the same machine, and the four Tornado servers are
running on ports 8000 - 8003:

\begin{sphinxVerbatim}[commandchars=\\\{\}]
user nginx;
worker\PYGZus{}processes 1;

error\PYGZus{}log /var/log/nginx/error.log;
pid /var/run/nginx.pid;

events \PYGZob{}
    worker\PYGZus{}connections 1024;
    use epoll;
\PYGZcb{}

http \PYGZob{}
    \PYGZsh{} Enumerate all the Tornado servers here
    upstream frontends \PYGZob{}
        server 127.0.0.1:8000;
        server 127.0.0.1:8001;
        server 127.0.0.1:8002;
        server 127.0.0.1:8003;
    \PYGZcb{}

    include /etc/nginx/mime.types;
    default\PYGZus{}type application/octet\PYGZhy{}stream;

    access\PYGZus{}log /var/log/nginx/access.log;

    keepalive\PYGZus{}timeout 65;
    proxy\PYGZus{}read\PYGZus{}timeout 200;
    sendfile on;
    tcp\PYGZus{}nopush on;
    tcp\PYGZus{}nodelay on;
    gzip on;
    gzip\PYGZus{}min\PYGZus{}length 1000;
    gzip\PYGZus{}proxied any;
    gzip\PYGZus{}types text/plain text/html text/css text/xml
               application/x\PYGZhy{}javascript application/xml
               application/atom+xml text/javascript;

    \PYGZsh{} Only retry if there was a communication error, not a timeout
    \PYGZsh{} on the Tornado server (to avoid propagating \PYGZdq{}queries of death\PYGZdq{}
    \PYGZsh{} to all frontends)
    proxy\PYGZus{}next\PYGZus{}upstream error;

    server \PYGZob{}
        listen 80;

        \PYGZsh{} Allow file uploads
        client\PYGZus{}max\PYGZus{}body\PYGZus{}size 50M;

        location \PYGZca{}\PYGZti{} /static/ \PYGZob{}
            root /var/www;
            if (\PYGZdl{}query\PYGZus{}string) \PYGZob{}
                expires max;
            \PYGZcb{}
        \PYGZcb{}
        location = /favicon.ico \PYGZob{}
            rewrite (.*) /static/favicon.ico;
        \PYGZcb{}
        location = /robots.txt \PYGZob{}
            rewrite (.*) /static/robots.txt;
        \PYGZcb{}

        location / \PYGZob{}
            proxy\PYGZus{}pass\PYGZus{}header Server;
            proxy\PYGZus{}set\PYGZus{}header Host \PYGZdl{}http\PYGZus{}host;
            proxy\PYGZus{}redirect off;
            proxy\PYGZus{}set\PYGZus{}header X\PYGZhy{}Real\PYGZhy{}IP \PYGZdl{}remote\PYGZus{}addr;
            proxy\PYGZus{}set\PYGZus{}header X\PYGZhy{}Scheme \PYGZdl{}scheme;
            proxy\PYGZus{}pass http://frontends;
        \PYGZcb{}
    \PYGZcb{}
\PYGZcb{}
\end{sphinxVerbatim}


\subsubsection{Static files and aggressive file caching}
\label{\detokenize{guide/running:static-files-and-aggressive-file-caching}}
You can serve static files from Tornado by specifying the
\sphinxcode{\sphinxupquote{static\_path}} setting in your application:

\begin{sphinxVerbatim}[commandchars=\\\{\}]
\PYG{n}{settings} \PYG{o}{=} \PYG{p}{\PYGZob{}}
    \PYG{l+s+s2}{\PYGZdq{}}\PYG{l+s+s2}{static\PYGZus{}path}\PYG{l+s+s2}{\PYGZdq{}}\PYG{p}{:} \PYG{n}{os}\PYG{o}{.}\PYG{n}{path}\PYG{o}{.}\PYG{n}{join}\PYG{p}{(}\PYG{n}{os}\PYG{o}{.}\PYG{n}{path}\PYG{o}{.}\PYG{n}{dirname}\PYG{p}{(}\PYG{n+nv+vm}{\PYGZus{}\PYGZus{}file\PYGZus{}\PYGZus{}}\PYG{p}{)}\PYG{p}{,} \PYG{l+s+s2}{\PYGZdq{}}\PYG{l+s+s2}{static}\PYG{l+s+s2}{\PYGZdq{}}\PYG{p}{)}\PYG{p}{,}
    \PYG{l+s+s2}{\PYGZdq{}}\PYG{l+s+s2}{cookie\PYGZus{}secret}\PYG{l+s+s2}{\PYGZdq{}}\PYG{p}{:} \PYG{l+s+s2}{\PYGZdq{}}\PYG{l+s+s2}{\PYGZus{}\PYGZus{}TODO:\PYGZus{}GENERATE\PYGZus{}YOUR\PYGZus{}OWN\PYGZus{}RANDOM\PYGZus{}VALUE\PYGZus{}HERE\PYGZus{}\PYGZus{}}\PYG{l+s+s2}{\PYGZdq{}}\PYG{p}{,}
    \PYG{l+s+s2}{\PYGZdq{}}\PYG{l+s+s2}{login\PYGZus{}url}\PYG{l+s+s2}{\PYGZdq{}}\PYG{p}{:} \PYG{l+s+s2}{\PYGZdq{}}\PYG{l+s+s2}{/login}\PYG{l+s+s2}{\PYGZdq{}}\PYG{p}{,}
    \PYG{l+s+s2}{\PYGZdq{}}\PYG{l+s+s2}{xsrf\PYGZus{}cookies}\PYG{l+s+s2}{\PYGZdq{}}\PYG{p}{:} \PYG{k+kc}{True}\PYG{p}{,}
\PYG{p}{\PYGZcb{}}
\PYG{n}{application} \PYG{o}{=} \PYG{n}{tornado}\PYG{o}{.}\PYG{n}{web}\PYG{o}{.}\PYG{n}{Application}\PYG{p}{(}\PYG{p}{[}
    \PYG{p}{(}\PYG{l+s+sa}{r}\PYG{l+s+s2}{\PYGZdq{}}\PYG{l+s+s2}{/}\PYG{l+s+s2}{\PYGZdq{}}\PYG{p}{,} \PYG{n}{MainHandler}\PYG{p}{)}\PYG{p}{,}
    \PYG{p}{(}\PYG{l+s+sa}{r}\PYG{l+s+s2}{\PYGZdq{}}\PYG{l+s+s2}{/login}\PYG{l+s+s2}{\PYGZdq{}}\PYG{p}{,} \PYG{n}{LoginHandler}\PYG{p}{)}\PYG{p}{,}
    \PYG{p}{(}\PYG{l+s+sa}{r}\PYG{l+s+s2}{\PYGZdq{}}\PYG{l+s+s2}{/(apple\PYGZhy{}touch\PYGZhy{}icon}\PYG{l+s+s2}{\PYGZbs{}}\PYG{l+s+s2}{.png)}\PYG{l+s+s2}{\PYGZdq{}}\PYG{p}{,} \PYG{n}{tornado}\PYG{o}{.}\PYG{n}{web}\PYG{o}{.}\PYG{n}{StaticFileHandler}\PYG{p}{,}
     \PYG{n+nb}{dict}\PYG{p}{(}\PYG{n}{path}\PYG{o}{=}\PYG{n}{settings}\PYG{p}{[}\PYG{l+s+s1}{\PYGZsq{}}\PYG{l+s+s1}{static\PYGZus{}path}\PYG{l+s+s1}{\PYGZsq{}}\PYG{p}{]}\PYG{p}{)}\PYG{p}{)}\PYG{p}{,}
\PYG{p}{]}\PYG{p}{,} \PYG{o}{*}\PYG{o}{*}\PYG{n}{settings}\PYG{p}{)}
\end{sphinxVerbatim}

This setting will automatically make all requests that start with
\sphinxcode{\sphinxupquote{/static/}} serve from that static directory, e.g.
\sphinxcode{\sphinxupquote{http://localhost:8888/static/foo.png}} will serve the file
\sphinxcode{\sphinxupquote{foo.png}} from the specified static directory. We also automatically
serve \sphinxcode{\sphinxupquote{/robots.txt}} and \sphinxcode{\sphinxupquote{/favicon.ico}} from the static directory
(even though they don’t start with the \sphinxcode{\sphinxupquote{/static/}} prefix).

In the above settings, we have explicitly configured Tornado to serve
\sphinxcode{\sphinxupquote{apple-touch-icon.png}} from the root with the {\hyperref[\detokenize{web:tornado.web.StaticFileHandler}]{\sphinxcrossref{\sphinxcode{\sphinxupquote{StaticFileHandler}}}}},
though it is physically in the static file directory. (The capturing
group in that regular expression is necessary to tell
{\hyperref[\detokenize{web:tornado.web.StaticFileHandler}]{\sphinxcrossref{\sphinxcode{\sphinxupquote{StaticFileHandler}}}}} the requested filename; recall that capturing
groups are passed to handlers as method arguments.) You could do the
same thing to serve e.g. \sphinxcode{\sphinxupquote{sitemap.xml}} from the site root. Of
course, you can also avoid faking a root \sphinxcode{\sphinxupquote{apple-touch-icon.png}} by
using the appropriate \sphinxcode{\sphinxupquote{\textless{}link /\textgreater{}}} tag in your HTML.

To improve performance, it is generally a good idea for browsers to
cache static resources aggressively so browsers won’t send unnecessary
\sphinxcode{\sphinxupquote{If-Modified-Since}} or \sphinxcode{\sphinxupquote{Etag}} requests that might block the
rendering of the page. Tornado supports this out of the box with \sphinxstyleemphasis{static
content versioning}.

To use this feature, use the {\hyperref[\detokenize{web:tornado.web.RequestHandler.static_url}]{\sphinxcrossref{\sphinxcode{\sphinxupquote{static\_url}}}}} method in
your templates rather than typing the URL of the static file directly
in your HTML:

\begin{sphinxVerbatim}[commandchars=\\\{\}]
\PYG{o}{\PYGZlt{}}\PYG{n}{html}\PYG{o}{\PYGZgt{}}
   \PYG{o}{\PYGZlt{}}\PYG{n}{head}\PYG{o}{\PYGZgt{}}
      \PYG{o}{\PYGZlt{}}\PYG{n}{title}\PYG{o}{\PYGZgt{}}\PYG{n}{FriendFeed} \PYG{o}{\PYGZhy{}} \PYG{p}{\PYGZob{}}\PYG{p}{\PYGZob{}} \PYG{n}{\PYGZus{}}\PYG{p}{(}\PYG{l+s+s2}{\PYGZdq{}}\PYG{l+s+s2}{Home}\PYG{l+s+s2}{\PYGZdq{}}\PYG{p}{)} \PYG{p}{\PYGZcb{}}\PYG{p}{\PYGZcb{}}\PYG{o}{\PYGZlt{}}\PYG{o}{/}\PYG{n}{title}\PYG{o}{\PYGZgt{}}
   \PYG{o}{\PYGZlt{}}\PYG{o}{/}\PYG{n}{head}\PYG{o}{\PYGZgt{}}
   \PYG{o}{\PYGZlt{}}\PYG{n}{body}\PYG{o}{\PYGZgt{}}
     \PYG{o}{\PYGZlt{}}\PYG{n}{div}\PYG{o}{\PYGZgt{}}\PYG{o}{\PYGZlt{}}\PYG{n}{img} \PYG{n}{src}\PYG{o}{=}\PYG{l+s+s2}{\PYGZdq{}}\PYG{l+s+s2}{\PYGZob{}\PYGZob{}}\PYG{l+s+s2}{ static\PYGZus{}url(}\PYG{l+s+s2}{\PYGZdq{}}\PYG{n}{images}\PYG{o}{/}\PYG{n}{logo}\PYG{o}{.}\PYG{n}{png}\PYG{l+s+s2}{\PYGZdq{}}\PYG{l+s+s2}{) \PYGZcb{}\PYGZcb{}}\PYG{l+s+s2}{\PYGZdq{}}\PYG{o}{/}\PYG{o}{\PYGZgt{}}\PYG{o}{\PYGZlt{}}\PYG{o}{/}\PYG{n}{div}\PYG{o}{\PYGZgt{}}
   \PYG{o}{\PYGZlt{}}\PYG{o}{/}\PYG{n}{body}\PYG{o}{\PYGZgt{}}
 \PYG{o}{\PYGZlt{}}\PYG{o}{/}\PYG{n}{html}\PYG{o}{\PYGZgt{}}
\end{sphinxVerbatim}

The \sphinxcode{\sphinxupquote{static\_url()}} function will translate that relative path to a URI
that looks like \sphinxcode{\sphinxupquote{/static/images/logo.png?v=aae54}}. The \sphinxcode{\sphinxupquote{v}} argument
is a hash of the content in \sphinxcode{\sphinxupquote{logo.png}}, and its presence makes the
Tornado server send cache headers to the user’s browser that will make
the browser cache the content indefinitely.

Since the \sphinxcode{\sphinxupquote{v}} argument is based on the content of the file, if you
update a file and restart your server, it will start sending a new \sphinxcode{\sphinxupquote{v}}
value, so the user’s browser will automatically fetch the new file. If
the file’s contents don’t change, the browser will continue to use a
locally cached copy without ever checking for updates on the server,
significantly improving rendering performance.

In production, you probably want to serve static files from a more
optimized static file server like \sphinxhref{http://nginx.net/}{nginx}. You
can configure almost any web server to recognize the version tags used
by \sphinxcode{\sphinxupquote{static\_url()}} and set caching headers accordingly.  Here is the
relevant portion of the nginx configuration we use at FriendFeed:

\begin{sphinxVerbatim}[commandchars=\\\{\}]
location /static/ \PYGZob{}
    root /var/friendfeed/static;
    if (\PYGZdl{}query\PYGZus{}string) \PYGZob{}
        expires max;
    \PYGZcb{}
 \PYGZcb{}
\end{sphinxVerbatim}


\subsubsection{Debug mode and automatic reloading}
\label{\detokenize{guide/running:debug-mode-and-automatic-reloading}}\label{\detokenize{guide/running:debug-mode}}
If you pass \sphinxcode{\sphinxupquote{debug=True}} to the \sphinxcode{\sphinxupquote{Application}} constructor, the app
will be run in debug/development mode. In this mode, several features
intended for convenience while developing will be enabled (each of which
is also available as an individual flag; if both are specified the
individual flag takes precedence):
\begin{itemize}
\item {} 
\sphinxcode{\sphinxupquote{autoreload=True}}: The app will watch for changes to its source
files and reload itself when anything changes. This reduces the need
to manually restart the server during development. However, certain
failures (such as syntax errors at import time) can still take the
server down in a way that debug mode cannot currently recover from.

\item {} 
\sphinxcode{\sphinxupquote{compiled\_template\_cache=False}}: Templates will not be cached.

\item {} 
\sphinxcode{\sphinxupquote{static\_hash\_cache=False}}: Static file hashes (used by the
\sphinxcode{\sphinxupquote{static\_url}} function) will not be cached.

\item {} 
\sphinxcode{\sphinxupquote{serve\_traceback=True}}: When an exception in a {\hyperref[\detokenize{web:tornado.web.RequestHandler}]{\sphinxcrossref{\sphinxcode{\sphinxupquote{RequestHandler}}}}}
is not caught, an error page including a stack trace will be
generated.

\end{itemize}

Autoreload mode is not compatible with the multi-process mode of {\hyperref[\detokenize{httpserver:tornado.httpserver.HTTPServer}]{\sphinxcrossref{\sphinxcode{\sphinxupquote{HTTPServer}}}}}.
You must not give {\hyperref[\detokenize{tcpserver:tornado.tcpserver.TCPServer.start}]{\sphinxcrossref{\sphinxcode{\sphinxupquote{HTTPServer.start}}}}} an argument other than 1 (or
call {\hyperref[\detokenize{process:tornado.process.fork_processes}]{\sphinxcrossref{\sphinxcode{\sphinxupquote{tornado.process.fork\_processes}}}}}) if you are using autoreload mode.

The automatic reloading feature of debug mode is available as a
standalone module in {\hyperref[\detokenize{autoreload:module-tornado.autoreload}]{\sphinxcrossref{\sphinxcode{\sphinxupquote{tornado.autoreload}}}}}.  The two can be used in
combination to provide extra robustness against syntax errors: set
\sphinxcode{\sphinxupquote{autoreload=True}} within the app to detect changes while it is running,
and start it with \sphinxcode{\sphinxupquote{python -m tornado.autoreload myserver.py}} to catch
any syntax errors or other errors at startup.

Reloading loses any Python interpreter command-line arguments (e.g. \sphinxcode{\sphinxupquote{-u}})
because it re-executes Python using \sphinxhref{https://docs.python.org/3.6/library/sys.html\#sys.executable}{\sphinxcode{\sphinxupquote{sys.executable}}} and \sphinxhref{https://docs.python.org/3.6/library/sys.html\#sys.argv}{\sphinxcode{\sphinxupquote{sys.argv}}}.
Additionally, modifying these variables will cause reloading to behave
incorrectly.

On some platforms (including Windows and Mac OSX prior to 10.6), the
process cannot be updated “in-place”, so when a code change is
detected the old server exits and a new one starts.  This has been
known to confuse some IDEs.


\section{Web framework}
\label{\detokenize{webframework:web-framework}}\label{\detokenize{webframework::doc}}

\subsection{\sphinxstyleliteralintitle{\sphinxupquote{tornado.web}} — \sphinxstyleliteralintitle{\sphinxupquote{RequestHandler}} and \sphinxstyleliteralintitle{\sphinxupquote{Application}} classes}
\label{\detokenize{web:tornado-web-requesthandler-and-application-classes}}\label{\detokenize{web::doc}}\phantomsection\label{\detokenize{web:module-tornado.web}}\index{tornado.web (module)@\spxentry{tornado.web}\spxextra{module}}
\sphinxcode{\sphinxupquote{tornado.web}} provides a simple web framework with asynchronous
features that allow it to scale to large numbers of open connections,
making it ideal for \sphinxhref{http://en.wikipedia.org/wiki/Push\_technology\#Long\_polling}{long polling}.

Here is a simple “Hello, world” example app:

\begin{sphinxVerbatim}[commandchars=\\\{\}]
\PYG{k+kn}{import} \PYG{n+nn}{tornado}\PYG{n+nn}{.}\PYG{n+nn}{ioloop}
\PYG{k+kn}{import} \PYG{n+nn}{tornado}\PYG{n+nn}{.}\PYG{n+nn}{web}

\PYG{k}{class} \PYG{n+nc}{MainHandler}\PYG{p}{(}\PYG{n}{tornado}\PYG{o}{.}\PYG{n}{web}\PYG{o}{.}\PYG{n}{RequestHandler}\PYG{p}{)}\PYG{p}{:}
    \PYG{k}{def} \PYG{n+nf}{get}\PYG{p}{(}\PYG{n+nb+bp}{self}\PYG{p}{)}\PYG{p}{:}
        \PYG{n+nb+bp}{self}\PYG{o}{.}\PYG{n}{write}\PYG{p}{(}\PYG{l+s+s2}{\PYGZdq{}}\PYG{l+s+s2}{Hello, world}\PYG{l+s+s2}{\PYGZdq{}}\PYG{p}{)}

\PYG{k}{if} \PYG{n+nv+vm}{\PYGZus{}\PYGZus{}name\PYGZus{}\PYGZus{}} \PYG{o}{==} \PYG{l+s+s2}{\PYGZdq{}}\PYG{l+s+s2}{\PYGZus{}\PYGZus{}main\PYGZus{}\PYGZus{}}\PYG{l+s+s2}{\PYGZdq{}}\PYG{p}{:}
    \PYG{n}{application} \PYG{o}{=} \PYG{n}{tornado}\PYG{o}{.}\PYG{n}{web}\PYG{o}{.}\PYG{n}{Application}\PYG{p}{(}\PYG{p}{[}
        \PYG{p}{(}\PYG{l+s+sa}{r}\PYG{l+s+s2}{\PYGZdq{}}\PYG{l+s+s2}{/}\PYG{l+s+s2}{\PYGZdq{}}\PYG{p}{,} \PYG{n}{MainHandler}\PYG{p}{)}\PYG{p}{,}
    \PYG{p}{]}\PYG{p}{)}
    \PYG{n}{application}\PYG{o}{.}\PYG{n}{listen}\PYG{p}{(}\PYG{l+m+mi}{8888}\PYG{p}{)}
    \PYG{n}{tornado}\PYG{o}{.}\PYG{n}{ioloop}\PYG{o}{.}\PYG{n}{IOLoop}\PYG{o}{.}\PYG{n}{current}\PYG{p}{(}\PYG{p}{)}\PYG{o}{.}\PYG{n}{start}\PYG{p}{(}\PYG{p}{)}
\end{sphinxVerbatim}

See the {\hyperref[\detokenize{guide::doc}]{\sphinxcrossref{\DUrole{doc}{User’s guide}}}} for additional information.


\subsubsection{Thread-safety notes}
\label{\detokenize{web:thread-safety-notes}}
In general, methods on {\hyperref[\detokenize{web:tornado.web.RequestHandler}]{\sphinxcrossref{\sphinxcode{\sphinxupquote{RequestHandler}}}}} and elsewhere in Tornado are
not thread-safe. In particular, methods such as
{\hyperref[\detokenize{web:tornado.web.RequestHandler.write}]{\sphinxcrossref{\sphinxcode{\sphinxupquote{write()}}}}}, {\hyperref[\detokenize{web:tornado.web.RequestHandler.finish}]{\sphinxcrossref{\sphinxcode{\sphinxupquote{finish()}}}}}, and
{\hyperref[\detokenize{web:tornado.web.RequestHandler.flush}]{\sphinxcrossref{\sphinxcode{\sphinxupquote{flush()}}}}} must only be called from the main thread. If
you use multiple threads it is important to use {\hyperref[\detokenize{ioloop:tornado.ioloop.IOLoop.add_callback}]{\sphinxcrossref{\sphinxcode{\sphinxupquote{IOLoop.add\_callback}}}}}
to transfer control back to the main thread before finishing the
request, or to limit your use of other threads to
{\hyperref[\detokenize{ioloop:tornado.ioloop.IOLoop.run_in_executor}]{\sphinxcrossref{\sphinxcode{\sphinxupquote{IOLoop.run\_in\_executor}}}}} and ensure that your callbacks running in
the executor do not refer to Tornado objects.


\subsubsection{Request handlers}
\label{\detokenize{web:request-handlers}}\index{RequestHandler (class in tornado.web)@\spxentry{RequestHandler}\spxextra{class in tornado.web}}

\begin{fulllineitems}
\phantomsection\label{\detokenize{web:tornado.web.RequestHandler}}\pysiglinewithargsret{\sphinxbfcode{\sphinxupquote{class }}\sphinxcode{\sphinxupquote{tornado.web.}}\sphinxbfcode{\sphinxupquote{RequestHandler}}}{\emph{...}}{}
Base class for HTTP request handlers.

Subclasses must define at least one of the methods defined in the
“Entry points” section below.

Applications should not construct {\hyperref[\detokenize{web:tornado.web.RequestHandler}]{\sphinxcrossref{\sphinxcode{\sphinxupquote{RequestHandler}}}}} objects
directly and subclasses should not override \sphinxcode{\sphinxupquote{\_\_init\_\_}} (override
{\hyperref[\detokenize{web:tornado.web.RequestHandler.initialize}]{\sphinxcrossref{\sphinxcode{\sphinxupquote{initialize}}}}} instead).

\end{fulllineitems}



\paragraph{Entry points}
\label{\detokenize{web:entry-points}}\index{initialize() (tornado.web.RequestHandler method)@\spxentry{initialize()}\spxextra{tornado.web.RequestHandler method}}

\begin{fulllineitems}
\phantomsection\label{\detokenize{web:tornado.web.RequestHandler.initialize}}\pysiglinewithargsret{\sphinxcode{\sphinxupquote{RequestHandler.}}\sphinxbfcode{\sphinxupquote{initialize}}}{}{{ $\rightarrow$ None}}
Hook for subclass initialization. Called for each request.

A dictionary passed as the third argument of a \sphinxcode{\sphinxupquote{URLSpec}} will be
supplied as keyword arguments to \sphinxcode{\sphinxupquote{initialize()}}.

Example:

\begin{sphinxVerbatim}[commandchars=\\\{\}]
\PYG{k}{class} \PYG{n+nc}{ProfileHandler}\PYG{p}{(}\PYG{n}{RequestHandler}\PYG{p}{)}\PYG{p}{:}
    \PYG{k}{def} \PYG{n+nf}{initialize}\PYG{p}{(}\PYG{n+nb+bp}{self}\PYG{p}{,} \PYG{n}{database}\PYG{p}{)}\PYG{p}{:}
        \PYG{n+nb+bp}{self}\PYG{o}{.}\PYG{n}{database} \PYG{o}{=} \PYG{n}{database}

    \PYG{k}{def} \PYG{n+nf}{get}\PYG{p}{(}\PYG{n+nb+bp}{self}\PYG{p}{,} \PYG{n}{username}\PYG{p}{)}\PYG{p}{:}
        \PYG{o}{.}\PYG{o}{.}\PYG{o}{.}

\PYG{n}{app} \PYG{o}{=} \PYG{n}{Application}\PYG{p}{(}\PYG{p}{[}
    \PYG{p}{(}\PYG{l+s+sa}{r}\PYG{l+s+s1}{\PYGZsq{}}\PYG{l+s+s1}{/user/(.*)}\PYG{l+s+s1}{\PYGZsq{}}\PYG{p}{,} \PYG{n}{ProfileHandler}\PYG{p}{,} \PYG{n+nb}{dict}\PYG{p}{(}\PYG{n}{database}\PYG{o}{=}\PYG{n}{database}\PYG{p}{)}\PYG{p}{)}\PYG{p}{,}
    \PYG{p}{]}\PYG{p}{)}
\end{sphinxVerbatim}

\end{fulllineitems}

\index{prepare() (tornado.web.RequestHandler method)@\spxentry{prepare()}\spxextra{tornado.web.RequestHandler method}}

\begin{fulllineitems}
\phantomsection\label{\detokenize{web:tornado.web.RequestHandler.prepare}}\pysiglinewithargsret{\sphinxcode{\sphinxupquote{RequestHandler.}}\sphinxbfcode{\sphinxupquote{prepare}}}{}{{ $\rightarrow$ Optional{[}Awaitable{[}None{]}{]}}}
Called at the beginning of a request before  {\hyperref[\detokenize{web:tornado.web.RequestHandler.get}]{\sphinxcrossref{\sphinxcode{\sphinxupquote{get}}}}}/{\hyperref[\detokenize{web:tornado.web.RequestHandler.post}]{\sphinxcrossref{\sphinxcode{\sphinxupquote{post}}}}}/etc.

Override this method to perform common initialization regardless
of the request method.

Asynchronous support: Use \sphinxcode{\sphinxupquote{async def}} or decorate this method with
{\hyperref[\detokenize{gen:tornado.gen.coroutine}]{\sphinxcrossref{\sphinxcode{\sphinxupquote{gen.coroutine}}}}} to make it asynchronous.
If this method returns an  \sphinxcode{\sphinxupquote{Awaitable}} execution will not proceed
until the \sphinxcode{\sphinxupquote{Awaitable}} is done.

\DUrole{versionmodified,added}{New in version 3.1: }Asynchronous support.

\end{fulllineitems}

\index{on\_finish() (tornado.web.RequestHandler method)@\spxentry{on\_finish()}\spxextra{tornado.web.RequestHandler method}}

\begin{fulllineitems}
\phantomsection\label{\detokenize{web:tornado.web.RequestHandler.on_finish}}\pysiglinewithargsret{\sphinxcode{\sphinxupquote{RequestHandler.}}\sphinxbfcode{\sphinxupquote{on\_finish}}}{}{{ $\rightarrow$ None}}
Called after the end of a request.

Override this method to perform cleanup, logging, etc.
This method is a counterpart to {\hyperref[\detokenize{web:tornado.web.RequestHandler.prepare}]{\sphinxcrossref{\sphinxcode{\sphinxupquote{prepare}}}}}.  \sphinxcode{\sphinxupquote{on\_finish}} may
not produce any output, as it is called after the response
has been sent to the client.

\end{fulllineitems}

\phantomsection\label{\detokenize{web:verbs}}
Implement any of the following methods (collectively known as the
HTTP verb methods) to handle the corresponding HTTP method. These
methods can be made asynchronous with the \sphinxcode{\sphinxupquote{async def}} keyword or
{\hyperref[\detokenize{gen:tornado.gen.coroutine}]{\sphinxcrossref{\sphinxcode{\sphinxupquote{gen.coroutine}}}}} decorator.

The arguments to these methods come from the {\hyperref[\detokenize{web:tornado.web.URLSpec}]{\sphinxcrossref{\sphinxcode{\sphinxupquote{URLSpec}}}}}: Any
capturing groups in the regular expression become arguments to the
HTTP verb methods (keyword arguments if the group is named,
positional arguments if it’s unnamed).

To support a method not on this list, override the class variable
\sphinxcode{\sphinxupquote{SUPPORTED\_METHODS}}:

\begin{sphinxVerbatim}[commandchars=\\\{\}]
\PYG{k}{class} \PYG{n+nc}{WebDAVHandler}\PYG{p}{(}\PYG{n}{RequestHandler}\PYG{p}{)}\PYG{p}{:}
    \PYG{n}{SUPPORTED\PYGZus{}METHODS} \PYG{o}{=} \PYG{n}{RequestHandler}\PYG{o}{.}\PYG{n}{SUPPORTED\PYGZus{}METHODS} \PYG{o}{+} \PYG{p}{(}\PYG{l+s+s1}{\PYGZsq{}}\PYG{l+s+s1}{PROPFIND}\PYG{l+s+s1}{\PYGZsq{}}\PYG{p}{,}\PYG{p}{)}

    \PYG{k}{def} \PYG{n+nf}{propfind}\PYG{p}{(}\PYG{n+nb+bp}{self}\PYG{p}{)}\PYG{p}{:}
        \PYG{k}{pass}
\end{sphinxVerbatim}
\index{get() (tornado.web.RequestHandler method)@\spxentry{get()}\spxextra{tornado.web.RequestHandler method}}

\begin{fulllineitems}
\phantomsection\label{\detokenize{web:tornado.web.RequestHandler.get}}\pysiglinewithargsret{\sphinxcode{\sphinxupquote{RequestHandler.}}\sphinxbfcode{\sphinxupquote{get}}}{\emph{*args}, \emph{**kwargs}}{{ $\rightarrow$ None}}
\end{fulllineitems}

\index{head() (tornado.web.RequestHandler method)@\spxentry{head()}\spxextra{tornado.web.RequestHandler method}}

\begin{fulllineitems}
\phantomsection\label{\detokenize{web:tornado.web.RequestHandler.head}}\pysiglinewithargsret{\sphinxcode{\sphinxupquote{RequestHandler.}}\sphinxbfcode{\sphinxupquote{head}}}{\emph{*args}, \emph{**kwargs}}{{ $\rightarrow$ None}}
\end{fulllineitems}

\index{post() (tornado.web.RequestHandler method)@\spxentry{post()}\spxextra{tornado.web.RequestHandler method}}

\begin{fulllineitems}
\phantomsection\label{\detokenize{web:tornado.web.RequestHandler.post}}\pysiglinewithargsret{\sphinxcode{\sphinxupquote{RequestHandler.}}\sphinxbfcode{\sphinxupquote{post}}}{\emph{*args}, \emph{**kwargs}}{{ $\rightarrow$ None}}
\end{fulllineitems}

\index{delete() (tornado.web.RequestHandler method)@\spxentry{delete()}\spxextra{tornado.web.RequestHandler method}}

\begin{fulllineitems}
\phantomsection\label{\detokenize{web:tornado.web.RequestHandler.delete}}\pysiglinewithargsret{\sphinxcode{\sphinxupquote{RequestHandler.}}\sphinxbfcode{\sphinxupquote{delete}}}{\emph{*args}, \emph{**kwargs}}{{ $\rightarrow$ None}}
\end{fulllineitems}

\index{patch() (tornado.web.RequestHandler method)@\spxentry{patch()}\spxextra{tornado.web.RequestHandler method}}

\begin{fulllineitems}
\phantomsection\label{\detokenize{web:tornado.web.RequestHandler.patch}}\pysiglinewithargsret{\sphinxcode{\sphinxupquote{RequestHandler.}}\sphinxbfcode{\sphinxupquote{patch}}}{\emph{*args}, \emph{**kwargs}}{{ $\rightarrow$ None}}
\end{fulllineitems}

\index{put() (tornado.web.RequestHandler method)@\spxentry{put()}\spxextra{tornado.web.RequestHandler method}}

\begin{fulllineitems}
\phantomsection\label{\detokenize{web:tornado.web.RequestHandler.put}}\pysiglinewithargsret{\sphinxcode{\sphinxupquote{RequestHandler.}}\sphinxbfcode{\sphinxupquote{put}}}{\emph{*args}, \emph{**kwargs}}{{ $\rightarrow$ None}}
\end{fulllineitems}

\index{options() (tornado.web.RequestHandler method)@\spxentry{options()}\spxextra{tornado.web.RequestHandler method}}

\begin{fulllineitems}
\phantomsection\label{\detokenize{web:tornado.web.RequestHandler.options}}\pysiglinewithargsret{\sphinxcode{\sphinxupquote{RequestHandler.}}\sphinxbfcode{\sphinxupquote{options}}}{\emph{*args}, \emph{**kwargs}}{{ $\rightarrow$ None}}
\end{fulllineitems}



\paragraph{Input}
\label{\detokenize{web:input}}
The \sphinxcode{\sphinxupquote{argument}} methods provide support for HTML form-style
arguments. These methods are available in both singular and plural
forms because HTML forms are ambiguous and do not distinguish
between a singular argument and a list containing one entry. If you
wish to use other formats for arguments (for example, JSON), parse
\sphinxcode{\sphinxupquote{self.request.body}} yourself:

\begin{sphinxVerbatim}[commandchars=\\\{\}]
\PYG{k}{def} \PYG{n+nf}{prepare}\PYG{p}{(}\PYG{n+nb+bp}{self}\PYG{p}{)}\PYG{p}{:}
    \PYG{k}{if} \PYG{n+nb+bp}{self}\PYG{o}{.}\PYG{n}{request}\PYG{o}{.}\PYG{n}{headers}\PYG{p}{[}\PYG{l+s+s1}{\PYGZsq{}}\PYG{l+s+s1}{Content\PYGZhy{}Type}\PYG{l+s+s1}{\PYGZsq{}}\PYG{p}{]} \PYG{o}{==} \PYG{l+s+s1}{\PYGZsq{}}\PYG{l+s+s1}{application/x\PYGZhy{}json}\PYG{l+s+s1}{\PYGZsq{}}\PYG{p}{:}
        \PYG{n+nb+bp}{self}\PYG{o}{.}\PYG{n}{args} \PYG{o}{=} \PYG{n}{json\PYGZus{}decode}\PYG{p}{(}\PYG{n+nb+bp}{self}\PYG{o}{.}\PYG{n}{request}\PYG{o}{.}\PYG{n}{body}\PYG{p}{)}
    \PYG{c+c1}{\PYGZsh{} Access self.args directly instead of using self.get\PYGZus{}argument.}
\end{sphinxVerbatim}
\index{get\_argument() (tornado.web.RequestHandler method)@\spxentry{get\_argument()}\spxextra{tornado.web.RequestHandler method}}

\begin{fulllineitems}
\phantomsection\label{\detokenize{web:tornado.web.RequestHandler.get_argument}}\pysiglinewithargsret{\sphinxcode{\sphinxupquote{RequestHandler.}}\sphinxbfcode{\sphinxupquote{get\_argument}}}{\emph{name: str}, \emph{default: Union{[}None}, \emph{str}, \emph{RAISE{]} = RAISE}, \emph{strip: bool = True}}{{ $\rightarrow$ Optional{[}str{]}}}
Returns the value of the argument with the given name.

If default is not provided, the argument is considered to be
required, and we raise a {\hyperref[\detokenize{web:tornado.web.MissingArgumentError}]{\sphinxcrossref{\sphinxcode{\sphinxupquote{MissingArgumentError}}}}} if it is missing.

If the argument appears in the request more than once, we return the
last value.

This method searches both the query and body arguments.

\end{fulllineitems}

\index{get\_arguments() (tornado.web.RequestHandler method)@\spxentry{get\_arguments()}\spxextra{tornado.web.RequestHandler method}}

\begin{fulllineitems}
\phantomsection\label{\detokenize{web:tornado.web.RequestHandler.get_arguments}}\pysiglinewithargsret{\sphinxcode{\sphinxupquote{RequestHandler.}}\sphinxbfcode{\sphinxupquote{get\_arguments}}}{\emph{name: str}, \emph{strip: bool = True}}{{ $\rightarrow$ List{[}str{]}}}
Returns a list of the arguments with the given name.

If the argument is not present, returns an empty list.

This method searches both the query and body arguments.

\end{fulllineitems}

\index{get\_query\_argument() (tornado.web.RequestHandler method)@\spxentry{get\_query\_argument()}\spxextra{tornado.web.RequestHandler method}}

\begin{fulllineitems}
\phantomsection\label{\detokenize{web:tornado.web.RequestHandler.get_query_argument}}\pysiglinewithargsret{\sphinxcode{\sphinxupquote{RequestHandler.}}\sphinxbfcode{\sphinxupquote{get\_query\_argument}}}{\emph{name: str}, \emph{default: Union{[}None}, \emph{str}, \emph{RAISE{]} = RAISE}, \emph{strip: bool = True}}{{ $\rightarrow$ Optional{[}str{]}}}
Returns the value of the argument with the given name
from the request query string.

If default is not provided, the argument is considered to be
required, and we raise a {\hyperref[\detokenize{web:tornado.web.MissingArgumentError}]{\sphinxcrossref{\sphinxcode{\sphinxupquote{MissingArgumentError}}}}} if it is missing.

If the argument appears in the url more than once, we return the
last value.

\DUrole{versionmodified,added}{New in version 3.2.}

\end{fulllineitems}

\index{get\_query\_arguments() (tornado.web.RequestHandler method)@\spxentry{get\_query\_arguments()}\spxextra{tornado.web.RequestHandler method}}

\begin{fulllineitems}
\phantomsection\label{\detokenize{web:tornado.web.RequestHandler.get_query_arguments}}\pysiglinewithargsret{\sphinxcode{\sphinxupquote{RequestHandler.}}\sphinxbfcode{\sphinxupquote{get\_query\_arguments}}}{\emph{name: str}, \emph{strip: bool = True}}{{ $\rightarrow$ List{[}str{]}}}
Returns a list of the query arguments with the given name.

If the argument is not present, returns an empty list.

\DUrole{versionmodified,added}{New in version 3.2.}

\end{fulllineitems}

\index{get\_body\_argument() (tornado.web.RequestHandler method)@\spxentry{get\_body\_argument()}\spxextra{tornado.web.RequestHandler method}}

\begin{fulllineitems}
\phantomsection\label{\detokenize{web:tornado.web.RequestHandler.get_body_argument}}\pysiglinewithargsret{\sphinxcode{\sphinxupquote{RequestHandler.}}\sphinxbfcode{\sphinxupquote{get\_body\_argument}}}{\emph{name: str}, \emph{default: Union{[}None}, \emph{str}, \emph{RAISE{]} = RAISE}, \emph{strip: bool = True}}{{ $\rightarrow$ Optional{[}str{]}}}
Returns the value of the argument with the given name
from the request body.

If default is not provided, the argument is considered to be
required, and we raise a {\hyperref[\detokenize{web:tornado.web.MissingArgumentError}]{\sphinxcrossref{\sphinxcode{\sphinxupquote{MissingArgumentError}}}}} if it is missing.

If the argument appears in the url more than once, we return the
last value.

\DUrole{versionmodified,added}{New in version 3.2.}

\end{fulllineitems}

\index{get\_body\_arguments() (tornado.web.RequestHandler method)@\spxentry{get\_body\_arguments()}\spxextra{tornado.web.RequestHandler method}}

\begin{fulllineitems}
\phantomsection\label{\detokenize{web:tornado.web.RequestHandler.get_body_arguments}}\pysiglinewithargsret{\sphinxcode{\sphinxupquote{RequestHandler.}}\sphinxbfcode{\sphinxupquote{get\_body\_arguments}}}{\emph{name: str}, \emph{strip: bool = True}}{{ $\rightarrow$ List{[}str{]}}}
Returns a list of the body arguments with the given name.

If the argument is not present, returns an empty list.

\DUrole{versionmodified,added}{New in version 3.2.}

\end{fulllineitems}

\index{decode\_argument() (tornado.web.RequestHandler method)@\spxentry{decode\_argument()}\spxextra{tornado.web.RequestHandler method}}

\begin{fulllineitems}
\phantomsection\label{\detokenize{web:tornado.web.RequestHandler.decode_argument}}\pysiglinewithargsret{\sphinxcode{\sphinxupquote{RequestHandler.}}\sphinxbfcode{\sphinxupquote{decode\_argument}}}{\emph{value: bytes}, \emph{name: str = None}}{{ $\rightarrow$ str}}
Decodes an argument from the request.

The argument has been percent-decoded and is now a byte string.
By default, this method decodes the argument as utf-8 and returns
a unicode string, but this may be overridden in subclasses.

This method is used as a filter for both {\hyperref[\detokenize{web:tornado.web.RequestHandler.get_argument}]{\sphinxcrossref{\sphinxcode{\sphinxupquote{get\_argument()}}}}} and for
values extracted from the url and passed to {\hyperref[\detokenize{web:tornado.web.RequestHandler.get}]{\sphinxcrossref{\sphinxcode{\sphinxupquote{get()}}}}}/{\hyperref[\detokenize{web:tornado.web.RequestHandler.post}]{\sphinxcrossref{\sphinxcode{\sphinxupquote{post()}}}}}/etc.

The name of the argument is provided if known, but may be None
(e.g. for unnamed groups in the url regex).

\end{fulllineitems}

\index{request (tornado.web.RequestHandler attribute)@\spxentry{request}\spxextra{tornado.web.RequestHandler attribute}}

\begin{fulllineitems}
\phantomsection\label{\detokenize{web:tornado.web.RequestHandler.request}}\pysigline{\sphinxcode{\sphinxupquote{RequestHandler.}}\sphinxbfcode{\sphinxupquote{request}}}
The {\hyperref[\detokenize{httputil:tornado.httputil.HTTPServerRequest}]{\sphinxcrossref{\sphinxcode{\sphinxupquote{tornado.httputil.HTTPServerRequest}}}}} object containing additional
request parameters including e.g. headers and body data.

\end{fulllineitems}

\index{path\_args (tornado.web.RequestHandler attribute)@\spxentry{path\_args}\spxextra{tornado.web.RequestHandler attribute}}

\begin{fulllineitems}
\phantomsection\label{\detokenize{web:tornado.web.RequestHandler.path_args}}\pysigline{\sphinxcode{\sphinxupquote{RequestHandler.}}\sphinxbfcode{\sphinxupquote{path\_args}}}
\end{fulllineitems}

\index{path\_kwargs (tornado.web.RequestHandler attribute)@\spxentry{path\_kwargs}\spxextra{tornado.web.RequestHandler attribute}}

\begin{fulllineitems}
\phantomsection\label{\detokenize{web:tornado.web.RequestHandler.path_kwargs}}\pysigline{\sphinxcode{\sphinxupquote{RequestHandler.}}\sphinxbfcode{\sphinxupquote{path\_kwargs}}}
The \sphinxcode{\sphinxupquote{path\_args}} and \sphinxcode{\sphinxupquote{path\_kwargs}} attributes contain the
positional and keyword arguments that are passed to the
{\hyperref[\detokenize{web:verbs}]{\sphinxcrossref{\DUrole{std,std-ref}{HTTP verb methods}}}}.  These attributes are set
before those methods are called, so the values are available
during {\hyperref[\detokenize{web:tornado.web.RequestHandler.prepare}]{\sphinxcrossref{\sphinxcode{\sphinxupquote{prepare}}}}}.

\end{fulllineitems}

\index{data\_received() (tornado.web.RequestHandler method)@\spxentry{data\_received()}\spxextra{tornado.web.RequestHandler method}}

\begin{fulllineitems}
\phantomsection\label{\detokenize{web:tornado.web.RequestHandler.data_received}}\pysiglinewithargsret{\sphinxcode{\sphinxupquote{RequestHandler.}}\sphinxbfcode{\sphinxupquote{data\_received}}}{\emph{chunk: bytes}}{{ $\rightarrow$ Optional{[}Awaitable{[}None{]}{]}}}
Implement this method to handle streamed request data.

Requires the {\hyperref[\detokenize{web:tornado.web.stream_request_body}]{\sphinxcrossref{\sphinxcode{\sphinxupquote{stream\_request\_body}}}}} decorator.

May be a coroutine for flow control.

\end{fulllineitems}



\paragraph{Output}
\label{\detokenize{web:output}}\index{set\_status() (tornado.web.RequestHandler method)@\spxentry{set\_status()}\spxextra{tornado.web.RequestHandler method}}

\begin{fulllineitems}
\phantomsection\label{\detokenize{web:tornado.web.RequestHandler.set_status}}\pysiglinewithargsret{\sphinxcode{\sphinxupquote{RequestHandler.}}\sphinxbfcode{\sphinxupquote{set\_status}}}{\emph{status\_code: int}, \emph{reason: str = None}}{{ $\rightarrow$ None}}
Sets the status code for our response.
\begin{quote}\begin{description}
\item[{Parameters}] \leavevmode\begin{itemize}
\item {} 
\sphinxstyleliteralstrong{\sphinxupquote{status\_code}} (\sphinxhref{https://docs.python.org/3.6/library/functions.html\#int}{\sphinxstyleliteralemphasis{\sphinxupquote{int}}}) \textendash{} Response status code.

\item {} 
\sphinxstyleliteralstrong{\sphinxupquote{reason}} (\sphinxhref{https://docs.python.org/3.6/library/stdtypes.html\#str}{\sphinxstyleliteralemphasis{\sphinxupquote{str}}}) \textendash{} Human-readable reason phrase describing the status
code. If \sphinxcode{\sphinxupquote{None}}, it will be filled in from
\sphinxhref{https://docs.python.org/3.6/library/http.client.html\#http.client.responses}{\sphinxcode{\sphinxupquote{http.client.responses}}} or “Unknown”.

\end{itemize}

\end{description}\end{quote}

\DUrole{versionmodified,changed}{Changed in version 5.0: }No longer validates that the response code is in
\sphinxhref{https://docs.python.org/3.6/library/http.client.html\#http.client.responses}{\sphinxcode{\sphinxupquote{http.client.responses}}}.

\end{fulllineitems}

\index{set\_header() (tornado.web.RequestHandler method)@\spxentry{set\_header()}\spxextra{tornado.web.RequestHandler method}}

\begin{fulllineitems}
\phantomsection\label{\detokenize{web:tornado.web.RequestHandler.set_header}}\pysiglinewithargsret{\sphinxcode{\sphinxupquote{RequestHandler.}}\sphinxbfcode{\sphinxupquote{set\_header}}}{\emph{name: str, value: Union{[}bytes, str, int, numbers.Integral, datetime.datetime{]}}}{{ $\rightarrow$ None}}
Sets the given response header name and value.

All header values are converted to strings (\sphinxhref{https://docs.python.org/3.6/library/datetime.html\#module-datetime}{\sphinxcode{\sphinxupquote{datetime}}} objects
are formatted according to the HTTP specification for the
\sphinxcode{\sphinxupquote{Date}} header).

\end{fulllineitems}

\index{add\_header() (tornado.web.RequestHandler method)@\spxentry{add\_header()}\spxextra{tornado.web.RequestHandler method}}

\begin{fulllineitems}
\phantomsection\label{\detokenize{web:tornado.web.RequestHandler.add_header}}\pysiglinewithargsret{\sphinxcode{\sphinxupquote{RequestHandler.}}\sphinxbfcode{\sphinxupquote{add\_header}}}{\emph{name: str, value: Union{[}bytes, str, int, numbers.Integral, datetime.datetime{]}}}{{ $\rightarrow$ None}}
Adds the given response header and value.

Unlike {\hyperref[\detokenize{web:tornado.web.RequestHandler.set_header}]{\sphinxcrossref{\sphinxcode{\sphinxupquote{set\_header}}}}}, {\hyperref[\detokenize{web:tornado.web.RequestHandler.add_header}]{\sphinxcrossref{\sphinxcode{\sphinxupquote{add\_header}}}}} may be called multiple times
to return multiple values for the same header.

\end{fulllineitems}

\index{clear\_header() (tornado.web.RequestHandler method)@\spxentry{clear\_header()}\spxextra{tornado.web.RequestHandler method}}

\begin{fulllineitems}
\phantomsection\label{\detokenize{web:tornado.web.RequestHandler.clear_header}}\pysiglinewithargsret{\sphinxcode{\sphinxupquote{RequestHandler.}}\sphinxbfcode{\sphinxupquote{clear\_header}}}{\emph{name: str}}{{ $\rightarrow$ None}}
Clears an outgoing header, undoing a previous {\hyperref[\detokenize{web:tornado.web.RequestHandler.set_header}]{\sphinxcrossref{\sphinxcode{\sphinxupquote{set\_header}}}}} call.

Note that this method does not apply to multi-valued headers
set by {\hyperref[\detokenize{web:tornado.web.RequestHandler.add_header}]{\sphinxcrossref{\sphinxcode{\sphinxupquote{add\_header}}}}}.

\end{fulllineitems}

\index{set\_default\_headers() (tornado.web.RequestHandler method)@\spxentry{set\_default\_headers()}\spxextra{tornado.web.RequestHandler method}}

\begin{fulllineitems}
\phantomsection\label{\detokenize{web:tornado.web.RequestHandler.set_default_headers}}\pysiglinewithargsret{\sphinxcode{\sphinxupquote{RequestHandler.}}\sphinxbfcode{\sphinxupquote{set\_default\_headers}}}{}{{ $\rightarrow$ None}}
Override this to set HTTP headers at the beginning of the request.

For example, this is the place to set a custom \sphinxcode{\sphinxupquote{Server}} header.
Note that setting such headers in the normal flow of request
processing may not do what you want, since headers may be reset
during error handling.

\end{fulllineitems}

\index{write() (tornado.web.RequestHandler method)@\spxentry{write()}\spxextra{tornado.web.RequestHandler method}}

\begin{fulllineitems}
\phantomsection\label{\detokenize{web:tornado.web.RequestHandler.write}}\pysiglinewithargsret{\sphinxcode{\sphinxupquote{RequestHandler.}}\sphinxbfcode{\sphinxupquote{write}}}{\emph{chunk: Union{[}str, bytes, dict{]}}}{{ $\rightarrow$ None}}
Writes the given chunk to the output buffer.

To write the output to the network, use the {\hyperref[\detokenize{web:tornado.web.RequestHandler.flush}]{\sphinxcrossref{\sphinxcode{\sphinxupquote{flush()}}}}} method below.

If the given chunk is a dictionary, we write it as JSON and set
the Content-Type of the response to be \sphinxcode{\sphinxupquote{application/json}}.
(if you want to send JSON as a different \sphinxcode{\sphinxupquote{Content-Type}}, call
\sphinxcode{\sphinxupquote{set\_header}} \sphinxstyleemphasis{after} calling \sphinxcode{\sphinxupquote{write()}}).

Note that lists are not converted to JSON because of a potential
cross-site security vulnerability.  All JSON output should be
wrapped in a dictionary.  More details at
\sphinxurl{http://haacked.com/archive/2009/06/25/json-hijacking.aspx/} and
\sphinxurl{https://github.com/facebook/tornado/issues/1009}

\end{fulllineitems}

\index{flush() (tornado.web.RequestHandler method)@\spxentry{flush()}\spxextra{tornado.web.RequestHandler method}}

\begin{fulllineitems}
\phantomsection\label{\detokenize{web:tornado.web.RequestHandler.flush}}\pysiglinewithargsret{\sphinxcode{\sphinxupquote{RequestHandler.}}\sphinxbfcode{\sphinxupquote{flush}}}{\emph{include\_footers: bool = False}}{{ $\rightarrow$ Future{[}None{]}}}
Flushes the current output buffer to the network.

The \sphinxcode{\sphinxupquote{callback}} argument, if given, can be used for flow control:
it will be run when all flushed data has been written to the socket.
Note that only one flush callback can be outstanding at a time;
if another flush occurs before the previous flush’s callback
has been run, the previous callback will be discarded.

\DUrole{versionmodified,changed}{Changed in version 4.0: }Now returns a {\hyperref[\detokenize{concurrent:tornado.concurrent.Future}]{\sphinxcrossref{\sphinxcode{\sphinxupquote{Future}}}}} if no callback is given.

\DUrole{versionmodified,changed}{Changed in version 6.0: }The \sphinxcode{\sphinxupquote{callback}} argument was removed.

\end{fulllineitems}

\index{finish() (tornado.web.RequestHandler method)@\spxentry{finish()}\spxextra{tornado.web.RequestHandler method}}

\begin{fulllineitems}
\phantomsection\label{\detokenize{web:tornado.web.RequestHandler.finish}}\pysiglinewithargsret{\sphinxcode{\sphinxupquote{RequestHandler.}}\sphinxbfcode{\sphinxupquote{finish}}}{\emph{chunk: Union{[}str}, \emph{bytes}, \emph{dict{]} = None}}{{ $\rightarrow$ Future{[}None{]}}}
Finishes this response, ending the HTTP request.

Passing a \sphinxcode{\sphinxupquote{chunk}} to \sphinxcode{\sphinxupquote{finish()}} is equivalent to passing that
chunk to \sphinxcode{\sphinxupquote{write()}} and then calling \sphinxcode{\sphinxupquote{finish()}} with no arguments.

Returns a {\hyperref[\detokenize{concurrent:tornado.concurrent.Future}]{\sphinxcrossref{\sphinxcode{\sphinxupquote{Future}}}}} which may optionally be awaited to track the sending
of the response to the client. This {\hyperref[\detokenize{concurrent:tornado.concurrent.Future}]{\sphinxcrossref{\sphinxcode{\sphinxupquote{Future}}}}} resolves when all the response
data has been sent, and raises an error if the connection is closed before all
data can be sent.

\DUrole{versionmodified,changed}{Changed in version 5.1: }Now returns a {\hyperref[\detokenize{concurrent:tornado.concurrent.Future}]{\sphinxcrossref{\sphinxcode{\sphinxupquote{Future}}}}} instead of \sphinxcode{\sphinxupquote{None}}.

\end{fulllineitems}

\index{render() (tornado.web.RequestHandler method)@\spxentry{render()}\spxextra{tornado.web.RequestHandler method}}

\begin{fulllineitems}
\phantomsection\label{\detokenize{web:tornado.web.RequestHandler.render}}\pysiglinewithargsret{\sphinxcode{\sphinxupquote{RequestHandler.}}\sphinxbfcode{\sphinxupquote{render}}}{\emph{template\_name: str}, \emph{**kwargs}}{{ $\rightarrow$ Future{[}None{]}}}
Renders the template with the given arguments as the response.

\sphinxcode{\sphinxupquote{render()}} calls \sphinxcode{\sphinxupquote{finish()}}, so no other output methods can be called
after it.

Returns a {\hyperref[\detokenize{concurrent:tornado.concurrent.Future}]{\sphinxcrossref{\sphinxcode{\sphinxupquote{Future}}}}} with the same semantics as the one returned by {\hyperref[\detokenize{web:tornado.web.RequestHandler.finish}]{\sphinxcrossref{\sphinxcode{\sphinxupquote{finish}}}}}.
Awaiting this {\hyperref[\detokenize{concurrent:tornado.concurrent.Future}]{\sphinxcrossref{\sphinxcode{\sphinxupquote{Future}}}}} is optional.

\DUrole{versionmodified,changed}{Changed in version 5.1: }Now returns a {\hyperref[\detokenize{concurrent:tornado.concurrent.Future}]{\sphinxcrossref{\sphinxcode{\sphinxupquote{Future}}}}} instead of \sphinxcode{\sphinxupquote{None}}.

\end{fulllineitems}

\index{render\_string() (tornado.web.RequestHandler method)@\spxentry{render\_string()}\spxextra{tornado.web.RequestHandler method}}

\begin{fulllineitems}
\phantomsection\label{\detokenize{web:tornado.web.RequestHandler.render_string}}\pysiglinewithargsret{\sphinxcode{\sphinxupquote{RequestHandler.}}\sphinxbfcode{\sphinxupquote{render\_string}}}{\emph{template\_name: str}, \emph{**kwargs}}{{ $\rightarrow$ bytes}}
Generate the given template with the given arguments.

We return the generated byte string (in utf8). To generate and
write a template as a response, use render() above.

\end{fulllineitems}

\index{get\_template\_namespace() (tornado.web.RequestHandler method)@\spxentry{get\_template\_namespace()}\spxextra{tornado.web.RequestHandler method}}

\begin{fulllineitems}
\phantomsection\label{\detokenize{web:tornado.web.RequestHandler.get_template_namespace}}\pysiglinewithargsret{\sphinxcode{\sphinxupquote{RequestHandler.}}\sphinxbfcode{\sphinxupquote{get\_template\_namespace}}}{}{{ $\rightarrow$ Dict{[}str, Any{]}}}
Returns a dictionary to be used as the default template namespace.

May be overridden by subclasses to add or modify values.

The results of this method will be combined with additional
defaults in the {\hyperref[\detokenize{template:module-tornado.template}]{\sphinxcrossref{\sphinxcode{\sphinxupquote{tornado.template}}}}} module and keyword arguments
to {\hyperref[\detokenize{web:tornado.web.RequestHandler.render}]{\sphinxcrossref{\sphinxcode{\sphinxupquote{render}}}}} or {\hyperref[\detokenize{web:tornado.web.RequestHandler.render_string}]{\sphinxcrossref{\sphinxcode{\sphinxupquote{render\_string}}}}}.

\end{fulllineitems}

\index{redirect() (tornado.web.RequestHandler method)@\spxentry{redirect()}\spxextra{tornado.web.RequestHandler method}}

\begin{fulllineitems}
\phantomsection\label{\detokenize{web:tornado.web.RequestHandler.redirect}}\pysiglinewithargsret{\sphinxcode{\sphinxupquote{RequestHandler.}}\sphinxbfcode{\sphinxupquote{redirect}}}{\emph{url: str}, \emph{permanent: bool = False}, \emph{status: int = None}}{{ $\rightarrow$ None}}
Sends a redirect to the given (optionally relative) URL.

If the \sphinxcode{\sphinxupquote{status}} argument is specified, that value is used as the
HTTP status code; otherwise either 301 (permanent) or 302
(temporary) is chosen based on the \sphinxcode{\sphinxupquote{permanent}} argument.
The default is 302 (temporary).

\end{fulllineitems}

\index{send\_error() (tornado.web.RequestHandler method)@\spxentry{send\_error()}\spxextra{tornado.web.RequestHandler method}}

\begin{fulllineitems}
\phantomsection\label{\detokenize{web:tornado.web.RequestHandler.send_error}}\pysiglinewithargsret{\sphinxcode{\sphinxupquote{RequestHandler.}}\sphinxbfcode{\sphinxupquote{send\_error}}}{\emph{status\_code: int = 500}, \emph{**kwargs}}{{ $\rightarrow$ None}}
Sends the given HTTP error code to the browser.

If {\hyperref[\detokenize{web:tornado.web.RequestHandler.flush}]{\sphinxcrossref{\sphinxcode{\sphinxupquote{flush()}}}}} has already been called, it is not possible to send
an error, so this method will simply terminate the response.
If output has been written but not yet flushed, it will be discarded
and replaced with the error page.

Override {\hyperref[\detokenize{web:tornado.web.RequestHandler.write_error}]{\sphinxcrossref{\sphinxcode{\sphinxupquote{write\_error()}}}}} to customize the error page that is returned.
Additional keyword arguments are passed through to {\hyperref[\detokenize{web:tornado.web.RequestHandler.write_error}]{\sphinxcrossref{\sphinxcode{\sphinxupquote{write\_error}}}}}.

\end{fulllineitems}

\index{write\_error() (tornado.web.RequestHandler method)@\spxentry{write\_error()}\spxextra{tornado.web.RequestHandler method}}

\begin{fulllineitems}
\phantomsection\label{\detokenize{web:tornado.web.RequestHandler.write_error}}\pysiglinewithargsret{\sphinxcode{\sphinxupquote{RequestHandler.}}\sphinxbfcode{\sphinxupquote{write\_error}}}{\emph{status\_code: int}, \emph{**kwargs}}{{ $\rightarrow$ None}}
Override to implement custom error pages.

\sphinxcode{\sphinxupquote{write\_error}} may call {\hyperref[\detokenize{web:tornado.web.RequestHandler.write}]{\sphinxcrossref{\sphinxcode{\sphinxupquote{write}}}}}, {\hyperref[\detokenize{web:tornado.web.RequestHandler.render}]{\sphinxcrossref{\sphinxcode{\sphinxupquote{render}}}}}, {\hyperref[\detokenize{web:tornado.web.RequestHandler.set_header}]{\sphinxcrossref{\sphinxcode{\sphinxupquote{set\_header}}}}}, etc
to produce output as usual.

If this error was caused by an uncaught exception (including
HTTPError), an \sphinxcode{\sphinxupquote{exc\_info}} triple will be available as
\sphinxcode{\sphinxupquote{kwargs{[}"exc\_info"{]}}}.  Note that this exception may not be
the “current” exception for purposes of methods like
\sphinxcode{\sphinxupquote{sys.exc\_info()}} or \sphinxcode{\sphinxupquote{traceback.format\_exc}}.

\end{fulllineitems}

\index{clear() (tornado.web.RequestHandler method)@\spxentry{clear()}\spxextra{tornado.web.RequestHandler method}}

\begin{fulllineitems}
\phantomsection\label{\detokenize{web:tornado.web.RequestHandler.clear}}\pysiglinewithargsret{\sphinxcode{\sphinxupquote{RequestHandler.}}\sphinxbfcode{\sphinxupquote{clear}}}{}{{ $\rightarrow$ None}}
Resets all headers and content for this response.

\end{fulllineitems}

\index{render\_linked\_js() (tornado.web.RequestHandler method)@\spxentry{render\_linked\_js()}\spxextra{tornado.web.RequestHandler method}}

\begin{fulllineitems}
\phantomsection\label{\detokenize{web:tornado.web.RequestHandler.render_linked_js}}\pysiglinewithargsret{\sphinxcode{\sphinxupquote{RequestHandler.}}\sphinxbfcode{\sphinxupquote{render\_linked\_js}}}{\emph{js\_files: Iterable{[}str{]}}}{{ $\rightarrow$ str}}
Default method used to render the final js links for the
rendered webpage.

Override this method in a sub-classed controller to change the output.

\end{fulllineitems}

\index{render\_embed\_js() (tornado.web.RequestHandler method)@\spxentry{render\_embed\_js()}\spxextra{tornado.web.RequestHandler method}}

\begin{fulllineitems}
\phantomsection\label{\detokenize{web:tornado.web.RequestHandler.render_embed_js}}\pysiglinewithargsret{\sphinxcode{\sphinxupquote{RequestHandler.}}\sphinxbfcode{\sphinxupquote{render\_embed\_js}}}{\emph{js\_embed: Iterable{[}bytes{]}}}{{ $\rightarrow$ bytes}}
Default method used to render the final embedded js for the
rendered webpage.

Override this method in a sub-classed controller to change the output.

\end{fulllineitems}

\index{render\_linked\_css() (tornado.web.RequestHandler method)@\spxentry{render\_linked\_css()}\spxextra{tornado.web.RequestHandler method}}

\begin{fulllineitems}
\phantomsection\label{\detokenize{web:tornado.web.RequestHandler.render_linked_css}}\pysiglinewithargsret{\sphinxcode{\sphinxupquote{RequestHandler.}}\sphinxbfcode{\sphinxupquote{render\_linked\_css}}}{\emph{css\_files: Iterable{[}str{]}}}{{ $\rightarrow$ str}}
Default method used to render the final css links for the
rendered webpage.

Override this method in a sub-classed controller to change the output.

\end{fulllineitems}

\index{render\_embed\_css() (tornado.web.RequestHandler method)@\spxentry{render\_embed\_css()}\spxextra{tornado.web.RequestHandler method}}

\begin{fulllineitems}
\phantomsection\label{\detokenize{web:tornado.web.RequestHandler.render_embed_css}}\pysiglinewithargsret{\sphinxcode{\sphinxupquote{RequestHandler.}}\sphinxbfcode{\sphinxupquote{render\_embed\_css}}}{\emph{css\_embed: Iterable{[}bytes{]}}}{{ $\rightarrow$ bytes}}
Default method used to render the final embedded css for the
rendered webpage.

Override this method in a sub-classed controller to change the output.

\end{fulllineitems}



\paragraph{Cookies}
\label{\detokenize{web:cookies}}\index{cookies (tornado.web.RequestHandler attribute)@\spxentry{cookies}\spxextra{tornado.web.RequestHandler attribute}}

\begin{fulllineitems}
\phantomsection\label{\detokenize{web:tornado.web.RequestHandler.cookies}}\pysigline{\sphinxcode{\sphinxupquote{RequestHandler.}}\sphinxbfcode{\sphinxupquote{cookies}}}
An alias for
{\hyperref[\detokenize{httputil:tornado.httputil.HTTPServerRequest.cookies}]{\sphinxcrossref{\sphinxcode{\sphinxupquote{self.request.cookies}}}}}.

\end{fulllineitems}

\index{get\_cookie() (tornado.web.RequestHandler method)@\spxentry{get\_cookie()}\spxextra{tornado.web.RequestHandler method}}

\begin{fulllineitems}
\phantomsection\label{\detokenize{web:tornado.web.RequestHandler.get_cookie}}\pysiglinewithargsret{\sphinxcode{\sphinxupquote{RequestHandler.}}\sphinxbfcode{\sphinxupquote{get\_cookie}}}{\emph{name: str}, \emph{default: str = None}}{{ $\rightarrow$ Optional{[}str{]}}}
Returns the value of the request cookie with the given name.

If the named cookie is not present, returns \sphinxcode{\sphinxupquote{default}}.

This method only returns cookies that were present in the request.
It does not see the outgoing cookies set by {\hyperref[\detokenize{web:tornado.web.RequestHandler.set_cookie}]{\sphinxcrossref{\sphinxcode{\sphinxupquote{set\_cookie}}}}} in this
handler.

\end{fulllineitems}

\index{set\_cookie() (tornado.web.RequestHandler method)@\spxentry{set\_cookie()}\spxextra{tornado.web.RequestHandler method}}

\begin{fulllineitems}
\phantomsection\label{\detokenize{web:tornado.web.RequestHandler.set_cookie}}\pysiglinewithargsret{\sphinxcode{\sphinxupquote{RequestHandler.}}\sphinxbfcode{\sphinxupquote{set\_cookie}}}{\emph{name: str, value: Union{[}str, bytes{]}, domain: str = None, expires: Union{[}float, Tuple, datetime.datetime{]} = None, path: str = '/', expires\_days: int = None, **kwargs}}{{ $\rightarrow$ None}}
Sets an outgoing cookie name/value with the given options.

Newly-set cookies are not immediately visible via {\hyperref[\detokenize{web:tornado.web.RequestHandler.get_cookie}]{\sphinxcrossref{\sphinxcode{\sphinxupquote{get\_cookie}}}}};
they are not present until the next request.

expires may be a numeric timestamp as returned by \sphinxhref{https://docs.python.org/3.6/library/time.html\#time.time}{\sphinxcode{\sphinxupquote{time.time}}},
a time tuple as returned by \sphinxhref{https://docs.python.org/3.6/library/time.html\#time.gmtime}{\sphinxcode{\sphinxupquote{time.gmtime}}}, or a
\sphinxhref{https://docs.python.org/3.6/library/datetime.html\#datetime.datetime}{\sphinxcode{\sphinxupquote{datetime.datetime}}} object.

Additional keyword arguments are set on the cookies.Morsel
directly.
See \sphinxurl{https://docs.python.org/3/library/http.cookies.html\#http.cookies.Morsel}
for available attributes.

\end{fulllineitems}

\index{clear\_cookie() (tornado.web.RequestHandler method)@\spxentry{clear\_cookie()}\spxextra{tornado.web.RequestHandler method}}

\begin{fulllineitems}
\phantomsection\label{\detokenize{web:tornado.web.RequestHandler.clear_cookie}}\pysiglinewithargsret{\sphinxcode{\sphinxupquote{RequestHandler.}}\sphinxbfcode{\sphinxupquote{clear\_cookie}}}{\emph{name: str}, \emph{path: str = '/'}, \emph{domain: str = None}}{{ $\rightarrow$ None}}
Deletes the cookie with the given name.

Due to limitations of the cookie protocol, you must pass the same
path and domain to clear a cookie as were used when that cookie
was set (but there is no way to find out on the server side
which values were used for a given cookie).

Similar to {\hyperref[\detokenize{web:tornado.web.RequestHandler.set_cookie}]{\sphinxcrossref{\sphinxcode{\sphinxupquote{set\_cookie}}}}}, the effect of this method will not be
seen until the following request.

\end{fulllineitems}

\index{clear\_all\_cookies() (tornado.web.RequestHandler method)@\spxentry{clear\_all\_cookies()}\spxextra{tornado.web.RequestHandler method}}

\begin{fulllineitems}
\phantomsection\label{\detokenize{web:tornado.web.RequestHandler.clear_all_cookies}}\pysiglinewithargsret{\sphinxcode{\sphinxupquote{RequestHandler.}}\sphinxbfcode{\sphinxupquote{clear\_all\_cookies}}}{\emph{path: str = '/'}, \emph{domain: str = None}}{{ $\rightarrow$ None}}
Deletes all the cookies the user sent with this request.

See {\hyperref[\detokenize{web:tornado.web.RequestHandler.clear_cookie}]{\sphinxcrossref{\sphinxcode{\sphinxupquote{clear\_cookie}}}}} for more information on the path and domain
parameters.

Similar to {\hyperref[\detokenize{web:tornado.web.RequestHandler.set_cookie}]{\sphinxcrossref{\sphinxcode{\sphinxupquote{set\_cookie}}}}}, the effect of this method will not be
seen until the following request.

\DUrole{versionmodified,changed}{Changed in version 3.2: }Added the \sphinxcode{\sphinxupquote{path}} and \sphinxcode{\sphinxupquote{domain}} parameters.

\end{fulllineitems}

\index{get\_secure\_cookie() (tornado.web.RequestHandler method)@\spxentry{get\_secure\_cookie()}\spxextra{tornado.web.RequestHandler method}}

\begin{fulllineitems}
\phantomsection\label{\detokenize{web:tornado.web.RequestHandler.get_secure_cookie}}\pysiglinewithargsret{\sphinxcode{\sphinxupquote{RequestHandler.}}\sphinxbfcode{\sphinxupquote{get\_secure\_cookie}}}{\emph{name: str}, \emph{value: str = None}, \emph{max\_age\_days: int = 31}, \emph{min\_version: int = None}}{{ $\rightarrow$ Optional{[}bytes{]}}}
Returns the given signed cookie if it validates, or None.

The decoded cookie value is returned as a byte string (unlike
{\hyperref[\detokenize{web:tornado.web.RequestHandler.get_cookie}]{\sphinxcrossref{\sphinxcode{\sphinxupquote{get\_cookie}}}}}).

Similar to {\hyperref[\detokenize{web:tornado.web.RequestHandler.get_cookie}]{\sphinxcrossref{\sphinxcode{\sphinxupquote{get\_cookie}}}}}, this method only returns cookies that
were present in the request. It does not see outgoing cookies set by
{\hyperref[\detokenize{web:tornado.web.RequestHandler.set_secure_cookie}]{\sphinxcrossref{\sphinxcode{\sphinxupquote{set\_secure\_cookie}}}}} in this handler.

\DUrole{versionmodified,changed}{Changed in version 3.2.1: }Added the \sphinxcode{\sphinxupquote{min\_version}} argument.  Introduced cookie version 2;
both versions 1 and 2 are accepted by default.

\end{fulllineitems}

\index{get\_secure\_cookie\_key\_version() (tornado.web.RequestHandler method)@\spxentry{get\_secure\_cookie\_key\_version()}\spxextra{tornado.web.RequestHandler method}}

\begin{fulllineitems}
\phantomsection\label{\detokenize{web:tornado.web.RequestHandler.get_secure_cookie_key_version}}\pysiglinewithargsret{\sphinxcode{\sphinxupquote{RequestHandler.}}\sphinxbfcode{\sphinxupquote{get\_secure\_cookie\_key\_version}}}{\emph{name: str}, \emph{value: str = None}}{{ $\rightarrow$ Optional{[}int{]}}}
Returns the signing key version of the secure cookie.

The version is returned as int.

\end{fulllineitems}

\index{set\_secure\_cookie() (tornado.web.RequestHandler method)@\spxentry{set\_secure\_cookie()}\spxextra{tornado.web.RequestHandler method}}

\begin{fulllineitems}
\phantomsection\label{\detokenize{web:tornado.web.RequestHandler.set_secure_cookie}}\pysiglinewithargsret{\sphinxcode{\sphinxupquote{RequestHandler.}}\sphinxbfcode{\sphinxupquote{set\_secure\_cookie}}}{\emph{name: str, value: Union{[}str, bytes{]}, expires\_days: int = 30, version: int = None, **kwargs}}{{ $\rightarrow$ None}}
Signs and timestamps a cookie so it cannot be forged.

You must specify the \sphinxcode{\sphinxupquote{cookie\_secret}} setting in your Application
to use this method. It should be a long, random sequence of bytes
to be used as the HMAC secret for the signature.

To read a cookie set with this method, use {\hyperref[\detokenize{web:tornado.web.RequestHandler.get_secure_cookie}]{\sphinxcrossref{\sphinxcode{\sphinxupquote{get\_secure\_cookie()}}}}}.

Note that the \sphinxcode{\sphinxupquote{expires\_days}} parameter sets the lifetime of the
cookie in the browser, but is independent of the \sphinxcode{\sphinxupquote{max\_age\_days}}
parameter to {\hyperref[\detokenize{web:tornado.web.RequestHandler.get_secure_cookie}]{\sphinxcrossref{\sphinxcode{\sphinxupquote{get\_secure\_cookie}}}}}.

Secure cookies may contain arbitrary byte values, not just unicode
strings (unlike regular cookies)

Similar to {\hyperref[\detokenize{web:tornado.web.RequestHandler.set_cookie}]{\sphinxcrossref{\sphinxcode{\sphinxupquote{set\_cookie}}}}}, the effect of this method will not be
seen until the following request.

\DUrole{versionmodified,changed}{Changed in version 3.2.1: }Added the \sphinxcode{\sphinxupquote{version}} argument.  Introduced cookie version 2
and made it the default.

\end{fulllineitems}

\index{create\_signed\_value() (tornado.web.RequestHandler method)@\spxentry{create\_signed\_value()}\spxextra{tornado.web.RequestHandler method}}

\begin{fulllineitems}
\phantomsection\label{\detokenize{web:tornado.web.RequestHandler.create_signed_value}}\pysiglinewithargsret{\sphinxcode{\sphinxupquote{RequestHandler.}}\sphinxbfcode{\sphinxupquote{create\_signed\_value}}}{\emph{name: str, value: Union{[}str, bytes{]}, version: int = None}}{{ $\rightarrow$ bytes}}
Signs and timestamps a string so it cannot be forged.

Normally used via set\_secure\_cookie, but provided as a separate
method for non-cookie uses.  To decode a value not stored
as a cookie use the optional value argument to get\_secure\_cookie.

\DUrole{versionmodified,changed}{Changed in version 3.2.1: }Added the \sphinxcode{\sphinxupquote{version}} argument.  Introduced cookie version 2
and made it the default.

\end{fulllineitems}

\index{MIN\_SUPPORTED\_SIGNED\_VALUE\_VERSION (in module tornado.web)@\spxentry{MIN\_SUPPORTED\_SIGNED\_VALUE\_VERSION}\spxextra{in module tornado.web}}

\begin{fulllineitems}
\phantomsection\label{\detokenize{web:tornado.web.MIN_SUPPORTED_SIGNED_VALUE_VERSION}}\pysigline{\sphinxcode{\sphinxupquote{tornado.web.}}\sphinxbfcode{\sphinxupquote{MIN\_SUPPORTED\_SIGNED\_VALUE\_VERSION}}\sphinxbfcode{\sphinxupquote{ = 1}}}
The oldest signed value version supported by this version of Tornado.

Signed values older than this version cannot be decoded.

\DUrole{versionmodified,added}{New in version 3.2.1.}

\end{fulllineitems}

\index{MAX\_SUPPORTED\_SIGNED\_VALUE\_VERSION (in module tornado.web)@\spxentry{MAX\_SUPPORTED\_SIGNED\_VALUE\_VERSION}\spxextra{in module tornado.web}}

\begin{fulllineitems}
\phantomsection\label{\detokenize{web:tornado.web.MAX_SUPPORTED_SIGNED_VALUE_VERSION}}\pysigline{\sphinxcode{\sphinxupquote{tornado.web.}}\sphinxbfcode{\sphinxupquote{MAX\_SUPPORTED\_SIGNED\_VALUE\_VERSION}}\sphinxbfcode{\sphinxupquote{ = 2}}}
The newest signed value version supported by this version of Tornado.

Signed values newer than this version cannot be decoded.

\DUrole{versionmodified,added}{New in version 3.2.1.}

\end{fulllineitems}

\index{DEFAULT\_SIGNED\_VALUE\_VERSION (in module tornado.web)@\spxentry{DEFAULT\_SIGNED\_VALUE\_VERSION}\spxextra{in module tornado.web}}

\begin{fulllineitems}
\phantomsection\label{\detokenize{web:tornado.web.DEFAULT_SIGNED_VALUE_VERSION}}\pysigline{\sphinxcode{\sphinxupquote{tornado.web.}}\sphinxbfcode{\sphinxupquote{DEFAULT\_SIGNED\_VALUE\_VERSION}}\sphinxbfcode{\sphinxupquote{ = 2}}}
The signed value version produced by {\hyperref[\detokenize{web:tornado.web.RequestHandler.create_signed_value}]{\sphinxcrossref{\sphinxcode{\sphinxupquote{RequestHandler.create\_signed\_value}}}}}.

May be overridden by passing a \sphinxcode{\sphinxupquote{version}} keyword argument.

\DUrole{versionmodified,added}{New in version 3.2.1.}

\end{fulllineitems}

\index{DEFAULT\_SIGNED\_VALUE\_MIN\_VERSION (in module tornado.web)@\spxentry{DEFAULT\_SIGNED\_VALUE\_MIN\_VERSION}\spxextra{in module tornado.web}}

\begin{fulllineitems}
\phantomsection\label{\detokenize{web:tornado.web.DEFAULT_SIGNED_VALUE_MIN_VERSION}}\pysigline{\sphinxcode{\sphinxupquote{tornado.web.}}\sphinxbfcode{\sphinxupquote{DEFAULT\_SIGNED\_VALUE\_MIN\_VERSION}}\sphinxbfcode{\sphinxupquote{ = 1}}}
The oldest signed value accepted by {\hyperref[\detokenize{web:tornado.web.RequestHandler.get_secure_cookie}]{\sphinxcrossref{\sphinxcode{\sphinxupquote{RequestHandler.get\_secure\_cookie}}}}}.

May be overridden by passing a \sphinxcode{\sphinxupquote{min\_version}} keyword argument.

\DUrole{versionmodified,added}{New in version 3.2.1.}

\end{fulllineitems}



\paragraph{Other}
\label{\detokenize{web:other}}\index{application (tornado.web.RequestHandler attribute)@\spxentry{application}\spxextra{tornado.web.RequestHandler attribute}}

\begin{fulllineitems}
\phantomsection\label{\detokenize{web:tornado.web.RequestHandler.application}}\pysigline{\sphinxcode{\sphinxupquote{RequestHandler.}}\sphinxbfcode{\sphinxupquote{application}}}
The {\hyperref[\detokenize{web:tornado.web.Application}]{\sphinxcrossref{\sphinxcode{\sphinxupquote{Application}}}}} object serving this request

\end{fulllineitems}

\index{check\_etag\_header() (tornado.web.RequestHandler method)@\spxentry{check\_etag\_header()}\spxextra{tornado.web.RequestHandler method}}

\begin{fulllineitems}
\phantomsection\label{\detokenize{web:tornado.web.RequestHandler.check_etag_header}}\pysiglinewithargsret{\sphinxcode{\sphinxupquote{RequestHandler.}}\sphinxbfcode{\sphinxupquote{check\_etag\_header}}}{}{{ $\rightarrow$ bool}}
Checks the \sphinxcode{\sphinxupquote{Etag}} header against requests’s \sphinxcode{\sphinxupquote{If-None-Match}}.

Returns \sphinxcode{\sphinxupquote{True}} if the request’s Etag matches and a 304 should be
returned. For example:

\begin{sphinxVerbatim}[commandchars=\\\{\}]
\PYG{n+nb+bp}{self}\PYG{o}{.}\PYG{n}{set\PYGZus{}etag\PYGZus{}header}\PYG{p}{(}\PYG{p}{)}
\PYG{k}{if} \PYG{n+nb+bp}{self}\PYG{o}{.}\PYG{n}{check\PYGZus{}etag\PYGZus{}header}\PYG{p}{(}\PYG{p}{)}\PYG{p}{:}
    \PYG{n+nb+bp}{self}\PYG{o}{.}\PYG{n}{set\PYGZus{}status}\PYG{p}{(}\PYG{l+m+mi}{304}\PYG{p}{)}
    \PYG{k}{return}
\end{sphinxVerbatim}

This method is called automatically when the request is finished,
but may be called earlier for applications that override
{\hyperref[\detokenize{web:tornado.web.RequestHandler.compute_etag}]{\sphinxcrossref{\sphinxcode{\sphinxupquote{compute\_etag}}}}} and want to do an early check for \sphinxcode{\sphinxupquote{If-None-Match}}
before completing the request.  The \sphinxcode{\sphinxupquote{Etag}} header should be set
(perhaps with {\hyperref[\detokenize{web:tornado.web.RequestHandler.set_etag_header}]{\sphinxcrossref{\sphinxcode{\sphinxupquote{set\_etag\_header}}}}}) before calling this method.

\end{fulllineitems}

\index{check\_xsrf\_cookie() (tornado.web.RequestHandler method)@\spxentry{check\_xsrf\_cookie()}\spxextra{tornado.web.RequestHandler method}}

\begin{fulllineitems}
\phantomsection\label{\detokenize{web:tornado.web.RequestHandler.check_xsrf_cookie}}\pysiglinewithargsret{\sphinxcode{\sphinxupquote{RequestHandler.}}\sphinxbfcode{\sphinxupquote{check\_xsrf\_cookie}}}{}{{ $\rightarrow$ None}}
Verifies that the \sphinxcode{\sphinxupquote{\_xsrf}} cookie matches the \sphinxcode{\sphinxupquote{\_xsrf}} argument.

To prevent cross-site request forgery, we set an \sphinxcode{\sphinxupquote{\_xsrf}}
cookie and include the same value as a non-cookie
field with all \sphinxcode{\sphinxupquote{POST}} requests. If the two do not match, we
reject the form submission as a potential forgery.

The \sphinxcode{\sphinxupquote{\_xsrf}} value may be set as either a form field named \sphinxcode{\sphinxupquote{\_xsrf}}
or in a custom HTTP header named \sphinxcode{\sphinxupquote{X-XSRFToken}} or \sphinxcode{\sphinxupquote{X-CSRFToken}}
(the latter is accepted for compatibility with Django).

See \sphinxurl{http://en.wikipedia.org/wiki/Cross-site\_request\_forgery}

\DUrole{versionmodified,changed}{Changed in version 3.2.2: }Added support for cookie version 2.  Both versions 1 and 2 are
supported.

\end{fulllineitems}

\index{compute\_etag() (tornado.web.RequestHandler method)@\spxentry{compute\_etag()}\spxextra{tornado.web.RequestHandler method}}

\begin{fulllineitems}
\phantomsection\label{\detokenize{web:tornado.web.RequestHandler.compute_etag}}\pysiglinewithargsret{\sphinxcode{\sphinxupquote{RequestHandler.}}\sphinxbfcode{\sphinxupquote{compute\_etag}}}{}{{ $\rightarrow$ Optional{[}str{]}}}
Computes the etag header to be used for this request.

By default uses a hash of the content written so far.

May be overridden to provide custom etag implementations,
or may return None to disable tornado’s default etag support.

\end{fulllineitems}

\index{create\_template\_loader() (tornado.web.RequestHandler method)@\spxentry{create\_template\_loader()}\spxextra{tornado.web.RequestHandler method}}

\begin{fulllineitems}
\phantomsection\label{\detokenize{web:tornado.web.RequestHandler.create_template_loader}}\pysiglinewithargsret{\sphinxcode{\sphinxupquote{RequestHandler.}}\sphinxbfcode{\sphinxupquote{create\_template\_loader}}}{\emph{template\_path: str}}{{ $\rightarrow$ tornado.template.BaseLoader}}
Returns a new template loader for the given path.

May be overridden by subclasses.  By default returns a
directory-based loader on the given path, using the
\sphinxcode{\sphinxupquote{autoescape}} and \sphinxcode{\sphinxupquote{template\_whitespace}} application
settings.  If a \sphinxcode{\sphinxupquote{template\_loader}} application setting is
supplied, uses that instead.

\end{fulllineitems}

\index{current\_user (tornado.web.RequestHandler attribute)@\spxentry{current\_user}\spxextra{tornado.web.RequestHandler attribute}}

\begin{fulllineitems}
\phantomsection\label{\detokenize{web:tornado.web.RequestHandler.current_user}}\pysigline{\sphinxcode{\sphinxupquote{RequestHandler.}}\sphinxbfcode{\sphinxupquote{current\_user}}}
The authenticated user for this request.

This is set in one of two ways:
\begin{itemize}
\item {} 
A subclass may override {\hyperref[\detokenize{web:tornado.web.RequestHandler.get_current_user}]{\sphinxcrossref{\sphinxcode{\sphinxupquote{get\_current\_user()}}}}}, which will be called
automatically the first time \sphinxcode{\sphinxupquote{self.current\_user}} is accessed.
{\hyperref[\detokenize{web:tornado.web.RequestHandler.get_current_user}]{\sphinxcrossref{\sphinxcode{\sphinxupquote{get\_current\_user()}}}}} will only be called once per request,
and is cached for future access:

\begin{sphinxVerbatim}[commandchars=\\\{\}]
\PYG{k}{def} \PYG{n+nf}{get\PYGZus{}current\PYGZus{}user}\PYG{p}{(}\PYG{n+nb+bp}{self}\PYG{p}{)}\PYG{p}{:}
    \PYG{n}{user\PYGZus{}cookie} \PYG{o}{=} \PYG{n+nb+bp}{self}\PYG{o}{.}\PYG{n}{get\PYGZus{}secure\PYGZus{}cookie}\PYG{p}{(}\PYG{l+s+s2}{\PYGZdq{}}\PYG{l+s+s2}{user}\PYG{l+s+s2}{\PYGZdq{}}\PYG{p}{)}
    \PYG{k}{if} \PYG{n}{user\PYGZus{}cookie}\PYG{p}{:}
        \PYG{k}{return} \PYG{n}{json}\PYG{o}{.}\PYG{n}{loads}\PYG{p}{(}\PYG{n}{user\PYGZus{}cookie}\PYG{p}{)}
    \PYG{k}{return} \PYG{k+kc}{None}
\end{sphinxVerbatim}

\item {} 
It may be set as a normal variable, typically from an overridden
{\hyperref[\detokenize{web:tornado.web.RequestHandler.prepare}]{\sphinxcrossref{\sphinxcode{\sphinxupquote{prepare()}}}}}:

\begin{sphinxVerbatim}[commandchars=\\\{\}]
\PYG{n+nd}{@gen}\PYG{o}{.}\PYG{n}{coroutine}
\PYG{k}{def} \PYG{n+nf}{prepare}\PYG{p}{(}\PYG{n+nb+bp}{self}\PYG{p}{)}\PYG{p}{:}
    \PYG{n}{user\PYGZus{}id\PYGZus{}cookie} \PYG{o}{=} \PYG{n+nb+bp}{self}\PYG{o}{.}\PYG{n}{get\PYGZus{}secure\PYGZus{}cookie}\PYG{p}{(}\PYG{l+s+s2}{\PYGZdq{}}\PYG{l+s+s2}{user\PYGZus{}id}\PYG{l+s+s2}{\PYGZdq{}}\PYG{p}{)}
    \PYG{k}{if} \PYG{n}{user\PYGZus{}id\PYGZus{}cookie}\PYG{p}{:}
        \PYG{n+nb+bp}{self}\PYG{o}{.}\PYG{n}{current\PYGZus{}user} \PYG{o}{=} \PYG{k}{yield} \PYG{n}{load\PYGZus{}user}\PYG{p}{(}\PYG{n}{user\PYGZus{}id\PYGZus{}cookie}\PYG{p}{)}
\end{sphinxVerbatim}

\end{itemize}

Note that {\hyperref[\detokenize{web:tornado.web.RequestHandler.prepare}]{\sphinxcrossref{\sphinxcode{\sphinxupquote{prepare()}}}}} may be a coroutine while {\hyperref[\detokenize{web:tornado.web.RequestHandler.get_current_user}]{\sphinxcrossref{\sphinxcode{\sphinxupquote{get\_current\_user()}}}}}
may not, so the latter form is necessary if loading the user requires
asynchronous operations.

The user object may be any type of the application’s choosing.

\end{fulllineitems}

\index{detach() (tornado.web.RequestHandler method)@\spxentry{detach()}\spxextra{tornado.web.RequestHandler method}}

\begin{fulllineitems}
\phantomsection\label{\detokenize{web:tornado.web.RequestHandler.detach}}\pysiglinewithargsret{\sphinxcode{\sphinxupquote{RequestHandler.}}\sphinxbfcode{\sphinxupquote{detach}}}{}{{ $\rightarrow$ tornado.iostream.IOStream}}
Take control of the underlying stream.

Returns the underlying {\hyperref[\detokenize{iostream:tornado.iostream.IOStream}]{\sphinxcrossref{\sphinxcode{\sphinxupquote{IOStream}}}}} object and stops all
further HTTP processing. Intended for implementing protocols
like websockets that tunnel over an HTTP handshake.

This method is only supported when HTTP/1.1 is used.

\DUrole{versionmodified,added}{New in version 5.1.}

\end{fulllineitems}

\index{get\_browser\_locale() (tornado.web.RequestHandler method)@\spxentry{get\_browser\_locale()}\spxextra{tornado.web.RequestHandler method}}

\begin{fulllineitems}
\phantomsection\label{\detokenize{web:tornado.web.RequestHandler.get_browser_locale}}\pysiglinewithargsret{\sphinxcode{\sphinxupquote{RequestHandler.}}\sphinxbfcode{\sphinxupquote{get\_browser\_locale}}}{\emph{default: str = 'en\_US'}}{{ $\rightarrow$ tornado.locale.Locale}}
Determines the user’s locale from \sphinxcode{\sphinxupquote{Accept-Language}} header.

See \sphinxurl{http://www.w3.org/Protocols/rfc2616/rfc2616-sec14.html\#sec14.4}

\end{fulllineitems}

\index{get\_current\_user() (tornado.web.RequestHandler method)@\spxentry{get\_current\_user()}\spxextra{tornado.web.RequestHandler method}}

\begin{fulllineitems}
\phantomsection\label{\detokenize{web:tornado.web.RequestHandler.get_current_user}}\pysiglinewithargsret{\sphinxcode{\sphinxupquote{RequestHandler.}}\sphinxbfcode{\sphinxupquote{get\_current\_user}}}{}{{ $\rightarrow$ Any}}
Override to determine the current user from, e.g., a cookie.

This method may not be a coroutine.

\end{fulllineitems}

\index{get\_login\_url() (tornado.web.RequestHandler method)@\spxentry{get\_login\_url()}\spxextra{tornado.web.RequestHandler method}}

\begin{fulllineitems}
\phantomsection\label{\detokenize{web:tornado.web.RequestHandler.get_login_url}}\pysiglinewithargsret{\sphinxcode{\sphinxupquote{RequestHandler.}}\sphinxbfcode{\sphinxupquote{get\_login\_url}}}{}{{ $\rightarrow$ str}}
Override to customize the login URL based on the request.

By default, we use the \sphinxcode{\sphinxupquote{login\_url}} application setting.

\end{fulllineitems}

\index{get\_status() (tornado.web.RequestHandler method)@\spxentry{get\_status()}\spxextra{tornado.web.RequestHandler method}}

\begin{fulllineitems}
\phantomsection\label{\detokenize{web:tornado.web.RequestHandler.get_status}}\pysiglinewithargsret{\sphinxcode{\sphinxupquote{RequestHandler.}}\sphinxbfcode{\sphinxupquote{get\_status}}}{}{{ $\rightarrow$ int}}
Returns the status code for our response.

\end{fulllineitems}

\index{get\_template\_path() (tornado.web.RequestHandler method)@\spxentry{get\_template\_path()}\spxextra{tornado.web.RequestHandler method}}

\begin{fulllineitems}
\phantomsection\label{\detokenize{web:tornado.web.RequestHandler.get_template_path}}\pysiglinewithargsret{\sphinxcode{\sphinxupquote{RequestHandler.}}\sphinxbfcode{\sphinxupquote{get\_template\_path}}}{}{{ $\rightarrow$ Optional{[}str{]}}}
Override to customize template path for each handler.

By default, we use the \sphinxcode{\sphinxupquote{template\_path}} application setting.
Return None to load templates relative to the calling file.

\end{fulllineitems}

\index{get\_user\_locale() (tornado.web.RequestHandler method)@\spxentry{get\_user\_locale()}\spxextra{tornado.web.RequestHandler method}}

\begin{fulllineitems}
\phantomsection\label{\detokenize{web:tornado.web.RequestHandler.get_user_locale}}\pysiglinewithargsret{\sphinxcode{\sphinxupquote{RequestHandler.}}\sphinxbfcode{\sphinxupquote{get\_user\_locale}}}{}{{ $\rightarrow$ Optional{[}tornado.locale.Locale{]}}}
Override to determine the locale from the authenticated user.

If None is returned, we fall back to {\hyperref[\detokenize{web:tornado.web.RequestHandler.get_browser_locale}]{\sphinxcrossref{\sphinxcode{\sphinxupquote{get\_browser\_locale()}}}}}.

This method should return a {\hyperref[\detokenize{locale:tornado.locale.Locale}]{\sphinxcrossref{\sphinxcode{\sphinxupquote{tornado.locale.Locale}}}}} object,
most likely obtained via a call like \sphinxcode{\sphinxupquote{tornado.locale.get("en")}}

\end{fulllineitems}

\index{locale (tornado.web.RequestHandler attribute)@\spxentry{locale}\spxextra{tornado.web.RequestHandler attribute}}

\begin{fulllineitems}
\phantomsection\label{\detokenize{web:tornado.web.RequestHandler.locale}}\pysigline{\sphinxcode{\sphinxupquote{RequestHandler.}}\sphinxbfcode{\sphinxupquote{locale}}}
The locale for the current session.

Determined by either {\hyperref[\detokenize{web:tornado.web.RequestHandler.get_user_locale}]{\sphinxcrossref{\sphinxcode{\sphinxupquote{get\_user\_locale}}}}}, which you can override to
set the locale based on, e.g., a user preference stored in a
database, or {\hyperref[\detokenize{web:tornado.web.RequestHandler.get_browser_locale}]{\sphinxcrossref{\sphinxcode{\sphinxupquote{get\_browser\_locale}}}}}, which uses the \sphinxcode{\sphinxupquote{Accept-Language}}
header.

\end{fulllineitems}

\index{log\_exception() (tornado.web.RequestHandler method)@\spxentry{log\_exception()}\spxextra{tornado.web.RequestHandler method}}

\begin{fulllineitems}
\phantomsection\label{\detokenize{web:tornado.web.RequestHandler.log_exception}}\pysiglinewithargsret{\sphinxcode{\sphinxupquote{RequestHandler.}}\sphinxbfcode{\sphinxupquote{log\_exception}}}{\emph{typ: Optional{[}Type{[}BaseException{]}{]}, value: Optional{[}BaseException{]}, tb: Optional{[}traceback{]}}}{{ $\rightarrow$ None}}
Override to customize logging of uncaught exceptions.

By default logs instances of {\hyperref[\detokenize{web:tornado.web.HTTPError}]{\sphinxcrossref{\sphinxcode{\sphinxupquote{HTTPError}}}}} as warnings without
stack traces (on the \sphinxcode{\sphinxupquote{tornado.general}} logger), and all
other exceptions as errors with stack traces (on the
\sphinxcode{\sphinxupquote{tornado.application}} logger).

\DUrole{versionmodified,added}{New in version 3.1.}

\end{fulllineitems}

\index{on\_connection\_close() (tornado.web.RequestHandler method)@\spxentry{on\_connection\_close()}\spxextra{tornado.web.RequestHandler method}}

\begin{fulllineitems}
\phantomsection\label{\detokenize{web:tornado.web.RequestHandler.on_connection_close}}\pysiglinewithargsret{\sphinxcode{\sphinxupquote{RequestHandler.}}\sphinxbfcode{\sphinxupquote{on\_connection\_close}}}{}{{ $\rightarrow$ None}}
Called in async handlers if the client closed the connection.

Override this to clean up resources associated with
long-lived connections.  Note that this method is called only if
the connection was closed during asynchronous processing; if you
need to do cleanup after every request override {\hyperref[\detokenize{web:tornado.web.RequestHandler.on_finish}]{\sphinxcrossref{\sphinxcode{\sphinxupquote{on\_finish}}}}}
instead.

Proxies may keep a connection open for a time (perhaps
indefinitely) after the client has gone away, so this method
may not be called promptly after the end user closes their
connection.

\end{fulllineitems}

\index{require\_setting() (tornado.web.RequestHandler method)@\spxentry{require\_setting()}\spxextra{tornado.web.RequestHandler method}}

\begin{fulllineitems}
\phantomsection\label{\detokenize{web:tornado.web.RequestHandler.require_setting}}\pysiglinewithargsret{\sphinxcode{\sphinxupquote{RequestHandler.}}\sphinxbfcode{\sphinxupquote{require\_setting}}}{\emph{name: str}, \emph{feature: str = 'this feature'}}{{ $\rightarrow$ None}}
Raises an exception if the given app setting is not defined.

\end{fulllineitems}

\index{reverse\_url() (tornado.web.RequestHandler method)@\spxentry{reverse\_url()}\spxextra{tornado.web.RequestHandler method}}

\begin{fulllineitems}
\phantomsection\label{\detokenize{web:tornado.web.RequestHandler.reverse_url}}\pysiglinewithargsret{\sphinxcode{\sphinxupquote{RequestHandler.}}\sphinxbfcode{\sphinxupquote{reverse\_url}}}{\emph{name: str}, \emph{*args}}{{ $\rightarrow$ str}}
Alias for {\hyperref[\detokenize{web:tornado.web.Application.reverse_url}]{\sphinxcrossref{\sphinxcode{\sphinxupquote{Application.reverse\_url}}}}}.

\end{fulllineitems}

\index{set\_etag\_header() (tornado.web.RequestHandler method)@\spxentry{set\_etag\_header()}\spxextra{tornado.web.RequestHandler method}}

\begin{fulllineitems}
\phantomsection\label{\detokenize{web:tornado.web.RequestHandler.set_etag_header}}\pysiglinewithargsret{\sphinxcode{\sphinxupquote{RequestHandler.}}\sphinxbfcode{\sphinxupquote{set\_etag\_header}}}{}{{ $\rightarrow$ None}}
Sets the response’s Etag header using \sphinxcode{\sphinxupquote{self.compute\_etag()}}.

Note: no header will be set if \sphinxcode{\sphinxupquote{compute\_etag()}} returns \sphinxcode{\sphinxupquote{None}}.

This method is called automatically when the request is finished.

\end{fulllineitems}

\index{settings (tornado.web.RequestHandler attribute)@\spxentry{settings}\spxextra{tornado.web.RequestHandler attribute}}

\begin{fulllineitems}
\phantomsection\label{\detokenize{web:tornado.web.RequestHandler.settings}}\pysigline{\sphinxcode{\sphinxupquote{RequestHandler.}}\sphinxbfcode{\sphinxupquote{settings}}}
An alias for {\hyperref[\detokenize{web:tornado.web.Application.settings}]{\sphinxcrossref{\sphinxcode{\sphinxupquote{self.application.settings}}}}}.

\end{fulllineitems}

\index{static\_url() (tornado.web.RequestHandler method)@\spxentry{static\_url()}\spxextra{tornado.web.RequestHandler method}}

\begin{fulllineitems}
\phantomsection\label{\detokenize{web:tornado.web.RequestHandler.static_url}}\pysiglinewithargsret{\sphinxcode{\sphinxupquote{RequestHandler.}}\sphinxbfcode{\sphinxupquote{static\_url}}}{\emph{path: str}, \emph{include\_host: bool = None}, \emph{**kwargs}}{{ $\rightarrow$ str}}
Returns a static URL for the given relative static file path.

This method requires you set the \sphinxcode{\sphinxupquote{static\_path}} setting in your
application (which specifies the root directory of your static
files).

This method returns a versioned url (by default appending
\sphinxcode{\sphinxupquote{?v=\textless{}signature\textgreater{}}}), which allows the static files to be
cached indefinitely.  This can be disabled by passing
\sphinxcode{\sphinxupquote{include\_version=False}} (in the default implementation;
other static file implementations are not required to support
this, but they may support other options).

By default this method returns URLs relative to the current
host, but if \sphinxcode{\sphinxupquote{include\_host}} is true the URL returned will be
absolute.  If this handler has an \sphinxcode{\sphinxupquote{include\_host}} attribute,
that value will be used as the default for all {\hyperref[\detokenize{web:tornado.web.RequestHandler.static_url}]{\sphinxcrossref{\sphinxcode{\sphinxupquote{static\_url}}}}}
calls that do not pass \sphinxcode{\sphinxupquote{include\_host}} as a keyword argument.

\end{fulllineitems}

\index{xsrf\_form\_html() (tornado.web.RequestHandler method)@\spxentry{xsrf\_form\_html()}\spxextra{tornado.web.RequestHandler method}}

\begin{fulllineitems}
\phantomsection\label{\detokenize{web:tornado.web.RequestHandler.xsrf_form_html}}\pysiglinewithargsret{\sphinxcode{\sphinxupquote{RequestHandler.}}\sphinxbfcode{\sphinxupquote{xsrf\_form\_html}}}{}{{ $\rightarrow$ str}}
An HTML \sphinxcode{\sphinxupquote{\textless{}input/\textgreater{}}} element to be included with all POST forms.

It defines the \sphinxcode{\sphinxupquote{\_xsrf}} input value, which we check on all POST
requests to prevent cross-site request forgery. If you have set
the \sphinxcode{\sphinxupquote{xsrf\_cookies}} application setting, you must include this
HTML within all of your HTML forms.

In a template, this method should be called with \sphinxcode{\sphinxupquote{\{\% module
xsrf\_form\_html() \%\}}}

See {\hyperref[\detokenize{web:tornado.web.RequestHandler.check_xsrf_cookie}]{\sphinxcrossref{\sphinxcode{\sphinxupquote{check\_xsrf\_cookie()}}}}} above for more information.

\end{fulllineitems}

\index{xsrf\_token (tornado.web.RequestHandler attribute)@\spxentry{xsrf\_token}\spxextra{tornado.web.RequestHandler attribute}}

\begin{fulllineitems}
\phantomsection\label{\detokenize{web:tornado.web.RequestHandler.xsrf_token}}\pysigline{\sphinxcode{\sphinxupquote{RequestHandler.}}\sphinxbfcode{\sphinxupquote{xsrf\_token}}}
The XSRF-prevention token for the current user/session.

To prevent cross-site request forgery, we set an ‘\_xsrf’ cookie
and include the same ‘\_xsrf’ value as an argument with all POST
requests. If the two do not match, we reject the form submission
as a potential forgery.

See \sphinxurl{http://en.wikipedia.org/wiki/Cross-site\_request\_forgery}

This property is of type \sphinxhref{https://docs.python.org/3.6/library/stdtypes.html\#bytes}{\sphinxcode{\sphinxupquote{bytes}}}, but it contains only ASCII
characters. If a character string is required, there is no
need to base64-encode it; just decode the byte string as
UTF-8.

\DUrole{versionmodified,changed}{Changed in version 3.2.2: }The xsrf token will now be have a random mask applied in every
request, which makes it safe to include the token in pages
that are compressed.  See \sphinxurl{http://breachattack.com} for more
information on the issue fixed by this change.  Old (version 1)
cookies will be converted to version 2 when this method is called
unless the \sphinxcode{\sphinxupquote{xsrf\_cookie\_version}} {\hyperref[\detokenize{web:tornado.web.Application}]{\sphinxcrossref{\sphinxcode{\sphinxupquote{Application}}}}} setting is
set to 1.

\DUrole{versionmodified,changed}{Changed in version 4.3: }The \sphinxcode{\sphinxupquote{xsrf\_cookie\_kwargs}} {\hyperref[\detokenize{web:tornado.web.Application}]{\sphinxcrossref{\sphinxcode{\sphinxupquote{Application}}}}} setting may be
used to supply additional cookie options (which will be
passed directly to {\hyperref[\detokenize{web:tornado.web.RequestHandler.set_cookie}]{\sphinxcrossref{\sphinxcode{\sphinxupquote{set\_cookie}}}}}). For example,
\sphinxcode{\sphinxupquote{xsrf\_cookie\_kwargs=dict(httponly=True, secure=True)}}
will set the \sphinxcode{\sphinxupquote{secure}} and \sphinxcode{\sphinxupquote{httponly}} flags on the
\sphinxcode{\sphinxupquote{\_xsrf}} cookie.

\end{fulllineitems}



\subsubsection{Application configuration}
\label{\detokenize{web:application-configuration}}\index{Application (class in tornado.web)@\spxentry{Application}\spxextra{class in tornado.web}}

\begin{fulllineitems}
\phantomsection\label{\detokenize{web:tornado.web.Application}}\pysiglinewithargsret{\sphinxbfcode{\sphinxupquote{class }}\sphinxcode{\sphinxupquote{tornado.web.}}\sphinxbfcode{\sphinxupquote{Application}}}{\emph{handlers: List{[}Union{[}Rule}, \emph{Tuple{]}{]} = None}, \emph{default\_host: str = None}, \emph{transforms: List{[}Type{[}OutputTransform{]}{]} = None}, \emph{**settings}}{}
A collection of request handlers that make up a web application.

Instances of this class are callable and can be passed directly to
HTTPServer to serve the application:

\begin{sphinxVerbatim}[commandchars=\\\{\}]
\PYG{n}{application} \PYG{o}{=} \PYG{n}{web}\PYG{o}{.}\PYG{n}{Application}\PYG{p}{(}\PYG{p}{[}
    \PYG{p}{(}\PYG{l+s+sa}{r}\PYG{l+s+s2}{\PYGZdq{}}\PYG{l+s+s2}{/}\PYG{l+s+s2}{\PYGZdq{}}\PYG{p}{,} \PYG{n}{MainPageHandler}\PYG{p}{)}\PYG{p}{,}
\PYG{p}{]}\PYG{p}{)}
\PYG{n}{http\PYGZus{}server} \PYG{o}{=} \PYG{n}{httpserver}\PYG{o}{.}\PYG{n}{HTTPServer}\PYG{p}{(}\PYG{n}{application}\PYG{p}{)}
\PYG{n}{http\PYGZus{}server}\PYG{o}{.}\PYG{n}{listen}\PYG{p}{(}\PYG{l+m+mi}{8080}\PYG{p}{)}
\PYG{n}{ioloop}\PYG{o}{.}\PYG{n}{IOLoop}\PYG{o}{.}\PYG{n}{current}\PYG{p}{(}\PYG{p}{)}\PYG{o}{.}\PYG{n}{start}\PYG{p}{(}\PYG{p}{)}
\end{sphinxVerbatim}

The constructor for this class takes in a list of {\hyperref[\detokenize{routing:tornado.routing.Rule}]{\sphinxcrossref{\sphinxcode{\sphinxupquote{Rule}}}}}
objects or tuples of values corresponding to the arguments of
{\hyperref[\detokenize{routing:tornado.routing.Rule}]{\sphinxcrossref{\sphinxcode{\sphinxupquote{Rule}}}}} constructor: \sphinxcode{\sphinxupquote{(matcher, target, {[}target\_kwargs{]}, {[}name{]})}},
the values in square brackets being optional. The default matcher is
{\hyperref[\detokenize{routing:tornado.routing.PathMatches}]{\sphinxcrossref{\sphinxcode{\sphinxupquote{PathMatches}}}}}, so \sphinxcode{\sphinxupquote{(regexp, target)}} tuples can also be used
instead of \sphinxcode{\sphinxupquote{(PathMatches(regexp), target)}}.

A common routing target is a {\hyperref[\detokenize{web:tornado.web.RequestHandler}]{\sphinxcrossref{\sphinxcode{\sphinxupquote{RequestHandler}}}}} subclass, but you can also
use lists of rules as a target, which create a nested routing configuration:

\begin{sphinxVerbatim}[commandchars=\\\{\}]
\PYG{n}{application} \PYG{o}{=} \PYG{n}{web}\PYG{o}{.}\PYG{n}{Application}\PYG{p}{(}\PYG{p}{[}
    \PYG{p}{(}\PYG{n}{HostMatches}\PYG{p}{(}\PYG{l+s+s2}{\PYGZdq{}}\PYG{l+s+s2}{example.com}\PYG{l+s+s2}{\PYGZdq{}}\PYG{p}{)}\PYG{p}{,} \PYG{p}{[}
        \PYG{p}{(}\PYG{l+s+sa}{r}\PYG{l+s+s2}{\PYGZdq{}}\PYG{l+s+s2}{/}\PYG{l+s+s2}{\PYGZdq{}}\PYG{p}{,} \PYG{n}{MainPageHandler}\PYG{p}{)}\PYG{p}{,}
        \PYG{p}{(}\PYG{l+s+sa}{r}\PYG{l+s+s2}{\PYGZdq{}}\PYG{l+s+s2}{/feed}\PYG{l+s+s2}{\PYGZdq{}}\PYG{p}{,} \PYG{n}{FeedHandler}\PYG{p}{)}\PYG{p}{,}
    \PYG{p}{]}\PYG{p}{)}\PYG{p}{,}
\PYG{p}{]}\PYG{p}{)}
\end{sphinxVerbatim}

In addition to this you can use nested {\hyperref[\detokenize{routing:tornado.routing.Router}]{\sphinxcrossref{\sphinxcode{\sphinxupquote{Router}}}}} instances,
{\hyperref[\detokenize{httputil:tornado.httputil.HTTPMessageDelegate}]{\sphinxcrossref{\sphinxcode{\sphinxupquote{HTTPMessageDelegate}}}}} subclasses and callables as routing targets
(see {\hyperref[\detokenize{routing:module-tornado.routing}]{\sphinxcrossref{\sphinxcode{\sphinxupquote{routing}}}}} module docs for more information).

When we receive requests, we iterate over the list in order and
instantiate an instance of the first request class whose regexp
matches the request path. The request class can be specified as
either a class object or a (fully-qualified) name.

A dictionary may be passed as the third element (\sphinxcode{\sphinxupquote{target\_kwargs}})
of the tuple, which will be used as keyword arguments to the handler’s
constructor and {\hyperref[\detokenize{web:tornado.web.RequestHandler.initialize}]{\sphinxcrossref{\sphinxcode{\sphinxupquote{initialize}}}}} method. This pattern
is used for the {\hyperref[\detokenize{web:tornado.web.StaticFileHandler}]{\sphinxcrossref{\sphinxcode{\sphinxupquote{StaticFileHandler}}}}} in this example (note that a
{\hyperref[\detokenize{web:tornado.web.StaticFileHandler}]{\sphinxcrossref{\sphinxcode{\sphinxupquote{StaticFileHandler}}}}} can be installed automatically with the
static\_path setting described below):

\begin{sphinxVerbatim}[commandchars=\\\{\}]
\PYG{n}{application} \PYG{o}{=} \PYG{n}{web}\PYG{o}{.}\PYG{n}{Application}\PYG{p}{(}\PYG{p}{[}
    \PYG{p}{(}\PYG{l+s+sa}{r}\PYG{l+s+s2}{\PYGZdq{}}\PYG{l+s+s2}{/static/(.*)}\PYG{l+s+s2}{\PYGZdq{}}\PYG{p}{,} \PYG{n}{web}\PYG{o}{.}\PYG{n}{StaticFileHandler}\PYG{p}{,} \PYG{p}{\PYGZob{}}\PYG{l+s+s2}{\PYGZdq{}}\PYG{l+s+s2}{path}\PYG{l+s+s2}{\PYGZdq{}}\PYG{p}{:} \PYG{l+s+s2}{\PYGZdq{}}\PYG{l+s+s2}{/var/www}\PYG{l+s+s2}{\PYGZdq{}}\PYG{p}{\PYGZcb{}}\PYG{p}{)}\PYG{p}{,}
\PYG{p}{]}\PYG{p}{)}
\end{sphinxVerbatim}

We support virtual hosts with the {\hyperref[\detokenize{web:tornado.web.Application.add_handlers}]{\sphinxcrossref{\sphinxcode{\sphinxupquote{add\_handlers}}}}} method, which takes in
a host regular expression as the first argument:

\begin{sphinxVerbatim}[commandchars=\\\{\}]
\PYG{n}{application}\PYG{o}{.}\PYG{n}{add\PYGZus{}handlers}\PYG{p}{(}\PYG{l+s+sa}{r}\PYG{l+s+s2}{\PYGZdq{}}\PYG{l+s+s2}{www}\PYG{l+s+s2}{\PYGZbs{}}\PYG{l+s+s2}{.myhost}\PYG{l+s+s2}{\PYGZbs{}}\PYG{l+s+s2}{.com}\PYG{l+s+s2}{\PYGZdq{}}\PYG{p}{,} \PYG{p}{[}
    \PYG{p}{(}\PYG{l+s+sa}{r}\PYG{l+s+s2}{\PYGZdq{}}\PYG{l+s+s2}{/article/([0\PYGZhy{}9]+)}\PYG{l+s+s2}{\PYGZdq{}}\PYG{p}{,} \PYG{n}{ArticleHandler}\PYG{p}{)}\PYG{p}{,}
\PYG{p}{]}\PYG{p}{)}
\end{sphinxVerbatim}

If there’s no match for the current request’s host, then \sphinxcode{\sphinxupquote{default\_host}}
parameter value is matched against host regular expressions.

\begin{sphinxadmonition}{warning}{Warning:}
Applications that do not use TLS may be vulnerable to {\hyperref[\detokenize{guide/security:dnsrebinding}]{\sphinxcrossref{\DUrole{std,std-ref}{DNS
rebinding}}}} attacks. This attack is especially
relevant to applications that only listen on \sphinxcode{\sphinxupquote{127.0.0.1}} or
other private networks. Appropriate host patterns must be used
(instead of the default of \sphinxcode{\sphinxupquote{r'.*'}}) to prevent this risk. The
\sphinxcode{\sphinxupquote{default\_host}} argument must not be used in applications that
may be vulnerable to DNS rebinding.
\end{sphinxadmonition}

You can serve static files by sending the \sphinxcode{\sphinxupquote{static\_path}} setting
as a keyword argument. We will serve those files from the
\sphinxcode{\sphinxupquote{/static/}} URI (this is configurable with the
\sphinxcode{\sphinxupquote{static\_url\_prefix}} setting), and we will serve \sphinxcode{\sphinxupquote{/favicon.ico}}
and \sphinxcode{\sphinxupquote{/robots.txt}} from the same directory.  A custom subclass of
{\hyperref[\detokenize{web:tornado.web.StaticFileHandler}]{\sphinxcrossref{\sphinxcode{\sphinxupquote{StaticFileHandler}}}}} can be specified with the
\sphinxcode{\sphinxupquote{static\_handler\_class}} setting.

\DUrole{versionmodified,changed}{Changed in version 4.5: }Integration with the new {\hyperref[\detokenize{routing:module-tornado.routing}]{\sphinxcrossref{\sphinxcode{\sphinxupquote{tornado.routing}}}}} module.
\index{settings (tornado.web.Application attribute)@\spxentry{settings}\spxextra{tornado.web.Application attribute}}

\begin{fulllineitems}
\phantomsection\label{\detokenize{web:tornado.web.Application.settings}}\pysigline{\sphinxbfcode{\sphinxupquote{settings}}}
Additional keyword arguments passed to the constructor are
saved in the {\hyperref[\detokenize{web:tornado.web.Application.settings}]{\sphinxcrossref{\sphinxcode{\sphinxupquote{settings}}}}} dictionary, and are often referred to
in documentation as “application settings”.  Settings are
used to customize various aspects of Tornado (although in
some cases richer customization is possible by overriding
methods in a subclass of {\hyperref[\detokenize{web:tornado.web.RequestHandler}]{\sphinxcrossref{\sphinxcode{\sphinxupquote{RequestHandler}}}}}).  Some
applications also like to use the {\hyperref[\detokenize{web:tornado.web.Application.settings}]{\sphinxcrossref{\sphinxcode{\sphinxupquote{settings}}}}} dictionary as a
way to make application-specific settings available to
handlers without using global variables.  Settings used in
Tornado are described below.

General settings:
\begin{itemize}
\item {} 
\sphinxcode{\sphinxupquote{autoreload}}: If \sphinxcode{\sphinxupquote{True}}, the server process will restart
when any source files change, as described in {\hyperref[\detokenize{guide/running:debug-mode}]{\sphinxcrossref{\DUrole{std,std-ref}{Debug mode and automatic reloading}}}}.
This option is new in Tornado 3.2; previously this functionality
was controlled by the \sphinxcode{\sphinxupquote{debug}} setting.

\item {} 
\sphinxcode{\sphinxupquote{debug}}: Shorthand for several debug mode settings,
described in {\hyperref[\detokenize{guide/running:debug-mode}]{\sphinxcrossref{\DUrole{std,std-ref}{Debug mode and automatic reloading}}}}.  Setting \sphinxcode{\sphinxupquote{debug=True}} is
equivalent to \sphinxcode{\sphinxupquote{autoreload=True}}, \sphinxcode{\sphinxupquote{compiled\_template\_cache=False}},
\sphinxcode{\sphinxupquote{static\_hash\_cache=False}}, \sphinxcode{\sphinxupquote{serve\_traceback=True}}.

\item {} 
\sphinxcode{\sphinxupquote{default\_handler\_class}} and \sphinxcode{\sphinxupquote{default\_handler\_args}}:
This handler will be used if no other match is found;
use this to implement custom 404 pages (new in Tornado 3.2).

\item {} 
\sphinxcode{\sphinxupquote{compress\_response}}: If \sphinxcode{\sphinxupquote{True}}, responses in textual formats
will be compressed automatically.  New in Tornado 4.0.

\item {} 
\sphinxcode{\sphinxupquote{gzip}}: Deprecated alias for \sphinxcode{\sphinxupquote{compress\_response}} since
Tornado 4.0.

\item {} 
\sphinxcode{\sphinxupquote{log\_function}}: This function will be called at the end
of every request to log the result (with one argument, the
{\hyperref[\detokenize{web:tornado.web.RequestHandler}]{\sphinxcrossref{\sphinxcode{\sphinxupquote{RequestHandler}}}}} object).  The default implementation
writes to the \sphinxhref{https://docs.python.org/3.6/library/logging.html\#module-logging}{\sphinxcode{\sphinxupquote{logging}}} module’s root logger.  May also be
customized by overriding {\hyperref[\detokenize{web:tornado.web.Application.log_request}]{\sphinxcrossref{\sphinxcode{\sphinxupquote{Application.log\_request}}}}}.

\item {} 
\sphinxcode{\sphinxupquote{serve\_traceback}}: If \sphinxcode{\sphinxupquote{True}}, the default error page
will include the traceback of the error.  This option is new in
Tornado 3.2; previously this functionality was controlled by
the \sphinxcode{\sphinxupquote{debug}} setting.

\item {} 
\sphinxcode{\sphinxupquote{ui\_modules}} and \sphinxcode{\sphinxupquote{ui\_methods}}: May be set to a mapping
of {\hyperref[\detokenize{web:tornado.web.UIModule}]{\sphinxcrossref{\sphinxcode{\sphinxupquote{UIModule}}}}} or UI methods to be made available to templates.
May be set to a module, dictionary, or a list of modules
and/or dicts.  See {\hyperref[\detokenize{guide/templates:ui-modules}]{\sphinxcrossref{\DUrole{std,std-ref}{UI modules}}}} for more details.

\item {} 
\sphinxcode{\sphinxupquote{websocket\_ping\_interval}}: If set to a number, all websockets will
be pinged every n seconds. This can help keep the connection alive
through certain proxy servers which close idle connections, and it
can detect if the websocket has failed without being properly closed.

\item {} 
\sphinxcode{\sphinxupquote{websocket\_ping\_timeout}}: If the ping interval is set, and the
server doesn’t receive a ‘pong’ in this many seconds, it will close
the websocket. The default is three times the ping interval, with a
minimum of 30 seconds. Ignored if the ping interval is not set.

\end{itemize}

Authentication and security settings:
\begin{itemize}
\item {} 
\sphinxcode{\sphinxupquote{cookie\_secret}}: Used by {\hyperref[\detokenize{web:tornado.web.RequestHandler.get_secure_cookie}]{\sphinxcrossref{\sphinxcode{\sphinxupquote{RequestHandler.get\_secure\_cookie}}}}}
and {\hyperref[\detokenize{web:tornado.web.RequestHandler.set_secure_cookie}]{\sphinxcrossref{\sphinxcode{\sphinxupquote{set\_secure\_cookie}}}}} to sign cookies.

\item {} 
\sphinxcode{\sphinxupquote{key\_version}}: Used by requestHandler {\hyperref[\detokenize{web:tornado.web.RequestHandler.set_secure_cookie}]{\sphinxcrossref{\sphinxcode{\sphinxupquote{set\_secure\_cookie}}}}}
to sign cookies with a specific key when \sphinxcode{\sphinxupquote{cookie\_secret}}
is a key dictionary.

\item {} 
\sphinxcode{\sphinxupquote{login\_url}}: The {\hyperref[\detokenize{web:tornado.web.authenticated}]{\sphinxcrossref{\sphinxcode{\sphinxupquote{authenticated}}}}} decorator will redirect
to this url if the user is not logged in.  Can be further
customized by overriding {\hyperref[\detokenize{web:tornado.web.RequestHandler.get_login_url}]{\sphinxcrossref{\sphinxcode{\sphinxupquote{RequestHandler.get\_login\_url}}}}}

\item {} 
\sphinxcode{\sphinxupquote{xsrf\_cookies}}: If \sphinxcode{\sphinxupquote{True}}, {\hyperref[\detokenize{guide/security:xsrf}]{\sphinxcrossref{\DUrole{std,std-ref}{Cross-site request forgery protection}}}} will be enabled.

\item {} 
\sphinxcode{\sphinxupquote{xsrf\_cookie\_version}}: Controls the version of new XSRF
cookies produced by this server.  Should generally be left
at the default (which will always be the highest supported
version), but may be set to a lower value temporarily
during version transitions.  New in Tornado 3.2.2, which
introduced XSRF cookie version 2.

\item {} 
\sphinxcode{\sphinxupquote{xsrf\_cookie\_kwargs}}: May be set to a dictionary of
additional arguments to be passed to {\hyperref[\detokenize{web:tornado.web.RequestHandler.set_cookie}]{\sphinxcrossref{\sphinxcode{\sphinxupquote{RequestHandler.set\_cookie}}}}}
for the XSRF cookie.

\item {} 
\sphinxcode{\sphinxupquote{twitter\_consumer\_key}}, \sphinxcode{\sphinxupquote{twitter\_consumer\_secret}},
\sphinxcode{\sphinxupquote{friendfeed\_consumer\_key}}, \sphinxcode{\sphinxupquote{friendfeed\_consumer\_secret}},
\sphinxcode{\sphinxupquote{google\_consumer\_key}}, \sphinxcode{\sphinxupquote{google\_consumer\_secret}},
\sphinxcode{\sphinxupquote{facebook\_api\_key}}, \sphinxcode{\sphinxupquote{facebook\_secret}}:  Used in the
{\hyperref[\detokenize{auth:module-tornado.auth}]{\sphinxcrossref{\sphinxcode{\sphinxupquote{tornado.auth}}}}} module to authenticate to various APIs.

\end{itemize}

Template settings:
\begin{itemize}
\item {} 
\sphinxcode{\sphinxupquote{autoescape}}: Controls automatic escaping for templates.
May be set to \sphinxcode{\sphinxupquote{None}} to disable escaping, or to the \sphinxstyleemphasis{name}
of a function that all output should be passed through.
Defaults to \sphinxcode{\sphinxupquote{"xhtml\_escape"}}.  Can be changed on a per-template
basis with the \sphinxcode{\sphinxupquote{\{\% autoescape \%\}}} directive.

\item {} 
\sphinxcode{\sphinxupquote{compiled\_template\_cache}}: Default is \sphinxcode{\sphinxupquote{True}}; if \sphinxcode{\sphinxupquote{False}}
templates will be recompiled on every request.  This option
is new in Tornado 3.2; previously this functionality was controlled
by the \sphinxcode{\sphinxupquote{debug}} setting.

\item {} 
\sphinxcode{\sphinxupquote{template\_path}}: Directory containing template files.  Can be
further customized by overriding {\hyperref[\detokenize{web:tornado.web.RequestHandler.get_template_path}]{\sphinxcrossref{\sphinxcode{\sphinxupquote{RequestHandler.get\_template\_path}}}}}

\item {} 
\sphinxcode{\sphinxupquote{template\_loader}}: Assign to an instance of
{\hyperref[\detokenize{template:tornado.template.BaseLoader}]{\sphinxcrossref{\sphinxcode{\sphinxupquote{tornado.template.BaseLoader}}}}} to customize template loading.
If this setting is used the \sphinxcode{\sphinxupquote{template\_path}} and \sphinxcode{\sphinxupquote{autoescape}}
settings are ignored.  Can be further customized by overriding
{\hyperref[\detokenize{web:tornado.web.RequestHandler.create_template_loader}]{\sphinxcrossref{\sphinxcode{\sphinxupquote{RequestHandler.create\_template\_loader}}}}}.

\item {} 
\sphinxcode{\sphinxupquote{template\_whitespace}}: Controls handling of whitespace in
templates; see {\hyperref[\detokenize{template:tornado.template.filter_whitespace}]{\sphinxcrossref{\sphinxcode{\sphinxupquote{tornado.template.filter\_whitespace}}}}} for allowed
values. New in Tornado 4.3.

\end{itemize}

Static file settings:
\begin{itemize}
\item {} 
\sphinxcode{\sphinxupquote{static\_hash\_cache}}: Default is \sphinxcode{\sphinxupquote{True}}; if \sphinxcode{\sphinxupquote{False}}
static urls will be recomputed on every request.  This option
is new in Tornado 3.2; previously this functionality was controlled
by the \sphinxcode{\sphinxupquote{debug}} setting.

\item {} 
\sphinxcode{\sphinxupquote{static\_path}}: Directory from which static files will be
served.

\item {} 
\sphinxcode{\sphinxupquote{static\_url\_prefix}}:  Url prefix for static files,
defaults to \sphinxcode{\sphinxupquote{"/static/"}}.

\item {} 
\sphinxcode{\sphinxupquote{static\_handler\_class}}, \sphinxcode{\sphinxupquote{static\_handler\_args}}: May be set to
use a different handler for static files instead of the default
{\hyperref[\detokenize{web:tornado.web.StaticFileHandler}]{\sphinxcrossref{\sphinxcode{\sphinxupquote{tornado.web.StaticFileHandler}}}}}.  \sphinxcode{\sphinxupquote{static\_handler\_args}}, if set,
should be a dictionary of keyword arguments to be passed to the
handler’s \sphinxcode{\sphinxupquote{initialize}} method.

\end{itemize}

\end{fulllineitems}


\end{fulllineitems}

\index{listen() (tornado.web.Application method)@\spxentry{listen()}\spxextra{tornado.web.Application method}}

\begin{fulllineitems}
\phantomsection\label{\detokenize{web:tornado.web.Application.listen}}\pysiglinewithargsret{\sphinxcode{\sphinxupquote{Application.}}\sphinxbfcode{\sphinxupquote{listen}}}{\emph{port: int}, \emph{address: str = ''}, \emph{**kwargs}}{{ $\rightarrow$ tornado.httpserver.HTTPServer}}
Starts an HTTP server for this application on the given port.

This is a convenience alias for creating an {\hyperref[\detokenize{httpserver:tornado.httpserver.HTTPServer}]{\sphinxcrossref{\sphinxcode{\sphinxupquote{HTTPServer}}}}}
object and calling its listen method.  Keyword arguments not
supported by {\hyperref[\detokenize{tcpserver:tornado.tcpserver.TCPServer.listen}]{\sphinxcrossref{\sphinxcode{\sphinxupquote{HTTPServer.listen}}}}} are passed to the
{\hyperref[\detokenize{httpserver:tornado.httpserver.HTTPServer}]{\sphinxcrossref{\sphinxcode{\sphinxupquote{HTTPServer}}}}} constructor.  For advanced uses
(e.g. multi-process mode), do not use this method; create an
{\hyperref[\detokenize{httpserver:tornado.httpserver.HTTPServer}]{\sphinxcrossref{\sphinxcode{\sphinxupquote{HTTPServer}}}}} and call its
{\hyperref[\detokenize{tcpserver:tornado.tcpserver.TCPServer.bind}]{\sphinxcrossref{\sphinxcode{\sphinxupquote{TCPServer.bind}}}}}/{\hyperref[\detokenize{tcpserver:tornado.tcpserver.TCPServer.start}]{\sphinxcrossref{\sphinxcode{\sphinxupquote{TCPServer.start}}}}} methods directly.

Note that after calling this method you still need to call
\sphinxcode{\sphinxupquote{IOLoop.current().start()}} to start the server.

Returns the {\hyperref[\detokenize{httpserver:tornado.httpserver.HTTPServer}]{\sphinxcrossref{\sphinxcode{\sphinxupquote{HTTPServer}}}}} object.

\DUrole{versionmodified,changed}{Changed in version 4.3: }Now returns the {\hyperref[\detokenize{httpserver:tornado.httpserver.HTTPServer}]{\sphinxcrossref{\sphinxcode{\sphinxupquote{HTTPServer}}}}} object.

\end{fulllineitems}

\index{add\_handlers() (tornado.web.Application method)@\spxentry{add\_handlers()}\spxextra{tornado.web.Application method}}

\begin{fulllineitems}
\phantomsection\label{\detokenize{web:tornado.web.Application.add_handlers}}\pysiglinewithargsret{\sphinxcode{\sphinxupquote{Application.}}\sphinxbfcode{\sphinxupquote{add\_handlers}}}{\emph{handlers: List{[}Union{[}Rule, Tuple{]}{]}}}{}
Appends the given handlers to our handler list.

Host patterns are processed sequentially in the order they were
added. All matching patterns will be considered.

\end{fulllineitems}

\index{get\_handler\_delegate() (tornado.web.Application method)@\spxentry{get\_handler\_delegate()}\spxextra{tornado.web.Application method}}

\begin{fulllineitems}
\phantomsection\label{\detokenize{web:tornado.web.Application.get_handler_delegate}}\pysiglinewithargsret{\sphinxcode{\sphinxupquote{Application.}}\sphinxbfcode{\sphinxupquote{get\_handler\_delegate}}}{\emph{request: tornado.httputil.HTTPServerRequest, target\_class: Type{[}tornado.web.RequestHandler{]}, target\_kwargs: Dict{[}str, Any{]} = None, path\_args: List{[}bytes{]} = None, path\_kwargs: Dict{[}str, bytes{]} = None}}{{ $\rightarrow$ tornado.web.\_HandlerDelegate}}
Returns {\hyperref[\detokenize{httputil:tornado.httputil.HTTPMessageDelegate}]{\sphinxcrossref{\sphinxcode{\sphinxupquote{HTTPMessageDelegate}}}}} that can serve a request
for application and {\hyperref[\detokenize{web:tornado.web.RequestHandler}]{\sphinxcrossref{\sphinxcode{\sphinxupquote{RequestHandler}}}}} subclass.
\begin{quote}\begin{description}
\item[{Parameters}] \leavevmode\begin{itemize}
\item {} 
\sphinxstyleliteralstrong{\sphinxupquote{request}} ({\hyperref[\detokenize{httputil:tornado.httputil.HTTPServerRequest}]{\sphinxcrossref{\sphinxstyleliteralemphasis{\sphinxupquote{httputil.HTTPServerRequest}}}}}) \textendash{} current HTTP request.

\item {} 
\sphinxstyleliteralstrong{\sphinxupquote{target\_class}} ({\hyperref[\detokenize{web:tornado.web.RequestHandler}]{\sphinxcrossref{\sphinxstyleliteralemphasis{\sphinxupquote{RequestHandler}}}}}) \textendash{} a {\hyperref[\detokenize{web:tornado.web.RequestHandler}]{\sphinxcrossref{\sphinxcode{\sphinxupquote{RequestHandler}}}}} class.

\item {} 
\sphinxstyleliteralstrong{\sphinxupquote{target\_kwargs}} (\sphinxhref{https://docs.python.org/3.6/library/stdtypes.html\#dict}{\sphinxstyleliteralemphasis{\sphinxupquote{dict}}}) \textendash{} keyword arguments for \sphinxcode{\sphinxupquote{target\_class}} constructor.

\item {} 
\sphinxstyleliteralstrong{\sphinxupquote{path\_args}} (\sphinxhref{https://docs.python.org/3.6/library/stdtypes.html\#list}{\sphinxstyleliteralemphasis{\sphinxupquote{list}}}) \textendash{} positional arguments for \sphinxcode{\sphinxupquote{target\_class}} HTTP method that
will be executed while handling a request (\sphinxcode{\sphinxupquote{get}}, \sphinxcode{\sphinxupquote{post}} or any other).

\item {} 
\sphinxstyleliteralstrong{\sphinxupquote{path\_kwargs}} (\sphinxhref{https://docs.python.org/3.6/library/stdtypes.html\#dict}{\sphinxstyleliteralemphasis{\sphinxupquote{dict}}}) \textendash{} keyword arguments for \sphinxcode{\sphinxupquote{target\_class}} HTTP method.

\end{itemize}

\end{description}\end{quote}

\end{fulllineitems}

\index{reverse\_url() (tornado.web.Application method)@\spxentry{reverse\_url()}\spxextra{tornado.web.Application method}}

\begin{fulllineitems}
\phantomsection\label{\detokenize{web:tornado.web.Application.reverse_url}}\pysiglinewithargsret{\sphinxcode{\sphinxupquote{Application.}}\sphinxbfcode{\sphinxupquote{reverse\_url}}}{\emph{name: str}, \emph{*args}}{{ $\rightarrow$ str}}
Returns a URL path for handler named \sphinxcode{\sphinxupquote{name}}

The handler must be added to the application as a named {\hyperref[\detokenize{web:tornado.web.URLSpec}]{\sphinxcrossref{\sphinxcode{\sphinxupquote{URLSpec}}}}}.

Args will be substituted for capturing groups in the {\hyperref[\detokenize{web:tornado.web.URLSpec}]{\sphinxcrossref{\sphinxcode{\sphinxupquote{URLSpec}}}}} regex.
They will be converted to strings if necessary, encoded as utf8,
and url-escaped.

\end{fulllineitems}

\index{log\_request() (tornado.web.Application method)@\spxentry{log\_request()}\spxextra{tornado.web.Application method}}

\begin{fulllineitems}
\phantomsection\label{\detokenize{web:tornado.web.Application.log_request}}\pysiglinewithargsret{\sphinxcode{\sphinxupquote{Application.}}\sphinxbfcode{\sphinxupquote{log\_request}}}{\emph{handler: tornado.web.RequestHandler}}{{ $\rightarrow$ None}}
Writes a completed HTTP request to the logs.

By default writes to the python root logger.  To change
this behavior either subclass Application and override this method,
or pass a function in the application settings dictionary as
\sphinxcode{\sphinxupquote{log\_function}}.

\end{fulllineitems}

\index{URLSpec (class in tornado.web)@\spxentry{URLSpec}\spxextra{class in tornado.web}}

\begin{fulllineitems}
\phantomsection\label{\detokenize{web:tornado.web.URLSpec}}\pysiglinewithargsret{\sphinxbfcode{\sphinxupquote{class }}\sphinxcode{\sphinxupquote{tornado.web.}}\sphinxbfcode{\sphinxupquote{URLSpec}}}{\emph{pattern: Union{[}str, Pattern{[}AnyStr{]}{]}, handler: Any, kwargs: Dict{[}str, Any{]} = None, name: str = None}}{}
Specifies mappings between URLs and handlers.

Parameters:
\begin{itemize}
\item {} 
\sphinxcode{\sphinxupquote{pattern}}: Regular expression to be matched. Any capturing
groups in the regex will be passed in to the handler’s
get/post/etc methods as arguments (by keyword if named, by
position if unnamed. Named and unnamed capturing groups
may not be mixed in the same rule).

\item {} 
\sphinxcode{\sphinxupquote{handler}}: {\hyperref[\detokenize{web:tornado.web.RequestHandler}]{\sphinxcrossref{\sphinxcode{\sphinxupquote{RequestHandler}}}}} subclass to be invoked.

\item {} 
\sphinxcode{\sphinxupquote{kwargs}} (optional): A dictionary of additional arguments
to be passed to the handler’s constructor.

\item {} 
\sphinxcode{\sphinxupquote{name}} (optional): A name for this handler.  Used by
{\hyperref[\detokenize{web:tornado.web.Application.reverse_url}]{\sphinxcrossref{\sphinxcode{\sphinxupquote{reverse\_url}}}}}.

\end{itemize}

The \sphinxcode{\sphinxupquote{URLSpec}} class is also available under the name \sphinxcode{\sphinxupquote{tornado.web.url}}.

\end{fulllineitems}



\subsubsection{Decorators}
\label{\detokenize{web:decorators}}\index{authenticated() (in module tornado.web)@\spxentry{authenticated()}\spxextra{in module tornado.web}}

\begin{fulllineitems}
\phantomsection\label{\detokenize{web:tornado.web.authenticated}}\pysiglinewithargsret{\sphinxcode{\sphinxupquote{tornado.web.}}\sphinxbfcode{\sphinxupquote{authenticated}}}{\emph{method: Callable{[}{[}...{]}, Optional{[}Awaitable{[}None{]}{]}{]}}}{{ $\rightarrow$ Callable{[}{[}...{]}, Optional{[}Awaitable{[}None{]}{]}{]}}}
Decorate methods with this to require that the user be logged in.

If the user is not logged in, they will be redirected to the configured
{\hyperref[\detokenize{web:tornado.web.RequestHandler.get_login_url}]{\sphinxcrossref{\sphinxcode{\sphinxupquote{login url}}}}}.

If you configure a login url with a query parameter, Tornado will
assume you know what you’re doing and use it as-is.  If not, it
will add a \sphinxhref{https://docs.python.org/3.6/library/functions.html\#next}{\sphinxcode{\sphinxupquote{next}}} parameter so the login page knows where to send
you once you’re logged in.

\end{fulllineitems}

\index{addslash() (in module tornado.web)@\spxentry{addslash()}\spxextra{in module tornado.web}}

\begin{fulllineitems}
\phantomsection\label{\detokenize{web:tornado.web.addslash}}\pysiglinewithargsret{\sphinxcode{\sphinxupquote{tornado.web.}}\sphinxbfcode{\sphinxupquote{addslash}}}{\emph{method: Callable{[}{[}...{]}, Optional{[}Awaitable{[}None{]}{]}{]}}}{{ $\rightarrow$ Callable{[}{[}...{]}, Optional{[}Awaitable{[}None{]}{]}{]}}}
Use this decorator to add a missing trailing slash to the request path.

For example, a request to \sphinxcode{\sphinxupquote{/foo}} would redirect to \sphinxcode{\sphinxupquote{/foo/}} with this
decorator. Your request handler mapping should use a regular expression
like \sphinxcode{\sphinxupquote{r'/foo/?'}} in conjunction with using the decorator.

\end{fulllineitems}

\index{removeslash() (in module tornado.web)@\spxentry{removeslash()}\spxextra{in module tornado.web}}

\begin{fulllineitems}
\phantomsection\label{\detokenize{web:tornado.web.removeslash}}\pysiglinewithargsret{\sphinxcode{\sphinxupquote{tornado.web.}}\sphinxbfcode{\sphinxupquote{removeslash}}}{\emph{method: Callable{[}{[}...{]}, Optional{[}Awaitable{[}None{]}{]}{]}}}{{ $\rightarrow$ Callable{[}{[}...{]}, Optional{[}Awaitable{[}None{]}{]}{]}}}
Use this decorator to remove trailing slashes from the request path.

For example, a request to \sphinxcode{\sphinxupquote{/foo/}} would redirect to \sphinxcode{\sphinxupquote{/foo}} with this
decorator. Your request handler mapping should use a regular expression
like \sphinxcode{\sphinxupquote{r'/foo/*'}} in conjunction with using the decorator.

\end{fulllineitems}

\index{stream\_request\_body() (in module tornado.web)@\spxentry{stream\_request\_body()}\spxextra{in module tornado.web}}

\begin{fulllineitems}
\phantomsection\label{\detokenize{web:tornado.web.stream_request_body}}\pysiglinewithargsret{\sphinxcode{\sphinxupquote{tornado.web.}}\sphinxbfcode{\sphinxupquote{stream\_request\_body}}}{\emph{cls: Type{[}tornado.web.RequestHandler{]}}}{{ $\rightarrow$ Type{[}tornado.web.RequestHandler{]}}}
Apply to {\hyperref[\detokenize{web:tornado.web.RequestHandler}]{\sphinxcrossref{\sphinxcode{\sphinxupquote{RequestHandler}}}}} subclasses to enable streaming body support.

This decorator implies the following changes:
\begin{itemize}
\item {} 
{\hyperref[\detokenize{httputil:tornado.httputil.HTTPServerRequest.body}]{\sphinxcrossref{\sphinxcode{\sphinxupquote{HTTPServerRequest.body}}}}} is undefined, and body arguments will not
be included in {\hyperref[\detokenize{web:tornado.web.RequestHandler.get_argument}]{\sphinxcrossref{\sphinxcode{\sphinxupquote{RequestHandler.get\_argument}}}}}.

\item {} 
{\hyperref[\detokenize{web:tornado.web.RequestHandler.prepare}]{\sphinxcrossref{\sphinxcode{\sphinxupquote{RequestHandler.prepare}}}}} is called when the request headers have been
read instead of after the entire body has been read.

\item {} 
The subclass must define a method \sphinxcode{\sphinxupquote{data\_received(self, data):}}, which
will be called zero or more times as data is available.  Note that
if the request has an empty body, \sphinxcode{\sphinxupquote{data\_received}} may not be called.

\item {} 
\sphinxcode{\sphinxupquote{prepare}} and \sphinxcode{\sphinxupquote{data\_received}} may return Futures (such as via
\sphinxcode{\sphinxupquote{@gen.coroutine}}, in which case the next method will not be called
until those futures have completed.

\item {} 
The regular HTTP method (\sphinxcode{\sphinxupquote{post}}, \sphinxcode{\sphinxupquote{put}}, etc) will be called after
the entire body has been read.

\end{itemize}

See the \sphinxhref{https://github.com/tornadoweb/tornado/tree/master/demos/file\_upload/}{file receiver demo}
for example usage.

\end{fulllineitems}



\subsubsection{Everything else}
\label{\detokenize{web:everything-else}}\index{HTTPError@\spxentry{HTTPError}}

\begin{fulllineitems}
\phantomsection\label{\detokenize{web:tornado.web.HTTPError}}\pysiglinewithargsret{\sphinxbfcode{\sphinxupquote{exception }}\sphinxcode{\sphinxupquote{tornado.web.}}\sphinxbfcode{\sphinxupquote{HTTPError}}}{\emph{status\_code: int = 500}, \emph{log\_message: str = None}, \emph{*args}, \emph{**kwargs}}{}
An exception that will turn into an HTTP error response.

Raising an {\hyperref[\detokenize{web:tornado.web.HTTPError}]{\sphinxcrossref{\sphinxcode{\sphinxupquote{HTTPError}}}}} is a convenient alternative to calling
{\hyperref[\detokenize{web:tornado.web.RequestHandler.send_error}]{\sphinxcrossref{\sphinxcode{\sphinxupquote{RequestHandler.send\_error}}}}} since it automatically ends the
current function.

To customize the response sent with an {\hyperref[\detokenize{web:tornado.web.HTTPError}]{\sphinxcrossref{\sphinxcode{\sphinxupquote{HTTPError}}}}}, override
{\hyperref[\detokenize{web:tornado.web.RequestHandler.write_error}]{\sphinxcrossref{\sphinxcode{\sphinxupquote{RequestHandler.write\_error}}}}}.
\begin{quote}\begin{description}
\item[{Parameters}] \leavevmode\begin{itemize}
\item {} 
\sphinxstyleliteralstrong{\sphinxupquote{status\_code}} (\sphinxhref{https://docs.python.org/3.6/library/functions.html\#int}{\sphinxstyleliteralemphasis{\sphinxupquote{int}}}) \textendash{} HTTP status code.  Must be listed in
\sphinxhref{https://docs.python.org/3.6/library/http.client.html\#http.client.responses}{\sphinxcode{\sphinxupquote{httplib.responses}}} unless the \sphinxcode{\sphinxupquote{reason}}
keyword argument is given.

\item {} 
\sphinxstyleliteralstrong{\sphinxupquote{log\_message}} (\sphinxhref{https://docs.python.org/3.6/library/stdtypes.html\#str}{\sphinxstyleliteralemphasis{\sphinxupquote{str}}}) \textendash{} Message to be written to the log for this error
(will not be shown to the user unless the {\hyperref[\detokenize{web:tornado.web.Application}]{\sphinxcrossref{\sphinxcode{\sphinxupquote{Application}}}}} is in debug
mode).  May contain \sphinxcode{\sphinxupquote{\%s}}-style placeholders, which will be filled
in with remaining positional parameters.

\item {} 
\sphinxstyleliteralstrong{\sphinxupquote{reason}} (\sphinxhref{https://docs.python.org/3.6/library/stdtypes.html\#str}{\sphinxstyleliteralemphasis{\sphinxupquote{str}}}) \textendash{} Keyword-only argument.  The HTTP “reason” phrase
to pass in the status line along with \sphinxcode{\sphinxupquote{status\_code}}.  Normally
determined automatically from \sphinxcode{\sphinxupquote{status\_code}}, but can be used
to use a non-standard numeric code.

\end{itemize}

\end{description}\end{quote}

\end{fulllineitems}

\index{Finish@\spxentry{Finish}}

\begin{fulllineitems}
\phantomsection\label{\detokenize{web:tornado.web.Finish}}\pysigline{\sphinxbfcode{\sphinxupquote{exception }}\sphinxcode{\sphinxupquote{tornado.web.}}\sphinxbfcode{\sphinxupquote{Finish}}}
An exception that ends the request without producing an error response.

When {\hyperref[\detokenize{web:tornado.web.Finish}]{\sphinxcrossref{\sphinxcode{\sphinxupquote{Finish}}}}} is raised in a {\hyperref[\detokenize{web:tornado.web.RequestHandler}]{\sphinxcrossref{\sphinxcode{\sphinxupquote{RequestHandler}}}}}, the request will
end (calling {\hyperref[\detokenize{web:tornado.web.RequestHandler.finish}]{\sphinxcrossref{\sphinxcode{\sphinxupquote{RequestHandler.finish}}}}} if it hasn’t already been
called), but the error-handling methods (including
{\hyperref[\detokenize{web:tornado.web.RequestHandler.write_error}]{\sphinxcrossref{\sphinxcode{\sphinxupquote{RequestHandler.write\_error}}}}}) will not be called.

If {\hyperref[\detokenize{web:tornado.web.Finish}]{\sphinxcrossref{\sphinxcode{\sphinxupquote{Finish()}}}}} was created with no arguments, the pending response
will be sent as-is. If {\hyperref[\detokenize{web:tornado.web.Finish}]{\sphinxcrossref{\sphinxcode{\sphinxupquote{Finish()}}}}} was given an argument, that
argument will be passed to {\hyperref[\detokenize{web:tornado.web.RequestHandler.finish}]{\sphinxcrossref{\sphinxcode{\sphinxupquote{RequestHandler.finish()}}}}}.

This can be a more convenient way to implement custom error pages
than overriding \sphinxcode{\sphinxupquote{write\_error}} (especially in library code):

\begin{sphinxVerbatim}[commandchars=\\\{\}]
\PYG{k}{if} \PYG{n+nb+bp}{self}\PYG{o}{.}\PYG{n}{current\PYGZus{}user} \PYG{o+ow}{is} \PYG{k+kc}{None}\PYG{p}{:}
    \PYG{n+nb+bp}{self}\PYG{o}{.}\PYG{n}{set\PYGZus{}status}\PYG{p}{(}\PYG{l+m+mi}{401}\PYG{p}{)}
    \PYG{n+nb+bp}{self}\PYG{o}{.}\PYG{n}{set\PYGZus{}header}\PYG{p}{(}\PYG{l+s+s1}{\PYGZsq{}}\PYG{l+s+s1}{WWW\PYGZhy{}Authenticate}\PYG{l+s+s1}{\PYGZsq{}}\PYG{p}{,} \PYG{l+s+s1}{\PYGZsq{}}\PYG{l+s+s1}{Basic realm=}\PYG{l+s+s1}{\PYGZdq{}}\PYG{l+s+s1}{something}\PYG{l+s+s1}{\PYGZdq{}}\PYG{l+s+s1}{\PYGZsq{}}\PYG{p}{)}
    \PYG{k}{raise} \PYG{n}{Finish}\PYG{p}{(}\PYG{p}{)}
\end{sphinxVerbatim}

\DUrole{versionmodified,changed}{Changed in version 4.3: }Arguments passed to \sphinxcode{\sphinxupquote{Finish()}} will be passed on to
{\hyperref[\detokenize{web:tornado.web.RequestHandler.finish}]{\sphinxcrossref{\sphinxcode{\sphinxupquote{RequestHandler.finish}}}}}.

\end{fulllineitems}

\index{MissingArgumentError@\spxentry{MissingArgumentError}}

\begin{fulllineitems}
\phantomsection\label{\detokenize{web:tornado.web.MissingArgumentError}}\pysiglinewithargsret{\sphinxbfcode{\sphinxupquote{exception }}\sphinxcode{\sphinxupquote{tornado.web.}}\sphinxbfcode{\sphinxupquote{MissingArgumentError}}}{\emph{arg\_name: str}}{}
Exception raised by {\hyperref[\detokenize{web:tornado.web.RequestHandler.get_argument}]{\sphinxcrossref{\sphinxcode{\sphinxupquote{RequestHandler.get\_argument}}}}}.

This is a subclass of {\hyperref[\detokenize{web:tornado.web.HTTPError}]{\sphinxcrossref{\sphinxcode{\sphinxupquote{HTTPError}}}}}, so if it is uncaught a 400 response
code will be used instead of 500 (and a stack trace will not be logged).

\DUrole{versionmodified,added}{New in version 3.1.}

\end{fulllineitems}

\index{UIModule (class in tornado.web)@\spxentry{UIModule}\spxextra{class in tornado.web}}

\begin{fulllineitems}
\phantomsection\label{\detokenize{web:tornado.web.UIModule}}\pysiglinewithargsret{\sphinxbfcode{\sphinxupquote{class }}\sphinxcode{\sphinxupquote{tornado.web.}}\sphinxbfcode{\sphinxupquote{UIModule}}}{\emph{handler: tornado.web.RequestHandler}}{}
A re-usable, modular UI unit on a page.

UI modules often execute additional queries, and they can include
additional CSS and JavaScript that will be included in the output
page, which is automatically inserted on page render.

Subclasses of UIModule must override the {\hyperref[\detokenize{web:tornado.web.UIModule.render}]{\sphinxcrossref{\sphinxcode{\sphinxupquote{render}}}}} method.
\index{render() (tornado.web.UIModule method)@\spxentry{render()}\spxextra{tornado.web.UIModule method}}

\begin{fulllineitems}
\phantomsection\label{\detokenize{web:tornado.web.UIModule.render}}\pysiglinewithargsret{\sphinxbfcode{\sphinxupquote{render}}}{\emph{*args}, \emph{**kwargs}}{{ $\rightarrow$ str}}
Override in subclasses to return this module’s output.

\end{fulllineitems}

\index{embedded\_javascript() (tornado.web.UIModule method)@\spxentry{embedded\_javascript()}\spxextra{tornado.web.UIModule method}}

\begin{fulllineitems}
\phantomsection\label{\detokenize{web:tornado.web.UIModule.embedded_javascript}}\pysiglinewithargsret{\sphinxbfcode{\sphinxupquote{embedded\_javascript}}}{}{{ $\rightarrow$ Optional{[}str{]}}}
Override to return a JavaScript string
to be embedded in the page.

\end{fulllineitems}

\index{javascript\_files() (tornado.web.UIModule method)@\spxentry{javascript\_files()}\spxextra{tornado.web.UIModule method}}

\begin{fulllineitems}
\phantomsection\label{\detokenize{web:tornado.web.UIModule.javascript_files}}\pysiglinewithargsret{\sphinxbfcode{\sphinxupquote{javascript\_files}}}{}{{ $\rightarrow$ Optional{[}Iterable{[}str{]}{]}}}
Override to return a list of JavaScript files needed by this module.

If the return values are relative paths, they will be passed to
{\hyperref[\detokenize{web:tornado.web.RequestHandler.static_url}]{\sphinxcrossref{\sphinxcode{\sphinxupquote{RequestHandler.static\_url}}}}}; otherwise they will be used as-is.

\end{fulllineitems}

\index{embedded\_css() (tornado.web.UIModule method)@\spxentry{embedded\_css()}\spxextra{tornado.web.UIModule method}}

\begin{fulllineitems}
\phantomsection\label{\detokenize{web:tornado.web.UIModule.embedded_css}}\pysiglinewithargsret{\sphinxbfcode{\sphinxupquote{embedded\_css}}}{}{{ $\rightarrow$ Optional{[}str{]}}}
Override to return a CSS string
that will be embedded in the page.

\end{fulllineitems}

\index{css\_files() (tornado.web.UIModule method)@\spxentry{css\_files()}\spxextra{tornado.web.UIModule method}}

\begin{fulllineitems}
\phantomsection\label{\detokenize{web:tornado.web.UIModule.css_files}}\pysiglinewithargsret{\sphinxbfcode{\sphinxupquote{css\_files}}}{}{{ $\rightarrow$ Optional{[}Iterable{[}str{]}{]}}}
Override to returns a list of CSS files required by this module.

If the return values are relative paths, they will be passed to
{\hyperref[\detokenize{web:tornado.web.RequestHandler.static_url}]{\sphinxcrossref{\sphinxcode{\sphinxupquote{RequestHandler.static\_url}}}}}; otherwise they will be used as-is.

\end{fulllineitems}

\index{html\_head() (tornado.web.UIModule method)@\spxentry{html\_head()}\spxextra{tornado.web.UIModule method}}

\begin{fulllineitems}
\phantomsection\label{\detokenize{web:tornado.web.UIModule.html_head}}\pysiglinewithargsret{\sphinxbfcode{\sphinxupquote{html\_head}}}{}{{ $\rightarrow$ Optional{[}str{]}}}
Override to return an HTML string that will be put in the \textless{}head/\textgreater{}
element.

\end{fulllineitems}

\index{html\_body() (tornado.web.UIModule method)@\spxentry{html\_body()}\spxextra{tornado.web.UIModule method}}

\begin{fulllineitems}
\phantomsection\label{\detokenize{web:tornado.web.UIModule.html_body}}\pysiglinewithargsret{\sphinxbfcode{\sphinxupquote{html\_body}}}{}{{ $\rightarrow$ Optional{[}str{]}}}
Override to return an HTML string that will be put at the end of
the \textless{}body/\textgreater{} element.

\end{fulllineitems}

\index{render\_string() (tornado.web.UIModule method)@\spxentry{render\_string()}\spxextra{tornado.web.UIModule method}}

\begin{fulllineitems}
\phantomsection\label{\detokenize{web:tornado.web.UIModule.render_string}}\pysiglinewithargsret{\sphinxbfcode{\sphinxupquote{render\_string}}}{\emph{path: str}, \emph{**kwargs}}{{ $\rightarrow$ bytes}}
Renders a template and returns it as a string.

\end{fulllineitems}


\end{fulllineitems}

\index{ErrorHandler (class in tornado.web)@\spxentry{ErrorHandler}\spxextra{class in tornado.web}}

\begin{fulllineitems}
\phantomsection\label{\detokenize{web:tornado.web.ErrorHandler}}\pysiglinewithargsret{\sphinxbfcode{\sphinxupquote{class }}\sphinxcode{\sphinxupquote{tornado.web.}}\sphinxbfcode{\sphinxupquote{ErrorHandler}}}{\emph{application: tornado.web.Application}, \emph{request: tornado.httputil.HTTPServerRequest}, \emph{**kwargs}}{}
Generates an error response with \sphinxcode{\sphinxupquote{status\_code}} for all requests.

\end{fulllineitems}

\index{FallbackHandler (class in tornado.web)@\spxentry{FallbackHandler}\spxextra{class in tornado.web}}

\begin{fulllineitems}
\phantomsection\label{\detokenize{web:tornado.web.FallbackHandler}}\pysiglinewithargsret{\sphinxbfcode{\sphinxupquote{class }}\sphinxcode{\sphinxupquote{tornado.web.}}\sphinxbfcode{\sphinxupquote{FallbackHandler}}}{\emph{application: tornado.web.Application}, \emph{request: tornado.httputil.HTTPServerRequest}, \emph{**kwargs}}{}
A {\hyperref[\detokenize{web:tornado.web.RequestHandler}]{\sphinxcrossref{\sphinxcode{\sphinxupquote{RequestHandler}}}}} that wraps another HTTP server callback.

The fallback is a callable object that accepts an
{\hyperref[\detokenize{httputil:tornado.httputil.HTTPServerRequest}]{\sphinxcrossref{\sphinxcode{\sphinxupquote{HTTPServerRequest}}}}}, such as an {\hyperref[\detokenize{web:tornado.web.Application}]{\sphinxcrossref{\sphinxcode{\sphinxupquote{Application}}}}} or
{\hyperref[\detokenize{wsgi:tornado.wsgi.WSGIContainer}]{\sphinxcrossref{\sphinxcode{\sphinxupquote{tornado.wsgi.WSGIContainer}}}}}.  This is most useful to use both
Tornado \sphinxcode{\sphinxupquote{RequestHandlers}} and WSGI in the same server.  Typical
usage:

\begin{sphinxVerbatim}[commandchars=\\\{\}]
\PYG{n}{wsgi\PYGZus{}app} \PYG{o}{=} \PYG{n}{tornado}\PYG{o}{.}\PYG{n}{wsgi}\PYG{o}{.}\PYG{n}{WSGIContainer}\PYG{p}{(}
    \PYG{n}{django}\PYG{o}{.}\PYG{n}{core}\PYG{o}{.}\PYG{n}{handlers}\PYG{o}{.}\PYG{n}{wsgi}\PYG{o}{.}\PYG{n}{WSGIHandler}\PYG{p}{(}\PYG{p}{)}\PYG{p}{)}
\PYG{n}{application} \PYG{o}{=} \PYG{n}{tornado}\PYG{o}{.}\PYG{n}{web}\PYG{o}{.}\PYG{n}{Application}\PYG{p}{(}\PYG{p}{[}
    \PYG{p}{(}\PYG{l+s+sa}{r}\PYG{l+s+s2}{\PYGZdq{}}\PYG{l+s+s2}{/foo}\PYG{l+s+s2}{\PYGZdq{}}\PYG{p}{,} \PYG{n}{FooHandler}\PYG{p}{)}\PYG{p}{,}
    \PYG{p}{(}\PYG{l+s+sa}{r}\PYG{l+s+s2}{\PYGZdq{}}\PYG{l+s+s2}{.*}\PYG{l+s+s2}{\PYGZdq{}}\PYG{p}{,} \PYG{n}{FallbackHandler}\PYG{p}{,} \PYG{n+nb}{dict}\PYG{p}{(}\PYG{n}{fallback}\PYG{o}{=}\PYG{n}{wsgi\PYGZus{}app}\PYG{p}{)}\PYG{p}{,}
\PYG{p}{]}\PYG{p}{)}
\end{sphinxVerbatim}

\end{fulllineitems}

\index{RedirectHandler (class in tornado.web)@\spxentry{RedirectHandler}\spxextra{class in tornado.web}}

\begin{fulllineitems}
\phantomsection\label{\detokenize{web:tornado.web.RedirectHandler}}\pysiglinewithargsret{\sphinxbfcode{\sphinxupquote{class }}\sphinxcode{\sphinxupquote{tornado.web.}}\sphinxbfcode{\sphinxupquote{RedirectHandler}}}{\emph{application: tornado.web.Application}, \emph{request: tornado.httputil.HTTPServerRequest}, \emph{**kwargs}}{}
Redirects the client to the given URL for all GET requests.

You should provide the keyword argument \sphinxcode{\sphinxupquote{url}} to the handler, e.g.:

\begin{sphinxVerbatim}[commandchars=\\\{\}]
\PYG{n}{application} \PYG{o}{=} \PYG{n}{web}\PYG{o}{.}\PYG{n}{Application}\PYG{p}{(}\PYG{p}{[}
    \PYG{p}{(}\PYG{l+s+sa}{r}\PYG{l+s+s2}{\PYGZdq{}}\PYG{l+s+s2}{/oldpath}\PYG{l+s+s2}{\PYGZdq{}}\PYG{p}{,} \PYG{n}{web}\PYG{o}{.}\PYG{n}{RedirectHandler}\PYG{p}{,} \PYG{p}{\PYGZob{}}\PYG{l+s+s2}{\PYGZdq{}}\PYG{l+s+s2}{url}\PYG{l+s+s2}{\PYGZdq{}}\PYG{p}{:} \PYG{l+s+s2}{\PYGZdq{}}\PYG{l+s+s2}{/newpath}\PYG{l+s+s2}{\PYGZdq{}}\PYG{p}{\PYGZcb{}}\PYG{p}{)}\PYG{p}{,}
\PYG{p}{]}\PYG{p}{)}
\end{sphinxVerbatim}

{\hyperref[\detokenize{web:tornado.web.RedirectHandler}]{\sphinxcrossref{\sphinxcode{\sphinxupquote{RedirectHandler}}}}} supports regular expression substitutions. E.g., to
swap the first and second parts of a path while preserving the remainder:

\begin{sphinxVerbatim}[commandchars=\\\{\}]
\PYG{n}{application} \PYG{o}{=} \PYG{n}{web}\PYG{o}{.}\PYG{n}{Application}\PYG{p}{(}\PYG{p}{[}
    \PYG{p}{(}\PYG{l+s+sa}{r}\PYG{l+s+s2}{\PYGZdq{}}\PYG{l+s+s2}{/(.*?)/(.*?)/(.*)}\PYG{l+s+s2}{\PYGZdq{}}\PYG{p}{,} \PYG{n}{web}\PYG{o}{.}\PYG{n}{RedirectHandler}\PYG{p}{,} \PYG{p}{\PYGZob{}}\PYG{l+s+s2}{\PYGZdq{}}\PYG{l+s+s2}{url}\PYG{l+s+s2}{\PYGZdq{}}\PYG{p}{:} \PYG{l+s+s2}{\PYGZdq{}}\PYG{l+s+s2}{/}\PYG{l+s+si}{\PYGZob{}1\PYGZcb{}}\PYG{l+s+s2}{/}\PYG{l+s+si}{\PYGZob{}0\PYGZcb{}}\PYG{l+s+s2}{/}\PYG{l+s+si}{\PYGZob{}2\PYGZcb{}}\PYG{l+s+s2}{\PYGZdq{}}\PYG{p}{\PYGZcb{}}\PYG{p}{)}\PYG{p}{,}
\PYG{p}{]}\PYG{p}{)}
\end{sphinxVerbatim}

The final URL is formatted with \sphinxhref{https://docs.python.org/3.6/library/stdtypes.html\#str.format}{\sphinxcode{\sphinxupquote{str.format}}} and the substrings that match
the capturing groups. In the above example, a request to “/a/b/c” would be
formatted like:

\begin{sphinxVerbatim}[commandchars=\\\{\}]
\PYG{n+nb}{str}\PYG{o}{.}\PYG{n}{format}\PYG{p}{(}\PYG{l+s+s2}{\PYGZdq{}}\PYG{l+s+s2}{/}\PYG{l+s+si}{\PYGZob{}1\PYGZcb{}}\PYG{l+s+s2}{/}\PYG{l+s+si}{\PYGZob{}0\PYGZcb{}}\PYG{l+s+s2}{/}\PYG{l+s+si}{\PYGZob{}2\PYGZcb{}}\PYG{l+s+s2}{\PYGZdq{}}\PYG{p}{,} \PYG{l+s+s2}{\PYGZdq{}}\PYG{l+s+s2}{a}\PYG{l+s+s2}{\PYGZdq{}}\PYG{p}{,} \PYG{l+s+s2}{\PYGZdq{}}\PYG{l+s+s2}{b}\PYG{l+s+s2}{\PYGZdq{}}\PYG{p}{,} \PYG{l+s+s2}{\PYGZdq{}}\PYG{l+s+s2}{c}\PYG{l+s+s2}{\PYGZdq{}}\PYG{p}{)}  \PYG{c+c1}{\PYGZsh{} \PYGZhy{}\PYGZgt{} \PYGZdq{}/b/a/c\PYGZdq{}}
\end{sphinxVerbatim}

Use Python’s \sphinxhref{https://docs.python.org/3.6/library/string.html\#formatstrings}{\DUrole{xref,std,std-ref}{format string syntax}} to customize how
values are substituted.

\DUrole{versionmodified,changed}{Changed in version 4.5: }Added support for substitutions into the destination URL.

\DUrole{versionmodified,changed}{Changed in version 5.0: }If any query arguments are present, they will be copied to the
destination URL.

\end{fulllineitems}

\index{StaticFileHandler (class in tornado.web)@\spxentry{StaticFileHandler}\spxextra{class in tornado.web}}

\begin{fulllineitems}
\phantomsection\label{\detokenize{web:tornado.web.StaticFileHandler}}\pysiglinewithargsret{\sphinxbfcode{\sphinxupquote{class }}\sphinxcode{\sphinxupquote{tornado.web.}}\sphinxbfcode{\sphinxupquote{StaticFileHandler}}}{\emph{application: tornado.web.Application}, \emph{request: tornado.httputil.HTTPServerRequest}, \emph{**kwargs}}{}
A simple handler that can serve static content from a directory.

A {\hyperref[\detokenize{web:tornado.web.StaticFileHandler}]{\sphinxcrossref{\sphinxcode{\sphinxupquote{StaticFileHandler}}}}} is configured automatically if you pass the
\sphinxcode{\sphinxupquote{static\_path}} keyword argument to {\hyperref[\detokenize{web:tornado.web.Application}]{\sphinxcrossref{\sphinxcode{\sphinxupquote{Application}}}}}.  This handler
can be customized with the \sphinxcode{\sphinxupquote{static\_url\_prefix}}, \sphinxcode{\sphinxupquote{static\_handler\_class}},
and \sphinxcode{\sphinxupquote{static\_handler\_args}} settings.

To map an additional path to this handler for a static data directory
you would add a line to your application like:

\begin{sphinxVerbatim}[commandchars=\\\{\}]
\PYG{n}{application} \PYG{o}{=} \PYG{n}{web}\PYG{o}{.}\PYG{n}{Application}\PYG{p}{(}\PYG{p}{[}
    \PYG{p}{(}\PYG{l+s+sa}{r}\PYG{l+s+s2}{\PYGZdq{}}\PYG{l+s+s2}{/content/(.*)}\PYG{l+s+s2}{\PYGZdq{}}\PYG{p}{,} \PYG{n}{web}\PYG{o}{.}\PYG{n}{StaticFileHandler}\PYG{p}{,} \PYG{p}{\PYGZob{}}\PYG{l+s+s2}{\PYGZdq{}}\PYG{l+s+s2}{path}\PYG{l+s+s2}{\PYGZdq{}}\PYG{p}{:} \PYG{l+s+s2}{\PYGZdq{}}\PYG{l+s+s2}{/var/www}\PYG{l+s+s2}{\PYGZdq{}}\PYG{p}{\PYGZcb{}}\PYG{p}{)}\PYG{p}{,}
\PYG{p}{]}\PYG{p}{)}
\end{sphinxVerbatim}

The handler constructor requires a \sphinxcode{\sphinxupquote{path}} argument, which specifies the
local root directory of the content to be served.

Note that a capture group in the regex is required to parse the value for
the \sphinxcode{\sphinxupquote{path}} argument to the get() method (different than the constructor
argument above); see {\hyperref[\detokenize{web:tornado.web.URLSpec}]{\sphinxcrossref{\sphinxcode{\sphinxupquote{URLSpec}}}}} for details.

To serve a file like \sphinxcode{\sphinxupquote{index.html}} automatically when a directory is
requested, set \sphinxcode{\sphinxupquote{static\_handler\_args=dict(default\_filename="index.html")}}
in your application settings, or add \sphinxcode{\sphinxupquote{default\_filename}} as an initializer
argument for your \sphinxcode{\sphinxupquote{StaticFileHandler}}.

To maximize the effectiveness of browser caching, this class supports
versioned urls (by default using the argument \sphinxcode{\sphinxupquote{?v=}}).  If a version
is given, we instruct the browser to cache this file indefinitely.
{\hyperref[\detokenize{web:tornado.web.StaticFileHandler.make_static_url}]{\sphinxcrossref{\sphinxcode{\sphinxupquote{make\_static\_url}}}}} (also available as {\hyperref[\detokenize{web:tornado.web.RequestHandler.static_url}]{\sphinxcrossref{\sphinxcode{\sphinxupquote{RequestHandler.static\_url}}}}}) can
be used to construct a versioned url.

This handler is intended primarily for use in development and light-duty
file serving; for heavy traffic it will be more efficient to use
a dedicated static file server (such as nginx or Apache).  We support
the HTTP \sphinxcode{\sphinxupquote{Accept-Ranges}} mechanism to return partial content (because
some browsers require this functionality to be present to seek in
HTML5 audio or video).

\sphinxstylestrong{Subclassing notes}

This class is designed to be extensible by subclassing, but because
of the way static urls are generated with class methods rather than
instance methods, the inheritance patterns are somewhat unusual.
Be sure to use the \sphinxcode{\sphinxupquote{@classmethod}} decorator when overriding a
class method.  Instance methods may use the attributes \sphinxcode{\sphinxupquote{self.path}}
\sphinxcode{\sphinxupquote{self.absolute\_path}}, and \sphinxcode{\sphinxupquote{self.modified}}.

Subclasses should only override methods discussed in this section;
overriding other methods is error-prone.  Overriding
\sphinxcode{\sphinxupquote{StaticFileHandler.get}} is particularly problematic due to the
tight coupling with \sphinxcode{\sphinxupquote{compute\_etag}} and other methods.

To change the way static urls are generated (e.g. to match the behavior
of another server or CDN), override {\hyperref[\detokenize{web:tornado.web.StaticFileHandler.make_static_url}]{\sphinxcrossref{\sphinxcode{\sphinxupquote{make\_static\_url}}}}}, {\hyperref[\detokenize{web:tornado.web.StaticFileHandler.parse_url_path}]{\sphinxcrossref{\sphinxcode{\sphinxupquote{parse\_url\_path}}}}},
{\hyperref[\detokenize{web:tornado.web.StaticFileHandler.get_cache_time}]{\sphinxcrossref{\sphinxcode{\sphinxupquote{get\_cache\_time}}}}}, and/or {\hyperref[\detokenize{web:tornado.web.StaticFileHandler.get_version}]{\sphinxcrossref{\sphinxcode{\sphinxupquote{get\_version}}}}}.

To replace all interaction with the filesystem (e.g. to serve
static content from a database), override {\hyperref[\detokenize{web:tornado.web.StaticFileHandler.get_content}]{\sphinxcrossref{\sphinxcode{\sphinxupquote{get\_content}}}}},
{\hyperref[\detokenize{web:tornado.web.StaticFileHandler.get_content_size}]{\sphinxcrossref{\sphinxcode{\sphinxupquote{get\_content\_size}}}}}, {\hyperref[\detokenize{web:tornado.web.StaticFileHandler.get_modified_time}]{\sphinxcrossref{\sphinxcode{\sphinxupquote{get\_modified\_time}}}}}, {\hyperref[\detokenize{web:tornado.web.StaticFileHandler.get_absolute_path}]{\sphinxcrossref{\sphinxcode{\sphinxupquote{get\_absolute\_path}}}}}, and
{\hyperref[\detokenize{web:tornado.web.StaticFileHandler.validate_absolute_path}]{\sphinxcrossref{\sphinxcode{\sphinxupquote{validate\_absolute\_path}}}}}.

\DUrole{versionmodified,changed}{Changed in version 3.1: }Many of the methods for subclasses were added in Tornado 3.1.
\index{compute\_etag() (tornado.web.StaticFileHandler method)@\spxentry{compute\_etag()}\spxextra{tornado.web.StaticFileHandler method}}

\begin{fulllineitems}
\phantomsection\label{\detokenize{web:tornado.web.StaticFileHandler.compute_etag}}\pysiglinewithargsret{\sphinxbfcode{\sphinxupquote{compute\_etag}}}{}{{ $\rightarrow$ Optional{[}str{]}}}
Sets the \sphinxcode{\sphinxupquote{Etag}} header based on static url version.

This allows efficient \sphinxcode{\sphinxupquote{If-None-Match}} checks against cached
versions, and sends the correct \sphinxcode{\sphinxupquote{Etag}} for a partial response
(i.e. the same \sphinxcode{\sphinxupquote{Etag}} as the full file).

\DUrole{versionmodified,added}{New in version 3.1.}

\end{fulllineitems}

\index{set\_headers() (tornado.web.StaticFileHandler method)@\spxentry{set\_headers()}\spxextra{tornado.web.StaticFileHandler method}}

\begin{fulllineitems}
\phantomsection\label{\detokenize{web:tornado.web.StaticFileHandler.set_headers}}\pysiglinewithargsret{\sphinxbfcode{\sphinxupquote{set\_headers}}}{}{{ $\rightarrow$ None}}
Sets the content and caching headers on the response.

\DUrole{versionmodified,added}{New in version 3.1.}

\end{fulllineitems}

\index{should\_return\_304() (tornado.web.StaticFileHandler method)@\spxentry{should\_return\_304()}\spxextra{tornado.web.StaticFileHandler method}}

\begin{fulllineitems}
\phantomsection\label{\detokenize{web:tornado.web.StaticFileHandler.should_return_304}}\pysiglinewithargsret{\sphinxbfcode{\sphinxupquote{should\_return\_304}}}{}{{ $\rightarrow$ bool}}
Returns True if the headers indicate that we should return 304.

\DUrole{versionmodified,added}{New in version 3.1.}

\end{fulllineitems}

\index{get\_absolute\_path() (tornado.web.StaticFileHandler class method)@\spxentry{get\_absolute\_path()}\spxextra{tornado.web.StaticFileHandler class method}}

\begin{fulllineitems}
\phantomsection\label{\detokenize{web:tornado.web.StaticFileHandler.get_absolute_path}}\pysiglinewithargsret{\sphinxbfcode{\sphinxupquote{classmethod }}\sphinxbfcode{\sphinxupquote{get\_absolute\_path}}}{\emph{root: str}, \emph{path: str}}{{ $\rightarrow$ str}}
Returns the absolute location of \sphinxcode{\sphinxupquote{path}} relative to \sphinxcode{\sphinxupquote{root}}.

\sphinxcode{\sphinxupquote{root}} is the path configured for this {\hyperref[\detokenize{web:tornado.web.StaticFileHandler}]{\sphinxcrossref{\sphinxcode{\sphinxupquote{StaticFileHandler}}}}}
(in most cases the \sphinxcode{\sphinxupquote{static\_path}} {\hyperref[\detokenize{web:tornado.web.Application}]{\sphinxcrossref{\sphinxcode{\sphinxupquote{Application}}}}} setting).

This class method may be overridden in subclasses.  By default
it returns a filesystem path, but other strings may be used
as long as they are unique and understood by the subclass’s
overridden {\hyperref[\detokenize{web:tornado.web.StaticFileHandler.get_content}]{\sphinxcrossref{\sphinxcode{\sphinxupquote{get\_content}}}}}.

\DUrole{versionmodified,added}{New in version 3.1.}

\end{fulllineitems}

\index{validate\_absolute\_path() (tornado.web.StaticFileHandler method)@\spxentry{validate\_absolute\_path()}\spxextra{tornado.web.StaticFileHandler method}}

\begin{fulllineitems}
\phantomsection\label{\detokenize{web:tornado.web.StaticFileHandler.validate_absolute_path}}\pysiglinewithargsret{\sphinxbfcode{\sphinxupquote{validate\_absolute\_path}}}{\emph{root: str}, \emph{absolute\_path: str}}{{ $\rightarrow$ Optional{[}str{]}}}
Validate and return the absolute path.

\sphinxcode{\sphinxupquote{root}} is the configured path for the {\hyperref[\detokenize{web:tornado.web.StaticFileHandler}]{\sphinxcrossref{\sphinxcode{\sphinxupquote{StaticFileHandler}}}}},
and \sphinxcode{\sphinxupquote{path}} is the result of {\hyperref[\detokenize{web:tornado.web.StaticFileHandler.get_absolute_path}]{\sphinxcrossref{\sphinxcode{\sphinxupquote{get\_absolute\_path}}}}}

This is an instance method called during request processing,
so it may raise {\hyperref[\detokenize{web:tornado.web.HTTPError}]{\sphinxcrossref{\sphinxcode{\sphinxupquote{HTTPError}}}}} or use methods like
{\hyperref[\detokenize{web:tornado.web.RequestHandler.redirect}]{\sphinxcrossref{\sphinxcode{\sphinxupquote{RequestHandler.redirect}}}}} (return None after redirecting to
halt further processing).  This is where 404 errors for missing files
are generated.

This method may modify the path before returning it, but note that
any such modifications will not be understood by {\hyperref[\detokenize{web:tornado.web.StaticFileHandler.make_static_url}]{\sphinxcrossref{\sphinxcode{\sphinxupquote{make\_static\_url}}}}}.

In instance methods, this method’s result is available as
\sphinxcode{\sphinxupquote{self.absolute\_path}}.

\DUrole{versionmodified,added}{New in version 3.1.}

\end{fulllineitems}

\index{get\_content() (tornado.web.StaticFileHandler class method)@\spxentry{get\_content()}\spxextra{tornado.web.StaticFileHandler class method}}

\begin{fulllineitems}
\phantomsection\label{\detokenize{web:tornado.web.StaticFileHandler.get_content}}\pysiglinewithargsret{\sphinxbfcode{\sphinxupquote{classmethod }}\sphinxbfcode{\sphinxupquote{get\_content}}}{\emph{abspath: str}, \emph{start: int = None}, \emph{end: int = None}}{{ $\rightarrow$ Generator{[}bytes, None, None{]}}}
Retrieve the content of the requested resource which is located
at the given absolute path.

This class method may be overridden by subclasses.  Note that its
signature is different from other overridable class methods
(no \sphinxcode{\sphinxupquote{settings}} argument); this is deliberate to ensure that
\sphinxcode{\sphinxupquote{abspath}} is able to stand on its own as a cache key.

This method should either return a byte string or an iterator
of byte strings.  The latter is preferred for large files
as it helps reduce memory fragmentation.

\DUrole{versionmodified,added}{New in version 3.1.}

\end{fulllineitems}

\index{get\_content\_version() (tornado.web.StaticFileHandler class method)@\spxentry{get\_content\_version()}\spxextra{tornado.web.StaticFileHandler class method}}

\begin{fulllineitems}
\phantomsection\label{\detokenize{web:tornado.web.StaticFileHandler.get_content_version}}\pysiglinewithargsret{\sphinxbfcode{\sphinxupquote{classmethod }}\sphinxbfcode{\sphinxupquote{get\_content\_version}}}{\emph{abspath: str}}{{ $\rightarrow$ str}}
Returns a version string for the resource at the given path.

This class method may be overridden by subclasses.  The
default implementation is a hash of the file’s contents.

\DUrole{versionmodified,added}{New in version 3.1.}

\end{fulllineitems}

\index{get\_content\_size() (tornado.web.StaticFileHandler method)@\spxentry{get\_content\_size()}\spxextra{tornado.web.StaticFileHandler method}}

\begin{fulllineitems}
\phantomsection\label{\detokenize{web:tornado.web.StaticFileHandler.get_content_size}}\pysiglinewithargsret{\sphinxbfcode{\sphinxupquote{get\_content\_size}}}{}{{ $\rightarrow$ int}}
Retrieve the total size of the resource at the given path.

This method may be overridden by subclasses.

\DUrole{versionmodified,added}{New in version 3.1.}

\DUrole{versionmodified,changed}{Changed in version 4.0: }This method is now always called, instead of only when
partial results are requested.

\end{fulllineitems}

\index{get\_modified\_time() (tornado.web.StaticFileHandler method)@\spxentry{get\_modified\_time()}\spxextra{tornado.web.StaticFileHandler method}}

\begin{fulllineitems}
\phantomsection\label{\detokenize{web:tornado.web.StaticFileHandler.get_modified_time}}\pysiglinewithargsret{\sphinxbfcode{\sphinxupquote{get\_modified\_time}}}{}{{ $\rightarrow$ Optional{[}datetime.datetime{]}}}
Returns the time that \sphinxcode{\sphinxupquote{self.absolute\_path}} was last modified.

May be overridden in subclasses.  Should return a \sphinxhref{https://docs.python.org/3.6/library/datetime.html\#datetime.datetime}{\sphinxcode{\sphinxupquote{datetime}}}
object or None.

\DUrole{versionmodified,added}{New in version 3.1.}

\end{fulllineitems}

\index{get\_content\_type() (tornado.web.StaticFileHandler method)@\spxentry{get\_content\_type()}\spxextra{tornado.web.StaticFileHandler method}}

\begin{fulllineitems}
\phantomsection\label{\detokenize{web:tornado.web.StaticFileHandler.get_content_type}}\pysiglinewithargsret{\sphinxbfcode{\sphinxupquote{get\_content\_type}}}{}{{ $\rightarrow$ str}}
Returns the \sphinxcode{\sphinxupquote{Content-Type}} header to be used for this request.

\DUrole{versionmodified,added}{New in version 3.1.}

\end{fulllineitems}

\index{set\_extra\_headers() (tornado.web.StaticFileHandler method)@\spxentry{set\_extra\_headers()}\spxextra{tornado.web.StaticFileHandler method}}

\begin{fulllineitems}
\phantomsection\label{\detokenize{web:tornado.web.StaticFileHandler.set_extra_headers}}\pysiglinewithargsret{\sphinxbfcode{\sphinxupquote{set\_extra\_headers}}}{\emph{path: str}}{{ $\rightarrow$ None}}
For subclass to add extra headers to the response

\end{fulllineitems}

\index{get\_cache\_time() (tornado.web.StaticFileHandler method)@\spxentry{get\_cache\_time()}\spxextra{tornado.web.StaticFileHandler method}}

\begin{fulllineitems}
\phantomsection\label{\detokenize{web:tornado.web.StaticFileHandler.get_cache_time}}\pysiglinewithargsret{\sphinxbfcode{\sphinxupquote{get\_cache\_time}}}{\emph{path: str, modified: Optional{[}datetime.datetime{]}, mime\_type: str}}{{ $\rightarrow$ int}}
Override to customize cache control behavior.

Return a positive number of seconds to make the result
cacheable for that amount of time or 0 to mark resource as
cacheable for an unspecified amount of time (subject to
browser heuristics).

By default returns cache expiry of 10 years for resources requested
with \sphinxcode{\sphinxupquote{v}} argument.

\end{fulllineitems}

\index{make\_static\_url() (tornado.web.StaticFileHandler class method)@\spxentry{make\_static\_url()}\spxextra{tornado.web.StaticFileHandler class method}}

\begin{fulllineitems}
\phantomsection\label{\detokenize{web:tornado.web.StaticFileHandler.make_static_url}}\pysiglinewithargsret{\sphinxbfcode{\sphinxupquote{classmethod }}\sphinxbfcode{\sphinxupquote{make\_static\_url}}}{\emph{settings: Dict{[}str, Any{]}, path: str, include\_version: bool = True}}{{ $\rightarrow$ str}}
Constructs a versioned url for the given path.

This method may be overridden in subclasses (but note that it
is a class method rather than an instance method).  Subclasses
are only required to implement the signature
\sphinxcode{\sphinxupquote{make\_static\_url(cls, settings, path)}}; other keyword
arguments may be passed through {\hyperref[\detokenize{web:tornado.web.RequestHandler.static_url}]{\sphinxcrossref{\sphinxcode{\sphinxupquote{static\_url}}}}}
but are not standard.

\sphinxcode{\sphinxupquote{settings}} is the {\hyperref[\detokenize{web:tornado.web.Application.settings}]{\sphinxcrossref{\sphinxcode{\sphinxupquote{Application.settings}}}}} dictionary.  \sphinxcode{\sphinxupquote{path}}
is the static path being requested.  The url returned should be
relative to the current host.

\sphinxcode{\sphinxupquote{include\_version}} determines whether the generated URL should
include the query string containing the version hash of the
file corresponding to the given \sphinxcode{\sphinxupquote{path}}.

\end{fulllineitems}

\index{parse\_url\_path() (tornado.web.StaticFileHandler method)@\spxentry{parse\_url\_path()}\spxextra{tornado.web.StaticFileHandler method}}

\begin{fulllineitems}
\phantomsection\label{\detokenize{web:tornado.web.StaticFileHandler.parse_url_path}}\pysiglinewithargsret{\sphinxbfcode{\sphinxupquote{parse\_url\_path}}}{\emph{url\_path: str}}{{ $\rightarrow$ str}}
Converts a static URL path into a filesystem path.

\sphinxcode{\sphinxupquote{url\_path}} is the path component of the URL with
\sphinxcode{\sphinxupquote{static\_url\_prefix}} removed.  The return value should be
filesystem path relative to \sphinxcode{\sphinxupquote{static\_path}}.

This is the inverse of {\hyperref[\detokenize{web:tornado.web.StaticFileHandler.make_static_url}]{\sphinxcrossref{\sphinxcode{\sphinxupquote{make\_static\_url}}}}}.

\end{fulllineitems}

\index{get\_version() (tornado.web.StaticFileHandler class method)@\spxentry{get\_version()}\spxextra{tornado.web.StaticFileHandler class method}}

\begin{fulllineitems}
\phantomsection\label{\detokenize{web:tornado.web.StaticFileHandler.get_version}}\pysiglinewithargsret{\sphinxbfcode{\sphinxupquote{classmethod }}\sphinxbfcode{\sphinxupquote{get\_version}}}{\emph{settings: Dict{[}str, Any{]}, path: str}}{{ $\rightarrow$ Optional{[}str{]}}}
Generate the version string to be used in static URLs.

\sphinxcode{\sphinxupquote{settings}} is the {\hyperref[\detokenize{web:tornado.web.Application.settings}]{\sphinxcrossref{\sphinxcode{\sphinxupquote{Application.settings}}}}} dictionary and \sphinxcode{\sphinxupquote{path}}
is the relative location of the requested asset on the filesystem.
The returned value should be a string, or \sphinxcode{\sphinxupquote{None}} if no version
could be determined.

\DUrole{versionmodified,changed}{Changed in version 3.1: }This method was previously recommended for subclasses to override;
{\hyperref[\detokenize{web:tornado.web.StaticFileHandler.get_content_version}]{\sphinxcrossref{\sphinxcode{\sphinxupquote{get\_content\_version}}}}} is now preferred as it allows the base
class to handle caching of the result.

\end{fulllineitems}


\end{fulllineitems}



\subsection{\sphinxstyleliteralintitle{\sphinxupquote{tornado.template}} — Flexible output generation}
\label{\detokenize{template:module-tornado.template}}\label{\detokenize{template:tornado-template-flexible-output-generation}}\label{\detokenize{template::doc}}\index{tornado.template (module)@\spxentry{tornado.template}\spxextra{module}}
A simple template system that compiles templates to Python code.

Basic usage looks like:

\begin{sphinxVerbatim}[commandchars=\\\{\}]
\PYG{n}{t} \PYG{o}{=} \PYG{n}{template}\PYG{o}{.}\PYG{n}{Template}\PYG{p}{(}\PYG{l+s+s2}{\PYGZdq{}}\PYG{l+s+s2}{\PYGZlt{}html\PYGZgt{}}\PYG{l+s+s2}{\PYGZob{}\PYGZob{}}\PYG{l+s+s2}{ myvalue \PYGZcb{}\PYGZcb{}\PYGZlt{}/html\PYGZgt{}}\PYG{l+s+s2}{\PYGZdq{}}\PYG{p}{)}
\PYG{n+nb}{print}\PYG{p}{(}\PYG{n}{t}\PYG{o}{.}\PYG{n}{generate}\PYG{p}{(}\PYG{n}{myvalue}\PYG{o}{=}\PYG{l+s+s2}{\PYGZdq{}}\PYG{l+s+s2}{XXX}\PYG{l+s+s2}{\PYGZdq{}}\PYG{p}{)}\PYG{p}{)}
\end{sphinxVerbatim}

{\hyperref[\detokenize{template:tornado.template.Loader}]{\sphinxcrossref{\sphinxcode{\sphinxupquote{Loader}}}}} is a class that loads templates from a root directory and caches
the compiled templates:

\begin{sphinxVerbatim}[commandchars=\\\{\}]
\PYG{n}{loader} \PYG{o}{=} \PYG{n}{template}\PYG{o}{.}\PYG{n}{Loader}\PYG{p}{(}\PYG{l+s+s2}{\PYGZdq{}}\PYG{l+s+s2}{/home/btaylor}\PYG{l+s+s2}{\PYGZdq{}}\PYG{p}{)}
\PYG{n+nb}{print}\PYG{p}{(}\PYG{n}{loader}\PYG{o}{.}\PYG{n}{load}\PYG{p}{(}\PYG{l+s+s2}{\PYGZdq{}}\PYG{l+s+s2}{test.html}\PYG{l+s+s2}{\PYGZdq{}}\PYG{p}{)}\PYG{o}{.}\PYG{n}{generate}\PYG{p}{(}\PYG{n}{myvalue}\PYG{o}{=}\PYG{l+s+s2}{\PYGZdq{}}\PYG{l+s+s2}{XXX}\PYG{l+s+s2}{\PYGZdq{}}\PYG{p}{)}\PYG{p}{)}
\end{sphinxVerbatim}

We compile all templates to raw Python. Error-reporting is currently… uh,
interesting. Syntax for the templates:

\begin{sphinxVerbatim}[commandchars=\\\{\}]
\PYG{c+c1}{\PYGZsh{}\PYGZsh{}\PYGZsh{} base.html}
\PYG{o}{\PYGZlt{}}\PYG{n}{html}\PYG{o}{\PYGZgt{}}
  \PYG{o}{\PYGZlt{}}\PYG{n}{head}\PYG{o}{\PYGZgt{}}
    \PYG{o}{\PYGZlt{}}\PYG{n}{title}\PYG{o}{\PYGZgt{}}\PYG{p}{\PYGZob{}}\PYG{o}{\PYGZpc{}} \PYG{n}{block} \PYG{n}{title} \PYG{o}{\PYGZpc{}}\PYG{p}{\PYGZcb{}}\PYG{n}{Default} \PYG{n}{title}\PYG{p}{\PYGZob{}}\PYG{o}{\PYGZpc{}} \PYG{n}{end} \PYG{o}{\PYGZpc{}}\PYG{p}{\PYGZcb{}}\PYG{o}{\PYGZlt{}}\PYG{o}{/}\PYG{n}{title}\PYG{o}{\PYGZgt{}}
  \PYG{o}{\PYGZlt{}}\PYG{o}{/}\PYG{n}{head}\PYG{o}{\PYGZgt{}}
  \PYG{o}{\PYGZlt{}}\PYG{n}{body}\PYG{o}{\PYGZgt{}}
    \PYG{o}{\PYGZlt{}}\PYG{n}{ul}\PYG{o}{\PYGZgt{}}
      \PYG{p}{\PYGZob{}}\PYG{o}{\PYGZpc{}} \PYG{k}{for} \PYG{n}{student} \PYG{o+ow}{in} \PYG{n}{students} \PYG{o}{\PYGZpc{}}\PYG{p}{\PYGZcb{}}
        \PYG{p}{\PYGZob{}}\PYG{o}{\PYGZpc{}} \PYG{n}{block} \PYG{n}{student} \PYG{o}{\PYGZpc{}}\PYG{p}{\PYGZcb{}}
          \PYG{o}{\PYGZlt{}}\PYG{n}{li}\PYG{o}{\PYGZgt{}}\PYG{p}{\PYGZob{}}\PYG{p}{\PYGZob{}} \PYG{n}{escape}\PYG{p}{(}\PYG{n}{student}\PYG{o}{.}\PYG{n}{name}\PYG{p}{)} \PYG{p}{\PYGZcb{}}\PYG{p}{\PYGZcb{}}\PYG{o}{\PYGZlt{}}\PYG{o}{/}\PYG{n}{li}\PYG{o}{\PYGZgt{}}
        \PYG{p}{\PYGZob{}}\PYG{o}{\PYGZpc{}} \PYG{n}{end} \PYG{o}{\PYGZpc{}}\PYG{p}{\PYGZcb{}}
      \PYG{p}{\PYGZob{}}\PYG{o}{\PYGZpc{}} \PYG{n}{end} \PYG{o}{\PYGZpc{}}\PYG{p}{\PYGZcb{}}
    \PYG{o}{\PYGZlt{}}\PYG{o}{/}\PYG{n}{ul}\PYG{o}{\PYGZgt{}}
  \PYG{o}{\PYGZlt{}}\PYG{o}{/}\PYG{n}{body}\PYG{o}{\PYGZgt{}}
\PYG{o}{\PYGZlt{}}\PYG{o}{/}\PYG{n}{html}\PYG{o}{\PYGZgt{}}

\PYG{c+c1}{\PYGZsh{}\PYGZsh{}\PYGZsh{} bold.html}
\PYG{p}{\PYGZob{}}\PYG{o}{\PYGZpc{}} \PYG{n}{extends} \PYG{l+s+s2}{\PYGZdq{}}\PYG{l+s+s2}{base.html}\PYG{l+s+s2}{\PYGZdq{}} \PYG{o}{\PYGZpc{}}\PYG{p}{\PYGZcb{}}

\PYG{p}{\PYGZob{}}\PYG{o}{\PYGZpc{}} \PYG{n}{block} \PYG{n}{title} \PYG{o}{\PYGZpc{}}\PYG{p}{\PYGZcb{}}\PYG{n}{A} \PYG{n}{bolder} \PYG{n}{title}\PYG{p}{\PYGZob{}}\PYG{o}{\PYGZpc{}} \PYG{n}{end} \PYG{o}{\PYGZpc{}}\PYG{p}{\PYGZcb{}}

\PYG{p}{\PYGZob{}}\PYG{o}{\PYGZpc{}} \PYG{n}{block} \PYG{n}{student} \PYG{o}{\PYGZpc{}}\PYG{p}{\PYGZcb{}}
  \PYG{o}{\PYGZlt{}}\PYG{n}{li}\PYG{o}{\PYGZgt{}}\PYG{o}{\PYGZlt{}}\PYG{n}{span} \PYG{n}{style}\PYG{o}{=}\PYG{l+s+s2}{\PYGZdq{}}\PYG{l+s+s2}{bold}\PYG{l+s+s2}{\PYGZdq{}}\PYG{o}{\PYGZgt{}}\PYG{p}{\PYGZob{}}\PYG{p}{\PYGZob{}} \PYG{n}{escape}\PYG{p}{(}\PYG{n}{student}\PYG{o}{.}\PYG{n}{name}\PYG{p}{)} \PYG{p}{\PYGZcb{}}\PYG{p}{\PYGZcb{}}\PYG{o}{\PYGZlt{}}\PYG{o}{/}\PYG{n}{span}\PYG{o}{\PYGZgt{}}\PYG{o}{\PYGZlt{}}\PYG{o}{/}\PYG{n}{li}\PYG{o}{\PYGZgt{}}
\PYG{p}{\PYGZob{}}\PYG{o}{\PYGZpc{}} \PYG{n}{end} \PYG{o}{\PYGZpc{}}\PYG{p}{\PYGZcb{}}
\end{sphinxVerbatim}

Unlike most other template systems, we do not put any restrictions on the
expressions you can include in your statements. \sphinxcode{\sphinxupquote{if}} and \sphinxcode{\sphinxupquote{for}} blocks get
translated exactly into Python, so you can do complex expressions like:

\begin{sphinxVerbatim}[commandchars=\\\{\}]
\PYG{p}{\PYGZob{}}\PYG{o}{\PYGZpc{}} \PYG{k}{for} \PYG{n}{student} \PYG{o+ow}{in} \PYG{p}{[}\PYG{n}{p} \PYG{k}{for} \PYG{n}{p} \PYG{o+ow}{in} \PYG{n}{people} \PYG{k}{if} \PYG{n}{p}\PYG{o}{.}\PYG{n}{student} \PYG{o+ow}{and} \PYG{n}{p}\PYG{o}{.}\PYG{n}{age} \PYG{o}{\PYGZgt{}} \PYG{l+m+mi}{23}\PYG{p}{]} \PYG{o}{\PYGZpc{}}\PYG{p}{\PYGZcb{}}
  \PYG{o}{\PYGZlt{}}\PYG{n}{li}\PYG{o}{\PYGZgt{}}\PYG{p}{\PYGZob{}}\PYG{p}{\PYGZob{}} \PYG{n}{escape}\PYG{p}{(}\PYG{n}{student}\PYG{o}{.}\PYG{n}{name}\PYG{p}{)} \PYG{p}{\PYGZcb{}}\PYG{p}{\PYGZcb{}}\PYG{o}{\PYGZlt{}}\PYG{o}{/}\PYG{n}{li}\PYG{o}{\PYGZgt{}}
\PYG{p}{\PYGZob{}}\PYG{o}{\PYGZpc{}} \PYG{n}{end} \PYG{o}{\PYGZpc{}}\PYG{p}{\PYGZcb{}}
\end{sphinxVerbatim}

Translating directly to Python means you can apply functions to expressions
easily, like the \sphinxcode{\sphinxupquote{escape()}} function in the examples above. You can pass
functions in to your template just like any other variable
(In a {\hyperref[\detokenize{web:tornado.web.RequestHandler}]{\sphinxcrossref{\sphinxcode{\sphinxupquote{RequestHandler}}}}}, override {\hyperref[\detokenize{web:tornado.web.RequestHandler.get_template_namespace}]{\sphinxcrossref{\sphinxcode{\sphinxupquote{RequestHandler.get\_template\_namespace}}}}}):

\begin{sphinxVerbatim}[commandchars=\\\{\}]
\PYG{c+c1}{\PYGZsh{}\PYGZsh{}\PYGZsh{} Python code}
\PYG{k}{def} \PYG{n+nf}{add}\PYG{p}{(}\PYG{n}{x}\PYG{p}{,} \PYG{n}{y}\PYG{p}{)}\PYG{p}{:}
   \PYG{k}{return} \PYG{n}{x} \PYG{o}{+} \PYG{n}{y}
\PYG{n}{template}\PYG{o}{.}\PYG{n}{execute}\PYG{p}{(}\PYG{n}{add}\PYG{o}{=}\PYG{n}{add}\PYG{p}{)}

\PYG{c+c1}{\PYGZsh{}\PYGZsh{}\PYGZsh{} The template}
\PYG{p}{\PYGZob{}}\PYG{p}{\PYGZob{}} \PYG{n}{add}\PYG{p}{(}\PYG{l+m+mi}{1}\PYG{p}{,} \PYG{l+m+mi}{2}\PYG{p}{)} \PYG{p}{\PYGZcb{}}\PYG{p}{\PYGZcb{}}
\end{sphinxVerbatim}

We provide the functions {\hyperref[\detokenize{escape:tornado.escape.xhtml_escape}]{\sphinxcrossref{\sphinxcode{\sphinxupquote{escape()}}}}}, {\hyperref[\detokenize{escape:tornado.escape.url_escape}]{\sphinxcrossref{\sphinxcode{\sphinxupquote{url\_escape()}}}}},
{\hyperref[\detokenize{escape:tornado.escape.json_encode}]{\sphinxcrossref{\sphinxcode{\sphinxupquote{json\_encode()}}}}}, and {\hyperref[\detokenize{escape:tornado.escape.squeeze}]{\sphinxcrossref{\sphinxcode{\sphinxupquote{squeeze()}}}}} to all templates by default.

Typical applications do not create {\hyperref[\detokenize{template:tornado.template.Template}]{\sphinxcrossref{\sphinxcode{\sphinxupquote{Template}}}}} or {\hyperref[\detokenize{template:tornado.template.Loader}]{\sphinxcrossref{\sphinxcode{\sphinxupquote{Loader}}}}} instances by
hand, but instead use the {\hyperref[\detokenize{web:tornado.web.RequestHandler.render}]{\sphinxcrossref{\sphinxcode{\sphinxupquote{render}}}}} and
{\hyperref[\detokenize{web:tornado.web.RequestHandler.render_string}]{\sphinxcrossref{\sphinxcode{\sphinxupquote{render\_string}}}}} methods of
{\hyperref[\detokenize{web:tornado.web.RequestHandler}]{\sphinxcrossref{\sphinxcode{\sphinxupquote{tornado.web.RequestHandler}}}}}, which load templates automatically based
on the \sphinxcode{\sphinxupquote{template\_path}} {\hyperref[\detokenize{web:tornado.web.Application}]{\sphinxcrossref{\sphinxcode{\sphinxupquote{Application}}}}} setting.

Variable names beginning with \sphinxcode{\sphinxupquote{\_tt\_}} are reserved by the template
system and should not be used by application code.


\subsubsection{Syntax Reference}
\label{\detokenize{template:syntax-reference}}
Template expressions are surrounded by double curly braces: \sphinxcode{\sphinxupquote{\{\{ ... \}\}}}.
The contents may be any python expression, which will be escaped according
to the current autoescape setting and inserted into the output.  Other
template directives use \sphinxcode{\sphinxupquote{\{\% \%\}}}.

To comment out a section so that it is omitted from the output, surround it
with \sphinxcode{\sphinxupquote{\{\# ... \#\}}}.

These tags may be escaped as \sphinxcode{\sphinxupquote{\{\{!}}, \sphinxcode{\sphinxupquote{\{\%!}}, and \sphinxcode{\sphinxupquote{\{\#!}}
if you need to include a literal \sphinxcode{\sphinxupquote{\{\{}}, \sphinxcode{\sphinxupquote{\{\%}}, or \sphinxcode{\sphinxupquote{\{\#}} in the output.
\begin{description}
\item[{\sphinxcode{\sphinxupquote{\{\% apply *function* \%\}...\{\% end \%\}}}}] \leavevmode
Applies a function to the output of all template code between \sphinxcode{\sphinxupquote{apply}}
and \sphinxcode{\sphinxupquote{end}}:

\begin{sphinxVerbatim}[commandchars=\\\{\}]
\PYG{p}{\PYGZob{}}\PYG{o}{\PYGZpc{}} \PYG{n}{apply} \PYG{n}{linkify} \PYG{o}{\PYGZpc{}}\PYG{p}{\PYGZcb{}}\PYG{p}{\PYGZob{}}\PYG{p}{\PYGZob{}}\PYG{n}{name}\PYG{p}{\PYGZcb{}}\PYG{p}{\PYGZcb{}} \PYG{n}{said}\PYG{p}{:} \PYG{p}{\PYGZob{}}\PYG{p}{\PYGZob{}}\PYG{n}{message}\PYG{p}{\PYGZcb{}}\PYG{p}{\PYGZcb{}}\PYG{p}{\PYGZob{}}\PYG{o}{\PYGZpc{}} \PYG{n}{end} \PYG{o}{\PYGZpc{}}\PYG{p}{\PYGZcb{}}
\end{sphinxVerbatim}

Note that as an implementation detail apply blocks are implemented
as nested functions and thus may interact strangely with variables
set via \sphinxcode{\sphinxupquote{\{\% set \%\}}}, or the use of \sphinxcode{\sphinxupquote{\{\% break \%\}}} or \sphinxcode{\sphinxupquote{\{\% continue \%\}}}
within loops.

\item[{\sphinxcode{\sphinxupquote{\{\% autoescape *function* \%\}}}}] \leavevmode
Sets the autoescape mode for the current file.  This does not affect
other files, even those referenced by \sphinxcode{\sphinxupquote{\{\% include \%\}}}.  Note that
autoescaping can also be configured globally, at the {\hyperref[\detokenize{web:tornado.web.Application}]{\sphinxcrossref{\sphinxcode{\sphinxupquote{Application}}}}}
or {\hyperref[\detokenize{template:tornado.template.Loader}]{\sphinxcrossref{\sphinxcode{\sphinxupquote{Loader}}}}}.:

\begin{sphinxVerbatim}[commandchars=\\\{\}]
\PYG{p}{\PYGZob{}}\PYG{o}{\PYGZpc{}} \PYG{n}{autoescape} \PYG{n}{xhtml\PYGZus{}escape} \PYG{o}{\PYGZpc{}}\PYG{p}{\PYGZcb{}}
\PYG{p}{\PYGZob{}}\PYG{o}{\PYGZpc{}} \PYG{n}{autoescape} \PYG{k+kc}{None} \PYG{o}{\PYGZpc{}}\PYG{p}{\PYGZcb{}}
\end{sphinxVerbatim}

\item[{\sphinxcode{\sphinxupquote{\{\% block *name* \%\}...\{\% end \%\}}}}] \leavevmode
Indicates a named, replaceable block for use with \sphinxcode{\sphinxupquote{\{\% extends \%\}}}.
Blocks in the parent template will be replaced with the contents of
the same-named block in a child template.:

\begin{sphinxVerbatim}[commandchars=\\\{\}]
\PYGZlt{}!\PYGZhy{}\PYGZhy{} base.html \PYGZhy{}\PYGZhy{}\PYGZgt{}
\PYGZlt{}title\PYGZgt{}\PYGZob{}\PYGZpc{} block title \PYGZpc{}\PYGZcb{}Default title\PYGZob{}\PYGZpc{} end \PYGZpc{}\PYGZcb{}\PYGZlt{}/title\PYGZgt{}

\PYGZlt{}!\PYGZhy{}\PYGZhy{} mypage.html \PYGZhy{}\PYGZhy{}\PYGZgt{}
\PYGZob{}\PYGZpc{} extends \PYGZdq{}base.html\PYGZdq{} \PYGZpc{}\PYGZcb{}
\PYGZob{}\PYGZpc{} block title \PYGZpc{}\PYGZcb{}My page title\PYGZob{}\PYGZpc{} end \PYGZpc{}\PYGZcb{}
\end{sphinxVerbatim}

\item[{\sphinxcode{\sphinxupquote{\{\% comment ... \%\}}}}] \leavevmode
A comment which will be removed from the template output.  Note that
there is no \sphinxcode{\sphinxupquote{\{\% end \%\}}} tag; the comment goes from the word \sphinxcode{\sphinxupquote{comment}}
to the closing \sphinxcode{\sphinxupquote{\%\}}} tag.

\item[{\sphinxcode{\sphinxupquote{\{\% extends *filename* \%\}}}}] \leavevmode
Inherit from another template.  Templates that use \sphinxcode{\sphinxupquote{extends}} should
contain one or more \sphinxcode{\sphinxupquote{block}} tags to replace content from the parent
template.  Anything in the child template not contained in a \sphinxcode{\sphinxupquote{block}}
tag will be ignored.  For an example, see the \sphinxcode{\sphinxupquote{\{\% block \%\}}} tag.

\item[{\sphinxcode{\sphinxupquote{\{\% for *var* in *expr* \%\}...\{\% end \%\}}}}] \leavevmode
Same as the python \sphinxcode{\sphinxupquote{for}} statement.  \sphinxcode{\sphinxupquote{\{\% break \%\}}} and
\sphinxcode{\sphinxupquote{\{\% continue \%\}}} may be used inside the loop.

\item[{\sphinxcode{\sphinxupquote{\{\% from *x* import *y* \%\}}}}] \leavevmode
Same as the python \sphinxcode{\sphinxupquote{import}} statement.

\item[{\sphinxcode{\sphinxupquote{\{\% if *condition* \%\}...\{\% elif *condition* \%\}...\{\% else \%\}...\{\% end \%\}}}}] \leavevmode
Conditional statement - outputs the first section whose condition is
true.  (The \sphinxcode{\sphinxupquote{elif}} and \sphinxcode{\sphinxupquote{else}} sections are optional)

\item[{\sphinxcode{\sphinxupquote{\{\% import *module* \%\}}}}] \leavevmode
Same as the python \sphinxcode{\sphinxupquote{import}} statement.

\item[{\sphinxcode{\sphinxupquote{\{\% include *filename* \%\}}}}] \leavevmode
Includes another template file.  The included file can see all the local
variables as if it were copied directly to the point of the \sphinxcode{\sphinxupquote{include}}
directive (the \sphinxcode{\sphinxupquote{\{\% autoescape \%\}}} directive is an exception).
Alternately, \sphinxcode{\sphinxupquote{\{\% module Template(filename, **kwargs) \%\}}} may be used
to include another template with an isolated namespace.

\item[{\sphinxcode{\sphinxupquote{\{\% module *expr* \%\}}}}] \leavevmode
Renders a {\hyperref[\detokenize{web:tornado.web.UIModule}]{\sphinxcrossref{\sphinxcode{\sphinxupquote{UIModule}}}}}.  The output of the \sphinxcode{\sphinxupquote{UIModule}} is
not escaped:

\begin{sphinxVerbatim}[commandchars=\\\{\}]
\PYG{p}{\PYGZob{}}\PYG{o}{\PYGZpc{}} \PYG{n}{module} \PYG{n}{Template}\PYG{p}{(}\PYG{l+s+s2}{\PYGZdq{}}\PYG{l+s+s2}{foo.html}\PYG{l+s+s2}{\PYGZdq{}}\PYG{p}{,} \PYG{n}{arg}\PYG{o}{=}\PYG{l+m+mi}{42}\PYG{p}{)} \PYG{o}{\PYGZpc{}}\PYG{p}{\PYGZcb{}}
\end{sphinxVerbatim}

\sphinxcode{\sphinxupquote{UIModules}} are a feature of the {\hyperref[\detokenize{web:tornado.web.RequestHandler}]{\sphinxcrossref{\sphinxcode{\sphinxupquote{tornado.web.RequestHandler}}}}}
class (and specifically its \sphinxcode{\sphinxupquote{render}} method) and will not work
when the template system is used on its own in other contexts.

\item[{\sphinxcode{\sphinxupquote{\{\% raw *expr* \%\}}}}] \leavevmode
Outputs the result of the given expression without autoescaping.

\item[{\sphinxcode{\sphinxupquote{\{\% set *x* = *y* \%\}}}}] \leavevmode
Sets a local variable.

\item[{\sphinxcode{\sphinxupquote{\{\% try \%\}...\{\% except \%\}...\{\% else \%\}...\{\% finally \%\}...\{\% end \%\}}}}] \leavevmode
Same as the python \sphinxcode{\sphinxupquote{try}} statement.

\item[{\sphinxcode{\sphinxupquote{\{\% while *condition* \%\}... \{\% end \%\}}}}] \leavevmode
Same as the python \sphinxcode{\sphinxupquote{while}} statement.  \sphinxcode{\sphinxupquote{\{\% break \%\}}} and
\sphinxcode{\sphinxupquote{\{\% continue \%\}}} may be used inside the loop.

\item[{\sphinxcode{\sphinxupquote{\{\% whitespace *mode* \%\}}}}] \leavevmode
Sets the whitespace mode for the remainder of the current file
(or until the next \sphinxcode{\sphinxupquote{\{\% whitespace \%\}}} directive). See
{\hyperref[\detokenize{template:tornado.template.filter_whitespace}]{\sphinxcrossref{\sphinxcode{\sphinxupquote{filter\_whitespace}}}}} for available options. New in Tornado 4.3.

\end{description}


\subsubsection{Class reference}
\label{\detokenize{template:class-reference}}\index{Template (class in tornado.template)@\spxentry{Template}\spxextra{class in tornado.template}}

\begin{fulllineitems}
\phantomsection\label{\detokenize{template:tornado.template.Template}}\pysiglinewithargsret{\sphinxbfcode{\sphinxupquote{class }}\sphinxcode{\sphinxupquote{tornado.template.}}\sphinxbfcode{\sphinxupquote{Template}}}{\emph{template\_string}, \emph{name="\textless{}string\textgreater{}"}, \emph{loader=None}, \emph{compress\_whitespace=None}, \emph{autoescape="xhtml\_escape"}, \emph{whitespace=None}}{}
A compiled template.

We compile into Python from the given template\_string. You can generate
the template from variables with generate().

Construct a Template.
\begin{quote}\begin{description}
\item[{Parameters}] \leavevmode\begin{itemize}
\item {} 
\sphinxstyleliteralstrong{\sphinxupquote{template\_string}} (\sphinxhref{https://docs.python.org/3.6/library/stdtypes.html\#str}{\sphinxstyleliteralemphasis{\sphinxupquote{str}}}) \textendash{} the contents of the template file.

\item {} 
\sphinxstyleliteralstrong{\sphinxupquote{name}} (\sphinxhref{https://docs.python.org/3.6/library/stdtypes.html\#str}{\sphinxstyleliteralemphasis{\sphinxupquote{str}}}) \textendash{} the filename from which the template was loaded
(used for error message).

\item {} 
\sphinxstyleliteralstrong{\sphinxupquote{loader}} ({\hyperref[\detokenize{template:tornado.template.BaseLoader}]{\sphinxcrossref{\sphinxstyleliteralemphasis{\sphinxupquote{tornado.template.BaseLoader}}}}}) \textendash{} the {\hyperref[\detokenize{template:tornado.template.BaseLoader}]{\sphinxcrossref{\sphinxcode{\sphinxupquote{BaseLoader}}}}} responsible
for this template, used to resolve \sphinxcode{\sphinxupquote{\{\% include \%\}}} and \sphinxcode{\sphinxupquote{\{\% extend \%\}}} directives.

\item {} 
\sphinxstyleliteralstrong{\sphinxupquote{compress\_whitespace}} (\sphinxhref{https://docs.python.org/3.6/library/functions.html\#bool}{\sphinxstyleliteralemphasis{\sphinxupquote{bool}}}) \textendash{} Deprecated since Tornado 4.3.
Equivalent to \sphinxcode{\sphinxupquote{whitespace="single"}} if true and
\sphinxcode{\sphinxupquote{whitespace="all"}} if false.

\item {} 
\sphinxstyleliteralstrong{\sphinxupquote{autoescape}} (\sphinxhref{https://docs.python.org/3.6/library/stdtypes.html\#str}{\sphinxstyleliteralemphasis{\sphinxupquote{str}}}) \textendash{} The name of a function in the template
namespace, or \sphinxcode{\sphinxupquote{None}} to disable escaping by default.

\item {} 
\sphinxstyleliteralstrong{\sphinxupquote{whitespace}} (\sphinxhref{https://docs.python.org/3.6/library/stdtypes.html\#str}{\sphinxstyleliteralemphasis{\sphinxupquote{str}}}) \textendash{} A string specifying treatment of whitespace;
see {\hyperref[\detokenize{template:tornado.template.filter_whitespace}]{\sphinxcrossref{\sphinxcode{\sphinxupquote{filter\_whitespace}}}}} for options.

\end{itemize}

\end{description}\end{quote}

\DUrole{versionmodified,changed}{Changed in version 4.3: }Added \sphinxcode{\sphinxupquote{whitespace}} parameter; deprecated \sphinxcode{\sphinxupquote{compress\_whitespace}}.
\index{generate() (tornado.template.Template method)@\spxentry{generate()}\spxextra{tornado.template.Template method}}

\begin{fulllineitems}
\phantomsection\label{\detokenize{template:tornado.template.Template.generate}}\pysiglinewithargsret{\sphinxbfcode{\sphinxupquote{generate}}}{\emph{**kwargs}}{{ $\rightarrow$ bytes}}
Generate this template with the given arguments.

\end{fulllineitems}


\end{fulllineitems}

\index{BaseLoader (class in tornado.template)@\spxentry{BaseLoader}\spxextra{class in tornado.template}}

\begin{fulllineitems}
\phantomsection\label{\detokenize{template:tornado.template.BaseLoader}}\pysiglinewithargsret{\sphinxbfcode{\sphinxupquote{class }}\sphinxcode{\sphinxupquote{tornado.template.}}\sphinxbfcode{\sphinxupquote{BaseLoader}}}{\emph{autoescape: str = 'xhtml\_escape'}, \emph{namespace: Dict{[}str}, \emph{Any{]} = None}, \emph{whitespace: str = None}}{}
Base class for template loaders.

You must use a template loader to use template constructs like
\sphinxcode{\sphinxupquote{\{\% extends \%\}}} and \sphinxcode{\sphinxupquote{\{\% include \%\}}}. The loader caches all
templates after they are loaded the first time.

Construct a template loader.
\begin{quote}\begin{description}
\item[{Parameters}] \leavevmode\begin{itemize}
\item {} 
\sphinxstyleliteralstrong{\sphinxupquote{autoescape}} (\sphinxhref{https://docs.python.org/3.6/library/stdtypes.html\#str}{\sphinxstyleliteralemphasis{\sphinxupquote{str}}}) \textendash{} The name of a function in the template
namespace, such as “xhtml\_escape”, or \sphinxcode{\sphinxupquote{None}} to disable
autoescaping by default.

\item {} 
\sphinxstyleliteralstrong{\sphinxupquote{namespace}} (\sphinxhref{https://docs.python.org/3.6/library/stdtypes.html\#dict}{\sphinxstyleliteralemphasis{\sphinxupquote{dict}}}) \textendash{} A dictionary to be added to the default template
namespace, or \sphinxcode{\sphinxupquote{None}}.

\item {} 
\sphinxstyleliteralstrong{\sphinxupquote{whitespace}} (\sphinxhref{https://docs.python.org/3.6/library/stdtypes.html\#str}{\sphinxstyleliteralemphasis{\sphinxupquote{str}}}) \textendash{} A string specifying default behavior for
whitespace in templates; see {\hyperref[\detokenize{template:tornado.template.filter_whitespace}]{\sphinxcrossref{\sphinxcode{\sphinxupquote{filter\_whitespace}}}}} for options.
Default is “single” for files ending in “.html” and “.js” and
“all” for other files.

\end{itemize}

\end{description}\end{quote}

\DUrole{versionmodified,changed}{Changed in version 4.3: }Added \sphinxcode{\sphinxupquote{whitespace}} parameter.
\index{reset() (tornado.template.BaseLoader method)@\spxentry{reset()}\spxextra{tornado.template.BaseLoader method}}

\begin{fulllineitems}
\phantomsection\label{\detokenize{template:tornado.template.BaseLoader.reset}}\pysiglinewithargsret{\sphinxbfcode{\sphinxupquote{reset}}}{}{{ $\rightarrow$ None}}
Resets the cache of compiled templates.

\end{fulllineitems}

\index{resolve\_path() (tornado.template.BaseLoader method)@\spxentry{resolve\_path()}\spxextra{tornado.template.BaseLoader method}}

\begin{fulllineitems}
\phantomsection\label{\detokenize{template:tornado.template.BaseLoader.resolve_path}}\pysiglinewithargsret{\sphinxbfcode{\sphinxupquote{resolve\_path}}}{\emph{name: str}, \emph{parent\_path: str = None}}{{ $\rightarrow$ str}}
Converts a possibly-relative path to absolute (used internally).

\end{fulllineitems}

\index{load() (tornado.template.BaseLoader method)@\spxentry{load()}\spxextra{tornado.template.BaseLoader method}}

\begin{fulllineitems}
\phantomsection\label{\detokenize{template:tornado.template.BaseLoader.load}}\pysiglinewithargsret{\sphinxbfcode{\sphinxupquote{load}}}{\emph{name: str}, \emph{parent\_path: str = None}}{{ $\rightarrow$ tornado.template.Template}}
Loads a template.

\end{fulllineitems}


\end{fulllineitems}

\index{Loader (class in tornado.template)@\spxentry{Loader}\spxextra{class in tornado.template}}

\begin{fulllineitems}
\phantomsection\label{\detokenize{template:tornado.template.Loader}}\pysiglinewithargsret{\sphinxbfcode{\sphinxupquote{class }}\sphinxcode{\sphinxupquote{tornado.template.}}\sphinxbfcode{\sphinxupquote{Loader}}}{\emph{root\_directory: str}, \emph{**kwargs}}{}
A template loader that loads from a single root directory.

\end{fulllineitems}

\index{DictLoader (class in tornado.template)@\spxentry{DictLoader}\spxextra{class in tornado.template}}

\begin{fulllineitems}
\phantomsection\label{\detokenize{template:tornado.template.DictLoader}}\pysiglinewithargsret{\sphinxbfcode{\sphinxupquote{class }}\sphinxcode{\sphinxupquote{tornado.template.}}\sphinxbfcode{\sphinxupquote{DictLoader}}}{\emph{dict: Dict{[}str, str{]}, **kwargs}}{}
A template loader that loads from a dictionary.

\end{fulllineitems}

\index{ParseError@\spxentry{ParseError}}

\begin{fulllineitems}
\phantomsection\label{\detokenize{template:tornado.template.ParseError}}\pysiglinewithargsret{\sphinxbfcode{\sphinxupquote{exception }}\sphinxcode{\sphinxupquote{tornado.template.}}\sphinxbfcode{\sphinxupquote{ParseError}}}{\emph{message: str}, \emph{filename: str = None}, \emph{lineno: int = 0}}{}
Raised for template syntax errors.

\sphinxcode{\sphinxupquote{ParseError}} instances have \sphinxcode{\sphinxupquote{filename}} and \sphinxcode{\sphinxupquote{lineno}} attributes
indicating the position of the error.

\DUrole{versionmodified,changed}{Changed in version 4.3: }Added \sphinxcode{\sphinxupquote{filename}} and \sphinxcode{\sphinxupquote{lineno}} attributes.

\end{fulllineitems}

\index{filter\_whitespace() (in module tornado.template)@\spxentry{filter\_whitespace()}\spxextra{in module tornado.template}}

\begin{fulllineitems}
\phantomsection\label{\detokenize{template:tornado.template.filter_whitespace}}\pysiglinewithargsret{\sphinxcode{\sphinxupquote{tornado.template.}}\sphinxbfcode{\sphinxupquote{filter\_whitespace}}}{\emph{mode: str}, \emph{text: str}}{{ $\rightarrow$ str}}
Transform whitespace in \sphinxcode{\sphinxupquote{text}} according to \sphinxcode{\sphinxupquote{mode}}.

Available modes are:
\begin{itemize}
\item {} 
\sphinxcode{\sphinxupquote{all}}: Return all whitespace unmodified.

\item {} 
\sphinxcode{\sphinxupquote{single}}: Collapse consecutive whitespace with a single whitespace
character, preserving newlines.

\item {} 
\sphinxcode{\sphinxupquote{oneline}}: Collapse all runs of whitespace into a single space
character, removing all newlines in the process.

\end{itemize}

\DUrole{versionmodified,added}{New in version 4.3.}

\end{fulllineitems}



\subsection{\sphinxstyleliteralintitle{\sphinxupquote{tornado.routing}} — Basic routing implementation}
\label{\detokenize{routing:module-tornado.routing}}\label{\detokenize{routing:tornado-routing-basic-routing-implementation}}\label{\detokenize{routing::doc}}\index{tornado.routing (module)@\spxentry{tornado.routing}\spxextra{module}}
Flexible routing implementation.

Tornado routes HTTP requests to appropriate handlers using {\hyperref[\detokenize{routing:tornado.routing.Router}]{\sphinxcrossref{\sphinxcode{\sphinxupquote{Router}}}}}
class implementations. The {\hyperref[\detokenize{web:tornado.web.Application}]{\sphinxcrossref{\sphinxcode{\sphinxupquote{tornado.web.Application}}}}} class is a
{\hyperref[\detokenize{routing:tornado.routing.Router}]{\sphinxcrossref{\sphinxcode{\sphinxupquote{Router}}}}} implementation and may be used directly, or the classes in
this module may be used for additional flexibility. The {\hyperref[\detokenize{routing:tornado.routing.RuleRouter}]{\sphinxcrossref{\sphinxcode{\sphinxupquote{RuleRouter}}}}}
class can match on more criteria than {\hyperref[\detokenize{web:tornado.web.Application}]{\sphinxcrossref{\sphinxcode{\sphinxupquote{Application}}}}}, or the {\hyperref[\detokenize{routing:tornado.routing.Router}]{\sphinxcrossref{\sphinxcode{\sphinxupquote{Router}}}}}
interface can be subclassed for maximum customization.

{\hyperref[\detokenize{routing:tornado.routing.Router}]{\sphinxcrossref{\sphinxcode{\sphinxupquote{Router}}}}} interface extends {\hyperref[\detokenize{httputil:tornado.httputil.HTTPServerConnectionDelegate}]{\sphinxcrossref{\sphinxcode{\sphinxupquote{HTTPServerConnectionDelegate}}}}}
to provide additional routing capabilities. This also means that any
{\hyperref[\detokenize{routing:tornado.routing.Router}]{\sphinxcrossref{\sphinxcode{\sphinxupquote{Router}}}}} implementation can be used directly as a \sphinxcode{\sphinxupquote{request\_callback}}
for {\hyperref[\detokenize{httpserver:tornado.httpserver.HTTPServer}]{\sphinxcrossref{\sphinxcode{\sphinxupquote{HTTPServer}}}}} constructor.

{\hyperref[\detokenize{routing:tornado.routing.Router}]{\sphinxcrossref{\sphinxcode{\sphinxupquote{Router}}}}} subclass must implement a \sphinxcode{\sphinxupquote{find\_handler}} method to provide
a suitable {\hyperref[\detokenize{httputil:tornado.httputil.HTTPMessageDelegate}]{\sphinxcrossref{\sphinxcode{\sphinxupquote{HTTPMessageDelegate}}}}} instance to handle the
request:

\begin{sphinxVerbatim}[commandchars=\\\{\}]
\PYG{k}{class} \PYG{n+nc}{CustomRouter}\PYG{p}{(}\PYG{n}{Router}\PYG{p}{)}\PYG{p}{:}
    \PYG{k}{def} \PYG{n+nf}{find\PYGZus{}handler}\PYG{p}{(}\PYG{n+nb+bp}{self}\PYG{p}{,} \PYG{n}{request}\PYG{p}{,} \PYG{o}{*}\PYG{o}{*}\PYG{n}{kwargs}\PYG{p}{)}\PYG{p}{:}
        \PYG{c+c1}{\PYGZsh{} some routing logic providing a suitable HTTPMessageDelegate instance}
        \PYG{k}{return} \PYG{n}{MessageDelegate}\PYG{p}{(}\PYG{n}{request}\PYG{o}{.}\PYG{n}{connection}\PYG{p}{)}

\PYG{k}{class} \PYG{n+nc}{MessageDelegate}\PYG{p}{(}\PYG{n}{HTTPMessageDelegate}\PYG{p}{)}\PYG{p}{:}
    \PYG{k}{def} \PYG{n+nf+fm}{\PYGZus{}\PYGZus{}init\PYGZus{}\PYGZus{}}\PYG{p}{(}\PYG{n+nb+bp}{self}\PYG{p}{,} \PYG{n}{connection}\PYG{p}{)}\PYG{p}{:}
        \PYG{n+nb+bp}{self}\PYG{o}{.}\PYG{n}{connection} \PYG{o}{=} \PYG{n}{connection}

    \PYG{k}{def} \PYG{n+nf}{finish}\PYG{p}{(}\PYG{n+nb+bp}{self}\PYG{p}{)}\PYG{p}{:}
        \PYG{n+nb+bp}{self}\PYG{o}{.}\PYG{n}{connection}\PYG{o}{.}\PYG{n}{write\PYGZus{}headers}\PYG{p}{(}
            \PYG{n}{ResponseStartLine}\PYG{p}{(}\PYG{l+s+s2}{\PYGZdq{}}\PYG{l+s+s2}{HTTP/1.1}\PYG{l+s+s2}{\PYGZdq{}}\PYG{p}{,} \PYG{l+m+mi}{200}\PYG{p}{,} \PYG{l+s+s2}{\PYGZdq{}}\PYG{l+s+s2}{OK}\PYG{l+s+s2}{\PYGZdq{}}\PYG{p}{)}\PYG{p}{,}
            \PYG{n}{HTTPHeaders}\PYG{p}{(}\PYG{p}{\PYGZob{}}\PYG{l+s+s2}{\PYGZdq{}}\PYG{l+s+s2}{Content\PYGZhy{}Length}\PYG{l+s+s2}{\PYGZdq{}}\PYG{p}{:} \PYG{l+s+s2}{\PYGZdq{}}\PYG{l+s+s2}{2}\PYG{l+s+s2}{\PYGZdq{}}\PYG{p}{\PYGZcb{}}\PYG{p}{)}\PYG{p}{,}
            \PYG{l+s+sa}{b}\PYG{l+s+s2}{\PYGZdq{}}\PYG{l+s+s2}{OK}\PYG{l+s+s2}{\PYGZdq{}}\PYG{p}{)}
        \PYG{n+nb+bp}{self}\PYG{o}{.}\PYG{n}{connection}\PYG{o}{.}\PYG{n}{finish}\PYG{p}{(}\PYG{p}{)}

\PYG{n}{router} \PYG{o}{=} \PYG{n}{CustomRouter}\PYG{p}{(}\PYG{p}{)}
\PYG{n}{server} \PYG{o}{=} \PYG{n}{HTTPServer}\PYG{p}{(}\PYG{n}{router}\PYG{p}{)}
\end{sphinxVerbatim}

The main responsibility of {\hyperref[\detokenize{routing:tornado.routing.Router}]{\sphinxcrossref{\sphinxcode{\sphinxupquote{Router}}}}} implementation is to provide a
mapping from a request to {\hyperref[\detokenize{httputil:tornado.httputil.HTTPMessageDelegate}]{\sphinxcrossref{\sphinxcode{\sphinxupquote{HTTPMessageDelegate}}}}} instance
that will handle this request. In the example above we can see that
routing is possible even without instantiating an {\hyperref[\detokenize{web:tornado.web.Application}]{\sphinxcrossref{\sphinxcode{\sphinxupquote{Application}}}}}.

For routing to {\hyperref[\detokenize{web:tornado.web.RequestHandler}]{\sphinxcrossref{\sphinxcode{\sphinxupquote{RequestHandler}}}}} implementations we need an
{\hyperref[\detokenize{web:tornado.web.Application}]{\sphinxcrossref{\sphinxcode{\sphinxupquote{Application}}}}} instance. {\hyperref[\detokenize{web:tornado.web.Application.get_handler_delegate}]{\sphinxcrossref{\sphinxcode{\sphinxupquote{get\_handler\_delegate}}}}}
provides a convenient way to create {\hyperref[\detokenize{httputil:tornado.httputil.HTTPMessageDelegate}]{\sphinxcrossref{\sphinxcode{\sphinxupquote{HTTPMessageDelegate}}}}}
for a given request and {\hyperref[\detokenize{web:tornado.web.RequestHandler}]{\sphinxcrossref{\sphinxcode{\sphinxupquote{RequestHandler}}}}}.

Here is a simple example of how we can we route to
{\hyperref[\detokenize{web:tornado.web.RequestHandler}]{\sphinxcrossref{\sphinxcode{\sphinxupquote{RequestHandler}}}}} subclasses by HTTP method:

\begin{sphinxVerbatim}[commandchars=\\\{\}]
\PYG{n}{resources} \PYG{o}{=} \PYG{p}{\PYGZob{}}\PYG{p}{\PYGZcb{}}

\PYG{k}{class} \PYG{n+nc}{GetResource}\PYG{p}{(}\PYG{n}{RequestHandler}\PYG{p}{)}\PYG{p}{:}
    \PYG{k}{def} \PYG{n+nf}{get}\PYG{p}{(}\PYG{n+nb+bp}{self}\PYG{p}{,} \PYG{n}{path}\PYG{p}{)}\PYG{p}{:}
        \PYG{k}{if} \PYG{n}{path} \PYG{o+ow}{not} \PYG{o+ow}{in} \PYG{n}{resources}\PYG{p}{:}
            \PYG{k}{raise} \PYG{n}{HTTPError}\PYG{p}{(}\PYG{l+m+mi}{404}\PYG{p}{)}

        \PYG{n+nb+bp}{self}\PYG{o}{.}\PYG{n}{finish}\PYG{p}{(}\PYG{n}{resources}\PYG{p}{[}\PYG{n}{path}\PYG{p}{]}\PYG{p}{)}

\PYG{k}{class} \PYG{n+nc}{PostResource}\PYG{p}{(}\PYG{n}{RequestHandler}\PYG{p}{)}\PYG{p}{:}
    \PYG{k}{def} \PYG{n+nf}{post}\PYG{p}{(}\PYG{n+nb+bp}{self}\PYG{p}{,} \PYG{n}{path}\PYG{p}{)}\PYG{p}{:}
        \PYG{n}{resources}\PYG{p}{[}\PYG{n}{path}\PYG{p}{]} \PYG{o}{=} \PYG{n+nb+bp}{self}\PYG{o}{.}\PYG{n}{request}\PYG{o}{.}\PYG{n}{body}

\PYG{k}{class} \PYG{n+nc}{HTTPMethodRouter}\PYG{p}{(}\PYG{n}{Router}\PYG{p}{)}\PYG{p}{:}
    \PYG{k}{def} \PYG{n+nf+fm}{\PYGZus{}\PYGZus{}init\PYGZus{}\PYGZus{}}\PYG{p}{(}\PYG{n+nb+bp}{self}\PYG{p}{,} \PYG{n}{app}\PYG{p}{)}\PYG{p}{:}
        \PYG{n+nb+bp}{self}\PYG{o}{.}\PYG{n}{app} \PYG{o}{=} \PYG{n}{app}

    \PYG{k}{def} \PYG{n+nf}{find\PYGZus{}handler}\PYG{p}{(}\PYG{n+nb+bp}{self}\PYG{p}{,} \PYG{n}{request}\PYG{p}{,} \PYG{o}{*}\PYG{o}{*}\PYG{n}{kwargs}\PYG{p}{)}\PYG{p}{:}
        \PYG{n}{handler} \PYG{o}{=} \PYG{n}{GetResource} \PYG{k}{if} \PYG{n}{request}\PYG{o}{.}\PYG{n}{method} \PYG{o}{==} \PYG{l+s+s2}{\PYGZdq{}}\PYG{l+s+s2}{GET}\PYG{l+s+s2}{\PYGZdq{}} \PYG{k}{else} \PYG{n}{PostResource}
        \PYG{k}{return} \PYG{n+nb+bp}{self}\PYG{o}{.}\PYG{n}{app}\PYG{o}{.}\PYG{n}{get\PYGZus{}handler\PYGZus{}delegate}\PYG{p}{(}\PYG{n}{request}\PYG{p}{,} \PYG{n}{handler}\PYG{p}{,} \PYG{n}{path\PYGZus{}args}\PYG{o}{=}\PYG{p}{[}\PYG{n}{request}\PYG{o}{.}\PYG{n}{path}\PYG{p}{]}\PYG{p}{)}

\PYG{n}{router} \PYG{o}{=} \PYG{n}{HTTPMethodRouter}\PYG{p}{(}\PYG{n}{Application}\PYG{p}{(}\PYG{p}{)}\PYG{p}{)}
\PYG{n}{server} \PYG{o}{=} \PYG{n}{HTTPServer}\PYG{p}{(}\PYG{n}{router}\PYG{p}{)}
\end{sphinxVerbatim}

{\hyperref[\detokenize{routing:tornado.routing.ReversibleRouter}]{\sphinxcrossref{\sphinxcode{\sphinxupquote{ReversibleRouter}}}}} interface adds the ability to distinguish between
the routes and reverse them to the original urls using route’s name
and additional arguments. {\hyperref[\detokenize{web:tornado.web.Application}]{\sphinxcrossref{\sphinxcode{\sphinxupquote{Application}}}}} is itself an
implementation of {\hyperref[\detokenize{routing:tornado.routing.ReversibleRouter}]{\sphinxcrossref{\sphinxcode{\sphinxupquote{ReversibleRouter}}}}} class.

{\hyperref[\detokenize{routing:tornado.routing.RuleRouter}]{\sphinxcrossref{\sphinxcode{\sphinxupquote{RuleRouter}}}}} and {\hyperref[\detokenize{routing:tornado.routing.ReversibleRuleRouter}]{\sphinxcrossref{\sphinxcode{\sphinxupquote{ReversibleRuleRouter}}}}} are implementations of
{\hyperref[\detokenize{routing:tornado.routing.Router}]{\sphinxcrossref{\sphinxcode{\sphinxupquote{Router}}}}} and {\hyperref[\detokenize{routing:tornado.routing.ReversibleRouter}]{\sphinxcrossref{\sphinxcode{\sphinxupquote{ReversibleRouter}}}}} interfaces and can be used for
creating rule-based routing configurations.

Rules are instances of {\hyperref[\detokenize{routing:tornado.routing.Rule}]{\sphinxcrossref{\sphinxcode{\sphinxupquote{Rule}}}}} class. They contain a {\hyperref[\detokenize{routing:tornado.routing.Matcher}]{\sphinxcrossref{\sphinxcode{\sphinxupquote{Matcher}}}}}, which
provides the logic for determining whether the rule is a match for a
particular request and a target, which can be one of the following.
\begin{enumerate}
\def\theenumi{\arabic{enumi}}
\def\labelenumi{\theenumi )}
\makeatletter\def\p@enumii{\p@enumi \theenumi )}\makeatother
\item {} 
An instance of {\hyperref[\detokenize{httputil:tornado.httputil.HTTPServerConnectionDelegate}]{\sphinxcrossref{\sphinxcode{\sphinxupquote{HTTPServerConnectionDelegate}}}}}:

\end{enumerate}

\begin{sphinxVerbatim}[commandchars=\\\{\}]
\PYG{n}{router} \PYG{o}{=} \PYG{n}{RuleRouter}\PYG{p}{(}\PYG{p}{[}
    \PYG{n}{Rule}\PYG{p}{(}\PYG{n}{PathMatches}\PYG{p}{(}\PYG{l+s+s2}{\PYGZdq{}}\PYG{l+s+s2}{/handler}\PYG{l+s+s2}{\PYGZdq{}}\PYG{p}{)}\PYG{p}{,} \PYG{n}{ConnectionDelegate}\PYG{p}{(}\PYG{p}{)}\PYG{p}{)}\PYG{p}{,}
    \PYG{c+c1}{\PYGZsh{} ... more rules}
\PYG{p}{]}\PYG{p}{)}

\PYG{k}{class} \PYG{n+nc}{ConnectionDelegate}\PYG{p}{(}\PYG{n}{HTTPServerConnectionDelegate}\PYG{p}{)}\PYG{p}{:}
    \PYG{k}{def} \PYG{n+nf}{start\PYGZus{}request}\PYG{p}{(}\PYG{n+nb+bp}{self}\PYG{p}{,} \PYG{n}{server\PYGZus{}conn}\PYG{p}{,} \PYG{n}{request\PYGZus{}conn}\PYG{p}{)}\PYG{p}{:}
        \PYG{k}{return} \PYG{n}{MessageDelegate}\PYG{p}{(}\PYG{n}{request\PYGZus{}conn}\PYG{p}{)}
\end{sphinxVerbatim}
\begin{enumerate}
\def\theenumi{\arabic{enumi}}
\def\labelenumi{\theenumi )}
\makeatletter\def\p@enumii{\p@enumi \theenumi )}\makeatother
\setcounter{enumi}{1}
\item {} 
A callable accepting a single argument of {\hyperref[\detokenize{httputil:tornado.httputil.HTTPServerRequest}]{\sphinxcrossref{\sphinxcode{\sphinxupquote{HTTPServerRequest}}}}} type:

\end{enumerate}

\begin{sphinxVerbatim}[commandchars=\\\{\}]
\PYG{n}{router} \PYG{o}{=} \PYG{n}{RuleRouter}\PYG{p}{(}\PYG{p}{[}
    \PYG{n}{Rule}\PYG{p}{(}\PYG{n}{PathMatches}\PYG{p}{(}\PYG{l+s+s2}{\PYGZdq{}}\PYG{l+s+s2}{/callable}\PYG{l+s+s2}{\PYGZdq{}}\PYG{p}{)}\PYG{p}{,} \PYG{n}{request\PYGZus{}callable}\PYG{p}{)}
\PYG{p}{]}\PYG{p}{)}

\PYG{k}{def} \PYG{n+nf}{request\PYGZus{}callable}\PYG{p}{(}\PYG{n}{request}\PYG{p}{)}\PYG{p}{:}
    \PYG{n}{request}\PYG{o}{.}\PYG{n}{write}\PYG{p}{(}\PYG{l+s+sa}{b}\PYG{l+s+s2}{\PYGZdq{}}\PYG{l+s+s2}{HTTP/1.1 200 OK}\PYG{l+s+se}{\PYGZbs{}r}\PYG{l+s+se}{\PYGZbs{}n}\PYG{l+s+s2}{Content\PYGZhy{}Length: 2}\PYG{l+s+se}{\PYGZbs{}r}\PYG{l+s+se}{\PYGZbs{}n}\PYG{l+s+se}{\PYGZbs{}r}\PYG{l+s+se}{\PYGZbs{}n}\PYG{l+s+s2}{OK}\PYG{l+s+s2}{\PYGZdq{}}\PYG{p}{)}
    \PYG{n}{request}\PYG{o}{.}\PYG{n}{finish}\PYG{p}{(}\PYG{p}{)}
\end{sphinxVerbatim}
\begin{enumerate}
\def\theenumi{\arabic{enumi}}
\def\labelenumi{\theenumi )}
\makeatletter\def\p@enumii{\p@enumi \theenumi )}\makeatother
\setcounter{enumi}{2}
\item {} 
Another {\hyperref[\detokenize{routing:tornado.routing.Router}]{\sphinxcrossref{\sphinxcode{\sphinxupquote{Router}}}}} instance:

\end{enumerate}

\begin{sphinxVerbatim}[commandchars=\\\{\}]
\PYG{n}{router} \PYG{o}{=} \PYG{n}{RuleRouter}\PYG{p}{(}\PYG{p}{[}
    \PYG{n}{Rule}\PYG{p}{(}\PYG{n}{PathMatches}\PYG{p}{(}\PYG{l+s+s2}{\PYGZdq{}}\PYG{l+s+s2}{/router.*}\PYG{l+s+s2}{\PYGZdq{}}\PYG{p}{)}\PYG{p}{,} \PYG{n}{CustomRouter}\PYG{p}{(}\PYG{p}{)}\PYG{p}{)}
\PYG{p}{]}\PYG{p}{)}
\end{sphinxVerbatim}

Of course a nested {\hyperref[\detokenize{routing:tornado.routing.RuleRouter}]{\sphinxcrossref{\sphinxcode{\sphinxupquote{RuleRouter}}}}} or a {\hyperref[\detokenize{web:tornado.web.Application}]{\sphinxcrossref{\sphinxcode{\sphinxupquote{Application}}}}} is allowed:

\begin{sphinxVerbatim}[commandchars=\\\{\}]
\PYG{n}{router} \PYG{o}{=} \PYG{n}{RuleRouter}\PYG{p}{(}\PYG{p}{[}
    \PYG{n}{Rule}\PYG{p}{(}\PYG{n}{HostMatches}\PYG{p}{(}\PYG{l+s+s2}{\PYGZdq{}}\PYG{l+s+s2}{example.com}\PYG{l+s+s2}{\PYGZdq{}}\PYG{p}{)}\PYG{p}{,} \PYG{n}{RuleRouter}\PYG{p}{(}\PYG{p}{[}
        \PYG{n}{Rule}\PYG{p}{(}\PYG{n}{PathMatches}\PYG{p}{(}\PYG{l+s+s2}{\PYGZdq{}}\PYG{l+s+s2}{/app1/.*}\PYG{l+s+s2}{\PYGZdq{}}\PYG{p}{)}\PYG{p}{,} \PYG{n}{Application}\PYG{p}{(}\PYG{p}{[}\PYG{p}{(}\PYG{l+s+sa}{r}\PYG{l+s+s2}{\PYGZdq{}}\PYG{l+s+s2}{/app1/handler}\PYG{l+s+s2}{\PYGZdq{}}\PYG{p}{,} \PYG{n}{Handler}\PYG{p}{)}\PYG{p}{]}\PYG{p}{)}\PYG{p}{)}\PYG{p}{)}\PYG{p}{,}
    \PYG{p}{]}\PYG{p}{)}\PYG{p}{)}
\PYG{p}{]}\PYG{p}{)}

\PYG{n}{server} \PYG{o}{=} \PYG{n}{HTTPServer}\PYG{p}{(}\PYG{n}{router}\PYG{p}{)}
\end{sphinxVerbatim}

In the example below {\hyperref[\detokenize{routing:tornado.routing.RuleRouter}]{\sphinxcrossref{\sphinxcode{\sphinxupquote{RuleRouter}}}}} is used to route between applications:

\begin{sphinxVerbatim}[commandchars=\\\{\}]
\PYG{n}{app1} \PYG{o}{=} \PYG{n}{Application}\PYG{p}{(}\PYG{p}{[}
    \PYG{p}{(}\PYG{l+s+sa}{r}\PYG{l+s+s2}{\PYGZdq{}}\PYG{l+s+s2}{/app1/handler}\PYG{l+s+s2}{\PYGZdq{}}\PYG{p}{,} \PYG{n}{Handler1}\PYG{p}{)}\PYG{p}{,}
    \PYG{c+c1}{\PYGZsh{} other handlers ...}
\PYG{p}{]}\PYG{p}{)}

\PYG{n}{app2} \PYG{o}{=} \PYG{n}{Application}\PYG{p}{(}\PYG{p}{[}
    \PYG{p}{(}\PYG{l+s+sa}{r}\PYG{l+s+s2}{\PYGZdq{}}\PYG{l+s+s2}{/app2/handler}\PYG{l+s+s2}{\PYGZdq{}}\PYG{p}{,} \PYG{n}{Handler2}\PYG{p}{)}\PYG{p}{,}
    \PYG{c+c1}{\PYGZsh{} other handlers ...}
\PYG{p}{]}\PYG{p}{)}

\PYG{n}{router} \PYG{o}{=} \PYG{n}{RuleRouter}\PYG{p}{(}\PYG{p}{[}
    \PYG{n}{Rule}\PYG{p}{(}\PYG{n}{PathMatches}\PYG{p}{(}\PYG{l+s+s2}{\PYGZdq{}}\PYG{l+s+s2}{/app1.*}\PYG{l+s+s2}{\PYGZdq{}}\PYG{p}{)}\PYG{p}{,} \PYG{n}{app1}\PYG{p}{)}\PYG{p}{,}
    \PYG{n}{Rule}\PYG{p}{(}\PYG{n}{PathMatches}\PYG{p}{(}\PYG{l+s+s2}{\PYGZdq{}}\PYG{l+s+s2}{/app2.*}\PYG{l+s+s2}{\PYGZdq{}}\PYG{p}{)}\PYG{p}{,} \PYG{n}{app2}\PYG{p}{)}
\PYG{p}{]}\PYG{p}{)}

\PYG{n}{server} \PYG{o}{=} \PYG{n}{HTTPServer}\PYG{p}{(}\PYG{n}{router}\PYG{p}{)}
\end{sphinxVerbatim}

For more information on application-level routing see docs for {\hyperref[\detokenize{web:tornado.web.Application}]{\sphinxcrossref{\sphinxcode{\sphinxupquote{Application}}}}}.

\DUrole{versionmodified,added}{New in version 4.5.}
\index{Router (class in tornado.routing)@\spxentry{Router}\spxextra{class in tornado.routing}}

\begin{fulllineitems}
\phantomsection\label{\detokenize{routing:tornado.routing.Router}}\pysigline{\sphinxbfcode{\sphinxupquote{class }}\sphinxcode{\sphinxupquote{tornado.routing.}}\sphinxbfcode{\sphinxupquote{Router}}}
Abstract router interface.
\index{find\_handler() (tornado.routing.Router method)@\spxentry{find\_handler()}\spxextra{tornado.routing.Router method}}

\begin{fulllineitems}
\phantomsection\label{\detokenize{routing:tornado.routing.Router.find_handler}}\pysiglinewithargsret{\sphinxbfcode{\sphinxupquote{find\_handler}}}{\emph{request: tornado.httputil.HTTPServerRequest}, \emph{**kwargs}}{{ $\rightarrow$ Optional{[}tornado.httputil.HTTPMessageDelegate{]}}}
Must be implemented to return an appropriate instance of {\hyperref[\detokenize{httputil:tornado.httputil.HTTPMessageDelegate}]{\sphinxcrossref{\sphinxcode{\sphinxupquote{HTTPMessageDelegate}}}}}
that can serve the request.
Routing implementations may pass additional kwargs to extend the routing logic.
\begin{quote}\begin{description}
\item[{Parameters}] \leavevmode\begin{itemize}
\item {} 
\sphinxstyleliteralstrong{\sphinxupquote{request}} ({\hyperref[\detokenize{httputil:tornado.httputil.HTTPServerRequest}]{\sphinxcrossref{\sphinxstyleliteralemphasis{\sphinxupquote{httputil.HTTPServerRequest}}}}}) \textendash{} current HTTP request.

\item {} 
\sphinxstyleliteralstrong{\sphinxupquote{kwargs}} \textendash{} additional keyword arguments passed by routing implementation.

\end{itemize}

\item[{Returns}] \leavevmode
an instance of {\hyperref[\detokenize{httputil:tornado.httputil.HTTPMessageDelegate}]{\sphinxcrossref{\sphinxcode{\sphinxupquote{HTTPMessageDelegate}}}}} that will be used to
process the request.

\end{description}\end{quote}

\end{fulllineitems}


\end{fulllineitems}

\index{ReversibleRouter (class in tornado.routing)@\spxentry{ReversibleRouter}\spxextra{class in tornado.routing}}

\begin{fulllineitems}
\phantomsection\label{\detokenize{routing:tornado.routing.ReversibleRouter}}\pysigline{\sphinxbfcode{\sphinxupquote{class }}\sphinxcode{\sphinxupquote{tornado.routing.}}\sphinxbfcode{\sphinxupquote{ReversibleRouter}}}
Abstract router interface for routers that can handle named routes
and support reversing them to original urls.
\index{reverse\_url() (tornado.routing.ReversibleRouter method)@\spxentry{reverse\_url()}\spxextra{tornado.routing.ReversibleRouter method}}

\begin{fulllineitems}
\phantomsection\label{\detokenize{routing:tornado.routing.ReversibleRouter.reverse_url}}\pysiglinewithargsret{\sphinxbfcode{\sphinxupquote{reverse\_url}}}{\emph{name: str}, \emph{*args}}{{ $\rightarrow$ Optional{[}str{]}}}
Returns url string for a given route name and arguments
or \sphinxcode{\sphinxupquote{None}} if no match is found.
\begin{quote}\begin{description}
\item[{Parameters}] \leavevmode\begin{itemize}
\item {} 
\sphinxstyleliteralstrong{\sphinxupquote{name}} (\sphinxhref{https://docs.python.org/3.6/library/stdtypes.html\#str}{\sphinxstyleliteralemphasis{\sphinxupquote{str}}}) \textendash{} route name.

\item {} 
\sphinxstyleliteralstrong{\sphinxupquote{args}} \textendash{} url parameters.

\end{itemize}

\item[{Returns}] \leavevmode
parametrized url string for a given route name (or \sphinxcode{\sphinxupquote{None}}).

\end{description}\end{quote}

\end{fulllineitems}


\end{fulllineitems}

\index{RuleRouter (class in tornado.routing)@\spxentry{RuleRouter}\spxextra{class in tornado.routing}}

\begin{fulllineitems}
\phantomsection\label{\detokenize{routing:tornado.routing.RuleRouter}}\pysiglinewithargsret{\sphinxbfcode{\sphinxupquote{class }}\sphinxcode{\sphinxupquote{tornado.routing.}}\sphinxbfcode{\sphinxupquote{RuleRouter}}}{\emph{rules: List{[}Union{[}Rule, List{[}Any{]}, Tuple{[}Union{[}str, Matcher{]}, Any{]}, Tuple{[}Union{[}str, Matcher{]}, Any, Dict{[}str, Any{]}{]}, Tuple{[}Union{[}str, Matcher{]}, Any, Dict{[}str, Any{]}, str{]}{]}{]} = None}}{}
Rule-based router implementation.

Constructs a router from an ordered list of rules:

\begin{sphinxVerbatim}[commandchars=\\\{\}]
\PYG{n}{RuleRouter}\PYG{p}{(}\PYG{p}{[}
    \PYG{n}{Rule}\PYG{p}{(}\PYG{n}{PathMatches}\PYG{p}{(}\PYG{l+s+s2}{\PYGZdq{}}\PYG{l+s+s2}{/handler}\PYG{l+s+s2}{\PYGZdq{}}\PYG{p}{)}\PYG{p}{,} \PYG{n}{Target}\PYG{p}{)}\PYG{p}{,}
    \PYG{c+c1}{\PYGZsh{} ... more rules}
\PYG{p}{]}\PYG{p}{)}
\end{sphinxVerbatim}

You can also omit explicit {\hyperref[\detokenize{routing:tornado.routing.Rule}]{\sphinxcrossref{\sphinxcode{\sphinxupquote{Rule}}}}} constructor and use tuples of arguments:

\begin{sphinxVerbatim}[commandchars=\\\{\}]
\PYG{n}{RuleRouter}\PYG{p}{(}\PYG{p}{[}
    \PYG{p}{(}\PYG{n}{PathMatches}\PYG{p}{(}\PYG{l+s+s2}{\PYGZdq{}}\PYG{l+s+s2}{/handler}\PYG{l+s+s2}{\PYGZdq{}}\PYG{p}{)}\PYG{p}{,} \PYG{n}{Target}\PYG{p}{)}\PYG{p}{,}
\PYG{p}{]}\PYG{p}{)}
\end{sphinxVerbatim}

{\hyperref[\detokenize{routing:tornado.routing.PathMatches}]{\sphinxcrossref{\sphinxcode{\sphinxupquote{PathMatches}}}}} is a default matcher, so the example above can be simplified:

\begin{sphinxVerbatim}[commandchars=\\\{\}]
\PYG{n}{RuleRouter}\PYG{p}{(}\PYG{p}{[}
    \PYG{p}{(}\PYG{l+s+s2}{\PYGZdq{}}\PYG{l+s+s2}{/handler}\PYG{l+s+s2}{\PYGZdq{}}\PYG{p}{,} \PYG{n}{Target}\PYG{p}{)}\PYG{p}{,}
\PYG{p}{]}\PYG{p}{)}
\end{sphinxVerbatim}

In the examples above, \sphinxcode{\sphinxupquote{Target}} can be a nested {\hyperref[\detokenize{routing:tornado.routing.Router}]{\sphinxcrossref{\sphinxcode{\sphinxupquote{Router}}}}} instance, an instance of
{\hyperref[\detokenize{httputil:tornado.httputil.HTTPServerConnectionDelegate}]{\sphinxcrossref{\sphinxcode{\sphinxupquote{HTTPServerConnectionDelegate}}}}} or an old-style callable,
accepting a request argument.
\begin{quote}\begin{description}
\item[{Parameters}] \leavevmode
\sphinxstyleliteralstrong{\sphinxupquote{rules}} \textendash{} a list of {\hyperref[\detokenize{routing:tornado.routing.Rule}]{\sphinxcrossref{\sphinxcode{\sphinxupquote{Rule}}}}} instances or tuples of {\hyperref[\detokenize{routing:tornado.routing.Rule}]{\sphinxcrossref{\sphinxcode{\sphinxupquote{Rule}}}}}
constructor arguments.

\end{description}\end{quote}
\index{add\_rules() (tornado.routing.RuleRouter method)@\spxentry{add\_rules()}\spxextra{tornado.routing.RuleRouter method}}

\begin{fulllineitems}
\phantomsection\label{\detokenize{routing:tornado.routing.RuleRouter.add_rules}}\pysiglinewithargsret{\sphinxbfcode{\sphinxupquote{add\_rules}}}{\emph{rules: List{[}Union{[}Rule, List{[}Any{]}, Tuple{[}Union{[}str, Matcher{]}, Any{]}, Tuple{[}Union{[}str, Matcher{]}, Any, Dict{[}str, Any{]}{]}, Tuple{[}Union{[}str, Matcher{]}, Any, Dict{[}str, Any{]}, str{]}{]}{]}}}{{ $\rightarrow$ None}}
Appends new rules to the router.
\begin{quote}\begin{description}
\item[{Parameters}] \leavevmode
\sphinxstyleliteralstrong{\sphinxupquote{rules}} \textendash{} a list of Rule instances (or tuples of arguments, which are
passed to Rule constructor).

\end{description}\end{quote}

\end{fulllineitems}

\index{process\_rule() (tornado.routing.RuleRouter method)@\spxentry{process\_rule()}\spxextra{tornado.routing.RuleRouter method}}

\begin{fulllineitems}
\phantomsection\label{\detokenize{routing:tornado.routing.RuleRouter.process_rule}}\pysiglinewithargsret{\sphinxbfcode{\sphinxupquote{process\_rule}}}{\emph{rule: tornado.routing.Rule}}{{ $\rightarrow$ tornado.routing.Rule}}
Override this method for additional preprocessing of each rule.
\begin{quote}\begin{description}
\item[{Parameters}] \leavevmode
\sphinxstyleliteralstrong{\sphinxupquote{rule}} ({\hyperref[\detokenize{routing:tornado.routing.Rule}]{\sphinxcrossref{\sphinxstyleliteralemphasis{\sphinxupquote{Rule}}}}}) \textendash{} a rule to be processed.

\item[{Returns}] \leavevmode
the same or modified Rule instance.

\end{description}\end{quote}

\end{fulllineitems}

\index{get\_target\_delegate() (tornado.routing.RuleRouter method)@\spxentry{get\_target\_delegate()}\spxextra{tornado.routing.RuleRouter method}}

\begin{fulllineitems}
\phantomsection\label{\detokenize{routing:tornado.routing.RuleRouter.get_target_delegate}}\pysiglinewithargsret{\sphinxbfcode{\sphinxupquote{get\_target\_delegate}}}{\emph{target: Any}, \emph{request: tornado.httputil.HTTPServerRequest}, \emph{**target\_params}}{{ $\rightarrow$ Optional{[}tornado.httputil.HTTPMessageDelegate{]}}}
Returns an instance of {\hyperref[\detokenize{httputil:tornado.httputil.HTTPMessageDelegate}]{\sphinxcrossref{\sphinxcode{\sphinxupquote{HTTPMessageDelegate}}}}} for a
Rule’s target. This method is called by {\hyperref[\detokenize{routing:tornado.routing.Router.find_handler}]{\sphinxcrossref{\sphinxcode{\sphinxupquote{find\_handler}}}}} and can be
extended to provide additional target types.
\begin{quote}\begin{description}
\item[{Parameters}] \leavevmode\begin{itemize}
\item {} 
\sphinxstyleliteralstrong{\sphinxupquote{target}} \textendash{} a Rule’s target.

\item {} 
\sphinxstyleliteralstrong{\sphinxupquote{request}} ({\hyperref[\detokenize{httputil:tornado.httputil.HTTPServerRequest}]{\sphinxcrossref{\sphinxstyleliteralemphasis{\sphinxupquote{httputil.HTTPServerRequest}}}}}) \textendash{} current request.

\item {} 
\sphinxstyleliteralstrong{\sphinxupquote{target\_params}} \textendash{} additional parameters that can be useful
for {\hyperref[\detokenize{httputil:tornado.httputil.HTTPMessageDelegate}]{\sphinxcrossref{\sphinxcode{\sphinxupquote{HTTPMessageDelegate}}}}} creation.

\end{itemize}

\end{description}\end{quote}

\end{fulllineitems}


\end{fulllineitems}

\index{ReversibleRuleRouter (class in tornado.routing)@\spxentry{ReversibleRuleRouter}\spxextra{class in tornado.routing}}

\begin{fulllineitems}
\phantomsection\label{\detokenize{routing:tornado.routing.ReversibleRuleRouter}}\pysiglinewithargsret{\sphinxbfcode{\sphinxupquote{class }}\sphinxcode{\sphinxupquote{tornado.routing.}}\sphinxbfcode{\sphinxupquote{ReversibleRuleRouter}}}{\emph{rules: List{[}Union{[}Rule, List{[}Any{]}, Tuple{[}Union{[}str, Matcher{]}, Any{]}, Tuple{[}Union{[}str, Matcher{]}, Any, Dict{[}str, Any{]}{]}, Tuple{[}Union{[}str, Matcher{]}, Any, Dict{[}str, Any{]}, str{]}{]}{]} = None}}{}
A rule-based router that implements \sphinxcode{\sphinxupquote{reverse\_url}} method.

Each rule added to this router may have a \sphinxcode{\sphinxupquote{name}} attribute that can be
used to reconstruct an original uri. The actual reconstruction takes place
in a rule’s matcher (see {\hyperref[\detokenize{routing:tornado.routing.Matcher.reverse}]{\sphinxcrossref{\sphinxcode{\sphinxupquote{Matcher.reverse}}}}}).

\end{fulllineitems}

\index{Rule (class in tornado.routing)@\spxentry{Rule}\spxextra{class in tornado.routing}}

\begin{fulllineitems}
\phantomsection\label{\detokenize{routing:tornado.routing.Rule}}\pysiglinewithargsret{\sphinxbfcode{\sphinxupquote{class }}\sphinxcode{\sphinxupquote{tornado.routing.}}\sphinxbfcode{\sphinxupquote{Rule}}}{\emph{matcher: tornado.routing.Matcher}, \emph{target: Any}, \emph{target\_kwargs: Dict{[}str}, \emph{Any{]} = None}, \emph{name: str = None}}{}
A routing rule.

Constructs a Rule instance.
\begin{quote}\begin{description}
\item[{Parameters}] \leavevmode\begin{itemize}
\item {} 
\sphinxstyleliteralstrong{\sphinxupquote{matcher}} ({\hyperref[\detokenize{routing:tornado.routing.Matcher}]{\sphinxcrossref{\sphinxstyleliteralemphasis{\sphinxupquote{Matcher}}}}}) \textendash{} a {\hyperref[\detokenize{routing:tornado.routing.Matcher}]{\sphinxcrossref{\sphinxcode{\sphinxupquote{Matcher}}}}} instance used for determining
whether the rule should be considered a match for a specific
request.

\item {} 
\sphinxstyleliteralstrong{\sphinxupquote{target}} \textendash{} a Rule’s target (typically a \sphinxcode{\sphinxupquote{RequestHandler}} or
{\hyperref[\detokenize{httputil:tornado.httputil.HTTPServerConnectionDelegate}]{\sphinxcrossref{\sphinxcode{\sphinxupquote{HTTPServerConnectionDelegate}}}}} subclass or even a nested {\hyperref[\detokenize{routing:tornado.routing.Router}]{\sphinxcrossref{\sphinxcode{\sphinxupquote{Router}}}}},
depending on routing implementation).

\item {} 
\sphinxstyleliteralstrong{\sphinxupquote{target\_kwargs}} (\sphinxhref{https://docs.python.org/3.6/library/stdtypes.html\#dict}{\sphinxstyleliteralemphasis{\sphinxupquote{dict}}}) \textendash{} a dict of parameters that can be useful
at the moment of target instantiation (for example, \sphinxcode{\sphinxupquote{status\_code}}
for a \sphinxcode{\sphinxupquote{RequestHandler}} subclass). They end up in
\sphinxcode{\sphinxupquote{target\_params{[}'target\_kwargs'{]}}} of {\hyperref[\detokenize{routing:tornado.routing.RuleRouter.get_target_delegate}]{\sphinxcrossref{\sphinxcode{\sphinxupquote{RuleRouter.get\_target\_delegate}}}}}
method.

\item {} 
\sphinxstyleliteralstrong{\sphinxupquote{name}} (\sphinxhref{https://docs.python.org/3.6/library/stdtypes.html\#str}{\sphinxstyleliteralemphasis{\sphinxupquote{str}}}) \textendash{} the name of the rule that can be used to find it
in {\hyperref[\detokenize{routing:tornado.routing.ReversibleRouter.reverse_url}]{\sphinxcrossref{\sphinxcode{\sphinxupquote{ReversibleRouter.reverse\_url}}}}} implementation.

\end{itemize}

\end{description}\end{quote}

\end{fulllineitems}

\index{Matcher (class in tornado.routing)@\spxentry{Matcher}\spxextra{class in tornado.routing}}

\begin{fulllineitems}
\phantomsection\label{\detokenize{routing:tornado.routing.Matcher}}\pysigline{\sphinxbfcode{\sphinxupquote{class }}\sphinxcode{\sphinxupquote{tornado.routing.}}\sphinxbfcode{\sphinxupquote{Matcher}}}
Represents a matcher for request features.
\index{match() (tornado.routing.Matcher method)@\spxentry{match()}\spxextra{tornado.routing.Matcher method}}

\begin{fulllineitems}
\phantomsection\label{\detokenize{routing:tornado.routing.Matcher.match}}\pysiglinewithargsret{\sphinxbfcode{\sphinxupquote{match}}}{\emph{request: tornado.httputil.HTTPServerRequest}}{{ $\rightarrow$ Optional{[}Dict{[}str, Any{]}{]}}}
Matches current instance against the request.
\begin{quote}\begin{description}
\item[{Parameters}] \leavevmode
\sphinxstyleliteralstrong{\sphinxupquote{request}} ({\hyperref[\detokenize{httputil:tornado.httputil.HTTPServerRequest}]{\sphinxcrossref{\sphinxstyleliteralemphasis{\sphinxupquote{httputil.HTTPServerRequest}}}}}) \textendash{} current HTTP request

\item[{Returns}] \leavevmode
a dict of parameters to be passed to the target handler
(for example, \sphinxcode{\sphinxupquote{handler\_kwargs}}, \sphinxcode{\sphinxupquote{path\_args}}, \sphinxcode{\sphinxupquote{path\_kwargs}}
can be passed for proper {\hyperref[\detokenize{web:tornado.web.RequestHandler}]{\sphinxcrossref{\sphinxcode{\sphinxupquote{RequestHandler}}}}} instantiation).
An empty dict is a valid (and common) return value to indicate a match
when the argument-passing features are not used.
\sphinxcode{\sphinxupquote{None}} must be returned to indicate that there is no match.

\end{description}\end{quote}

\end{fulllineitems}

\index{reverse() (tornado.routing.Matcher method)@\spxentry{reverse()}\spxextra{tornado.routing.Matcher method}}

\begin{fulllineitems}
\phantomsection\label{\detokenize{routing:tornado.routing.Matcher.reverse}}\pysiglinewithargsret{\sphinxbfcode{\sphinxupquote{reverse}}}{\emph{*args}}{{ $\rightarrow$ Optional{[}str{]}}}
Reconstructs full url from matcher instance and additional arguments.

\end{fulllineitems}


\end{fulllineitems}

\index{AnyMatches (class in tornado.routing)@\spxentry{AnyMatches}\spxextra{class in tornado.routing}}

\begin{fulllineitems}
\phantomsection\label{\detokenize{routing:tornado.routing.AnyMatches}}\pysigline{\sphinxbfcode{\sphinxupquote{class }}\sphinxcode{\sphinxupquote{tornado.routing.}}\sphinxbfcode{\sphinxupquote{AnyMatches}}}
Matches any request.

\end{fulllineitems}

\index{HostMatches (class in tornado.routing)@\spxentry{HostMatches}\spxextra{class in tornado.routing}}

\begin{fulllineitems}
\phantomsection\label{\detokenize{routing:tornado.routing.HostMatches}}\pysiglinewithargsret{\sphinxbfcode{\sphinxupquote{class }}\sphinxcode{\sphinxupquote{tornado.routing.}}\sphinxbfcode{\sphinxupquote{HostMatches}}}{\emph{host\_pattern: Union{[}str, Pattern{[}AnyStr{]}{]}}}{}
Matches requests from hosts specified by \sphinxcode{\sphinxupquote{host\_pattern}} regex.

\end{fulllineitems}

\index{DefaultHostMatches (class in tornado.routing)@\spxentry{DefaultHostMatches}\spxextra{class in tornado.routing}}

\begin{fulllineitems}
\phantomsection\label{\detokenize{routing:tornado.routing.DefaultHostMatches}}\pysiglinewithargsret{\sphinxbfcode{\sphinxupquote{class }}\sphinxcode{\sphinxupquote{tornado.routing.}}\sphinxbfcode{\sphinxupquote{DefaultHostMatches}}}{\emph{application: Any, host\_pattern: Pattern{[}AnyStr{]}}}{}
Matches requests from host that is equal to application’s default\_host.
Always returns no match if \sphinxcode{\sphinxupquote{X-Real-Ip}} header is present.

\end{fulllineitems}

\index{PathMatches (class in tornado.routing)@\spxentry{PathMatches}\spxextra{class in tornado.routing}}

\begin{fulllineitems}
\phantomsection\label{\detokenize{routing:tornado.routing.PathMatches}}\pysiglinewithargsret{\sphinxbfcode{\sphinxupquote{class }}\sphinxcode{\sphinxupquote{tornado.routing.}}\sphinxbfcode{\sphinxupquote{PathMatches}}}{\emph{path\_pattern: Union{[}str, Pattern{[}AnyStr{]}{]}}}{}
Matches requests with paths specified by \sphinxcode{\sphinxupquote{path\_pattern}} regex.

\end{fulllineitems}

\index{URLSpec (class in tornado.routing)@\spxentry{URLSpec}\spxextra{class in tornado.routing}}

\begin{fulllineitems}
\phantomsection\label{\detokenize{routing:tornado.routing.URLSpec}}\pysiglinewithargsret{\sphinxbfcode{\sphinxupquote{class }}\sphinxcode{\sphinxupquote{tornado.routing.}}\sphinxbfcode{\sphinxupquote{URLSpec}}}{\emph{pattern: Union{[}str, Pattern{[}AnyStr{]}{]}, handler: Any, kwargs: Dict{[}str, Any{]} = None, name: str = None}}{}
Specifies mappings between URLs and handlers.

Parameters:
\begin{itemize}
\item {} 
\sphinxcode{\sphinxupquote{pattern}}: Regular expression to be matched. Any capturing
groups in the regex will be passed in to the handler’s
get/post/etc methods as arguments (by keyword if named, by
position if unnamed. Named and unnamed capturing groups
may not be mixed in the same rule).

\item {} 
\sphinxcode{\sphinxupquote{handler}}: {\hyperref[\detokenize{web:tornado.web.RequestHandler}]{\sphinxcrossref{\sphinxcode{\sphinxupquote{RequestHandler}}}}} subclass to be invoked.

\item {} 
\sphinxcode{\sphinxupquote{kwargs}} (optional): A dictionary of additional arguments
to be passed to the handler’s constructor.

\item {} 
\sphinxcode{\sphinxupquote{name}} (optional): A name for this handler.  Used by
{\hyperref[\detokenize{web:tornado.web.Application.reverse_url}]{\sphinxcrossref{\sphinxcode{\sphinxupquote{reverse\_url}}}}}.

\end{itemize}

\end{fulllineitems}



\subsection{\sphinxstyleliteralintitle{\sphinxupquote{tornado.escape}} — Escaping and string manipulation}
\label{\detokenize{escape:module-tornado.escape}}\label{\detokenize{escape:tornado-escape-escaping-and-string-manipulation}}\label{\detokenize{escape::doc}}\index{tornado.escape (module)@\spxentry{tornado.escape}\spxextra{module}}
Escaping/unescaping methods for HTML, JSON, URLs, and others.

Also includes a few other miscellaneous string manipulation functions that
have crept in over time.


\subsubsection{Escaping functions}
\label{\detokenize{escape:escaping-functions}}\index{xhtml\_escape() (in module tornado.escape)@\spxentry{xhtml\_escape()}\spxextra{in module tornado.escape}}

\begin{fulllineitems}
\phantomsection\label{\detokenize{escape:tornado.escape.xhtml_escape}}\pysiglinewithargsret{\sphinxcode{\sphinxupquote{tornado.escape.}}\sphinxbfcode{\sphinxupquote{xhtml\_escape}}}{\emph{value: Union{[}str, bytes{]}}}{{ $\rightarrow$ str}}
Escapes a string so it is valid within HTML or XML.

Escapes the characters \sphinxcode{\sphinxupquote{\textless{}}}, \sphinxcode{\sphinxupquote{\textgreater{}}}, \sphinxcode{\sphinxupquote{"}}, \sphinxcode{\sphinxupquote{'}}, and \sphinxcode{\sphinxupquote{\&}}.
When used in attribute values the escaped strings must be enclosed
in quotes.

\DUrole{versionmodified,changed}{Changed in version 3.2: }Added the single quote to the list of escaped characters.

\end{fulllineitems}

\index{xhtml\_unescape() (in module tornado.escape)@\spxentry{xhtml\_unescape()}\spxextra{in module tornado.escape}}

\begin{fulllineitems}
\phantomsection\label{\detokenize{escape:tornado.escape.xhtml_unescape}}\pysiglinewithargsret{\sphinxcode{\sphinxupquote{tornado.escape.}}\sphinxbfcode{\sphinxupquote{xhtml\_unescape}}}{\emph{value: Union{[}str, bytes{]}}}{{ $\rightarrow$ str}}
Un-escapes an XML-escaped string.

\end{fulllineitems}

\index{url\_escape() (in module tornado.escape)@\spxentry{url\_escape()}\spxextra{in module tornado.escape}}

\begin{fulllineitems}
\phantomsection\label{\detokenize{escape:tornado.escape.url_escape}}\pysiglinewithargsret{\sphinxcode{\sphinxupquote{tornado.escape.}}\sphinxbfcode{\sphinxupquote{url\_escape}}}{\emph{value: Union{[}str, bytes{]}, plus: bool = True}}{{ $\rightarrow$ str}}
Returns a URL-encoded version of the given value.

If \sphinxcode{\sphinxupquote{plus}} is true (the default), spaces will be represented
as “+” instead of “\%20”.  This is appropriate for query strings
but not for the path component of a URL.  Note that this default
is the reverse of Python’s urllib module.

\DUrole{versionmodified,added}{New in version 3.1: }The \sphinxcode{\sphinxupquote{plus}} argument

\end{fulllineitems}

\index{url\_unescape() (in module tornado.escape)@\spxentry{url\_unescape()}\spxextra{in module tornado.escape}}

\begin{fulllineitems}
\phantomsection\label{\detokenize{escape:tornado.escape.url_unescape}}\pysiglinewithargsret{\sphinxcode{\sphinxupquote{tornado.escape.}}\sphinxbfcode{\sphinxupquote{url\_unescape}}}{\emph{value: Union{[}str, bytes{]}, encoding: Optional{[}str{]} = 'utf-8', plus: bool = True}}{{ $\rightarrow$ Union{[}str, bytes{]}}}
Decodes the given value from a URL.

The argument may be either a byte or unicode string.

If encoding is None, the result will be a byte string.  Otherwise,
the result is a unicode string in the specified encoding.

If \sphinxcode{\sphinxupquote{plus}} is true (the default), plus signs will be interpreted
as spaces (literal plus signs must be represented as “\%2B”).  This
is appropriate for query strings and form-encoded values but not
for the path component of a URL.  Note that this default is the
reverse of Python’s urllib module.

\DUrole{versionmodified,added}{New in version 3.1: }The \sphinxcode{\sphinxupquote{plus}} argument

\end{fulllineitems}

\index{json\_encode() (in module tornado.escape)@\spxentry{json\_encode()}\spxextra{in module tornado.escape}}

\begin{fulllineitems}
\phantomsection\label{\detokenize{escape:tornado.escape.json_encode}}\pysiglinewithargsret{\sphinxcode{\sphinxupquote{tornado.escape.}}\sphinxbfcode{\sphinxupquote{json\_encode}}}{\emph{value: Any}}{{ $\rightarrow$ str}}
JSON-encodes the given Python object.

\end{fulllineitems}

\index{json\_decode() (in module tornado.escape)@\spxentry{json\_decode()}\spxextra{in module tornado.escape}}

\begin{fulllineitems}
\phantomsection\label{\detokenize{escape:tornado.escape.json_decode}}\pysiglinewithargsret{\sphinxcode{\sphinxupquote{tornado.escape.}}\sphinxbfcode{\sphinxupquote{json\_decode}}}{\emph{value: Union{[}str, bytes{]}}}{{ $\rightarrow$ Any}}
Returns Python objects for the given JSON string.

Supports both \sphinxhref{https://docs.python.org/3.6/library/stdtypes.html\#str}{\sphinxcode{\sphinxupquote{str}}} and \sphinxhref{https://docs.python.org/3.6/library/stdtypes.html\#bytes}{\sphinxcode{\sphinxupquote{bytes}}} inputs.

\end{fulllineitems}



\subsubsection{Byte/unicode conversions}
\label{\detokenize{escape:byte-unicode-conversions}}\index{utf8() (in module tornado.escape)@\spxentry{utf8()}\spxextra{in module tornado.escape}}

\begin{fulllineitems}
\phantomsection\label{\detokenize{escape:tornado.escape.utf8}}\pysiglinewithargsret{\sphinxcode{\sphinxupquote{tornado.escape.}}\sphinxbfcode{\sphinxupquote{utf8}}}{\emph{value: Union{[}None, str, bytes{]}}}{{ $\rightarrow$ Optional{[}bytes{]}}}
Converts a string argument to a byte string.

If the argument is already a byte string or None, it is returned unchanged.
Otherwise it must be a unicode string and is encoded as utf8.

\end{fulllineitems}

\index{to\_unicode() (in module tornado.escape)@\spxentry{to\_unicode()}\spxextra{in module tornado.escape}}

\begin{fulllineitems}
\phantomsection\label{\detokenize{escape:tornado.escape.to_unicode}}\pysiglinewithargsret{\sphinxcode{\sphinxupquote{tornado.escape.}}\sphinxbfcode{\sphinxupquote{to\_unicode}}}{\emph{value: Union{[}None, str, bytes{]}}}{{ $\rightarrow$ Optional{[}str{]}}}
Converts a string argument to a unicode string.

If the argument is already a unicode string or None, it is returned
unchanged.  Otherwise it must be a byte string and is decoded as utf8.

\end{fulllineitems}

\index{native\_str() (in module tornado.escape)@\spxentry{native\_str()}\spxextra{in module tornado.escape}}

\begin{fulllineitems}
\phantomsection\label{\detokenize{escape:tornado.escape.native_str}}\pysiglinewithargsret{\sphinxcode{\sphinxupquote{tornado.escape.}}\sphinxbfcode{\sphinxupquote{native\_str}}}{}{}
\end{fulllineitems}

\index{to\_basestring() (in module tornado.escape)@\spxentry{to\_basestring()}\spxextra{in module tornado.escape}}

\begin{fulllineitems}
\phantomsection\label{\detokenize{escape:tornado.escape.to_basestring}}\pysiglinewithargsret{\sphinxcode{\sphinxupquote{tornado.escape.}}\sphinxbfcode{\sphinxupquote{to\_basestring}}}{}{}
Converts a byte or unicode string into type \sphinxhref{https://docs.python.org/3.6/library/stdtypes.html\#str}{\sphinxcode{\sphinxupquote{str}}}. These functions
were used to help transition from Python 2 to Python 3 but are now
deprecated aliases for {\hyperref[\detokenize{escape:tornado.escape.to_unicode}]{\sphinxcrossref{\sphinxcode{\sphinxupquote{to\_unicode}}}}}.

\end{fulllineitems}

\index{recursive\_unicode() (in module tornado.escape)@\spxentry{recursive\_unicode()}\spxextra{in module tornado.escape}}

\begin{fulllineitems}
\phantomsection\label{\detokenize{escape:tornado.escape.recursive_unicode}}\pysiglinewithargsret{\sphinxcode{\sphinxupquote{tornado.escape.}}\sphinxbfcode{\sphinxupquote{recursive\_unicode}}}{\emph{obj: Any}}{{ $\rightarrow$ Any}}
Walks a simple data structure, converting byte strings to unicode.

Supports lists, tuples, and dictionaries.

\end{fulllineitems}



\subsubsection{Miscellaneous functions}
\label{\detokenize{escape:miscellaneous-functions}}\index{linkify() (in module tornado.escape)@\spxentry{linkify()}\spxextra{in module tornado.escape}}

\begin{fulllineitems}
\phantomsection\label{\detokenize{escape:tornado.escape.linkify}}\pysiglinewithargsret{\sphinxcode{\sphinxupquote{tornado.escape.}}\sphinxbfcode{\sphinxupquote{linkify}}}{\emph{text: Union{[}str, bytes{]}, shorten: bool = False, extra\_params: Union{[}str, Callable{[}{[}str{]}, str{]}{]} = '', require\_protocol: bool = False, permitted\_protocols: List{[}str{]} = {[}'http', 'https'{]}}}{{ $\rightarrow$ str}}
Converts plain text into HTML with links.

For example: \sphinxcode{\sphinxupquote{linkify("Hello http://tornadoweb.org!")}} would return
\sphinxcode{\sphinxupquote{Hello \textless{}a href="http://tornadoweb.org"\textgreater{}http://tornadoweb.org\textless{}/a\textgreater{}!}}

Parameters:
\begin{itemize}
\item {} 
\sphinxcode{\sphinxupquote{shorten}}: Long urls will be shortened for display.

\item {} 
\sphinxcode{\sphinxupquote{extra\_params}}: Extra text to include in the link tag, or a callable
taking the link as an argument and returning the extra text
e.g. \sphinxcode{\sphinxupquote{linkify(text, extra\_params='rel="nofollow" class="external"')}},
or:

\begin{sphinxVerbatim}[commandchars=\\\{\}]
\PYG{k}{def} \PYG{n+nf}{extra\PYGZus{}params\PYGZus{}cb}\PYG{p}{(}\PYG{n}{url}\PYG{p}{)}\PYG{p}{:}
    \PYG{k}{if} \PYG{n}{url}\PYG{o}{.}\PYG{n}{startswith}\PYG{p}{(}\PYG{l+s+s2}{\PYGZdq{}}\PYG{l+s+s2}{http://example.com}\PYG{l+s+s2}{\PYGZdq{}}\PYG{p}{)}\PYG{p}{:}
        \PYG{k}{return} \PYG{l+s+s1}{\PYGZsq{}}\PYG{l+s+s1}{class=}\PYG{l+s+s1}{\PYGZdq{}}\PYG{l+s+s1}{internal}\PYG{l+s+s1}{\PYGZdq{}}\PYG{l+s+s1}{\PYGZsq{}}
    \PYG{k}{else}\PYG{p}{:}
        \PYG{k}{return} \PYG{l+s+s1}{\PYGZsq{}}\PYG{l+s+s1}{class=}\PYG{l+s+s1}{\PYGZdq{}}\PYG{l+s+s1}{external}\PYG{l+s+s1}{\PYGZdq{}}\PYG{l+s+s1}{ rel=}\PYG{l+s+s1}{\PYGZdq{}}\PYG{l+s+s1}{nofollow}\PYG{l+s+s1}{\PYGZdq{}}\PYG{l+s+s1}{\PYGZsq{}}
\PYG{n}{linkify}\PYG{p}{(}\PYG{n}{text}\PYG{p}{,} \PYG{n}{extra\PYGZus{}params}\PYG{o}{=}\PYG{n}{extra\PYGZus{}params\PYGZus{}cb}\PYG{p}{)}
\end{sphinxVerbatim}

\item {} 
\sphinxcode{\sphinxupquote{require\_protocol}}: Only linkify urls which include a protocol. If
this is False, urls such as www.facebook.com will also be linkified.

\item {} 
\sphinxcode{\sphinxupquote{permitted\_protocols}}: List (or set) of protocols which should be
linkified, e.g. \sphinxcode{\sphinxupquote{linkify(text, permitted\_protocols={[}"http", "ftp",
"mailto"{]})}}. It is very unsafe to include protocols such as
\sphinxcode{\sphinxupquote{javascript}}.

\end{itemize}

\end{fulllineitems}

\index{squeeze() (in module tornado.escape)@\spxentry{squeeze()}\spxextra{in module tornado.escape}}

\begin{fulllineitems}
\phantomsection\label{\detokenize{escape:tornado.escape.squeeze}}\pysiglinewithargsret{\sphinxcode{\sphinxupquote{tornado.escape.}}\sphinxbfcode{\sphinxupquote{squeeze}}}{\emph{value: str}}{{ $\rightarrow$ str}}
Replace all sequences of whitespace chars with a single space.

\end{fulllineitems}



\subsection{\sphinxstyleliteralintitle{\sphinxupquote{tornado.locale}} — Internationalization support}
\label{\detokenize{locale:module-tornado.locale}}\label{\detokenize{locale:tornado-locale-internationalization-support}}\label{\detokenize{locale::doc}}\index{tornado.locale (module)@\spxentry{tornado.locale}\spxextra{module}}
Translation methods for generating localized strings.

To load a locale and generate a translated string:

\begin{sphinxVerbatim}[commandchars=\\\{\}]
\PYG{n}{user\PYGZus{}locale} \PYG{o}{=} \PYG{n}{tornado}\PYG{o}{.}\PYG{n}{locale}\PYG{o}{.}\PYG{n}{get}\PYG{p}{(}\PYG{l+s+s2}{\PYGZdq{}}\PYG{l+s+s2}{es\PYGZus{}LA}\PYG{l+s+s2}{\PYGZdq{}}\PYG{p}{)}
\PYG{n+nb}{print}\PYG{p}{(}\PYG{n}{user\PYGZus{}locale}\PYG{o}{.}\PYG{n}{translate}\PYG{p}{(}\PYG{l+s+s2}{\PYGZdq{}}\PYG{l+s+s2}{Sign out}\PYG{l+s+s2}{\PYGZdq{}}\PYG{p}{)}\PYG{p}{)}
\end{sphinxVerbatim}

{\hyperref[\detokenize{locale:tornado.locale.get}]{\sphinxcrossref{\sphinxcode{\sphinxupquote{tornado.locale.get()}}}}} returns the closest matching locale, not necessarily the
specific locale you requested. You can support pluralization with
additional arguments to {\hyperref[\detokenize{locale:tornado.locale.Locale.translate}]{\sphinxcrossref{\sphinxcode{\sphinxupquote{translate()}}}}}, e.g.:

\begin{sphinxVerbatim}[commandchars=\\\{\}]
\PYG{n}{people} \PYG{o}{=} \PYG{p}{[}\PYG{o}{.}\PYG{o}{.}\PYG{o}{.}\PYG{p}{]}
\PYG{n}{message} \PYG{o}{=} \PYG{n}{user\PYGZus{}locale}\PYG{o}{.}\PYG{n}{translate}\PYG{p}{(}
    \PYG{l+s+s2}{\PYGZdq{}}\PYG{l+s+si}{\PYGZpc{}(list)s}\PYG{l+s+s2}{ is online}\PYG{l+s+s2}{\PYGZdq{}}\PYG{p}{,} \PYG{l+s+s2}{\PYGZdq{}}\PYG{l+s+si}{\PYGZpc{}(list)s}\PYG{l+s+s2}{ are online}\PYG{l+s+s2}{\PYGZdq{}}\PYG{p}{,} \PYG{n+nb}{len}\PYG{p}{(}\PYG{n}{people}\PYG{p}{)}\PYG{p}{)}
\PYG{n+nb}{print}\PYG{p}{(}\PYG{n}{message} \PYG{o}{\PYGZpc{}} \PYG{p}{\PYGZob{}}\PYG{l+s+s2}{\PYGZdq{}}\PYG{l+s+s2}{list}\PYG{l+s+s2}{\PYGZdq{}}\PYG{p}{:} \PYG{n}{user\PYGZus{}locale}\PYG{o}{.}\PYG{n}{list}\PYG{p}{(}\PYG{n}{people}\PYG{p}{)}\PYG{p}{\PYGZcb{}}\PYG{p}{)}
\end{sphinxVerbatim}

The first string is chosen if \sphinxcode{\sphinxupquote{len(people) == 1}}, otherwise the second
string is chosen.

Applications should call one of {\hyperref[\detokenize{locale:tornado.locale.load_translations}]{\sphinxcrossref{\sphinxcode{\sphinxupquote{load\_translations}}}}} (which uses a simple
CSV format) or {\hyperref[\detokenize{locale:tornado.locale.load_gettext_translations}]{\sphinxcrossref{\sphinxcode{\sphinxupquote{load\_gettext\_translations}}}}} (which uses the \sphinxcode{\sphinxupquote{.mo}} format
supported by \sphinxhref{https://docs.python.org/3.6/library/gettext.html\#module-gettext}{\sphinxcode{\sphinxupquote{gettext}}} and related tools).  If neither method is called,
the {\hyperref[\detokenize{locale:tornado.locale.Locale.translate}]{\sphinxcrossref{\sphinxcode{\sphinxupquote{Locale.translate}}}}} method will simply return the original string.
\index{get() (in module tornado.locale)@\spxentry{get()}\spxextra{in module tornado.locale}}

\begin{fulllineitems}
\phantomsection\label{\detokenize{locale:tornado.locale.get}}\pysiglinewithargsret{\sphinxcode{\sphinxupquote{tornado.locale.}}\sphinxbfcode{\sphinxupquote{get}}}{\emph{*locale\_codes}}{{ $\rightarrow$ tornado.locale.Locale}}
Returns the closest match for the given locale codes.

We iterate over all given locale codes in order. If we have a tight
or a loose match for the code (e.g., “en” for “en\_US”), we return
the locale. Otherwise we move to the next code in the list.

By default we return \sphinxcode{\sphinxupquote{en\_US}} if no translations are found for any of
the specified locales. You can change the default locale with
{\hyperref[\detokenize{locale:tornado.locale.set_default_locale}]{\sphinxcrossref{\sphinxcode{\sphinxupquote{set\_default\_locale()}}}}}.

\end{fulllineitems}

\index{set\_default\_locale() (in module tornado.locale)@\spxentry{set\_default\_locale()}\spxextra{in module tornado.locale}}

\begin{fulllineitems}
\phantomsection\label{\detokenize{locale:tornado.locale.set_default_locale}}\pysiglinewithargsret{\sphinxcode{\sphinxupquote{tornado.locale.}}\sphinxbfcode{\sphinxupquote{set\_default\_locale}}}{\emph{code: str}}{{ $\rightarrow$ None}}
Sets the default locale.

The default locale is assumed to be the language used for all strings
in the system. The translations loaded from disk are mappings from
the default locale to the destination locale. Consequently, you don’t
need to create a translation file for the default locale.

\end{fulllineitems}

\index{load\_translations() (in module tornado.locale)@\spxentry{load\_translations()}\spxextra{in module tornado.locale}}

\begin{fulllineitems}
\phantomsection\label{\detokenize{locale:tornado.locale.load_translations}}\pysiglinewithargsret{\sphinxcode{\sphinxupquote{tornado.locale.}}\sphinxbfcode{\sphinxupquote{load\_translations}}}{\emph{directory: str}, \emph{encoding: str = None}}{{ $\rightarrow$ None}}
Loads translations from CSV files in a directory.

Translations are strings with optional Python-style named placeholders
(e.g., \sphinxcode{\sphinxupquote{My name is \%(name)s}}) and their associated translations.

The directory should have translation files of the form \sphinxcode{\sphinxupquote{LOCALE.csv}},
e.g. \sphinxcode{\sphinxupquote{es\_GT.csv}}. The CSV files should have two or three columns: string,
translation, and an optional plural indicator. Plural indicators should
be one of “plural” or “singular”. A given string can have both singular
and plural forms. For example \sphinxcode{\sphinxupquote{\%(name)s liked this}} may have a
different verb conjugation depending on whether \%(name)s is one
name or a list of names. There should be two rows in the CSV file for
that string, one with plural indicator “singular”, and one “plural”.
For strings with no verbs that would change on translation, simply
use “unknown” or the empty string (or don’t include the column at all).

The file is read using the \sphinxhref{https://docs.python.org/3.6/library/csv.html\#module-csv}{\sphinxcode{\sphinxupquote{csv}}} module in the default “excel” dialect.
In this format there should not be spaces after the commas.

If no \sphinxcode{\sphinxupquote{encoding}} parameter is given, the encoding will be
detected automatically (among UTF-8 and UTF-16) if the file
contains a byte-order marker (BOM), defaulting to UTF-8 if no BOM
is present.

Example translation \sphinxcode{\sphinxupquote{es\_LA.csv}}:

\begin{sphinxVerbatim}[commandchars=\\\{\}]
\PYG{l+s+s2}{\PYGZdq{}}\PYG{l+s+s2}{I love you}\PYG{l+s+s2}{\PYGZdq{}}\PYG{p}{,}\PYG{l+s+s2}{\PYGZdq{}}\PYG{l+s+s2}{Te amo}\PYG{l+s+s2}{\PYGZdq{}}
\PYG{l+s+s2}{\PYGZdq{}}\PYG{l+s+si}{\PYGZpc{}(name)s}\PYG{l+s+s2}{ liked this}\PYG{l+s+s2}{\PYGZdq{}}\PYG{p}{,}\PYG{l+s+s2}{\PYGZdq{}}\PYG{l+s+s2}{A }\PYG{l+s+si}{\PYGZpc{}(name)s}\PYG{l+s+s2}{ les gustó esto}\PYG{l+s+s2}{\PYGZdq{}}\PYG{p}{,}\PYG{l+s+s2}{\PYGZdq{}}\PYG{l+s+s2}{plural}\PYG{l+s+s2}{\PYGZdq{}}
\PYG{l+s+s2}{\PYGZdq{}}\PYG{l+s+si}{\PYGZpc{}(name)s}\PYG{l+s+s2}{ liked this}\PYG{l+s+s2}{\PYGZdq{}}\PYG{p}{,}\PYG{l+s+s2}{\PYGZdq{}}\PYG{l+s+s2}{A }\PYG{l+s+si}{\PYGZpc{}(name)s}\PYG{l+s+s2}{ le gustó esto}\PYG{l+s+s2}{\PYGZdq{}}\PYG{p}{,}\PYG{l+s+s2}{\PYGZdq{}}\PYG{l+s+s2}{singular}\PYG{l+s+s2}{\PYGZdq{}}
\end{sphinxVerbatim}

\DUrole{versionmodified,changed}{Changed in version 4.3: }Added \sphinxcode{\sphinxupquote{encoding}} parameter. Added support for BOM-based encoding
detection, UTF-16, and UTF-8-with-BOM.

\end{fulllineitems}

\index{load\_gettext\_translations() (in module tornado.locale)@\spxentry{load\_gettext\_translations()}\spxextra{in module tornado.locale}}

\begin{fulllineitems}
\phantomsection\label{\detokenize{locale:tornado.locale.load_gettext_translations}}\pysiglinewithargsret{\sphinxcode{\sphinxupquote{tornado.locale.}}\sphinxbfcode{\sphinxupquote{load\_gettext\_translations}}}{\emph{directory: str}, \emph{domain: str}}{{ $\rightarrow$ None}}
Loads translations from \sphinxhref{https://docs.python.org/3.6/library/gettext.html\#module-gettext}{\sphinxcode{\sphinxupquote{gettext}}}’s locale tree

Locale tree is similar to system’s \sphinxcode{\sphinxupquote{/usr/share/locale}}, like:

\begin{sphinxVerbatim}[commandchars=\\\{\}]
\PYG{p}{\PYGZob{}}\PYG{n}{directory}\PYG{p}{\PYGZcb{}}\PYG{o}{/}\PYG{p}{\PYGZob{}}\PYG{n}{lang}\PYG{p}{\PYGZcb{}}\PYG{o}{/}\PYG{n}{LC\PYGZus{}MESSAGES}\PYG{o}{/}\PYG{p}{\PYGZob{}}\PYG{n}{domain}\PYG{p}{\PYGZcb{}}\PYG{o}{.}\PYG{n}{mo}
\end{sphinxVerbatim}

Three steps are required to have your app translated:
\begin{enumerate}
\def\theenumi{\arabic{enumi}}
\def\labelenumi{\theenumi .}
\makeatletter\def\p@enumii{\p@enumi \theenumi .}\makeatother
\item {} 
Generate POT translation file:

\begin{sphinxVerbatim}[commandchars=\\\{\}]
\PYG{n}{xgettext} \PYG{o}{\PYGZhy{}}\PYG{o}{\PYGZhy{}}\PYG{n}{language}\PYG{o}{=}\PYG{n}{Python} \PYG{o}{\PYGZhy{}}\PYG{o}{\PYGZhy{}}\PYG{n}{keyword}\PYG{o}{=}\PYG{n}{\PYGZus{}}\PYG{p}{:}\PYG{l+m+mi}{1}\PYG{p}{,}\PYG{l+m+mi}{2} \PYG{o}{\PYGZhy{}}\PYG{n}{d} \PYG{n}{mydomain} \PYG{n}{file1}\PYG{o}{.}\PYG{n}{py} \PYG{n}{file2}\PYG{o}{.}\PYG{n}{html} \PYG{n}{etc}
\end{sphinxVerbatim}

\item {} 
Merge against existing POT file:

\begin{sphinxVerbatim}[commandchars=\\\{\}]
\PYG{n}{msgmerge} \PYG{n}{old}\PYG{o}{.}\PYG{n}{po} \PYG{n}{mydomain}\PYG{o}{.}\PYG{n}{po} \PYG{o}{\PYGZgt{}} \PYG{n}{new}\PYG{o}{.}\PYG{n}{po}
\end{sphinxVerbatim}

\item {} 
Compile:

\begin{sphinxVerbatim}[commandchars=\\\{\}]
\PYG{n}{msgfmt} \PYG{n}{mydomain}\PYG{o}{.}\PYG{n}{po} \PYG{o}{\PYGZhy{}}\PYG{n}{o} \PYG{p}{\PYGZob{}}\PYG{n}{directory}\PYG{p}{\PYGZcb{}}\PYG{o}{/}\PYG{n}{pt\PYGZus{}BR}\PYG{o}{/}\PYG{n}{LC\PYGZus{}MESSAGES}\PYG{o}{/}\PYG{n}{mydomain}\PYG{o}{.}\PYG{n}{mo}
\end{sphinxVerbatim}

\end{enumerate}

\end{fulllineitems}

\index{get\_supported\_locales() (in module tornado.locale)@\spxentry{get\_supported\_locales()}\spxextra{in module tornado.locale}}

\begin{fulllineitems}
\phantomsection\label{\detokenize{locale:tornado.locale.get_supported_locales}}\pysiglinewithargsret{\sphinxcode{\sphinxupquote{tornado.locale.}}\sphinxbfcode{\sphinxupquote{get\_supported\_locales}}}{}{{ $\rightarrow$ Iterable{[}str{]}}}
Returns a list of all the supported locale codes.

\end{fulllineitems}

\index{Locale (class in tornado.locale)@\spxentry{Locale}\spxextra{class in tornado.locale}}

\begin{fulllineitems}
\phantomsection\label{\detokenize{locale:tornado.locale.Locale}}\pysiglinewithargsret{\sphinxbfcode{\sphinxupquote{class }}\sphinxcode{\sphinxupquote{tornado.locale.}}\sphinxbfcode{\sphinxupquote{Locale}}}{\emph{code: str}}{}
Object representing a locale.

After calling one of {\hyperref[\detokenize{locale:tornado.locale.load_translations}]{\sphinxcrossref{\sphinxcode{\sphinxupquote{load\_translations}}}}} or {\hyperref[\detokenize{locale:tornado.locale.load_gettext_translations}]{\sphinxcrossref{\sphinxcode{\sphinxupquote{load\_gettext\_translations}}}}},
call {\hyperref[\detokenize{locale:tornado.locale.get}]{\sphinxcrossref{\sphinxcode{\sphinxupquote{get}}}}} or {\hyperref[\detokenize{locale:tornado.locale.Locale.get_closest}]{\sphinxcrossref{\sphinxcode{\sphinxupquote{get\_closest}}}}} to get a Locale object.
\index{get\_closest() (tornado.locale.Locale class method)@\spxentry{get\_closest()}\spxextra{tornado.locale.Locale class method}}

\begin{fulllineitems}
\phantomsection\label{\detokenize{locale:tornado.locale.Locale.get_closest}}\pysiglinewithargsret{\sphinxbfcode{\sphinxupquote{classmethod }}\sphinxbfcode{\sphinxupquote{get\_closest}}}{\emph{*locale\_codes}}{{ $\rightarrow$ tornado.locale.Locale}}
Returns the closest match for the given locale code.

\end{fulllineitems}

\index{get() (tornado.locale.Locale class method)@\spxentry{get()}\spxextra{tornado.locale.Locale class method}}

\begin{fulllineitems}
\phantomsection\label{\detokenize{locale:tornado.locale.Locale.get}}\pysiglinewithargsret{\sphinxbfcode{\sphinxupquote{classmethod }}\sphinxbfcode{\sphinxupquote{get}}}{\emph{code: str}}{{ $\rightarrow$ tornado.locale.Locale}}
Returns the Locale for the given locale code.

If it is not supported, we raise an exception.

\end{fulllineitems}

\index{translate() (tornado.locale.Locale method)@\spxentry{translate()}\spxextra{tornado.locale.Locale method}}

\begin{fulllineitems}
\phantomsection\label{\detokenize{locale:tornado.locale.Locale.translate}}\pysiglinewithargsret{\sphinxbfcode{\sphinxupquote{translate}}}{\emph{message: str}, \emph{plural\_message: str = None}, \emph{count: int = None}}{{ $\rightarrow$ str}}
Returns the translation for the given message for this locale.

If \sphinxcode{\sphinxupquote{plural\_message}} is given, you must also provide
\sphinxcode{\sphinxupquote{count}}. We return \sphinxcode{\sphinxupquote{plural\_message}} when \sphinxcode{\sphinxupquote{count != 1}},
and we return the singular form for the given message when
\sphinxcode{\sphinxupquote{count == 1}}.

\end{fulllineitems}

\index{format\_date() (tornado.locale.Locale method)@\spxentry{format\_date()}\spxextra{tornado.locale.Locale method}}

\begin{fulllineitems}
\phantomsection\label{\detokenize{locale:tornado.locale.Locale.format_date}}\pysiglinewithargsret{\sphinxbfcode{\sphinxupquote{format\_date}}}{\emph{date: Union{[}int, float, datetime.datetime{]}, gmt\_offset: int = 0, relative: bool = True, shorter: bool = False, full\_format: bool = False}}{{ $\rightarrow$ str}}
Formats the given date (which should be GMT).

By default, we return a relative time (e.g., “2 minutes ago”). You
can return an absolute date string with \sphinxcode{\sphinxupquote{relative=False}}.

You can force a full format date (“July 10, 1980”) with
\sphinxcode{\sphinxupquote{full\_format=True}}.

This method is primarily intended for dates in the past.
For dates in the future, we fall back to full format.

\end{fulllineitems}

\index{format\_day() (tornado.locale.Locale method)@\spxentry{format\_day()}\spxextra{tornado.locale.Locale method}}

\begin{fulllineitems}
\phantomsection\label{\detokenize{locale:tornado.locale.Locale.format_day}}\pysiglinewithargsret{\sphinxbfcode{\sphinxupquote{format\_day}}}{\emph{date: datetime.datetime}, \emph{gmt\_offset: int = 0}, \emph{dow: bool = True}}{{ $\rightarrow$ bool}}
Formats the given date as a day of week.

Example: “Monday, January 22”. You can remove the day of week with
\sphinxcode{\sphinxupquote{dow=False}}.

\end{fulllineitems}

\index{list() (tornado.locale.Locale method)@\spxentry{list()}\spxextra{tornado.locale.Locale method}}

\begin{fulllineitems}
\phantomsection\label{\detokenize{locale:tornado.locale.Locale.list}}\pysiglinewithargsret{\sphinxbfcode{\sphinxupquote{list}}}{\emph{parts: Any}}{{ $\rightarrow$ str}}
Returns a comma-separated list for the given list of parts.

The format is, e.g., “A, B and C”, “A and B” or just “A” for lists
of size 1.

\end{fulllineitems}

\index{friendly\_number() (tornado.locale.Locale method)@\spxentry{friendly\_number()}\spxextra{tornado.locale.Locale method}}

\begin{fulllineitems}
\phantomsection\label{\detokenize{locale:tornado.locale.Locale.friendly_number}}\pysiglinewithargsret{\sphinxbfcode{\sphinxupquote{friendly\_number}}}{\emph{value: int}}{{ $\rightarrow$ str}}
Returns a comma-separated number for the given integer.

\end{fulllineitems}


\end{fulllineitems}

\index{CSVLocale (class in tornado.locale)@\spxentry{CSVLocale}\spxextra{class in tornado.locale}}

\begin{fulllineitems}
\phantomsection\label{\detokenize{locale:tornado.locale.CSVLocale}}\pysiglinewithargsret{\sphinxbfcode{\sphinxupquote{class }}\sphinxcode{\sphinxupquote{tornado.locale.}}\sphinxbfcode{\sphinxupquote{CSVLocale}}}{\emph{code: str, translations: Dict{[}str, Dict{[}str, str{]}{]}}}{}
Locale implementation using tornado’s CSV translation format.

\end{fulllineitems}

\index{GettextLocale (class in tornado.locale)@\spxentry{GettextLocale}\spxextra{class in tornado.locale}}

\begin{fulllineitems}
\phantomsection\label{\detokenize{locale:tornado.locale.GettextLocale}}\pysiglinewithargsret{\sphinxbfcode{\sphinxupquote{class }}\sphinxcode{\sphinxupquote{tornado.locale.}}\sphinxbfcode{\sphinxupquote{GettextLocale}}}{\emph{code: str}, \emph{translations: gettext.NullTranslations}}{}
Locale implementation using the \sphinxhref{https://docs.python.org/3.6/library/gettext.html\#module-gettext}{\sphinxcode{\sphinxupquote{gettext}}} module.
\index{pgettext() (tornado.locale.GettextLocale method)@\spxentry{pgettext()}\spxextra{tornado.locale.GettextLocale method}}

\begin{fulllineitems}
\phantomsection\label{\detokenize{locale:tornado.locale.GettextLocale.pgettext}}\pysiglinewithargsret{\sphinxbfcode{\sphinxupquote{pgettext}}}{\emph{context: str}, \emph{message: str}, \emph{plural\_message: str = None}, \emph{count: int = None}}{{ $\rightarrow$ str}}
Allows to set context for translation, accepts plural forms.

Usage example:

\begin{sphinxVerbatim}[commandchars=\\\{\}]
\PYG{n}{pgettext}\PYG{p}{(}\PYG{l+s+s2}{\PYGZdq{}}\PYG{l+s+s2}{law}\PYG{l+s+s2}{\PYGZdq{}}\PYG{p}{,} \PYG{l+s+s2}{\PYGZdq{}}\PYG{l+s+s2}{right}\PYG{l+s+s2}{\PYGZdq{}}\PYG{p}{)}
\PYG{n}{pgettext}\PYG{p}{(}\PYG{l+s+s2}{\PYGZdq{}}\PYG{l+s+s2}{good}\PYG{l+s+s2}{\PYGZdq{}}\PYG{p}{,} \PYG{l+s+s2}{\PYGZdq{}}\PYG{l+s+s2}{right}\PYG{l+s+s2}{\PYGZdq{}}\PYG{p}{)}
\end{sphinxVerbatim}

Plural message example:

\begin{sphinxVerbatim}[commandchars=\\\{\}]
\PYG{n}{pgettext}\PYG{p}{(}\PYG{l+s+s2}{\PYGZdq{}}\PYG{l+s+s2}{organization}\PYG{l+s+s2}{\PYGZdq{}}\PYG{p}{,} \PYG{l+s+s2}{\PYGZdq{}}\PYG{l+s+s2}{club}\PYG{l+s+s2}{\PYGZdq{}}\PYG{p}{,} \PYG{l+s+s2}{\PYGZdq{}}\PYG{l+s+s2}{clubs}\PYG{l+s+s2}{\PYGZdq{}}\PYG{p}{,} \PYG{n+nb}{len}\PYG{p}{(}\PYG{n}{clubs}\PYG{p}{)}\PYG{p}{)}
\PYG{n}{pgettext}\PYG{p}{(}\PYG{l+s+s2}{\PYGZdq{}}\PYG{l+s+s2}{stick}\PYG{l+s+s2}{\PYGZdq{}}\PYG{p}{,} \PYG{l+s+s2}{\PYGZdq{}}\PYG{l+s+s2}{club}\PYG{l+s+s2}{\PYGZdq{}}\PYG{p}{,} \PYG{l+s+s2}{\PYGZdq{}}\PYG{l+s+s2}{clubs}\PYG{l+s+s2}{\PYGZdq{}}\PYG{p}{,} \PYG{n+nb}{len}\PYG{p}{(}\PYG{n}{clubs}\PYG{p}{)}\PYG{p}{)}
\end{sphinxVerbatim}

To generate POT file with context, add following options to step 1
of {\hyperref[\detokenize{locale:tornado.locale.load_gettext_translations}]{\sphinxcrossref{\sphinxcode{\sphinxupquote{load\_gettext\_translations}}}}} sequence:

\begin{sphinxVerbatim}[commandchars=\\\{\}]
\PYG{n}{xgettext} \PYG{p}{[}\PYG{n}{basic} \PYG{n}{options}\PYG{p}{]} \PYG{o}{\PYGZhy{}}\PYG{o}{\PYGZhy{}}\PYG{n}{keyword}\PYG{o}{=}\PYG{n}{pgettext}\PYG{p}{:}\PYG{l+m+mi}{1}\PYG{n}{c}\PYG{p}{,}\PYG{l+m+mi}{2} \PYG{o}{\PYGZhy{}}\PYG{o}{\PYGZhy{}}\PYG{n}{keyword}\PYG{o}{=}\PYG{n}{pgettext}\PYG{p}{:}\PYG{l+m+mi}{1}\PYG{n}{c}\PYG{p}{,}\PYG{l+m+mi}{2}\PYG{p}{,}\PYG{l+m+mi}{3}
\end{sphinxVerbatim}

\DUrole{versionmodified,added}{New in version 4.2.}

\end{fulllineitems}


\end{fulllineitems}



\subsection{\sphinxstyleliteralintitle{\sphinxupquote{tornado.websocket}} — Bidirectional communication to the browser}
\label{\detokenize{websocket:tornado-websocket-bidirectional-communication-to-the-browser}}\label{\detokenize{websocket::doc}}\phantomsection\label{\detokenize{websocket:module-tornado.websocket}}\index{tornado.websocket (module)@\spxentry{tornado.websocket}\spxextra{module}}
Implementation of the WebSocket protocol.

\sphinxhref{http://dev.w3.org/html5/websockets/}{WebSockets} allow for bidirectional
communication between the browser and server.

WebSockets are supported in the current versions of all major browsers,
although older versions that do not support WebSockets are still in use
(refer to \sphinxurl{http://caniuse.com/websockets} for details).

This module implements the final version of the WebSocket protocol as
defined in \sphinxhref{http://tools.ietf.org/html/rfc6455}{RFC 6455}.  Certain
browser versions (notably Safari 5.x) implemented an earlier draft of
the protocol (known as “draft 76”) and are not compatible with this module.

\DUrole{versionmodified,changed}{Changed in version 4.0: }Removed support for the draft 76 protocol version.
\index{WebSocketHandler (class in tornado.websocket)@\spxentry{WebSocketHandler}\spxextra{class in tornado.websocket}}

\begin{fulllineitems}
\phantomsection\label{\detokenize{websocket:tornado.websocket.WebSocketHandler}}\pysiglinewithargsret{\sphinxbfcode{\sphinxupquote{class }}\sphinxcode{\sphinxupquote{tornado.websocket.}}\sphinxbfcode{\sphinxupquote{WebSocketHandler}}}{\emph{application: tornado.web.Application}, \emph{request: tornado.httputil.HTTPServerRequest}, \emph{**kwargs}}{}
Subclass this class to create a basic WebSocket handler.

Override {\hyperref[\detokenize{websocket:tornado.websocket.WebSocketHandler.on_message}]{\sphinxcrossref{\sphinxcode{\sphinxupquote{on\_message}}}}} to handle incoming messages, and use
{\hyperref[\detokenize{websocket:tornado.websocket.WebSocketHandler.write_message}]{\sphinxcrossref{\sphinxcode{\sphinxupquote{write\_message}}}}} to send messages to the client. You can also
override {\hyperref[\detokenize{websocket:tornado.websocket.WebSocketHandler.open}]{\sphinxcrossref{\sphinxcode{\sphinxupquote{open}}}}} and {\hyperref[\detokenize{websocket:tornado.websocket.WebSocketHandler.on_close}]{\sphinxcrossref{\sphinxcode{\sphinxupquote{on\_close}}}}} to handle opened and closed
connections.

Custom upgrade response headers can be sent by overriding
{\hyperref[\detokenize{web:tornado.web.RequestHandler.set_default_headers}]{\sphinxcrossref{\sphinxcode{\sphinxupquote{set\_default\_headers}}}}} or
{\hyperref[\detokenize{web:tornado.web.RequestHandler.prepare}]{\sphinxcrossref{\sphinxcode{\sphinxupquote{prepare}}}}}.

See \sphinxurl{http://dev.w3.org/html5/websockets/} for details on the
JavaScript interface.  The protocol is specified at
\sphinxurl{http://tools.ietf.org/html/rfc6455}.

Here is an example WebSocket handler that echos back all received messages
back to the client:

\begin{sphinxVerbatim}[commandchars=\\\{\}]
\PYG{k}{class} \PYG{n+nc}{EchoWebSocket}\PYG{p}{(}\PYG{n}{tornado}\PYG{o}{.}\PYG{n}{websocket}\PYG{o}{.}\PYG{n}{WebSocketHandler}\PYG{p}{)}\PYG{p}{:}
    \PYG{k}{def} \PYG{n+nf}{open}\PYG{p}{(}\PYG{n+nb+bp}{self}\PYG{p}{)}\PYG{p}{:}
        \PYG{n+nb}{print}\PYG{p}{(}\PYG{l+s+s2}{\PYGZdq{}}\PYG{l+s+s2}{WebSocket opened}\PYG{l+s+s2}{\PYGZdq{}}\PYG{p}{)}

    \PYG{k}{def} \PYG{n+nf}{on\PYGZus{}message}\PYG{p}{(}\PYG{n+nb+bp}{self}\PYG{p}{,} \PYG{n}{message}\PYG{p}{)}\PYG{p}{:}
        \PYG{n+nb+bp}{self}\PYG{o}{.}\PYG{n}{write\PYGZus{}message}\PYG{p}{(}\PYG{l+s+sa}{u}\PYG{l+s+s2}{\PYGZdq{}}\PYG{l+s+s2}{You said: }\PYG{l+s+s2}{\PYGZdq{}} \PYG{o}{+} \PYG{n}{message}\PYG{p}{)}

    \PYG{k}{def} \PYG{n+nf}{on\PYGZus{}close}\PYG{p}{(}\PYG{n+nb+bp}{self}\PYG{p}{)}\PYG{p}{:}
        \PYG{n+nb}{print}\PYG{p}{(}\PYG{l+s+s2}{\PYGZdq{}}\PYG{l+s+s2}{WebSocket closed}\PYG{l+s+s2}{\PYGZdq{}}\PYG{p}{)}
\end{sphinxVerbatim}

WebSockets are not standard HTTP connections. The “handshake” is
HTTP, but after the handshake, the protocol is
message-based. Consequently, most of the Tornado HTTP facilities
are not available in handlers of this type. The only communication
methods available to you are {\hyperref[\detokenize{websocket:tornado.websocket.WebSocketHandler.write_message}]{\sphinxcrossref{\sphinxcode{\sphinxupquote{write\_message()}}}}}, {\hyperref[\detokenize{websocket:tornado.websocket.WebSocketHandler.ping}]{\sphinxcrossref{\sphinxcode{\sphinxupquote{ping()}}}}}, and
{\hyperref[\detokenize{websocket:tornado.websocket.WebSocketHandler.close}]{\sphinxcrossref{\sphinxcode{\sphinxupquote{close()}}}}}. Likewise, your request handler class should implement
{\hyperref[\detokenize{websocket:tornado.websocket.WebSocketHandler.open}]{\sphinxcrossref{\sphinxcode{\sphinxupquote{open()}}}}} method rather than \sphinxcode{\sphinxupquote{get()}} or \sphinxcode{\sphinxupquote{post()}}.

If you map the handler above to \sphinxcode{\sphinxupquote{/websocket}} in your application, you can
invoke it in JavaScript with:

\begin{sphinxVerbatim}[commandchars=\\\{\}]
\PYG{n}{var} \PYG{n}{ws} \PYG{o}{=} \PYG{n}{new} \PYG{n}{WebSocket}\PYG{p}{(}\PYG{l+s+s2}{\PYGZdq{}}\PYG{l+s+s2}{ws://localhost:8888/websocket}\PYG{l+s+s2}{\PYGZdq{}}\PYG{p}{)}\PYG{p}{;}
\PYG{n}{ws}\PYG{o}{.}\PYG{n}{onopen} \PYG{o}{=} \PYG{n}{function}\PYG{p}{(}\PYG{p}{)} \PYG{p}{\PYGZob{}}
   \PYG{n}{ws}\PYG{o}{.}\PYG{n}{send}\PYG{p}{(}\PYG{l+s+s2}{\PYGZdq{}}\PYG{l+s+s2}{Hello, world}\PYG{l+s+s2}{\PYGZdq{}}\PYG{p}{)}\PYG{p}{;}
\PYG{p}{\PYGZcb{}}\PYG{p}{;}
\PYG{n}{ws}\PYG{o}{.}\PYG{n}{onmessage} \PYG{o}{=} \PYG{n}{function} \PYG{p}{(}\PYG{n}{evt}\PYG{p}{)} \PYG{p}{\PYGZob{}}
   \PYG{n}{alert}\PYG{p}{(}\PYG{n}{evt}\PYG{o}{.}\PYG{n}{data}\PYG{p}{)}\PYG{p}{;}
\PYG{p}{\PYGZcb{}}\PYG{p}{;}
\end{sphinxVerbatim}

This script pops up an alert box that says “You said: Hello, world”.

Web browsers allow any site to open a websocket connection to any other,
instead of using the same-origin policy that governs other network
access from javascript.  This can be surprising and is a potential
security hole, so since Tornado 4.0 {\hyperref[\detokenize{websocket:tornado.websocket.WebSocketHandler}]{\sphinxcrossref{\sphinxcode{\sphinxupquote{WebSocketHandler}}}}} requires
applications that wish to receive cross-origin websockets to opt in
by overriding the {\hyperref[\detokenize{websocket:tornado.websocket.WebSocketHandler.check_origin}]{\sphinxcrossref{\sphinxcode{\sphinxupquote{check\_origin}}}}} method (see that
method’s docs for details).  Failure to do so is the most likely
cause of 403 errors when making a websocket connection.

When using a secure websocket connection (\sphinxcode{\sphinxupquote{wss://}}) with a self-signed
certificate, the connection from a browser may fail because it wants
to show the “accept this certificate” dialog but has nowhere to show it.
You must first visit a regular HTML page using the same certificate
to accept it before the websocket connection will succeed.

If the application setting \sphinxcode{\sphinxupquote{websocket\_ping\_interval}} has a non-zero
value, a ping will be sent periodically, and the connection will be
closed if a response is not received before the \sphinxcode{\sphinxupquote{websocket\_ping\_timeout}}.

Messages larger than the \sphinxcode{\sphinxupquote{websocket\_max\_message\_size}} application setting
(default 10MiB) will not be accepted.

\DUrole{versionmodified,changed}{Changed in version 4.5: }Added \sphinxcode{\sphinxupquote{websocket\_ping\_interval}}, \sphinxcode{\sphinxupquote{websocket\_ping\_timeout}}, and
\sphinxcode{\sphinxupquote{websocket\_max\_message\_size}}.

\end{fulllineitems}



\subsubsection{Event handlers}
\label{\detokenize{websocket:event-handlers}}\index{open() (tornado.websocket.WebSocketHandler method)@\spxentry{open()}\spxextra{tornado.websocket.WebSocketHandler method}}

\begin{fulllineitems}
\phantomsection\label{\detokenize{websocket:tornado.websocket.WebSocketHandler.open}}\pysiglinewithargsret{\sphinxcode{\sphinxupquote{WebSocketHandler.}}\sphinxbfcode{\sphinxupquote{open}}}{\emph{*args}, \emph{**kwargs}}{{ $\rightarrow$ Optional{[}Awaitable{[}None{]}{]}}}
Invoked when a new WebSocket is opened.

The arguments to {\hyperref[\detokenize{websocket:tornado.websocket.WebSocketHandler.open}]{\sphinxcrossref{\sphinxcode{\sphinxupquote{open}}}}} are extracted from the {\hyperref[\detokenize{web:tornado.web.URLSpec}]{\sphinxcrossref{\sphinxcode{\sphinxupquote{tornado.web.URLSpec}}}}}
regular expression, just like the arguments to
{\hyperref[\detokenize{web:tornado.web.RequestHandler.get}]{\sphinxcrossref{\sphinxcode{\sphinxupquote{tornado.web.RequestHandler.get}}}}}.

{\hyperref[\detokenize{websocket:tornado.websocket.WebSocketHandler.open}]{\sphinxcrossref{\sphinxcode{\sphinxupquote{open}}}}} may be a coroutine. {\hyperref[\detokenize{websocket:tornado.websocket.WebSocketHandler.on_message}]{\sphinxcrossref{\sphinxcode{\sphinxupquote{on\_message}}}}} will not be called until
{\hyperref[\detokenize{websocket:tornado.websocket.WebSocketHandler.open}]{\sphinxcrossref{\sphinxcode{\sphinxupquote{open}}}}} has returned.

\DUrole{versionmodified,changed}{Changed in version 5.1: }\sphinxcode{\sphinxupquote{open}} may be a coroutine.

\end{fulllineitems}

\index{on\_message() (tornado.websocket.WebSocketHandler method)@\spxentry{on\_message()}\spxextra{tornado.websocket.WebSocketHandler method}}

\begin{fulllineitems}
\phantomsection\label{\detokenize{websocket:tornado.websocket.WebSocketHandler.on_message}}\pysiglinewithargsret{\sphinxcode{\sphinxupquote{WebSocketHandler.}}\sphinxbfcode{\sphinxupquote{on\_message}}}{\emph{message: Union{[}str, bytes{]}}}{{ $\rightarrow$ Optional{[}Awaitable{[}None{]}{]}}}
Handle incoming messages on the WebSocket

This method must be overridden.

\DUrole{versionmodified,changed}{Changed in version 4.5: }\sphinxcode{\sphinxupquote{on\_message}} can be a coroutine.

\end{fulllineitems}

\index{on\_close() (tornado.websocket.WebSocketHandler method)@\spxentry{on\_close()}\spxextra{tornado.websocket.WebSocketHandler method}}

\begin{fulllineitems}
\phantomsection\label{\detokenize{websocket:tornado.websocket.WebSocketHandler.on_close}}\pysiglinewithargsret{\sphinxcode{\sphinxupquote{WebSocketHandler.}}\sphinxbfcode{\sphinxupquote{on\_close}}}{}{{ $\rightarrow$ None}}
Invoked when the WebSocket is closed.

If the connection was closed cleanly and a status code or reason
phrase was supplied, these values will be available as the attributes
\sphinxcode{\sphinxupquote{self.close\_code}} and \sphinxcode{\sphinxupquote{self.close\_reason}}.

\DUrole{versionmodified,changed}{Changed in version 4.0: }Added \sphinxcode{\sphinxupquote{close\_code}} and \sphinxcode{\sphinxupquote{close\_reason}} attributes.

\end{fulllineitems}

\index{select\_subprotocol() (tornado.websocket.WebSocketHandler method)@\spxentry{select\_subprotocol()}\spxextra{tornado.websocket.WebSocketHandler method}}

\begin{fulllineitems}
\phantomsection\label{\detokenize{websocket:tornado.websocket.WebSocketHandler.select_subprotocol}}\pysiglinewithargsret{\sphinxcode{\sphinxupquote{WebSocketHandler.}}\sphinxbfcode{\sphinxupquote{select\_subprotocol}}}{\emph{subprotocols: List{[}str{]}}}{{ $\rightarrow$ Optional{[}str{]}}}
Override to implement subprotocol negotiation.

\sphinxcode{\sphinxupquote{subprotocols}} is a list of strings identifying the
subprotocols proposed by the client.  This method may be
overridden to return one of those strings to select it, or
\sphinxcode{\sphinxupquote{None}} to not select a subprotocol.

Failure to select a subprotocol does not automatically abort
the connection, although clients may close the connection if
none of their proposed subprotocols was selected.

The list may be empty, in which case this method must return
None. This method is always called exactly once even if no
subprotocols were proposed so that the handler can be advised
of this fact.

\DUrole{versionmodified,changed}{Changed in version 5.1: }Previously, this method was called with a list containing
an empty string instead of an empty list if no subprotocols
were proposed by the client.

\end{fulllineitems}

\index{selected\_subprotocol (tornado.websocket.WebSocketHandler attribute)@\spxentry{selected\_subprotocol}\spxextra{tornado.websocket.WebSocketHandler attribute}}

\begin{fulllineitems}
\phantomsection\label{\detokenize{websocket:tornado.websocket.WebSocketHandler.selected_subprotocol}}\pysigline{\sphinxcode{\sphinxupquote{WebSocketHandler.}}\sphinxbfcode{\sphinxupquote{selected\_subprotocol}}}
The subprotocol returned by {\hyperref[\detokenize{websocket:tornado.websocket.WebSocketHandler.select_subprotocol}]{\sphinxcrossref{\sphinxcode{\sphinxupquote{select\_subprotocol}}}}}.

\DUrole{versionmodified,added}{New in version 5.1.}

\end{fulllineitems}

\index{on\_ping() (tornado.websocket.WebSocketHandler method)@\spxentry{on\_ping()}\spxextra{tornado.websocket.WebSocketHandler method}}

\begin{fulllineitems}
\phantomsection\label{\detokenize{websocket:tornado.websocket.WebSocketHandler.on_ping}}\pysiglinewithargsret{\sphinxcode{\sphinxupquote{WebSocketHandler.}}\sphinxbfcode{\sphinxupquote{on\_ping}}}{\emph{data: bytes}}{{ $\rightarrow$ None}}
Invoked when the a ping frame is received.

\end{fulllineitems}



\subsubsection{Output}
\label{\detokenize{websocket:output}}\index{write\_message() (tornado.websocket.WebSocketHandler method)@\spxentry{write\_message()}\spxextra{tornado.websocket.WebSocketHandler method}}

\begin{fulllineitems}
\phantomsection\label{\detokenize{websocket:tornado.websocket.WebSocketHandler.write_message}}\pysiglinewithargsret{\sphinxcode{\sphinxupquote{WebSocketHandler.}}\sphinxbfcode{\sphinxupquote{write\_message}}}{\emph{message: Union{[}bytes, str, Dict{[}str, Any{]}{]}, binary: bool = False}}{{ $\rightarrow$ Future{[}None{]}}}
Sends the given message to the client of this Web Socket.

The message may be either a string or a dict (which will be
encoded as json).  If the \sphinxcode{\sphinxupquote{binary}} argument is false, the
message will be sent as utf8; in binary mode any byte string
is allowed.

If the connection is already closed, raises {\hyperref[\detokenize{websocket:tornado.websocket.WebSocketClosedError}]{\sphinxcrossref{\sphinxcode{\sphinxupquote{WebSocketClosedError}}}}}.
Returns a {\hyperref[\detokenize{concurrent:tornado.concurrent.Future}]{\sphinxcrossref{\sphinxcode{\sphinxupquote{Future}}}}} which can be used for flow control.

\DUrole{versionmodified,changed}{Changed in version 3.2: }{\hyperref[\detokenize{websocket:tornado.websocket.WebSocketClosedError}]{\sphinxcrossref{\sphinxcode{\sphinxupquote{WebSocketClosedError}}}}} was added (previously a closed connection
would raise an \sphinxhref{https://docs.python.org/3.6/library/exceptions.html\#AttributeError}{\sphinxcode{\sphinxupquote{AttributeError}}})

\DUrole{versionmodified,changed}{Changed in version 4.3: }Returns a {\hyperref[\detokenize{concurrent:tornado.concurrent.Future}]{\sphinxcrossref{\sphinxcode{\sphinxupquote{Future}}}}} which can be used for flow control.

\DUrole{versionmodified,changed}{Changed in version 5.0: }Consistently raises {\hyperref[\detokenize{websocket:tornado.websocket.WebSocketClosedError}]{\sphinxcrossref{\sphinxcode{\sphinxupquote{WebSocketClosedError}}}}}. Previously could
sometimes raise {\hyperref[\detokenize{iostream:tornado.iostream.StreamClosedError}]{\sphinxcrossref{\sphinxcode{\sphinxupquote{StreamClosedError}}}}}.

\end{fulllineitems}

\index{close() (tornado.websocket.WebSocketHandler method)@\spxentry{close()}\spxextra{tornado.websocket.WebSocketHandler method}}

\begin{fulllineitems}
\phantomsection\label{\detokenize{websocket:tornado.websocket.WebSocketHandler.close}}\pysiglinewithargsret{\sphinxcode{\sphinxupquote{WebSocketHandler.}}\sphinxbfcode{\sphinxupquote{close}}}{\emph{code: int = None}, \emph{reason: str = None}}{{ $\rightarrow$ None}}
Closes this Web Socket.

Once the close handshake is successful the socket will be closed.

\sphinxcode{\sphinxupquote{code}} may be a numeric status code, taken from the values
defined in \sphinxhref{https://tools.ietf.org/html/rfc6455\#section-7.4.1}{RFC 6455 section 7.4.1}.
\sphinxcode{\sphinxupquote{reason}} may be a textual message about why the connection is
closing.  These values are made available to the client, but are
not otherwise interpreted by the websocket protocol.

\DUrole{versionmodified,changed}{Changed in version 4.0: }Added the \sphinxcode{\sphinxupquote{code}} and \sphinxcode{\sphinxupquote{reason}} arguments.

\end{fulllineitems}



\subsubsection{Configuration}
\label{\detokenize{websocket:configuration}}\index{check\_origin() (tornado.websocket.WebSocketHandler method)@\spxentry{check\_origin()}\spxextra{tornado.websocket.WebSocketHandler method}}

\begin{fulllineitems}
\phantomsection\label{\detokenize{websocket:tornado.websocket.WebSocketHandler.check_origin}}\pysiglinewithargsret{\sphinxcode{\sphinxupquote{WebSocketHandler.}}\sphinxbfcode{\sphinxupquote{check\_origin}}}{\emph{origin: str}}{{ $\rightarrow$ bool}}
Override to enable support for allowing alternate origins.

The \sphinxcode{\sphinxupquote{origin}} argument is the value of the \sphinxcode{\sphinxupquote{Origin}} HTTP
header, the url responsible for initiating this request.  This
method is not called for clients that do not send this header;
such requests are always allowed (because all browsers that
implement WebSockets support this header, and non-browser
clients do not have the same cross-site security concerns).

Should return \sphinxcode{\sphinxupquote{True}} to accept the request or \sphinxcode{\sphinxupquote{False}} to
reject it. By default, rejects all requests with an origin on
a host other than this one.

This is a security protection against cross site scripting attacks on
browsers, since WebSockets are allowed to bypass the usual same-origin
policies and don’t use CORS headers.

\begin{sphinxadmonition}{warning}{Warning:}
This is an important security measure; don’t disable it
without understanding the security implications. In
particular, if your authentication is cookie-based, you
must either restrict the origins allowed by
\sphinxcode{\sphinxupquote{check\_origin()}} or implement your own XSRF-like
protection for websocket connections. See \sphinxhref{https://www.christian-schneider.net/CrossSiteWebSocketHijacking.html}{these}
\sphinxhref{https://devcenter.heroku.com/articles/websocket-security}{articles}
for more.
\end{sphinxadmonition}

To accept all cross-origin traffic (which was the default prior to
Tornado 4.0), simply override this method to always return \sphinxcode{\sphinxupquote{True}}:

\begin{sphinxVerbatim}[commandchars=\\\{\}]
\PYG{k}{def} \PYG{n+nf}{check\PYGZus{}origin}\PYG{p}{(}\PYG{n+nb+bp}{self}\PYG{p}{,} \PYG{n}{origin}\PYG{p}{)}\PYG{p}{:}
    \PYG{k}{return} \PYG{k+kc}{True}
\end{sphinxVerbatim}

To allow connections from any subdomain of your site, you might
do something like:

\begin{sphinxVerbatim}[commandchars=\\\{\}]
\PYG{k}{def} \PYG{n+nf}{check\PYGZus{}origin}\PYG{p}{(}\PYG{n+nb+bp}{self}\PYG{p}{,} \PYG{n}{origin}\PYG{p}{)}\PYG{p}{:}
    \PYG{n}{parsed\PYGZus{}origin} \PYG{o}{=} \PYG{n}{urllib}\PYG{o}{.}\PYG{n}{parse}\PYG{o}{.}\PYG{n}{urlparse}\PYG{p}{(}\PYG{n}{origin}\PYG{p}{)}
    \PYG{k}{return} \PYG{n}{parsed\PYGZus{}origin}\PYG{o}{.}\PYG{n}{netloc}\PYG{o}{.}\PYG{n}{endswith}\PYG{p}{(}\PYG{l+s+s2}{\PYGZdq{}}\PYG{l+s+s2}{.mydomain.com}\PYG{l+s+s2}{\PYGZdq{}}\PYG{p}{)}
\end{sphinxVerbatim}

\DUrole{versionmodified,added}{New in version 4.0.}

\end{fulllineitems}

\index{get\_compression\_options() (tornado.websocket.WebSocketHandler method)@\spxentry{get\_compression\_options()}\spxextra{tornado.websocket.WebSocketHandler method}}

\begin{fulllineitems}
\phantomsection\label{\detokenize{websocket:tornado.websocket.WebSocketHandler.get_compression_options}}\pysiglinewithargsret{\sphinxcode{\sphinxupquote{WebSocketHandler.}}\sphinxbfcode{\sphinxupquote{get\_compression\_options}}}{}{{ $\rightarrow$ Optional{[}Dict{[}str, Any{]}{]}}}
Override to return compression options for the connection.

If this method returns None (the default), compression will
be disabled.  If it returns a dict (even an empty one), it
will be enabled.  The contents of the dict may be used to
control the following compression options:

\sphinxcode{\sphinxupquote{compression\_level}} specifies the compression level.

\sphinxcode{\sphinxupquote{mem\_level}} specifies the amount of memory used for the internal compression state.
\begin{quote}

These parameters are documented in details here:
\sphinxurl{https://docs.python.org/3.6/library/zlib.html\#zlib.compressobj}
\end{quote}

\DUrole{versionmodified,added}{New in version 4.1.}

\DUrole{versionmodified,changed}{Changed in version 4.5: }Added \sphinxcode{\sphinxupquote{compression\_level}} and \sphinxcode{\sphinxupquote{mem\_level}}.

\end{fulllineitems}

\index{set\_nodelay() (tornado.websocket.WebSocketHandler method)@\spxentry{set\_nodelay()}\spxextra{tornado.websocket.WebSocketHandler method}}

\begin{fulllineitems}
\phantomsection\label{\detokenize{websocket:tornado.websocket.WebSocketHandler.set_nodelay}}\pysiglinewithargsret{\sphinxcode{\sphinxupquote{WebSocketHandler.}}\sphinxbfcode{\sphinxupquote{set\_nodelay}}}{\emph{value: bool}}{{ $\rightarrow$ None}}
Set the no-delay flag for this stream.

By default, small messages may be delayed and/or combined to minimize
the number of packets sent.  This can sometimes cause 200-500ms delays
due to the interaction between Nagle’s algorithm and TCP delayed
ACKs.  To reduce this delay (at the expense of possibly increasing
bandwidth usage), call \sphinxcode{\sphinxupquote{self.set\_nodelay(True)}} once the websocket
connection is established.

See {\hyperref[\detokenize{iostream:tornado.iostream.BaseIOStream.set_nodelay}]{\sphinxcrossref{\sphinxcode{\sphinxupquote{BaseIOStream.set\_nodelay}}}}} for additional details.

\DUrole{versionmodified,added}{New in version 3.1.}

\end{fulllineitems}



\subsubsection{Other}
\label{\detokenize{websocket:other}}\index{ping() (tornado.websocket.WebSocketHandler method)@\spxentry{ping()}\spxextra{tornado.websocket.WebSocketHandler method}}

\begin{fulllineitems}
\phantomsection\label{\detokenize{websocket:tornado.websocket.WebSocketHandler.ping}}\pysiglinewithargsret{\sphinxcode{\sphinxupquote{WebSocketHandler.}}\sphinxbfcode{\sphinxupquote{ping}}}{\emph{data: Union{[}str}, \emph{bytes{]} = b''}}{{ $\rightarrow$ None}}
Send ping frame to the remote end.

The data argument allows a small amount of data (up to 125
bytes) to be sent as a part of the ping message. Note that not
all websocket implementations expose this data to
applications.

Consider using the \sphinxcode{\sphinxupquote{websocket\_ping\_interval}} application
setting instead of sending pings manually.

\DUrole{versionmodified,changed}{Changed in version 5.1: }The data argument is now optional.

\end{fulllineitems}

\index{on\_pong() (tornado.websocket.WebSocketHandler method)@\spxentry{on\_pong()}\spxextra{tornado.websocket.WebSocketHandler method}}

\begin{fulllineitems}
\phantomsection\label{\detokenize{websocket:tornado.websocket.WebSocketHandler.on_pong}}\pysiglinewithargsret{\sphinxcode{\sphinxupquote{WebSocketHandler.}}\sphinxbfcode{\sphinxupquote{on\_pong}}}{\emph{data: bytes}}{{ $\rightarrow$ None}}
Invoked when the response to a ping frame is received.

\end{fulllineitems}

\index{WebSocketClosedError@\spxentry{WebSocketClosedError}}

\begin{fulllineitems}
\phantomsection\label{\detokenize{websocket:tornado.websocket.WebSocketClosedError}}\pysigline{\sphinxbfcode{\sphinxupquote{exception }}\sphinxcode{\sphinxupquote{tornado.websocket.}}\sphinxbfcode{\sphinxupquote{WebSocketClosedError}}}
Raised by operations on a closed connection.

\DUrole{versionmodified,added}{New in version 3.2.}

\end{fulllineitems}



\subsubsection{Client-side support}
\label{\detokenize{websocket:client-side-support}}\index{websocket\_connect() (in module tornado.websocket)@\spxentry{websocket\_connect()}\spxextra{in module tornado.websocket}}

\begin{fulllineitems}
\phantomsection\label{\detokenize{websocket:tornado.websocket.websocket_connect}}\pysiglinewithargsret{\sphinxcode{\sphinxupquote{tornado.websocket.}}\sphinxbfcode{\sphinxupquote{websocket\_connect}}}{\emph{url: Union{[}str, tornado.httpclient.HTTPRequest{]}, callback: Callable{[}{[}Future{[}WebSocketClientConnection{]}{]}, None{]} = None, connect\_timeout: float = None, on\_message\_callback: Callable{[}{[}Union{[}None, str, bytes{]}{]}, None{]} = None, compression\_options: Dict{[}str, Any{]} = None, ping\_interval: float = None, ping\_timeout: float = None, max\_message\_size: int = 10485760, subprotocols: List{[}str{]} = None}}{{ $\rightarrow$ Awaitable{[}WebSocketClientConnection{]}}}
Client-side websocket support.

Takes a url and returns a Future whose result is a
{\hyperref[\detokenize{websocket:tornado.websocket.WebSocketClientConnection}]{\sphinxcrossref{\sphinxcode{\sphinxupquote{WebSocketClientConnection}}}}}.

\sphinxcode{\sphinxupquote{compression\_options}} is interpreted in the same way as the
return value of {\hyperref[\detokenize{websocket:tornado.websocket.WebSocketHandler.get_compression_options}]{\sphinxcrossref{\sphinxcode{\sphinxupquote{WebSocketHandler.get\_compression\_options}}}}}.

The connection supports two styles of operation. In the coroutine
style, the application typically calls
{\hyperref[\detokenize{websocket:tornado.websocket.WebSocketClientConnection.read_message}]{\sphinxcrossref{\sphinxcode{\sphinxupquote{read\_message}}}}} in a loop:

\begin{sphinxVerbatim}[commandchars=\\\{\}]
\PYG{n}{conn} \PYG{o}{=} \PYG{k}{yield} \PYG{n}{websocket\PYGZus{}connect}\PYG{p}{(}\PYG{n}{url}\PYG{p}{)}
\PYG{k}{while} \PYG{k+kc}{True}\PYG{p}{:}
    \PYG{n}{msg} \PYG{o}{=} \PYG{k}{yield} \PYG{n}{conn}\PYG{o}{.}\PYG{n}{read\PYGZus{}message}\PYG{p}{(}\PYG{p}{)}
    \PYG{k}{if} \PYG{n}{msg} \PYG{o+ow}{is} \PYG{k+kc}{None}\PYG{p}{:} \PYG{k}{break}
    \PYG{c+c1}{\PYGZsh{} Do something with msg}
\end{sphinxVerbatim}

In the callback style, pass an \sphinxcode{\sphinxupquote{on\_message\_callback}} to
\sphinxcode{\sphinxupquote{websocket\_connect}}. In both styles, a message of \sphinxcode{\sphinxupquote{None}}
indicates that the connection has been closed.

\sphinxcode{\sphinxupquote{subprotocols}} may be a list of strings specifying proposed
subprotocols. The selected protocol may be found on the
\sphinxcode{\sphinxupquote{selected\_subprotocol}} attribute of the connection object
when the connection is complete.

\DUrole{versionmodified,changed}{Changed in version 3.2: }Also accepts \sphinxcode{\sphinxupquote{HTTPRequest}} objects in place of urls.

\DUrole{versionmodified,changed}{Changed in version 4.1: }Added \sphinxcode{\sphinxupquote{compression\_options}} and \sphinxcode{\sphinxupquote{on\_message\_callback}}.

\DUrole{versionmodified,changed}{Changed in version 4.5: }Added the \sphinxcode{\sphinxupquote{ping\_interval}}, \sphinxcode{\sphinxupquote{ping\_timeout}}, and \sphinxcode{\sphinxupquote{max\_message\_size}}
arguments, which have the same meaning as in {\hyperref[\detokenize{websocket:tornado.websocket.WebSocketHandler}]{\sphinxcrossref{\sphinxcode{\sphinxupquote{WebSocketHandler}}}}}.

\DUrole{versionmodified,changed}{Changed in version 5.0: }The \sphinxcode{\sphinxupquote{io\_loop}} argument (deprecated since version 4.1) has been removed.

\DUrole{versionmodified,changed}{Changed in version 5.1: }Added the \sphinxcode{\sphinxupquote{subprotocols}} argument.

\end{fulllineitems}

\index{WebSocketClientConnection (class in tornado.websocket)@\spxentry{WebSocketClientConnection}\spxextra{class in tornado.websocket}}

\begin{fulllineitems}
\phantomsection\label{\detokenize{websocket:tornado.websocket.WebSocketClientConnection}}\pysiglinewithargsret{\sphinxbfcode{\sphinxupquote{class }}\sphinxcode{\sphinxupquote{tornado.websocket.}}\sphinxbfcode{\sphinxupquote{WebSocketClientConnection}}}{\emph{request: tornado.httpclient.HTTPRequest, on\_message\_callback: Callable{[}{[}Union{[}None, str, bytes{]}{]}, None{]} = None, compression\_options: Dict{[}str, Any{]} = None, ping\_interval: float = None, ping\_timeout: float = None, max\_message\_size: int = 10485760, subprotocols: Optional{[}List{[}str{]}{]} = {[}{]}}}{}
WebSocket client connection.

This class should not be instantiated directly; use the
{\hyperref[\detokenize{websocket:tornado.websocket.websocket_connect}]{\sphinxcrossref{\sphinxcode{\sphinxupquote{websocket\_connect}}}}} function instead.
\index{close() (tornado.websocket.WebSocketClientConnection method)@\spxentry{close()}\spxextra{tornado.websocket.WebSocketClientConnection method}}

\begin{fulllineitems}
\phantomsection\label{\detokenize{websocket:tornado.websocket.WebSocketClientConnection.close}}\pysiglinewithargsret{\sphinxbfcode{\sphinxupquote{close}}}{\emph{code: int = None}, \emph{reason: str = None}}{{ $\rightarrow$ None}}
Closes the websocket connection.

\sphinxcode{\sphinxupquote{code}} and \sphinxcode{\sphinxupquote{reason}} are documented under
{\hyperref[\detokenize{websocket:tornado.websocket.WebSocketHandler.close}]{\sphinxcrossref{\sphinxcode{\sphinxupquote{WebSocketHandler.close}}}}}.

\DUrole{versionmodified,added}{New in version 3.2.}

\DUrole{versionmodified,changed}{Changed in version 4.0: }Added the \sphinxcode{\sphinxupquote{code}} and \sphinxcode{\sphinxupquote{reason}} arguments.

\end{fulllineitems}

\index{write\_message() (tornado.websocket.WebSocketClientConnection method)@\spxentry{write\_message()}\spxextra{tornado.websocket.WebSocketClientConnection method}}

\begin{fulllineitems}
\phantomsection\label{\detokenize{websocket:tornado.websocket.WebSocketClientConnection.write_message}}\pysiglinewithargsret{\sphinxbfcode{\sphinxupquote{write\_message}}}{\emph{message: Union{[}str, bytes{]}, binary: bool = False}}{{ $\rightarrow$ Future{[}None{]}}}
Sends a message to the WebSocket server.

If the stream is closed, raises {\hyperref[\detokenize{websocket:tornado.websocket.WebSocketClosedError}]{\sphinxcrossref{\sphinxcode{\sphinxupquote{WebSocketClosedError}}}}}.
Returns a {\hyperref[\detokenize{concurrent:tornado.concurrent.Future}]{\sphinxcrossref{\sphinxcode{\sphinxupquote{Future}}}}} which can be used for flow control.

\DUrole{versionmodified,changed}{Changed in version 5.0: }Exception raised on a closed stream changed from {\hyperref[\detokenize{iostream:tornado.iostream.StreamClosedError}]{\sphinxcrossref{\sphinxcode{\sphinxupquote{StreamClosedError}}}}}
to {\hyperref[\detokenize{websocket:tornado.websocket.WebSocketClosedError}]{\sphinxcrossref{\sphinxcode{\sphinxupquote{WebSocketClosedError}}}}}.

\end{fulllineitems}

\index{read\_message() (tornado.websocket.WebSocketClientConnection method)@\spxentry{read\_message()}\spxextra{tornado.websocket.WebSocketClientConnection method}}

\begin{fulllineitems}
\phantomsection\label{\detokenize{websocket:tornado.websocket.WebSocketClientConnection.read_message}}\pysiglinewithargsret{\sphinxbfcode{\sphinxupquote{read\_message}}}{\emph{callback: Callable{[}{[}Future{[}Union{[}None, str, bytes{]}{]}{]}, None{]} = None}}{{ $\rightarrow$ Awaitable{[}Union{[}None, str, bytes{]}{]}}}
Reads a message from the WebSocket server.

If on\_message\_callback was specified at WebSocket
initialization, this function will never return messages

Returns a future whose result is the message, or None
if the connection is closed.  If a callback argument
is given it will be called with the future when it is
ready.

\end{fulllineitems}

\index{ping() (tornado.websocket.WebSocketClientConnection method)@\spxentry{ping()}\spxextra{tornado.websocket.WebSocketClientConnection method}}

\begin{fulllineitems}
\phantomsection\label{\detokenize{websocket:tornado.websocket.WebSocketClientConnection.ping}}\pysiglinewithargsret{\sphinxbfcode{\sphinxupquote{ping}}}{\emph{data: bytes = b''}}{{ $\rightarrow$ None}}
Send ping frame to the remote end.

The data argument allows a small amount of data (up to 125
bytes) to be sent as a part of the ping message. Note that not
all websocket implementations expose this data to
applications.

Consider using the \sphinxcode{\sphinxupquote{ping\_interval}} argument to
{\hyperref[\detokenize{websocket:tornado.websocket.websocket_connect}]{\sphinxcrossref{\sphinxcode{\sphinxupquote{websocket\_connect}}}}} instead of sending pings manually.

\DUrole{versionmodified,added}{New in version 5.1.}

\end{fulllineitems}

\index{selected\_subprotocol (tornado.websocket.WebSocketClientConnection attribute)@\spxentry{selected\_subprotocol}\spxextra{tornado.websocket.WebSocketClientConnection attribute}}

\begin{fulllineitems}
\phantomsection\label{\detokenize{websocket:tornado.websocket.WebSocketClientConnection.selected_subprotocol}}\pysigline{\sphinxbfcode{\sphinxupquote{selected\_subprotocol}}}
The subprotocol selected by the server.

\DUrole{versionmodified,added}{New in version 5.1.}

\end{fulllineitems}


\end{fulllineitems}



\section{HTTP servers and clients}
\label{\detokenize{http:http-servers-and-clients}}\label{\detokenize{http::doc}}

\subsection{\sphinxstyleliteralintitle{\sphinxupquote{tornado.httpserver}} — Non-blocking HTTP server}
\label{\detokenize{httpserver:module-tornado.httpserver}}\label{\detokenize{httpserver:tornado-httpserver-non-blocking-http-server}}\label{\detokenize{httpserver::doc}}\index{tornado.httpserver (module)@\spxentry{tornado.httpserver}\spxextra{module}}
A non-blocking, single-threaded HTTP server.

Typical applications have little direct interaction with the {\hyperref[\detokenize{httpserver:tornado.httpserver.HTTPServer}]{\sphinxcrossref{\sphinxcode{\sphinxupquote{HTTPServer}}}}}
class except to start a server at the beginning of the process
(and even that is often done indirectly via {\hyperref[\detokenize{web:tornado.web.Application.listen}]{\sphinxcrossref{\sphinxcode{\sphinxupquote{tornado.web.Application.listen}}}}}).

\DUrole{versionmodified,changed}{Changed in version 4.0: }The \sphinxcode{\sphinxupquote{HTTPRequest}} class that used to live in this module has been moved
to {\hyperref[\detokenize{httputil:tornado.httputil.HTTPServerRequest}]{\sphinxcrossref{\sphinxcode{\sphinxupquote{tornado.httputil.HTTPServerRequest}}}}}.  The old name remains as an alias.


\subsubsection{HTTP Server}
\label{\detokenize{httpserver:http-server}}\index{HTTPServer (class in tornado.httpserver)@\spxentry{HTTPServer}\spxextra{class in tornado.httpserver}}

\begin{fulllineitems}
\phantomsection\label{\detokenize{httpserver:tornado.httpserver.HTTPServer}}\pysiglinewithargsret{\sphinxbfcode{\sphinxupquote{class }}\sphinxcode{\sphinxupquote{tornado.httpserver.}}\sphinxbfcode{\sphinxupquote{HTTPServer}}}{\emph{request\_callback: Union{[}httputil.HTTPServerConnectionDelegate, Callable{[}{[}httputil.HTTPServerRequest{]}, None{]}{]}, no\_keep\_alive: bool = False, xheaders: bool = False, ssl\_options: Union{[}Dict{[}str, Any{]}, ssl.SSLContext{]} = None, protocol: str = None, decompress\_request: bool = False, chunk\_size: int = None, max\_header\_size: int = None, idle\_connection\_timeout: float = None, body\_timeout: float = None, max\_body\_size: int = None, max\_buffer\_size: int = None, trusted\_downstream: List{[}str{]} = None}}{}
A non-blocking, single-threaded HTTP server.

A server is defined by a subclass of {\hyperref[\detokenize{httputil:tornado.httputil.HTTPServerConnectionDelegate}]{\sphinxcrossref{\sphinxcode{\sphinxupquote{HTTPServerConnectionDelegate}}}}},
or, for backwards compatibility, a callback that takes an
{\hyperref[\detokenize{httputil:tornado.httputil.HTTPServerRequest}]{\sphinxcrossref{\sphinxcode{\sphinxupquote{HTTPServerRequest}}}}} as an argument. The delegate is usually a
{\hyperref[\detokenize{web:tornado.web.Application}]{\sphinxcrossref{\sphinxcode{\sphinxupquote{tornado.web.Application}}}}}.

{\hyperref[\detokenize{httpserver:tornado.httpserver.HTTPServer}]{\sphinxcrossref{\sphinxcode{\sphinxupquote{HTTPServer}}}}} supports keep-alive connections by default
(automatically for HTTP/1.1, or for HTTP/1.0 when the client
requests \sphinxcode{\sphinxupquote{Connection: keep-alive}}).

If \sphinxcode{\sphinxupquote{xheaders}} is \sphinxcode{\sphinxupquote{True}}, we support the
\sphinxcode{\sphinxupquote{X-Real-Ip}}/\sphinxcode{\sphinxupquote{X-Forwarded-For}} and
\sphinxcode{\sphinxupquote{X-Scheme}}/\sphinxcode{\sphinxupquote{X-Forwarded-Proto}} headers, which override the
remote IP and URI scheme/protocol for all requests.  These headers
are useful when running Tornado behind a reverse proxy or load
balancer.  The \sphinxcode{\sphinxupquote{protocol}} argument can also be set to \sphinxcode{\sphinxupquote{https}}
if Tornado is run behind an SSL-decoding proxy that does not set one of
the supported \sphinxcode{\sphinxupquote{xheaders}}.

By default, when parsing the \sphinxcode{\sphinxupquote{X-Forwarded-For}} header, Tornado will
select the last (i.e., the closest) address on the list of hosts as the
remote host IP address.  To select the next server in the chain, a list of
trusted downstream hosts may be passed as the \sphinxcode{\sphinxupquote{trusted\_downstream}}
argument.  These hosts will be skipped when parsing the \sphinxcode{\sphinxupquote{X-Forwarded-For}}
header.

To make this server serve SSL traffic, send the \sphinxcode{\sphinxupquote{ssl\_options}} keyword
argument with an \sphinxhref{https://docs.python.org/3.6/library/ssl.html\#ssl.SSLContext}{\sphinxcode{\sphinxupquote{ssl.SSLContext}}} object. For compatibility with older
versions of Python \sphinxcode{\sphinxupquote{ssl\_options}} may also be a dictionary of keyword
arguments for the \sphinxhref{https://docs.python.org/3.6/library/ssl.html\#ssl.wrap\_socket}{\sphinxcode{\sphinxupquote{ssl.wrap\_socket}}} method.:

\begin{sphinxVerbatim}[commandchars=\\\{\}]
\PYG{n}{ssl\PYGZus{}ctx} \PYG{o}{=} \PYG{n}{ssl}\PYG{o}{.}\PYG{n}{create\PYGZus{}default\PYGZus{}context}\PYG{p}{(}\PYG{n}{ssl}\PYG{o}{.}\PYG{n}{Purpose}\PYG{o}{.}\PYG{n}{CLIENT\PYGZus{}AUTH}\PYG{p}{)}
\PYG{n}{ssl\PYGZus{}ctx}\PYG{o}{.}\PYG{n}{load\PYGZus{}cert\PYGZus{}chain}\PYG{p}{(}\PYG{n}{os}\PYG{o}{.}\PYG{n}{path}\PYG{o}{.}\PYG{n}{join}\PYG{p}{(}\PYG{n}{data\PYGZus{}dir}\PYG{p}{,} \PYG{l+s+s2}{\PYGZdq{}}\PYG{l+s+s2}{mydomain.crt}\PYG{l+s+s2}{\PYGZdq{}}\PYG{p}{)}\PYG{p}{,}
                        \PYG{n}{os}\PYG{o}{.}\PYG{n}{path}\PYG{o}{.}\PYG{n}{join}\PYG{p}{(}\PYG{n}{data\PYGZus{}dir}\PYG{p}{,} \PYG{l+s+s2}{\PYGZdq{}}\PYG{l+s+s2}{mydomain.key}\PYG{l+s+s2}{\PYGZdq{}}\PYG{p}{)}\PYG{p}{)}
\PYG{n}{HTTPServer}\PYG{p}{(}\PYG{n}{application}\PYG{p}{,} \PYG{n}{ssl\PYGZus{}options}\PYG{o}{=}\PYG{n}{ssl\PYGZus{}ctx}\PYG{p}{)}
\end{sphinxVerbatim}

{\hyperref[\detokenize{httpserver:tornado.httpserver.HTTPServer}]{\sphinxcrossref{\sphinxcode{\sphinxupquote{HTTPServer}}}}} initialization follows one of three patterns (the
initialization methods are defined on {\hyperref[\detokenize{tcpserver:tornado.tcpserver.TCPServer}]{\sphinxcrossref{\sphinxcode{\sphinxupquote{tornado.tcpserver.TCPServer}}}}}):
\begin{enumerate}
\def\theenumi{\arabic{enumi}}
\def\labelenumi{\theenumi .}
\makeatletter\def\p@enumii{\p@enumi \theenumi .}\makeatother
\item {} 
{\hyperref[\detokenize{tcpserver:tornado.tcpserver.TCPServer.listen}]{\sphinxcrossref{\sphinxcode{\sphinxupquote{listen}}}}}: simple single-process:

\begin{sphinxVerbatim}[commandchars=\\\{\}]
\PYG{n}{server} \PYG{o}{=} \PYG{n}{HTTPServer}\PYG{p}{(}\PYG{n}{app}\PYG{p}{)}
\PYG{n}{server}\PYG{o}{.}\PYG{n}{listen}\PYG{p}{(}\PYG{l+m+mi}{8888}\PYG{p}{)}
\PYG{n}{IOLoop}\PYG{o}{.}\PYG{n}{current}\PYG{p}{(}\PYG{p}{)}\PYG{o}{.}\PYG{n}{start}\PYG{p}{(}\PYG{p}{)}
\end{sphinxVerbatim}

In many cases, {\hyperref[\detokenize{web:tornado.web.Application.listen}]{\sphinxcrossref{\sphinxcode{\sphinxupquote{tornado.web.Application.listen}}}}} can be used to avoid
the need to explicitly create the {\hyperref[\detokenize{httpserver:tornado.httpserver.HTTPServer}]{\sphinxcrossref{\sphinxcode{\sphinxupquote{HTTPServer}}}}}.

\item {} 
{\hyperref[\detokenize{tcpserver:tornado.tcpserver.TCPServer.bind}]{\sphinxcrossref{\sphinxcode{\sphinxupquote{bind}}}}}/{\hyperref[\detokenize{tcpserver:tornado.tcpserver.TCPServer.start}]{\sphinxcrossref{\sphinxcode{\sphinxupquote{start}}}}}:
simple multi-process:

\begin{sphinxVerbatim}[commandchars=\\\{\}]
\PYG{n}{server} \PYG{o}{=} \PYG{n}{HTTPServer}\PYG{p}{(}\PYG{n}{app}\PYG{p}{)}
\PYG{n}{server}\PYG{o}{.}\PYG{n}{bind}\PYG{p}{(}\PYG{l+m+mi}{8888}\PYG{p}{)}
\PYG{n}{server}\PYG{o}{.}\PYG{n}{start}\PYG{p}{(}\PYG{l+m+mi}{0}\PYG{p}{)}  \PYG{c+c1}{\PYGZsh{} Forks multiple sub\PYGZhy{}processes}
\PYG{n}{IOLoop}\PYG{o}{.}\PYG{n}{current}\PYG{p}{(}\PYG{p}{)}\PYG{o}{.}\PYG{n}{start}\PYG{p}{(}\PYG{p}{)}
\end{sphinxVerbatim}

When using this interface, an {\hyperref[\detokenize{ioloop:tornado.ioloop.IOLoop}]{\sphinxcrossref{\sphinxcode{\sphinxupquote{IOLoop}}}}} must \sphinxstyleemphasis{not} be passed
to the {\hyperref[\detokenize{httpserver:tornado.httpserver.HTTPServer}]{\sphinxcrossref{\sphinxcode{\sphinxupquote{HTTPServer}}}}} constructor.  {\hyperref[\detokenize{tcpserver:tornado.tcpserver.TCPServer.start}]{\sphinxcrossref{\sphinxcode{\sphinxupquote{start}}}}} will always start
the server on the default singleton {\hyperref[\detokenize{ioloop:tornado.ioloop.IOLoop}]{\sphinxcrossref{\sphinxcode{\sphinxupquote{IOLoop}}}}}.

\item {} 
{\hyperref[\detokenize{tcpserver:tornado.tcpserver.TCPServer.add_sockets}]{\sphinxcrossref{\sphinxcode{\sphinxupquote{add\_sockets}}}}}: advanced multi-process:

\begin{sphinxVerbatim}[commandchars=\\\{\}]
\PYG{n}{sockets} \PYG{o}{=} \PYG{n}{tornado}\PYG{o}{.}\PYG{n}{netutil}\PYG{o}{.}\PYG{n}{bind\PYGZus{}sockets}\PYG{p}{(}\PYG{l+m+mi}{8888}\PYG{p}{)}
\PYG{n}{tornado}\PYG{o}{.}\PYG{n}{process}\PYG{o}{.}\PYG{n}{fork\PYGZus{}processes}\PYG{p}{(}\PYG{l+m+mi}{0}\PYG{p}{)}
\PYG{n}{server} \PYG{o}{=} \PYG{n}{HTTPServer}\PYG{p}{(}\PYG{n}{app}\PYG{p}{)}
\PYG{n}{server}\PYG{o}{.}\PYG{n}{add\PYGZus{}sockets}\PYG{p}{(}\PYG{n}{sockets}\PYG{p}{)}
\PYG{n}{IOLoop}\PYG{o}{.}\PYG{n}{current}\PYG{p}{(}\PYG{p}{)}\PYG{o}{.}\PYG{n}{start}\PYG{p}{(}\PYG{p}{)}
\end{sphinxVerbatim}

The {\hyperref[\detokenize{tcpserver:tornado.tcpserver.TCPServer.add_sockets}]{\sphinxcrossref{\sphinxcode{\sphinxupquote{add\_sockets}}}}} interface is more complicated,
but it can be used with {\hyperref[\detokenize{process:tornado.process.fork_processes}]{\sphinxcrossref{\sphinxcode{\sphinxupquote{tornado.process.fork\_processes}}}}} to
give you more flexibility in when the fork happens.
{\hyperref[\detokenize{tcpserver:tornado.tcpserver.TCPServer.add_sockets}]{\sphinxcrossref{\sphinxcode{\sphinxupquote{add\_sockets}}}}} can also be used in single-process
servers if you want to create your listening sockets in some
way other than {\hyperref[\detokenize{netutil:tornado.netutil.bind_sockets}]{\sphinxcrossref{\sphinxcode{\sphinxupquote{tornado.netutil.bind\_sockets}}}}}.

\end{enumerate}

\DUrole{versionmodified,changed}{Changed in version 4.0: }Added \sphinxcode{\sphinxupquote{decompress\_request}}, \sphinxcode{\sphinxupquote{chunk\_size}}, \sphinxcode{\sphinxupquote{max\_header\_size}},
\sphinxcode{\sphinxupquote{idle\_connection\_timeout}}, \sphinxcode{\sphinxupquote{body\_timeout}}, \sphinxcode{\sphinxupquote{max\_body\_size}}
arguments.  Added support for {\hyperref[\detokenize{httputil:tornado.httputil.HTTPServerConnectionDelegate}]{\sphinxcrossref{\sphinxcode{\sphinxupquote{HTTPServerConnectionDelegate}}}}}
instances as \sphinxcode{\sphinxupquote{request\_callback}}.

\DUrole{versionmodified,changed}{Changed in version 4.1: }{\hyperref[\detokenize{httputil:tornado.httputil.HTTPServerConnectionDelegate.start_request}]{\sphinxcrossref{\sphinxcode{\sphinxupquote{HTTPServerConnectionDelegate.start\_request}}}}} is now called with
two arguments \sphinxcode{\sphinxupquote{(server\_conn, request\_conn)}} (in accordance with the
documentation) instead of one \sphinxcode{\sphinxupquote{(request\_conn)}}.

\DUrole{versionmodified,changed}{Changed in version 4.2: }{\hyperref[\detokenize{httpserver:tornado.httpserver.HTTPServer}]{\sphinxcrossref{\sphinxcode{\sphinxupquote{HTTPServer}}}}} is now a subclass of {\hyperref[\detokenize{util:tornado.util.Configurable}]{\sphinxcrossref{\sphinxcode{\sphinxupquote{tornado.util.Configurable}}}}}.

\DUrole{versionmodified,changed}{Changed in version 4.5: }Added the \sphinxcode{\sphinxupquote{trusted\_downstream}} argument.

\DUrole{versionmodified,changed}{Changed in version 5.0: }The \sphinxcode{\sphinxupquote{io\_loop}} argument has been removed.

The public interface of this class is mostly inherited from
{\hyperref[\detokenize{tcpserver:tornado.tcpserver.TCPServer}]{\sphinxcrossref{\sphinxcode{\sphinxupquote{TCPServer}}}}} and is documented under that class.
\index{close\_all\_connections() (tornado.httpserver.HTTPServer method)@\spxentry{close\_all\_connections()}\spxextra{tornado.httpserver.HTTPServer method}}

\begin{fulllineitems}
\phantomsection\label{\detokenize{httpserver:tornado.httpserver.HTTPServer.close_all_connections}}\pysiglinewithargsret{\sphinxbfcode{\sphinxupquote{coroutine }}\sphinxbfcode{\sphinxupquote{close\_all\_connections}}}{}{{ $\rightarrow$ None}}
Close all open connections and asynchronously wait for them to finish.

This method is used in combination with {\hyperref[\detokenize{tcpserver:tornado.tcpserver.TCPServer.stop}]{\sphinxcrossref{\sphinxcode{\sphinxupquote{stop}}}}} to
support clean shutdowns (especially for unittests). Typical
usage would call \sphinxcode{\sphinxupquote{stop()}} first to stop accepting new
connections, then \sphinxcode{\sphinxupquote{await close\_all\_connections()}} to wait for
existing connections to finish.

This method does not currently close open websocket connections.

Note that this method is a coroutine and must be caled with \sphinxcode{\sphinxupquote{await}}.

\end{fulllineitems}


\end{fulllineitems}



\subsection{\sphinxstyleliteralintitle{\sphinxupquote{tornado.httpclient}} — Asynchronous HTTP client}
\label{\detokenize{httpclient:module-tornado.httpclient}}\label{\detokenize{httpclient:tornado-httpclient-asynchronous-http-client}}\label{\detokenize{httpclient::doc}}\index{tornado.httpclient (module)@\spxentry{tornado.httpclient}\spxextra{module}}
Blocking and non-blocking HTTP client interfaces.

This module defines a common interface shared by two implementations,
\sphinxcode{\sphinxupquote{simple\_httpclient}} and \sphinxcode{\sphinxupquote{curl\_httpclient}}.  Applications may either
instantiate their chosen implementation class directly or use the
{\hyperref[\detokenize{httpclient:tornado.httpclient.AsyncHTTPClient}]{\sphinxcrossref{\sphinxcode{\sphinxupquote{AsyncHTTPClient}}}}} class from this module, which selects an implementation
that can be overridden with the {\hyperref[\detokenize{httpclient:tornado.httpclient.AsyncHTTPClient.configure}]{\sphinxcrossref{\sphinxcode{\sphinxupquote{AsyncHTTPClient.configure}}}}} method.

The default implementation is \sphinxcode{\sphinxupquote{simple\_httpclient}}, and this is expected
to be suitable for most users’ needs.  However, some applications may wish
to switch to \sphinxcode{\sphinxupquote{curl\_httpclient}} for reasons such as the following:
\begin{itemize}
\item {} 
\sphinxcode{\sphinxupquote{curl\_httpclient}} has some features not found in \sphinxcode{\sphinxupquote{simple\_httpclient}},
including support for HTTP proxies and the ability to use a specified
network interface.

\item {} 
\sphinxcode{\sphinxupquote{curl\_httpclient}} is more likely to be compatible with sites that are
not-quite-compliant with the HTTP spec, or sites that use little-exercised
features of HTTP.

\item {} 
\sphinxcode{\sphinxupquote{curl\_httpclient}} is faster.

\end{itemize}

Note that if you are using \sphinxcode{\sphinxupquote{curl\_httpclient}}, it is highly
recommended that you use a recent version of \sphinxcode{\sphinxupquote{libcurl}} and
\sphinxcode{\sphinxupquote{pycurl}}.  Currently the minimum supported version of libcurl is
7.22.0, and the minimum version of pycurl is 7.18.2.  It is highly
recommended that your \sphinxcode{\sphinxupquote{libcurl}} installation is built with
asynchronous DNS resolver (threaded or c-ares), otherwise you may
encounter various problems with request timeouts (for more
information, see
\sphinxurl{http://curl.haxx.se/libcurl/c/curl\_easy\_setopt.html\#CURLOPTCONNECTTIMEOUTMS}
and comments in curl\_httpclient.py).

To select \sphinxcode{\sphinxupquote{curl\_httpclient}}, call {\hyperref[\detokenize{httpclient:tornado.httpclient.AsyncHTTPClient.configure}]{\sphinxcrossref{\sphinxcode{\sphinxupquote{AsyncHTTPClient.configure}}}}} at startup:

\begin{sphinxVerbatim}[commandchars=\\\{\}]
\PYG{n}{AsyncHTTPClient}\PYG{o}{.}\PYG{n}{configure}\PYG{p}{(}\PYG{l+s+s2}{\PYGZdq{}}\PYG{l+s+s2}{tornado.curl\PYGZus{}httpclient.CurlAsyncHTTPClient}\PYG{l+s+s2}{\PYGZdq{}}\PYG{p}{)}
\end{sphinxVerbatim}


\subsubsection{HTTP client interfaces}
\label{\detokenize{httpclient:http-client-interfaces}}\index{HTTPClient (class in tornado.httpclient)@\spxentry{HTTPClient}\spxextra{class in tornado.httpclient}}

\begin{fulllineitems}
\phantomsection\label{\detokenize{httpclient:tornado.httpclient.HTTPClient}}\pysiglinewithargsret{\sphinxbfcode{\sphinxupquote{class }}\sphinxcode{\sphinxupquote{tornado.httpclient.}}\sphinxbfcode{\sphinxupquote{HTTPClient}}}{\emph{async\_client\_class: Type{[}AsyncHTTPClient{]} = None}, \emph{**kwargs}}{}
A blocking HTTP client.

This interface is provided to make it easier to share code between
synchronous and asynchronous applications. Applications that are
running an {\hyperref[\detokenize{ioloop:tornado.ioloop.IOLoop}]{\sphinxcrossref{\sphinxcode{\sphinxupquote{IOLoop}}}}} must use {\hyperref[\detokenize{httpclient:tornado.httpclient.AsyncHTTPClient}]{\sphinxcrossref{\sphinxcode{\sphinxupquote{AsyncHTTPClient}}}}} instead.

Typical usage looks like this:

\begin{sphinxVerbatim}[commandchars=\\\{\}]
\PYG{n}{http\PYGZus{}client} \PYG{o}{=} \PYG{n}{httpclient}\PYG{o}{.}\PYG{n}{HTTPClient}\PYG{p}{(}\PYG{p}{)}
\PYG{k}{try}\PYG{p}{:}
    \PYG{n}{response} \PYG{o}{=} \PYG{n}{http\PYGZus{}client}\PYG{o}{.}\PYG{n}{fetch}\PYG{p}{(}\PYG{l+s+s2}{\PYGZdq{}}\PYG{l+s+s2}{http://www.google.com/}\PYG{l+s+s2}{\PYGZdq{}}\PYG{p}{)}
    \PYG{n+nb}{print}\PYG{p}{(}\PYG{n}{response}\PYG{o}{.}\PYG{n}{body}\PYG{p}{)}
\PYG{k}{except} \PYG{n}{httpclient}\PYG{o}{.}\PYG{n}{HTTPError} \PYG{k}{as} \PYG{n}{e}\PYG{p}{:}
    \PYG{c+c1}{\PYGZsh{} HTTPError is raised for non\PYGZhy{}200 responses; the response}
    \PYG{c+c1}{\PYGZsh{} can be found in e.response.}
    \PYG{n+nb}{print}\PYG{p}{(}\PYG{l+s+s2}{\PYGZdq{}}\PYG{l+s+s2}{Error: }\PYG{l+s+s2}{\PYGZdq{}} \PYG{o}{+} \PYG{n+nb}{str}\PYG{p}{(}\PYG{n}{e}\PYG{p}{)}\PYG{p}{)}
\PYG{k}{except} \PYG{n+ne}{Exception} \PYG{k}{as} \PYG{n}{e}\PYG{p}{:}
    \PYG{c+c1}{\PYGZsh{} Other errors are possible, such as IOError.}
    \PYG{n+nb}{print}\PYG{p}{(}\PYG{l+s+s2}{\PYGZdq{}}\PYG{l+s+s2}{Error: }\PYG{l+s+s2}{\PYGZdq{}} \PYG{o}{+} \PYG{n+nb}{str}\PYG{p}{(}\PYG{n}{e}\PYG{p}{)}\PYG{p}{)}
\PYG{n}{http\PYGZus{}client}\PYG{o}{.}\PYG{n}{close}\PYG{p}{(}\PYG{p}{)}
\end{sphinxVerbatim}

\DUrole{versionmodified,changed}{Changed in version 5.0: }Due to limitations in \sphinxhref{https://docs.python.org/3.6/library/asyncio.html\#module-asyncio}{\sphinxcode{\sphinxupquote{asyncio}}}, it is no longer possible to
use the synchronous \sphinxcode{\sphinxupquote{HTTPClient}} while an {\hyperref[\detokenize{ioloop:tornado.ioloop.IOLoop}]{\sphinxcrossref{\sphinxcode{\sphinxupquote{IOLoop}}}}} is running.
Use {\hyperref[\detokenize{httpclient:tornado.httpclient.AsyncHTTPClient}]{\sphinxcrossref{\sphinxcode{\sphinxupquote{AsyncHTTPClient}}}}} instead.
\index{close() (tornado.httpclient.HTTPClient method)@\spxentry{close()}\spxextra{tornado.httpclient.HTTPClient method}}

\begin{fulllineitems}
\phantomsection\label{\detokenize{httpclient:tornado.httpclient.HTTPClient.close}}\pysiglinewithargsret{\sphinxbfcode{\sphinxupquote{close}}}{}{{ $\rightarrow$ None}}
Closes the HTTPClient, freeing any resources used.

\end{fulllineitems}

\index{fetch() (tornado.httpclient.HTTPClient method)@\spxentry{fetch()}\spxextra{tornado.httpclient.HTTPClient method}}

\begin{fulllineitems}
\phantomsection\label{\detokenize{httpclient:tornado.httpclient.HTTPClient.fetch}}\pysiglinewithargsret{\sphinxbfcode{\sphinxupquote{fetch}}}{\emph{request: Union{[}HTTPRequest, str{]}, **kwargs}}{{ $\rightarrow$ tornado.httpclient.HTTPResponse}}
Executes a request, returning an {\hyperref[\detokenize{httpclient:tornado.httpclient.HTTPResponse}]{\sphinxcrossref{\sphinxcode{\sphinxupquote{HTTPResponse}}}}}.

The request may be either a string URL or an {\hyperref[\detokenize{httpclient:tornado.httpclient.HTTPRequest}]{\sphinxcrossref{\sphinxcode{\sphinxupquote{HTTPRequest}}}}} object.
If it is a string, we construct an {\hyperref[\detokenize{httpclient:tornado.httpclient.HTTPRequest}]{\sphinxcrossref{\sphinxcode{\sphinxupquote{HTTPRequest}}}}} using any additional
kwargs: \sphinxcode{\sphinxupquote{HTTPRequest(request, **kwargs)}}

If an error occurs during the fetch, we raise an {\hyperref[\detokenize{httpclient:tornado.httpclient.HTTPError}]{\sphinxcrossref{\sphinxcode{\sphinxupquote{HTTPError}}}}} unless
the \sphinxcode{\sphinxupquote{raise\_error}} keyword argument is set to False.

\end{fulllineitems}


\end{fulllineitems}

\index{AsyncHTTPClient (class in tornado.httpclient)@\spxentry{AsyncHTTPClient}\spxextra{class in tornado.httpclient}}

\begin{fulllineitems}
\phantomsection\label{\detokenize{httpclient:tornado.httpclient.AsyncHTTPClient}}\pysigline{\sphinxbfcode{\sphinxupquote{class }}\sphinxcode{\sphinxupquote{tornado.httpclient.}}\sphinxbfcode{\sphinxupquote{AsyncHTTPClient}}}
An non-blocking HTTP client.

Example usage:

\begin{sphinxVerbatim}[commandchars=\\\{\}]
\PYG{k}{async} \PYG{k}{def} \PYG{n+nf}{f}\PYG{p}{(}\PYG{p}{)}\PYG{p}{:}
    \PYG{n}{http\PYGZus{}client} \PYG{o}{=} \PYG{n}{AsyncHTTPClient}\PYG{p}{(}\PYG{p}{)}
    \PYG{k}{try}\PYG{p}{:}
        \PYG{n}{response} \PYG{o}{=} \PYG{k}{await} \PYG{n}{http\PYGZus{}client}\PYG{o}{.}\PYG{n}{fetch}\PYG{p}{(}\PYG{l+s+s2}{\PYGZdq{}}\PYG{l+s+s2}{http://www.google.com}\PYG{l+s+s2}{\PYGZdq{}}\PYG{p}{)}
    \PYG{k}{except} \PYG{n+ne}{Exception} \PYG{k}{as} \PYG{n}{e}\PYG{p}{:}
        \PYG{n+nb}{print}\PYG{p}{(}\PYG{l+s+s2}{\PYGZdq{}}\PYG{l+s+s2}{Error: }\PYG{l+s+si}{\PYGZpc{}s}\PYG{l+s+s2}{\PYGZdq{}} \PYG{o}{\PYGZpc{}} \PYG{n}{e}\PYG{p}{)}
    \PYG{k}{else}\PYG{p}{:}
        \PYG{n+nb}{print}\PYG{p}{(}\PYG{n}{response}\PYG{o}{.}\PYG{n}{body}\PYG{p}{)}
\end{sphinxVerbatim}

The constructor for this class is magic in several respects: It
actually creates an instance of an implementation-specific
subclass, and instances are reused as a kind of pseudo-singleton
(one per {\hyperref[\detokenize{ioloop:tornado.ioloop.IOLoop}]{\sphinxcrossref{\sphinxcode{\sphinxupquote{IOLoop}}}}}). The keyword argument \sphinxcode{\sphinxupquote{force\_instance=True}}
can be used to suppress this singleton behavior. Unless
\sphinxcode{\sphinxupquote{force\_instance=True}} is used, no arguments should be passed to
the {\hyperref[\detokenize{httpclient:tornado.httpclient.AsyncHTTPClient}]{\sphinxcrossref{\sphinxcode{\sphinxupquote{AsyncHTTPClient}}}}} constructor. The implementation subclass as
well as arguments to its constructor can be set with the static
method {\hyperref[\detokenize{httpclient:tornado.httpclient.AsyncHTTPClient.configure}]{\sphinxcrossref{\sphinxcode{\sphinxupquote{configure()}}}}}

All {\hyperref[\detokenize{httpclient:tornado.httpclient.AsyncHTTPClient}]{\sphinxcrossref{\sphinxcode{\sphinxupquote{AsyncHTTPClient}}}}} implementations support a \sphinxcode{\sphinxupquote{defaults}}
keyword argument, which can be used to set default values for
{\hyperref[\detokenize{httpclient:tornado.httpclient.HTTPRequest}]{\sphinxcrossref{\sphinxcode{\sphinxupquote{HTTPRequest}}}}} attributes.  For example:

\begin{sphinxVerbatim}[commandchars=\\\{\}]
\PYG{n}{AsyncHTTPClient}\PYG{o}{.}\PYG{n}{configure}\PYG{p}{(}
    \PYG{k+kc}{None}\PYG{p}{,} \PYG{n}{defaults}\PYG{o}{=}\PYG{n+nb}{dict}\PYG{p}{(}\PYG{n}{user\PYGZus{}agent}\PYG{o}{=}\PYG{l+s+s2}{\PYGZdq{}}\PYG{l+s+s2}{MyUserAgent}\PYG{l+s+s2}{\PYGZdq{}}\PYG{p}{)}\PYG{p}{)}
\PYG{c+c1}{\PYGZsh{} or with force\PYGZus{}instance:}
\PYG{n}{client} \PYG{o}{=} \PYG{n}{AsyncHTTPClient}\PYG{p}{(}\PYG{n}{force\PYGZus{}instance}\PYG{o}{=}\PYG{k+kc}{True}\PYG{p}{,}
    \PYG{n}{defaults}\PYG{o}{=}\PYG{n+nb}{dict}\PYG{p}{(}\PYG{n}{user\PYGZus{}agent}\PYG{o}{=}\PYG{l+s+s2}{\PYGZdq{}}\PYG{l+s+s2}{MyUserAgent}\PYG{l+s+s2}{\PYGZdq{}}\PYG{p}{)}\PYG{p}{)}
\end{sphinxVerbatim}

\DUrole{versionmodified,changed}{Changed in version 5.0: }The \sphinxcode{\sphinxupquote{io\_loop}} argument (deprecated since version 4.1) has been removed.
\index{close() (tornado.httpclient.AsyncHTTPClient method)@\spxentry{close()}\spxextra{tornado.httpclient.AsyncHTTPClient method}}

\begin{fulllineitems}
\phantomsection\label{\detokenize{httpclient:tornado.httpclient.AsyncHTTPClient.close}}\pysiglinewithargsret{\sphinxbfcode{\sphinxupquote{close}}}{}{{ $\rightarrow$ None}}
Destroys this HTTP client, freeing any file descriptors used.

This method is \sphinxstylestrong{not needed in normal use} due to the way
that {\hyperref[\detokenize{httpclient:tornado.httpclient.AsyncHTTPClient}]{\sphinxcrossref{\sphinxcode{\sphinxupquote{AsyncHTTPClient}}}}} objects are transparently reused.
\sphinxcode{\sphinxupquote{close()}} is generally only necessary when either the
{\hyperref[\detokenize{ioloop:tornado.ioloop.IOLoop}]{\sphinxcrossref{\sphinxcode{\sphinxupquote{IOLoop}}}}} is also being closed, or the \sphinxcode{\sphinxupquote{force\_instance=True}}
argument was used when creating the {\hyperref[\detokenize{httpclient:tornado.httpclient.AsyncHTTPClient}]{\sphinxcrossref{\sphinxcode{\sphinxupquote{AsyncHTTPClient}}}}}.

No other methods may be called on the {\hyperref[\detokenize{httpclient:tornado.httpclient.AsyncHTTPClient}]{\sphinxcrossref{\sphinxcode{\sphinxupquote{AsyncHTTPClient}}}}} after
\sphinxcode{\sphinxupquote{close()}}.

\end{fulllineitems}

\index{fetch() (tornado.httpclient.AsyncHTTPClient method)@\spxentry{fetch()}\spxextra{tornado.httpclient.AsyncHTTPClient method}}

\begin{fulllineitems}
\phantomsection\label{\detokenize{httpclient:tornado.httpclient.AsyncHTTPClient.fetch}}\pysiglinewithargsret{\sphinxbfcode{\sphinxupquote{fetch}}}{\emph{request: Union{[}str, HTTPRequest{]}, raise\_error: bool = True, **kwargs}}{{ $\rightarrow$ Awaitable{[}tornado.httpclient.HTTPResponse{]}}}
Executes a request, asynchronously returning an {\hyperref[\detokenize{httpclient:tornado.httpclient.HTTPResponse}]{\sphinxcrossref{\sphinxcode{\sphinxupquote{HTTPResponse}}}}}.

The request may be either a string URL or an {\hyperref[\detokenize{httpclient:tornado.httpclient.HTTPRequest}]{\sphinxcrossref{\sphinxcode{\sphinxupquote{HTTPRequest}}}}} object.
If it is a string, we construct an {\hyperref[\detokenize{httpclient:tornado.httpclient.HTTPRequest}]{\sphinxcrossref{\sphinxcode{\sphinxupquote{HTTPRequest}}}}} using any additional
kwargs: \sphinxcode{\sphinxupquote{HTTPRequest(request, **kwargs)}}

This method returns a {\hyperref[\detokenize{concurrent:tornado.concurrent.Future}]{\sphinxcrossref{\sphinxcode{\sphinxupquote{Future}}}}} whose result is an
{\hyperref[\detokenize{httpclient:tornado.httpclient.HTTPResponse}]{\sphinxcrossref{\sphinxcode{\sphinxupquote{HTTPResponse}}}}}. By default, the \sphinxcode{\sphinxupquote{Future}} will raise an
{\hyperref[\detokenize{httpclient:tornado.httpclient.HTTPError}]{\sphinxcrossref{\sphinxcode{\sphinxupquote{HTTPError}}}}} if the request returned a non-200 response code
(other errors may also be raised if the server could not be
contacted). Instead, if \sphinxcode{\sphinxupquote{raise\_error}} is set to False, the
response will always be returned regardless of the response
code.

If a \sphinxcode{\sphinxupquote{callback}} is given, it will be invoked with the {\hyperref[\detokenize{httpclient:tornado.httpclient.HTTPResponse}]{\sphinxcrossref{\sphinxcode{\sphinxupquote{HTTPResponse}}}}}.
In the callback interface, {\hyperref[\detokenize{httpclient:tornado.httpclient.HTTPError}]{\sphinxcrossref{\sphinxcode{\sphinxupquote{HTTPError}}}}} is not automatically raised.
Instead, you must check the response’s \sphinxcode{\sphinxupquote{error}} attribute or
call its {\hyperref[\detokenize{httpclient:tornado.httpclient.HTTPResponse.rethrow}]{\sphinxcrossref{\sphinxcode{\sphinxupquote{rethrow}}}}} method.

\DUrole{versionmodified,changed}{Changed in version 6.0: }The \sphinxcode{\sphinxupquote{callback}} argument was removed. Use the returned
{\hyperref[\detokenize{concurrent:tornado.concurrent.Future}]{\sphinxcrossref{\sphinxcode{\sphinxupquote{Future}}}}} instead.

The \sphinxcode{\sphinxupquote{raise\_error=False}} argument only affects the
{\hyperref[\detokenize{httpclient:tornado.httpclient.HTTPError}]{\sphinxcrossref{\sphinxcode{\sphinxupquote{HTTPError}}}}} raised when a non-200 response code is used,
instead of suppressing all errors.

\end{fulllineitems}

\index{configure() (tornado.httpclient.AsyncHTTPClient class method)@\spxentry{configure()}\spxextra{tornado.httpclient.AsyncHTTPClient class method}}

\begin{fulllineitems}
\phantomsection\label{\detokenize{httpclient:tornado.httpclient.AsyncHTTPClient.configure}}\pysiglinewithargsret{\sphinxbfcode{\sphinxupquote{classmethod }}\sphinxbfcode{\sphinxupquote{configure}}}{\emph{impl: Union{[}None, str, Type{[}tornado.util.Configurable{]}{]}, **kwargs}}{{ $\rightarrow$ None}}
Configures the {\hyperref[\detokenize{httpclient:tornado.httpclient.AsyncHTTPClient}]{\sphinxcrossref{\sphinxcode{\sphinxupquote{AsyncHTTPClient}}}}} subclass to use.

\sphinxcode{\sphinxupquote{AsyncHTTPClient()}} actually creates an instance of a subclass.
This method may be called with either a class object or the
fully-qualified name of such a class (or \sphinxcode{\sphinxupquote{None}} to use the default,
\sphinxcode{\sphinxupquote{SimpleAsyncHTTPClient}})

If additional keyword arguments are given, they will be passed
to the constructor of each subclass instance created.  The
keyword argument \sphinxcode{\sphinxupquote{max\_clients}} determines the maximum number
of simultaneous {\hyperref[\detokenize{httpclient:tornado.httpclient.AsyncHTTPClient.fetch}]{\sphinxcrossref{\sphinxcode{\sphinxupquote{fetch()}}}}} operations that can
execute in parallel on each {\hyperref[\detokenize{ioloop:tornado.ioloop.IOLoop}]{\sphinxcrossref{\sphinxcode{\sphinxupquote{IOLoop}}}}}.  Additional arguments
may be supported depending on the implementation class in use.

Example:

\begin{sphinxVerbatim}[commandchars=\\\{\}]
\PYG{n}{AsyncHTTPClient}\PYG{o}{.}\PYG{n}{configure}\PYG{p}{(}\PYG{l+s+s2}{\PYGZdq{}}\PYG{l+s+s2}{tornado.curl\PYGZus{}httpclient.CurlAsyncHTTPClient}\PYG{l+s+s2}{\PYGZdq{}}\PYG{p}{)}
\end{sphinxVerbatim}

\end{fulllineitems}


\end{fulllineitems}



\subsubsection{Request objects}
\label{\detokenize{httpclient:request-objects}}\index{HTTPRequest (class in tornado.httpclient)@\spxentry{HTTPRequest}\spxextra{class in tornado.httpclient}}

\begin{fulllineitems}
\phantomsection\label{\detokenize{httpclient:tornado.httpclient.HTTPRequest}}\pysiglinewithargsret{\sphinxbfcode{\sphinxupquote{class }}\sphinxcode{\sphinxupquote{tornado.httpclient.}}\sphinxbfcode{\sphinxupquote{HTTPRequest}}}{\emph{url: str, method: str = 'GET', headers: Union{[}Dict{[}str, str{]}, tornado.httputil.HTTPHeaders{]} = None, body: Union{[}bytes, str{]} = None, auth\_username: str = None, auth\_password: str = None, auth\_mode: str = None, connect\_timeout: float = None, request\_timeout: float = None, if\_modified\_since: Union{[}float, datetime.datetime{]} = None, follow\_redirects: bool = None, max\_redirects: int = None, user\_agent: str = None, use\_gzip: bool = None, network\_interface: str = None, streaming\_callback: Callable{[}{[}bytes{]}, None{]} = None, header\_callback: Callable{[}{[}str{]}, None{]} = None, prepare\_curl\_callback: Callable{[}{[}Any{]}, None{]} = None, proxy\_host: str = None, proxy\_port: int = None, proxy\_username: str = None, proxy\_password: str = None, proxy\_auth\_mode: str = None, allow\_nonstandard\_methods: bool = None, validate\_cert: bool = None, ca\_certs: str = None, allow\_ipv6: bool = None, client\_key: str = None, client\_cert: str = None, body\_producer: Callable{[}{[}Callable{[}{[}bytes{]}, None{]}{]}, Future{[}None{]}{]} = None, expect\_100\_continue: bool = False, decompress\_response: bool = None, ssl\_options: Union{[}Dict{[}str, Any{]}, ssl.SSLContext{]} = None}}{}
HTTP client request object.

All parameters except \sphinxcode{\sphinxupquote{url}} are optional.
\begin{quote}\begin{description}
\item[{Parameters}] \leavevmode\begin{itemize}
\item {} 
\sphinxstyleliteralstrong{\sphinxupquote{url}} (\sphinxhref{https://docs.python.org/3.6/library/stdtypes.html\#str}{\sphinxstyleliteralemphasis{\sphinxupquote{str}}}) \textendash{} URL to fetch

\item {} 
\sphinxstyleliteralstrong{\sphinxupquote{method}} (\sphinxhref{https://docs.python.org/3.6/library/stdtypes.html\#str}{\sphinxstyleliteralemphasis{\sphinxupquote{str}}}) \textendash{} HTTP method, e.g. “GET” or “POST”

\item {} 
\sphinxstyleliteralstrong{\sphinxupquote{headers}} ({\hyperref[\detokenize{httputil:tornado.httputil.HTTPHeaders}]{\sphinxcrossref{\sphinxcode{\sphinxupquote{HTTPHeaders}}}}} or \sphinxhref{https://docs.python.org/3.6/library/stdtypes.html\#dict}{\sphinxcode{\sphinxupquote{dict}}}) \textendash{} Additional HTTP headers to pass on the request

\item {} 
\sphinxstyleliteralstrong{\sphinxupquote{body}} \textendash{} HTTP request body as a string (byte or unicode; if unicode
the utf-8 encoding will be used)

\item {} 
\sphinxstyleliteralstrong{\sphinxupquote{body\_producer}} \textendash{} Callable used for lazy/asynchronous request bodies.
It is called with one argument, a \sphinxcode{\sphinxupquote{write}} function, and should
return a {\hyperref[\detokenize{concurrent:tornado.concurrent.Future}]{\sphinxcrossref{\sphinxcode{\sphinxupquote{Future}}}}}.  It should call the write function with new
data as it becomes available.  The write function returns a
{\hyperref[\detokenize{concurrent:tornado.concurrent.Future}]{\sphinxcrossref{\sphinxcode{\sphinxupquote{Future}}}}} which can be used for flow control.
Only one of \sphinxcode{\sphinxupquote{body}} and \sphinxcode{\sphinxupquote{body\_producer}} may
be specified.  \sphinxcode{\sphinxupquote{body\_producer}} is not supported on
\sphinxcode{\sphinxupquote{curl\_httpclient}}.  When using \sphinxcode{\sphinxupquote{body\_producer}} it is recommended
to pass a \sphinxcode{\sphinxupquote{Content-Length}} in the headers as otherwise chunked
encoding will be used, and many servers do not support chunked
encoding on requests.  New in Tornado 4.0

\item {} 
\sphinxstyleliteralstrong{\sphinxupquote{auth\_username}} (\sphinxhref{https://docs.python.org/3.6/library/stdtypes.html\#str}{\sphinxstyleliteralemphasis{\sphinxupquote{str}}}) \textendash{} Username for HTTP authentication

\item {} 
\sphinxstyleliteralstrong{\sphinxupquote{auth\_password}} (\sphinxhref{https://docs.python.org/3.6/library/stdtypes.html\#str}{\sphinxstyleliteralemphasis{\sphinxupquote{str}}}) \textendash{} Password for HTTP authentication

\item {} 
\sphinxstyleliteralstrong{\sphinxupquote{auth\_mode}} (\sphinxhref{https://docs.python.org/3.6/library/stdtypes.html\#str}{\sphinxstyleliteralemphasis{\sphinxupquote{str}}}) \textendash{} Authentication mode; default is “basic”.
Allowed values are implementation-defined; \sphinxcode{\sphinxupquote{curl\_httpclient}}
supports “basic” and “digest”; \sphinxcode{\sphinxupquote{simple\_httpclient}} only supports
“basic”

\item {} 
\sphinxstyleliteralstrong{\sphinxupquote{connect\_timeout}} (\sphinxhref{https://docs.python.org/3.6/library/functions.html\#float}{\sphinxstyleliteralemphasis{\sphinxupquote{float}}}) \textendash{} Timeout for initial connection in seconds,
default 20 seconds

\item {} 
\sphinxstyleliteralstrong{\sphinxupquote{request\_timeout}} (\sphinxhref{https://docs.python.org/3.6/library/functions.html\#float}{\sphinxstyleliteralemphasis{\sphinxupquote{float}}}) \textendash{} Timeout for entire request in seconds,
default 20 seconds

\item {} 
\sphinxstyleliteralstrong{\sphinxupquote{if\_modified\_since}} (\sphinxhref{https://docs.python.org/3.6/library/datetime.html\#module-datetime}{\sphinxcode{\sphinxupquote{datetime}}} or \sphinxhref{https://docs.python.org/3.6/library/functions.html\#float}{\sphinxcode{\sphinxupquote{float}}}) \textendash{} Timestamp for \sphinxcode{\sphinxupquote{If-Modified-Since}} header

\item {} 
\sphinxstyleliteralstrong{\sphinxupquote{follow\_redirects}} (\sphinxhref{https://docs.python.org/3.6/library/functions.html\#bool}{\sphinxstyleliteralemphasis{\sphinxupquote{bool}}}) \textendash{} Should redirects be followed automatically
or return the 3xx response? Default True.

\item {} 
\sphinxstyleliteralstrong{\sphinxupquote{max\_redirects}} (\sphinxhref{https://docs.python.org/3.6/library/functions.html\#int}{\sphinxstyleliteralemphasis{\sphinxupquote{int}}}) \textendash{} Limit for \sphinxcode{\sphinxupquote{follow\_redirects}}, default 5.

\item {} 
\sphinxstyleliteralstrong{\sphinxupquote{user\_agent}} (\sphinxhref{https://docs.python.org/3.6/library/stdtypes.html\#str}{\sphinxstyleliteralemphasis{\sphinxupquote{str}}}) \textendash{} String to send as \sphinxcode{\sphinxupquote{User-Agent}} header

\item {} 
\sphinxstyleliteralstrong{\sphinxupquote{decompress\_response}} (\sphinxhref{https://docs.python.org/3.6/library/functions.html\#bool}{\sphinxstyleliteralemphasis{\sphinxupquote{bool}}}) \textendash{} Request a compressed response from
the server and decompress it after downloading.  Default is True.
New in Tornado 4.0.

\item {} 
\sphinxstyleliteralstrong{\sphinxupquote{use\_gzip}} (\sphinxhref{https://docs.python.org/3.6/library/functions.html\#bool}{\sphinxstyleliteralemphasis{\sphinxupquote{bool}}}) \textendash{} Deprecated alias for \sphinxcode{\sphinxupquote{decompress\_response}}
since Tornado 4.0.

\item {} 
\sphinxstyleliteralstrong{\sphinxupquote{network\_interface}} (\sphinxhref{https://docs.python.org/3.6/library/stdtypes.html\#str}{\sphinxstyleliteralemphasis{\sphinxupquote{str}}}) \textendash{} Network interface or source IP to use for request.
See \sphinxcode{\sphinxupquote{curl\_httpclient}} note below.

\item {} 
\sphinxstyleliteralstrong{\sphinxupquote{streaming\_callback}} (\sphinxhref{https://docs.python.org/3.6/library/collections.abc.html\#collections.abc.Callable}{\sphinxstyleliteralemphasis{\sphinxupquote{collections.abc.Callable}}}) \textendash{} If set, \sphinxcode{\sphinxupquote{streaming\_callback}} will
be run with each chunk of data as it is received, and
\sphinxcode{\sphinxupquote{HTTPResponse.body}} and \sphinxcode{\sphinxupquote{HTTPResponse.buffer}} will be empty in
the final response.

\item {} 
\sphinxstyleliteralstrong{\sphinxupquote{header\_callback}} (\sphinxhref{https://docs.python.org/3.6/library/collections.abc.html\#collections.abc.Callable}{\sphinxstyleliteralemphasis{\sphinxupquote{collections.abc.Callable}}}) \textendash{} If set, \sphinxcode{\sphinxupquote{header\_callback}} will
be run with each header line as it is received (including the
first line, e.g. \sphinxcode{\sphinxupquote{HTTP/1.0 200 OK\textbackslash{}r\textbackslash{}n}}, and a final line
containing only \sphinxcode{\sphinxupquote{\textbackslash{}r\textbackslash{}n}}.  All lines include the trailing newline
characters).  \sphinxcode{\sphinxupquote{HTTPResponse.headers}} will be empty in the final
response.  This is most useful in conjunction with
\sphinxcode{\sphinxupquote{streaming\_callback}}, because it’s the only way to get access to
header data while the request is in progress.

\item {} 
\sphinxstyleliteralstrong{\sphinxupquote{prepare\_curl\_callback}} (\sphinxhref{https://docs.python.org/3.6/library/collections.abc.html\#collections.abc.Callable}{\sphinxstyleliteralemphasis{\sphinxupquote{collections.abc.Callable}}}) \textendash{} If set, will be called with
a \sphinxcode{\sphinxupquote{pycurl.Curl}} object to allow the application to make additional
\sphinxcode{\sphinxupquote{setopt}} calls.

\item {} 
\sphinxstyleliteralstrong{\sphinxupquote{proxy\_host}} (\sphinxhref{https://docs.python.org/3.6/library/stdtypes.html\#str}{\sphinxstyleliteralemphasis{\sphinxupquote{str}}}) \textendash{} HTTP proxy hostname.  To use proxies,
\sphinxcode{\sphinxupquote{proxy\_host}} and \sphinxcode{\sphinxupquote{proxy\_port}} must be set; \sphinxcode{\sphinxupquote{proxy\_username}},
\sphinxcode{\sphinxupquote{proxy\_pass}} and \sphinxcode{\sphinxupquote{proxy\_auth\_mode}} are optional.  Proxies are
currently only supported with \sphinxcode{\sphinxupquote{curl\_httpclient}}.

\item {} 
\sphinxstyleliteralstrong{\sphinxupquote{proxy\_port}} (\sphinxhref{https://docs.python.org/3.6/library/functions.html\#int}{\sphinxstyleliteralemphasis{\sphinxupquote{int}}}) \textendash{} HTTP proxy port

\item {} 
\sphinxstyleliteralstrong{\sphinxupquote{proxy\_username}} (\sphinxhref{https://docs.python.org/3.6/library/stdtypes.html\#str}{\sphinxstyleliteralemphasis{\sphinxupquote{str}}}) \textendash{} HTTP proxy username

\item {} 
\sphinxstyleliteralstrong{\sphinxupquote{proxy\_password}} (\sphinxhref{https://docs.python.org/3.6/library/stdtypes.html\#str}{\sphinxstyleliteralemphasis{\sphinxupquote{str}}}) \textendash{} HTTP proxy password

\item {} 
\sphinxstyleliteralstrong{\sphinxupquote{proxy\_auth\_mode}} (\sphinxhref{https://docs.python.org/3.6/library/stdtypes.html\#str}{\sphinxstyleliteralemphasis{\sphinxupquote{str}}}) \textendash{} HTTP proxy Authentication mode;
default is “basic”. supports “basic” and “digest”

\item {} 
\sphinxstyleliteralstrong{\sphinxupquote{allow\_nonstandard\_methods}} (\sphinxhref{https://docs.python.org/3.6/library/functions.html\#bool}{\sphinxstyleliteralemphasis{\sphinxupquote{bool}}}) \textendash{} Allow unknown values for \sphinxcode{\sphinxupquote{method}}
argument? Default is False.

\item {} 
\sphinxstyleliteralstrong{\sphinxupquote{validate\_cert}} (\sphinxhref{https://docs.python.org/3.6/library/functions.html\#bool}{\sphinxstyleliteralemphasis{\sphinxupquote{bool}}}) \textendash{} For HTTPS requests, validate the server’s
certificate? Default is True.

\item {} 
\sphinxstyleliteralstrong{\sphinxupquote{ca\_certs}} (\sphinxhref{https://docs.python.org/3.6/library/stdtypes.html\#str}{\sphinxstyleliteralemphasis{\sphinxupquote{str}}}) \textendash{} filename of CA certificates in PEM format,
or None to use defaults.  See note below when used with
\sphinxcode{\sphinxupquote{curl\_httpclient}}.

\item {} 
\sphinxstyleliteralstrong{\sphinxupquote{client\_key}} (\sphinxhref{https://docs.python.org/3.6/library/stdtypes.html\#str}{\sphinxstyleliteralemphasis{\sphinxupquote{str}}}) \textendash{} Filename for client SSL key, if any.  See
note below when used with \sphinxcode{\sphinxupquote{curl\_httpclient}}.

\item {} 
\sphinxstyleliteralstrong{\sphinxupquote{client\_cert}} (\sphinxhref{https://docs.python.org/3.6/library/stdtypes.html\#str}{\sphinxstyleliteralemphasis{\sphinxupquote{str}}}) \textendash{} Filename for client SSL certificate, if any.
See note below when used with \sphinxcode{\sphinxupquote{curl\_httpclient}}.

\item {} 
\sphinxstyleliteralstrong{\sphinxupquote{ssl\_options}} (\sphinxhref{https://docs.python.org/3.6/library/ssl.html\#ssl.SSLContext}{\sphinxstyleliteralemphasis{\sphinxupquote{ssl.SSLContext}}}) \textendash{} \sphinxhref{https://docs.python.org/3.6/library/ssl.html\#ssl.SSLContext}{\sphinxcode{\sphinxupquote{ssl.SSLContext}}} object for use in
\sphinxcode{\sphinxupquote{simple\_httpclient}} (unsupported by \sphinxcode{\sphinxupquote{curl\_httpclient}}).
Overrides \sphinxcode{\sphinxupquote{validate\_cert}}, \sphinxcode{\sphinxupquote{ca\_certs}}, \sphinxcode{\sphinxupquote{client\_key}},
and \sphinxcode{\sphinxupquote{client\_cert}}.

\item {} 
\sphinxstyleliteralstrong{\sphinxupquote{allow\_ipv6}} (\sphinxhref{https://docs.python.org/3.6/library/functions.html\#bool}{\sphinxstyleliteralemphasis{\sphinxupquote{bool}}}) \textendash{} Use IPv6 when available?  Default is True.

\item {} 
\sphinxstyleliteralstrong{\sphinxupquote{expect\_100\_continue}} (\sphinxhref{https://docs.python.org/3.6/library/functions.html\#bool}{\sphinxstyleliteralemphasis{\sphinxupquote{bool}}}) \textendash{} If true, send the
\sphinxcode{\sphinxupquote{Expect: 100-continue}} header and wait for a continue response
before sending the request body.  Only supported with
\sphinxcode{\sphinxupquote{simple\_httpclient}}.

\end{itemize}

\end{description}\end{quote}

\begin{sphinxadmonition}{note}{Note:}
When using \sphinxcode{\sphinxupquote{curl\_httpclient}} certain options may be
inherited by subsequent fetches because \sphinxcode{\sphinxupquote{pycurl}} does
not allow them to be cleanly reset.  This applies to the
\sphinxcode{\sphinxupquote{ca\_certs}}, \sphinxcode{\sphinxupquote{client\_key}}, \sphinxcode{\sphinxupquote{client\_cert}}, and
\sphinxcode{\sphinxupquote{network\_interface}} arguments.  If you use these
options, you should pass them on every request (you don’t
have to always use the same values, but it’s not possible
to mix requests that specify these options with ones that
use the defaults).
\end{sphinxadmonition}

\DUrole{versionmodified,added}{New in version 3.1: }The \sphinxcode{\sphinxupquote{auth\_mode}} argument.

\DUrole{versionmodified,added}{New in version 4.0: }The \sphinxcode{\sphinxupquote{body\_producer}} and \sphinxcode{\sphinxupquote{expect\_100\_continue}} arguments.

\DUrole{versionmodified,added}{New in version 4.2: }The \sphinxcode{\sphinxupquote{ssl\_options}} argument.

\DUrole{versionmodified,added}{New in version 4.5: }The \sphinxcode{\sphinxupquote{proxy\_auth\_mode}} argument.

\end{fulllineitems}



\subsubsection{Response objects}
\label{\detokenize{httpclient:response-objects}}\index{HTTPResponse (class in tornado.httpclient)@\spxentry{HTTPResponse}\spxextra{class in tornado.httpclient}}

\begin{fulllineitems}
\phantomsection\label{\detokenize{httpclient:tornado.httpclient.HTTPResponse}}\pysiglinewithargsret{\sphinxbfcode{\sphinxupquote{class }}\sphinxcode{\sphinxupquote{tornado.httpclient.}}\sphinxbfcode{\sphinxupquote{HTTPResponse}}}{\emph{request: tornado.httpclient.HTTPRequest}, \emph{code: int}, \emph{headers: tornado.httputil.HTTPHeaders = None}, \emph{buffer: \_io.BytesIO = None}, \emph{effective\_url: str = None}, \emph{error: BaseException = None}, \emph{request\_time: float = None}, \emph{time\_info: Dict{[}str}, \emph{float{]} = None}, \emph{reason: str = None}, \emph{start\_time: float = None}}{}
HTTP Response object.

Attributes:
\begin{itemize}
\item {} 
\sphinxcode{\sphinxupquote{request}}: HTTPRequest object

\item {} 
\sphinxcode{\sphinxupquote{code}}: numeric HTTP status code, e.g. 200 or 404

\item {} 
\sphinxcode{\sphinxupquote{reason}}: human-readable reason phrase describing the status code

\item {} 
\sphinxcode{\sphinxupquote{headers}}: {\hyperref[\detokenize{httputil:tornado.httputil.HTTPHeaders}]{\sphinxcrossref{\sphinxcode{\sphinxupquote{tornado.httputil.HTTPHeaders}}}}} object

\item {} 
\sphinxcode{\sphinxupquote{effective\_url}}: final location of the resource after following any
redirects

\item {} 
\sphinxcode{\sphinxupquote{buffer}}: \sphinxcode{\sphinxupquote{cStringIO}} object for response body

\item {} 
\sphinxcode{\sphinxupquote{body}}: response body as bytes (created on demand from \sphinxcode{\sphinxupquote{self.buffer}})

\item {} 
\sphinxcode{\sphinxupquote{error}}: Exception object, if any

\item {} 
\sphinxcode{\sphinxupquote{request\_time}}: seconds from request start to finish. Includes all
network operations from DNS resolution to receiving the last byte of
data. Does not include time spent in the queue (due to the
\sphinxcode{\sphinxupquote{max\_clients}} option). If redirects were followed, only includes
the final request.

\item {} 
\sphinxcode{\sphinxupquote{start\_time}}: Time at which the HTTP operation started, based on
\sphinxhref{https://docs.python.org/3.6/library/time.html\#time.time}{\sphinxcode{\sphinxupquote{time.time}}} (not the monotonic clock used by {\hyperref[\detokenize{ioloop:tornado.ioloop.IOLoop.time}]{\sphinxcrossref{\sphinxcode{\sphinxupquote{IOLoop.time}}}}}). May
be \sphinxcode{\sphinxupquote{None}} if the request timed out while in the queue.

\item {} 
\sphinxcode{\sphinxupquote{time\_info}}: dictionary of diagnostic timing information from the
request. Available data are subject to change, but currently uses timings
available from \sphinxurl{http://curl.haxx.se/libcurl/c/curl\_easy\_getinfo.html},
plus \sphinxcode{\sphinxupquote{queue}}, which is the delay (if any) introduced by waiting for
a slot under {\hyperref[\detokenize{httpclient:tornado.httpclient.AsyncHTTPClient}]{\sphinxcrossref{\sphinxcode{\sphinxupquote{AsyncHTTPClient}}}}}’s \sphinxcode{\sphinxupquote{max\_clients}} setting.

\end{itemize}

\DUrole{versionmodified,added}{New in version 5.1: }Added the \sphinxcode{\sphinxupquote{start\_time}} attribute.

\DUrole{versionmodified,changed}{Changed in version 5.1: }The \sphinxcode{\sphinxupquote{request\_time}} attribute previously included time spent in the queue
for \sphinxcode{\sphinxupquote{simple\_httpclient}}, but not in \sphinxcode{\sphinxupquote{curl\_httpclient}}. Now queueing time
is excluded in both implementations. \sphinxcode{\sphinxupquote{request\_time}} is now more accurate for
\sphinxcode{\sphinxupquote{curl\_httpclient}} because it uses a monotonic clock when available.
\index{rethrow() (tornado.httpclient.HTTPResponse method)@\spxentry{rethrow()}\spxextra{tornado.httpclient.HTTPResponse method}}

\begin{fulllineitems}
\phantomsection\label{\detokenize{httpclient:tornado.httpclient.HTTPResponse.rethrow}}\pysiglinewithargsret{\sphinxbfcode{\sphinxupquote{rethrow}}}{}{{ $\rightarrow$ None}}
If there was an error on the request, raise an {\hyperref[\detokenize{httpclient:tornado.httpclient.HTTPError}]{\sphinxcrossref{\sphinxcode{\sphinxupquote{HTTPError}}}}}.

\end{fulllineitems}


\end{fulllineitems}



\subsubsection{Exceptions}
\label{\detokenize{httpclient:exceptions}}\index{HTTPClientError@\spxentry{HTTPClientError}}

\begin{fulllineitems}
\phantomsection\label{\detokenize{httpclient:tornado.httpclient.HTTPClientError}}\pysiglinewithargsret{\sphinxbfcode{\sphinxupquote{exception }}\sphinxcode{\sphinxupquote{tornado.httpclient.}}\sphinxbfcode{\sphinxupquote{HTTPClientError}}}{\emph{code: int}, \emph{message: str = None}, \emph{response: tornado.httpclient.HTTPResponse = None}}{}
Exception thrown for an unsuccessful HTTP request.

Attributes:
\begin{itemize}
\item {} 
\sphinxcode{\sphinxupquote{code}} - HTTP error integer error code, e.g. 404.  Error code 599 is
used when no HTTP response was received, e.g. for a timeout.

\item {} 
\sphinxcode{\sphinxupquote{response}} - {\hyperref[\detokenize{httpclient:tornado.httpclient.HTTPResponse}]{\sphinxcrossref{\sphinxcode{\sphinxupquote{HTTPResponse}}}}} object, if any.

\end{itemize}

Note that if \sphinxcode{\sphinxupquote{follow\_redirects}} is False, redirects become HTTPErrors,
and you can look at \sphinxcode{\sphinxupquote{error.response.headers{[}'Location'{]}}} to see the
destination of the redirect.

\DUrole{versionmodified,changed}{Changed in version 5.1: }Renamed from \sphinxcode{\sphinxupquote{HTTPError}} to \sphinxcode{\sphinxupquote{HTTPClientError}} to avoid collisions with
{\hyperref[\detokenize{web:tornado.web.HTTPError}]{\sphinxcrossref{\sphinxcode{\sphinxupquote{tornado.web.HTTPError}}}}}. The name \sphinxcode{\sphinxupquote{tornado.httpclient.HTTPError}} remains
as an alias.

\end{fulllineitems}

\index{HTTPError@\spxentry{HTTPError}}

\begin{fulllineitems}
\phantomsection\label{\detokenize{httpclient:tornado.httpclient.HTTPError}}\pysigline{\sphinxbfcode{\sphinxupquote{exception }}\sphinxcode{\sphinxupquote{tornado.httpclient.}}\sphinxbfcode{\sphinxupquote{HTTPError}}}
Alias for {\hyperref[\detokenize{httpclient:tornado.httpclient.HTTPClientError}]{\sphinxcrossref{\sphinxcode{\sphinxupquote{HTTPClientError}}}}}.

\end{fulllineitems}



\subsubsection{Command-line interface}
\label{\detokenize{httpclient:command-line-interface}}
This module provides a simple command-line interface to fetch a url
using Tornado’s HTTP client.  Example usage:

\begin{sphinxVerbatim}[commandchars=\\\{\}]
\PYG{c+c1}{\PYGZsh{} Fetch the url and print its body}
\PYG{n}{python} \PYG{o}{\PYGZhy{}}\PYG{n}{m} \PYG{n}{tornado}\PYG{o}{.}\PYG{n}{httpclient} \PYG{n}{http}\PYG{p}{:}\PYG{o}{/}\PYG{o}{/}\PYG{n}{www}\PYG{o}{.}\PYG{n}{google}\PYG{o}{.}\PYG{n}{com}

\PYG{c+c1}{\PYGZsh{} Just print the headers}
\PYG{n}{python} \PYG{o}{\PYGZhy{}}\PYG{n}{m} \PYG{n}{tornado}\PYG{o}{.}\PYG{n}{httpclient} \PYG{o}{\PYGZhy{}}\PYG{o}{\PYGZhy{}}\PYG{n}{print\PYGZus{}headers} \PYG{o}{\PYGZhy{}}\PYG{o}{\PYGZhy{}}\PYG{n}{print\PYGZus{}body}\PYG{o}{=}\PYG{n}{false} \PYG{n}{http}\PYG{p}{:}\PYG{o}{/}\PYG{o}{/}\PYG{n}{www}\PYG{o}{.}\PYG{n}{google}\PYG{o}{.}\PYG{n}{com}
\end{sphinxVerbatim}


\subsubsection{Implementations}
\label{\detokenize{httpclient:module-tornado.simple_httpclient}}\label{\detokenize{httpclient:implementations}}\index{tornado.simple\_httpclient (module)@\spxentry{tornado.simple\_httpclient}\spxextra{module}}\index{SimpleAsyncHTTPClient (class in tornado.simple\_httpclient)@\spxentry{SimpleAsyncHTTPClient}\spxextra{class in tornado.simple\_httpclient}}

\begin{fulllineitems}
\phantomsection\label{\detokenize{httpclient:tornado.simple_httpclient.SimpleAsyncHTTPClient}}\pysigline{\sphinxbfcode{\sphinxupquote{class }}\sphinxcode{\sphinxupquote{tornado.simple\_httpclient.}}\sphinxbfcode{\sphinxupquote{SimpleAsyncHTTPClient}}}
Non-blocking HTTP client with no external dependencies.

This class implements an HTTP 1.1 client on top of Tornado’s IOStreams.
Some features found in the curl-based AsyncHTTPClient are not yet
supported.  In particular, proxies are not supported, connections
are not reused, and callers cannot select the network interface to be
used.
\index{initialize() (tornado.simple\_httpclient.SimpleAsyncHTTPClient method)@\spxentry{initialize()}\spxextra{tornado.simple\_httpclient.SimpleAsyncHTTPClient method}}

\begin{fulllineitems}
\phantomsection\label{\detokenize{httpclient:tornado.simple_httpclient.SimpleAsyncHTTPClient.initialize}}\pysiglinewithargsret{\sphinxbfcode{\sphinxupquote{initialize}}}{\emph{max\_clients: int = 10}, \emph{hostname\_mapping: Dict{[}str}, \emph{str{]} = None}, \emph{max\_buffer\_size: int = 104857600}, \emph{resolver: tornado.netutil.Resolver = None}, \emph{defaults: Dict{[}str}, \emph{Any{]} = None}, \emph{max\_header\_size: int = None}, \emph{max\_body\_size: int = None}}{{ $\rightarrow$ None}}
Creates a AsyncHTTPClient.

Only a single AsyncHTTPClient instance exists per IOLoop
in order to provide limitations on the number of pending connections.
\sphinxcode{\sphinxupquote{force\_instance=True}} may be used to suppress this behavior.

Note that because of this implicit reuse, unless \sphinxcode{\sphinxupquote{force\_instance}}
is used, only the first call to the constructor actually uses
its arguments. It is recommended to use the \sphinxcode{\sphinxupquote{configure}} method
instead of the constructor to ensure that arguments take effect.

\sphinxcode{\sphinxupquote{max\_clients}} is the number of concurrent requests that can be
in progress; when this limit is reached additional requests will be
queued. Note that time spent waiting in this queue still counts
against the \sphinxcode{\sphinxupquote{request\_timeout}}.

\sphinxcode{\sphinxupquote{hostname\_mapping}} is a dictionary mapping hostnames to IP addresses.
It can be used to make local DNS changes when modifying system-wide
settings like \sphinxcode{\sphinxupquote{/etc/hosts}} is not possible or desirable (e.g. in
unittests).

\sphinxcode{\sphinxupquote{max\_buffer\_size}} (default 100MB) is the number of bytes
that can be read into memory at once. \sphinxcode{\sphinxupquote{max\_body\_size}}
(defaults to \sphinxcode{\sphinxupquote{max\_buffer\_size}}) is the largest response body
that the client will accept.  Without a
\sphinxcode{\sphinxupquote{streaming\_callback}}, the smaller of these two limits
applies; with a \sphinxcode{\sphinxupquote{streaming\_callback}} only \sphinxcode{\sphinxupquote{max\_body\_size}}
does.

\DUrole{versionmodified,changed}{Changed in version 4.2: }Added the \sphinxcode{\sphinxupquote{max\_body\_size}} argument.

\end{fulllineitems}


\end{fulllineitems}

\phantomsection\label{\detokenize{httpclient:module-tornado.curl_httpclient}}\index{tornado.curl\_httpclient (module)@\spxentry{tornado.curl\_httpclient}\spxextra{module}}\index{CurlAsyncHTTPClient (class in tornado.curl\_httpclient)@\spxentry{CurlAsyncHTTPClient}\spxextra{class in tornado.curl\_httpclient}}

\begin{fulllineitems}
\phantomsection\label{\detokenize{httpclient:tornado.curl_httpclient.CurlAsyncHTTPClient}}\pysiglinewithargsret{\sphinxbfcode{\sphinxupquote{class }}\sphinxcode{\sphinxupquote{tornado.curl\_httpclient.}}\sphinxbfcode{\sphinxupquote{CurlAsyncHTTPClient}}}{\emph{max\_clients=10}, \emph{defaults=None}}{}
\sphinxcode{\sphinxupquote{libcurl}}-based HTTP client.

\end{fulllineitems}



\subsubsection{Example Code}
\label{\detokenize{httpclient:example-code}}\begin{itemize}
\item {} 
\sphinxhref{https://github.com/tornadoweb/tornado/blob/master/demos/webspider/webspider.py}{A simple webspider}
shows how to fetch URLs concurrently.

\item {} 
\sphinxhref{https://github.com/tornadoweb/tornado/tree/master/demos/file\_upload/}{The file uploader demo}
uses either HTTP POST or HTTP PUT to upload files to a server.

\end{itemize}


\subsection{\sphinxstyleliteralintitle{\sphinxupquote{tornado.httputil}} — Manipulate HTTP headers and URLs}
\label{\detokenize{httputil:tornado-httputil-manipulate-http-headers-and-urls}}\label{\detokenize{httputil::doc}}\phantomsection\label{\detokenize{httputil:module-tornado.httputil}}\index{tornado.httputil (module)@\spxentry{tornado.httputil}\spxextra{module}}
HTTP utility code shared by clients and servers.

This module also defines the {\hyperref[\detokenize{httputil:tornado.httputil.HTTPServerRequest}]{\sphinxcrossref{\sphinxcode{\sphinxupquote{HTTPServerRequest}}}}} class which is exposed
via {\hyperref[\detokenize{web:tornado.web.RequestHandler.request}]{\sphinxcrossref{\sphinxcode{\sphinxupquote{tornado.web.RequestHandler.request}}}}}.
\index{HTTPHeaders (class in tornado.httputil)@\spxentry{HTTPHeaders}\spxextra{class in tornado.httputil}}

\begin{fulllineitems}
\phantomsection\label{\detokenize{httputil:tornado.httputil.HTTPHeaders}}\pysiglinewithargsret{\sphinxbfcode{\sphinxupquote{class }}\sphinxcode{\sphinxupquote{tornado.httputil.}}\sphinxbfcode{\sphinxupquote{HTTPHeaders}}}{\emph{*args}, \emph{**kwargs}}{}
A dictionary that maintains \sphinxcode{\sphinxupquote{Http-Header-Case}} for all keys.

Supports multiple values per key via a pair of new methods,
{\hyperref[\detokenize{httputil:tornado.httputil.HTTPHeaders.add}]{\sphinxcrossref{\sphinxcode{\sphinxupquote{add()}}}}} and {\hyperref[\detokenize{httputil:tornado.httputil.HTTPHeaders.get_list}]{\sphinxcrossref{\sphinxcode{\sphinxupquote{get\_list()}}}}}.  The regular dictionary interface
returns a single value per key, with multiple values joined by a
comma.

\begin{sphinxVerbatim}[commandchars=\\\{\}]
\PYG{g+gp}{\PYGZgt{}\PYGZgt{}\PYGZgt{} }\PYG{n}{h} \PYG{o}{=} \PYG{n}{HTTPHeaders}\PYG{p}{(}\PYG{p}{\PYGZob{}}\PYG{l+s+s2}{\PYGZdq{}}\PYG{l+s+s2}{content\PYGZhy{}type}\PYG{l+s+s2}{\PYGZdq{}}\PYG{p}{:} \PYG{l+s+s2}{\PYGZdq{}}\PYG{l+s+s2}{text/html}\PYG{l+s+s2}{\PYGZdq{}}\PYG{p}{\PYGZcb{}}\PYG{p}{)}
\PYG{g+gp}{\PYGZgt{}\PYGZgt{}\PYGZgt{} }\PYG{n+nb}{list}\PYG{p}{(}\PYG{n}{h}\PYG{o}{.}\PYG{n}{keys}\PYG{p}{(}\PYG{p}{)}\PYG{p}{)}
\PYG{g+go}{[\PYGZsq{}Content\PYGZhy{}Type\PYGZsq{}]}
\PYG{g+gp}{\PYGZgt{}\PYGZgt{}\PYGZgt{} }\PYG{n}{h}\PYG{p}{[}\PYG{l+s+s2}{\PYGZdq{}}\PYG{l+s+s2}{Content\PYGZhy{}Type}\PYG{l+s+s2}{\PYGZdq{}}\PYG{p}{]}
\PYG{g+go}{\PYGZsq{}text/html\PYGZsq{}}
\end{sphinxVerbatim}

\begin{sphinxVerbatim}[commandchars=\\\{\}]
\PYG{g+gp}{\PYGZgt{}\PYGZgt{}\PYGZgt{} }\PYG{n}{h}\PYG{o}{.}\PYG{n}{add}\PYG{p}{(}\PYG{l+s+s2}{\PYGZdq{}}\PYG{l+s+s2}{Set\PYGZhy{}Cookie}\PYG{l+s+s2}{\PYGZdq{}}\PYG{p}{,} \PYG{l+s+s2}{\PYGZdq{}}\PYG{l+s+s2}{A=B}\PYG{l+s+s2}{\PYGZdq{}}\PYG{p}{)}
\PYG{g+gp}{\PYGZgt{}\PYGZgt{}\PYGZgt{} }\PYG{n}{h}\PYG{o}{.}\PYG{n}{add}\PYG{p}{(}\PYG{l+s+s2}{\PYGZdq{}}\PYG{l+s+s2}{Set\PYGZhy{}Cookie}\PYG{l+s+s2}{\PYGZdq{}}\PYG{p}{,} \PYG{l+s+s2}{\PYGZdq{}}\PYG{l+s+s2}{C=D}\PYG{l+s+s2}{\PYGZdq{}}\PYG{p}{)}
\PYG{g+gp}{\PYGZgt{}\PYGZgt{}\PYGZgt{} }\PYG{n}{h}\PYG{p}{[}\PYG{l+s+s2}{\PYGZdq{}}\PYG{l+s+s2}{set\PYGZhy{}cookie}\PYG{l+s+s2}{\PYGZdq{}}\PYG{p}{]}
\PYG{g+go}{\PYGZsq{}A=B,C=D\PYGZsq{}}
\PYG{g+gp}{\PYGZgt{}\PYGZgt{}\PYGZgt{} }\PYG{n}{h}\PYG{o}{.}\PYG{n}{get\PYGZus{}list}\PYG{p}{(}\PYG{l+s+s2}{\PYGZdq{}}\PYG{l+s+s2}{set\PYGZhy{}cookie}\PYG{l+s+s2}{\PYGZdq{}}\PYG{p}{)}
\PYG{g+go}{[\PYGZsq{}A=B\PYGZsq{}, \PYGZsq{}C=D\PYGZsq{}]}
\end{sphinxVerbatim}

\begin{sphinxVerbatim}[commandchars=\\\{\}]
\PYG{g+gp}{\PYGZgt{}\PYGZgt{}\PYGZgt{} }\PYG{k}{for} \PYG{p}{(}\PYG{n}{k}\PYG{p}{,}\PYG{n}{v}\PYG{p}{)} \PYG{o+ow}{in} \PYG{n+nb}{sorted}\PYG{p}{(}\PYG{n}{h}\PYG{o}{.}\PYG{n}{get\PYGZus{}all}\PYG{p}{(}\PYG{p}{)}\PYG{p}{)}\PYG{p}{:}
\PYG{g+gp}{... }   \PYG{n+nb}{print}\PYG{p}{(}\PYG{l+s+s1}{\PYGZsq{}}\PYG{l+s+si}{\PYGZpc{}s}\PYG{l+s+s1}{: }\PYG{l+s+si}{\PYGZpc{}s}\PYG{l+s+s1}{\PYGZsq{}} \PYG{o}{\PYGZpc{}} \PYG{p}{(}\PYG{n}{k}\PYG{p}{,}\PYG{n}{v}\PYG{p}{)}\PYG{p}{)}
\PYG{g+gp}{...}
\PYG{g+go}{Content\PYGZhy{}Type: text/html}
\PYG{g+go}{Set\PYGZhy{}Cookie: A=B}
\PYG{g+go}{Set\PYGZhy{}Cookie: C=D}
\end{sphinxVerbatim}
\index{add() (tornado.httputil.HTTPHeaders method)@\spxentry{add()}\spxextra{tornado.httputil.HTTPHeaders method}}

\begin{fulllineitems}
\phantomsection\label{\detokenize{httputil:tornado.httputil.HTTPHeaders.add}}\pysiglinewithargsret{\sphinxbfcode{\sphinxupquote{add}}}{\emph{name: str}, \emph{value: str}}{{ $\rightarrow$ None}}
Adds a new value for the given key.

\end{fulllineitems}

\index{get\_list() (tornado.httputil.HTTPHeaders method)@\spxentry{get\_list()}\spxextra{tornado.httputil.HTTPHeaders method}}

\begin{fulllineitems}
\phantomsection\label{\detokenize{httputil:tornado.httputil.HTTPHeaders.get_list}}\pysiglinewithargsret{\sphinxbfcode{\sphinxupquote{get\_list}}}{\emph{name: str}}{{ $\rightarrow$ List{[}str{]}}}
Returns all values for the given header as a list.

\end{fulllineitems}

\index{get\_all() (tornado.httputil.HTTPHeaders method)@\spxentry{get\_all()}\spxextra{tornado.httputil.HTTPHeaders method}}

\begin{fulllineitems}
\phantomsection\label{\detokenize{httputil:tornado.httputil.HTTPHeaders.get_all}}\pysiglinewithargsret{\sphinxbfcode{\sphinxupquote{get\_all}}}{}{{ $\rightarrow$ Iterable{[}Tuple{[}str, str{]}{]}}}
Returns an iterable of all (name, value) pairs.

If a header has multiple values, multiple pairs will be
returned with the same name.

\end{fulllineitems}

\index{parse\_line() (tornado.httputil.HTTPHeaders method)@\spxentry{parse\_line()}\spxextra{tornado.httputil.HTTPHeaders method}}

\begin{fulllineitems}
\phantomsection\label{\detokenize{httputil:tornado.httputil.HTTPHeaders.parse_line}}\pysiglinewithargsret{\sphinxbfcode{\sphinxupquote{parse\_line}}}{\emph{line: str}}{{ $\rightarrow$ None}}
Updates the dictionary with a single header line.

\begin{sphinxVerbatim}[commandchars=\\\{\}]
\PYG{g+gp}{\PYGZgt{}\PYGZgt{}\PYGZgt{} }\PYG{n}{h} \PYG{o}{=} \PYG{n}{HTTPHeaders}\PYG{p}{(}\PYG{p}{)}
\PYG{g+gp}{\PYGZgt{}\PYGZgt{}\PYGZgt{} }\PYG{n}{h}\PYG{o}{.}\PYG{n}{parse\PYGZus{}line}\PYG{p}{(}\PYG{l+s+s2}{\PYGZdq{}}\PYG{l+s+s2}{Content\PYGZhy{}Type: text/html}\PYG{l+s+s2}{\PYGZdq{}}\PYG{p}{)}
\PYG{g+gp}{\PYGZgt{}\PYGZgt{}\PYGZgt{} }\PYG{n}{h}\PYG{o}{.}\PYG{n}{get}\PYG{p}{(}\PYG{l+s+s1}{\PYGZsq{}}\PYG{l+s+s1}{content\PYGZhy{}type}\PYG{l+s+s1}{\PYGZsq{}}\PYG{p}{)}
\PYG{g+go}{\PYGZsq{}text/html\PYGZsq{}}
\end{sphinxVerbatim}

\end{fulllineitems}

\index{parse() (tornado.httputil.HTTPHeaders class method)@\spxentry{parse()}\spxextra{tornado.httputil.HTTPHeaders class method}}

\begin{fulllineitems}
\phantomsection\label{\detokenize{httputil:tornado.httputil.HTTPHeaders.parse}}\pysiglinewithargsret{\sphinxbfcode{\sphinxupquote{classmethod }}\sphinxbfcode{\sphinxupquote{parse}}}{\emph{headers: str}}{{ $\rightarrow$ tornado.httputil.HTTPHeaders}}
Returns a dictionary from HTTP header text.

\begin{sphinxVerbatim}[commandchars=\\\{\}]
\PYG{g+gp}{\PYGZgt{}\PYGZgt{}\PYGZgt{} }\PYG{n}{h} \PYG{o}{=} \PYG{n}{HTTPHeaders}\PYG{o}{.}\PYG{n}{parse}\PYG{p}{(}\PYG{l+s+s2}{\PYGZdq{}}\PYG{l+s+s2}{Content\PYGZhy{}Type: text/html}\PYG{l+s+se}{\PYGZbs{}r}\PYG{l+s+se}{\PYGZbs{}n}\PYG{l+s+s2}{Content\PYGZhy{}Length: 42}\PYG{l+s+se}{\PYGZbs{}r}\PYG{l+s+se}{\PYGZbs{}n}\PYG{l+s+s2}{\PYGZdq{}}\PYG{p}{)}
\PYG{g+gp}{\PYGZgt{}\PYGZgt{}\PYGZgt{} }\PYG{n+nb}{sorted}\PYG{p}{(}\PYG{n}{h}\PYG{o}{.}\PYG{n}{items}\PYG{p}{(}\PYG{p}{)}\PYG{p}{)}
\PYG{g+go}{[(\PYGZsq{}Content\PYGZhy{}Length\PYGZsq{}, \PYGZsq{}42\PYGZsq{}), (\PYGZsq{}Content\PYGZhy{}Type\PYGZsq{}, \PYGZsq{}text/html\PYGZsq{})]}
\end{sphinxVerbatim}

\DUrole{versionmodified,changed}{Changed in version 5.1: }Raises {\hyperref[\detokenize{httputil:tornado.httputil.HTTPInputError}]{\sphinxcrossref{\sphinxcode{\sphinxupquote{HTTPInputError}}}}} on malformed headers instead of a
mix of \sphinxhref{https://docs.python.org/3.6/library/exceptions.html\#KeyError}{\sphinxcode{\sphinxupquote{KeyError}}}, and \sphinxhref{https://docs.python.org/3.6/library/exceptions.html\#ValueError}{\sphinxcode{\sphinxupquote{ValueError}}}.

\end{fulllineitems}


\end{fulllineitems}

\index{HTTPServerRequest (class in tornado.httputil)@\spxentry{HTTPServerRequest}\spxextra{class in tornado.httputil}}

\begin{fulllineitems}
\phantomsection\label{\detokenize{httputil:tornado.httputil.HTTPServerRequest}}\pysiglinewithargsret{\sphinxbfcode{\sphinxupquote{class }}\sphinxcode{\sphinxupquote{tornado.httputil.}}\sphinxbfcode{\sphinxupquote{HTTPServerRequest}}}{\emph{method: str = None}, \emph{uri: str = None}, \emph{version: str = 'HTTP/1.0'}, \emph{headers: tornado.httputil.HTTPHeaders = None}, \emph{body: bytes = None}, \emph{host: str = None}, \emph{files: Dict{[}str}, \emph{List{[}HTTPFile{]}{]} = None}, \emph{connection: Optional{[}tornado.httputil.HTTPConnection{]} = None}, \emph{start\_line: Optional{[}tornado.httputil.RequestStartLine{]} = None}, \emph{server\_connection: object = None}}{}
A single HTTP request.

All attributes are type \sphinxhref{https://docs.python.org/3.6/library/stdtypes.html\#str}{\sphinxcode{\sphinxupquote{str}}} unless otherwise noted.
\index{method (tornado.httputil.HTTPServerRequest attribute)@\spxentry{method}\spxextra{tornado.httputil.HTTPServerRequest attribute}}

\begin{fulllineitems}
\phantomsection\label{\detokenize{httputil:tornado.httputil.HTTPServerRequest.method}}\pysigline{\sphinxbfcode{\sphinxupquote{method}}}
HTTP request method, e.g. “GET” or “POST”

\end{fulllineitems}

\index{uri (tornado.httputil.HTTPServerRequest attribute)@\spxentry{uri}\spxextra{tornado.httputil.HTTPServerRequest attribute}}

\begin{fulllineitems}
\phantomsection\label{\detokenize{httputil:tornado.httputil.HTTPServerRequest.uri}}\pysigline{\sphinxbfcode{\sphinxupquote{uri}}}
The requested uri.

\end{fulllineitems}

\index{path (tornado.httputil.HTTPServerRequest attribute)@\spxentry{path}\spxextra{tornado.httputil.HTTPServerRequest attribute}}

\begin{fulllineitems}
\phantomsection\label{\detokenize{httputil:tornado.httputil.HTTPServerRequest.path}}\pysigline{\sphinxbfcode{\sphinxupquote{path}}}
The path portion of {\hyperref[\detokenize{httputil:tornado.httputil.HTTPServerRequest.uri}]{\sphinxcrossref{\sphinxcode{\sphinxupquote{uri}}}}}

\end{fulllineitems}

\index{query (tornado.httputil.HTTPServerRequest attribute)@\spxentry{query}\spxextra{tornado.httputil.HTTPServerRequest attribute}}

\begin{fulllineitems}
\phantomsection\label{\detokenize{httputil:tornado.httputil.HTTPServerRequest.query}}\pysigline{\sphinxbfcode{\sphinxupquote{query}}}
The query portion of {\hyperref[\detokenize{httputil:tornado.httputil.HTTPServerRequest.uri}]{\sphinxcrossref{\sphinxcode{\sphinxupquote{uri}}}}}

\end{fulllineitems}

\index{version (tornado.httputil.HTTPServerRequest attribute)@\spxentry{version}\spxextra{tornado.httputil.HTTPServerRequest attribute}}

\begin{fulllineitems}
\phantomsection\label{\detokenize{httputil:tornado.httputil.HTTPServerRequest.version}}\pysigline{\sphinxbfcode{\sphinxupquote{version}}}
HTTP version specified in request, e.g. “HTTP/1.1”

\end{fulllineitems}

\index{headers (tornado.httputil.HTTPServerRequest attribute)@\spxentry{headers}\spxextra{tornado.httputil.HTTPServerRequest attribute}}

\begin{fulllineitems}
\phantomsection\label{\detokenize{httputil:tornado.httputil.HTTPServerRequest.headers}}\pysigline{\sphinxbfcode{\sphinxupquote{headers}}}
{\hyperref[\detokenize{httputil:tornado.httputil.HTTPHeaders}]{\sphinxcrossref{\sphinxcode{\sphinxupquote{HTTPHeaders}}}}} dictionary-like object for request headers.  Acts like
a case-insensitive dictionary with additional methods for repeated
headers.

\end{fulllineitems}

\index{body (tornado.httputil.HTTPServerRequest attribute)@\spxentry{body}\spxextra{tornado.httputil.HTTPServerRequest attribute}}

\begin{fulllineitems}
\phantomsection\label{\detokenize{httputil:tornado.httputil.HTTPServerRequest.body}}\pysigline{\sphinxbfcode{\sphinxupquote{body}}}
Request body, if present, as a byte string.

\end{fulllineitems}

\index{remote\_ip (tornado.httputil.HTTPServerRequest attribute)@\spxentry{remote\_ip}\spxextra{tornado.httputil.HTTPServerRequest attribute}}

\begin{fulllineitems}
\phantomsection\label{\detokenize{httputil:tornado.httputil.HTTPServerRequest.remote_ip}}\pysigline{\sphinxbfcode{\sphinxupquote{remote\_ip}}}
Client’s IP address as a string.  If \sphinxcode{\sphinxupquote{HTTPServer.xheaders}} is set,
will pass along the real IP address provided by a load balancer
in the \sphinxcode{\sphinxupquote{X-Real-Ip}} or \sphinxcode{\sphinxupquote{X-Forwarded-For}} header.

\end{fulllineitems}


\DUrole{versionmodified,changed}{Changed in version 3.1: }The list format of \sphinxcode{\sphinxupquote{X-Forwarded-For}} is now supported.
\index{protocol (tornado.httputil.HTTPServerRequest attribute)@\spxentry{protocol}\spxextra{tornado.httputil.HTTPServerRequest attribute}}

\begin{fulllineitems}
\phantomsection\label{\detokenize{httputil:tornado.httputil.HTTPServerRequest.protocol}}\pysigline{\sphinxbfcode{\sphinxupquote{protocol}}}
The protocol used, either “http” or “https”.  If \sphinxcode{\sphinxupquote{HTTPServer.xheaders}}
is set, will pass along the protocol used by a load balancer if
reported via an \sphinxcode{\sphinxupquote{X-Scheme}} header.

\end{fulllineitems}

\index{host (tornado.httputil.HTTPServerRequest attribute)@\spxentry{host}\spxextra{tornado.httputil.HTTPServerRequest attribute}}

\begin{fulllineitems}
\phantomsection\label{\detokenize{httputil:tornado.httputil.HTTPServerRequest.host}}\pysigline{\sphinxbfcode{\sphinxupquote{host}}}
The requested hostname, usually taken from the \sphinxcode{\sphinxupquote{Host}} header.

\end{fulllineitems}

\index{arguments (tornado.httputil.HTTPServerRequest attribute)@\spxentry{arguments}\spxextra{tornado.httputil.HTTPServerRequest attribute}}

\begin{fulllineitems}
\phantomsection\label{\detokenize{httputil:tornado.httputil.HTTPServerRequest.arguments}}\pysigline{\sphinxbfcode{\sphinxupquote{arguments}}}
GET/POST arguments are available in the arguments property, which
maps arguments names to lists of values (to support multiple values
for individual names). Names are of type \sphinxhref{https://docs.python.org/3.6/library/stdtypes.html\#str}{\sphinxcode{\sphinxupquote{str}}}, while arguments
are byte strings.  Note that this is different from
{\hyperref[\detokenize{web:tornado.web.RequestHandler.get_argument}]{\sphinxcrossref{\sphinxcode{\sphinxupquote{RequestHandler.get\_argument}}}}}, which returns argument values as
unicode strings.

\end{fulllineitems}

\index{query\_arguments (tornado.httputil.HTTPServerRequest attribute)@\spxentry{query\_arguments}\spxextra{tornado.httputil.HTTPServerRequest attribute}}

\begin{fulllineitems}
\phantomsection\label{\detokenize{httputil:tornado.httputil.HTTPServerRequest.query_arguments}}\pysigline{\sphinxbfcode{\sphinxupquote{query\_arguments}}}
Same format as \sphinxcode{\sphinxupquote{arguments}}, but contains only arguments extracted
from the query string.

\DUrole{versionmodified,added}{New in version 3.2.}

\end{fulllineitems}

\index{body\_arguments (tornado.httputil.HTTPServerRequest attribute)@\spxentry{body\_arguments}\spxextra{tornado.httputil.HTTPServerRequest attribute}}

\begin{fulllineitems}
\phantomsection\label{\detokenize{httputil:tornado.httputil.HTTPServerRequest.body_arguments}}\pysigline{\sphinxbfcode{\sphinxupquote{body\_arguments}}}
Same format as \sphinxcode{\sphinxupquote{arguments}}, but contains only arguments extracted
from the request body.

\DUrole{versionmodified,added}{New in version 3.2.}

\end{fulllineitems}

\index{files (tornado.httputil.HTTPServerRequest attribute)@\spxentry{files}\spxextra{tornado.httputil.HTTPServerRequest attribute}}

\begin{fulllineitems}
\phantomsection\label{\detokenize{httputil:tornado.httputil.HTTPServerRequest.files}}\pysigline{\sphinxbfcode{\sphinxupquote{files}}}
File uploads are available in the files property, which maps file
names to lists of {\hyperref[\detokenize{httputil:tornado.httputil.HTTPFile}]{\sphinxcrossref{\sphinxcode{\sphinxupquote{HTTPFile}}}}}.

\end{fulllineitems}

\index{connection (tornado.httputil.HTTPServerRequest attribute)@\spxentry{connection}\spxextra{tornado.httputil.HTTPServerRequest attribute}}

\begin{fulllineitems}
\phantomsection\label{\detokenize{httputil:tornado.httputil.HTTPServerRequest.connection}}\pysigline{\sphinxbfcode{\sphinxupquote{connection}}}
An HTTP request is attached to a single HTTP connection, which can
be accessed through the “connection” attribute. Since connections
are typically kept open in HTTP/1.1, multiple requests can be handled
sequentially on a single connection.

\end{fulllineitems}


\DUrole{versionmodified,changed}{Changed in version 4.0: }Moved from \sphinxcode{\sphinxupquote{tornado.httpserver.HTTPRequest}}.
\index{cookies (tornado.httputil.HTTPServerRequest attribute)@\spxentry{cookies}\spxextra{tornado.httputil.HTTPServerRequest attribute}}

\begin{fulllineitems}
\phantomsection\label{\detokenize{httputil:tornado.httputil.HTTPServerRequest.cookies}}\pysigline{\sphinxbfcode{\sphinxupquote{cookies}}}
A dictionary of \sphinxcode{\sphinxupquote{http.cookies.Morsel}} objects.

\end{fulllineitems}

\index{full\_url() (tornado.httputil.HTTPServerRequest method)@\spxentry{full\_url()}\spxextra{tornado.httputil.HTTPServerRequest method}}

\begin{fulllineitems}
\phantomsection\label{\detokenize{httputil:tornado.httputil.HTTPServerRequest.full_url}}\pysiglinewithargsret{\sphinxbfcode{\sphinxupquote{full\_url}}}{}{{ $\rightarrow$ str}}
Reconstructs the full URL for this request.

\end{fulllineitems}

\index{request\_time() (tornado.httputil.HTTPServerRequest method)@\spxentry{request\_time()}\spxextra{tornado.httputil.HTTPServerRequest method}}

\begin{fulllineitems}
\phantomsection\label{\detokenize{httputil:tornado.httputil.HTTPServerRequest.request_time}}\pysiglinewithargsret{\sphinxbfcode{\sphinxupquote{request\_time}}}{}{{ $\rightarrow$ float}}
Returns the amount of time it took for this request to execute.

\end{fulllineitems}

\index{get\_ssl\_certificate() (tornado.httputil.HTTPServerRequest method)@\spxentry{get\_ssl\_certificate()}\spxextra{tornado.httputil.HTTPServerRequest method}}

\begin{fulllineitems}
\phantomsection\label{\detokenize{httputil:tornado.httputil.HTTPServerRequest.get_ssl_certificate}}\pysiglinewithargsret{\sphinxbfcode{\sphinxupquote{get\_ssl\_certificate}}}{\emph{binary\_form: bool = False}}{{ $\rightarrow$ Union{[}None, Dict{[}KT, VT{]}, bytes{]}}}
Returns the client’s SSL certificate, if any.

To use client certificates, the HTTPServer’s
\sphinxhref{https://docs.python.org/3.6/library/ssl.html\#ssl.SSLContext.verify\_mode}{\sphinxcode{\sphinxupquote{ssl.SSLContext.verify\_mode}}} field must be set, e.g.:

\begin{sphinxVerbatim}[commandchars=\\\{\}]
\PYG{n}{ssl\PYGZus{}ctx} \PYG{o}{=} \PYG{n}{ssl}\PYG{o}{.}\PYG{n}{create\PYGZus{}default\PYGZus{}context}\PYG{p}{(}\PYG{n}{ssl}\PYG{o}{.}\PYG{n}{Purpose}\PYG{o}{.}\PYG{n}{CLIENT\PYGZus{}AUTH}\PYG{p}{)}
\PYG{n}{ssl\PYGZus{}ctx}\PYG{o}{.}\PYG{n}{load\PYGZus{}cert\PYGZus{}chain}\PYG{p}{(}\PYG{l+s+s2}{\PYGZdq{}}\PYG{l+s+s2}{foo.crt}\PYG{l+s+s2}{\PYGZdq{}}\PYG{p}{,} \PYG{l+s+s2}{\PYGZdq{}}\PYG{l+s+s2}{foo.key}\PYG{l+s+s2}{\PYGZdq{}}\PYG{p}{)}
\PYG{n}{ssl\PYGZus{}ctx}\PYG{o}{.}\PYG{n}{load\PYGZus{}verify\PYGZus{}locations}\PYG{p}{(}\PYG{l+s+s2}{\PYGZdq{}}\PYG{l+s+s2}{cacerts.pem}\PYG{l+s+s2}{\PYGZdq{}}\PYG{p}{)}
\PYG{n}{ssl\PYGZus{}ctx}\PYG{o}{.}\PYG{n}{verify\PYGZus{}mode} \PYG{o}{=} \PYG{n}{ssl}\PYG{o}{.}\PYG{n}{CERT\PYGZus{}REQUIRED}
\PYG{n}{server} \PYG{o}{=} \PYG{n}{HTTPServer}\PYG{p}{(}\PYG{n}{app}\PYG{p}{,} \PYG{n}{ssl\PYGZus{}options}\PYG{o}{=}\PYG{n}{ssl\PYGZus{}ctx}\PYG{p}{)}
\end{sphinxVerbatim}

By default, the return value is a dictionary (or None, if no
client certificate is present).  If \sphinxcode{\sphinxupquote{binary\_form}} is true, a
DER-encoded form of the certificate is returned instead.  See
SSLSocket.getpeercert() in the standard library for more
details.
\sphinxurl{http://docs.python.org/library/ssl.html\#sslsocket-objects}

\end{fulllineitems}


\end{fulllineitems}

\index{HTTPInputError@\spxentry{HTTPInputError}}

\begin{fulllineitems}
\phantomsection\label{\detokenize{httputil:tornado.httputil.HTTPInputError}}\pysigline{\sphinxbfcode{\sphinxupquote{exception }}\sphinxcode{\sphinxupquote{tornado.httputil.}}\sphinxbfcode{\sphinxupquote{HTTPInputError}}}
Exception class for malformed HTTP requests or responses
from remote sources.

\DUrole{versionmodified,added}{New in version 4.0.}

\end{fulllineitems}

\index{HTTPOutputError@\spxentry{HTTPOutputError}}

\begin{fulllineitems}
\phantomsection\label{\detokenize{httputil:tornado.httputil.HTTPOutputError}}\pysigline{\sphinxbfcode{\sphinxupquote{exception }}\sphinxcode{\sphinxupquote{tornado.httputil.}}\sphinxbfcode{\sphinxupquote{HTTPOutputError}}}
Exception class for errors in HTTP output.

\DUrole{versionmodified,added}{New in version 4.0.}

\end{fulllineitems}

\index{HTTPServerConnectionDelegate (class in tornado.httputil)@\spxentry{HTTPServerConnectionDelegate}\spxextra{class in tornado.httputil}}

\begin{fulllineitems}
\phantomsection\label{\detokenize{httputil:tornado.httputil.HTTPServerConnectionDelegate}}\pysigline{\sphinxbfcode{\sphinxupquote{class }}\sphinxcode{\sphinxupquote{tornado.httputil.}}\sphinxbfcode{\sphinxupquote{HTTPServerConnectionDelegate}}}
Implement this interface to handle requests from {\hyperref[\detokenize{httpserver:tornado.httpserver.HTTPServer}]{\sphinxcrossref{\sphinxcode{\sphinxupquote{HTTPServer}}}}}.

\DUrole{versionmodified,added}{New in version 4.0.}
\index{start\_request() (tornado.httputil.HTTPServerConnectionDelegate method)@\spxentry{start\_request()}\spxextra{tornado.httputil.HTTPServerConnectionDelegate method}}

\begin{fulllineitems}
\phantomsection\label{\detokenize{httputil:tornado.httputil.HTTPServerConnectionDelegate.start_request}}\pysiglinewithargsret{\sphinxbfcode{\sphinxupquote{start\_request}}}{\emph{server\_conn: object}, \emph{request\_conn: tornado.httputil.HTTPConnection}}{{ $\rightarrow$ tornado.httputil.HTTPMessageDelegate}}
This method is called by the server when a new request has started.
\begin{quote}\begin{description}
\item[{Parameters}] \leavevmode\begin{itemize}
\item {} 
\sphinxstyleliteralstrong{\sphinxupquote{server\_conn}} \textendash{} is an opaque object representing the long-lived
(e.g. tcp-level) connection.

\item {} 
\sphinxstyleliteralstrong{\sphinxupquote{request\_conn}} \textendash{} is a {\hyperref[\detokenize{httputil:tornado.httputil.HTTPConnection}]{\sphinxcrossref{\sphinxcode{\sphinxupquote{HTTPConnection}}}}} object for a single
request/response exchange.

\end{itemize}

\end{description}\end{quote}

This method should return a {\hyperref[\detokenize{httputil:tornado.httputil.HTTPMessageDelegate}]{\sphinxcrossref{\sphinxcode{\sphinxupquote{HTTPMessageDelegate}}}}}.

\end{fulllineitems}

\index{on\_close() (tornado.httputil.HTTPServerConnectionDelegate method)@\spxentry{on\_close()}\spxextra{tornado.httputil.HTTPServerConnectionDelegate method}}

\begin{fulllineitems}
\phantomsection\label{\detokenize{httputil:tornado.httputil.HTTPServerConnectionDelegate.on_close}}\pysiglinewithargsret{\sphinxbfcode{\sphinxupquote{on\_close}}}{\emph{server\_conn: object}}{{ $\rightarrow$ None}}
This method is called when a connection has been closed.
\begin{quote}\begin{description}
\item[{Parameters}] \leavevmode
\sphinxstyleliteralstrong{\sphinxupquote{server\_conn}} \textendash{} is a server connection that has previously been
passed to \sphinxcode{\sphinxupquote{start\_request}}.

\end{description}\end{quote}

\end{fulllineitems}


\end{fulllineitems}

\index{HTTPMessageDelegate (class in tornado.httputil)@\spxentry{HTTPMessageDelegate}\spxextra{class in tornado.httputil}}

\begin{fulllineitems}
\phantomsection\label{\detokenize{httputil:tornado.httputil.HTTPMessageDelegate}}\pysigline{\sphinxbfcode{\sphinxupquote{class }}\sphinxcode{\sphinxupquote{tornado.httputil.}}\sphinxbfcode{\sphinxupquote{HTTPMessageDelegate}}}
Implement this interface to handle an HTTP request or response.

\DUrole{versionmodified,added}{New in version 4.0.}
\index{headers\_received() (tornado.httputil.HTTPMessageDelegate method)@\spxentry{headers\_received()}\spxextra{tornado.httputil.HTTPMessageDelegate method}}

\begin{fulllineitems}
\phantomsection\label{\detokenize{httputil:tornado.httputil.HTTPMessageDelegate.headers_received}}\pysiglinewithargsret{\sphinxbfcode{\sphinxupquote{headers\_received}}}{\emph{start\_line: Union{[}RequestStartLine, ResponseStartLine{]}, headers: tornado.httputil.HTTPHeaders}}{{ $\rightarrow$ Optional{[}Awaitable{[}None{]}{]}}}
Called when the HTTP headers have been received and parsed.
\begin{quote}\begin{description}
\item[{Parameters}] \leavevmode\begin{itemize}
\item {} 
\sphinxstyleliteralstrong{\sphinxupquote{start\_line}} \textendash{} a {\hyperref[\detokenize{httputil:tornado.httputil.RequestStartLine}]{\sphinxcrossref{\sphinxcode{\sphinxupquote{RequestStartLine}}}}} or {\hyperref[\detokenize{httputil:tornado.httputil.ResponseStartLine}]{\sphinxcrossref{\sphinxcode{\sphinxupquote{ResponseStartLine}}}}}
depending on whether this is a client or server message.

\item {} 
\sphinxstyleliteralstrong{\sphinxupquote{headers}} \textendash{} a {\hyperref[\detokenize{httputil:tornado.httputil.HTTPHeaders}]{\sphinxcrossref{\sphinxcode{\sphinxupquote{HTTPHeaders}}}}} instance.

\end{itemize}

\end{description}\end{quote}

Some {\hyperref[\detokenize{httputil:tornado.httputil.HTTPConnection}]{\sphinxcrossref{\sphinxcode{\sphinxupquote{HTTPConnection}}}}} methods can only be called during
\sphinxcode{\sphinxupquote{headers\_received}}.

May return a {\hyperref[\detokenize{concurrent:tornado.concurrent.Future}]{\sphinxcrossref{\sphinxcode{\sphinxupquote{Future}}}}}; if it does the body will not be read
until it is done.

\end{fulllineitems}

\index{data\_received() (tornado.httputil.HTTPMessageDelegate method)@\spxentry{data\_received()}\spxextra{tornado.httputil.HTTPMessageDelegate method}}

\begin{fulllineitems}
\phantomsection\label{\detokenize{httputil:tornado.httputil.HTTPMessageDelegate.data_received}}\pysiglinewithargsret{\sphinxbfcode{\sphinxupquote{data\_received}}}{\emph{chunk: bytes}}{{ $\rightarrow$ Optional{[}Awaitable{[}None{]}{]}}}
Called when a chunk of data has been received.

May return a {\hyperref[\detokenize{concurrent:tornado.concurrent.Future}]{\sphinxcrossref{\sphinxcode{\sphinxupquote{Future}}}}} for flow control.

\end{fulllineitems}

\index{finish() (tornado.httputil.HTTPMessageDelegate method)@\spxentry{finish()}\spxextra{tornado.httputil.HTTPMessageDelegate method}}

\begin{fulllineitems}
\phantomsection\label{\detokenize{httputil:tornado.httputil.HTTPMessageDelegate.finish}}\pysiglinewithargsret{\sphinxbfcode{\sphinxupquote{finish}}}{}{{ $\rightarrow$ None}}
Called after the last chunk of data has been received.

\end{fulllineitems}

\index{on\_connection\_close() (tornado.httputil.HTTPMessageDelegate method)@\spxentry{on\_connection\_close()}\spxextra{tornado.httputil.HTTPMessageDelegate method}}

\begin{fulllineitems}
\phantomsection\label{\detokenize{httputil:tornado.httputil.HTTPMessageDelegate.on_connection_close}}\pysiglinewithargsret{\sphinxbfcode{\sphinxupquote{on\_connection\_close}}}{}{{ $\rightarrow$ None}}
Called if the connection is closed without finishing the request.

If \sphinxcode{\sphinxupquote{headers\_received}} is called, either \sphinxcode{\sphinxupquote{finish}} or
\sphinxcode{\sphinxupquote{on\_connection\_close}} will be called, but not both.

\end{fulllineitems}


\end{fulllineitems}

\index{HTTPConnection (class in tornado.httputil)@\spxentry{HTTPConnection}\spxextra{class in tornado.httputil}}

\begin{fulllineitems}
\phantomsection\label{\detokenize{httputil:tornado.httputil.HTTPConnection}}\pysigline{\sphinxbfcode{\sphinxupquote{class }}\sphinxcode{\sphinxupquote{tornado.httputil.}}\sphinxbfcode{\sphinxupquote{HTTPConnection}}}
Applications use this interface to write their responses.

\DUrole{versionmodified,added}{New in version 4.0.}
\index{write\_headers() (tornado.httputil.HTTPConnection method)@\spxentry{write\_headers()}\spxextra{tornado.httputil.HTTPConnection method}}

\begin{fulllineitems}
\phantomsection\label{\detokenize{httputil:tornado.httputil.HTTPConnection.write_headers}}\pysiglinewithargsret{\sphinxbfcode{\sphinxupquote{write\_headers}}}{\emph{start\_line: Union{[}RequestStartLine, ResponseStartLine{]}, headers: tornado.httputil.HTTPHeaders, chunk: bytes = None}}{{ $\rightarrow$ Future{[}None{]}}}
Write an HTTP header block.
\begin{quote}\begin{description}
\item[{Parameters}] \leavevmode\begin{itemize}
\item {} 
\sphinxstyleliteralstrong{\sphinxupquote{start\_line}} \textendash{} a {\hyperref[\detokenize{httputil:tornado.httputil.RequestStartLine}]{\sphinxcrossref{\sphinxcode{\sphinxupquote{RequestStartLine}}}}} or {\hyperref[\detokenize{httputil:tornado.httputil.ResponseStartLine}]{\sphinxcrossref{\sphinxcode{\sphinxupquote{ResponseStartLine}}}}}.

\item {} 
\sphinxstyleliteralstrong{\sphinxupquote{headers}} \textendash{} a {\hyperref[\detokenize{httputil:tornado.httputil.HTTPHeaders}]{\sphinxcrossref{\sphinxcode{\sphinxupquote{HTTPHeaders}}}}} instance.

\item {} 
\sphinxstyleliteralstrong{\sphinxupquote{chunk}} \textendash{} the first (optional) chunk of data.  This is an optimization
so that small responses can be written in the same call as their
headers.

\end{itemize}

\end{description}\end{quote}

The \sphinxcode{\sphinxupquote{version}} field of \sphinxcode{\sphinxupquote{start\_line}} is ignored.

Returns a future for flow control.

\DUrole{versionmodified,changed}{Changed in version 6.0: }The \sphinxcode{\sphinxupquote{callback}} argument was removed.

\end{fulllineitems}

\index{write() (tornado.httputil.HTTPConnection method)@\spxentry{write()}\spxextra{tornado.httputil.HTTPConnection method}}

\begin{fulllineitems}
\phantomsection\label{\detokenize{httputil:tornado.httputil.HTTPConnection.write}}\pysiglinewithargsret{\sphinxbfcode{\sphinxupquote{write}}}{\emph{chunk: bytes}}{{ $\rightarrow$ Future{[}None{]}}}
Writes a chunk of body data.

Returns a future for flow control.

\DUrole{versionmodified,changed}{Changed in version 6.0: }The \sphinxcode{\sphinxupquote{callback}} argument was removed.

\end{fulllineitems}

\index{finish() (tornado.httputil.HTTPConnection method)@\spxentry{finish()}\spxextra{tornado.httputil.HTTPConnection method}}

\begin{fulllineitems}
\phantomsection\label{\detokenize{httputil:tornado.httputil.HTTPConnection.finish}}\pysiglinewithargsret{\sphinxbfcode{\sphinxupquote{finish}}}{}{{ $\rightarrow$ None}}
Indicates that the last body data has been written.

\end{fulllineitems}


\end{fulllineitems}

\index{url\_concat() (in module tornado.httputil)@\spxentry{url\_concat()}\spxextra{in module tornado.httputil}}

\begin{fulllineitems}
\phantomsection\label{\detokenize{httputil:tornado.httputil.url_concat}}\pysiglinewithargsret{\sphinxcode{\sphinxupquote{tornado.httputil.}}\sphinxbfcode{\sphinxupquote{url\_concat}}}{\emph{url: str, args: Union{[}None, Dict{[}str, str{]}, List{[}Tuple{[}str, str{]}{]}, Tuple{[}Tuple{[}str, str{]}, ...{]}{]}}}{{ $\rightarrow$ str}}
Concatenate url and arguments regardless of whether
url has existing query parameters.

\sphinxcode{\sphinxupquote{args}} may be either a dictionary or a list of key-value pairs
(the latter allows for multiple values with the same key.

\begin{sphinxVerbatim}[commandchars=\\\{\}]
\PYG{g+gp}{\PYGZgt{}\PYGZgt{}\PYGZgt{} }\PYG{n}{url\PYGZus{}concat}\PYG{p}{(}\PYG{l+s+s2}{\PYGZdq{}}\PYG{l+s+s2}{http://example.com/foo}\PYG{l+s+s2}{\PYGZdq{}}\PYG{p}{,} \PYG{n+nb}{dict}\PYG{p}{(}\PYG{n}{c}\PYG{o}{=}\PYG{l+s+s2}{\PYGZdq{}}\PYG{l+s+s2}{d}\PYG{l+s+s2}{\PYGZdq{}}\PYG{p}{)}\PYG{p}{)}
\PYG{g+go}{\PYGZsq{}http://example.com/foo?c=d\PYGZsq{}}
\PYG{g+gp}{\PYGZgt{}\PYGZgt{}\PYGZgt{} }\PYG{n}{url\PYGZus{}concat}\PYG{p}{(}\PYG{l+s+s2}{\PYGZdq{}}\PYG{l+s+s2}{http://example.com/foo?a=b}\PYG{l+s+s2}{\PYGZdq{}}\PYG{p}{,} \PYG{n+nb}{dict}\PYG{p}{(}\PYG{n}{c}\PYG{o}{=}\PYG{l+s+s2}{\PYGZdq{}}\PYG{l+s+s2}{d}\PYG{l+s+s2}{\PYGZdq{}}\PYG{p}{)}\PYG{p}{)}
\PYG{g+go}{\PYGZsq{}http://example.com/foo?a=b\PYGZam{}c=d\PYGZsq{}}
\PYG{g+gp}{\PYGZgt{}\PYGZgt{}\PYGZgt{} }\PYG{n}{url\PYGZus{}concat}\PYG{p}{(}\PYG{l+s+s2}{\PYGZdq{}}\PYG{l+s+s2}{http://example.com/foo?a=b}\PYG{l+s+s2}{\PYGZdq{}}\PYG{p}{,} \PYG{p}{[}\PYG{p}{(}\PYG{l+s+s2}{\PYGZdq{}}\PYG{l+s+s2}{c}\PYG{l+s+s2}{\PYGZdq{}}\PYG{p}{,} \PYG{l+s+s2}{\PYGZdq{}}\PYG{l+s+s2}{d}\PYG{l+s+s2}{\PYGZdq{}}\PYG{p}{)}\PYG{p}{,} \PYG{p}{(}\PYG{l+s+s2}{\PYGZdq{}}\PYG{l+s+s2}{c}\PYG{l+s+s2}{\PYGZdq{}}\PYG{p}{,} \PYG{l+s+s2}{\PYGZdq{}}\PYG{l+s+s2}{d2}\PYG{l+s+s2}{\PYGZdq{}}\PYG{p}{)}\PYG{p}{]}\PYG{p}{)}
\PYG{g+go}{\PYGZsq{}http://example.com/foo?a=b\PYGZam{}c=d\PYGZam{}c=d2\PYGZsq{}}
\end{sphinxVerbatim}

\end{fulllineitems}

\index{HTTPFile (class in tornado.httputil)@\spxentry{HTTPFile}\spxextra{class in tornado.httputil}}

\begin{fulllineitems}
\phantomsection\label{\detokenize{httputil:tornado.httputil.HTTPFile}}\pysigline{\sphinxbfcode{\sphinxupquote{class }}\sphinxcode{\sphinxupquote{tornado.httputil.}}\sphinxbfcode{\sphinxupquote{HTTPFile}}}
Represents a file uploaded via a form.

For backwards compatibility, its instance attributes are also
accessible as dictionary keys.
\begin{itemize}
\item {} 
\sphinxcode{\sphinxupquote{filename}}

\item {} 
\sphinxcode{\sphinxupquote{body}}

\item {} 
\sphinxcode{\sphinxupquote{content\_type}}

\end{itemize}

\end{fulllineitems}

\index{parse\_body\_arguments() (in module tornado.httputil)@\spxentry{parse\_body\_arguments()}\spxextra{in module tornado.httputil}}

\begin{fulllineitems}
\phantomsection\label{\detokenize{httputil:tornado.httputil.parse_body_arguments}}\pysiglinewithargsret{\sphinxcode{\sphinxupquote{tornado.httputil.}}\sphinxbfcode{\sphinxupquote{parse\_body\_arguments}}}{\emph{content\_type: str, body: bytes, arguments: Dict{[}str, List{[}bytes{]}{]}, files: Dict{[}str, List{[}tornado.httputil.HTTPFile{]}{]}, headers: tornado.httputil.HTTPHeaders = None}}{{ $\rightarrow$ None}}
Parses a form request body.

Supports \sphinxcode{\sphinxupquote{application/x-www-form-urlencoded}} and
\sphinxcode{\sphinxupquote{multipart/form-data}}.  The \sphinxcode{\sphinxupquote{content\_type}} parameter should be
a string and \sphinxcode{\sphinxupquote{body}} should be a byte string.  The \sphinxcode{\sphinxupquote{arguments}}
and \sphinxcode{\sphinxupquote{files}} parameters are dictionaries that will be updated
with the parsed contents.

\end{fulllineitems}

\index{parse\_multipart\_form\_data() (in module tornado.httputil)@\spxentry{parse\_multipart\_form\_data()}\spxextra{in module tornado.httputil}}

\begin{fulllineitems}
\phantomsection\label{\detokenize{httputil:tornado.httputil.parse_multipart_form_data}}\pysiglinewithargsret{\sphinxcode{\sphinxupquote{tornado.httputil.}}\sphinxbfcode{\sphinxupquote{parse\_multipart\_form\_data}}}{\emph{boundary: bytes, data: bytes, arguments: Dict{[}str, List{[}bytes{]}{]}, files: Dict{[}str, List{[}tornado.httputil.HTTPFile{]}{]}}}{{ $\rightarrow$ None}}
Parses a \sphinxcode{\sphinxupquote{multipart/form-data}} body.

The \sphinxcode{\sphinxupquote{boundary}} and \sphinxcode{\sphinxupquote{data}} parameters are both byte strings.
The dictionaries given in the arguments and files parameters
will be updated with the contents of the body.

\DUrole{versionmodified,changed}{Changed in version 5.1: }Now recognizes non-ASCII filenames in RFC 2231/5987
(\sphinxcode{\sphinxupquote{filename*=}}) format.

\end{fulllineitems}

\index{format\_timestamp() (in module tornado.httputil)@\spxentry{format\_timestamp()}\spxextra{in module tornado.httputil}}

\begin{fulllineitems}
\phantomsection\label{\detokenize{httputil:tornado.httputil.format_timestamp}}\pysiglinewithargsret{\sphinxcode{\sphinxupquote{tornado.httputil.}}\sphinxbfcode{\sphinxupquote{format\_timestamp}}}{\emph{ts: Union{[}int, float, tuple, time.struct\_time, datetime.datetime{]}}}{{ $\rightarrow$ str}}
Formats a timestamp in the format used by HTTP.

The argument may be a numeric timestamp as returned by \sphinxhref{https://docs.python.org/3.6/library/time.html\#time.time}{\sphinxcode{\sphinxupquote{time.time}}},
a time tuple as returned by \sphinxhref{https://docs.python.org/3.6/library/time.html\#time.gmtime}{\sphinxcode{\sphinxupquote{time.gmtime}}}, or a \sphinxhref{https://docs.python.org/3.6/library/datetime.html\#datetime.datetime}{\sphinxcode{\sphinxupquote{datetime.datetime}}}
object.

\begin{sphinxVerbatim}[commandchars=\\\{\}]
\PYG{g+gp}{\PYGZgt{}\PYGZgt{}\PYGZgt{} }\PYG{n}{format\PYGZus{}timestamp}\PYG{p}{(}\PYG{l+m+mi}{1359312200}\PYG{p}{)}
\PYG{g+go}{\PYGZsq{}Sun, 27 Jan 2013 18:43:20 GMT\PYGZsq{}}
\end{sphinxVerbatim}

\end{fulllineitems}

\index{RequestStartLine (class in tornado.httputil)@\spxentry{RequestStartLine}\spxextra{class in tornado.httputil}}

\begin{fulllineitems}
\phantomsection\label{\detokenize{httputil:tornado.httputil.RequestStartLine}}\pysigline{\sphinxbfcode{\sphinxupquote{class }}\sphinxcode{\sphinxupquote{tornado.httputil.}}\sphinxbfcode{\sphinxupquote{RequestStartLine}}}
RequestStartLine(method, path, version)

Create new instance of RequestStartLine(method, path, version)
\index{method (tornado.httputil.RequestStartLine attribute)@\spxentry{method}\spxextra{tornado.httputil.RequestStartLine attribute}}

\begin{fulllineitems}
\phantomsection\label{\detokenize{httputil:tornado.httputil.RequestStartLine.method}}\pysigline{\sphinxbfcode{\sphinxupquote{method}}}
Alias for field number 0

\end{fulllineitems}

\index{path (tornado.httputil.RequestStartLine attribute)@\spxentry{path}\spxextra{tornado.httputil.RequestStartLine attribute}}

\begin{fulllineitems}
\phantomsection\label{\detokenize{httputil:tornado.httputil.RequestStartLine.path}}\pysigline{\sphinxbfcode{\sphinxupquote{path}}}
Alias for field number 1

\end{fulllineitems}

\index{version (tornado.httputil.RequestStartLine attribute)@\spxentry{version}\spxextra{tornado.httputil.RequestStartLine attribute}}

\begin{fulllineitems}
\phantomsection\label{\detokenize{httputil:tornado.httputil.RequestStartLine.version}}\pysigline{\sphinxbfcode{\sphinxupquote{version}}}
Alias for field number 2

\end{fulllineitems}


\end{fulllineitems}

\index{parse\_request\_start\_line() (in module tornado.httputil)@\spxentry{parse\_request\_start\_line()}\spxextra{in module tornado.httputil}}

\begin{fulllineitems}
\phantomsection\label{\detokenize{httputil:tornado.httputil.parse_request_start_line}}\pysiglinewithargsret{\sphinxcode{\sphinxupquote{tornado.httputil.}}\sphinxbfcode{\sphinxupquote{parse\_request\_start\_line}}}{\emph{line: str}}{{ $\rightarrow$ tornado.httputil.RequestStartLine}}
Returns a (method, path, version) tuple for an HTTP 1.x request line.

The response is a \sphinxhref{https://docs.python.org/3.6/library/collections.html\#collections.namedtuple}{\sphinxcode{\sphinxupquote{collections.namedtuple}}}.

\begin{sphinxVerbatim}[commandchars=\\\{\}]
\PYG{g+gp}{\PYGZgt{}\PYGZgt{}\PYGZgt{} }\PYG{n}{parse\PYGZus{}request\PYGZus{}start\PYGZus{}line}\PYG{p}{(}\PYG{l+s+s2}{\PYGZdq{}}\PYG{l+s+s2}{GET /foo HTTP/1.1}\PYG{l+s+s2}{\PYGZdq{}}\PYG{p}{)}
\PYG{g+go}{RequestStartLine(method=\PYGZsq{}GET\PYGZsq{}, path=\PYGZsq{}/foo\PYGZsq{}, version=\PYGZsq{}HTTP/1.1\PYGZsq{})}
\end{sphinxVerbatim}

\end{fulllineitems}

\index{ResponseStartLine (class in tornado.httputil)@\spxentry{ResponseStartLine}\spxextra{class in tornado.httputil}}

\begin{fulllineitems}
\phantomsection\label{\detokenize{httputil:tornado.httputil.ResponseStartLine}}\pysigline{\sphinxbfcode{\sphinxupquote{class }}\sphinxcode{\sphinxupquote{tornado.httputil.}}\sphinxbfcode{\sphinxupquote{ResponseStartLine}}}
ResponseStartLine(version, code, reason)

Create new instance of ResponseStartLine(version, code, reason)
\index{code (tornado.httputil.ResponseStartLine attribute)@\spxentry{code}\spxextra{tornado.httputil.ResponseStartLine attribute}}

\begin{fulllineitems}
\phantomsection\label{\detokenize{httputil:tornado.httputil.ResponseStartLine.code}}\pysigline{\sphinxbfcode{\sphinxupquote{code}}}
Alias for field number 1

\end{fulllineitems}

\index{reason (tornado.httputil.ResponseStartLine attribute)@\spxentry{reason}\spxextra{tornado.httputil.ResponseStartLine attribute}}

\begin{fulllineitems}
\phantomsection\label{\detokenize{httputil:tornado.httputil.ResponseStartLine.reason}}\pysigline{\sphinxbfcode{\sphinxupquote{reason}}}
Alias for field number 2

\end{fulllineitems}

\index{version (tornado.httputil.ResponseStartLine attribute)@\spxentry{version}\spxextra{tornado.httputil.ResponseStartLine attribute}}

\begin{fulllineitems}
\phantomsection\label{\detokenize{httputil:tornado.httputil.ResponseStartLine.version}}\pysigline{\sphinxbfcode{\sphinxupquote{version}}}
Alias for field number 0

\end{fulllineitems}


\end{fulllineitems}

\index{parse\_response\_start\_line() (in module tornado.httputil)@\spxentry{parse\_response\_start\_line()}\spxextra{in module tornado.httputil}}

\begin{fulllineitems}
\phantomsection\label{\detokenize{httputil:tornado.httputil.parse_response_start_line}}\pysiglinewithargsret{\sphinxcode{\sphinxupquote{tornado.httputil.}}\sphinxbfcode{\sphinxupquote{parse\_response\_start\_line}}}{\emph{line: str}}{{ $\rightarrow$ tornado.httputil.ResponseStartLine}}
Returns a (version, code, reason) tuple for an HTTP 1.x response line.

The response is a \sphinxhref{https://docs.python.org/3.6/library/collections.html\#collections.namedtuple}{\sphinxcode{\sphinxupquote{collections.namedtuple}}}.

\begin{sphinxVerbatim}[commandchars=\\\{\}]
\PYG{g+gp}{\PYGZgt{}\PYGZgt{}\PYGZgt{} }\PYG{n}{parse\PYGZus{}response\PYGZus{}start\PYGZus{}line}\PYG{p}{(}\PYG{l+s+s2}{\PYGZdq{}}\PYG{l+s+s2}{HTTP/1.1 200 OK}\PYG{l+s+s2}{\PYGZdq{}}\PYG{p}{)}
\PYG{g+go}{ResponseStartLine(version=\PYGZsq{}HTTP/1.1\PYGZsq{}, code=200, reason=\PYGZsq{}OK\PYGZsq{})}
\end{sphinxVerbatim}

\end{fulllineitems}

\index{encode\_username\_password() (in module tornado.httputil)@\spxentry{encode\_username\_password()}\spxextra{in module tornado.httputil}}

\begin{fulllineitems}
\phantomsection\label{\detokenize{httputil:tornado.httputil.encode_username_password}}\pysiglinewithargsret{\sphinxcode{\sphinxupquote{tornado.httputil.}}\sphinxbfcode{\sphinxupquote{encode\_username\_password}}}{\emph{username: Union{[}str, bytes{]}, password: Union{[}str, bytes{]}}}{{ $\rightarrow$ bytes}}
Encodes a username/password pair in the format used by HTTP auth.

The return value is a byte string in the form \sphinxcode{\sphinxupquote{username:password}}.

\DUrole{versionmodified,added}{New in version 5.1.}

\end{fulllineitems}

\index{split\_host\_and\_port() (in module tornado.httputil)@\spxentry{split\_host\_and\_port()}\spxextra{in module tornado.httputil}}

\begin{fulllineitems}
\phantomsection\label{\detokenize{httputil:tornado.httputil.split_host_and_port}}\pysiglinewithargsret{\sphinxcode{\sphinxupquote{tornado.httputil.}}\sphinxbfcode{\sphinxupquote{split\_host\_and\_port}}}{\emph{netloc: str}}{{ $\rightarrow$ Tuple{[}str, Optional{[}int{]}{]}}}
Returns \sphinxcode{\sphinxupquote{(host, port)}} tuple from \sphinxcode{\sphinxupquote{netloc}}.

Returned \sphinxcode{\sphinxupquote{port}} will be \sphinxcode{\sphinxupquote{None}} if not present.

\DUrole{versionmodified,added}{New in version 4.1.}

\end{fulllineitems}

\index{qs\_to\_qsl() (in module tornado.httputil)@\spxentry{qs\_to\_qsl()}\spxextra{in module tornado.httputil}}

\begin{fulllineitems}
\phantomsection\label{\detokenize{httputil:tornado.httputil.qs_to_qsl}}\pysiglinewithargsret{\sphinxcode{\sphinxupquote{tornado.httputil.}}\sphinxbfcode{\sphinxupquote{qs\_to\_qsl}}}{\emph{qs: Dict{[}str, List{[}AnyStr{]}{]}}}{{ $\rightarrow$ Iterable{[}Tuple{[}str, AnyStr{]}{]}}}
Generator converting a result of \sphinxcode{\sphinxupquote{parse\_qs}} back to name-value pairs.

\DUrole{versionmodified,added}{New in version 5.0.}

\end{fulllineitems}

\index{parse\_cookie() (in module tornado.httputil)@\spxentry{parse\_cookie()}\spxextra{in module tornado.httputil}}

\begin{fulllineitems}
\phantomsection\label{\detokenize{httputil:tornado.httputil.parse_cookie}}\pysiglinewithargsret{\sphinxcode{\sphinxupquote{tornado.httputil.}}\sphinxbfcode{\sphinxupquote{parse\_cookie}}}{\emph{cookie: str}}{{ $\rightarrow$ Dict{[}str, str{]}}}
Parse a \sphinxcode{\sphinxupquote{Cookie}} HTTP header into a dict of name/value pairs.

This function attempts to mimic browser cookie parsing behavior;
it specifically does not follow any of the cookie-related RFCs
(because browsers don’t either).

The algorithm used is identical to that used by Django version 1.9.10.

\DUrole{versionmodified,added}{New in version 4.4.2.}

\end{fulllineitems}



\subsection{\sphinxstyleliteralintitle{\sphinxupquote{tornado.http1connection}} \textendash{} HTTP/1.x client/server implementation}
\label{\detokenize{http1connection:module-tornado.http1connection}}\label{\detokenize{http1connection:tornado-http1connection-http-1-x-client-server-implementation}}\label{\detokenize{http1connection::doc}}\index{tornado.http1connection (module)@\spxentry{tornado.http1connection}\spxextra{module}}
Client and server implementations of HTTP/1.x.

\DUrole{versionmodified,added}{New in version 4.0.}
\index{HTTP1ConnectionParameters (class in tornado.http1connection)@\spxentry{HTTP1ConnectionParameters}\spxextra{class in tornado.http1connection}}

\begin{fulllineitems}
\phantomsection\label{\detokenize{http1connection:tornado.http1connection.HTTP1ConnectionParameters}}\pysiglinewithargsret{\sphinxbfcode{\sphinxupquote{class }}\sphinxcode{\sphinxupquote{tornado.http1connection.}}\sphinxbfcode{\sphinxupquote{HTTP1ConnectionParameters}}}{\emph{no\_keep\_alive: bool = False}, \emph{chunk\_size: int = None}, \emph{max\_header\_size: int = None}, \emph{header\_timeout: float = None}, \emph{max\_body\_size: int = None}, \emph{body\_timeout: float = None}, \emph{decompress: bool = False}}{}
Parameters for {\hyperref[\detokenize{http1connection:tornado.http1connection.HTTP1Connection}]{\sphinxcrossref{\sphinxcode{\sphinxupquote{HTTP1Connection}}}}} and {\hyperref[\detokenize{http1connection:tornado.http1connection.HTTP1ServerConnection}]{\sphinxcrossref{\sphinxcode{\sphinxupquote{HTTP1ServerConnection}}}}}.
\begin{quote}\begin{description}
\item[{Parameters}] \leavevmode\begin{itemize}
\item {} 
\sphinxstyleliteralstrong{\sphinxupquote{no\_keep\_alive}} (\sphinxhref{https://docs.python.org/3.6/library/functions.html\#bool}{\sphinxstyleliteralemphasis{\sphinxupquote{bool}}}) \textendash{} If true, always close the connection after
one request.

\item {} 
\sphinxstyleliteralstrong{\sphinxupquote{chunk\_size}} (\sphinxhref{https://docs.python.org/3.6/library/functions.html\#int}{\sphinxstyleliteralemphasis{\sphinxupquote{int}}}) \textendash{} how much data to read into memory at once

\item {} 
\sphinxstyleliteralstrong{\sphinxupquote{max\_header\_size}} (\sphinxhref{https://docs.python.org/3.6/library/functions.html\#int}{\sphinxstyleliteralemphasis{\sphinxupquote{int}}}) \textendash{} maximum amount of data for HTTP headers

\item {} 
\sphinxstyleliteralstrong{\sphinxupquote{header\_timeout}} (\sphinxhref{https://docs.python.org/3.6/library/functions.html\#float}{\sphinxstyleliteralemphasis{\sphinxupquote{float}}}) \textendash{} how long to wait for all headers (seconds)

\item {} 
\sphinxstyleliteralstrong{\sphinxupquote{max\_body\_size}} (\sphinxhref{https://docs.python.org/3.6/library/functions.html\#int}{\sphinxstyleliteralemphasis{\sphinxupquote{int}}}) \textendash{} maximum amount of data for body

\item {} 
\sphinxstyleliteralstrong{\sphinxupquote{body\_timeout}} (\sphinxhref{https://docs.python.org/3.6/library/functions.html\#float}{\sphinxstyleliteralemphasis{\sphinxupquote{float}}}) \textendash{} how long to wait while reading body (seconds)

\item {} 
\sphinxstyleliteralstrong{\sphinxupquote{decompress}} (\sphinxhref{https://docs.python.org/3.6/library/functions.html\#bool}{\sphinxstyleliteralemphasis{\sphinxupquote{bool}}}) \textendash{} if true, decode incoming
\sphinxcode{\sphinxupquote{Content-Encoding: gzip}}

\end{itemize}

\end{description}\end{quote}

\end{fulllineitems}

\index{HTTP1Connection (class in tornado.http1connection)@\spxentry{HTTP1Connection}\spxextra{class in tornado.http1connection}}

\begin{fulllineitems}
\phantomsection\label{\detokenize{http1connection:tornado.http1connection.HTTP1Connection}}\pysiglinewithargsret{\sphinxbfcode{\sphinxupquote{class }}\sphinxcode{\sphinxupquote{tornado.http1connection.}}\sphinxbfcode{\sphinxupquote{HTTP1Connection}}}{\emph{stream: tornado.iostream.IOStream}, \emph{is\_client: bool}, \emph{params: tornado.http1connection.HTTP1ConnectionParameters = None}, \emph{context: object = None}}{}
Implements the HTTP/1.x protocol.

This class can be on its own for clients, or via {\hyperref[\detokenize{http1connection:tornado.http1connection.HTTP1ServerConnection}]{\sphinxcrossref{\sphinxcode{\sphinxupquote{HTTP1ServerConnection}}}}}
for servers.
\begin{quote}\begin{description}
\item[{Parameters}] \leavevmode\begin{itemize}
\item {} 
\sphinxstyleliteralstrong{\sphinxupquote{stream}} \textendash{} an {\hyperref[\detokenize{iostream:tornado.iostream.IOStream}]{\sphinxcrossref{\sphinxcode{\sphinxupquote{IOStream}}}}}

\item {} 
\sphinxstyleliteralstrong{\sphinxupquote{is\_client}} (\sphinxhref{https://docs.python.org/3.6/library/functions.html\#bool}{\sphinxstyleliteralemphasis{\sphinxupquote{bool}}}) \textendash{} client or server

\item {} 
\sphinxstyleliteralstrong{\sphinxupquote{params}} \textendash{} a {\hyperref[\detokenize{http1connection:tornado.http1connection.HTTP1ConnectionParameters}]{\sphinxcrossref{\sphinxcode{\sphinxupquote{HTTP1ConnectionParameters}}}}} instance or \sphinxcode{\sphinxupquote{None}}

\item {} 
\sphinxstyleliteralstrong{\sphinxupquote{context}} \textendash{} an opaque application-defined object that can be accessed
as \sphinxcode{\sphinxupquote{connection.context}}.

\end{itemize}

\end{description}\end{quote}
\index{read\_response() (tornado.http1connection.HTTP1Connection method)@\spxentry{read\_response()}\spxextra{tornado.http1connection.HTTP1Connection method}}

\begin{fulllineitems}
\phantomsection\label{\detokenize{http1connection:tornado.http1connection.HTTP1Connection.read_response}}\pysiglinewithargsret{\sphinxbfcode{\sphinxupquote{read\_response}}}{\emph{delegate: tornado.httputil.HTTPMessageDelegate}}{{ $\rightarrow$ Awaitable{[}bool{]}}}
Read a single HTTP response.

Typical client-mode usage is to write a request using {\hyperref[\detokenize{http1connection:tornado.http1connection.HTTP1Connection.write_headers}]{\sphinxcrossref{\sphinxcode{\sphinxupquote{write\_headers}}}}},
{\hyperref[\detokenize{http1connection:tornado.http1connection.HTTP1Connection.write}]{\sphinxcrossref{\sphinxcode{\sphinxupquote{write}}}}}, and {\hyperref[\detokenize{http1connection:tornado.http1connection.HTTP1Connection.finish}]{\sphinxcrossref{\sphinxcode{\sphinxupquote{finish}}}}}, and then call \sphinxcode{\sphinxupquote{read\_response}}.
\begin{quote}\begin{description}
\item[{Parameters}] \leavevmode
\sphinxstyleliteralstrong{\sphinxupquote{delegate}} \textendash{} a {\hyperref[\detokenize{httputil:tornado.httputil.HTTPMessageDelegate}]{\sphinxcrossref{\sphinxcode{\sphinxupquote{HTTPMessageDelegate}}}}}

\end{description}\end{quote}

Returns a {\hyperref[\detokenize{concurrent:tornado.concurrent.Future}]{\sphinxcrossref{\sphinxcode{\sphinxupquote{Future}}}}} that resolves to a bool after the full response has
been read. The result is true if the stream is still open.

\end{fulllineitems}

\index{set\_close\_callback() (tornado.http1connection.HTTP1Connection method)@\spxentry{set\_close\_callback()}\spxextra{tornado.http1connection.HTTP1Connection method}}

\begin{fulllineitems}
\phantomsection\label{\detokenize{http1connection:tornado.http1connection.HTTP1Connection.set_close_callback}}\pysiglinewithargsret{\sphinxbfcode{\sphinxupquote{set\_close\_callback}}}{\emph{callback: Optional{[}Callable{[}{[}{]}, None{]}{]}}}{{ $\rightarrow$ None}}
Sets a callback that will be run when the connection is closed.

Note that this callback is slightly different from
{\hyperref[\detokenize{httputil:tornado.httputil.HTTPMessageDelegate.on_connection_close}]{\sphinxcrossref{\sphinxcode{\sphinxupquote{HTTPMessageDelegate.on\_connection\_close}}}}}: The
{\hyperref[\detokenize{httputil:tornado.httputil.HTTPMessageDelegate}]{\sphinxcrossref{\sphinxcode{\sphinxupquote{HTTPMessageDelegate}}}}} method is called when the connection is
closed while recieving a message. This callback is used when
there is not an active delegate (for example, on the server
side this callback is used if the client closes the connection
after sending its request but before receiving all the
response.

\end{fulllineitems}

\index{detach() (tornado.http1connection.HTTP1Connection method)@\spxentry{detach()}\spxextra{tornado.http1connection.HTTP1Connection method}}

\begin{fulllineitems}
\phantomsection\label{\detokenize{http1connection:tornado.http1connection.HTTP1Connection.detach}}\pysiglinewithargsret{\sphinxbfcode{\sphinxupquote{detach}}}{}{{ $\rightarrow$ tornado.iostream.IOStream}}
Take control of the underlying stream.

Returns the underlying {\hyperref[\detokenize{iostream:tornado.iostream.IOStream}]{\sphinxcrossref{\sphinxcode{\sphinxupquote{IOStream}}}}} object and stops all further
HTTP processing.  May only be called during
{\hyperref[\detokenize{httputil:tornado.httputil.HTTPMessageDelegate.headers_received}]{\sphinxcrossref{\sphinxcode{\sphinxupquote{HTTPMessageDelegate.headers\_received}}}}}.  Intended for implementing
protocols like websockets that tunnel over an HTTP handshake.

\end{fulllineitems}

\index{set\_body\_timeout() (tornado.http1connection.HTTP1Connection method)@\spxentry{set\_body\_timeout()}\spxextra{tornado.http1connection.HTTP1Connection method}}

\begin{fulllineitems}
\phantomsection\label{\detokenize{http1connection:tornado.http1connection.HTTP1Connection.set_body_timeout}}\pysiglinewithargsret{\sphinxbfcode{\sphinxupquote{set\_body\_timeout}}}{\emph{timeout: float}}{{ $\rightarrow$ None}}
Sets the body timeout for a single request.

Overrides the value from {\hyperref[\detokenize{http1connection:tornado.http1connection.HTTP1ConnectionParameters}]{\sphinxcrossref{\sphinxcode{\sphinxupquote{HTTP1ConnectionParameters}}}}}.

\end{fulllineitems}

\index{set\_max\_body\_size() (tornado.http1connection.HTTP1Connection method)@\spxentry{set\_max\_body\_size()}\spxextra{tornado.http1connection.HTTP1Connection method}}

\begin{fulllineitems}
\phantomsection\label{\detokenize{http1connection:tornado.http1connection.HTTP1Connection.set_max_body_size}}\pysiglinewithargsret{\sphinxbfcode{\sphinxupquote{set\_max\_body\_size}}}{\emph{max\_body\_size: int}}{{ $\rightarrow$ None}}
Sets the body size limit for a single request.

Overrides the value from {\hyperref[\detokenize{http1connection:tornado.http1connection.HTTP1ConnectionParameters}]{\sphinxcrossref{\sphinxcode{\sphinxupquote{HTTP1ConnectionParameters}}}}}.

\end{fulllineitems}

\index{write\_headers() (tornado.http1connection.HTTP1Connection method)@\spxentry{write\_headers()}\spxextra{tornado.http1connection.HTTP1Connection method}}

\begin{fulllineitems}
\phantomsection\label{\detokenize{http1connection:tornado.http1connection.HTTP1Connection.write_headers}}\pysiglinewithargsret{\sphinxbfcode{\sphinxupquote{write\_headers}}}{\emph{start\_line: Union{[}tornado.httputil.RequestStartLine, tornado.httputil.ResponseStartLine{]}, headers: tornado.httputil.HTTPHeaders, chunk: bytes = None}}{{ $\rightarrow$ Future{[}None{]}}}
Implements {\hyperref[\detokenize{httputil:tornado.httputil.HTTPConnection.write_headers}]{\sphinxcrossref{\sphinxcode{\sphinxupquote{HTTPConnection.write\_headers}}}}}.

\end{fulllineitems}

\index{write() (tornado.http1connection.HTTP1Connection method)@\spxentry{write()}\spxextra{tornado.http1connection.HTTP1Connection method}}

\begin{fulllineitems}
\phantomsection\label{\detokenize{http1connection:tornado.http1connection.HTTP1Connection.write}}\pysiglinewithargsret{\sphinxbfcode{\sphinxupquote{write}}}{\emph{chunk: bytes}}{{ $\rightarrow$ Future{[}None{]}}}
Implements {\hyperref[\detokenize{httputil:tornado.httputil.HTTPConnection.write}]{\sphinxcrossref{\sphinxcode{\sphinxupquote{HTTPConnection.write}}}}}.

For backwards compatibility it is allowed but deprecated to
skip {\hyperref[\detokenize{http1connection:tornado.http1connection.HTTP1Connection.write_headers}]{\sphinxcrossref{\sphinxcode{\sphinxupquote{write\_headers}}}}} and instead call {\hyperref[\detokenize{http1connection:tornado.http1connection.HTTP1Connection.write}]{\sphinxcrossref{\sphinxcode{\sphinxupquote{write()}}}}} with a
pre-encoded header block.

\end{fulllineitems}

\index{finish() (tornado.http1connection.HTTP1Connection method)@\spxentry{finish()}\spxextra{tornado.http1connection.HTTP1Connection method}}

\begin{fulllineitems}
\phantomsection\label{\detokenize{http1connection:tornado.http1connection.HTTP1Connection.finish}}\pysiglinewithargsret{\sphinxbfcode{\sphinxupquote{finish}}}{}{{ $\rightarrow$ None}}
Implements {\hyperref[\detokenize{httputil:tornado.httputil.HTTPConnection.finish}]{\sphinxcrossref{\sphinxcode{\sphinxupquote{HTTPConnection.finish}}}}}.

\end{fulllineitems}


\end{fulllineitems}

\index{HTTP1ServerConnection (class in tornado.http1connection)@\spxentry{HTTP1ServerConnection}\spxextra{class in tornado.http1connection}}

\begin{fulllineitems}
\phantomsection\label{\detokenize{http1connection:tornado.http1connection.HTTP1ServerConnection}}\pysiglinewithargsret{\sphinxbfcode{\sphinxupquote{class }}\sphinxcode{\sphinxupquote{tornado.http1connection.}}\sphinxbfcode{\sphinxupquote{HTTP1ServerConnection}}}{\emph{stream: tornado.iostream.IOStream}, \emph{params: tornado.http1connection.HTTP1ConnectionParameters = None}, \emph{context: object = None}}{}
An HTTP/1.x server.
\begin{quote}\begin{description}
\item[{Parameters}] \leavevmode\begin{itemize}
\item {} 
\sphinxstyleliteralstrong{\sphinxupquote{stream}} \textendash{} an {\hyperref[\detokenize{iostream:tornado.iostream.IOStream}]{\sphinxcrossref{\sphinxcode{\sphinxupquote{IOStream}}}}}

\item {} 
\sphinxstyleliteralstrong{\sphinxupquote{params}} \textendash{} a {\hyperref[\detokenize{http1connection:tornado.http1connection.HTTP1ConnectionParameters}]{\sphinxcrossref{\sphinxcode{\sphinxupquote{HTTP1ConnectionParameters}}}}} or None

\item {} 
\sphinxstyleliteralstrong{\sphinxupquote{context}} \textendash{} an opaque application-defined object that is accessible
as \sphinxcode{\sphinxupquote{connection.context}}

\end{itemize}

\end{description}\end{quote}
\index{close() (tornado.http1connection.HTTP1ServerConnection method)@\spxentry{close()}\spxextra{tornado.http1connection.HTTP1ServerConnection method}}

\begin{fulllineitems}
\phantomsection\label{\detokenize{http1connection:tornado.http1connection.HTTP1ServerConnection.close}}\pysiglinewithargsret{\sphinxbfcode{\sphinxupquote{coroutine }}\sphinxbfcode{\sphinxupquote{close}}}{}{{ $\rightarrow$ None}}
Closes the connection.

Returns a {\hyperref[\detokenize{concurrent:tornado.concurrent.Future}]{\sphinxcrossref{\sphinxcode{\sphinxupquote{Future}}}}} that resolves after the serving loop has exited.

\end{fulllineitems}

\index{start\_serving() (tornado.http1connection.HTTP1ServerConnection method)@\spxentry{start\_serving()}\spxextra{tornado.http1connection.HTTP1ServerConnection method}}

\begin{fulllineitems}
\phantomsection\label{\detokenize{http1connection:tornado.http1connection.HTTP1ServerConnection.start_serving}}\pysiglinewithargsret{\sphinxbfcode{\sphinxupquote{start\_serving}}}{\emph{delegate: tornado.httputil.HTTPServerConnectionDelegate}}{{ $\rightarrow$ None}}
Starts serving requests on this connection.
\begin{quote}\begin{description}
\item[{Parameters}] \leavevmode
\sphinxstyleliteralstrong{\sphinxupquote{delegate}} \textendash{} a {\hyperref[\detokenize{httputil:tornado.httputil.HTTPServerConnectionDelegate}]{\sphinxcrossref{\sphinxcode{\sphinxupquote{HTTPServerConnectionDelegate}}}}}

\end{description}\end{quote}

\end{fulllineitems}


\end{fulllineitems}



\section{Asynchronous networking}
\label{\detokenize{networking:asynchronous-networking}}\label{\detokenize{networking::doc}}

\subsection{\sphinxstyleliteralintitle{\sphinxupquote{tornado.ioloop}} — Main event loop}
\label{\detokenize{ioloop:module-tornado.ioloop}}\label{\detokenize{ioloop:tornado-ioloop-main-event-loop}}\label{\detokenize{ioloop::doc}}\index{tornado.ioloop (module)@\spxentry{tornado.ioloop}\spxextra{module}}
An I/O event loop for non-blocking sockets.

In Tornado 6.0, {\hyperref[\detokenize{ioloop:tornado.ioloop.IOLoop}]{\sphinxcrossref{\sphinxcode{\sphinxupquote{IOLoop}}}}} is a wrapper around the \sphinxhref{https://docs.python.org/3.6/library/asyncio.html\#module-asyncio}{\sphinxcode{\sphinxupquote{asyncio}}} event
loop, with a slightly different interface for historical reasons.
Applications can use either the {\hyperref[\detokenize{ioloop:tornado.ioloop.IOLoop}]{\sphinxcrossref{\sphinxcode{\sphinxupquote{IOLoop}}}}} interface or the underlying
\sphinxhref{https://docs.python.org/3.6/library/asyncio.html\#module-asyncio}{\sphinxcode{\sphinxupquote{asyncio}}} event loop directly (unless compatibility with older
versions of Tornado is desired, in which case {\hyperref[\detokenize{ioloop:tornado.ioloop.IOLoop}]{\sphinxcrossref{\sphinxcode{\sphinxupquote{IOLoop}}}}} must be used).

Typical applications will use a single {\hyperref[\detokenize{ioloop:tornado.ioloop.IOLoop}]{\sphinxcrossref{\sphinxcode{\sphinxupquote{IOLoop}}}}} object, accessed via
{\hyperref[\detokenize{ioloop:tornado.ioloop.IOLoop.current}]{\sphinxcrossref{\sphinxcode{\sphinxupquote{IOLoop.current}}}}} class method. The {\hyperref[\detokenize{ioloop:tornado.ioloop.IOLoop.start}]{\sphinxcrossref{\sphinxcode{\sphinxupquote{IOLoop.start}}}}} method (or
equivalently, \sphinxhref{https://docs.python.org/3.6/library/asyncio-eventloop.html\#asyncio.AbstractEventLoop.run\_forever}{\sphinxcode{\sphinxupquote{asyncio.AbstractEventLoop.run\_forever}}}) should usually
be called at the end of the \sphinxcode{\sphinxupquote{main()}} function. Atypical applications
may use more than one {\hyperref[\detokenize{ioloop:tornado.ioloop.IOLoop}]{\sphinxcrossref{\sphinxcode{\sphinxupquote{IOLoop}}}}}, such as one {\hyperref[\detokenize{ioloop:tornado.ioloop.IOLoop}]{\sphinxcrossref{\sphinxcode{\sphinxupquote{IOLoop}}}}} per thread, or
per \sphinxhref{https://docs.python.org/3.6/library/unittest.html\#module-unittest}{\sphinxcode{\sphinxupquote{unittest}}} case.


\subsubsection{IOLoop objects}
\label{\detokenize{ioloop:ioloop-objects}}\index{IOLoop (class in tornado.ioloop)@\spxentry{IOLoop}\spxextra{class in tornado.ioloop}}

\begin{fulllineitems}
\phantomsection\label{\detokenize{ioloop:tornado.ioloop.IOLoop}}\pysigline{\sphinxbfcode{\sphinxupquote{class }}\sphinxcode{\sphinxupquote{tornado.ioloop.}}\sphinxbfcode{\sphinxupquote{IOLoop}}}
An I/O event loop.

As of Tornado 6.0, {\hyperref[\detokenize{ioloop:tornado.ioloop.IOLoop}]{\sphinxcrossref{\sphinxcode{\sphinxupquote{IOLoop}}}}} is a wrapper around the \sphinxhref{https://docs.python.org/3.6/library/asyncio.html\#module-asyncio}{\sphinxcode{\sphinxupquote{asyncio}}} event
loop.

Example usage for a simple TCP server:

\begin{sphinxVerbatim}[commandchars=\\\{\}]
\PYG{k+kn}{import} \PYG{n+nn}{errno}
\PYG{k+kn}{import} \PYG{n+nn}{functools}
\PYG{k+kn}{import} \PYG{n+nn}{socket}

\PYG{k+kn}{import} \PYG{n+nn}{tornado}\PYG{n+nn}{.}\PYG{n+nn}{ioloop}
\PYG{k+kn}{from} \PYG{n+nn}{tornado}\PYG{n+nn}{.}\PYG{n+nn}{iostream} \PYG{k}{import} \PYG{n}{IOStream}

\PYG{k}{async} \PYG{k}{def} \PYG{n+nf}{handle\PYGZus{}connection}\PYG{p}{(}\PYG{n}{connection}\PYG{p}{,} \PYG{n}{address}\PYG{p}{)}\PYG{p}{:}
    \PYG{n}{stream} \PYG{o}{=} \PYG{n}{IOStream}\PYG{p}{(}\PYG{n}{connection}\PYG{p}{)}
    \PYG{n}{message} \PYG{o}{=} \PYG{k}{await} \PYG{n}{stream}\PYG{o}{.}\PYG{n}{read\PYGZus{}until\PYGZus{}close}\PYG{p}{(}\PYG{p}{)}
    \PYG{n+nb}{print}\PYG{p}{(}\PYG{l+s+s2}{\PYGZdq{}}\PYG{l+s+s2}{message from client:}\PYG{l+s+s2}{\PYGZdq{}}\PYG{p}{,} \PYG{n}{message}\PYG{o}{.}\PYG{n}{decode}\PYG{p}{(}\PYG{p}{)}\PYG{o}{.}\PYG{n}{strip}\PYG{p}{(}\PYG{p}{)}\PYG{p}{)}

\PYG{k}{def} \PYG{n+nf}{connection\PYGZus{}ready}\PYG{p}{(}\PYG{n}{sock}\PYG{p}{,} \PYG{n}{fd}\PYG{p}{,} \PYG{n}{events}\PYG{p}{)}\PYG{p}{:}
    \PYG{k}{while} \PYG{k+kc}{True}\PYG{p}{:}
        \PYG{k}{try}\PYG{p}{:}
            \PYG{n}{connection}\PYG{p}{,} \PYG{n}{address} \PYG{o}{=} \PYG{n}{sock}\PYG{o}{.}\PYG{n}{accept}\PYG{p}{(}\PYG{p}{)}
        \PYG{k}{except} \PYG{n}{socket}\PYG{o}{.}\PYG{n}{error} \PYG{k}{as} \PYG{n}{e}\PYG{p}{:}
            \PYG{k}{if} \PYG{n}{e}\PYG{o}{.}\PYG{n}{args}\PYG{p}{[}\PYG{l+m+mi}{0}\PYG{p}{]} \PYG{o+ow}{not} \PYG{o+ow}{in} \PYG{p}{(}\PYG{n}{errno}\PYG{o}{.}\PYG{n}{EWOULDBLOCK}\PYG{p}{,} \PYG{n}{errno}\PYG{o}{.}\PYG{n}{EAGAIN}\PYG{p}{)}\PYG{p}{:}
                \PYG{k}{raise}
            \PYG{k}{return}
        \PYG{n}{connection}\PYG{o}{.}\PYG{n}{setblocking}\PYG{p}{(}\PYG{l+m+mi}{0}\PYG{p}{)}
        \PYG{n}{io\PYGZus{}loop} \PYG{o}{=} \PYG{n}{tornado}\PYG{o}{.}\PYG{n}{ioloop}\PYG{o}{.}\PYG{n}{IOLoop}\PYG{o}{.}\PYG{n}{current}\PYG{p}{(}\PYG{p}{)}
        \PYG{n}{io\PYGZus{}loop}\PYG{o}{.}\PYG{n}{spawn\PYGZus{}callback}\PYG{p}{(}\PYG{n}{handle\PYGZus{}connection}\PYG{p}{,} \PYG{n}{connection}\PYG{p}{,} \PYG{n}{address}\PYG{p}{)}

\PYG{k}{if} \PYG{n+nv+vm}{\PYGZus{}\PYGZus{}name\PYGZus{}\PYGZus{}} \PYG{o}{==} \PYG{l+s+s1}{\PYGZsq{}}\PYG{l+s+s1}{\PYGZus{}\PYGZus{}main\PYGZus{}\PYGZus{}}\PYG{l+s+s1}{\PYGZsq{}}\PYG{p}{:}
    \PYG{n}{sock} \PYG{o}{=} \PYG{n}{socket}\PYG{o}{.}\PYG{n}{socket}\PYG{p}{(}\PYG{n}{socket}\PYG{o}{.}\PYG{n}{AF\PYGZus{}INET}\PYG{p}{,} \PYG{n}{socket}\PYG{o}{.}\PYG{n}{SOCK\PYGZus{}STREAM}\PYG{p}{,} \PYG{l+m+mi}{0}\PYG{p}{)}
    \PYG{n}{sock}\PYG{o}{.}\PYG{n}{setsockopt}\PYG{p}{(}\PYG{n}{socket}\PYG{o}{.}\PYG{n}{SOL\PYGZus{}SOCKET}\PYG{p}{,} \PYG{n}{socket}\PYG{o}{.}\PYG{n}{SO\PYGZus{}REUSEADDR}\PYG{p}{,} \PYG{l+m+mi}{1}\PYG{p}{)}
    \PYG{n}{sock}\PYG{o}{.}\PYG{n}{setblocking}\PYG{p}{(}\PYG{l+m+mi}{0}\PYG{p}{)}
    \PYG{n}{sock}\PYG{o}{.}\PYG{n}{bind}\PYG{p}{(}\PYG{p}{(}\PYG{l+s+s2}{\PYGZdq{}}\PYG{l+s+s2}{\PYGZdq{}}\PYG{p}{,} \PYG{l+m+mi}{8888}\PYG{p}{)}\PYG{p}{)}
    \PYG{n}{sock}\PYG{o}{.}\PYG{n}{listen}\PYG{p}{(}\PYG{l+m+mi}{128}\PYG{p}{)}

    \PYG{n}{io\PYGZus{}loop} \PYG{o}{=} \PYG{n}{tornado}\PYG{o}{.}\PYG{n}{ioloop}\PYG{o}{.}\PYG{n}{IOLoop}\PYG{o}{.}\PYG{n}{current}\PYG{p}{(}\PYG{p}{)}
    \PYG{n}{callback} \PYG{o}{=} \PYG{n}{functools}\PYG{o}{.}\PYG{n}{partial}\PYG{p}{(}\PYG{n}{connection\PYGZus{}ready}\PYG{p}{,} \PYG{n}{sock}\PYG{p}{)}
    \PYG{n}{io\PYGZus{}loop}\PYG{o}{.}\PYG{n}{add\PYGZus{}handler}\PYG{p}{(}\PYG{n}{sock}\PYG{o}{.}\PYG{n}{fileno}\PYG{p}{(}\PYG{p}{)}\PYG{p}{,} \PYG{n}{callback}\PYG{p}{,} \PYG{n}{io\PYGZus{}loop}\PYG{o}{.}\PYG{n}{READ}\PYG{p}{)}
    \PYG{n}{io\PYGZus{}loop}\PYG{o}{.}\PYG{n}{start}\PYG{p}{(}\PYG{p}{)}
\end{sphinxVerbatim}

By default, a newly-constructed {\hyperref[\detokenize{ioloop:tornado.ioloop.IOLoop}]{\sphinxcrossref{\sphinxcode{\sphinxupquote{IOLoop}}}}} becomes the thread’s current
{\hyperref[\detokenize{ioloop:tornado.ioloop.IOLoop}]{\sphinxcrossref{\sphinxcode{\sphinxupquote{IOLoop}}}}}, unless there already is a current {\hyperref[\detokenize{ioloop:tornado.ioloop.IOLoop}]{\sphinxcrossref{\sphinxcode{\sphinxupquote{IOLoop}}}}}. This behavior
can be controlled with the \sphinxcode{\sphinxupquote{make\_current}} argument to the {\hyperref[\detokenize{ioloop:tornado.ioloop.IOLoop}]{\sphinxcrossref{\sphinxcode{\sphinxupquote{IOLoop}}}}}
constructor: if \sphinxcode{\sphinxupquote{make\_current=True}}, the new {\hyperref[\detokenize{ioloop:tornado.ioloop.IOLoop}]{\sphinxcrossref{\sphinxcode{\sphinxupquote{IOLoop}}}}} will always
try to become current and it raises an error if there is already a
current instance. If \sphinxcode{\sphinxupquote{make\_current=False}}, the new {\hyperref[\detokenize{ioloop:tornado.ioloop.IOLoop}]{\sphinxcrossref{\sphinxcode{\sphinxupquote{IOLoop}}}}} will
not try to become current.

In general, an {\hyperref[\detokenize{ioloop:tornado.ioloop.IOLoop}]{\sphinxcrossref{\sphinxcode{\sphinxupquote{IOLoop}}}}} cannot survive a fork or be shared across
processes in any way. When multiple processes are being used, each
process should create its own {\hyperref[\detokenize{ioloop:tornado.ioloop.IOLoop}]{\sphinxcrossref{\sphinxcode{\sphinxupquote{IOLoop}}}}}, which also implies that
any objects which depend on the {\hyperref[\detokenize{ioloop:tornado.ioloop.IOLoop}]{\sphinxcrossref{\sphinxcode{\sphinxupquote{IOLoop}}}}} (such as
{\hyperref[\detokenize{httpclient:tornado.httpclient.AsyncHTTPClient}]{\sphinxcrossref{\sphinxcode{\sphinxupquote{AsyncHTTPClient}}}}}) must also be created in the child processes.
As a guideline, anything that starts processes (including the
{\hyperref[\detokenize{process:module-tornado.process}]{\sphinxcrossref{\sphinxcode{\sphinxupquote{tornado.process}}}}} and \sphinxhref{https://docs.python.org/3.6/library/multiprocessing.html\#module-multiprocessing}{\sphinxcode{\sphinxupquote{multiprocessing}}} modules) should do so as
early as possible, ideally the first thing the application does
after loading its configuration in \sphinxcode{\sphinxupquote{main()}}.

\DUrole{versionmodified,changed}{Changed in version 4.2: }Added the \sphinxcode{\sphinxupquote{make\_current}} keyword argument to the {\hyperref[\detokenize{ioloop:tornado.ioloop.IOLoop}]{\sphinxcrossref{\sphinxcode{\sphinxupquote{IOLoop}}}}}
constructor.

\DUrole{versionmodified,changed}{Changed in version 5.0: }Uses the \sphinxhref{https://docs.python.org/3.6/library/asyncio.html\#module-asyncio}{\sphinxcode{\sphinxupquote{asyncio}}} event loop by default. The
\sphinxcode{\sphinxupquote{IOLoop.configure}} method cannot be used on Python 3 except
to redundantly specify the \sphinxhref{https://docs.python.org/3.6/library/asyncio.html\#module-asyncio}{\sphinxcode{\sphinxupquote{asyncio}}} event loop.

\end{fulllineitems}



\paragraph{Running an IOLoop}
\label{\detokenize{ioloop:running-an-ioloop}}\index{current() (tornado.ioloop.IOLoop static method)@\spxentry{current()}\spxextra{tornado.ioloop.IOLoop static method}}

\begin{fulllineitems}
\phantomsection\label{\detokenize{ioloop:tornado.ioloop.IOLoop.current}}\pysiglinewithargsret{\sphinxbfcode{\sphinxupquote{static }}\sphinxcode{\sphinxupquote{IOLoop.}}\sphinxbfcode{\sphinxupquote{current}}}{\emph{instance: bool = True}}{{ $\rightarrow$ Optional{[}tornado.ioloop.IOLoop{]}}}
Returns the current thread’s {\hyperref[\detokenize{ioloop:tornado.ioloop.IOLoop}]{\sphinxcrossref{\sphinxcode{\sphinxupquote{IOLoop}}}}}.

If an {\hyperref[\detokenize{ioloop:tornado.ioloop.IOLoop}]{\sphinxcrossref{\sphinxcode{\sphinxupquote{IOLoop}}}}} is currently running or has been marked as
current by {\hyperref[\detokenize{ioloop:tornado.ioloop.IOLoop.make_current}]{\sphinxcrossref{\sphinxcode{\sphinxupquote{make\_current}}}}}, returns that instance.  If there is
no current {\hyperref[\detokenize{ioloop:tornado.ioloop.IOLoop}]{\sphinxcrossref{\sphinxcode{\sphinxupquote{IOLoop}}}}} and \sphinxcode{\sphinxupquote{instance}} is true, creates one.

\DUrole{versionmodified,changed}{Changed in version 4.1: }Added \sphinxcode{\sphinxupquote{instance}} argument to control the fallback to
{\hyperref[\detokenize{ioloop:tornado.ioloop.IOLoop.instance}]{\sphinxcrossref{\sphinxcode{\sphinxupquote{IOLoop.instance()}}}}}.

\DUrole{versionmodified,changed}{Changed in version 5.0: }On Python 3, control of the current {\hyperref[\detokenize{ioloop:tornado.ioloop.IOLoop}]{\sphinxcrossref{\sphinxcode{\sphinxupquote{IOLoop}}}}} is delegated
to \sphinxhref{https://docs.python.org/3.6/library/asyncio.html\#module-asyncio}{\sphinxcode{\sphinxupquote{asyncio}}}, with this and other methods as pass-through accessors.
The \sphinxcode{\sphinxupquote{instance}} argument now controls whether an {\hyperref[\detokenize{ioloop:tornado.ioloop.IOLoop}]{\sphinxcrossref{\sphinxcode{\sphinxupquote{IOLoop}}}}}
is created automatically when there is none, instead of
whether we fall back to {\hyperref[\detokenize{ioloop:tornado.ioloop.IOLoop.instance}]{\sphinxcrossref{\sphinxcode{\sphinxupquote{IOLoop.instance()}}}}} (which is now
an alias for this method). \sphinxcode{\sphinxupquote{instance=False}} is deprecated,
since even if we do not create an {\hyperref[\detokenize{ioloop:tornado.ioloop.IOLoop}]{\sphinxcrossref{\sphinxcode{\sphinxupquote{IOLoop}}}}}, this method
may initialize the asyncio loop.

\end{fulllineitems}

\index{make\_current() (tornado.ioloop.IOLoop method)@\spxentry{make\_current()}\spxextra{tornado.ioloop.IOLoop method}}

\begin{fulllineitems}
\phantomsection\label{\detokenize{ioloop:tornado.ioloop.IOLoop.make_current}}\pysiglinewithargsret{\sphinxcode{\sphinxupquote{IOLoop.}}\sphinxbfcode{\sphinxupquote{make\_current}}}{}{{ $\rightarrow$ None}}
Makes this the {\hyperref[\detokenize{ioloop:tornado.ioloop.IOLoop}]{\sphinxcrossref{\sphinxcode{\sphinxupquote{IOLoop}}}}} for the current thread.

An {\hyperref[\detokenize{ioloop:tornado.ioloop.IOLoop}]{\sphinxcrossref{\sphinxcode{\sphinxupquote{IOLoop}}}}} automatically becomes current for its thread
when it is started, but it is sometimes useful to call
{\hyperref[\detokenize{ioloop:tornado.ioloop.IOLoop.make_current}]{\sphinxcrossref{\sphinxcode{\sphinxupquote{make\_current}}}}} explicitly before starting the {\hyperref[\detokenize{ioloop:tornado.ioloop.IOLoop}]{\sphinxcrossref{\sphinxcode{\sphinxupquote{IOLoop}}}}},
so that code run at startup time can find the right
instance.

\DUrole{versionmodified,changed}{Changed in version 4.1: }An {\hyperref[\detokenize{ioloop:tornado.ioloop.IOLoop}]{\sphinxcrossref{\sphinxcode{\sphinxupquote{IOLoop}}}}} created while there is no current {\hyperref[\detokenize{ioloop:tornado.ioloop.IOLoop}]{\sphinxcrossref{\sphinxcode{\sphinxupquote{IOLoop}}}}}
will automatically become current.

\DUrole{versionmodified,changed}{Changed in version 5.0: }This method also sets the current \sphinxhref{https://docs.python.org/3.6/library/asyncio.html\#module-asyncio}{\sphinxcode{\sphinxupquote{asyncio}}} event loop.

\end{fulllineitems}

\index{clear\_current() (tornado.ioloop.IOLoop static method)@\spxentry{clear\_current()}\spxextra{tornado.ioloop.IOLoop static method}}

\begin{fulllineitems}
\phantomsection\label{\detokenize{ioloop:tornado.ioloop.IOLoop.clear_current}}\pysiglinewithargsret{\sphinxbfcode{\sphinxupquote{static }}\sphinxcode{\sphinxupquote{IOLoop.}}\sphinxbfcode{\sphinxupquote{clear\_current}}}{}{{ $\rightarrow$ None}}
Clears the {\hyperref[\detokenize{ioloop:tornado.ioloop.IOLoop}]{\sphinxcrossref{\sphinxcode{\sphinxupquote{IOLoop}}}}} for the current thread.

Intended primarily for use by test frameworks in between tests.

\DUrole{versionmodified,changed}{Changed in version 5.0: }This method also clears the current \sphinxhref{https://docs.python.org/3.6/library/asyncio.html\#module-asyncio}{\sphinxcode{\sphinxupquote{asyncio}}} event loop.

\end{fulllineitems}

\index{start() (tornado.ioloop.IOLoop method)@\spxentry{start()}\spxextra{tornado.ioloop.IOLoop method}}

\begin{fulllineitems}
\phantomsection\label{\detokenize{ioloop:tornado.ioloop.IOLoop.start}}\pysiglinewithargsret{\sphinxcode{\sphinxupquote{IOLoop.}}\sphinxbfcode{\sphinxupquote{start}}}{}{{ $\rightarrow$ None}}
Starts the I/O loop.

The loop will run until one of the callbacks calls {\hyperref[\detokenize{ioloop:tornado.ioloop.IOLoop.stop}]{\sphinxcrossref{\sphinxcode{\sphinxupquote{stop()}}}}}, which
will make the loop stop after the current event iteration completes.

\end{fulllineitems}

\index{stop() (tornado.ioloop.IOLoop method)@\spxentry{stop()}\spxextra{tornado.ioloop.IOLoop method}}

\begin{fulllineitems}
\phantomsection\label{\detokenize{ioloop:tornado.ioloop.IOLoop.stop}}\pysiglinewithargsret{\sphinxcode{\sphinxupquote{IOLoop.}}\sphinxbfcode{\sphinxupquote{stop}}}{}{{ $\rightarrow$ None}}
Stop the I/O loop.

If the event loop is not currently running, the next call to {\hyperref[\detokenize{ioloop:tornado.ioloop.IOLoop.start}]{\sphinxcrossref{\sphinxcode{\sphinxupquote{start()}}}}}
will return immediately.

Note that even after {\hyperref[\detokenize{ioloop:tornado.ioloop.IOLoop.stop}]{\sphinxcrossref{\sphinxcode{\sphinxupquote{stop}}}}} has been called, the {\hyperref[\detokenize{ioloop:tornado.ioloop.IOLoop}]{\sphinxcrossref{\sphinxcode{\sphinxupquote{IOLoop}}}}} is not
completely stopped until {\hyperref[\detokenize{ioloop:tornado.ioloop.IOLoop.start}]{\sphinxcrossref{\sphinxcode{\sphinxupquote{IOLoop.start}}}}} has also returned.
Some work that was scheduled before the call to {\hyperref[\detokenize{ioloop:tornado.ioloop.IOLoop.stop}]{\sphinxcrossref{\sphinxcode{\sphinxupquote{stop}}}}} may still
be run before the {\hyperref[\detokenize{ioloop:tornado.ioloop.IOLoop}]{\sphinxcrossref{\sphinxcode{\sphinxupquote{IOLoop}}}}} shuts down.

\end{fulllineitems}

\index{run\_sync() (tornado.ioloop.IOLoop method)@\spxentry{run\_sync()}\spxextra{tornado.ioloop.IOLoop method}}

\begin{fulllineitems}
\phantomsection\label{\detokenize{ioloop:tornado.ioloop.IOLoop.run_sync}}\pysiglinewithargsret{\sphinxcode{\sphinxupquote{IOLoop.}}\sphinxbfcode{\sphinxupquote{run\_sync}}}{\emph{func: Callable}, \emph{timeout: float = None}}{{ $\rightarrow$ Any}}
Starts the {\hyperref[\detokenize{ioloop:tornado.ioloop.IOLoop}]{\sphinxcrossref{\sphinxcode{\sphinxupquote{IOLoop}}}}}, runs the given function, and stops the loop.

The function must return either an awaitable object or
\sphinxcode{\sphinxupquote{None}}. If the function returns an awaitable object, the
{\hyperref[\detokenize{ioloop:tornado.ioloop.IOLoop}]{\sphinxcrossref{\sphinxcode{\sphinxupquote{IOLoop}}}}} will run until the awaitable is resolved (and
{\hyperref[\detokenize{ioloop:tornado.ioloop.IOLoop.run_sync}]{\sphinxcrossref{\sphinxcode{\sphinxupquote{run\_sync()}}}}} will return the awaitable’s result). If it raises
an exception, the {\hyperref[\detokenize{ioloop:tornado.ioloop.IOLoop}]{\sphinxcrossref{\sphinxcode{\sphinxupquote{IOLoop}}}}} will stop and the exception will be
re-raised to the caller.

The keyword-only argument \sphinxcode{\sphinxupquote{timeout}} may be used to set
a maximum duration for the function.  If the timeout expires,
a {\hyperref[\detokenize{util:tornado.util.TimeoutError}]{\sphinxcrossref{\sphinxcode{\sphinxupquote{tornado.util.TimeoutError}}}}} is raised.

This method is useful to allow asynchronous calls in a
\sphinxcode{\sphinxupquote{main()}} function:

\begin{sphinxVerbatim}[commandchars=\\\{\}]
\PYG{k}{async} \PYG{k}{def} \PYG{n+nf}{main}\PYG{p}{(}\PYG{p}{)}\PYG{p}{:}
    \PYG{c+c1}{\PYGZsh{} do stuff...}

\PYG{k}{if} \PYG{n+nv+vm}{\PYGZus{}\PYGZus{}name\PYGZus{}\PYGZus{}} \PYG{o}{==} \PYG{l+s+s1}{\PYGZsq{}}\PYG{l+s+s1}{\PYGZus{}\PYGZus{}main\PYGZus{}\PYGZus{}}\PYG{l+s+s1}{\PYGZsq{}}\PYG{p}{:}
    \PYG{n}{IOLoop}\PYG{o}{.}\PYG{n}{current}\PYG{p}{(}\PYG{p}{)}\PYG{o}{.}\PYG{n}{run\PYGZus{}sync}\PYG{p}{(}\PYG{n}{main}\PYG{p}{)}
\end{sphinxVerbatim}

\DUrole{versionmodified,changed}{Changed in version 4.3: }Returning a non-\sphinxcode{\sphinxupquote{None}}, non-awaitable value is now an error.

\DUrole{versionmodified,changed}{Changed in version 5.0: }If a timeout occurs, the \sphinxcode{\sphinxupquote{func}} coroutine will be cancelled.

\end{fulllineitems}

\index{close() (tornado.ioloop.IOLoop method)@\spxentry{close()}\spxextra{tornado.ioloop.IOLoop method}}

\begin{fulllineitems}
\phantomsection\label{\detokenize{ioloop:tornado.ioloop.IOLoop.close}}\pysiglinewithargsret{\sphinxcode{\sphinxupquote{IOLoop.}}\sphinxbfcode{\sphinxupquote{close}}}{\emph{all\_fds: bool = False}}{{ $\rightarrow$ None}}
Closes the {\hyperref[\detokenize{ioloop:tornado.ioloop.IOLoop}]{\sphinxcrossref{\sphinxcode{\sphinxupquote{IOLoop}}}}}, freeing any resources used.

If \sphinxcode{\sphinxupquote{all\_fds}} is true, all file descriptors registered on the
IOLoop will be closed (not just the ones created by the
{\hyperref[\detokenize{ioloop:tornado.ioloop.IOLoop}]{\sphinxcrossref{\sphinxcode{\sphinxupquote{IOLoop}}}}} itself).

Many applications will only use a single {\hyperref[\detokenize{ioloop:tornado.ioloop.IOLoop}]{\sphinxcrossref{\sphinxcode{\sphinxupquote{IOLoop}}}}} that runs for the
entire lifetime of the process.  In that case closing the {\hyperref[\detokenize{ioloop:tornado.ioloop.IOLoop}]{\sphinxcrossref{\sphinxcode{\sphinxupquote{IOLoop}}}}}
is not necessary since everything will be cleaned up when the
process exits.  {\hyperref[\detokenize{ioloop:tornado.ioloop.IOLoop.close}]{\sphinxcrossref{\sphinxcode{\sphinxupquote{IOLoop.close}}}}} is provided mainly for scenarios
such as unit tests, which create and destroy a large number of
\sphinxcode{\sphinxupquote{IOLoops}}.

An {\hyperref[\detokenize{ioloop:tornado.ioloop.IOLoop}]{\sphinxcrossref{\sphinxcode{\sphinxupquote{IOLoop}}}}} must be completely stopped before it can be closed.  This
means that {\hyperref[\detokenize{ioloop:tornado.ioloop.IOLoop.stop}]{\sphinxcrossref{\sphinxcode{\sphinxupquote{IOLoop.stop()}}}}} must be called \sphinxstyleemphasis{and} {\hyperref[\detokenize{ioloop:tornado.ioloop.IOLoop.start}]{\sphinxcrossref{\sphinxcode{\sphinxupquote{IOLoop.start()}}}}} must
be allowed to return before attempting to call {\hyperref[\detokenize{ioloop:tornado.ioloop.IOLoop.close}]{\sphinxcrossref{\sphinxcode{\sphinxupquote{IOLoop.close()}}}}}.
Therefore the call to {\hyperref[\detokenize{ioloop:tornado.ioloop.IOLoop.close}]{\sphinxcrossref{\sphinxcode{\sphinxupquote{close}}}}} will usually appear just after
the call to {\hyperref[\detokenize{ioloop:tornado.ioloop.IOLoop.start}]{\sphinxcrossref{\sphinxcode{\sphinxupquote{start}}}}} rather than near the call to {\hyperref[\detokenize{ioloop:tornado.ioloop.IOLoop.stop}]{\sphinxcrossref{\sphinxcode{\sphinxupquote{stop}}}}}.

\DUrole{versionmodified,changed}{Changed in version 3.1: }If the {\hyperref[\detokenize{ioloop:tornado.ioloop.IOLoop}]{\sphinxcrossref{\sphinxcode{\sphinxupquote{IOLoop}}}}} implementation supports non-integer objects
for “file descriptors”, those objects will have their
\sphinxcode{\sphinxupquote{close}} method when \sphinxcode{\sphinxupquote{all\_fds}} is true.

\end{fulllineitems}

\index{instance() (tornado.ioloop.IOLoop static method)@\spxentry{instance()}\spxextra{tornado.ioloop.IOLoop static method}}

\begin{fulllineitems}
\phantomsection\label{\detokenize{ioloop:tornado.ioloop.IOLoop.instance}}\pysiglinewithargsret{\sphinxbfcode{\sphinxupquote{static }}\sphinxcode{\sphinxupquote{IOLoop.}}\sphinxbfcode{\sphinxupquote{instance}}}{}{{ $\rightarrow$ tornado.ioloop.IOLoop}}
Deprecated alias for {\hyperref[\detokenize{ioloop:tornado.ioloop.IOLoop.current}]{\sphinxcrossref{\sphinxcode{\sphinxupquote{IOLoop.current()}}}}}.

\DUrole{versionmodified,changed}{Changed in version 5.0: }Previously, this method returned a global singleton
{\hyperref[\detokenize{ioloop:tornado.ioloop.IOLoop}]{\sphinxcrossref{\sphinxcode{\sphinxupquote{IOLoop}}}}}, in contrast with the per-thread {\hyperref[\detokenize{ioloop:tornado.ioloop.IOLoop}]{\sphinxcrossref{\sphinxcode{\sphinxupquote{IOLoop}}}}} returned
by {\hyperref[\detokenize{ioloop:tornado.ioloop.IOLoop.current}]{\sphinxcrossref{\sphinxcode{\sphinxupquote{current()}}}}}. In nearly all cases the two were the same
(when they differed, it was generally used from non-Tornado
threads to communicate back to the main thread’s {\hyperref[\detokenize{ioloop:tornado.ioloop.IOLoop}]{\sphinxcrossref{\sphinxcode{\sphinxupquote{IOLoop}}}}}).
This distinction is not present in \sphinxhref{https://docs.python.org/3.6/library/asyncio.html\#module-asyncio}{\sphinxcode{\sphinxupquote{asyncio}}}, so in order
to facilitate integration with that package {\hyperref[\detokenize{ioloop:tornado.ioloop.IOLoop.instance}]{\sphinxcrossref{\sphinxcode{\sphinxupquote{instance()}}}}}
was changed to be an alias to {\hyperref[\detokenize{ioloop:tornado.ioloop.IOLoop.current}]{\sphinxcrossref{\sphinxcode{\sphinxupquote{current()}}}}}. Applications
using the cross-thread communications aspect of
{\hyperref[\detokenize{ioloop:tornado.ioloop.IOLoop.instance}]{\sphinxcrossref{\sphinxcode{\sphinxupquote{instance()}}}}} should instead set their own global variable
to point to the {\hyperref[\detokenize{ioloop:tornado.ioloop.IOLoop}]{\sphinxcrossref{\sphinxcode{\sphinxupquote{IOLoop}}}}} they want to use.

\DUrole{versionmodified,deprecated}{Deprecated since version 5.0.}

\end{fulllineitems}

\index{install() (tornado.ioloop.IOLoop method)@\spxentry{install()}\spxextra{tornado.ioloop.IOLoop method}}

\begin{fulllineitems}
\phantomsection\label{\detokenize{ioloop:tornado.ioloop.IOLoop.install}}\pysiglinewithargsret{\sphinxcode{\sphinxupquote{IOLoop.}}\sphinxbfcode{\sphinxupquote{install}}}{}{{ $\rightarrow$ None}}
Deprecated alias for {\hyperref[\detokenize{ioloop:tornado.ioloop.IOLoop.make_current}]{\sphinxcrossref{\sphinxcode{\sphinxupquote{make\_current()}}}}}.

\DUrole{versionmodified,changed}{Changed in version 5.0: }Previously, this method would set this {\hyperref[\detokenize{ioloop:tornado.ioloop.IOLoop}]{\sphinxcrossref{\sphinxcode{\sphinxupquote{IOLoop}}}}} as the
global singleton used by {\hyperref[\detokenize{ioloop:tornado.ioloop.IOLoop.instance}]{\sphinxcrossref{\sphinxcode{\sphinxupquote{IOLoop.instance()}}}}}. Now that
{\hyperref[\detokenize{ioloop:tornado.ioloop.IOLoop.instance}]{\sphinxcrossref{\sphinxcode{\sphinxupquote{instance()}}}}} is an alias for {\hyperref[\detokenize{ioloop:tornado.ioloop.IOLoop.current}]{\sphinxcrossref{\sphinxcode{\sphinxupquote{current()}}}}}, {\hyperref[\detokenize{ioloop:tornado.ioloop.IOLoop.install}]{\sphinxcrossref{\sphinxcode{\sphinxupquote{install()}}}}}
is an alias for {\hyperref[\detokenize{ioloop:tornado.ioloop.IOLoop.make_current}]{\sphinxcrossref{\sphinxcode{\sphinxupquote{make\_current()}}}}}.

\DUrole{versionmodified,deprecated}{Deprecated since version 5.0.}

\end{fulllineitems}

\index{clear\_instance() (tornado.ioloop.IOLoop static method)@\spxentry{clear\_instance()}\spxextra{tornado.ioloop.IOLoop static method}}

\begin{fulllineitems}
\phantomsection\label{\detokenize{ioloop:tornado.ioloop.IOLoop.clear_instance}}\pysiglinewithargsret{\sphinxbfcode{\sphinxupquote{static }}\sphinxcode{\sphinxupquote{IOLoop.}}\sphinxbfcode{\sphinxupquote{clear\_instance}}}{}{{ $\rightarrow$ None}}
Deprecated alias for {\hyperref[\detokenize{ioloop:tornado.ioloop.IOLoop.clear_current}]{\sphinxcrossref{\sphinxcode{\sphinxupquote{clear\_current()}}}}}.

\DUrole{versionmodified,changed}{Changed in version 5.0: }Previously, this method would clear the {\hyperref[\detokenize{ioloop:tornado.ioloop.IOLoop}]{\sphinxcrossref{\sphinxcode{\sphinxupquote{IOLoop}}}}} used as
the global singleton by {\hyperref[\detokenize{ioloop:tornado.ioloop.IOLoop.instance}]{\sphinxcrossref{\sphinxcode{\sphinxupquote{IOLoop.instance()}}}}}. Now that
{\hyperref[\detokenize{ioloop:tornado.ioloop.IOLoop.instance}]{\sphinxcrossref{\sphinxcode{\sphinxupquote{instance()}}}}} is an alias for {\hyperref[\detokenize{ioloop:tornado.ioloop.IOLoop.current}]{\sphinxcrossref{\sphinxcode{\sphinxupquote{current()}}}}},
{\hyperref[\detokenize{ioloop:tornado.ioloop.IOLoop.clear_instance}]{\sphinxcrossref{\sphinxcode{\sphinxupquote{clear\_instance()}}}}} is an alias for {\hyperref[\detokenize{ioloop:tornado.ioloop.IOLoop.clear_current}]{\sphinxcrossref{\sphinxcode{\sphinxupquote{clear\_current()}}}}}.

\DUrole{versionmodified,deprecated}{Deprecated since version 5.0.}

\end{fulllineitems}



\paragraph{I/O events}
\label{\detokenize{ioloop:i-o-events}}\index{add\_handler() (tornado.ioloop.IOLoop method)@\spxentry{add\_handler()}\spxextra{tornado.ioloop.IOLoop method}}

\begin{fulllineitems}
\phantomsection\label{\detokenize{ioloop:tornado.ioloop.IOLoop.add_handler}}\pysiglinewithargsret{\sphinxcode{\sphinxupquote{IOLoop.}}\sphinxbfcode{\sphinxupquote{add\_handler}}}{\emph{fd: Union{[}int, tornado.ioloop.\_Selectable{]}, handler: Callable{[}{[}...{]}, None{]}, events: int}}{{ $\rightarrow$ None}}
Registers the given handler to receive the given events for \sphinxcode{\sphinxupquote{fd}}.

The \sphinxcode{\sphinxupquote{fd}} argument may either be an integer file descriptor or
a file-like object with a \sphinxcode{\sphinxupquote{fileno()}} and \sphinxcode{\sphinxupquote{close()}} method.

The \sphinxcode{\sphinxupquote{events}} argument is a bitwise or of the constants
\sphinxcode{\sphinxupquote{IOLoop.READ}}, \sphinxcode{\sphinxupquote{IOLoop.WRITE}}, and \sphinxcode{\sphinxupquote{IOLoop.ERROR}}.

When an event occurs, \sphinxcode{\sphinxupquote{handler(fd, events)}} will be run.

\DUrole{versionmodified,changed}{Changed in version 4.0: }Added the ability to pass file-like objects in addition to
raw file descriptors.

\end{fulllineitems}

\index{update\_handler() (tornado.ioloop.IOLoop method)@\spxentry{update\_handler()}\spxextra{tornado.ioloop.IOLoop method}}

\begin{fulllineitems}
\phantomsection\label{\detokenize{ioloop:tornado.ioloop.IOLoop.update_handler}}\pysiglinewithargsret{\sphinxcode{\sphinxupquote{IOLoop.}}\sphinxbfcode{\sphinxupquote{update\_handler}}}{\emph{fd: Union{[}int, tornado.ioloop.\_Selectable{]}, events: int}}{{ $\rightarrow$ None}}
Changes the events we listen for \sphinxcode{\sphinxupquote{fd}}.

\DUrole{versionmodified,changed}{Changed in version 4.0: }Added the ability to pass file-like objects in addition to
raw file descriptors.

\end{fulllineitems}

\index{remove\_handler() (tornado.ioloop.IOLoop method)@\spxentry{remove\_handler()}\spxextra{tornado.ioloop.IOLoop method}}

\begin{fulllineitems}
\phantomsection\label{\detokenize{ioloop:tornado.ioloop.IOLoop.remove_handler}}\pysiglinewithargsret{\sphinxcode{\sphinxupquote{IOLoop.}}\sphinxbfcode{\sphinxupquote{remove\_handler}}}{\emph{fd: Union{[}int, tornado.ioloop.\_Selectable{]}}}{{ $\rightarrow$ None}}
Stop listening for events on \sphinxcode{\sphinxupquote{fd}}.

\DUrole{versionmodified,changed}{Changed in version 4.0: }Added the ability to pass file-like objects in addition to
raw file descriptors.

\end{fulllineitems}



\paragraph{Callbacks and timeouts}
\label{\detokenize{ioloop:callbacks-and-timeouts}}\index{add\_callback() (tornado.ioloop.IOLoop method)@\spxentry{add\_callback()}\spxextra{tornado.ioloop.IOLoop method}}

\begin{fulllineitems}
\phantomsection\label{\detokenize{ioloop:tornado.ioloop.IOLoop.add_callback}}\pysiglinewithargsret{\sphinxcode{\sphinxupquote{IOLoop.}}\sphinxbfcode{\sphinxupquote{add\_callback}}}{\emph{callback: Callable}, \emph{*args}, \emph{**kwargs}}{{ $\rightarrow$ None}}
Calls the given callback on the next I/O loop iteration.

It is safe to call this method from any thread at any time,
except from a signal handler.  Note that this is the \sphinxstylestrong{only}
method in {\hyperref[\detokenize{ioloop:tornado.ioloop.IOLoop}]{\sphinxcrossref{\sphinxcode{\sphinxupquote{IOLoop}}}}} that makes this thread-safety guarantee; all
other interaction with the {\hyperref[\detokenize{ioloop:tornado.ioloop.IOLoop}]{\sphinxcrossref{\sphinxcode{\sphinxupquote{IOLoop}}}}} must be done from that
{\hyperref[\detokenize{ioloop:tornado.ioloop.IOLoop}]{\sphinxcrossref{\sphinxcode{\sphinxupquote{IOLoop}}}}}’s thread.  {\hyperref[\detokenize{ioloop:tornado.ioloop.IOLoop.add_callback}]{\sphinxcrossref{\sphinxcode{\sphinxupquote{add\_callback()}}}}} may be used to transfer
control from other threads to the {\hyperref[\detokenize{ioloop:tornado.ioloop.IOLoop}]{\sphinxcrossref{\sphinxcode{\sphinxupquote{IOLoop}}}}}’s thread.

To add a callback from a signal handler, see
{\hyperref[\detokenize{ioloop:tornado.ioloop.IOLoop.add_callback_from_signal}]{\sphinxcrossref{\sphinxcode{\sphinxupquote{add\_callback\_from\_signal}}}}}.

\end{fulllineitems}

\index{add\_callback\_from\_signal() (tornado.ioloop.IOLoop method)@\spxentry{add\_callback\_from\_signal()}\spxextra{tornado.ioloop.IOLoop method}}

\begin{fulllineitems}
\phantomsection\label{\detokenize{ioloop:tornado.ioloop.IOLoop.add_callback_from_signal}}\pysiglinewithargsret{\sphinxcode{\sphinxupquote{IOLoop.}}\sphinxbfcode{\sphinxupquote{add\_callback\_from\_signal}}}{\emph{callback: Callable}, \emph{*args}, \emph{**kwargs}}{{ $\rightarrow$ None}}
Calls the given callback on the next I/O loop iteration.

Safe for use from a Python signal handler; should not be used
otherwise.

\end{fulllineitems}

\index{add\_future() (tornado.ioloop.IOLoop method)@\spxentry{add\_future()}\spxextra{tornado.ioloop.IOLoop method}}

\begin{fulllineitems}
\phantomsection\label{\detokenize{ioloop:tornado.ioloop.IOLoop.add_future}}\pysiglinewithargsret{\sphinxcode{\sphinxupquote{IOLoop.}}\sphinxbfcode{\sphinxupquote{add\_future}}}{\emph{future: Union{[}Future{[}\_T{]}, concurrent.futures.Future{[}\_T{]}{]}, callback: Callable{[}{[}Future{[}\_T{]}{]}, None{]}}}{{ $\rightarrow$ None}}
Schedules a callback on the \sphinxcode{\sphinxupquote{IOLoop}} when the given
{\hyperref[\detokenize{concurrent:tornado.concurrent.Future}]{\sphinxcrossref{\sphinxcode{\sphinxupquote{Future}}}}} is finished.

The callback is invoked with one argument, the
{\hyperref[\detokenize{concurrent:tornado.concurrent.Future}]{\sphinxcrossref{\sphinxcode{\sphinxupquote{Future}}}}}.

This method only accepts {\hyperref[\detokenize{concurrent:tornado.concurrent.Future}]{\sphinxcrossref{\sphinxcode{\sphinxupquote{Future}}}}} objects and not other
awaitables (unlike most of Tornado where the two are
interchangeable).

\end{fulllineitems}

\index{add\_timeout() (tornado.ioloop.IOLoop method)@\spxentry{add\_timeout()}\spxextra{tornado.ioloop.IOLoop method}}

\begin{fulllineitems}
\phantomsection\label{\detokenize{ioloop:tornado.ioloop.IOLoop.add_timeout}}\pysiglinewithargsret{\sphinxcode{\sphinxupquote{IOLoop.}}\sphinxbfcode{\sphinxupquote{add\_timeout}}}{\emph{deadline: Union{[}float, datetime.timedelta{]}, callback: Callable{[}{[}...{]}, None{]}, *args, **kwargs}}{{ $\rightarrow$ object}}
Runs the \sphinxcode{\sphinxupquote{callback}} at the time \sphinxcode{\sphinxupquote{deadline}} from the I/O loop.

Returns an opaque handle that may be passed to
{\hyperref[\detokenize{ioloop:tornado.ioloop.IOLoop.remove_timeout}]{\sphinxcrossref{\sphinxcode{\sphinxupquote{remove\_timeout}}}}} to cancel.

\sphinxcode{\sphinxupquote{deadline}} may be a number denoting a time (on the same
scale as {\hyperref[\detokenize{ioloop:tornado.ioloop.IOLoop.time}]{\sphinxcrossref{\sphinxcode{\sphinxupquote{IOLoop.time}}}}}, normally \sphinxhref{https://docs.python.org/3.6/library/time.html\#time.time}{\sphinxcode{\sphinxupquote{time.time}}}), or a
\sphinxhref{https://docs.python.org/3.6/library/datetime.html\#datetime.timedelta}{\sphinxcode{\sphinxupquote{datetime.timedelta}}} object for a deadline relative to the
current time.  Since Tornado 4.0, {\hyperref[\detokenize{ioloop:tornado.ioloop.IOLoop.call_later}]{\sphinxcrossref{\sphinxcode{\sphinxupquote{call\_later}}}}} is a more
convenient alternative for the relative case since it does not
require a timedelta object.

Note that it is not safe to call {\hyperref[\detokenize{ioloop:tornado.ioloop.IOLoop.add_timeout}]{\sphinxcrossref{\sphinxcode{\sphinxupquote{add\_timeout}}}}} from other threads.
Instead, you must use {\hyperref[\detokenize{ioloop:tornado.ioloop.IOLoop.add_callback}]{\sphinxcrossref{\sphinxcode{\sphinxupquote{add\_callback}}}}} to transfer control to the
{\hyperref[\detokenize{ioloop:tornado.ioloop.IOLoop}]{\sphinxcrossref{\sphinxcode{\sphinxupquote{IOLoop}}}}}’s thread, and then call {\hyperref[\detokenize{ioloop:tornado.ioloop.IOLoop.add_timeout}]{\sphinxcrossref{\sphinxcode{\sphinxupquote{add\_timeout}}}}} from there.

Subclasses of IOLoop must implement either {\hyperref[\detokenize{ioloop:tornado.ioloop.IOLoop.add_timeout}]{\sphinxcrossref{\sphinxcode{\sphinxupquote{add\_timeout}}}}} or
{\hyperref[\detokenize{ioloop:tornado.ioloop.IOLoop.call_at}]{\sphinxcrossref{\sphinxcode{\sphinxupquote{call\_at}}}}}; the default implementations of each will call
the other.  {\hyperref[\detokenize{ioloop:tornado.ioloop.IOLoop.call_at}]{\sphinxcrossref{\sphinxcode{\sphinxupquote{call\_at}}}}} is usually easier to implement, but
subclasses that wish to maintain compatibility with Tornado
versions prior to 4.0 must use {\hyperref[\detokenize{ioloop:tornado.ioloop.IOLoop.add_timeout}]{\sphinxcrossref{\sphinxcode{\sphinxupquote{add\_timeout}}}}} instead.

\DUrole{versionmodified,changed}{Changed in version 4.0: }Now passes through \sphinxcode{\sphinxupquote{*args}} and \sphinxcode{\sphinxupquote{**kwargs}} to the callback.

\end{fulllineitems}

\index{call\_at() (tornado.ioloop.IOLoop method)@\spxentry{call\_at()}\spxextra{tornado.ioloop.IOLoop method}}

\begin{fulllineitems}
\phantomsection\label{\detokenize{ioloop:tornado.ioloop.IOLoop.call_at}}\pysiglinewithargsret{\sphinxcode{\sphinxupquote{IOLoop.}}\sphinxbfcode{\sphinxupquote{call\_at}}}{\emph{when: float, callback: Callable{[}{[}...{]}, None{]}, *args, **kwargs}}{{ $\rightarrow$ object}}
Runs the \sphinxcode{\sphinxupquote{callback}} at the absolute time designated by \sphinxcode{\sphinxupquote{when}}.

\sphinxcode{\sphinxupquote{when}} must be a number using the same reference point as
{\hyperref[\detokenize{ioloop:tornado.ioloop.IOLoop.time}]{\sphinxcrossref{\sphinxcode{\sphinxupquote{IOLoop.time}}}}}.

Returns an opaque handle that may be passed to {\hyperref[\detokenize{ioloop:tornado.ioloop.IOLoop.remove_timeout}]{\sphinxcrossref{\sphinxcode{\sphinxupquote{remove\_timeout}}}}}
to cancel.  Note that unlike the \sphinxhref{https://docs.python.org/3.6/library/asyncio.html\#module-asyncio}{\sphinxcode{\sphinxupquote{asyncio}}} method of the same
name, the returned object does not have a \sphinxcode{\sphinxupquote{cancel()}} method.

See {\hyperref[\detokenize{ioloop:tornado.ioloop.IOLoop.add_timeout}]{\sphinxcrossref{\sphinxcode{\sphinxupquote{add\_timeout}}}}} for comments on thread-safety and subclassing.

\DUrole{versionmodified,added}{New in version 4.0.}

\end{fulllineitems}

\index{call\_later() (tornado.ioloop.IOLoop method)@\spxentry{call\_later()}\spxextra{tornado.ioloop.IOLoop method}}

\begin{fulllineitems}
\phantomsection\label{\detokenize{ioloop:tornado.ioloop.IOLoop.call_later}}\pysiglinewithargsret{\sphinxcode{\sphinxupquote{IOLoop.}}\sphinxbfcode{\sphinxupquote{call\_later}}}{\emph{delay: float, callback: Callable{[}{[}...{]}, None{]}, *args, **kwargs}}{{ $\rightarrow$ object}}
Runs the \sphinxcode{\sphinxupquote{callback}} after \sphinxcode{\sphinxupquote{delay}} seconds have passed.

Returns an opaque handle that may be passed to {\hyperref[\detokenize{ioloop:tornado.ioloop.IOLoop.remove_timeout}]{\sphinxcrossref{\sphinxcode{\sphinxupquote{remove\_timeout}}}}}
to cancel.  Note that unlike the \sphinxhref{https://docs.python.org/3.6/library/asyncio.html\#module-asyncio}{\sphinxcode{\sphinxupquote{asyncio}}} method of the same
name, the returned object does not have a \sphinxcode{\sphinxupquote{cancel()}} method.

See {\hyperref[\detokenize{ioloop:tornado.ioloop.IOLoop.add_timeout}]{\sphinxcrossref{\sphinxcode{\sphinxupquote{add\_timeout}}}}} for comments on thread-safety and subclassing.

\DUrole{versionmodified,added}{New in version 4.0.}

\end{fulllineitems}

\index{remove\_timeout() (tornado.ioloop.IOLoop method)@\spxentry{remove\_timeout()}\spxextra{tornado.ioloop.IOLoop method}}

\begin{fulllineitems}
\phantomsection\label{\detokenize{ioloop:tornado.ioloop.IOLoop.remove_timeout}}\pysiglinewithargsret{\sphinxcode{\sphinxupquote{IOLoop.}}\sphinxbfcode{\sphinxupquote{remove\_timeout}}}{\emph{timeout: object}}{{ $\rightarrow$ None}}
Cancels a pending timeout.

The argument is a handle as returned by {\hyperref[\detokenize{ioloop:tornado.ioloop.IOLoop.add_timeout}]{\sphinxcrossref{\sphinxcode{\sphinxupquote{add\_timeout}}}}}.  It is
safe to call {\hyperref[\detokenize{ioloop:tornado.ioloop.IOLoop.remove_timeout}]{\sphinxcrossref{\sphinxcode{\sphinxupquote{remove\_timeout}}}}} even if the callback has already
been run.

\end{fulllineitems}

\index{spawn\_callback() (tornado.ioloop.IOLoop method)@\spxentry{spawn\_callback()}\spxextra{tornado.ioloop.IOLoop method}}

\begin{fulllineitems}
\phantomsection\label{\detokenize{ioloop:tornado.ioloop.IOLoop.spawn_callback}}\pysiglinewithargsret{\sphinxcode{\sphinxupquote{IOLoop.}}\sphinxbfcode{\sphinxupquote{spawn\_callback}}}{\emph{callback: Callable}, \emph{*args}, \emph{**kwargs}}{{ $\rightarrow$ None}}
Calls the given callback on the next IOLoop iteration.

As of Tornado 6.0, this method is equivalent to {\hyperref[\detokenize{ioloop:tornado.ioloop.IOLoop.add_callback}]{\sphinxcrossref{\sphinxcode{\sphinxupquote{add\_callback}}}}}.

\DUrole{versionmodified,added}{New in version 4.0.}

\end{fulllineitems}

\index{run\_in\_executor() (tornado.ioloop.IOLoop method)@\spxentry{run\_in\_executor()}\spxextra{tornado.ioloop.IOLoop method}}

\begin{fulllineitems}
\phantomsection\label{\detokenize{ioloop:tornado.ioloop.IOLoop.run_in_executor}}\pysiglinewithargsret{\sphinxcode{\sphinxupquote{IOLoop.}}\sphinxbfcode{\sphinxupquote{run\_in\_executor}}}{\emph{executor: Optional{[}concurrent.futures.\_base.Executor{]}, func: Callable{[}{[}...{]}, \_T{]}, *args}}{{ $\rightarrow$ Awaitable{[}\_T{]}}}
Runs a function in a \sphinxcode{\sphinxupquote{concurrent.futures.Executor}}. If
\sphinxcode{\sphinxupquote{executor}} is \sphinxcode{\sphinxupquote{None}}, the IO loop’s default executor will be used.

Use \sphinxhref{https://docs.python.org/3.6/library/functools.html\#functools.partial}{\sphinxcode{\sphinxupquote{functools.partial}}} to pass keyword arguments to \sphinxcode{\sphinxupquote{func}}.

\DUrole{versionmodified,added}{New in version 5.0.}

\end{fulllineitems}

\index{set\_default\_executor() (tornado.ioloop.IOLoop method)@\spxentry{set\_default\_executor()}\spxextra{tornado.ioloop.IOLoop method}}

\begin{fulllineitems}
\phantomsection\label{\detokenize{ioloop:tornado.ioloop.IOLoop.set_default_executor}}\pysiglinewithargsret{\sphinxcode{\sphinxupquote{IOLoop.}}\sphinxbfcode{\sphinxupquote{set\_default\_executor}}}{\emph{executor: concurrent.futures.\_base.Executor}}{{ $\rightarrow$ None}}
Sets the default executor to use with {\hyperref[\detokenize{ioloop:tornado.ioloop.IOLoop.run_in_executor}]{\sphinxcrossref{\sphinxcode{\sphinxupquote{run\_in\_executor()}}}}}.

\DUrole{versionmodified,added}{New in version 5.0.}

\end{fulllineitems}

\index{time() (tornado.ioloop.IOLoop method)@\spxentry{time()}\spxextra{tornado.ioloop.IOLoop method}}

\begin{fulllineitems}
\phantomsection\label{\detokenize{ioloop:tornado.ioloop.IOLoop.time}}\pysiglinewithargsret{\sphinxcode{\sphinxupquote{IOLoop.}}\sphinxbfcode{\sphinxupquote{time}}}{}{{ $\rightarrow$ float}}
Returns the current time according to the {\hyperref[\detokenize{ioloop:tornado.ioloop.IOLoop}]{\sphinxcrossref{\sphinxcode{\sphinxupquote{IOLoop}}}}}’s clock.

The return value is a floating-point number relative to an
unspecified time in the past.

Historically, the IOLoop could be customized to use e.g.
\sphinxhref{https://docs.python.org/3.6/library/time.html\#time.monotonic}{\sphinxcode{\sphinxupquote{time.monotonic}}} instead of \sphinxhref{https://docs.python.org/3.6/library/time.html\#time.time}{\sphinxcode{\sphinxupquote{time.time}}}, but this is not
currently supported and so this method is equivalent to
\sphinxhref{https://docs.python.org/3.6/library/time.html\#time.time}{\sphinxcode{\sphinxupquote{time.time}}}.

\end{fulllineitems}

\index{PeriodicCallback (class in tornado.ioloop)@\spxentry{PeriodicCallback}\spxextra{class in tornado.ioloop}}

\begin{fulllineitems}
\phantomsection\label{\detokenize{ioloop:tornado.ioloop.PeriodicCallback}}\pysiglinewithargsret{\sphinxbfcode{\sphinxupquote{class }}\sphinxcode{\sphinxupquote{tornado.ioloop.}}\sphinxbfcode{\sphinxupquote{PeriodicCallback}}}{\emph{callback: Callable{[}{[}{]}, None{]}, callback\_time: float, jitter: float = 0}}{}
Schedules the given callback to be called periodically.

The callback is called every \sphinxcode{\sphinxupquote{callback\_time}} milliseconds.
Note that the timeout is given in milliseconds, while most other
time-related functions in Tornado use seconds.

If \sphinxcode{\sphinxupquote{jitter}} is specified, each callback time will be randomly selected
within a window of \sphinxcode{\sphinxupquote{jitter * callback\_time}} milliseconds.
Jitter can be used to reduce alignment of events with similar periods.
A jitter of 0.1 means allowing a 10\% variation in callback time.
The window is centered on \sphinxcode{\sphinxupquote{callback\_time}} so the total number of calls
within a given interval should not be significantly affected by adding
jitter.

If the callback runs for longer than \sphinxcode{\sphinxupquote{callback\_time}} milliseconds,
subsequent invocations will be skipped to get back on schedule.

{\hyperref[\detokenize{ioloop:tornado.ioloop.PeriodicCallback.start}]{\sphinxcrossref{\sphinxcode{\sphinxupquote{start}}}}} must be called after the {\hyperref[\detokenize{ioloop:tornado.ioloop.PeriodicCallback}]{\sphinxcrossref{\sphinxcode{\sphinxupquote{PeriodicCallback}}}}} is created.

\DUrole{versionmodified,changed}{Changed in version 5.0: }The \sphinxcode{\sphinxupquote{io\_loop}} argument (deprecated since version 4.1) has been removed.

\DUrole{versionmodified,changed}{Changed in version 5.1: }The \sphinxcode{\sphinxupquote{jitter}} argument is added.
\index{start() (tornado.ioloop.PeriodicCallback method)@\spxentry{start()}\spxextra{tornado.ioloop.PeriodicCallback method}}

\begin{fulllineitems}
\phantomsection\label{\detokenize{ioloop:tornado.ioloop.PeriodicCallback.start}}\pysiglinewithargsret{\sphinxbfcode{\sphinxupquote{start}}}{}{{ $\rightarrow$ None}}
Starts the timer.

\end{fulllineitems}

\index{stop() (tornado.ioloop.PeriodicCallback method)@\spxentry{stop()}\spxextra{tornado.ioloop.PeriodicCallback method}}

\begin{fulllineitems}
\phantomsection\label{\detokenize{ioloop:tornado.ioloop.PeriodicCallback.stop}}\pysiglinewithargsret{\sphinxbfcode{\sphinxupquote{stop}}}{}{{ $\rightarrow$ None}}
Stops the timer.

\end{fulllineitems}

\index{is\_running() (tornado.ioloop.PeriodicCallback method)@\spxentry{is\_running()}\spxextra{tornado.ioloop.PeriodicCallback method}}

\begin{fulllineitems}
\phantomsection\label{\detokenize{ioloop:tornado.ioloop.PeriodicCallback.is_running}}\pysiglinewithargsret{\sphinxbfcode{\sphinxupquote{is\_running}}}{}{{ $\rightarrow$ bool}}
Returns \sphinxcode{\sphinxupquote{True}} if this {\hyperref[\detokenize{ioloop:tornado.ioloop.PeriodicCallback}]{\sphinxcrossref{\sphinxcode{\sphinxupquote{PeriodicCallback}}}}} has been started.

\DUrole{versionmodified,added}{New in version 4.1.}

\end{fulllineitems}


\end{fulllineitems}



\subsection{\sphinxstyleliteralintitle{\sphinxupquote{tornado.iostream}} — Convenient wrappers for non-blocking sockets}
\label{\detokenize{iostream:module-tornado.iostream}}\label{\detokenize{iostream:tornado-iostream-convenient-wrappers-for-non-blocking-sockets}}\label{\detokenize{iostream::doc}}\index{tornado.iostream (module)@\spxentry{tornado.iostream}\spxextra{module}}
Utility classes to write to and read from non-blocking files and sockets.

Contents:
\begin{itemize}
\item {} 
{\hyperref[\detokenize{iostream:tornado.iostream.BaseIOStream}]{\sphinxcrossref{\sphinxcode{\sphinxupquote{BaseIOStream}}}}}: Generic interface for reading and writing.

\item {} 
{\hyperref[\detokenize{iostream:tornado.iostream.IOStream}]{\sphinxcrossref{\sphinxcode{\sphinxupquote{IOStream}}}}}: Implementation of BaseIOStream using non-blocking sockets.

\item {} 
{\hyperref[\detokenize{iostream:tornado.iostream.SSLIOStream}]{\sphinxcrossref{\sphinxcode{\sphinxupquote{SSLIOStream}}}}}: SSL-aware version of IOStream.

\item {} 
{\hyperref[\detokenize{iostream:tornado.iostream.PipeIOStream}]{\sphinxcrossref{\sphinxcode{\sphinxupquote{PipeIOStream}}}}}: Pipe-based IOStream implementation.

\end{itemize}


\subsubsection{Base class}
\label{\detokenize{iostream:base-class}}\index{BaseIOStream (class in tornado.iostream)@\spxentry{BaseIOStream}\spxextra{class in tornado.iostream}}

\begin{fulllineitems}
\phantomsection\label{\detokenize{iostream:tornado.iostream.BaseIOStream}}\pysiglinewithargsret{\sphinxbfcode{\sphinxupquote{class }}\sphinxcode{\sphinxupquote{tornado.iostream.}}\sphinxbfcode{\sphinxupquote{BaseIOStream}}}{\emph{max\_buffer\_size: int = None}, \emph{read\_chunk\_size: int = None}, \emph{max\_write\_buffer\_size: int = None}}{}
A utility class to write to and read from a non-blocking file or socket.

We support a non-blocking \sphinxcode{\sphinxupquote{write()}} and a family of \sphinxcode{\sphinxupquote{read\_*()}}
methods. When the operation completes, the \sphinxcode{\sphinxupquote{Awaitable}} will resolve
with the data read (or \sphinxcode{\sphinxupquote{None}} for \sphinxcode{\sphinxupquote{write()}}). All outstanding
\sphinxcode{\sphinxupquote{Awaitables}} will resolve with a {\hyperref[\detokenize{iostream:tornado.iostream.StreamClosedError}]{\sphinxcrossref{\sphinxcode{\sphinxupquote{StreamClosedError}}}}} when the
stream is closed; {\hyperref[\detokenize{iostream:tornado.iostream.BaseIOStream.set_close_callback}]{\sphinxcrossref{\sphinxcode{\sphinxupquote{BaseIOStream.set\_close\_callback}}}}} can also be used
to be notified of a closed stream.

When a stream is closed due to an error, the IOStream’s \sphinxcode{\sphinxupquote{error}}
attribute contains the exception object.

Subclasses must implement {\hyperref[\detokenize{iostream:tornado.iostream.BaseIOStream.fileno}]{\sphinxcrossref{\sphinxcode{\sphinxupquote{fileno}}}}}, {\hyperref[\detokenize{iostream:tornado.iostream.BaseIOStream.close_fd}]{\sphinxcrossref{\sphinxcode{\sphinxupquote{close\_fd}}}}}, {\hyperref[\detokenize{iostream:tornado.iostream.BaseIOStream.write_to_fd}]{\sphinxcrossref{\sphinxcode{\sphinxupquote{write\_to\_fd}}}}},
{\hyperref[\detokenize{iostream:tornado.iostream.BaseIOStream.read_from_fd}]{\sphinxcrossref{\sphinxcode{\sphinxupquote{read\_from\_fd}}}}}, and optionally {\hyperref[\detokenize{iostream:tornado.iostream.BaseIOStream.get_fd_error}]{\sphinxcrossref{\sphinxcode{\sphinxupquote{get\_fd\_error}}}}}.

{\hyperref[\detokenize{iostream:tornado.iostream.BaseIOStream}]{\sphinxcrossref{\sphinxcode{\sphinxupquote{BaseIOStream}}}}} constructor.
\begin{quote}\begin{description}
\item[{Parameters}] \leavevmode\begin{itemize}
\item {} 
\sphinxstyleliteralstrong{\sphinxupquote{max\_buffer\_size}} \textendash{} Maximum amount of incoming data to buffer;
defaults to 100MB.

\item {} 
\sphinxstyleliteralstrong{\sphinxupquote{read\_chunk\_size}} \textendash{} Amount of data to read at one time from the
underlying transport; defaults to 64KB.

\item {} 
\sphinxstyleliteralstrong{\sphinxupquote{max\_write\_buffer\_size}} \textendash{} Amount of outgoing data to buffer;
defaults to unlimited.

\end{itemize}

\end{description}\end{quote}

\DUrole{versionmodified,changed}{Changed in version 4.0: }Add the \sphinxcode{\sphinxupquote{max\_write\_buffer\_size}} parameter.  Changed default
\sphinxcode{\sphinxupquote{read\_chunk\_size}} to 64KB.

\DUrole{versionmodified,changed}{Changed in version 5.0: }The \sphinxcode{\sphinxupquote{io\_loop}} argument (deprecated since version 4.1) has been
removed.

\end{fulllineitems}



\paragraph{Main interface}
\label{\detokenize{iostream:main-interface}}\index{write() (tornado.iostream.BaseIOStream method)@\spxentry{write()}\spxextra{tornado.iostream.BaseIOStream method}}

\begin{fulllineitems}
\phantomsection\label{\detokenize{iostream:tornado.iostream.BaseIOStream.write}}\pysiglinewithargsret{\sphinxcode{\sphinxupquote{BaseIOStream.}}\sphinxbfcode{\sphinxupquote{write}}}{\emph{data: Union{[}bytes, memoryview{]}}}{{ $\rightarrow$ Future{[}None{]}}}
Asynchronously write the given data to this stream.

This method returns a {\hyperref[\detokenize{concurrent:tornado.concurrent.Future}]{\sphinxcrossref{\sphinxcode{\sphinxupquote{Future}}}}} that resolves (with a result
of \sphinxcode{\sphinxupquote{None}}) when the write has been completed.

The \sphinxcode{\sphinxupquote{data}} argument may be of type \sphinxhref{https://docs.python.org/3.6/library/stdtypes.html\#bytes}{\sphinxcode{\sphinxupquote{bytes}}} or \sphinxhref{https://docs.python.org/3.6/library/stdtypes.html\#memoryview}{\sphinxcode{\sphinxupquote{memoryview}}}.

\DUrole{versionmodified,changed}{Changed in version 4.0: }Now returns a {\hyperref[\detokenize{concurrent:tornado.concurrent.Future}]{\sphinxcrossref{\sphinxcode{\sphinxupquote{Future}}}}} if no callback is given.

\DUrole{versionmodified,changed}{Changed in version 4.5: }Added support for \sphinxhref{https://docs.python.org/3.6/library/stdtypes.html\#memoryview}{\sphinxcode{\sphinxupquote{memoryview}}} arguments.

\DUrole{versionmodified,changed}{Changed in version 6.0: }The \sphinxcode{\sphinxupquote{callback}} argument was removed. Use the returned
{\hyperref[\detokenize{concurrent:tornado.concurrent.Future}]{\sphinxcrossref{\sphinxcode{\sphinxupquote{Future}}}}} instead.

\end{fulllineitems}

\index{read\_bytes() (tornado.iostream.BaseIOStream method)@\spxentry{read\_bytes()}\spxextra{tornado.iostream.BaseIOStream method}}

\begin{fulllineitems}
\phantomsection\label{\detokenize{iostream:tornado.iostream.BaseIOStream.read_bytes}}\pysiglinewithargsret{\sphinxcode{\sphinxupquote{BaseIOStream.}}\sphinxbfcode{\sphinxupquote{read\_bytes}}}{\emph{num\_bytes: int}, \emph{partial: bool = False}}{{ $\rightarrow$ Awaitable{[}bytes{]}}}
Asynchronously read a number of bytes.

If \sphinxcode{\sphinxupquote{partial}} is true, data is returned as soon as we have
any bytes to return (but never more than \sphinxcode{\sphinxupquote{num\_bytes}})

\DUrole{versionmodified,changed}{Changed in version 4.0: }Added the \sphinxcode{\sphinxupquote{partial}} argument.  The callback argument is now
optional and a {\hyperref[\detokenize{concurrent:tornado.concurrent.Future}]{\sphinxcrossref{\sphinxcode{\sphinxupquote{Future}}}}} will be returned if it is omitted.

\DUrole{versionmodified,changed}{Changed in version 6.0: }The \sphinxcode{\sphinxupquote{callback}} and \sphinxcode{\sphinxupquote{streaming\_callback}} arguments have
been removed. Use the returned {\hyperref[\detokenize{concurrent:tornado.concurrent.Future}]{\sphinxcrossref{\sphinxcode{\sphinxupquote{Future}}}}} (and
\sphinxcode{\sphinxupquote{partial=True}} for \sphinxcode{\sphinxupquote{streaming\_callback}}) instead.

\end{fulllineitems}

\index{read\_into() (tornado.iostream.BaseIOStream method)@\spxentry{read\_into()}\spxextra{tornado.iostream.BaseIOStream method}}

\begin{fulllineitems}
\phantomsection\label{\detokenize{iostream:tornado.iostream.BaseIOStream.read_into}}\pysiglinewithargsret{\sphinxcode{\sphinxupquote{BaseIOStream.}}\sphinxbfcode{\sphinxupquote{read\_into}}}{\emph{buf: bytearray}, \emph{partial: bool = False}}{{ $\rightarrow$ Awaitable{[}int{]}}}
Asynchronously read a number of bytes.

\sphinxcode{\sphinxupquote{buf}} must be a writable buffer into which data will be read.

If \sphinxcode{\sphinxupquote{partial}} is true, the callback is run as soon as any bytes
have been read.  Otherwise, it is run when the \sphinxcode{\sphinxupquote{buf}} has been
entirely filled with read data.

\DUrole{versionmodified,added}{New in version 5.0.}

\DUrole{versionmodified,changed}{Changed in version 6.0: }The \sphinxcode{\sphinxupquote{callback}} argument was removed. Use the returned
{\hyperref[\detokenize{concurrent:tornado.concurrent.Future}]{\sphinxcrossref{\sphinxcode{\sphinxupquote{Future}}}}} instead.

\end{fulllineitems}

\index{read\_until() (tornado.iostream.BaseIOStream method)@\spxentry{read\_until()}\spxextra{tornado.iostream.BaseIOStream method}}

\begin{fulllineitems}
\phantomsection\label{\detokenize{iostream:tornado.iostream.BaseIOStream.read_until}}\pysiglinewithargsret{\sphinxcode{\sphinxupquote{BaseIOStream.}}\sphinxbfcode{\sphinxupquote{read\_until}}}{\emph{delimiter: bytes}, \emph{max\_bytes: int = None}}{{ $\rightarrow$ Awaitable{[}bytes{]}}}
Asynchronously read until we have found the given delimiter.

The result includes all the data read including the delimiter.

If \sphinxcode{\sphinxupquote{max\_bytes}} is not None, the connection will be closed
if more than \sphinxcode{\sphinxupquote{max\_bytes}} bytes have been read and the delimiter
is not found.

\DUrole{versionmodified,changed}{Changed in version 4.0: }Added the \sphinxcode{\sphinxupquote{max\_bytes}} argument.  The \sphinxcode{\sphinxupquote{callback}} argument is
now optional and a {\hyperref[\detokenize{concurrent:tornado.concurrent.Future}]{\sphinxcrossref{\sphinxcode{\sphinxupquote{Future}}}}} will be returned if it is omitted.

\DUrole{versionmodified,changed}{Changed in version 6.0: }The \sphinxcode{\sphinxupquote{callback}} argument was removed. Use the returned
{\hyperref[\detokenize{concurrent:tornado.concurrent.Future}]{\sphinxcrossref{\sphinxcode{\sphinxupquote{Future}}}}} instead.

\end{fulllineitems}

\index{read\_until\_regex() (tornado.iostream.BaseIOStream method)@\spxentry{read\_until\_regex()}\spxextra{tornado.iostream.BaseIOStream method}}

\begin{fulllineitems}
\phantomsection\label{\detokenize{iostream:tornado.iostream.BaseIOStream.read_until_regex}}\pysiglinewithargsret{\sphinxcode{\sphinxupquote{BaseIOStream.}}\sphinxbfcode{\sphinxupquote{read\_until\_regex}}}{\emph{regex: bytes}, \emph{max\_bytes: int = None}}{{ $\rightarrow$ Awaitable{[}bytes{]}}}
Asynchronously read until we have matched the given regex.

The result includes the data that matches the regex and anything
that came before it.

If \sphinxcode{\sphinxupquote{max\_bytes}} is not None, the connection will be closed
if more than \sphinxcode{\sphinxupquote{max\_bytes}} bytes have been read and the regex is
not satisfied.

\DUrole{versionmodified,changed}{Changed in version 4.0: }Added the \sphinxcode{\sphinxupquote{max\_bytes}} argument.  The \sphinxcode{\sphinxupquote{callback}} argument is
now optional and a {\hyperref[\detokenize{concurrent:tornado.concurrent.Future}]{\sphinxcrossref{\sphinxcode{\sphinxupquote{Future}}}}} will be returned if it is omitted.

\DUrole{versionmodified,changed}{Changed in version 6.0: }The \sphinxcode{\sphinxupquote{callback}} argument was removed. Use the returned
{\hyperref[\detokenize{concurrent:tornado.concurrent.Future}]{\sphinxcrossref{\sphinxcode{\sphinxupquote{Future}}}}} instead.

\end{fulllineitems}

\index{read\_until\_close() (tornado.iostream.BaseIOStream method)@\spxentry{read\_until\_close()}\spxextra{tornado.iostream.BaseIOStream method}}

\begin{fulllineitems}
\phantomsection\label{\detokenize{iostream:tornado.iostream.BaseIOStream.read_until_close}}\pysiglinewithargsret{\sphinxcode{\sphinxupquote{BaseIOStream.}}\sphinxbfcode{\sphinxupquote{read\_until\_close}}}{}{{ $\rightarrow$ Awaitable{[}bytes{]}}}
Asynchronously reads all data from the socket until it is closed.

This will buffer all available data until \sphinxcode{\sphinxupquote{max\_buffer\_size}}
is reached. If flow control or cancellation are desired, use a
loop with {\hyperref[\detokenize{iostream:tornado.iostream.BaseIOStream.read_bytes}]{\sphinxcrossref{\sphinxcode{\sphinxupquote{read\_bytes(partial=True)}}}}} instead.

\DUrole{versionmodified,changed}{Changed in version 4.0: }The callback argument is now optional and a {\hyperref[\detokenize{concurrent:tornado.concurrent.Future}]{\sphinxcrossref{\sphinxcode{\sphinxupquote{Future}}}}} will
be returned if it is omitted.

\DUrole{versionmodified,changed}{Changed in version 6.0: }The \sphinxcode{\sphinxupquote{callback}} and \sphinxcode{\sphinxupquote{streaming\_callback}} arguments have
been removed. Use the returned {\hyperref[\detokenize{concurrent:tornado.concurrent.Future}]{\sphinxcrossref{\sphinxcode{\sphinxupquote{Future}}}}} (and {\hyperref[\detokenize{iostream:tornado.iostream.BaseIOStream.read_bytes}]{\sphinxcrossref{\sphinxcode{\sphinxupquote{read\_bytes}}}}}
with \sphinxcode{\sphinxupquote{partial=True}} for \sphinxcode{\sphinxupquote{streaming\_callback}}) instead.

\end{fulllineitems}

\index{close() (tornado.iostream.BaseIOStream method)@\spxentry{close()}\spxextra{tornado.iostream.BaseIOStream method}}

\begin{fulllineitems}
\phantomsection\label{\detokenize{iostream:tornado.iostream.BaseIOStream.close}}\pysiglinewithargsret{\sphinxcode{\sphinxupquote{BaseIOStream.}}\sphinxbfcode{\sphinxupquote{close}}}{\emph{exc\_info: Union{[}None, bool, BaseException, Tuple{[}Optional{[}Type{[}BaseException{]}{]}, Optional{[}BaseException{]}, Optional{[}traceback{]}{]}{]} = False}}{{ $\rightarrow$ None}}
Close this stream.

If \sphinxcode{\sphinxupquote{exc\_info}} is true, set the \sphinxcode{\sphinxupquote{error}} attribute to the current
exception from \sphinxhref{https://docs.python.org/3.6/library/sys.html\#sys.exc\_info}{\sphinxcode{\sphinxupquote{sys.exc\_info}}} (or if \sphinxcode{\sphinxupquote{exc\_info}} is a tuple,
use that instead of \sphinxhref{https://docs.python.org/3.6/library/sys.html\#sys.exc\_info}{\sphinxcode{\sphinxupquote{sys.exc\_info}}}).

\end{fulllineitems}

\index{set\_close\_callback() (tornado.iostream.BaseIOStream method)@\spxentry{set\_close\_callback()}\spxextra{tornado.iostream.BaseIOStream method}}

\begin{fulllineitems}
\phantomsection\label{\detokenize{iostream:tornado.iostream.BaseIOStream.set_close_callback}}\pysiglinewithargsret{\sphinxcode{\sphinxupquote{BaseIOStream.}}\sphinxbfcode{\sphinxupquote{set\_close\_callback}}}{\emph{callback: Optional{[}Callable{[}{[}{]}, None{]}{]}}}{{ $\rightarrow$ None}}
Call the given callback when the stream is closed.

This mostly is not necessary for applications that use the
{\hyperref[\detokenize{concurrent:tornado.concurrent.Future}]{\sphinxcrossref{\sphinxcode{\sphinxupquote{Future}}}}} interface; all outstanding \sphinxcode{\sphinxupquote{Futures}} will resolve
with a {\hyperref[\detokenize{iostream:tornado.iostream.StreamClosedError}]{\sphinxcrossref{\sphinxcode{\sphinxupquote{StreamClosedError}}}}} when the stream is closed. However,
it is still useful as a way to signal that the stream has been
closed while no other read or write is in progress.

Unlike other callback-based interfaces, \sphinxcode{\sphinxupquote{set\_close\_callback}}
was not removed in Tornado 6.0.

\end{fulllineitems}

\index{closed() (tornado.iostream.BaseIOStream method)@\spxentry{closed()}\spxextra{tornado.iostream.BaseIOStream method}}

\begin{fulllineitems}
\phantomsection\label{\detokenize{iostream:tornado.iostream.BaseIOStream.closed}}\pysiglinewithargsret{\sphinxcode{\sphinxupquote{BaseIOStream.}}\sphinxbfcode{\sphinxupquote{closed}}}{}{{ $\rightarrow$ bool}}
Returns \sphinxcode{\sphinxupquote{True}} if the stream has been closed.

\end{fulllineitems}

\index{reading() (tornado.iostream.BaseIOStream method)@\spxentry{reading()}\spxextra{tornado.iostream.BaseIOStream method}}

\begin{fulllineitems}
\phantomsection\label{\detokenize{iostream:tornado.iostream.BaseIOStream.reading}}\pysiglinewithargsret{\sphinxcode{\sphinxupquote{BaseIOStream.}}\sphinxbfcode{\sphinxupquote{reading}}}{}{{ $\rightarrow$ bool}}
Returns \sphinxcode{\sphinxupquote{True}} if we are currently reading from the stream.

\end{fulllineitems}

\index{writing() (tornado.iostream.BaseIOStream method)@\spxentry{writing()}\spxextra{tornado.iostream.BaseIOStream method}}

\begin{fulllineitems}
\phantomsection\label{\detokenize{iostream:tornado.iostream.BaseIOStream.writing}}\pysiglinewithargsret{\sphinxcode{\sphinxupquote{BaseIOStream.}}\sphinxbfcode{\sphinxupquote{writing}}}{}{{ $\rightarrow$ bool}}
Returns \sphinxcode{\sphinxupquote{True}} if we are currently writing to the stream.

\end{fulllineitems}

\index{set\_nodelay() (tornado.iostream.BaseIOStream method)@\spxentry{set\_nodelay()}\spxextra{tornado.iostream.BaseIOStream method}}

\begin{fulllineitems}
\phantomsection\label{\detokenize{iostream:tornado.iostream.BaseIOStream.set_nodelay}}\pysiglinewithargsret{\sphinxcode{\sphinxupquote{BaseIOStream.}}\sphinxbfcode{\sphinxupquote{set\_nodelay}}}{\emph{value: bool}}{{ $\rightarrow$ None}}
Sets the no-delay flag for this stream.

By default, data written to TCP streams may be held for a time
to make the most efficient use of bandwidth (according to
Nagle’s algorithm).  The no-delay flag requests that data be
written as soon as possible, even if doing so would consume
additional bandwidth.

This flag is currently defined only for TCP-based \sphinxcode{\sphinxupquote{IOStreams}}.

\DUrole{versionmodified,added}{New in version 3.1.}

\end{fulllineitems}



\paragraph{Methods for subclasses}
\label{\detokenize{iostream:methods-for-subclasses}}\index{fileno() (tornado.iostream.BaseIOStream method)@\spxentry{fileno()}\spxextra{tornado.iostream.BaseIOStream method}}

\begin{fulllineitems}
\phantomsection\label{\detokenize{iostream:tornado.iostream.BaseIOStream.fileno}}\pysiglinewithargsret{\sphinxcode{\sphinxupquote{BaseIOStream.}}\sphinxbfcode{\sphinxupquote{fileno}}}{}{{ $\rightarrow$ Union{[}int, tornado.ioloop.\_Selectable{]}}}
Returns the file descriptor for this stream.

\end{fulllineitems}

\index{close\_fd() (tornado.iostream.BaseIOStream method)@\spxentry{close\_fd()}\spxextra{tornado.iostream.BaseIOStream method}}

\begin{fulllineitems}
\phantomsection\label{\detokenize{iostream:tornado.iostream.BaseIOStream.close_fd}}\pysiglinewithargsret{\sphinxcode{\sphinxupquote{BaseIOStream.}}\sphinxbfcode{\sphinxupquote{close\_fd}}}{}{{ $\rightarrow$ None}}
Closes the file underlying this stream.

\sphinxcode{\sphinxupquote{close\_fd}} is called by {\hyperref[\detokenize{iostream:tornado.iostream.BaseIOStream}]{\sphinxcrossref{\sphinxcode{\sphinxupquote{BaseIOStream}}}}} and should not be called
elsewhere; other users should call {\hyperref[\detokenize{iostream:tornado.iostream.BaseIOStream.close}]{\sphinxcrossref{\sphinxcode{\sphinxupquote{close}}}}} instead.

\end{fulllineitems}

\index{write\_to\_fd() (tornado.iostream.BaseIOStream method)@\spxentry{write\_to\_fd()}\spxextra{tornado.iostream.BaseIOStream method}}

\begin{fulllineitems}
\phantomsection\label{\detokenize{iostream:tornado.iostream.BaseIOStream.write_to_fd}}\pysiglinewithargsret{\sphinxcode{\sphinxupquote{BaseIOStream.}}\sphinxbfcode{\sphinxupquote{write\_to\_fd}}}{\emph{data: memoryview}}{{ $\rightarrow$ int}}
Attempts to write \sphinxcode{\sphinxupquote{data}} to the underlying file.

Returns the number of bytes written.

\end{fulllineitems}

\index{read\_from\_fd() (tornado.iostream.BaseIOStream method)@\spxentry{read\_from\_fd()}\spxextra{tornado.iostream.BaseIOStream method}}

\begin{fulllineitems}
\phantomsection\label{\detokenize{iostream:tornado.iostream.BaseIOStream.read_from_fd}}\pysiglinewithargsret{\sphinxcode{\sphinxupquote{BaseIOStream.}}\sphinxbfcode{\sphinxupquote{read\_from\_fd}}}{\emph{buf: Union{[}bytearray, memoryview{]}}}{{ $\rightarrow$ Optional{[}int{]}}}
Attempts to read from the underlying file.

Reads up to \sphinxcode{\sphinxupquote{len(buf)}} bytes, storing them in the buffer.
Returns the number of bytes read. Returns None if there was
nothing to read (the socket returned \sphinxhref{https://docs.python.org/3.6/library/errno.html\#errno.EWOULDBLOCK}{\sphinxcode{\sphinxupquote{EWOULDBLOCK}}} or
equivalent), and zero on EOF.

\DUrole{versionmodified,changed}{Changed in version 5.0: }Interface redesigned to take a buffer and return a number
of bytes instead of a freshly-allocated object.

\end{fulllineitems}

\index{get\_fd\_error() (tornado.iostream.BaseIOStream method)@\spxentry{get\_fd\_error()}\spxextra{tornado.iostream.BaseIOStream method}}

\begin{fulllineitems}
\phantomsection\label{\detokenize{iostream:tornado.iostream.BaseIOStream.get_fd_error}}\pysiglinewithargsret{\sphinxcode{\sphinxupquote{BaseIOStream.}}\sphinxbfcode{\sphinxupquote{get\_fd\_error}}}{}{{ $\rightarrow$ Optional{[}Exception{]}}}
Returns information about any error on the underlying file.

This method is called after the {\hyperref[\detokenize{ioloop:tornado.ioloop.IOLoop}]{\sphinxcrossref{\sphinxcode{\sphinxupquote{IOLoop}}}}} has signaled an error on the
file descriptor, and should return an Exception (such as \sphinxhref{https://docs.python.org/3.6/library/socket.html\#socket.error}{\sphinxcode{\sphinxupquote{socket.error}}}
with additional information, or None if no such information is
available.

\end{fulllineitems}



\subsubsection{Implementations}
\label{\detokenize{iostream:implementations}}\index{IOStream (class in tornado.iostream)@\spxentry{IOStream}\spxextra{class in tornado.iostream}}

\begin{fulllineitems}
\phantomsection\label{\detokenize{iostream:tornado.iostream.IOStream}}\pysiglinewithargsret{\sphinxbfcode{\sphinxupquote{class }}\sphinxcode{\sphinxupquote{tornado.iostream.}}\sphinxbfcode{\sphinxupquote{IOStream}}}{\emph{socket: socket.socket}, \emph{*args}, \emph{**kwargs}}{}
Socket-based {\hyperref[\detokenize{iostream:tornado.iostream.IOStream}]{\sphinxcrossref{\sphinxcode{\sphinxupquote{IOStream}}}}} implementation.

This class supports the read and write methods from {\hyperref[\detokenize{iostream:tornado.iostream.BaseIOStream}]{\sphinxcrossref{\sphinxcode{\sphinxupquote{BaseIOStream}}}}}
plus a {\hyperref[\detokenize{iostream:tornado.iostream.IOStream.connect}]{\sphinxcrossref{\sphinxcode{\sphinxupquote{connect}}}}} method.

The \sphinxcode{\sphinxupquote{socket}} parameter may either be connected or unconnected.
For server operations the socket is the result of calling
\sphinxhref{https://docs.python.org/3.6/library/socket.html\#socket.socket.accept}{\sphinxcode{\sphinxupquote{socket.accept}}}.  For client operations the
socket is created with \sphinxhref{https://docs.python.org/3.6/library/socket.html\#socket.socket}{\sphinxcode{\sphinxupquote{socket.socket}}}, and may either be
connected before passing it to the {\hyperref[\detokenize{iostream:tornado.iostream.IOStream}]{\sphinxcrossref{\sphinxcode{\sphinxupquote{IOStream}}}}} or connected with
{\hyperref[\detokenize{iostream:tornado.iostream.IOStream.connect}]{\sphinxcrossref{\sphinxcode{\sphinxupquote{IOStream.connect}}}}}.

A very simple (and broken) HTTP client using this class:

\begin{sphinxVerbatim}[commandchars=\\\{\}]
\PYG{k+kn}{import} \PYG{n+nn}{tornado}\PYG{n+nn}{.}\PYG{n+nn}{ioloop}
\PYG{k+kn}{import} \PYG{n+nn}{tornado}\PYG{n+nn}{.}\PYG{n+nn}{iostream}
\PYG{k+kn}{import} \PYG{n+nn}{socket}

\PYG{k}{async} \PYG{k}{def} \PYG{n+nf}{main}\PYG{p}{(}\PYG{p}{)}\PYG{p}{:}
    \PYG{n}{s} \PYG{o}{=} \PYG{n}{socket}\PYG{o}{.}\PYG{n}{socket}\PYG{p}{(}\PYG{n}{socket}\PYG{o}{.}\PYG{n}{AF\PYGZus{}INET}\PYG{p}{,} \PYG{n}{socket}\PYG{o}{.}\PYG{n}{SOCK\PYGZus{}STREAM}\PYG{p}{,} \PYG{l+m+mi}{0}\PYG{p}{)}
    \PYG{n}{stream} \PYG{o}{=} \PYG{n}{tornado}\PYG{o}{.}\PYG{n}{iostream}\PYG{o}{.}\PYG{n}{IOStream}\PYG{p}{(}\PYG{n}{s}\PYG{p}{)}
    \PYG{k}{await} \PYG{n}{stream}\PYG{o}{.}\PYG{n}{connect}\PYG{p}{(}\PYG{p}{(}\PYG{l+s+s2}{\PYGZdq{}}\PYG{l+s+s2}{friendfeed.com}\PYG{l+s+s2}{\PYGZdq{}}\PYG{p}{,} \PYG{l+m+mi}{80}\PYG{p}{)}\PYG{p}{)}
    \PYG{k}{await} \PYG{n}{stream}\PYG{o}{.}\PYG{n}{write}\PYG{p}{(}\PYG{l+s+sa}{b}\PYG{l+s+s2}{\PYGZdq{}}\PYG{l+s+s2}{GET / HTTP/1.0}\PYG{l+s+se}{\PYGZbs{}r}\PYG{l+s+se}{\PYGZbs{}n}\PYG{l+s+s2}{Host: friendfeed.com}\PYG{l+s+se}{\PYGZbs{}r}\PYG{l+s+se}{\PYGZbs{}n}\PYG{l+s+se}{\PYGZbs{}r}\PYG{l+s+se}{\PYGZbs{}n}\PYG{l+s+s2}{\PYGZdq{}}\PYG{p}{)}
    \PYG{n}{header\PYGZus{}data} \PYG{o}{=} \PYG{k}{await} \PYG{n}{stream}\PYG{o}{.}\PYG{n}{read\PYGZus{}until}\PYG{p}{(}\PYG{l+s+sa}{b}\PYG{l+s+s2}{\PYGZdq{}}\PYG{l+s+se}{\PYGZbs{}r}\PYG{l+s+se}{\PYGZbs{}n}\PYG{l+s+se}{\PYGZbs{}r}\PYG{l+s+se}{\PYGZbs{}n}\PYG{l+s+s2}{\PYGZdq{}}\PYG{p}{)}
    \PYG{n}{headers} \PYG{o}{=} \PYG{p}{\PYGZob{}}\PYG{p}{\PYGZcb{}}
    \PYG{k}{for} \PYG{n}{line} \PYG{o+ow}{in} \PYG{n}{header\PYGZus{}data}\PYG{o}{.}\PYG{n}{split}\PYG{p}{(}\PYG{l+s+sa}{b}\PYG{l+s+s2}{\PYGZdq{}}\PYG{l+s+se}{\PYGZbs{}r}\PYG{l+s+se}{\PYGZbs{}n}\PYG{l+s+s2}{\PYGZdq{}}\PYG{p}{)}\PYG{p}{:}
        \PYG{n}{parts} \PYG{o}{=} \PYG{n}{line}\PYG{o}{.}\PYG{n}{split}\PYG{p}{(}\PYG{l+s+sa}{b}\PYG{l+s+s2}{\PYGZdq{}}\PYG{l+s+s2}{:}\PYG{l+s+s2}{\PYGZdq{}}\PYG{p}{)}
        \PYG{k}{if} \PYG{n+nb}{len}\PYG{p}{(}\PYG{n}{parts}\PYG{p}{)} \PYG{o}{==} \PYG{l+m+mi}{2}\PYG{p}{:}
            \PYG{n}{headers}\PYG{p}{[}\PYG{n}{parts}\PYG{p}{[}\PYG{l+m+mi}{0}\PYG{p}{]}\PYG{o}{.}\PYG{n}{strip}\PYG{p}{(}\PYG{p}{)}\PYG{p}{]} \PYG{o}{=} \PYG{n}{parts}\PYG{p}{[}\PYG{l+m+mi}{1}\PYG{p}{]}\PYG{o}{.}\PYG{n}{strip}\PYG{p}{(}\PYG{p}{)}
    \PYG{n}{body\PYGZus{}data} \PYG{o}{=} \PYG{k}{await} \PYG{n}{stream}\PYG{o}{.}\PYG{n}{read\PYGZus{}bytes}\PYG{p}{(}\PYG{n+nb}{int}\PYG{p}{(}\PYG{n}{headers}\PYG{p}{[}\PYG{l+s+sa}{b}\PYG{l+s+s2}{\PYGZdq{}}\PYG{l+s+s2}{Content\PYGZhy{}Length}\PYG{l+s+s2}{\PYGZdq{}}\PYG{p}{]}\PYG{p}{)}\PYG{p}{)}
    \PYG{n+nb}{print}\PYG{p}{(}\PYG{n}{body\PYGZus{}data}\PYG{p}{)}
    \PYG{n}{stream}\PYG{o}{.}\PYG{n}{close}\PYG{p}{(}\PYG{p}{)}

\PYG{k}{if} \PYG{n+nv+vm}{\PYGZus{}\PYGZus{}name\PYGZus{}\PYGZus{}} \PYG{o}{==} \PYG{l+s+s1}{\PYGZsq{}}\PYG{l+s+s1}{\PYGZus{}\PYGZus{}main\PYGZus{}\PYGZus{}}\PYG{l+s+s1}{\PYGZsq{}}\PYG{p}{:}
    \PYG{n}{tornado}\PYG{o}{.}\PYG{n}{ioloop}\PYG{o}{.}\PYG{n}{IOLoop}\PYG{o}{.}\PYG{n}{current}\PYG{p}{(}\PYG{p}{)}\PYG{o}{.}\PYG{n}{run\PYGZus{}sync}\PYG{p}{(}\PYG{n}{main}\PYG{p}{)}
    \PYG{n}{s} \PYG{o}{=} \PYG{n}{socket}\PYG{o}{.}\PYG{n}{socket}\PYG{p}{(}\PYG{n}{socket}\PYG{o}{.}\PYG{n}{AF\PYGZus{}INET}\PYG{p}{,} \PYG{n}{socket}\PYG{o}{.}\PYG{n}{SOCK\PYGZus{}STREAM}\PYG{p}{,} \PYG{l+m+mi}{0}\PYG{p}{)}
    \PYG{n}{stream} \PYG{o}{=} \PYG{n}{tornado}\PYG{o}{.}\PYG{n}{iostream}\PYG{o}{.}\PYG{n}{IOStream}\PYG{p}{(}\PYG{n}{s}\PYG{p}{)}
    \PYG{n}{stream}\PYG{o}{.}\PYG{n}{connect}\PYG{p}{(}\PYG{p}{(}\PYG{l+s+s2}{\PYGZdq{}}\PYG{l+s+s2}{friendfeed.com}\PYG{l+s+s2}{\PYGZdq{}}\PYG{p}{,} \PYG{l+m+mi}{80}\PYG{p}{)}\PYG{p}{,} \PYG{n}{send\PYGZus{}request}\PYG{p}{)}
    \PYG{n}{tornado}\PYG{o}{.}\PYG{n}{ioloop}\PYG{o}{.}\PYG{n}{IOLoop}\PYG{o}{.}\PYG{n}{current}\PYG{p}{(}\PYG{p}{)}\PYG{o}{.}\PYG{n}{start}\PYG{p}{(}\PYG{p}{)}
\end{sphinxVerbatim}
\index{connect() (tornado.iostream.IOStream method)@\spxentry{connect()}\spxextra{tornado.iostream.IOStream method}}

\begin{fulllineitems}
\phantomsection\label{\detokenize{iostream:tornado.iostream.IOStream.connect}}\pysiglinewithargsret{\sphinxbfcode{\sphinxupquote{connect}}}{\emph{address: tuple}, \emph{server\_hostname: str = None}}{{ $\rightarrow$ Future{[}\_IOStreamType{]}}}
Connects the socket to a remote address without blocking.

May only be called if the socket passed to the constructor was
not previously connected.  The address parameter is in the
same format as for \sphinxhref{https://docs.python.org/3.6/library/socket.html\#socket.socket.connect}{\sphinxcode{\sphinxupquote{socket.connect}}} for
the type of socket passed to the IOStream constructor,
e.g. an \sphinxcode{\sphinxupquote{(ip, port)}} tuple.  Hostnames are accepted here,
but will be resolved synchronously and block the IOLoop.
If you have a hostname instead of an IP address, the {\hyperref[\detokenize{tcpclient:tornado.tcpclient.TCPClient}]{\sphinxcrossref{\sphinxcode{\sphinxupquote{TCPClient}}}}}
class is recommended instead of calling this method directly.
{\hyperref[\detokenize{tcpclient:tornado.tcpclient.TCPClient}]{\sphinxcrossref{\sphinxcode{\sphinxupquote{TCPClient}}}}} will do asynchronous DNS resolution and handle
both IPv4 and IPv6.

If \sphinxcode{\sphinxupquote{callback}} is specified, it will be called with no
arguments when the connection is completed; if not this method
returns a {\hyperref[\detokenize{concurrent:tornado.concurrent.Future}]{\sphinxcrossref{\sphinxcode{\sphinxupquote{Future}}}}} (whose result after a successful
connection will be the stream itself).

In SSL mode, the \sphinxcode{\sphinxupquote{server\_hostname}} parameter will be used
for certificate validation (unless disabled in the
\sphinxcode{\sphinxupquote{ssl\_options}}) and SNI (if supported; requires Python
2.7.9+).

Note that it is safe to call {\hyperref[\detokenize{iostream:tornado.iostream.BaseIOStream.write}]{\sphinxcrossref{\sphinxcode{\sphinxupquote{IOStream.write}}}}} while the connection is pending, in
which case the data will be written as soon as the connection
is ready.  Calling {\hyperref[\detokenize{iostream:tornado.iostream.IOStream}]{\sphinxcrossref{\sphinxcode{\sphinxupquote{IOStream}}}}} read methods before the socket is
connected works on some platforms but is non-portable.

\DUrole{versionmodified,changed}{Changed in version 4.0: }If no callback is given, returns a {\hyperref[\detokenize{concurrent:tornado.concurrent.Future}]{\sphinxcrossref{\sphinxcode{\sphinxupquote{Future}}}}}.

\DUrole{versionmodified,changed}{Changed in version 4.2: }SSL certificates are validated by default; pass
\sphinxcode{\sphinxupquote{ssl\_options=dict(cert\_reqs=ssl.CERT\_NONE)}} or a
suitably-configured \sphinxhref{https://docs.python.org/3.6/library/ssl.html\#ssl.SSLContext}{\sphinxcode{\sphinxupquote{ssl.SSLContext}}} to the
{\hyperref[\detokenize{iostream:tornado.iostream.SSLIOStream}]{\sphinxcrossref{\sphinxcode{\sphinxupquote{SSLIOStream}}}}} constructor to disable.

\DUrole{versionmodified,changed}{Changed in version 6.0: }The \sphinxcode{\sphinxupquote{callback}} argument was removed. Use the returned
{\hyperref[\detokenize{concurrent:tornado.concurrent.Future}]{\sphinxcrossref{\sphinxcode{\sphinxupquote{Future}}}}} instead.

\end{fulllineitems}

\index{start\_tls() (tornado.iostream.IOStream method)@\spxentry{start\_tls()}\spxextra{tornado.iostream.IOStream method}}

\begin{fulllineitems}
\phantomsection\label{\detokenize{iostream:tornado.iostream.IOStream.start_tls}}\pysiglinewithargsret{\sphinxbfcode{\sphinxupquote{start\_tls}}}{\emph{server\_side: bool, ssl\_options: Union{[}Dict{[}str, Any{]}, ssl.SSLContext{]} = None, server\_hostname: str = None}}{{ $\rightarrow$ Awaitable{[}tornado.iostream.SSLIOStream{]}}}
Convert this {\hyperref[\detokenize{iostream:tornado.iostream.IOStream}]{\sphinxcrossref{\sphinxcode{\sphinxupquote{IOStream}}}}} to an {\hyperref[\detokenize{iostream:tornado.iostream.SSLIOStream}]{\sphinxcrossref{\sphinxcode{\sphinxupquote{SSLIOStream}}}}}.

This enables protocols that begin in clear-text mode and
switch to SSL after some initial negotiation (such as the
\sphinxcode{\sphinxupquote{STARTTLS}} extension to SMTP and IMAP).

This method cannot be used if there are outstanding reads
or writes on the stream, or if there is any data in the
IOStream’s buffer (data in the operating system’s socket
buffer is allowed).  This means it must generally be used
immediately after reading or writing the last clear-text
data.  It can also be used immediately after connecting,
before any reads or writes.

The \sphinxcode{\sphinxupquote{ssl\_options}} argument may be either an \sphinxhref{https://docs.python.org/3.6/library/ssl.html\#ssl.SSLContext}{\sphinxcode{\sphinxupquote{ssl.SSLContext}}}
object or a dictionary of keyword arguments for the
\sphinxhref{https://docs.python.org/3.6/library/ssl.html\#ssl.wrap\_socket}{\sphinxcode{\sphinxupquote{ssl.wrap\_socket}}} function.  The \sphinxcode{\sphinxupquote{server\_hostname}} argument
will be used for certificate validation unless disabled
in the \sphinxcode{\sphinxupquote{ssl\_options}}.

This method returns a {\hyperref[\detokenize{concurrent:tornado.concurrent.Future}]{\sphinxcrossref{\sphinxcode{\sphinxupquote{Future}}}}} whose result is the new
{\hyperref[\detokenize{iostream:tornado.iostream.SSLIOStream}]{\sphinxcrossref{\sphinxcode{\sphinxupquote{SSLIOStream}}}}}.  After this method has been called,
any other operation on the original stream is undefined.

If a close callback is defined on this stream, it will be
transferred to the new stream.

\DUrole{versionmodified,added}{New in version 4.0.}

\DUrole{versionmodified,changed}{Changed in version 4.2: }SSL certificates are validated by default; pass
\sphinxcode{\sphinxupquote{ssl\_options=dict(cert\_reqs=ssl.CERT\_NONE)}} or a
suitably-configured \sphinxhref{https://docs.python.org/3.6/library/ssl.html\#ssl.SSLContext}{\sphinxcode{\sphinxupquote{ssl.SSLContext}}} to disable.

\end{fulllineitems}


\end{fulllineitems}

\index{SSLIOStream (class in tornado.iostream)@\spxentry{SSLIOStream}\spxextra{class in tornado.iostream}}

\begin{fulllineitems}
\phantomsection\label{\detokenize{iostream:tornado.iostream.SSLIOStream}}\pysiglinewithargsret{\sphinxbfcode{\sphinxupquote{class }}\sphinxcode{\sphinxupquote{tornado.iostream.}}\sphinxbfcode{\sphinxupquote{SSLIOStream}}}{\emph{*args}, \emph{**kwargs}}{}
A utility class to write to and read from a non-blocking SSL socket.

If the socket passed to the constructor is already connected,
it should be wrapped with:

\begin{sphinxVerbatim}[commandchars=\\\{\}]
\PYG{n}{ssl}\PYG{o}{.}\PYG{n}{wrap\PYGZus{}socket}\PYG{p}{(}\PYG{n}{sock}\PYG{p}{,} \PYG{n}{do\PYGZus{}handshake\PYGZus{}on\PYGZus{}connect}\PYG{o}{=}\PYG{k+kc}{False}\PYG{p}{,} \PYG{o}{*}\PYG{o}{*}\PYG{n}{kwargs}\PYG{p}{)}
\end{sphinxVerbatim}

before constructing the {\hyperref[\detokenize{iostream:tornado.iostream.SSLIOStream}]{\sphinxcrossref{\sphinxcode{\sphinxupquote{SSLIOStream}}}}}.  Unconnected sockets will be
wrapped when {\hyperref[\detokenize{iostream:tornado.iostream.IOStream.connect}]{\sphinxcrossref{\sphinxcode{\sphinxupquote{IOStream.connect}}}}} is finished.

The \sphinxcode{\sphinxupquote{ssl\_options}} keyword argument may either be an
\sphinxhref{https://docs.python.org/3.6/library/ssl.html\#ssl.SSLContext}{\sphinxcode{\sphinxupquote{ssl.SSLContext}}} object or a dictionary of keywords arguments
for \sphinxhref{https://docs.python.org/3.6/library/ssl.html\#ssl.wrap\_socket}{\sphinxcode{\sphinxupquote{ssl.wrap\_socket}}}
\index{wait\_for\_handshake() (tornado.iostream.SSLIOStream method)@\spxentry{wait\_for\_handshake()}\spxextra{tornado.iostream.SSLIOStream method}}

\begin{fulllineitems}
\phantomsection\label{\detokenize{iostream:tornado.iostream.SSLIOStream.wait_for_handshake}}\pysiglinewithargsret{\sphinxbfcode{\sphinxupquote{wait\_for\_handshake}}}{}{{ $\rightarrow$ Future{[}SSLIOStream{]}}}
Wait for the initial SSL handshake to complete.

If a \sphinxcode{\sphinxupquote{callback}} is given, it will be called with no
arguments once the handshake is complete; otherwise this
method returns a {\hyperref[\detokenize{concurrent:tornado.concurrent.Future}]{\sphinxcrossref{\sphinxcode{\sphinxupquote{Future}}}}} which will resolve to the
stream itself after the handshake is complete.

Once the handshake is complete, information such as
the peer’s certificate and NPN/ALPN selections may be
accessed on \sphinxcode{\sphinxupquote{self.socket}}.

This method is intended for use on server-side streams
or after using {\hyperref[\detokenize{iostream:tornado.iostream.IOStream.start_tls}]{\sphinxcrossref{\sphinxcode{\sphinxupquote{IOStream.start\_tls}}}}}; it should not be used
with {\hyperref[\detokenize{iostream:tornado.iostream.IOStream.connect}]{\sphinxcrossref{\sphinxcode{\sphinxupquote{IOStream.connect}}}}} (which already waits for the
handshake to complete). It may only be called once per stream.

\DUrole{versionmodified,added}{New in version 4.2.}

\DUrole{versionmodified,changed}{Changed in version 6.0: }The \sphinxcode{\sphinxupquote{callback}} argument was removed. Use the returned
{\hyperref[\detokenize{concurrent:tornado.concurrent.Future}]{\sphinxcrossref{\sphinxcode{\sphinxupquote{Future}}}}} instead.

\end{fulllineitems}


\end{fulllineitems}

\index{PipeIOStream (class in tornado.iostream)@\spxentry{PipeIOStream}\spxextra{class in tornado.iostream}}

\begin{fulllineitems}
\phantomsection\label{\detokenize{iostream:tornado.iostream.PipeIOStream}}\pysiglinewithargsret{\sphinxbfcode{\sphinxupquote{class }}\sphinxcode{\sphinxupquote{tornado.iostream.}}\sphinxbfcode{\sphinxupquote{PipeIOStream}}}{\emph{fd: int}, \emph{*args}, \emph{**kwargs}}{}
Pipe-based {\hyperref[\detokenize{iostream:tornado.iostream.IOStream}]{\sphinxcrossref{\sphinxcode{\sphinxupquote{IOStream}}}}} implementation.

The constructor takes an integer file descriptor (such as one returned
by \sphinxhref{https://docs.python.org/3.6/library/os.html\#os.pipe}{\sphinxcode{\sphinxupquote{os.pipe}}}) rather than an open file object.  Pipes are generally
one-way, so a {\hyperref[\detokenize{iostream:tornado.iostream.PipeIOStream}]{\sphinxcrossref{\sphinxcode{\sphinxupquote{PipeIOStream}}}}} can be used for reading or writing but not
both.

\end{fulllineitems}



\subsubsection{Exceptions}
\label{\detokenize{iostream:exceptions}}\index{StreamBufferFullError@\spxentry{StreamBufferFullError}}

\begin{fulllineitems}
\phantomsection\label{\detokenize{iostream:tornado.iostream.StreamBufferFullError}}\pysigline{\sphinxbfcode{\sphinxupquote{exception }}\sphinxcode{\sphinxupquote{tornado.iostream.}}\sphinxbfcode{\sphinxupquote{StreamBufferFullError}}}
Exception raised by {\hyperref[\detokenize{iostream:tornado.iostream.IOStream}]{\sphinxcrossref{\sphinxcode{\sphinxupquote{IOStream}}}}} methods when the buffer is full.

\end{fulllineitems}

\index{StreamClosedError@\spxentry{StreamClosedError}}

\begin{fulllineitems}
\phantomsection\label{\detokenize{iostream:tornado.iostream.StreamClosedError}}\pysiglinewithargsret{\sphinxbfcode{\sphinxupquote{exception }}\sphinxcode{\sphinxupquote{tornado.iostream.}}\sphinxbfcode{\sphinxupquote{StreamClosedError}}}{\emph{real\_error: BaseException = None}}{}
Exception raised by {\hyperref[\detokenize{iostream:tornado.iostream.IOStream}]{\sphinxcrossref{\sphinxcode{\sphinxupquote{IOStream}}}}} methods when the stream is closed.

Note that the close callback is scheduled to run \sphinxstyleemphasis{after} other
callbacks on the stream (to allow for buffered data to be processed),
so you may see this error before you see the close callback.

The \sphinxcode{\sphinxupquote{real\_error}} attribute contains the underlying error that caused
the stream to close (if any).

\DUrole{versionmodified,changed}{Changed in version 4.3: }Added the \sphinxcode{\sphinxupquote{real\_error}} attribute.

\end{fulllineitems}

\index{UnsatisfiableReadError@\spxentry{UnsatisfiableReadError}}

\begin{fulllineitems}
\phantomsection\label{\detokenize{iostream:tornado.iostream.UnsatisfiableReadError}}\pysigline{\sphinxbfcode{\sphinxupquote{exception }}\sphinxcode{\sphinxupquote{tornado.iostream.}}\sphinxbfcode{\sphinxupquote{UnsatisfiableReadError}}}
Exception raised when a read cannot be satisfied.

Raised by \sphinxcode{\sphinxupquote{read\_until}} and \sphinxcode{\sphinxupquote{read\_until\_regex}} with a \sphinxcode{\sphinxupquote{max\_bytes}}
argument.

\end{fulllineitems}



\subsection{\sphinxstyleliteralintitle{\sphinxupquote{tornado.netutil}} — Miscellaneous network utilities}
\label{\detokenize{netutil:module-tornado.netutil}}\label{\detokenize{netutil:tornado-netutil-miscellaneous-network-utilities}}\label{\detokenize{netutil::doc}}\index{tornado.netutil (module)@\spxentry{tornado.netutil}\spxextra{module}}
Miscellaneous network utility code.
\index{bind\_sockets() (in module tornado.netutil)@\spxentry{bind\_sockets()}\spxextra{in module tornado.netutil}}

\begin{fulllineitems}
\phantomsection\label{\detokenize{netutil:tornado.netutil.bind_sockets}}\pysiglinewithargsret{\sphinxcode{\sphinxupquote{tornado.netutil.}}\sphinxbfcode{\sphinxupquote{bind\_sockets}}}{\emph{port: int}, \emph{address: str = None}, \emph{family: socket.AddressFamily = \textless{}AddressFamily.AF\_UNSPEC: 0\textgreater{}}, \emph{backlog: int = 128}, \emph{flags: int = None}, \emph{reuse\_port: bool = False}}{{ $\rightarrow$ List{[}socket.socket{]}}}
Creates listening sockets bound to the given port and address.

Returns a list of socket objects (multiple sockets are returned if
the given address maps to multiple IP addresses, which is most common
for mixed IPv4 and IPv6 use).

Address may be either an IP address or hostname.  If it’s a hostname,
the server will listen on all IP addresses associated with the
name.  Address may be an empty string or None to listen on all
available interfaces.  Family may be set to either \sphinxhref{https://docs.python.org/3.6/library/socket.html\#socket.AF\_INET}{\sphinxcode{\sphinxupquote{socket.AF\_INET}}}
or \sphinxhref{https://docs.python.org/3.6/library/socket.html\#socket.AF\_INET6}{\sphinxcode{\sphinxupquote{socket.AF\_INET6}}} to restrict to IPv4 or IPv6 addresses, otherwise
both will be used if available.

The \sphinxcode{\sphinxupquote{backlog}} argument has the same meaning as for
\sphinxhref{https://docs.python.org/3.6/library/socket.html\#socket.socket.listen}{\sphinxcode{\sphinxupquote{socket.listen()}}}.

\sphinxcode{\sphinxupquote{flags}} is a bitmask of AI\_* flags to \sphinxhref{https://docs.python.org/3.6/library/socket.html\#socket.getaddrinfo}{\sphinxcode{\sphinxupquote{getaddrinfo}}}, like
\sphinxcode{\sphinxupquote{socket.AI\_PASSIVE \textbar{} socket.AI\_NUMERICHOST}}.

\sphinxcode{\sphinxupquote{reuse\_port}} option sets \sphinxcode{\sphinxupquote{SO\_REUSEPORT}} option for every socket
in the list. If your platform doesn’t support this option ValueError will
be raised.

\end{fulllineitems}

\index{bind\_unix\_socket() (in module tornado.netutil)@\spxentry{bind\_unix\_socket()}\spxextra{in module tornado.netutil}}

\begin{fulllineitems}
\phantomsection\label{\detokenize{netutil:tornado.netutil.bind_unix_socket}}\pysiglinewithargsret{\sphinxcode{\sphinxupquote{tornado.netutil.}}\sphinxbfcode{\sphinxupquote{bind\_unix\_socket}}}{\emph{file: str}, \emph{mode: int = 384}, \emph{backlog: int = 128}}{{ $\rightarrow$ socket.socket}}
Creates a listening unix socket.

If a socket with the given name already exists, it will be deleted.
If any other file with that name exists, an exception will be
raised.

Returns a socket object (not a list of socket objects like
{\hyperref[\detokenize{netutil:tornado.netutil.bind_sockets}]{\sphinxcrossref{\sphinxcode{\sphinxupquote{bind\_sockets}}}}})

\end{fulllineitems}

\index{add\_accept\_handler() (in module tornado.netutil)@\spxentry{add\_accept\_handler()}\spxextra{in module tornado.netutil}}

\begin{fulllineitems}
\phantomsection\label{\detokenize{netutil:tornado.netutil.add_accept_handler}}\pysiglinewithargsret{\sphinxcode{\sphinxupquote{tornado.netutil.}}\sphinxbfcode{\sphinxupquote{add\_accept\_handler}}}{\emph{sock: socket.socket, callback: Callable{[}{[}socket.socket, Any{]}, None{]}}}{{ $\rightarrow$ Callable{[}{[}{]}, None{]}}}
Adds an {\hyperref[\detokenize{ioloop:tornado.ioloop.IOLoop}]{\sphinxcrossref{\sphinxcode{\sphinxupquote{IOLoop}}}}} event handler to accept new connections on \sphinxcode{\sphinxupquote{sock}}.

When a connection is accepted, \sphinxcode{\sphinxupquote{callback(connection, address)}} will
be run (\sphinxcode{\sphinxupquote{connection}} is a socket object, and \sphinxcode{\sphinxupquote{address}} is the
address of the other end of the connection).  Note that this signature
is different from the \sphinxcode{\sphinxupquote{callback(fd, events)}} signature used for
{\hyperref[\detokenize{ioloop:tornado.ioloop.IOLoop}]{\sphinxcrossref{\sphinxcode{\sphinxupquote{IOLoop}}}}} handlers.

A callable is returned which, when called, will remove the {\hyperref[\detokenize{ioloop:tornado.ioloop.IOLoop}]{\sphinxcrossref{\sphinxcode{\sphinxupquote{IOLoop}}}}}
event handler and stop processing further incoming connections.

\DUrole{versionmodified,changed}{Changed in version 5.0: }The \sphinxcode{\sphinxupquote{io\_loop}} argument (deprecated since version 4.1) has been removed.

\DUrole{versionmodified,changed}{Changed in version 5.0: }A callable is returned (\sphinxcode{\sphinxupquote{None}} was returned before).

\end{fulllineitems}

\index{is\_valid\_ip() (in module tornado.netutil)@\spxentry{is\_valid\_ip()}\spxextra{in module tornado.netutil}}

\begin{fulllineitems}
\phantomsection\label{\detokenize{netutil:tornado.netutil.is_valid_ip}}\pysiglinewithargsret{\sphinxcode{\sphinxupquote{tornado.netutil.}}\sphinxbfcode{\sphinxupquote{is\_valid\_ip}}}{\emph{ip: str}}{{ $\rightarrow$ bool}}
Returns \sphinxcode{\sphinxupquote{True}} if the given string is a well-formed IP address.

Supports IPv4 and IPv6.

\end{fulllineitems}

\index{Resolver (class in tornado.netutil)@\spxentry{Resolver}\spxextra{class in tornado.netutil}}

\begin{fulllineitems}
\phantomsection\label{\detokenize{netutil:tornado.netutil.Resolver}}\pysigline{\sphinxbfcode{\sphinxupquote{class }}\sphinxcode{\sphinxupquote{tornado.netutil.}}\sphinxbfcode{\sphinxupquote{Resolver}}}
Configurable asynchronous DNS resolver interface.

By default, a blocking implementation is used (which simply calls
\sphinxhref{https://docs.python.org/3.6/library/socket.html\#socket.getaddrinfo}{\sphinxcode{\sphinxupquote{socket.getaddrinfo}}}).  An alternative implementation can be
chosen with the {\hyperref[\detokenize{util:tornado.util.Configurable.configure}]{\sphinxcrossref{\sphinxcode{\sphinxupquote{Resolver.configure}}}}}
class method:

\begin{sphinxVerbatim}[commandchars=\\\{\}]
\PYG{n}{Resolver}\PYG{o}{.}\PYG{n}{configure}\PYG{p}{(}\PYG{l+s+s1}{\PYGZsq{}}\PYG{l+s+s1}{tornado.netutil.ThreadedResolver}\PYG{l+s+s1}{\PYGZsq{}}\PYG{p}{)}
\end{sphinxVerbatim}

The implementations of this interface included with Tornado are
\begin{itemize}
\item {} 
{\hyperref[\detokenize{netutil:tornado.netutil.DefaultExecutorResolver}]{\sphinxcrossref{\sphinxcode{\sphinxupquote{tornado.netutil.DefaultExecutorResolver}}}}}

\item {} 
{\hyperref[\detokenize{netutil:tornado.netutil.BlockingResolver}]{\sphinxcrossref{\sphinxcode{\sphinxupquote{tornado.netutil.BlockingResolver}}}}} (deprecated)

\item {} 
{\hyperref[\detokenize{netutil:tornado.netutil.ThreadedResolver}]{\sphinxcrossref{\sphinxcode{\sphinxupquote{tornado.netutil.ThreadedResolver}}}}} (deprecated)

\item {} 
{\hyperref[\detokenize{netutil:tornado.netutil.OverrideResolver}]{\sphinxcrossref{\sphinxcode{\sphinxupquote{tornado.netutil.OverrideResolver}}}}}

\item {} 
{\hyperref[\detokenize{twisted:tornado.platform.twisted.TwistedResolver}]{\sphinxcrossref{\sphinxcode{\sphinxupquote{tornado.platform.twisted.TwistedResolver}}}}}

\item {} 
{\hyperref[\detokenize{caresresolver:tornado.platform.caresresolver.CaresResolver}]{\sphinxcrossref{\sphinxcode{\sphinxupquote{tornado.platform.caresresolver.CaresResolver}}}}}

\end{itemize}

\DUrole{versionmodified,changed}{Changed in version 5.0: }The default implementation has changed from {\hyperref[\detokenize{netutil:tornado.netutil.BlockingResolver}]{\sphinxcrossref{\sphinxcode{\sphinxupquote{BlockingResolver}}}}} to
{\hyperref[\detokenize{netutil:tornado.netutil.DefaultExecutorResolver}]{\sphinxcrossref{\sphinxcode{\sphinxupquote{DefaultExecutorResolver}}}}}.
\index{resolve() (tornado.netutil.Resolver method)@\spxentry{resolve()}\spxextra{tornado.netutil.Resolver method}}

\begin{fulllineitems}
\phantomsection\label{\detokenize{netutil:tornado.netutil.Resolver.resolve}}\pysiglinewithargsret{\sphinxbfcode{\sphinxupquote{resolve}}}{\emph{host: str}, \emph{port: int}, \emph{family: socket.AddressFamily = \textless{}AddressFamily.AF\_UNSPEC: 0\textgreater{}}}{{ $\rightarrow$ Awaitable{[}List{[}Tuple{[}int, Any{]}{]}{]}}}
Resolves an address.

The \sphinxcode{\sphinxupquote{host}} argument is a string which may be a hostname or a
literal IP address.

Returns a {\hyperref[\detokenize{concurrent:tornado.concurrent.Future}]{\sphinxcrossref{\sphinxcode{\sphinxupquote{Future}}}}} whose result is a list of (family,
address) pairs, where address is a tuple suitable to pass to
\sphinxhref{https://docs.python.org/3.6/library/socket.html\#socket.socket.connect}{\sphinxcode{\sphinxupquote{socket.connect}}} (i.e. a \sphinxcode{\sphinxupquote{(host,
port)}} pair for IPv4; additional fields may be present for
IPv6). If a \sphinxcode{\sphinxupquote{callback}} is passed, it will be run with the
result as an argument when it is complete.
\begin{quote}\begin{description}
\item[{Raises}] \leavevmode
\sphinxhref{https://docs.python.org/3.6/library/exceptions.html\#IOError}{\sphinxstyleliteralstrong{\sphinxupquote{IOError}}} \textendash{} if the address cannot be resolved.

\end{description}\end{quote}

\DUrole{versionmodified,changed}{Changed in version 4.4: }Standardized all implementations to raise \sphinxhref{https://docs.python.org/3.6/library/exceptions.html\#IOError}{\sphinxcode{\sphinxupquote{IOError}}}.

\DUrole{versionmodified,changed}{Changed in version 6.0: }The \sphinxcode{\sphinxupquote{callback}} argument was removed.
Use the returned awaitable object instead.

\end{fulllineitems}

\index{close() (tornado.netutil.Resolver method)@\spxentry{close()}\spxextra{tornado.netutil.Resolver method}}

\begin{fulllineitems}
\phantomsection\label{\detokenize{netutil:tornado.netutil.Resolver.close}}\pysiglinewithargsret{\sphinxbfcode{\sphinxupquote{close}}}{}{{ $\rightarrow$ None}}
Closes the {\hyperref[\detokenize{netutil:tornado.netutil.Resolver}]{\sphinxcrossref{\sphinxcode{\sphinxupquote{Resolver}}}}}, freeing any resources used.

\DUrole{versionmodified,added}{New in version 3.1.}

\end{fulllineitems}


\end{fulllineitems}

\index{DefaultExecutorResolver (class in tornado.netutil)@\spxentry{DefaultExecutorResolver}\spxextra{class in tornado.netutil}}

\begin{fulllineitems}
\phantomsection\label{\detokenize{netutil:tornado.netutil.DefaultExecutorResolver}}\pysigline{\sphinxbfcode{\sphinxupquote{class }}\sphinxcode{\sphinxupquote{tornado.netutil.}}\sphinxbfcode{\sphinxupquote{DefaultExecutorResolver}}}
Resolver implementation using {\hyperref[\detokenize{ioloop:tornado.ioloop.IOLoop.run_in_executor}]{\sphinxcrossref{\sphinxcode{\sphinxupquote{IOLoop.run\_in\_executor}}}}}.

\DUrole{versionmodified,added}{New in version 5.0.}

\end{fulllineitems}

\index{ExecutorResolver (class in tornado.netutil)@\spxentry{ExecutorResolver}\spxextra{class in tornado.netutil}}

\begin{fulllineitems}
\phantomsection\label{\detokenize{netutil:tornado.netutil.ExecutorResolver}}\pysigline{\sphinxbfcode{\sphinxupquote{class }}\sphinxcode{\sphinxupquote{tornado.netutil.}}\sphinxbfcode{\sphinxupquote{ExecutorResolver}}}
Resolver implementation using a \sphinxhref{https://docs.python.org/3.6/library/concurrent.futures.html\#concurrent.futures.Executor}{\sphinxcode{\sphinxupquote{concurrent.futures.Executor}}}.

Use this instead of {\hyperref[\detokenize{netutil:tornado.netutil.ThreadedResolver}]{\sphinxcrossref{\sphinxcode{\sphinxupquote{ThreadedResolver}}}}} when you require additional
control over the executor being used.

The executor will be shut down when the resolver is closed unless
\sphinxcode{\sphinxupquote{close\_resolver=False}}; use this if you want to reuse the same
executor elsewhere.

\DUrole{versionmodified,changed}{Changed in version 5.0: }The \sphinxcode{\sphinxupquote{io\_loop}} argument (deprecated since version 4.1) has been removed.

\DUrole{versionmodified,deprecated}{Deprecated since version 5.0: }The default {\hyperref[\detokenize{netutil:tornado.netutil.Resolver}]{\sphinxcrossref{\sphinxcode{\sphinxupquote{Resolver}}}}} now uses {\hyperref[\detokenize{ioloop:tornado.ioloop.IOLoop.run_in_executor}]{\sphinxcrossref{\sphinxcode{\sphinxupquote{IOLoop.run\_in\_executor}}}}}; use that instead
of this class.

\end{fulllineitems}

\index{BlockingResolver (class in tornado.netutil)@\spxentry{BlockingResolver}\spxextra{class in tornado.netutil}}

\begin{fulllineitems}
\phantomsection\label{\detokenize{netutil:tornado.netutil.BlockingResolver}}\pysigline{\sphinxbfcode{\sphinxupquote{class }}\sphinxcode{\sphinxupquote{tornado.netutil.}}\sphinxbfcode{\sphinxupquote{BlockingResolver}}}
Default {\hyperref[\detokenize{netutil:tornado.netutil.Resolver}]{\sphinxcrossref{\sphinxcode{\sphinxupquote{Resolver}}}}} implementation, using \sphinxhref{https://docs.python.org/3.6/library/socket.html\#socket.getaddrinfo}{\sphinxcode{\sphinxupquote{socket.getaddrinfo}}}.

The {\hyperref[\detokenize{ioloop:tornado.ioloop.IOLoop}]{\sphinxcrossref{\sphinxcode{\sphinxupquote{IOLoop}}}}} will be blocked during the resolution, although the
callback will not be run until the next {\hyperref[\detokenize{ioloop:tornado.ioloop.IOLoop}]{\sphinxcrossref{\sphinxcode{\sphinxupquote{IOLoop}}}}} iteration.

\DUrole{versionmodified,deprecated}{Deprecated since version 5.0: }The default {\hyperref[\detokenize{netutil:tornado.netutil.Resolver}]{\sphinxcrossref{\sphinxcode{\sphinxupquote{Resolver}}}}} now uses {\hyperref[\detokenize{ioloop:tornado.ioloop.IOLoop.run_in_executor}]{\sphinxcrossref{\sphinxcode{\sphinxupquote{IOLoop.run\_in\_executor}}}}}; use that instead
of this class.

\end{fulllineitems}

\index{ThreadedResolver (class in tornado.netutil)@\spxentry{ThreadedResolver}\spxextra{class in tornado.netutil}}

\begin{fulllineitems}
\phantomsection\label{\detokenize{netutil:tornado.netutil.ThreadedResolver}}\pysigline{\sphinxbfcode{\sphinxupquote{class }}\sphinxcode{\sphinxupquote{tornado.netutil.}}\sphinxbfcode{\sphinxupquote{ThreadedResolver}}}
Multithreaded non-blocking {\hyperref[\detokenize{netutil:tornado.netutil.Resolver}]{\sphinxcrossref{\sphinxcode{\sphinxupquote{Resolver}}}}} implementation.

Requires the \sphinxhref{https://docs.python.org/3.6/library/concurrent.futures.html\#module-concurrent.futures}{\sphinxcode{\sphinxupquote{concurrent.futures}}} package to be installed
(available in the standard library since Python 3.2,
installable with \sphinxcode{\sphinxupquote{pip install futures}} in older versions).

The thread pool size can be configured with:

\begin{sphinxVerbatim}[commandchars=\\\{\}]
\PYG{n}{Resolver}\PYG{o}{.}\PYG{n}{configure}\PYG{p}{(}\PYG{l+s+s1}{\PYGZsq{}}\PYG{l+s+s1}{tornado.netutil.ThreadedResolver}\PYG{l+s+s1}{\PYGZsq{}}\PYG{p}{,}
                   \PYG{n}{num\PYGZus{}threads}\PYG{o}{=}\PYG{l+m+mi}{10}\PYG{p}{)}
\end{sphinxVerbatim}

\DUrole{versionmodified,changed}{Changed in version 3.1: }All \sphinxcode{\sphinxupquote{ThreadedResolvers}} share a single thread pool, whose
size is set by the first one to be created.

\DUrole{versionmodified,deprecated}{Deprecated since version 5.0: }The default {\hyperref[\detokenize{netutil:tornado.netutil.Resolver}]{\sphinxcrossref{\sphinxcode{\sphinxupquote{Resolver}}}}} now uses {\hyperref[\detokenize{ioloop:tornado.ioloop.IOLoop.run_in_executor}]{\sphinxcrossref{\sphinxcode{\sphinxupquote{IOLoop.run\_in\_executor}}}}}; use that instead
of this class.

\end{fulllineitems}

\index{OverrideResolver (class in tornado.netutil)@\spxentry{OverrideResolver}\spxextra{class in tornado.netutil}}

\begin{fulllineitems}
\phantomsection\label{\detokenize{netutil:tornado.netutil.OverrideResolver}}\pysigline{\sphinxbfcode{\sphinxupquote{class }}\sphinxcode{\sphinxupquote{tornado.netutil.}}\sphinxbfcode{\sphinxupquote{OverrideResolver}}}
Wraps a resolver with a mapping of overrides.

This can be used to make local DNS changes (e.g. for testing)
without modifying system-wide settings.

The mapping can be in three formats:

\begin{sphinxVerbatim}[commandchars=\\\{\}]
\PYG{p}{\PYGZob{}}
    \PYG{c+c1}{\PYGZsh{} Hostname to host or ip}
    \PYG{l+s+s2}{\PYGZdq{}}\PYG{l+s+s2}{example.com}\PYG{l+s+s2}{\PYGZdq{}}\PYG{p}{:} \PYG{l+s+s2}{\PYGZdq{}}\PYG{l+s+s2}{127.0.1.1}\PYG{l+s+s2}{\PYGZdq{}}\PYG{p}{,}

    \PYG{c+c1}{\PYGZsh{} Host+port to host+port}
    \PYG{p}{(}\PYG{l+s+s2}{\PYGZdq{}}\PYG{l+s+s2}{login.example.com}\PYG{l+s+s2}{\PYGZdq{}}\PYG{p}{,} \PYG{l+m+mi}{443}\PYG{p}{)}\PYG{p}{:} \PYG{p}{(}\PYG{l+s+s2}{\PYGZdq{}}\PYG{l+s+s2}{localhost}\PYG{l+s+s2}{\PYGZdq{}}\PYG{p}{,} \PYG{l+m+mi}{1443}\PYG{p}{)}\PYG{p}{,}

    \PYG{c+c1}{\PYGZsh{} Host+port+address family to host+port}
    \PYG{p}{(}\PYG{l+s+s2}{\PYGZdq{}}\PYG{l+s+s2}{login.example.com}\PYG{l+s+s2}{\PYGZdq{}}\PYG{p}{,} \PYG{l+m+mi}{443}\PYG{p}{,} \PYG{n}{socket}\PYG{o}{.}\PYG{n}{AF\PYGZus{}INET6}\PYG{p}{)}\PYG{p}{:} \PYG{p}{(}\PYG{l+s+s2}{\PYGZdq{}}\PYG{l+s+s2}{::1}\PYG{l+s+s2}{\PYGZdq{}}\PYG{p}{,} \PYG{l+m+mi}{1443}\PYG{p}{)}\PYG{p}{,}
\PYG{p}{\PYGZcb{}}
\end{sphinxVerbatim}

\DUrole{versionmodified,changed}{Changed in version 5.0: }Added support for host-port-family triplets.

\end{fulllineitems}

\index{ssl\_options\_to\_context() (in module tornado.netutil)@\spxentry{ssl\_options\_to\_context()}\spxextra{in module tornado.netutil}}

\begin{fulllineitems}
\phantomsection\label{\detokenize{netutil:tornado.netutil.ssl_options_to_context}}\pysiglinewithargsret{\sphinxcode{\sphinxupquote{tornado.netutil.}}\sphinxbfcode{\sphinxupquote{ssl\_options\_to\_context}}}{\emph{ssl\_options: Union{[}Dict{[}str, Any{]}, ssl.SSLContext{]}}}{{ $\rightarrow$ ssl.SSLContext}}
Try to convert an \sphinxcode{\sphinxupquote{ssl\_options}} dictionary to an
\sphinxhref{https://docs.python.org/3.6/library/ssl.html\#ssl.SSLContext}{\sphinxcode{\sphinxupquote{SSLContext}}} object.

The \sphinxcode{\sphinxupquote{ssl\_options}} dictionary contains keywords to be passed to
\sphinxhref{https://docs.python.org/3.6/library/ssl.html\#ssl.wrap\_socket}{\sphinxcode{\sphinxupquote{ssl.wrap\_socket}}}.  In Python 2.7.9+, \sphinxhref{https://docs.python.org/3.6/library/ssl.html\#ssl.SSLContext}{\sphinxcode{\sphinxupquote{ssl.SSLContext}}} objects can
be used instead.  This function converts the dict form to its
\sphinxhref{https://docs.python.org/3.6/library/ssl.html\#ssl.SSLContext}{\sphinxcode{\sphinxupquote{SSLContext}}} equivalent, and may be used when a component which
accepts both forms needs to upgrade to the \sphinxhref{https://docs.python.org/3.6/library/ssl.html\#ssl.SSLContext}{\sphinxcode{\sphinxupquote{SSLContext}}} version
to use features like SNI or NPN.

\end{fulllineitems}

\index{ssl\_wrap\_socket() (in module tornado.netutil)@\spxentry{ssl\_wrap\_socket()}\spxextra{in module tornado.netutil}}

\begin{fulllineitems}
\phantomsection\label{\detokenize{netutil:tornado.netutil.ssl_wrap_socket}}\pysiglinewithargsret{\sphinxcode{\sphinxupquote{tornado.netutil.}}\sphinxbfcode{\sphinxupquote{ssl\_wrap\_socket}}}{\emph{socket: socket.socket, ssl\_options: Union{[}Dict{[}str, Any{]}, ssl.SSLContext{]}, server\_hostname: str = None, **kwargs}}{{ $\rightarrow$ ssl.SSLSocket}}
Returns an \sphinxcode{\sphinxupquote{ssl.SSLSocket}} wrapping the given socket.

\sphinxcode{\sphinxupquote{ssl\_options}} may be either an \sphinxhref{https://docs.python.org/3.6/library/ssl.html\#ssl.SSLContext}{\sphinxcode{\sphinxupquote{ssl.SSLContext}}} object or a
dictionary (as accepted by {\hyperref[\detokenize{netutil:tornado.netutil.ssl_options_to_context}]{\sphinxcrossref{\sphinxcode{\sphinxupquote{ssl\_options\_to\_context}}}}}).  Additional
keyword arguments are passed to \sphinxcode{\sphinxupquote{wrap\_socket}} (either the
\sphinxhref{https://docs.python.org/3.6/library/ssl.html\#ssl.SSLContext}{\sphinxcode{\sphinxupquote{SSLContext}}} method or the \sphinxhref{https://docs.python.org/3.6/library/ssl.html\#module-ssl}{\sphinxcode{\sphinxupquote{ssl}}} module function as
appropriate).

\end{fulllineitems}



\subsection{\sphinxstyleliteralintitle{\sphinxupquote{tornado.tcpclient}} — \sphinxstyleliteralintitle{\sphinxupquote{IOStream}} connection factory}
\label{\detokenize{tcpclient:module-tornado.tcpclient}}\label{\detokenize{tcpclient:tornado-tcpclient-iostream-connection-factory}}\label{\detokenize{tcpclient::doc}}\index{tornado.tcpclient (module)@\spxentry{tornado.tcpclient}\spxextra{module}}
A non-blocking TCP connection factory.
\index{TCPClient (class in tornado.tcpclient)@\spxentry{TCPClient}\spxextra{class in tornado.tcpclient}}

\begin{fulllineitems}
\phantomsection\label{\detokenize{tcpclient:tornado.tcpclient.TCPClient}}\pysiglinewithargsret{\sphinxbfcode{\sphinxupquote{class }}\sphinxcode{\sphinxupquote{tornado.tcpclient.}}\sphinxbfcode{\sphinxupquote{TCPClient}}}{\emph{resolver: tornado.netutil.Resolver = None}}{}
A non-blocking TCP connection factory.

\DUrole{versionmodified,changed}{Changed in version 5.0: }The \sphinxcode{\sphinxupquote{io\_loop}} argument (deprecated since version 4.1) has been removed.
\index{connect() (tornado.tcpclient.TCPClient method)@\spxentry{connect()}\spxextra{tornado.tcpclient.TCPClient method}}

\begin{fulllineitems}
\phantomsection\label{\detokenize{tcpclient:tornado.tcpclient.TCPClient.connect}}\pysiglinewithargsret{\sphinxbfcode{\sphinxupquote{coroutine }}\sphinxbfcode{\sphinxupquote{connect}}}{\emph{host: str, port: int, af: socket.AddressFamily = \textless{}AddressFamily.AF\_UNSPEC: 0\textgreater{}, ssl\_options: Union{[}Dict{[}str, Any{]}, ssl.SSLContext{]} = None, max\_buffer\_size: int = None, source\_ip: str = None, source\_port: int = None, timeout: Union{[}float, datetime.timedelta{]} = None}}{{ $\rightarrow$ tornado.iostream.IOStream}}
Connect to the given host and port.

Asynchronously returns an {\hyperref[\detokenize{iostream:tornado.iostream.IOStream}]{\sphinxcrossref{\sphinxcode{\sphinxupquote{IOStream}}}}} (or {\hyperref[\detokenize{iostream:tornado.iostream.SSLIOStream}]{\sphinxcrossref{\sphinxcode{\sphinxupquote{SSLIOStream}}}}} if
\sphinxcode{\sphinxupquote{ssl\_options}} is not None).

Using the \sphinxcode{\sphinxupquote{source\_ip}} kwarg, one can specify the source
IP address to use when establishing the connection.
In case the user needs to resolve and
use a specific interface, it has to be handled outside
of Tornado as this depends very much on the platform.

Raises \sphinxhref{https://docs.python.org/3.6/library/exceptions.html\#TimeoutError}{\sphinxcode{\sphinxupquote{TimeoutError}}} if the input future does not complete before
\sphinxcode{\sphinxupquote{timeout}}, which may be specified in any form allowed by
{\hyperref[\detokenize{ioloop:tornado.ioloop.IOLoop.add_timeout}]{\sphinxcrossref{\sphinxcode{\sphinxupquote{IOLoop.add\_timeout}}}}} (i.e. a \sphinxhref{https://docs.python.org/3.6/library/datetime.html\#datetime.timedelta}{\sphinxcode{\sphinxupquote{datetime.timedelta}}} or an absolute time
relative to {\hyperref[\detokenize{ioloop:tornado.ioloop.IOLoop.time}]{\sphinxcrossref{\sphinxcode{\sphinxupquote{IOLoop.time}}}}})

Similarly, when the user requires a certain source port, it can
be specified using the \sphinxcode{\sphinxupquote{source\_port}} arg.

\DUrole{versionmodified,changed}{Changed in version 4.5: }Added the \sphinxcode{\sphinxupquote{source\_ip}} and \sphinxcode{\sphinxupquote{source\_port}} arguments.

\DUrole{versionmodified,changed}{Changed in version 5.0: }Added the \sphinxcode{\sphinxupquote{timeout}} argument.

\end{fulllineitems}


\end{fulllineitems}



\subsection{\sphinxstyleliteralintitle{\sphinxupquote{tornado.tcpserver}} — Basic \sphinxstyleliteralintitle{\sphinxupquote{IOStream}}-based TCP server}
\label{\detokenize{tcpserver:module-tornado.tcpserver}}\label{\detokenize{tcpserver:tornado-tcpserver-basic-iostream-based-tcp-server}}\label{\detokenize{tcpserver::doc}}\index{tornado.tcpserver (module)@\spxentry{tornado.tcpserver}\spxextra{module}}
A non-blocking, single-threaded TCP server.
\index{TCPServer (class in tornado.tcpserver)@\spxentry{TCPServer}\spxextra{class in tornado.tcpserver}}

\begin{fulllineitems}
\phantomsection\label{\detokenize{tcpserver:tornado.tcpserver.TCPServer}}\pysiglinewithargsret{\sphinxbfcode{\sphinxupquote{class }}\sphinxcode{\sphinxupquote{tornado.tcpserver.}}\sphinxbfcode{\sphinxupquote{TCPServer}}}{\emph{ssl\_options: Union{[}Dict{[}str, Any{]}, ssl.SSLContext{]} = None, max\_buffer\_size: int = None, read\_chunk\_size: int = None}}{}
A non-blocking, single-threaded TCP server.

To use {\hyperref[\detokenize{tcpserver:tornado.tcpserver.TCPServer}]{\sphinxcrossref{\sphinxcode{\sphinxupquote{TCPServer}}}}}, define a subclass which overrides the {\hyperref[\detokenize{tcpserver:tornado.tcpserver.TCPServer.handle_stream}]{\sphinxcrossref{\sphinxcode{\sphinxupquote{handle\_stream}}}}}
method. For example, a simple echo server could be defined like this:

\begin{sphinxVerbatim}[commandchars=\\\{\}]
\PYG{k+kn}{from} \PYG{n+nn}{tornado}\PYG{n+nn}{.}\PYG{n+nn}{tcpserver} \PYG{k}{import} \PYG{n}{TCPServer}
\PYG{k+kn}{from} \PYG{n+nn}{tornado}\PYG{n+nn}{.}\PYG{n+nn}{iostream} \PYG{k}{import} \PYG{n}{StreamClosedError}
\PYG{k+kn}{from} \PYG{n+nn}{tornado} \PYG{k}{import} \PYG{n}{gen}

\PYG{k}{class} \PYG{n+nc}{EchoServer}\PYG{p}{(}\PYG{n}{TCPServer}\PYG{p}{)}\PYG{p}{:}
    \PYG{k}{async} \PYG{k}{def} \PYG{n+nf}{handle\PYGZus{}stream}\PYG{p}{(}\PYG{n+nb+bp}{self}\PYG{p}{,} \PYG{n}{stream}\PYG{p}{,} \PYG{n}{address}\PYG{p}{)}\PYG{p}{:}
        \PYG{k}{while} \PYG{k+kc}{True}\PYG{p}{:}
            \PYG{k}{try}\PYG{p}{:}
                \PYG{n}{data} \PYG{o}{=} \PYG{k}{await} \PYG{n}{stream}\PYG{o}{.}\PYG{n}{read\PYGZus{}until}\PYG{p}{(}\PYG{l+s+sa}{b}\PYG{l+s+s2}{\PYGZdq{}}\PYG{l+s+se}{\PYGZbs{}n}\PYG{l+s+s2}{\PYGZdq{}}\PYG{p}{)}
                \PYG{k}{await} \PYG{n}{stream}\PYG{o}{.}\PYG{n}{write}\PYG{p}{(}\PYG{n}{data}\PYG{p}{)}
            \PYG{k}{except} \PYG{n}{StreamClosedError}\PYG{p}{:}
                \PYG{k}{break}
\end{sphinxVerbatim}

To make this server serve SSL traffic, send the \sphinxcode{\sphinxupquote{ssl\_options}} keyword
argument with an \sphinxhref{https://docs.python.org/3.6/library/ssl.html\#ssl.SSLContext}{\sphinxcode{\sphinxupquote{ssl.SSLContext}}} object. For compatibility with older
versions of Python \sphinxcode{\sphinxupquote{ssl\_options}} may also be a dictionary of keyword
arguments for the \sphinxhref{https://docs.python.org/3.6/library/ssl.html\#ssl.wrap\_socket}{\sphinxcode{\sphinxupquote{ssl.wrap\_socket}}} method.:

\begin{sphinxVerbatim}[commandchars=\\\{\}]
\PYG{n}{ssl\PYGZus{}ctx} \PYG{o}{=} \PYG{n}{ssl}\PYG{o}{.}\PYG{n}{create\PYGZus{}default\PYGZus{}context}\PYG{p}{(}\PYG{n}{ssl}\PYG{o}{.}\PYG{n}{Purpose}\PYG{o}{.}\PYG{n}{CLIENT\PYGZus{}AUTH}\PYG{p}{)}
\PYG{n}{ssl\PYGZus{}ctx}\PYG{o}{.}\PYG{n}{load\PYGZus{}cert\PYGZus{}chain}\PYG{p}{(}\PYG{n}{os}\PYG{o}{.}\PYG{n}{path}\PYG{o}{.}\PYG{n}{join}\PYG{p}{(}\PYG{n}{data\PYGZus{}dir}\PYG{p}{,} \PYG{l+s+s2}{\PYGZdq{}}\PYG{l+s+s2}{mydomain.crt}\PYG{l+s+s2}{\PYGZdq{}}\PYG{p}{)}\PYG{p}{,}
                        \PYG{n}{os}\PYG{o}{.}\PYG{n}{path}\PYG{o}{.}\PYG{n}{join}\PYG{p}{(}\PYG{n}{data\PYGZus{}dir}\PYG{p}{,} \PYG{l+s+s2}{\PYGZdq{}}\PYG{l+s+s2}{mydomain.key}\PYG{l+s+s2}{\PYGZdq{}}\PYG{p}{)}\PYG{p}{)}
\PYG{n}{TCPServer}\PYG{p}{(}\PYG{n}{ssl\PYGZus{}options}\PYG{o}{=}\PYG{n}{ssl\PYGZus{}ctx}\PYG{p}{)}
\end{sphinxVerbatim}

{\hyperref[\detokenize{tcpserver:tornado.tcpserver.TCPServer}]{\sphinxcrossref{\sphinxcode{\sphinxupquote{TCPServer}}}}} initialization follows one of three patterns:
\begin{enumerate}
\def\theenumi{\arabic{enumi}}
\def\labelenumi{\theenumi .}
\makeatletter\def\p@enumii{\p@enumi \theenumi .}\makeatother
\item {} 
{\hyperref[\detokenize{tcpserver:tornado.tcpserver.TCPServer.listen}]{\sphinxcrossref{\sphinxcode{\sphinxupquote{listen}}}}}: simple single-process:

\begin{sphinxVerbatim}[commandchars=\\\{\}]
\PYG{n}{server} \PYG{o}{=} \PYG{n}{TCPServer}\PYG{p}{(}\PYG{p}{)}
\PYG{n}{server}\PYG{o}{.}\PYG{n}{listen}\PYG{p}{(}\PYG{l+m+mi}{8888}\PYG{p}{)}
\PYG{n}{IOLoop}\PYG{o}{.}\PYG{n}{current}\PYG{p}{(}\PYG{p}{)}\PYG{o}{.}\PYG{n}{start}\PYG{p}{(}\PYG{p}{)}
\end{sphinxVerbatim}

\item {} 
{\hyperref[\detokenize{tcpserver:tornado.tcpserver.TCPServer.bind}]{\sphinxcrossref{\sphinxcode{\sphinxupquote{bind}}}}}/{\hyperref[\detokenize{tcpserver:tornado.tcpserver.TCPServer.start}]{\sphinxcrossref{\sphinxcode{\sphinxupquote{start}}}}}: simple multi-process:

\begin{sphinxVerbatim}[commandchars=\\\{\}]
\PYG{n}{server} \PYG{o}{=} \PYG{n}{TCPServer}\PYG{p}{(}\PYG{p}{)}
\PYG{n}{server}\PYG{o}{.}\PYG{n}{bind}\PYG{p}{(}\PYG{l+m+mi}{8888}\PYG{p}{)}
\PYG{n}{server}\PYG{o}{.}\PYG{n}{start}\PYG{p}{(}\PYG{l+m+mi}{0}\PYG{p}{)}  \PYG{c+c1}{\PYGZsh{} Forks multiple sub\PYGZhy{}processes}
\PYG{n}{IOLoop}\PYG{o}{.}\PYG{n}{current}\PYG{p}{(}\PYG{p}{)}\PYG{o}{.}\PYG{n}{start}\PYG{p}{(}\PYG{p}{)}
\end{sphinxVerbatim}

When using this interface, an {\hyperref[\detokenize{ioloop:tornado.ioloop.IOLoop}]{\sphinxcrossref{\sphinxcode{\sphinxupquote{IOLoop}}}}} must \sphinxstyleemphasis{not} be passed
to the {\hyperref[\detokenize{tcpserver:tornado.tcpserver.TCPServer}]{\sphinxcrossref{\sphinxcode{\sphinxupquote{TCPServer}}}}} constructor.  {\hyperref[\detokenize{tcpserver:tornado.tcpserver.TCPServer.start}]{\sphinxcrossref{\sphinxcode{\sphinxupquote{start}}}}} will always start
the server on the default singleton {\hyperref[\detokenize{ioloop:tornado.ioloop.IOLoop}]{\sphinxcrossref{\sphinxcode{\sphinxupquote{IOLoop}}}}}.

\item {} 
{\hyperref[\detokenize{tcpserver:tornado.tcpserver.TCPServer.add_sockets}]{\sphinxcrossref{\sphinxcode{\sphinxupquote{add\_sockets}}}}}: advanced multi-process:

\begin{sphinxVerbatim}[commandchars=\\\{\}]
\PYG{n}{sockets} \PYG{o}{=} \PYG{n}{bind\PYGZus{}sockets}\PYG{p}{(}\PYG{l+m+mi}{8888}\PYG{p}{)}
\PYG{n}{tornado}\PYG{o}{.}\PYG{n}{process}\PYG{o}{.}\PYG{n}{fork\PYGZus{}processes}\PYG{p}{(}\PYG{l+m+mi}{0}\PYG{p}{)}
\PYG{n}{server} \PYG{o}{=} \PYG{n}{TCPServer}\PYG{p}{(}\PYG{p}{)}
\PYG{n}{server}\PYG{o}{.}\PYG{n}{add\PYGZus{}sockets}\PYG{p}{(}\PYG{n}{sockets}\PYG{p}{)}
\PYG{n}{IOLoop}\PYG{o}{.}\PYG{n}{current}\PYG{p}{(}\PYG{p}{)}\PYG{o}{.}\PYG{n}{start}\PYG{p}{(}\PYG{p}{)}
\end{sphinxVerbatim}

The {\hyperref[\detokenize{tcpserver:tornado.tcpserver.TCPServer.add_sockets}]{\sphinxcrossref{\sphinxcode{\sphinxupquote{add\_sockets}}}}} interface is more complicated, but it can be
used with {\hyperref[\detokenize{process:tornado.process.fork_processes}]{\sphinxcrossref{\sphinxcode{\sphinxupquote{tornado.process.fork\_processes}}}}} to give you more
flexibility in when the fork happens.  {\hyperref[\detokenize{tcpserver:tornado.tcpserver.TCPServer.add_sockets}]{\sphinxcrossref{\sphinxcode{\sphinxupquote{add\_sockets}}}}} can
also be used in single-process servers if you want to create
your listening sockets in some way other than
{\hyperref[\detokenize{netutil:tornado.netutil.bind_sockets}]{\sphinxcrossref{\sphinxcode{\sphinxupquote{bind\_sockets}}}}}.

\end{enumerate}

\DUrole{versionmodified,added}{New in version 3.1: }The \sphinxcode{\sphinxupquote{max\_buffer\_size}} argument.

\DUrole{versionmodified,changed}{Changed in version 5.0: }The \sphinxcode{\sphinxupquote{io\_loop}} argument has been removed.
\index{listen() (tornado.tcpserver.TCPServer method)@\spxentry{listen()}\spxextra{tornado.tcpserver.TCPServer method}}

\begin{fulllineitems}
\phantomsection\label{\detokenize{tcpserver:tornado.tcpserver.TCPServer.listen}}\pysiglinewithargsret{\sphinxbfcode{\sphinxupquote{listen}}}{\emph{port: int}, \emph{address: str = ''}}{{ $\rightarrow$ None}}
Starts accepting connections on the given port.

This method may be called more than once to listen on multiple ports.
{\hyperref[\detokenize{tcpserver:tornado.tcpserver.TCPServer.listen}]{\sphinxcrossref{\sphinxcode{\sphinxupquote{listen}}}}} takes effect immediately; it is not necessary to call
{\hyperref[\detokenize{tcpserver:tornado.tcpserver.TCPServer.start}]{\sphinxcrossref{\sphinxcode{\sphinxupquote{TCPServer.start}}}}} afterwards.  It is, however, necessary to start
the {\hyperref[\detokenize{ioloop:tornado.ioloop.IOLoop}]{\sphinxcrossref{\sphinxcode{\sphinxupquote{IOLoop}}}}}.

\end{fulllineitems}

\index{add\_sockets() (tornado.tcpserver.TCPServer method)@\spxentry{add\_sockets()}\spxextra{tornado.tcpserver.TCPServer method}}

\begin{fulllineitems}
\phantomsection\label{\detokenize{tcpserver:tornado.tcpserver.TCPServer.add_sockets}}\pysiglinewithargsret{\sphinxbfcode{\sphinxupquote{add\_sockets}}}{\emph{sockets: Iterable{[}socket.socket{]}}}{{ $\rightarrow$ None}}
Makes this server start accepting connections on the given sockets.

The \sphinxcode{\sphinxupquote{sockets}} parameter is a list of socket objects such as
those returned by {\hyperref[\detokenize{netutil:tornado.netutil.bind_sockets}]{\sphinxcrossref{\sphinxcode{\sphinxupquote{bind\_sockets}}}}}.
{\hyperref[\detokenize{tcpserver:tornado.tcpserver.TCPServer.add_sockets}]{\sphinxcrossref{\sphinxcode{\sphinxupquote{add\_sockets}}}}} is typically used in combination with that
method and {\hyperref[\detokenize{process:tornado.process.fork_processes}]{\sphinxcrossref{\sphinxcode{\sphinxupquote{tornado.process.fork\_processes}}}}} to provide greater
control over the initialization of a multi-process server.

\end{fulllineitems}

\index{add\_socket() (tornado.tcpserver.TCPServer method)@\spxentry{add\_socket()}\spxextra{tornado.tcpserver.TCPServer method}}

\begin{fulllineitems}
\phantomsection\label{\detokenize{tcpserver:tornado.tcpserver.TCPServer.add_socket}}\pysiglinewithargsret{\sphinxbfcode{\sphinxupquote{add\_socket}}}{\emph{socket: socket.socket}}{{ $\rightarrow$ None}}
Singular version of {\hyperref[\detokenize{tcpserver:tornado.tcpserver.TCPServer.add_sockets}]{\sphinxcrossref{\sphinxcode{\sphinxupquote{add\_sockets}}}}}.  Takes a single socket object.

\end{fulllineitems}

\index{bind() (tornado.tcpserver.TCPServer method)@\spxentry{bind()}\spxextra{tornado.tcpserver.TCPServer method}}

\begin{fulllineitems}
\phantomsection\label{\detokenize{tcpserver:tornado.tcpserver.TCPServer.bind}}\pysiglinewithargsret{\sphinxbfcode{\sphinxupquote{bind}}}{\emph{port: int}, \emph{address: str = None}, \emph{family: socket.AddressFamily = \textless{}AddressFamily.AF\_UNSPEC: 0\textgreater{}}, \emph{backlog: int = 128}, \emph{reuse\_port: bool = False}}{{ $\rightarrow$ None}}
Binds this server to the given port on the given address.

To start the server, call {\hyperref[\detokenize{tcpserver:tornado.tcpserver.TCPServer.start}]{\sphinxcrossref{\sphinxcode{\sphinxupquote{start}}}}}. If you want to run this server
in a single process, you can call {\hyperref[\detokenize{tcpserver:tornado.tcpserver.TCPServer.listen}]{\sphinxcrossref{\sphinxcode{\sphinxupquote{listen}}}}} as a shortcut to the
sequence of {\hyperref[\detokenize{tcpserver:tornado.tcpserver.TCPServer.bind}]{\sphinxcrossref{\sphinxcode{\sphinxupquote{bind}}}}} and {\hyperref[\detokenize{tcpserver:tornado.tcpserver.TCPServer.start}]{\sphinxcrossref{\sphinxcode{\sphinxupquote{start}}}}} calls.

Address may be either an IP address or hostname.  If it’s a hostname,
the server will listen on all IP addresses associated with the
name.  Address may be an empty string or None to listen on all
available interfaces.  Family may be set to either \sphinxhref{https://docs.python.org/3.6/library/socket.html\#socket.AF\_INET}{\sphinxcode{\sphinxupquote{socket.AF\_INET}}}
or \sphinxhref{https://docs.python.org/3.6/library/socket.html\#socket.AF\_INET6}{\sphinxcode{\sphinxupquote{socket.AF\_INET6}}} to restrict to IPv4 or IPv6 addresses, otherwise
both will be used if available.

The \sphinxcode{\sphinxupquote{backlog}} argument has the same meaning as for
\sphinxhref{https://docs.python.org/3.6/library/socket.html\#socket.socket.listen}{\sphinxcode{\sphinxupquote{socket.listen}}}. The \sphinxcode{\sphinxupquote{reuse\_port}} argument
has the same meaning as for {\hyperref[\detokenize{netutil:tornado.netutil.bind_sockets}]{\sphinxcrossref{\sphinxcode{\sphinxupquote{bind\_sockets}}}}}.

This method may be called multiple times prior to {\hyperref[\detokenize{tcpserver:tornado.tcpserver.TCPServer.start}]{\sphinxcrossref{\sphinxcode{\sphinxupquote{start}}}}} to listen
on multiple ports or interfaces.

\DUrole{versionmodified,changed}{Changed in version 4.4: }Added the \sphinxcode{\sphinxupquote{reuse\_port}} argument.

\end{fulllineitems}

\index{start() (tornado.tcpserver.TCPServer method)@\spxentry{start()}\spxextra{tornado.tcpserver.TCPServer method}}

\begin{fulllineitems}
\phantomsection\label{\detokenize{tcpserver:tornado.tcpserver.TCPServer.start}}\pysiglinewithargsret{\sphinxbfcode{\sphinxupquote{start}}}{\emph{num\_processes: Optional{[}int{]} = 1}, \emph{max\_restarts: int = None}}{{ $\rightarrow$ None}}
Starts this server in the {\hyperref[\detokenize{ioloop:tornado.ioloop.IOLoop}]{\sphinxcrossref{\sphinxcode{\sphinxupquote{IOLoop}}}}}.

By default, we run the server in this process and do not fork any
additional child process.

If num\_processes is \sphinxcode{\sphinxupquote{None}} or \textless{}= 0, we detect the number of cores
available on this machine and fork that number of child
processes. If num\_processes is given and \textgreater{} 1, we fork that
specific number of sub-processes.

Since we use processes and not threads, there is no shared memory
between any server code.

Note that multiple processes are not compatible with the autoreload
module (or the \sphinxcode{\sphinxupquote{autoreload=True}} option to {\hyperref[\detokenize{web:tornado.web.Application}]{\sphinxcrossref{\sphinxcode{\sphinxupquote{tornado.web.Application}}}}}
which defaults to True when \sphinxcode{\sphinxupquote{debug=True}}).
When using multiple processes, no IOLoops can be created or
referenced until after the call to \sphinxcode{\sphinxupquote{TCPServer.start(n)}}.

The \sphinxcode{\sphinxupquote{max\_restarts}} argument is passed to {\hyperref[\detokenize{process:tornado.process.fork_processes}]{\sphinxcrossref{\sphinxcode{\sphinxupquote{fork\_processes}}}}}.

\DUrole{versionmodified,changed}{Changed in version 6.0: }Added \sphinxcode{\sphinxupquote{max\_restarts}} argument.

\end{fulllineitems}

\index{stop() (tornado.tcpserver.TCPServer method)@\spxentry{stop()}\spxextra{tornado.tcpserver.TCPServer method}}

\begin{fulllineitems}
\phantomsection\label{\detokenize{tcpserver:tornado.tcpserver.TCPServer.stop}}\pysiglinewithargsret{\sphinxbfcode{\sphinxupquote{stop}}}{}{{ $\rightarrow$ None}}
Stops listening for new connections.

Requests currently in progress may still continue after the
server is stopped.

\end{fulllineitems}

\index{handle\_stream() (tornado.tcpserver.TCPServer method)@\spxentry{handle\_stream()}\spxextra{tornado.tcpserver.TCPServer method}}

\begin{fulllineitems}
\phantomsection\label{\detokenize{tcpserver:tornado.tcpserver.TCPServer.handle_stream}}\pysiglinewithargsret{\sphinxbfcode{\sphinxupquote{handle\_stream}}}{\emph{stream: tornado.iostream.IOStream}, \emph{address: tuple}}{{ $\rightarrow$ Optional{[}Awaitable{[}None{]}{]}}}
Override to handle a new {\hyperref[\detokenize{iostream:tornado.iostream.IOStream}]{\sphinxcrossref{\sphinxcode{\sphinxupquote{IOStream}}}}} from an incoming connection.

This method may be a coroutine; if so any exceptions it raises
asynchronously will be logged. Accepting of incoming connections
will not be blocked by this coroutine.

If this {\hyperref[\detokenize{tcpserver:tornado.tcpserver.TCPServer}]{\sphinxcrossref{\sphinxcode{\sphinxupquote{TCPServer}}}}} is configured for SSL, \sphinxcode{\sphinxupquote{handle\_stream}}
may be called before the SSL handshake has completed. Use
{\hyperref[\detokenize{iostream:tornado.iostream.SSLIOStream.wait_for_handshake}]{\sphinxcrossref{\sphinxcode{\sphinxupquote{SSLIOStream.wait\_for\_handshake}}}}} if you need to verify the client’s
certificate or use NPN/ALPN.

\DUrole{versionmodified,changed}{Changed in version 4.2: }Added the option for this method to be a coroutine.

\end{fulllineitems}


\end{fulllineitems}



\section{Coroutines and concurrency}
\label{\detokenize{coroutine:coroutines-and-concurrency}}\label{\detokenize{coroutine::doc}}

\subsection{\sphinxstyleliteralintitle{\sphinxupquote{tornado.gen}} — Generator-based coroutines}
\label{\detokenize{gen:tornado-gen-generator-based-coroutines}}\label{\detokenize{gen::doc}}\phantomsection\label{\detokenize{gen:module-tornado.gen}}\index{tornado.gen (module)@\spxentry{tornado.gen}\spxextra{module}}
\sphinxcode{\sphinxupquote{tornado.gen}} implements generator-based coroutines.

\begin{sphinxadmonition}{note}{Note:}
The “decorator and generator” approach in this module is a
precursor to native coroutines (using \sphinxcode{\sphinxupquote{async def}} and \sphinxcode{\sphinxupquote{await}})
which were introduced in Python 3.5. Applications that do not
require compatibility with older versions of Python should use
native coroutines instead. Some parts of this module are still
useful with native coroutines, notably {\hyperref[\detokenize{gen:tornado.gen.multi}]{\sphinxcrossref{\sphinxcode{\sphinxupquote{multi}}}}}, {\hyperref[\detokenize{gen:tornado.gen.sleep}]{\sphinxcrossref{\sphinxcode{\sphinxupquote{sleep}}}}},
{\hyperref[\detokenize{gen:tornado.gen.WaitIterator}]{\sphinxcrossref{\sphinxcode{\sphinxupquote{WaitIterator}}}}}, and {\hyperref[\detokenize{gen:tornado.gen.with_timeout}]{\sphinxcrossref{\sphinxcode{\sphinxupquote{with\_timeout}}}}}. Some of these functions have
counterparts in the \sphinxhref{https://docs.python.org/3.6/library/asyncio.html\#module-asyncio}{\sphinxcode{\sphinxupquote{asyncio}}} module which may be used as well,
although the two may not necessarily be 100\% compatible.
\end{sphinxadmonition}

Coroutines provide an easier way to work in an asynchronous
environment than chaining callbacks. Code using coroutines is
technically asynchronous, but it is written as a single generator
instead of a collection of separate functions.

For example, here’s a coroutine-based handler:

\begin{sphinxVerbatim}[commandchars=\\\{\}]
\PYG{k}{class} \PYG{n+nc}{GenAsyncHandler}\PYG{p}{(}\PYG{n}{RequestHandler}\PYG{p}{)}\PYG{p}{:}
    \PYG{n+nd}{@gen}\PYG{o}{.}\PYG{n}{coroutine}
    \PYG{k}{def} \PYG{n+nf}{get}\PYG{p}{(}\PYG{n+nb+bp}{self}\PYG{p}{)}\PYG{p}{:}
        \PYG{n}{http\PYGZus{}client} \PYG{o}{=} \PYG{n}{AsyncHTTPClient}\PYG{p}{(}\PYG{p}{)}
        \PYG{n}{response} \PYG{o}{=} \PYG{k}{yield} \PYG{n}{http\PYGZus{}client}\PYG{o}{.}\PYG{n}{fetch}\PYG{p}{(}\PYG{l+s+s2}{\PYGZdq{}}\PYG{l+s+s2}{http://example.com}\PYG{l+s+s2}{\PYGZdq{}}\PYG{p}{)}
        \PYG{n}{do\PYGZus{}something\PYGZus{}with\PYGZus{}response}\PYG{p}{(}\PYG{n}{response}\PYG{p}{)}
        \PYG{n+nb+bp}{self}\PYG{o}{.}\PYG{n}{render}\PYG{p}{(}\PYG{l+s+s2}{\PYGZdq{}}\PYG{l+s+s2}{template.html}\PYG{l+s+s2}{\PYGZdq{}}\PYG{p}{)}
\end{sphinxVerbatim}

Asynchronous functions in Tornado return an \sphinxcode{\sphinxupquote{Awaitable}} or {\hyperref[\detokenize{concurrent:tornado.concurrent.Future}]{\sphinxcrossref{\sphinxcode{\sphinxupquote{Future}}}}};
yielding this object returns its result.

You can also yield a list or dict of other yieldable objects, which
will be started at the same time and run in parallel; a list or dict
of results will be returned when they are all finished:

\begin{sphinxVerbatim}[commandchars=\\\{\}]
\PYG{n+nd}{@gen}\PYG{o}{.}\PYG{n}{coroutine}
\PYG{k}{def} \PYG{n+nf}{get}\PYG{p}{(}\PYG{n+nb+bp}{self}\PYG{p}{)}\PYG{p}{:}
    \PYG{n}{http\PYGZus{}client} \PYG{o}{=} \PYG{n}{AsyncHTTPClient}\PYG{p}{(}\PYG{p}{)}
    \PYG{n}{response1}\PYG{p}{,} \PYG{n}{response2} \PYG{o}{=} \PYG{k}{yield} \PYG{p}{[}\PYG{n}{http\PYGZus{}client}\PYG{o}{.}\PYG{n}{fetch}\PYG{p}{(}\PYG{n}{url1}\PYG{p}{)}\PYG{p}{,}
                                  \PYG{n}{http\PYGZus{}client}\PYG{o}{.}\PYG{n}{fetch}\PYG{p}{(}\PYG{n}{url2}\PYG{p}{)}\PYG{p}{]}
    \PYG{n}{response\PYGZus{}dict} \PYG{o}{=} \PYG{k}{yield} \PYG{n+nb}{dict}\PYG{p}{(}\PYG{n}{response3}\PYG{o}{=}\PYG{n}{http\PYGZus{}client}\PYG{o}{.}\PYG{n}{fetch}\PYG{p}{(}\PYG{n}{url3}\PYG{p}{)}\PYG{p}{,}
                               \PYG{n}{response4}\PYG{o}{=}\PYG{n}{http\PYGZus{}client}\PYG{o}{.}\PYG{n}{fetch}\PYG{p}{(}\PYG{n}{url4}\PYG{p}{)}\PYG{p}{)}
    \PYG{n}{response3} \PYG{o}{=} \PYG{n}{response\PYGZus{}dict}\PYG{p}{[}\PYG{l+s+s1}{\PYGZsq{}}\PYG{l+s+s1}{response3}\PYG{l+s+s1}{\PYGZsq{}}\PYG{p}{]}
    \PYG{n}{response4} \PYG{o}{=} \PYG{n}{response\PYGZus{}dict}\PYG{p}{[}\PYG{l+s+s1}{\PYGZsq{}}\PYG{l+s+s1}{response4}\PYG{l+s+s1}{\PYGZsq{}}\PYG{p}{]}
\end{sphinxVerbatim}

If \sphinxcode{\sphinxupquote{tornado.platform.twisted}} is imported, it is also possible to
yield Twisted’s \sphinxcode{\sphinxupquote{Deferred}} objects. See the {\hyperref[\detokenize{gen:tornado.gen.convert_yielded}]{\sphinxcrossref{\sphinxcode{\sphinxupquote{convert\_yielded}}}}}
function to extend this mechanism.

\DUrole{versionmodified,changed}{Changed in version 3.2: }Dict support added.

\DUrole{versionmodified,changed}{Changed in version 4.1: }Support added for yielding \sphinxcode{\sphinxupquote{asyncio}} Futures and Twisted Deferreds
via \sphinxcode{\sphinxupquote{singledispatch}}.


\subsubsection{Decorators}
\label{\detokenize{gen:decorators}}\index{coroutine() (in module tornado.gen)@\spxentry{coroutine()}\spxextra{in module tornado.gen}}

\begin{fulllineitems}
\phantomsection\label{\detokenize{gen:tornado.gen.coroutine}}\pysiglinewithargsret{\sphinxcode{\sphinxupquote{tornado.gen.}}\sphinxbfcode{\sphinxupquote{coroutine}}}{\emph{func: Callable{[}{[}...{]}, Generator{[}Any, Any, \_T{]}{]}}}{{ $\rightarrow$ Callable{[}{[}...{]}, Future{[}\_T{]}{]}}}
Decorator for asynchronous generators.

For compatibility with older versions of Python, coroutines may
also “return” by raising the special exception {\hyperref[\detokenize{gen:tornado.gen.Return}]{\sphinxcrossref{\sphinxcode{\sphinxupquote{Return(value)}}}}}.

Functions with this decorator return a {\hyperref[\detokenize{concurrent:tornado.concurrent.Future}]{\sphinxcrossref{\sphinxcode{\sphinxupquote{Future}}}}}.

\begin{sphinxadmonition}{warning}{Warning:}
When exceptions occur inside a coroutine, the exception
information will be stored in the {\hyperref[\detokenize{concurrent:tornado.concurrent.Future}]{\sphinxcrossref{\sphinxcode{\sphinxupquote{Future}}}}} object. You must
examine the result of the {\hyperref[\detokenize{concurrent:tornado.concurrent.Future}]{\sphinxcrossref{\sphinxcode{\sphinxupquote{Future}}}}} object, or the exception
may go unnoticed by your code. This means yielding the function
if called from another coroutine, using something like
{\hyperref[\detokenize{ioloop:tornado.ioloop.IOLoop.run_sync}]{\sphinxcrossref{\sphinxcode{\sphinxupquote{IOLoop.run\_sync}}}}} for top-level calls, or passing the {\hyperref[\detokenize{concurrent:tornado.concurrent.Future}]{\sphinxcrossref{\sphinxcode{\sphinxupquote{Future}}}}}
to {\hyperref[\detokenize{ioloop:tornado.ioloop.IOLoop.add_future}]{\sphinxcrossref{\sphinxcode{\sphinxupquote{IOLoop.add\_future}}}}}.
\end{sphinxadmonition}

\DUrole{versionmodified,changed}{Changed in version 6.0: }The \sphinxcode{\sphinxupquote{callback}} argument was removed. Use the returned
awaitable object instead.

\end{fulllineitems}

\index{Return@\spxentry{Return}}

\begin{fulllineitems}
\phantomsection\label{\detokenize{gen:tornado.gen.Return}}\pysiglinewithargsret{\sphinxbfcode{\sphinxupquote{exception }}\sphinxcode{\sphinxupquote{tornado.gen.}}\sphinxbfcode{\sphinxupquote{Return}}}{\emph{value: Any = None}}{}
Special exception to return a value from a {\hyperref[\detokenize{gen:tornado.gen.coroutine}]{\sphinxcrossref{\sphinxcode{\sphinxupquote{coroutine}}}}}.

If this exception is raised, its value argument is used as the
result of the coroutine:

\begin{sphinxVerbatim}[commandchars=\\\{\}]
\PYG{n+nd}{@gen}\PYG{o}{.}\PYG{n}{coroutine}
\PYG{k}{def} \PYG{n+nf}{fetch\PYGZus{}json}\PYG{p}{(}\PYG{n}{url}\PYG{p}{)}\PYG{p}{:}
    \PYG{n}{response} \PYG{o}{=} \PYG{k}{yield} \PYG{n}{AsyncHTTPClient}\PYG{p}{(}\PYG{p}{)}\PYG{o}{.}\PYG{n}{fetch}\PYG{p}{(}\PYG{n}{url}\PYG{p}{)}
    \PYG{k}{raise} \PYG{n}{gen}\PYG{o}{.}\PYG{n}{Return}\PYG{p}{(}\PYG{n}{json\PYGZus{}decode}\PYG{p}{(}\PYG{n}{response}\PYG{o}{.}\PYG{n}{body}\PYG{p}{)}\PYG{p}{)}
\end{sphinxVerbatim}

In Python 3.3, this exception is no longer necessary: the \sphinxcode{\sphinxupquote{return}}
statement can be used directly to return a value (previously
\sphinxcode{\sphinxupquote{yield}} and \sphinxcode{\sphinxupquote{return}} with a value could not be combined in the
same function).

By analogy with the return statement, the value argument is optional,
but it is never necessary to \sphinxcode{\sphinxupquote{raise gen.Return()}}.  The \sphinxcode{\sphinxupquote{return}}
statement can be used with no arguments instead.

\end{fulllineitems}



\subsubsection{Utility functions}
\label{\detokenize{gen:utility-functions}}\index{with\_timeout() (in module tornado.gen)@\spxentry{with\_timeout()}\spxextra{in module tornado.gen}}

\begin{fulllineitems}
\phantomsection\label{\detokenize{gen:tornado.gen.with_timeout}}\pysiglinewithargsret{\sphinxcode{\sphinxupquote{tornado.gen.}}\sphinxbfcode{\sphinxupquote{with\_timeout}}}{\emph{timeout: Union{[}float, datetime.timedelta{]}, future: Yieldable, quiet\_exceptions: Union{[}Type{[}Exception{]}, Tuple{[}Type{[}Exception{]}, ...{]}{]} = ()}}{}
Wraps a {\hyperref[\detokenize{concurrent:tornado.concurrent.Future}]{\sphinxcrossref{\sphinxcode{\sphinxupquote{Future}}}}} (or other yieldable object) in a timeout.

Raises {\hyperref[\detokenize{util:tornado.util.TimeoutError}]{\sphinxcrossref{\sphinxcode{\sphinxupquote{tornado.util.TimeoutError}}}}} if the input future does not
complete before \sphinxcode{\sphinxupquote{timeout}}, which may be specified in any form
allowed by {\hyperref[\detokenize{ioloop:tornado.ioloop.IOLoop.add_timeout}]{\sphinxcrossref{\sphinxcode{\sphinxupquote{IOLoop.add\_timeout}}}}} (i.e. a \sphinxhref{https://docs.python.org/3.6/library/datetime.html\#datetime.timedelta}{\sphinxcode{\sphinxupquote{datetime.timedelta}}} or
an absolute time relative to {\hyperref[\detokenize{ioloop:tornado.ioloop.IOLoop.time}]{\sphinxcrossref{\sphinxcode{\sphinxupquote{IOLoop.time}}}}})

If the wrapped {\hyperref[\detokenize{concurrent:tornado.concurrent.Future}]{\sphinxcrossref{\sphinxcode{\sphinxupquote{Future}}}}} fails after it has timed out, the exception
will be logged unless it is of a type contained in \sphinxcode{\sphinxupquote{quiet\_exceptions}}
(which may be an exception type or a sequence of types).

The wrapped {\hyperref[\detokenize{concurrent:tornado.concurrent.Future}]{\sphinxcrossref{\sphinxcode{\sphinxupquote{Future}}}}} is not canceled when the timeout expires,
permitting it to be reused. \sphinxhref{https://docs.python.org/3.6/library/asyncio-task.html\#asyncio.wait\_for}{\sphinxcode{\sphinxupquote{asyncio.wait\_for}}} is similar to this
function but it does cancel the wrapped {\hyperref[\detokenize{concurrent:tornado.concurrent.Future}]{\sphinxcrossref{\sphinxcode{\sphinxupquote{Future}}}}} on timeout.

\DUrole{versionmodified,added}{New in version 4.0.}

\DUrole{versionmodified,changed}{Changed in version 4.1: }Added the \sphinxcode{\sphinxupquote{quiet\_exceptions}} argument and the logging of unhandled
exceptions.

\DUrole{versionmodified,changed}{Changed in version 4.4: }Added support for yieldable objects other than {\hyperref[\detokenize{concurrent:tornado.concurrent.Future}]{\sphinxcrossref{\sphinxcode{\sphinxupquote{Future}}}}}.

\end{fulllineitems}

\index{sleep() (in module tornado.gen)@\spxentry{sleep()}\spxextra{in module tornado.gen}}

\begin{fulllineitems}
\phantomsection\label{\detokenize{gen:tornado.gen.sleep}}\pysiglinewithargsret{\sphinxcode{\sphinxupquote{tornado.gen.}}\sphinxbfcode{\sphinxupquote{sleep}}}{\emph{duration: float}}{{ $\rightarrow$ Future{[}None{]}}}
Return a {\hyperref[\detokenize{concurrent:tornado.concurrent.Future}]{\sphinxcrossref{\sphinxcode{\sphinxupquote{Future}}}}} that resolves after the given number of seconds.

When used with \sphinxcode{\sphinxupquote{yield}} in a coroutine, this is a non-blocking
analogue to \sphinxhref{https://docs.python.org/3.6/library/time.html\#time.sleep}{\sphinxcode{\sphinxupquote{time.sleep}}} (which should not be used in coroutines
because it is blocking):

\begin{sphinxVerbatim}[commandchars=\\\{\}]
\PYG{k}{yield} \PYG{n}{gen}\PYG{o}{.}\PYG{n}{sleep}\PYG{p}{(}\PYG{l+m+mf}{0.5}\PYG{p}{)}
\end{sphinxVerbatim}

Note that calling this function on its own does nothing; you must
wait on the {\hyperref[\detokenize{concurrent:tornado.concurrent.Future}]{\sphinxcrossref{\sphinxcode{\sphinxupquote{Future}}}}} it returns (usually by yielding it).

\DUrole{versionmodified,added}{New in version 4.1.}

\end{fulllineitems}

\index{WaitIterator (class in tornado.gen)@\spxentry{WaitIterator}\spxextra{class in tornado.gen}}

\begin{fulllineitems}
\phantomsection\label{\detokenize{gen:tornado.gen.WaitIterator}}\pysiglinewithargsret{\sphinxbfcode{\sphinxupquote{class }}\sphinxcode{\sphinxupquote{tornado.gen.}}\sphinxbfcode{\sphinxupquote{WaitIterator}}}{\emph{*args}, \emph{**kwargs}}{}
Provides an iterator to yield the results of awaitables as they finish.

Yielding a set of awaitables like this:

\sphinxcode{\sphinxupquote{results = yield {[}awaitable1, awaitable2{]}}}

pauses the coroutine until both \sphinxcode{\sphinxupquote{awaitable1}} and \sphinxcode{\sphinxupquote{awaitable2}}
return, and then restarts the coroutine with the results of both
awaitables. If either awaitable raises an exception, the
expression will raise that exception and all the results will be
lost.

If you need to get the result of each awaitable as soon as possible,
or if you need the result of some awaitables even if others produce
errors, you can use \sphinxcode{\sphinxupquote{WaitIterator}}:

\begin{sphinxVerbatim}[commandchars=\\\{\}]
\PYG{n}{wait\PYGZus{}iterator} \PYG{o}{=} \PYG{n}{gen}\PYG{o}{.}\PYG{n}{WaitIterator}\PYG{p}{(}\PYG{n}{awaitable1}\PYG{p}{,} \PYG{n}{awaitable2}\PYG{p}{)}
\PYG{k}{while} \PYG{o+ow}{not} \PYG{n}{wait\PYGZus{}iterator}\PYG{o}{.}\PYG{n}{done}\PYG{p}{(}\PYG{p}{)}\PYG{p}{:}
    \PYG{k}{try}\PYG{p}{:}
        \PYG{n}{result} \PYG{o}{=} \PYG{k}{yield} \PYG{n}{wait\PYGZus{}iterator}\PYG{o}{.}\PYG{n}{next}\PYG{p}{(}\PYG{p}{)}
    \PYG{k}{except} \PYG{n+ne}{Exception} \PYG{k}{as} \PYG{n}{e}\PYG{p}{:}
        \PYG{n+nb}{print}\PYG{p}{(}\PYG{l+s+s2}{\PYGZdq{}}\PYG{l+s+s2}{Error }\PYG{l+s+si}{\PYGZob{}\PYGZcb{}}\PYG{l+s+s2}{ from }\PYG{l+s+si}{\PYGZob{}\PYGZcb{}}\PYG{l+s+s2}{\PYGZdq{}}\PYG{o}{.}\PYG{n}{format}\PYG{p}{(}\PYG{n}{e}\PYG{p}{,} \PYG{n}{wait\PYGZus{}iterator}\PYG{o}{.}\PYG{n}{current\PYGZus{}future}\PYG{p}{)}\PYG{p}{)}
    \PYG{k}{else}\PYG{p}{:}
        \PYG{n+nb}{print}\PYG{p}{(}\PYG{l+s+s2}{\PYGZdq{}}\PYG{l+s+s2}{Result }\PYG{l+s+si}{\PYGZob{}\PYGZcb{}}\PYG{l+s+s2}{ received from }\PYG{l+s+si}{\PYGZob{}\PYGZcb{}}\PYG{l+s+s2}{ at }\PYG{l+s+si}{\PYGZob{}\PYGZcb{}}\PYG{l+s+s2}{\PYGZdq{}}\PYG{o}{.}\PYG{n}{format}\PYG{p}{(}
            \PYG{n}{result}\PYG{p}{,} \PYG{n}{wait\PYGZus{}iterator}\PYG{o}{.}\PYG{n}{current\PYGZus{}future}\PYG{p}{,}
            \PYG{n}{wait\PYGZus{}iterator}\PYG{o}{.}\PYG{n}{current\PYGZus{}index}\PYG{p}{)}\PYG{p}{)}
\end{sphinxVerbatim}

Because results are returned as soon as they are available the
output from the iterator \sphinxstyleemphasis{will not be in the same order as the
input arguments}. If you need to know which future produced the
current result, you can use the attributes
\sphinxcode{\sphinxupquote{WaitIterator.current\_future}}, or \sphinxcode{\sphinxupquote{WaitIterator.current\_index}}
to get the index of the awaitable from the input list. (if keyword
arguments were used in the construction of the {\hyperref[\detokenize{gen:tornado.gen.WaitIterator}]{\sphinxcrossref{\sphinxcode{\sphinxupquote{WaitIterator}}}}},
\sphinxcode{\sphinxupquote{current\_index}} will use the corresponding keyword).

On Python 3.5, {\hyperref[\detokenize{gen:tornado.gen.WaitIterator}]{\sphinxcrossref{\sphinxcode{\sphinxupquote{WaitIterator}}}}} implements the async iterator
protocol, so it can be used with the \sphinxcode{\sphinxupquote{async for}} statement (note
that in this version the entire iteration is aborted if any value
raises an exception, while the previous example can continue past
individual errors):

\begin{sphinxVerbatim}[commandchars=\\\{\}]
\PYG{k}{async} \PYG{k}{for} \PYG{n}{result} \PYG{o+ow}{in} \PYG{n}{gen}\PYG{o}{.}\PYG{n}{WaitIterator}\PYG{p}{(}\PYG{n}{future1}\PYG{p}{,} \PYG{n}{future2}\PYG{p}{)}\PYG{p}{:}
    \PYG{n+nb}{print}\PYG{p}{(}\PYG{l+s+s2}{\PYGZdq{}}\PYG{l+s+s2}{Result }\PYG{l+s+si}{\PYGZob{}\PYGZcb{}}\PYG{l+s+s2}{ received from }\PYG{l+s+si}{\PYGZob{}\PYGZcb{}}\PYG{l+s+s2}{ at }\PYG{l+s+si}{\PYGZob{}\PYGZcb{}}\PYG{l+s+s2}{\PYGZdq{}}\PYG{o}{.}\PYG{n}{format}\PYG{p}{(}
        \PYG{n}{result}\PYG{p}{,} \PYG{n}{wait\PYGZus{}iterator}\PYG{o}{.}\PYG{n}{current\PYGZus{}future}\PYG{p}{,}
        \PYG{n}{wait\PYGZus{}iterator}\PYG{o}{.}\PYG{n}{current\PYGZus{}index}\PYG{p}{)}\PYG{p}{)}
\end{sphinxVerbatim}

\DUrole{versionmodified,added}{New in version 4.1.}

\DUrole{versionmodified,changed}{Changed in version 4.3: }Added \sphinxcode{\sphinxupquote{async for}} support in Python 3.5.
\index{done() (tornado.gen.WaitIterator method)@\spxentry{done()}\spxextra{tornado.gen.WaitIterator method}}

\begin{fulllineitems}
\phantomsection\label{\detokenize{gen:tornado.gen.WaitIterator.done}}\pysiglinewithargsret{\sphinxbfcode{\sphinxupquote{done}}}{}{{ $\rightarrow$ bool}}
Returns True if this iterator has no more results.

\end{fulllineitems}

\index{next() (tornado.gen.WaitIterator method)@\spxentry{next()}\spxextra{tornado.gen.WaitIterator method}}

\begin{fulllineitems}
\phantomsection\label{\detokenize{gen:tornado.gen.WaitIterator.next}}\pysiglinewithargsret{\sphinxbfcode{\sphinxupquote{next}}}{}{{ $\rightarrow$ \_asyncio.Future}}
Returns a {\hyperref[\detokenize{concurrent:tornado.concurrent.Future}]{\sphinxcrossref{\sphinxcode{\sphinxupquote{Future}}}}} that will yield the next available result.

Note that this {\hyperref[\detokenize{concurrent:tornado.concurrent.Future}]{\sphinxcrossref{\sphinxcode{\sphinxupquote{Future}}}}} will not be the same object as any of
the inputs.

\end{fulllineitems}


\end{fulllineitems}

\index{multi() (in module tornado.gen)@\spxentry{multi()}\spxextra{in module tornado.gen}}

\begin{fulllineitems}
\phantomsection\label{\detokenize{gen:tornado.gen.multi}}\pysiglinewithargsret{\sphinxcode{\sphinxupquote{tornado.gen.}}\sphinxbfcode{\sphinxupquote{multi}}}{\emph{Union{[}List{[}Yieldable{]}, Dict{[}Any, Yieldable{]}{]}, quiet\_exceptions: Union{[}Type{[}Exception{]}, Tuple{[}Type{[}Exception{]}, ...{]}{]} = ()}}{}
Runs multiple asynchronous operations in parallel.

\sphinxcode{\sphinxupquote{children}} may either be a list or a dict whose values are
yieldable objects. \sphinxcode{\sphinxupquote{multi()}} returns a new yieldable
object that resolves to a parallel structure containing their
results. If \sphinxcode{\sphinxupquote{children}} is a list, the result is a list of
results in the same order; if it is a dict, the result is a dict
with the same keys.

That is, \sphinxcode{\sphinxupquote{results = yield multi(list\_of\_futures)}} is equivalent
to:

\begin{sphinxVerbatim}[commandchars=\\\{\}]
\PYG{n}{results} \PYG{o}{=} \PYG{p}{[}\PYG{p}{]}
\PYG{k}{for} \PYG{n}{future} \PYG{o+ow}{in} \PYG{n}{list\PYGZus{}of\PYGZus{}futures}\PYG{p}{:}
    \PYG{n}{results}\PYG{o}{.}\PYG{n}{append}\PYG{p}{(}\PYG{k}{yield} \PYG{n}{future}\PYG{p}{)}
\end{sphinxVerbatim}

If any children raise exceptions, \sphinxcode{\sphinxupquote{multi()}} will raise the first
one. All others will be logged, unless they are of types
contained in the \sphinxcode{\sphinxupquote{quiet\_exceptions}} argument.

In a \sphinxcode{\sphinxupquote{yield}}-based coroutine, it is not normally necessary to
call this function directly, since the coroutine runner will
do it automatically when a list or dict is yielded. However,
it is necessary in \sphinxcode{\sphinxupquote{await}}-based coroutines, or to pass
the \sphinxcode{\sphinxupquote{quiet\_exceptions}} argument.

This function is available under the names \sphinxcode{\sphinxupquote{multi()}} and \sphinxcode{\sphinxupquote{Multi()}}
for historical reasons.

Cancelling a {\hyperref[\detokenize{concurrent:tornado.concurrent.Future}]{\sphinxcrossref{\sphinxcode{\sphinxupquote{Future}}}}} returned by \sphinxcode{\sphinxupquote{multi()}} does not cancel its
children. \sphinxhref{https://docs.python.org/3.6/library/asyncio-task.html\#asyncio.gather}{\sphinxcode{\sphinxupquote{asyncio.gather}}} is similar to \sphinxcode{\sphinxupquote{multi()}}, but it does
cancel its children.

\DUrole{versionmodified,changed}{Changed in version 4.2: }If multiple yieldables fail, any exceptions after the first
(which is raised) will be logged. Added the \sphinxcode{\sphinxupquote{quiet\_exceptions}}
argument to suppress this logging for selected exception types.

\DUrole{versionmodified,changed}{Changed in version 4.3: }Replaced the class \sphinxcode{\sphinxupquote{Multi}} and the function \sphinxcode{\sphinxupquote{multi\_future}}
with a unified function \sphinxcode{\sphinxupquote{multi}}. Added support for yieldables
other than \sphinxcode{\sphinxupquote{YieldPoint}} and {\hyperref[\detokenize{concurrent:tornado.concurrent.Future}]{\sphinxcrossref{\sphinxcode{\sphinxupquote{Future}}}}}.

\end{fulllineitems}

\index{multi\_future() (in module tornado.gen)@\spxentry{multi\_future()}\spxextra{in module tornado.gen}}

\begin{fulllineitems}
\phantomsection\label{\detokenize{gen:tornado.gen.multi_future}}\pysiglinewithargsret{\sphinxcode{\sphinxupquote{tornado.gen.}}\sphinxbfcode{\sphinxupquote{multi\_future}}}{\emph{Union{[}List{[}Yieldable{]}, Dict{[}Any, Yieldable{]}{]}, quiet\_exceptions: Union{[}Type{[}Exception{]}, Tuple{[}Type{[}Exception{]}, ...{]}{]} = ()}}{}
Wait for multiple asynchronous futures in parallel.

Since Tornado 6.0, this function is exactly the same as {\hyperref[\detokenize{gen:tornado.gen.multi}]{\sphinxcrossref{\sphinxcode{\sphinxupquote{multi}}}}}.

\DUrole{versionmodified,added}{New in version 4.0.}

\DUrole{versionmodified,changed}{Changed in version 4.2: }If multiple \sphinxcode{\sphinxupquote{Futures}} fail, any exceptions after the first (which is
raised) will be logged. Added the \sphinxcode{\sphinxupquote{quiet\_exceptions}}
argument to suppress this logging for selected exception types.

\DUrole{versionmodified,deprecated}{Deprecated since version 4.3: }Use {\hyperref[\detokenize{gen:tornado.gen.multi}]{\sphinxcrossref{\sphinxcode{\sphinxupquote{multi}}}}} instead.

\end{fulllineitems}

\index{convert\_yielded() (in module tornado.gen)@\spxentry{convert\_yielded()}\spxextra{in module tornado.gen}}

\begin{fulllineitems}
\phantomsection\label{\detokenize{gen:tornado.gen.convert_yielded}}\pysiglinewithargsret{\sphinxcode{\sphinxupquote{tornado.gen.}}\sphinxbfcode{\sphinxupquote{convert\_yielded}}}{\emph{yielded: Union{[}None, Awaitable{[}T\_co{]}, List{[}Awaitable{[}T\_co{]}{]}, Dict{[}Any, Awaitable{[}T\_co{]}{]}, concurrent.futures.\_base.Future{]}}}{{ $\rightarrow$ \_asyncio.Future}}
Convert a yielded object into a {\hyperref[\detokenize{concurrent:tornado.concurrent.Future}]{\sphinxcrossref{\sphinxcode{\sphinxupquote{Future}}}}}.

The default implementation accepts lists, dictionaries, and
Futures. This has the side effect of starting any coroutines that
did not start themselves, similar to \sphinxhref{https://docs.python.org/3.6/library/asyncio-task.html\#asyncio.ensure\_future}{\sphinxcode{\sphinxupquote{asyncio.ensure\_future}}}.

If the \sphinxhref{https://docs.python.org/3.6/library/functools.html\#functools.singledispatch}{\sphinxcode{\sphinxupquote{singledispatch}}} library is available, this function
may be extended to support additional types. For example:

\begin{sphinxVerbatim}[commandchars=\\\{\}]
\PYG{n+nd}{@convert\PYGZus{}yielded}\PYG{o}{.}\PYG{n}{register}\PYG{p}{(}\PYG{n}{asyncio}\PYG{o}{.}\PYG{n}{Future}\PYG{p}{)}
\PYG{k}{def} \PYG{n+nf}{\PYGZus{}}\PYG{p}{(}\PYG{n}{asyncio\PYGZus{}future}\PYG{p}{)}\PYG{p}{:}
    \PYG{k}{return} \PYG{n}{tornado}\PYG{o}{.}\PYG{n}{platform}\PYG{o}{.}\PYG{n}{asyncio}\PYG{o}{.}\PYG{n}{to\PYGZus{}tornado\PYGZus{}future}\PYG{p}{(}\PYG{n}{asyncio\PYGZus{}future}\PYG{p}{)}
\end{sphinxVerbatim}

\DUrole{versionmodified,added}{New in version 4.1.}

\end{fulllineitems}

\index{maybe\_future() (in module tornado.gen)@\spxentry{maybe\_future()}\spxextra{in module tornado.gen}}

\begin{fulllineitems}
\phantomsection\label{\detokenize{gen:tornado.gen.maybe_future}}\pysiglinewithargsret{\sphinxcode{\sphinxupquote{tornado.gen.}}\sphinxbfcode{\sphinxupquote{maybe\_future}}}{\emph{x: Any}}{{ $\rightarrow$ \_asyncio.Future}}
Converts \sphinxcode{\sphinxupquote{x}} into a {\hyperref[\detokenize{concurrent:tornado.concurrent.Future}]{\sphinxcrossref{\sphinxcode{\sphinxupquote{Future}}}}}.

If \sphinxcode{\sphinxupquote{x}} is already a {\hyperref[\detokenize{concurrent:tornado.concurrent.Future}]{\sphinxcrossref{\sphinxcode{\sphinxupquote{Future}}}}}, it is simply returned; otherwise
it is wrapped in a new {\hyperref[\detokenize{concurrent:tornado.concurrent.Future}]{\sphinxcrossref{\sphinxcode{\sphinxupquote{Future}}}}}.  This is suitable for use as
\sphinxcode{\sphinxupquote{result = yield gen.maybe\_future(f())}} when you don’t know whether
\sphinxcode{\sphinxupquote{f()}} returns a {\hyperref[\detokenize{concurrent:tornado.concurrent.Future}]{\sphinxcrossref{\sphinxcode{\sphinxupquote{Future}}}}} or not.

\DUrole{versionmodified,deprecated}{Deprecated since version 4.3: }This function only handles \sphinxcode{\sphinxupquote{Futures}}, not other yieldable objects.
Instead of {\hyperref[\detokenize{gen:tornado.gen.maybe_future}]{\sphinxcrossref{\sphinxcode{\sphinxupquote{maybe\_future}}}}}, check for the non-future result types
you expect (often just \sphinxcode{\sphinxupquote{None}}), and \sphinxcode{\sphinxupquote{yield}} anything unknown.

\end{fulllineitems}

\index{is\_coroutine\_function() (in module tornado.gen)@\spxentry{is\_coroutine\_function()}\spxextra{in module tornado.gen}}

\begin{fulllineitems}
\phantomsection\label{\detokenize{gen:tornado.gen.is_coroutine_function}}\pysiglinewithargsret{\sphinxcode{\sphinxupquote{tornado.gen.}}\sphinxbfcode{\sphinxupquote{is\_coroutine\_function}}}{\emph{func: Any}}{{ $\rightarrow$ bool}}
Return whether \sphinxstyleemphasis{func} is a coroutine function, i.e. a function
wrapped with {\hyperref[\detokenize{gen:tornado.gen.coroutine}]{\sphinxcrossref{\sphinxcode{\sphinxupquote{coroutine}}}}}.

\DUrole{versionmodified,added}{New in version 4.5.}

\end{fulllineitems}

\index{moment (in module tornado.gen)@\spxentry{moment}\spxextra{in module tornado.gen}}

\begin{fulllineitems}
\phantomsection\label{\detokenize{gen:tornado.gen.moment}}\pysigline{\sphinxcode{\sphinxupquote{tornado.gen.}}\sphinxbfcode{\sphinxupquote{moment}}}
A special object which may be yielded to allow the IOLoop to run for
one iteration.

This is not needed in normal use but it can be helpful in long-running
coroutines that are likely to yield Futures that are ready instantly.

Usage: \sphinxcode{\sphinxupquote{yield gen.moment}}

In native coroutines, the equivalent of \sphinxcode{\sphinxupquote{yield gen.moment}} is
\sphinxcode{\sphinxupquote{await asyncio.sleep(0)}}.

\DUrole{versionmodified,added}{New in version 4.0.}

\DUrole{versionmodified,deprecated}{Deprecated since version 4.5: }\sphinxcode{\sphinxupquote{yield None}} (or \sphinxcode{\sphinxupquote{yield}} with no argument) is now equivalent to
 \sphinxcode{\sphinxupquote{yield gen.moment}}.

\end{fulllineitems}



\subsection{\sphinxstyleliteralintitle{\sphinxupquote{tornado.locks}} \textendash{} Synchronization primitives}
\label{\detokenize{locks:tornado-locks-synchronization-primitives}}\label{\detokenize{locks::doc}}
\DUrole{versionmodified,added}{New in version 4.2.}

Coordinate coroutines with synchronization primitives analogous to
those the standard library provides to threads. These classes are very
similar to those provided in the standard library’s \sphinxhref{https://docs.python.org/3/library/asyncio-sync.html}{asyncio package}.

\begin{sphinxadmonition}{warning}{Warning:}
Note that these primitives are not actually thread-safe and cannot
be used in place of those from the standard library’s \sphinxhref{https://docs.python.org/3.6/library/threading.html\#module-threading}{\sphinxcode{\sphinxupquote{threading}}}
module\textendash{}they are meant to coordinate Tornado coroutines in a
single-threaded app, not to protect shared objects in a
multithreaded app.
\end{sphinxadmonition}
\phantomsection\label{\detokenize{locks:module-tornado.locks}}\index{tornado.locks (module)@\spxentry{tornado.locks}\spxextra{module}}

\subsubsection{Condition}
\label{\detokenize{locks:condition}}\index{Condition (class in tornado.locks)@\spxentry{Condition}\spxextra{class in tornado.locks}}

\begin{fulllineitems}
\phantomsection\label{\detokenize{locks:tornado.locks.Condition}}\pysigline{\sphinxbfcode{\sphinxupquote{class }}\sphinxcode{\sphinxupquote{tornado.locks.}}\sphinxbfcode{\sphinxupquote{Condition}}}
A condition allows one or more coroutines to wait until notified.

Like a standard \sphinxhref{https://docs.python.org/3.6/library/threading.html\#threading.Condition}{\sphinxcode{\sphinxupquote{threading.Condition}}}, but does not need an underlying lock
that is acquired and released.

With a {\hyperref[\detokenize{locks:tornado.locks.Condition}]{\sphinxcrossref{\sphinxcode{\sphinxupquote{Condition}}}}}, coroutines can wait to be notified by other coroutines:

\begin{sphinxVerbatim}[commandchars=\\\{\}]
\PYG{k+kn}{from} \PYG{n+nn}{tornado} \PYG{k}{import} \PYG{n}{gen}
\PYG{k+kn}{from} \PYG{n+nn}{tornado}\PYG{n+nn}{.}\PYG{n+nn}{ioloop} \PYG{k}{import} \PYG{n}{IOLoop}
\PYG{k+kn}{from} \PYG{n+nn}{tornado}\PYG{n+nn}{.}\PYG{n+nn}{locks} \PYG{k}{import} \PYG{n}{Condition}

\PYG{n}{condition} \PYG{o}{=} \PYG{n}{Condition}\PYG{p}{(}\PYG{p}{)}

\PYG{k}{async} \PYG{k}{def} \PYG{n+nf}{waiter}\PYG{p}{(}\PYG{p}{)}\PYG{p}{:}
    \PYG{n+nb}{print}\PYG{p}{(}\PYG{l+s+s2}{\PYGZdq{}}\PYG{l+s+s2}{I}\PYG{l+s+s2}{\PYGZsq{}}\PYG{l+s+s2}{ll wait right here}\PYG{l+s+s2}{\PYGZdq{}}\PYG{p}{)}
    \PYG{k}{await} \PYG{n}{condition}\PYG{o}{.}\PYG{n}{wait}\PYG{p}{(}\PYG{p}{)}
    \PYG{n+nb}{print}\PYG{p}{(}\PYG{l+s+s2}{\PYGZdq{}}\PYG{l+s+s2}{I}\PYG{l+s+s2}{\PYGZsq{}}\PYG{l+s+s2}{m done waiting}\PYG{l+s+s2}{\PYGZdq{}}\PYG{p}{)}

\PYG{k}{async} \PYG{k}{def} \PYG{n+nf}{notifier}\PYG{p}{(}\PYG{p}{)}\PYG{p}{:}
    \PYG{n+nb}{print}\PYG{p}{(}\PYG{l+s+s2}{\PYGZdq{}}\PYG{l+s+s2}{About to notify}\PYG{l+s+s2}{\PYGZdq{}}\PYG{p}{)}
    \PYG{n}{condition}\PYG{o}{.}\PYG{n}{notify}\PYG{p}{(}\PYG{p}{)}
    \PYG{n+nb}{print}\PYG{p}{(}\PYG{l+s+s2}{\PYGZdq{}}\PYG{l+s+s2}{Done notifying}\PYG{l+s+s2}{\PYGZdq{}}\PYG{p}{)}

\PYG{k}{async} \PYG{k}{def} \PYG{n+nf}{runner}\PYG{p}{(}\PYG{p}{)}\PYG{p}{:}
    \PYG{c+c1}{\PYGZsh{} Wait for waiter() and notifier() in parallel}
    \PYG{k}{await} \PYG{n}{gen}\PYG{o}{.}\PYG{n}{multi}\PYG{p}{(}\PYG{p}{[}\PYG{n}{waiter}\PYG{p}{(}\PYG{p}{)}\PYG{p}{,} \PYG{n}{notifier}\PYG{p}{(}\PYG{p}{)}\PYG{p}{]}\PYG{p}{)}

\PYG{n}{IOLoop}\PYG{o}{.}\PYG{n}{current}\PYG{p}{(}\PYG{p}{)}\PYG{o}{.}\PYG{n}{run\PYGZus{}sync}\PYG{p}{(}\PYG{n}{runner}\PYG{p}{)}
\end{sphinxVerbatim}

\begin{sphinxVerbatim}[commandchars=\\\{\}]
I\PYGZsq{}ll wait right here
About to notify
Done notifying
I\PYGZsq{}m done waiting
\end{sphinxVerbatim}

{\hyperref[\detokenize{locks:tornado.locks.Condition.wait}]{\sphinxcrossref{\sphinxcode{\sphinxupquote{wait}}}}} takes an optional \sphinxcode{\sphinxupquote{timeout}} argument, which is either an absolute
timestamp:

\begin{sphinxVerbatim}[commandchars=\\\{\}]
\PYG{n}{io\PYGZus{}loop} \PYG{o}{=} \PYG{n}{IOLoop}\PYG{o}{.}\PYG{n}{current}\PYG{p}{(}\PYG{p}{)}

\PYG{c+c1}{\PYGZsh{} Wait up to 1 second for a notification.}
\PYG{k}{await} \PYG{n}{condition}\PYG{o}{.}\PYG{n}{wait}\PYG{p}{(}\PYG{n}{timeout}\PYG{o}{=}\PYG{n}{io\PYGZus{}loop}\PYG{o}{.}\PYG{n}{time}\PYG{p}{(}\PYG{p}{)} \PYG{o}{+} \PYG{l+m+mi}{1}\PYG{p}{)}
\end{sphinxVerbatim}

…or a \sphinxhref{https://docs.python.org/3.6/library/datetime.html\#datetime.timedelta}{\sphinxcode{\sphinxupquote{datetime.timedelta}}} for a timeout relative to the current time:

\begin{sphinxVerbatim}[commandchars=\\\{\}]
\PYG{c+c1}{\PYGZsh{} Wait up to 1 second.}
\PYG{k}{await} \PYG{n}{condition}\PYG{o}{.}\PYG{n}{wait}\PYG{p}{(}\PYG{n}{timeout}\PYG{o}{=}\PYG{n}{datetime}\PYG{o}{.}\PYG{n}{timedelta}\PYG{p}{(}\PYG{n}{seconds}\PYG{o}{=}\PYG{l+m+mi}{1}\PYG{p}{)}\PYG{p}{)}
\end{sphinxVerbatim}

The method returns False if there’s no notification before the deadline.

\DUrole{versionmodified,changed}{Changed in version 5.0: }Previously, waiters could be notified synchronously from within
{\hyperref[\detokenize{locks:tornado.locks.Condition.notify}]{\sphinxcrossref{\sphinxcode{\sphinxupquote{notify}}}}}. Now, the notification will always be received on the
next iteration of the {\hyperref[\detokenize{ioloop:tornado.ioloop.IOLoop}]{\sphinxcrossref{\sphinxcode{\sphinxupquote{IOLoop}}}}}.
\index{wait() (tornado.locks.Condition method)@\spxentry{wait()}\spxextra{tornado.locks.Condition method}}

\begin{fulllineitems}
\phantomsection\label{\detokenize{locks:tornado.locks.Condition.wait}}\pysiglinewithargsret{\sphinxbfcode{\sphinxupquote{wait}}}{\emph{timeout: Union{[}float}, \emph{datetime.timedelta{]} = None}}{{ $\rightarrow$ Awaitable{[}bool{]}}}
Wait for {\hyperref[\detokenize{locks:tornado.locks.Condition.notify}]{\sphinxcrossref{\sphinxcode{\sphinxupquote{notify}}}}}.

Returns a {\hyperref[\detokenize{concurrent:tornado.concurrent.Future}]{\sphinxcrossref{\sphinxcode{\sphinxupquote{Future}}}}} that resolves \sphinxcode{\sphinxupquote{True}} if the condition is notified,
or \sphinxcode{\sphinxupquote{False}} after a timeout.

\end{fulllineitems}

\index{notify() (tornado.locks.Condition method)@\spxentry{notify()}\spxextra{tornado.locks.Condition method}}

\begin{fulllineitems}
\phantomsection\label{\detokenize{locks:tornado.locks.Condition.notify}}\pysiglinewithargsret{\sphinxbfcode{\sphinxupquote{notify}}}{\emph{n: int = 1}}{{ $\rightarrow$ None}}
Wake \sphinxcode{\sphinxupquote{n}} waiters.

\end{fulllineitems}

\index{notify\_all() (tornado.locks.Condition method)@\spxentry{notify\_all()}\spxextra{tornado.locks.Condition method}}

\begin{fulllineitems}
\phantomsection\label{\detokenize{locks:tornado.locks.Condition.notify_all}}\pysiglinewithargsret{\sphinxbfcode{\sphinxupquote{notify\_all}}}{}{{ $\rightarrow$ None}}
Wake all waiters.

\end{fulllineitems}


\end{fulllineitems}



\subsubsection{Event}
\label{\detokenize{locks:event}}\index{Event (class in tornado.locks)@\spxentry{Event}\spxextra{class in tornado.locks}}

\begin{fulllineitems}
\phantomsection\label{\detokenize{locks:tornado.locks.Event}}\pysigline{\sphinxbfcode{\sphinxupquote{class }}\sphinxcode{\sphinxupquote{tornado.locks.}}\sphinxbfcode{\sphinxupquote{Event}}}
An event blocks coroutines until its internal flag is set to True.

Similar to \sphinxhref{https://docs.python.org/3.6/library/threading.html\#threading.Event}{\sphinxcode{\sphinxupquote{threading.Event}}}.

A coroutine can wait for an event to be set. Once it is set, calls to
\sphinxcode{\sphinxupquote{yield event.wait()}} will not block unless the event has been cleared:

\begin{sphinxVerbatim}[commandchars=\\\{\}]
\PYG{k+kn}{from} \PYG{n+nn}{tornado} \PYG{k}{import} \PYG{n}{gen}
\PYG{k+kn}{from} \PYG{n+nn}{tornado}\PYG{n+nn}{.}\PYG{n+nn}{ioloop} \PYG{k}{import} \PYG{n}{IOLoop}
\PYG{k+kn}{from} \PYG{n+nn}{tornado}\PYG{n+nn}{.}\PYG{n+nn}{locks} \PYG{k}{import} \PYG{n}{Event}

\PYG{n}{event} \PYG{o}{=} \PYG{n}{Event}\PYG{p}{(}\PYG{p}{)}

\PYG{k}{async} \PYG{k}{def} \PYG{n+nf}{waiter}\PYG{p}{(}\PYG{p}{)}\PYG{p}{:}
    \PYG{n+nb}{print}\PYG{p}{(}\PYG{l+s+s2}{\PYGZdq{}}\PYG{l+s+s2}{Waiting for event}\PYG{l+s+s2}{\PYGZdq{}}\PYG{p}{)}
    \PYG{k}{await} \PYG{n}{event}\PYG{o}{.}\PYG{n}{wait}\PYG{p}{(}\PYG{p}{)}
    \PYG{n+nb}{print}\PYG{p}{(}\PYG{l+s+s2}{\PYGZdq{}}\PYG{l+s+s2}{Not waiting this time}\PYG{l+s+s2}{\PYGZdq{}}\PYG{p}{)}
    \PYG{k}{await} \PYG{n}{event}\PYG{o}{.}\PYG{n}{wait}\PYG{p}{(}\PYG{p}{)}
    \PYG{n+nb}{print}\PYG{p}{(}\PYG{l+s+s2}{\PYGZdq{}}\PYG{l+s+s2}{Done}\PYG{l+s+s2}{\PYGZdq{}}\PYG{p}{)}

\PYG{k}{async} \PYG{k}{def} \PYG{n+nf}{setter}\PYG{p}{(}\PYG{p}{)}\PYG{p}{:}
    \PYG{n+nb}{print}\PYG{p}{(}\PYG{l+s+s2}{\PYGZdq{}}\PYG{l+s+s2}{About to set the event}\PYG{l+s+s2}{\PYGZdq{}}\PYG{p}{)}
    \PYG{n}{event}\PYG{o}{.}\PYG{n}{set}\PYG{p}{(}\PYG{p}{)}

\PYG{k}{async} \PYG{k}{def} \PYG{n+nf}{runner}\PYG{p}{(}\PYG{p}{)}\PYG{p}{:}
    \PYG{k}{await} \PYG{n}{gen}\PYG{o}{.}\PYG{n}{multi}\PYG{p}{(}\PYG{p}{[}\PYG{n}{waiter}\PYG{p}{(}\PYG{p}{)}\PYG{p}{,} \PYG{n}{setter}\PYG{p}{(}\PYG{p}{)}\PYG{p}{]}\PYG{p}{)}

\PYG{n}{IOLoop}\PYG{o}{.}\PYG{n}{current}\PYG{p}{(}\PYG{p}{)}\PYG{o}{.}\PYG{n}{run\PYGZus{}sync}\PYG{p}{(}\PYG{n}{runner}\PYG{p}{)}
\end{sphinxVerbatim}

\begin{sphinxVerbatim}[commandchars=\\\{\}]
Waiting for event
About to set the event
Not waiting this time
Done
\end{sphinxVerbatim}
\index{is\_set() (tornado.locks.Event method)@\spxentry{is\_set()}\spxextra{tornado.locks.Event method}}

\begin{fulllineitems}
\phantomsection\label{\detokenize{locks:tornado.locks.Event.is_set}}\pysiglinewithargsret{\sphinxbfcode{\sphinxupquote{is\_set}}}{}{{ $\rightarrow$ bool}}
Return \sphinxcode{\sphinxupquote{True}} if the internal flag is true.

\end{fulllineitems}

\index{set() (tornado.locks.Event method)@\spxentry{set()}\spxextra{tornado.locks.Event method}}

\begin{fulllineitems}
\phantomsection\label{\detokenize{locks:tornado.locks.Event.set}}\pysiglinewithargsret{\sphinxbfcode{\sphinxupquote{set}}}{}{{ $\rightarrow$ None}}
Set the internal flag to \sphinxcode{\sphinxupquote{True}}. All waiters are awakened.

Calling {\hyperref[\detokenize{locks:tornado.locks.Event.wait}]{\sphinxcrossref{\sphinxcode{\sphinxupquote{wait}}}}} once the flag is set will not block.

\end{fulllineitems}

\index{clear() (tornado.locks.Event method)@\spxentry{clear()}\spxextra{tornado.locks.Event method}}

\begin{fulllineitems}
\phantomsection\label{\detokenize{locks:tornado.locks.Event.clear}}\pysiglinewithargsret{\sphinxbfcode{\sphinxupquote{clear}}}{}{{ $\rightarrow$ None}}
Reset the internal flag to \sphinxcode{\sphinxupquote{False}}.

Calls to {\hyperref[\detokenize{locks:tornado.locks.Event.wait}]{\sphinxcrossref{\sphinxcode{\sphinxupquote{wait}}}}} will block until {\hyperref[\detokenize{locks:tornado.locks.Event.set}]{\sphinxcrossref{\sphinxcode{\sphinxupquote{set}}}}} is called.

\end{fulllineitems}

\index{wait() (tornado.locks.Event method)@\spxentry{wait()}\spxextra{tornado.locks.Event method}}

\begin{fulllineitems}
\phantomsection\label{\detokenize{locks:tornado.locks.Event.wait}}\pysiglinewithargsret{\sphinxbfcode{\sphinxupquote{wait}}}{\emph{timeout: Union{[}float}, \emph{datetime.timedelta{]} = None}}{{ $\rightarrow$ Awaitable{[}None{]}}}
Block until the internal flag is true.

Returns an awaitable, which raises {\hyperref[\detokenize{util:tornado.util.TimeoutError}]{\sphinxcrossref{\sphinxcode{\sphinxupquote{tornado.util.TimeoutError}}}}} after a
timeout.

\end{fulllineitems}


\end{fulllineitems}



\subsubsection{Semaphore}
\label{\detokenize{locks:semaphore}}\index{Semaphore (class in tornado.locks)@\spxentry{Semaphore}\spxextra{class in tornado.locks}}

\begin{fulllineitems}
\phantomsection\label{\detokenize{locks:tornado.locks.Semaphore}}\pysiglinewithargsret{\sphinxbfcode{\sphinxupquote{class }}\sphinxcode{\sphinxupquote{tornado.locks.}}\sphinxbfcode{\sphinxupquote{Semaphore}}}{\emph{value: int = 1}}{}
A lock that can be acquired a fixed number of times before blocking.

A Semaphore manages a counter representing the number of {\hyperref[\detokenize{locks:tornado.locks.Semaphore.release}]{\sphinxcrossref{\sphinxcode{\sphinxupquote{release}}}}} calls
minus the number of {\hyperref[\detokenize{locks:tornado.locks.Semaphore.acquire}]{\sphinxcrossref{\sphinxcode{\sphinxupquote{acquire}}}}} calls, plus an initial value. The {\hyperref[\detokenize{locks:tornado.locks.Semaphore.acquire}]{\sphinxcrossref{\sphinxcode{\sphinxupquote{acquire}}}}}
method blocks if necessary until it can return without making the counter
negative.

Semaphores limit access to a shared resource. To allow access for two
workers at a time:

\begin{sphinxVerbatim}[commandchars=\\\{\}]
\PYG{k+kn}{from} \PYG{n+nn}{tornado} \PYG{k}{import} \PYG{n}{gen}
\PYG{k+kn}{from} \PYG{n+nn}{tornado}\PYG{n+nn}{.}\PYG{n+nn}{ioloop} \PYG{k}{import} \PYG{n}{IOLoop}
\PYG{k+kn}{from} \PYG{n+nn}{tornado}\PYG{n+nn}{.}\PYG{n+nn}{locks} \PYG{k}{import} \PYG{n}{Semaphore}

\PYG{n}{sem} \PYG{o}{=} \PYG{n}{Semaphore}\PYG{p}{(}\PYG{l+m+mi}{2}\PYG{p}{)}

\PYG{k}{async} \PYG{k}{def} \PYG{n+nf}{worker}\PYG{p}{(}\PYG{n}{worker\PYGZus{}id}\PYG{p}{)}\PYG{p}{:}
    \PYG{k}{await} \PYG{n}{sem}\PYG{o}{.}\PYG{n}{acquire}\PYG{p}{(}\PYG{p}{)}
    \PYG{k}{try}\PYG{p}{:}
        \PYG{n+nb}{print}\PYG{p}{(}\PYG{l+s+s2}{\PYGZdq{}}\PYG{l+s+s2}{Worker }\PYG{l+s+si}{\PYGZpc{}d}\PYG{l+s+s2}{ is working}\PYG{l+s+s2}{\PYGZdq{}} \PYG{o}{\PYGZpc{}} \PYG{n}{worker\PYGZus{}id}\PYG{p}{)}
        \PYG{k}{await} \PYG{n}{use\PYGZus{}some\PYGZus{}resource}\PYG{p}{(}\PYG{p}{)}
    \PYG{k}{finally}\PYG{p}{:}
        \PYG{n+nb}{print}\PYG{p}{(}\PYG{l+s+s2}{\PYGZdq{}}\PYG{l+s+s2}{Worker }\PYG{l+s+si}{\PYGZpc{}d}\PYG{l+s+s2}{ is done}\PYG{l+s+s2}{\PYGZdq{}} \PYG{o}{\PYGZpc{}} \PYG{n}{worker\PYGZus{}id}\PYG{p}{)}
        \PYG{n}{sem}\PYG{o}{.}\PYG{n}{release}\PYG{p}{(}\PYG{p}{)}

\PYG{k}{async} \PYG{k}{def} \PYG{n+nf}{runner}\PYG{p}{(}\PYG{p}{)}\PYG{p}{:}
    \PYG{c+c1}{\PYGZsh{} Join all workers.}
    \PYG{k}{await} \PYG{n}{gen}\PYG{o}{.}\PYG{n}{multi}\PYG{p}{(}\PYG{p}{[}\PYG{n}{worker}\PYG{p}{(}\PYG{n}{i}\PYG{p}{)} \PYG{k}{for} \PYG{n}{i} \PYG{o+ow}{in} \PYG{n+nb}{range}\PYG{p}{(}\PYG{l+m+mi}{3}\PYG{p}{)}\PYG{p}{]}\PYG{p}{)}

\PYG{n}{IOLoop}\PYG{o}{.}\PYG{n}{current}\PYG{p}{(}\PYG{p}{)}\PYG{o}{.}\PYG{n}{run\PYGZus{}sync}\PYG{p}{(}\PYG{n}{runner}\PYG{p}{)}
\end{sphinxVerbatim}

\begin{sphinxVerbatim}[commandchars=\\\{\}]
Worker 0 is working
Worker 1 is working
Worker 0 is done
Worker 2 is working
Worker 1 is done
Worker 2 is done
\end{sphinxVerbatim}

Workers 0 and 1 are allowed to run concurrently, but worker 2 waits until
the semaphore has been released once, by worker 0.

The semaphore can be used as an async context manager:

\begin{sphinxVerbatim}[commandchars=\\\{\}]
\PYG{k}{async} \PYG{k}{def} \PYG{n+nf}{worker}\PYG{p}{(}\PYG{n}{worker\PYGZus{}id}\PYG{p}{)}\PYG{p}{:}
    \PYG{k}{async} \PYG{k}{with} \PYG{n}{sem}\PYG{p}{:}
        \PYG{n+nb}{print}\PYG{p}{(}\PYG{l+s+s2}{\PYGZdq{}}\PYG{l+s+s2}{Worker }\PYG{l+s+si}{\PYGZpc{}d}\PYG{l+s+s2}{ is working}\PYG{l+s+s2}{\PYGZdq{}} \PYG{o}{\PYGZpc{}} \PYG{n}{worker\PYGZus{}id}\PYG{p}{)}
        \PYG{k}{await} \PYG{n}{use\PYGZus{}some\PYGZus{}resource}\PYG{p}{(}\PYG{p}{)}

    \PYG{c+c1}{\PYGZsh{} Now the semaphore has been released.}
    \PYG{n+nb}{print}\PYG{p}{(}\PYG{l+s+s2}{\PYGZdq{}}\PYG{l+s+s2}{Worker }\PYG{l+s+si}{\PYGZpc{}d}\PYG{l+s+s2}{ is done}\PYG{l+s+s2}{\PYGZdq{}} \PYG{o}{\PYGZpc{}} \PYG{n}{worker\PYGZus{}id}\PYG{p}{)}
\end{sphinxVerbatim}

For compatibility with older versions of Python, {\hyperref[\detokenize{locks:tornado.locks.Semaphore.acquire}]{\sphinxcrossref{\sphinxcode{\sphinxupquote{acquire}}}}} is a
context manager, so \sphinxcode{\sphinxupquote{worker}} could also be written as:

\begin{sphinxVerbatim}[commandchars=\\\{\}]
\PYG{n+nd}{@gen}\PYG{o}{.}\PYG{n}{coroutine}
\PYG{k}{def} \PYG{n+nf}{worker}\PYG{p}{(}\PYG{n}{worker\PYGZus{}id}\PYG{p}{)}\PYG{p}{:}
    \PYG{k}{with} \PYG{p}{(}\PYG{k}{yield} \PYG{n}{sem}\PYG{o}{.}\PYG{n}{acquire}\PYG{p}{(}\PYG{p}{)}\PYG{p}{)}\PYG{p}{:}
        \PYG{n+nb}{print}\PYG{p}{(}\PYG{l+s+s2}{\PYGZdq{}}\PYG{l+s+s2}{Worker }\PYG{l+s+si}{\PYGZpc{}d}\PYG{l+s+s2}{ is working}\PYG{l+s+s2}{\PYGZdq{}} \PYG{o}{\PYGZpc{}} \PYG{n}{worker\PYGZus{}id}\PYG{p}{)}
        \PYG{k}{yield} \PYG{n}{use\PYGZus{}some\PYGZus{}resource}\PYG{p}{(}\PYG{p}{)}

    \PYG{c+c1}{\PYGZsh{} Now the semaphore has been released.}
    \PYG{n+nb}{print}\PYG{p}{(}\PYG{l+s+s2}{\PYGZdq{}}\PYG{l+s+s2}{Worker }\PYG{l+s+si}{\PYGZpc{}d}\PYG{l+s+s2}{ is done}\PYG{l+s+s2}{\PYGZdq{}} \PYG{o}{\PYGZpc{}} \PYG{n}{worker\PYGZus{}id}\PYG{p}{)}
\end{sphinxVerbatim}

\DUrole{versionmodified,changed}{Changed in version 4.3: }Added \sphinxcode{\sphinxupquote{async with}} support in Python 3.5.
\index{release() (tornado.locks.Semaphore method)@\spxentry{release()}\spxextra{tornado.locks.Semaphore method}}

\begin{fulllineitems}
\phantomsection\label{\detokenize{locks:tornado.locks.Semaphore.release}}\pysiglinewithargsret{\sphinxbfcode{\sphinxupquote{release}}}{}{{ $\rightarrow$ None}}
Increment the counter and wake one waiter.

\end{fulllineitems}

\index{acquire() (tornado.locks.Semaphore method)@\spxentry{acquire()}\spxextra{tornado.locks.Semaphore method}}

\begin{fulllineitems}
\phantomsection\label{\detokenize{locks:tornado.locks.Semaphore.acquire}}\pysiglinewithargsret{\sphinxbfcode{\sphinxupquote{acquire}}}{\emph{timeout: Union{[}float}, \emph{datetime.timedelta{]} = None}}{{ $\rightarrow$ Awaitable{[}tornado.locks.\_ReleasingContextManager{]}}}
Decrement the counter. Returns an awaitable.

Block if the counter is zero and wait for a {\hyperref[\detokenize{locks:tornado.locks.Semaphore.release}]{\sphinxcrossref{\sphinxcode{\sphinxupquote{release}}}}}. The awaitable
raises {\hyperref[\detokenize{util:tornado.util.TimeoutError}]{\sphinxcrossref{\sphinxcode{\sphinxupquote{TimeoutError}}}}} after the deadline.

\end{fulllineitems}


\end{fulllineitems}



\subsubsection{BoundedSemaphore}
\label{\detokenize{locks:boundedsemaphore}}\index{BoundedSemaphore (class in tornado.locks)@\spxentry{BoundedSemaphore}\spxextra{class in tornado.locks}}

\begin{fulllineitems}
\phantomsection\label{\detokenize{locks:tornado.locks.BoundedSemaphore}}\pysiglinewithargsret{\sphinxbfcode{\sphinxupquote{class }}\sphinxcode{\sphinxupquote{tornado.locks.}}\sphinxbfcode{\sphinxupquote{BoundedSemaphore}}}{\emph{value: int = 1}}{}
A semaphore that prevents release() being called too many times.

If {\hyperref[\detokenize{locks:tornado.locks.BoundedSemaphore.release}]{\sphinxcrossref{\sphinxcode{\sphinxupquote{release}}}}} would increment the semaphore’s value past the initial
value, it raises \sphinxhref{https://docs.python.org/3.6/library/exceptions.html\#ValueError}{\sphinxcode{\sphinxupquote{ValueError}}}. Semaphores are mostly used to guard
resources with limited capacity, so a semaphore released too many times
is a sign of a bug.
\index{release() (tornado.locks.BoundedSemaphore method)@\spxentry{release()}\spxextra{tornado.locks.BoundedSemaphore method}}

\begin{fulllineitems}
\phantomsection\label{\detokenize{locks:tornado.locks.BoundedSemaphore.release}}\pysiglinewithargsret{\sphinxbfcode{\sphinxupquote{release}}}{}{{ $\rightarrow$ None}}
Increment the counter and wake one waiter.

\end{fulllineitems}

\index{acquire() (tornado.locks.BoundedSemaphore method)@\spxentry{acquire()}\spxextra{tornado.locks.BoundedSemaphore method}}

\begin{fulllineitems}
\phantomsection\label{\detokenize{locks:tornado.locks.BoundedSemaphore.acquire}}\pysiglinewithargsret{\sphinxbfcode{\sphinxupquote{acquire}}}{\emph{timeout: Union{[}float}, \emph{datetime.timedelta{]} = None}}{{ $\rightarrow$ Awaitable{[}tornado.locks.\_ReleasingContextManager{]}}}
Decrement the counter. Returns an awaitable.

Block if the counter is zero and wait for a {\hyperref[\detokenize{locks:tornado.locks.BoundedSemaphore.release}]{\sphinxcrossref{\sphinxcode{\sphinxupquote{release}}}}}. The awaitable
raises {\hyperref[\detokenize{util:tornado.util.TimeoutError}]{\sphinxcrossref{\sphinxcode{\sphinxupquote{TimeoutError}}}}} after the deadline.

\end{fulllineitems}


\end{fulllineitems}



\subsubsection{Lock}
\label{\detokenize{locks:lock}}\index{Lock (class in tornado.locks)@\spxentry{Lock}\spxextra{class in tornado.locks}}

\begin{fulllineitems}
\phantomsection\label{\detokenize{locks:tornado.locks.Lock}}\pysigline{\sphinxbfcode{\sphinxupquote{class }}\sphinxcode{\sphinxupquote{tornado.locks.}}\sphinxbfcode{\sphinxupquote{Lock}}}
A lock for coroutines.

A Lock begins unlocked, and {\hyperref[\detokenize{locks:tornado.locks.Lock.acquire}]{\sphinxcrossref{\sphinxcode{\sphinxupquote{acquire}}}}} locks it immediately. While it is
locked, a coroutine that yields {\hyperref[\detokenize{locks:tornado.locks.Lock.acquire}]{\sphinxcrossref{\sphinxcode{\sphinxupquote{acquire}}}}} waits until another coroutine
calls {\hyperref[\detokenize{locks:tornado.locks.Lock.release}]{\sphinxcrossref{\sphinxcode{\sphinxupquote{release}}}}}.

Releasing an unlocked lock raises \sphinxhref{https://docs.python.org/3.6/library/exceptions.html\#RuntimeError}{\sphinxcode{\sphinxupquote{RuntimeError}}}.

A Lock can be used as an async context manager with the \sphinxcode{\sphinxupquote{async
with}} statement:

\begin{sphinxVerbatim}[commandchars=\\\{\}]
\PYG{g+gp}{\PYGZgt{}\PYGZgt{}\PYGZgt{} }\PYG{k+kn}{from} \PYG{n+nn}{tornado} \PYG{k}{import} \PYG{n}{locks}
\PYG{g+gp}{\PYGZgt{}\PYGZgt{}\PYGZgt{} }\PYG{n}{lock} \PYG{o}{=} \PYG{n}{locks}\PYG{o}{.}\PYG{n}{Lock}\PYG{p}{(}\PYG{p}{)}
\PYG{g+go}{\PYGZgt{}\PYGZgt{}\PYGZgt{}}
\PYG{g+gp}{\PYGZgt{}\PYGZgt{}\PYGZgt{} }\PYG{k}{async} \PYG{k}{def} \PYG{n+nf}{f}\PYG{p}{(}\PYG{p}{)}\PYG{p}{:}
\PYG{g+gp}{... }   \PYG{k}{async} \PYG{k}{with} \PYG{n}{lock}\PYG{p}{:}
\PYG{g+gp}{... }       \PYG{c+c1}{\PYGZsh{} Do something holding the lock.}
\PYG{g+gp}{... }       \PYG{k}{pass}
\PYG{g+gp}{...}
\PYG{g+gp}{... }   \PYG{c+c1}{\PYGZsh{} Now the lock is released.}
\end{sphinxVerbatim}

For compatibility with older versions of Python, the {\hyperref[\detokenize{locks:tornado.locks.Lock.acquire}]{\sphinxcrossref{\sphinxcode{\sphinxupquote{acquire}}}}}
method asynchronously returns a regular context manager:

\begin{sphinxVerbatim}[commandchars=\\\{\}]
\PYG{g+gp}{\PYGZgt{}\PYGZgt{}\PYGZgt{} }\PYG{k}{async} \PYG{k}{def} \PYG{n+nf}{f2}\PYG{p}{(}\PYG{p}{)}\PYG{p}{:}
\PYG{g+gp}{... }   \PYG{k}{with} \PYG{p}{(}\PYG{k}{yield} \PYG{n}{lock}\PYG{o}{.}\PYG{n}{acquire}\PYG{p}{(}\PYG{p}{)}\PYG{p}{)}\PYG{p}{:}
\PYG{g+gp}{... }       \PYG{c+c1}{\PYGZsh{} Do something holding the lock.}
\PYG{g+gp}{... }       \PYG{k}{pass}
\PYG{g+gp}{...}
\PYG{g+gp}{... }   \PYG{c+c1}{\PYGZsh{} Now the lock is released.}
\end{sphinxVerbatim}

\DUrole{versionmodified,changed}{Changed in version 4.3: }Added \sphinxcode{\sphinxupquote{async with}} support in Python 3.5.
\index{acquire() (tornado.locks.Lock method)@\spxentry{acquire()}\spxextra{tornado.locks.Lock method}}

\begin{fulllineitems}
\phantomsection\label{\detokenize{locks:tornado.locks.Lock.acquire}}\pysiglinewithargsret{\sphinxbfcode{\sphinxupquote{acquire}}}{\emph{timeout: Union{[}float}, \emph{datetime.timedelta{]} = None}}{{ $\rightarrow$ Awaitable{[}tornado.locks.\_ReleasingContextManager{]}}}
Attempt to lock. Returns an awaitable.

Returns an awaitable, which raises {\hyperref[\detokenize{util:tornado.util.TimeoutError}]{\sphinxcrossref{\sphinxcode{\sphinxupquote{tornado.util.TimeoutError}}}}} after a
timeout.

\end{fulllineitems}

\index{release() (tornado.locks.Lock method)@\spxentry{release()}\spxextra{tornado.locks.Lock method}}

\begin{fulllineitems}
\phantomsection\label{\detokenize{locks:tornado.locks.Lock.release}}\pysiglinewithargsret{\sphinxbfcode{\sphinxupquote{release}}}{}{{ $\rightarrow$ None}}
Unlock.

The first coroutine in line waiting for {\hyperref[\detokenize{locks:tornado.locks.Lock.acquire}]{\sphinxcrossref{\sphinxcode{\sphinxupquote{acquire}}}}} gets the lock.

If not locked, raise a \sphinxhref{https://docs.python.org/3.6/library/exceptions.html\#RuntimeError}{\sphinxcode{\sphinxupquote{RuntimeError}}}.

\end{fulllineitems}


\end{fulllineitems}



\subsection{\sphinxstyleliteralintitle{\sphinxupquote{tornado.queues}} \textendash{} Queues for coroutines}
\label{\detokenize{queues:tornado-queues-queues-for-coroutines}}\label{\detokenize{queues::doc}}
\DUrole{versionmodified,added}{New in version 4.2.}
\phantomsection\label{\detokenize{queues:module-tornado.queues}}\index{tornado.queues (module)@\spxentry{tornado.queues}\spxextra{module}}
Asynchronous queues for coroutines. These classes are very similar
to those provided in the standard library’s \sphinxhref{https://docs.python.org/3/library/asyncio-queue.html}{asyncio package}.

\begin{sphinxadmonition}{warning}{Warning:}
Unlike the standard library’s \sphinxhref{https://docs.python.org/3.6/library/queue.html\#module-queue}{\sphinxcode{\sphinxupquote{queue}}} module, the classes defined here
are \sphinxstyleemphasis{not} thread-safe. To use these queues from another thread,
use {\hyperref[\detokenize{ioloop:tornado.ioloop.IOLoop.add_callback}]{\sphinxcrossref{\sphinxcode{\sphinxupquote{IOLoop.add\_callback}}}}} to transfer control to the {\hyperref[\detokenize{ioloop:tornado.ioloop.IOLoop}]{\sphinxcrossref{\sphinxcode{\sphinxupquote{IOLoop}}}}} thread
before calling any queue methods.
\end{sphinxadmonition}


\subsubsection{Classes}
\label{\detokenize{queues:classes}}

\paragraph{Queue}
\label{\detokenize{queues:queue}}\index{Queue (class in tornado.queues)@\spxentry{Queue}\spxextra{class in tornado.queues}}

\begin{fulllineitems}
\phantomsection\label{\detokenize{queues:tornado.queues.Queue}}\pysiglinewithargsret{\sphinxbfcode{\sphinxupquote{class }}\sphinxcode{\sphinxupquote{tornado.queues.}}\sphinxbfcode{\sphinxupquote{Queue}}}{\emph{maxsize: int = 0}}{}
Coordinate producer and consumer coroutines.

If maxsize is 0 (the default) the queue size is unbounded.

\begin{sphinxVerbatim}[commandchars=\\\{\}]
\PYG{k+kn}{from} \PYG{n+nn}{tornado} \PYG{k}{import} \PYG{n}{gen}
\PYG{k+kn}{from} \PYG{n+nn}{tornado}\PYG{n+nn}{.}\PYG{n+nn}{ioloop} \PYG{k}{import} \PYG{n}{IOLoop}
\PYG{k+kn}{from} \PYG{n+nn}{tornado}\PYG{n+nn}{.}\PYG{n+nn}{queues} \PYG{k}{import} \PYG{n}{Queue}

\PYG{n}{q} \PYG{o}{=} \PYG{n}{Queue}\PYG{p}{(}\PYG{n}{maxsize}\PYG{o}{=}\PYG{l+m+mi}{2}\PYG{p}{)}

\PYG{k}{async} \PYG{k}{def} \PYG{n+nf}{consumer}\PYG{p}{(}\PYG{p}{)}\PYG{p}{:}
    \PYG{k}{async} \PYG{k}{for} \PYG{n}{item} \PYG{o+ow}{in} \PYG{n}{q}\PYG{p}{:}
        \PYG{k}{try}\PYG{p}{:}
            \PYG{n+nb}{print}\PYG{p}{(}\PYG{l+s+s1}{\PYGZsq{}}\PYG{l+s+s1}{Doing work on }\PYG{l+s+si}{\PYGZpc{}s}\PYG{l+s+s1}{\PYGZsq{}} \PYG{o}{\PYGZpc{}} \PYG{n}{item}\PYG{p}{)}
            \PYG{k}{await} \PYG{n}{gen}\PYG{o}{.}\PYG{n}{sleep}\PYG{p}{(}\PYG{l+m+mf}{0.01}\PYG{p}{)}
        \PYG{k}{finally}\PYG{p}{:}
            \PYG{n}{q}\PYG{o}{.}\PYG{n}{task\PYGZus{}done}\PYG{p}{(}\PYG{p}{)}

\PYG{k}{async} \PYG{k}{def} \PYG{n+nf}{producer}\PYG{p}{(}\PYG{p}{)}\PYG{p}{:}
    \PYG{k}{for} \PYG{n}{item} \PYG{o+ow}{in} \PYG{n+nb}{range}\PYG{p}{(}\PYG{l+m+mi}{5}\PYG{p}{)}\PYG{p}{:}
        \PYG{k}{await} \PYG{n}{q}\PYG{o}{.}\PYG{n}{put}\PYG{p}{(}\PYG{n}{item}\PYG{p}{)}
        \PYG{n+nb}{print}\PYG{p}{(}\PYG{l+s+s1}{\PYGZsq{}}\PYG{l+s+s1}{Put }\PYG{l+s+si}{\PYGZpc{}s}\PYG{l+s+s1}{\PYGZsq{}} \PYG{o}{\PYGZpc{}} \PYG{n}{item}\PYG{p}{)}

\PYG{k}{async} \PYG{k}{def} \PYG{n+nf}{main}\PYG{p}{(}\PYG{p}{)}\PYG{p}{:}
    \PYG{c+c1}{\PYGZsh{} Start consumer without waiting (since it never finishes).}
    \PYG{n}{IOLoop}\PYG{o}{.}\PYG{n}{current}\PYG{p}{(}\PYG{p}{)}\PYG{o}{.}\PYG{n}{spawn\PYGZus{}callback}\PYG{p}{(}\PYG{n}{consumer}\PYG{p}{)}
    \PYG{k}{await} \PYG{n}{producer}\PYG{p}{(}\PYG{p}{)}     \PYG{c+c1}{\PYGZsh{} Wait for producer to put all tasks.}
    \PYG{k}{await} \PYG{n}{q}\PYG{o}{.}\PYG{n}{join}\PYG{p}{(}\PYG{p}{)}       \PYG{c+c1}{\PYGZsh{} Wait for consumer to finish all tasks.}
    \PYG{n+nb}{print}\PYG{p}{(}\PYG{l+s+s1}{\PYGZsq{}}\PYG{l+s+s1}{Done}\PYG{l+s+s1}{\PYGZsq{}}\PYG{p}{)}

\PYG{n}{IOLoop}\PYG{o}{.}\PYG{n}{current}\PYG{p}{(}\PYG{p}{)}\PYG{o}{.}\PYG{n}{run\PYGZus{}sync}\PYG{p}{(}\PYG{n}{main}\PYG{p}{)}
\end{sphinxVerbatim}

\begin{sphinxVerbatim}[commandchars=\\\{\}]
Put 0
Put 1
Doing work on 0
Put 2
Doing work on 1
Put 3
Doing work on 2
Put 4
Doing work on 3
Doing work on 4
Done
\end{sphinxVerbatim}

In versions of Python without native coroutines (before 3.5),
\sphinxcode{\sphinxupquote{consumer()}} could be written as:

\begin{sphinxVerbatim}[commandchars=\\\{\}]
\PYG{n+nd}{@gen}\PYG{o}{.}\PYG{n}{coroutine}
\PYG{k}{def} \PYG{n+nf}{consumer}\PYG{p}{(}\PYG{p}{)}\PYG{p}{:}
    \PYG{k}{while} \PYG{k+kc}{True}\PYG{p}{:}
        \PYG{n}{item} \PYG{o}{=} \PYG{k}{yield} \PYG{n}{q}\PYG{o}{.}\PYG{n}{get}\PYG{p}{(}\PYG{p}{)}
        \PYG{k}{try}\PYG{p}{:}
            \PYG{n+nb}{print}\PYG{p}{(}\PYG{l+s+s1}{\PYGZsq{}}\PYG{l+s+s1}{Doing work on }\PYG{l+s+si}{\PYGZpc{}s}\PYG{l+s+s1}{\PYGZsq{}} \PYG{o}{\PYGZpc{}} \PYG{n}{item}\PYG{p}{)}
            \PYG{k}{yield} \PYG{n}{gen}\PYG{o}{.}\PYG{n}{sleep}\PYG{p}{(}\PYG{l+m+mf}{0.01}\PYG{p}{)}
        \PYG{k}{finally}\PYG{p}{:}
            \PYG{n}{q}\PYG{o}{.}\PYG{n}{task\PYGZus{}done}\PYG{p}{(}\PYG{p}{)}
\end{sphinxVerbatim}

\DUrole{versionmodified,changed}{Changed in version 4.3: }Added \sphinxcode{\sphinxupquote{async for}} support in Python 3.5.
\index{maxsize (tornado.queues.Queue attribute)@\spxentry{maxsize}\spxextra{tornado.queues.Queue attribute}}

\begin{fulllineitems}
\phantomsection\label{\detokenize{queues:tornado.queues.Queue.maxsize}}\pysigline{\sphinxbfcode{\sphinxupquote{maxsize}}}
Number of items allowed in the queue.

\end{fulllineitems}

\index{qsize() (tornado.queues.Queue method)@\spxentry{qsize()}\spxextra{tornado.queues.Queue method}}

\begin{fulllineitems}
\phantomsection\label{\detokenize{queues:tornado.queues.Queue.qsize}}\pysiglinewithargsret{\sphinxbfcode{\sphinxupquote{qsize}}}{}{{ $\rightarrow$ int}}
Number of items in the queue.

\end{fulllineitems}

\index{put() (tornado.queues.Queue method)@\spxentry{put()}\spxextra{tornado.queues.Queue method}}

\begin{fulllineitems}
\phantomsection\label{\detokenize{queues:tornado.queues.Queue.put}}\pysiglinewithargsret{\sphinxbfcode{\sphinxupquote{put}}}{\emph{item: \_T}, \emph{timeout: Union{[}float}, \emph{datetime.timedelta{]} = None}}{{ $\rightarrow$ Future{[}None{]}}}
Put an item into the queue, perhaps waiting until there is room.

Returns a Future, which raises {\hyperref[\detokenize{util:tornado.util.TimeoutError}]{\sphinxcrossref{\sphinxcode{\sphinxupquote{tornado.util.TimeoutError}}}}} after a
timeout.

\sphinxcode{\sphinxupquote{timeout}} may be a number denoting a time (on the same
scale as {\hyperref[\detokenize{ioloop:tornado.ioloop.IOLoop.time}]{\sphinxcrossref{\sphinxcode{\sphinxupquote{tornado.ioloop.IOLoop.time}}}}}, normally \sphinxhref{https://docs.python.org/3.6/library/time.html\#time.time}{\sphinxcode{\sphinxupquote{time.time}}}), or a
\sphinxhref{https://docs.python.org/3.6/library/datetime.html\#datetime.timedelta}{\sphinxcode{\sphinxupquote{datetime.timedelta}}} object for a deadline relative to the
current time.

\end{fulllineitems}

\index{put\_nowait() (tornado.queues.Queue method)@\spxentry{put\_nowait()}\spxextra{tornado.queues.Queue method}}

\begin{fulllineitems}
\phantomsection\label{\detokenize{queues:tornado.queues.Queue.put_nowait}}\pysiglinewithargsret{\sphinxbfcode{\sphinxupquote{put\_nowait}}}{\emph{item: \_T}}{{ $\rightarrow$ None}}
Put an item into the queue without blocking.

If no free slot is immediately available, raise {\hyperref[\detokenize{queues:tornado.queues.QueueFull}]{\sphinxcrossref{\sphinxcode{\sphinxupquote{QueueFull}}}}}.

\end{fulllineitems}

\index{get() (tornado.queues.Queue method)@\spxentry{get()}\spxextra{tornado.queues.Queue method}}

\begin{fulllineitems}
\phantomsection\label{\detokenize{queues:tornado.queues.Queue.get}}\pysiglinewithargsret{\sphinxbfcode{\sphinxupquote{get}}}{\emph{timeout: Union{[}float}, \emph{datetime.timedelta{]} = None}}{{ $\rightarrow$ Awaitable{[}\_T{]}}}
Remove and return an item from the queue.

Returns an awaitable which resolves once an item is available, or raises
{\hyperref[\detokenize{util:tornado.util.TimeoutError}]{\sphinxcrossref{\sphinxcode{\sphinxupquote{tornado.util.TimeoutError}}}}} after a timeout.

\sphinxcode{\sphinxupquote{timeout}} may be a number denoting a time (on the same
scale as {\hyperref[\detokenize{ioloop:tornado.ioloop.IOLoop.time}]{\sphinxcrossref{\sphinxcode{\sphinxupquote{tornado.ioloop.IOLoop.time}}}}}, normally \sphinxhref{https://docs.python.org/3.6/library/time.html\#time.time}{\sphinxcode{\sphinxupquote{time.time}}}), or a
\sphinxhref{https://docs.python.org/3.6/library/datetime.html\#datetime.timedelta}{\sphinxcode{\sphinxupquote{datetime.timedelta}}} object for a deadline relative to the
current time.

\begin{sphinxadmonition}{note}{Note:}
The \sphinxcode{\sphinxupquote{timeout}} argument of this method differs from that
of the standard library’s \sphinxhref{https://docs.python.org/3.6/library/queue.html\#queue.Queue.get}{\sphinxcode{\sphinxupquote{queue.Queue.get}}}. That method
interprets numeric values as relative timeouts; this one
interprets them as absolute deadlines and requires
\sphinxcode{\sphinxupquote{timedelta}} objects for relative timeouts (consistent
with other timeouts in Tornado).
\end{sphinxadmonition}

\end{fulllineitems}

\index{get\_nowait() (tornado.queues.Queue method)@\spxentry{get\_nowait()}\spxextra{tornado.queues.Queue method}}

\begin{fulllineitems}
\phantomsection\label{\detokenize{queues:tornado.queues.Queue.get_nowait}}\pysiglinewithargsret{\sphinxbfcode{\sphinxupquote{get\_nowait}}}{}{{ $\rightarrow$ \_T}}
Remove and return an item from the queue without blocking.

Return an item if one is immediately available, else raise
{\hyperref[\detokenize{queues:tornado.queues.QueueEmpty}]{\sphinxcrossref{\sphinxcode{\sphinxupquote{QueueEmpty}}}}}.

\end{fulllineitems}

\index{task\_done() (tornado.queues.Queue method)@\spxentry{task\_done()}\spxextra{tornado.queues.Queue method}}

\begin{fulllineitems}
\phantomsection\label{\detokenize{queues:tornado.queues.Queue.task_done}}\pysiglinewithargsret{\sphinxbfcode{\sphinxupquote{task\_done}}}{}{{ $\rightarrow$ None}}
Indicate that a formerly enqueued task is complete.

Used by queue consumers. For each {\hyperref[\detokenize{queues:tornado.queues.Queue.get}]{\sphinxcrossref{\sphinxcode{\sphinxupquote{get}}}}} used to fetch a task, a
subsequent call to {\hyperref[\detokenize{queues:tornado.queues.Queue.task_done}]{\sphinxcrossref{\sphinxcode{\sphinxupquote{task\_done}}}}} tells the queue that the processing
on the task is complete.

If a {\hyperref[\detokenize{queues:tornado.queues.Queue.join}]{\sphinxcrossref{\sphinxcode{\sphinxupquote{join}}}}} is blocking, it resumes when all items have been
processed; that is, when every {\hyperref[\detokenize{queues:tornado.queues.Queue.put}]{\sphinxcrossref{\sphinxcode{\sphinxupquote{put}}}}} is matched by a {\hyperref[\detokenize{queues:tornado.queues.Queue.task_done}]{\sphinxcrossref{\sphinxcode{\sphinxupquote{task\_done}}}}}.

Raises \sphinxhref{https://docs.python.org/3.6/library/exceptions.html\#ValueError}{\sphinxcode{\sphinxupquote{ValueError}}} if called more times than {\hyperref[\detokenize{queues:tornado.queues.Queue.put}]{\sphinxcrossref{\sphinxcode{\sphinxupquote{put}}}}}.

\end{fulllineitems}

\index{join() (tornado.queues.Queue method)@\spxentry{join()}\spxextra{tornado.queues.Queue method}}

\begin{fulllineitems}
\phantomsection\label{\detokenize{queues:tornado.queues.Queue.join}}\pysiglinewithargsret{\sphinxbfcode{\sphinxupquote{join}}}{\emph{timeout: Union{[}float}, \emph{datetime.timedelta{]} = None}}{{ $\rightarrow$ Awaitable{[}None{]}}}
Block until all items in the queue are processed.

Returns an awaitable, which raises {\hyperref[\detokenize{util:tornado.util.TimeoutError}]{\sphinxcrossref{\sphinxcode{\sphinxupquote{tornado.util.TimeoutError}}}}} after a
timeout.

\end{fulllineitems}


\end{fulllineitems}



\paragraph{PriorityQueue}
\label{\detokenize{queues:priorityqueue}}\index{PriorityQueue (class in tornado.queues)@\spxentry{PriorityQueue}\spxextra{class in tornado.queues}}

\begin{fulllineitems}
\phantomsection\label{\detokenize{queues:tornado.queues.PriorityQueue}}\pysiglinewithargsret{\sphinxbfcode{\sphinxupquote{class }}\sphinxcode{\sphinxupquote{tornado.queues.}}\sphinxbfcode{\sphinxupquote{PriorityQueue}}}{\emph{maxsize: int = 0}}{}
A {\hyperref[\detokenize{queues:tornado.queues.Queue}]{\sphinxcrossref{\sphinxcode{\sphinxupquote{Queue}}}}} that retrieves entries in priority order, lowest first.

Entries are typically tuples like \sphinxcode{\sphinxupquote{(priority number, data)}}.

\begin{sphinxVerbatim}[commandchars=\\\{\}]
\PYG{k+kn}{from} \PYG{n+nn}{tornado}\PYG{n+nn}{.}\PYG{n+nn}{queues} \PYG{k}{import} \PYG{n}{PriorityQueue}

\PYG{n}{q} \PYG{o}{=} \PYG{n}{PriorityQueue}\PYG{p}{(}\PYG{p}{)}
\PYG{n}{q}\PYG{o}{.}\PYG{n}{put}\PYG{p}{(}\PYG{p}{(}\PYG{l+m+mi}{1}\PYG{p}{,} \PYG{l+s+s1}{\PYGZsq{}}\PYG{l+s+s1}{medium\PYGZhy{}priority item}\PYG{l+s+s1}{\PYGZsq{}}\PYG{p}{)}\PYG{p}{)}
\PYG{n}{q}\PYG{o}{.}\PYG{n}{put}\PYG{p}{(}\PYG{p}{(}\PYG{l+m+mi}{0}\PYG{p}{,} \PYG{l+s+s1}{\PYGZsq{}}\PYG{l+s+s1}{high\PYGZhy{}priority item}\PYG{l+s+s1}{\PYGZsq{}}\PYG{p}{)}\PYG{p}{)}
\PYG{n}{q}\PYG{o}{.}\PYG{n}{put}\PYG{p}{(}\PYG{p}{(}\PYG{l+m+mi}{10}\PYG{p}{,} \PYG{l+s+s1}{\PYGZsq{}}\PYG{l+s+s1}{low\PYGZhy{}priority item}\PYG{l+s+s1}{\PYGZsq{}}\PYG{p}{)}\PYG{p}{)}

\PYG{n+nb}{print}\PYG{p}{(}\PYG{n}{q}\PYG{o}{.}\PYG{n}{get\PYGZus{}nowait}\PYG{p}{(}\PYG{p}{)}\PYG{p}{)}
\PYG{n+nb}{print}\PYG{p}{(}\PYG{n}{q}\PYG{o}{.}\PYG{n}{get\PYGZus{}nowait}\PYG{p}{(}\PYG{p}{)}\PYG{p}{)}
\PYG{n+nb}{print}\PYG{p}{(}\PYG{n}{q}\PYG{o}{.}\PYG{n}{get\PYGZus{}nowait}\PYG{p}{(}\PYG{p}{)}\PYG{p}{)}
\end{sphinxVerbatim}

\begin{sphinxVerbatim}[commandchars=\\\{\}]
(0, \PYGZsq{}high\PYGZhy{}priority item\PYGZsq{})
(1, \PYGZsq{}medium\PYGZhy{}priority item\PYGZsq{})
(10, \PYGZsq{}low\PYGZhy{}priority item\PYGZsq{})
\end{sphinxVerbatim}

\end{fulllineitems}



\paragraph{LifoQueue}
\label{\detokenize{queues:lifoqueue}}\index{LifoQueue (class in tornado.queues)@\spxentry{LifoQueue}\spxextra{class in tornado.queues}}

\begin{fulllineitems}
\phantomsection\label{\detokenize{queues:tornado.queues.LifoQueue}}\pysiglinewithargsret{\sphinxbfcode{\sphinxupquote{class }}\sphinxcode{\sphinxupquote{tornado.queues.}}\sphinxbfcode{\sphinxupquote{LifoQueue}}}{\emph{maxsize: int = 0}}{}
A {\hyperref[\detokenize{queues:tornado.queues.Queue}]{\sphinxcrossref{\sphinxcode{\sphinxupquote{Queue}}}}} that retrieves the most recently put items first.

\begin{sphinxVerbatim}[commandchars=\\\{\}]
\PYG{k+kn}{from} \PYG{n+nn}{tornado}\PYG{n+nn}{.}\PYG{n+nn}{queues} \PYG{k}{import} \PYG{n}{LifoQueue}

\PYG{n}{q} \PYG{o}{=} \PYG{n}{LifoQueue}\PYG{p}{(}\PYG{p}{)}
\PYG{n}{q}\PYG{o}{.}\PYG{n}{put}\PYG{p}{(}\PYG{l+m+mi}{3}\PYG{p}{)}
\PYG{n}{q}\PYG{o}{.}\PYG{n}{put}\PYG{p}{(}\PYG{l+m+mi}{2}\PYG{p}{)}
\PYG{n}{q}\PYG{o}{.}\PYG{n}{put}\PYG{p}{(}\PYG{l+m+mi}{1}\PYG{p}{)}

\PYG{n+nb}{print}\PYG{p}{(}\PYG{n}{q}\PYG{o}{.}\PYG{n}{get\PYGZus{}nowait}\PYG{p}{(}\PYG{p}{)}\PYG{p}{)}
\PYG{n+nb}{print}\PYG{p}{(}\PYG{n}{q}\PYG{o}{.}\PYG{n}{get\PYGZus{}nowait}\PYG{p}{(}\PYG{p}{)}\PYG{p}{)}
\PYG{n+nb}{print}\PYG{p}{(}\PYG{n}{q}\PYG{o}{.}\PYG{n}{get\PYGZus{}nowait}\PYG{p}{(}\PYG{p}{)}\PYG{p}{)}
\end{sphinxVerbatim}

\begin{sphinxVerbatim}[commandchars=\\\{\}]
1
2
3
\end{sphinxVerbatim}

\end{fulllineitems}



\subsubsection{Exceptions}
\label{\detokenize{queues:exceptions}}

\paragraph{QueueEmpty}
\label{\detokenize{queues:queueempty}}\index{QueueEmpty@\spxentry{QueueEmpty}}

\begin{fulllineitems}
\phantomsection\label{\detokenize{queues:tornado.queues.QueueEmpty}}\pysigline{\sphinxbfcode{\sphinxupquote{exception }}\sphinxcode{\sphinxupquote{tornado.queues.}}\sphinxbfcode{\sphinxupquote{QueueEmpty}}}
Raised by {\hyperref[\detokenize{queues:tornado.queues.Queue.get_nowait}]{\sphinxcrossref{\sphinxcode{\sphinxupquote{Queue.get\_nowait}}}}} when the queue has no items.

\end{fulllineitems}



\paragraph{QueueFull}
\label{\detokenize{queues:queuefull}}\index{QueueFull@\spxentry{QueueFull}}

\begin{fulllineitems}
\phantomsection\label{\detokenize{queues:tornado.queues.QueueFull}}\pysigline{\sphinxbfcode{\sphinxupquote{exception }}\sphinxcode{\sphinxupquote{tornado.queues.}}\sphinxbfcode{\sphinxupquote{QueueFull}}}
Raised by {\hyperref[\detokenize{queues:tornado.queues.Queue.put_nowait}]{\sphinxcrossref{\sphinxcode{\sphinxupquote{Queue.put\_nowait}}}}} when a queue is at its maximum size.

\end{fulllineitems}



\subsection{\sphinxstyleliteralintitle{\sphinxupquote{tornado.process}} — Utilities for multiple processes}
\label{\detokenize{process:module-tornado.process}}\label{\detokenize{process:tornado-process-utilities-for-multiple-processes}}\label{\detokenize{process::doc}}\index{tornado.process (module)@\spxentry{tornado.process}\spxextra{module}}
Utilities for working with multiple processes, including both forking
the server into multiple processes and managing subprocesses.
\index{CalledProcessError@\spxentry{CalledProcessError}}

\begin{fulllineitems}
\phantomsection\label{\detokenize{process:tornado.process.CalledProcessError}}\pysigline{\sphinxbfcode{\sphinxupquote{exception }}\sphinxcode{\sphinxupquote{tornado.process.}}\sphinxbfcode{\sphinxupquote{CalledProcessError}}}
An alias for \sphinxhref{https://docs.python.org/3.6/library/subprocess.html\#subprocess.CalledProcessError}{\sphinxcode{\sphinxupquote{subprocess.CalledProcessError}}}.

\end{fulllineitems}

\index{cpu\_count() (in module tornado.process)@\spxentry{cpu\_count()}\spxextra{in module tornado.process}}

\begin{fulllineitems}
\phantomsection\label{\detokenize{process:tornado.process.cpu_count}}\pysiglinewithargsret{\sphinxcode{\sphinxupquote{tornado.process.}}\sphinxbfcode{\sphinxupquote{cpu\_count}}}{}{{ $\rightarrow$ int}}
Returns the number of processors on this machine.

\end{fulllineitems}

\index{fork\_processes() (in module tornado.process)@\spxentry{fork\_processes()}\spxextra{in module tornado.process}}

\begin{fulllineitems}
\phantomsection\label{\detokenize{process:tornado.process.fork_processes}}\pysiglinewithargsret{\sphinxcode{\sphinxupquote{tornado.process.}}\sphinxbfcode{\sphinxupquote{fork\_processes}}}{\emph{num\_processes: Optional{[}int{]}, max\_restarts: int = None}}{{ $\rightarrow$ int}}
Starts multiple worker processes.

If \sphinxcode{\sphinxupquote{num\_processes}} is None or \textless{}= 0, we detect the number of cores
available on this machine and fork that number of child
processes. If \sphinxcode{\sphinxupquote{num\_processes}} is given and \textgreater{} 0, we fork that
specific number of sub-processes.

Since we use processes and not threads, there is no shared memory
between any server code.

Note that multiple processes are not compatible with the autoreload
module (or the \sphinxcode{\sphinxupquote{autoreload=True}} option to {\hyperref[\detokenize{web:tornado.web.Application}]{\sphinxcrossref{\sphinxcode{\sphinxupquote{tornado.web.Application}}}}}
which defaults to True when \sphinxcode{\sphinxupquote{debug=True}}).
When using multiple processes, no IOLoops can be created or
referenced until after the call to \sphinxcode{\sphinxupquote{fork\_processes}}.

In each child process, \sphinxcode{\sphinxupquote{fork\_processes}} returns its \sphinxstyleemphasis{task id}, a
number between 0 and \sphinxcode{\sphinxupquote{num\_processes}}.  Processes that exit
abnormally (due to a signal or non-zero exit status) are restarted
with the same id (up to \sphinxcode{\sphinxupquote{max\_restarts}} times).  In the parent
process, \sphinxcode{\sphinxupquote{fork\_processes}} returns None if all child processes
have exited normally, but will otherwise only exit by throwing an
exception.

max\_restarts defaults to 100.

\end{fulllineitems}

\index{task\_id() (in module tornado.process)@\spxentry{task\_id()}\spxextra{in module tornado.process}}

\begin{fulllineitems}
\phantomsection\label{\detokenize{process:tornado.process.task_id}}\pysiglinewithargsret{\sphinxcode{\sphinxupquote{tornado.process.}}\sphinxbfcode{\sphinxupquote{task\_id}}}{}{{ $\rightarrow$ Optional{[}int{]}}}
Returns the current task id, if any.

Returns None if this process was not created by {\hyperref[\detokenize{process:tornado.process.fork_processes}]{\sphinxcrossref{\sphinxcode{\sphinxupquote{fork\_processes}}}}}.

\end{fulllineitems}

\index{Subprocess (class in tornado.process)@\spxentry{Subprocess}\spxextra{class in tornado.process}}

\begin{fulllineitems}
\phantomsection\label{\detokenize{process:tornado.process.Subprocess}}\pysiglinewithargsret{\sphinxbfcode{\sphinxupquote{class }}\sphinxcode{\sphinxupquote{tornado.process.}}\sphinxbfcode{\sphinxupquote{Subprocess}}}{\emph{*args}, \emph{**kwargs}}{}
Wraps \sphinxcode{\sphinxupquote{subprocess.Popen}} with IOStream support.

The constructor is the same as \sphinxcode{\sphinxupquote{subprocess.Popen}} with the following
additions:
\begin{itemize}
\item {} 
\sphinxcode{\sphinxupquote{stdin}}, \sphinxcode{\sphinxupquote{stdout}}, and \sphinxcode{\sphinxupquote{stderr}} may have the value
\sphinxcode{\sphinxupquote{tornado.process.Subprocess.STREAM}}, which will make the corresponding
attribute of the resulting Subprocess a {\hyperref[\detokenize{iostream:tornado.iostream.PipeIOStream}]{\sphinxcrossref{\sphinxcode{\sphinxupquote{PipeIOStream}}}}}. If this option
is used, the caller is responsible for closing the streams when done
with them.

\end{itemize}

The \sphinxcode{\sphinxupquote{Subprocess.STREAM}} option and the \sphinxcode{\sphinxupquote{set\_exit\_callback}} and
\sphinxcode{\sphinxupquote{wait\_for\_exit}} methods do not work on Windows. There is
therefore no reason to use this class instead of
\sphinxcode{\sphinxupquote{subprocess.Popen}} on that platform.

\DUrole{versionmodified,changed}{Changed in version 5.0: }The \sphinxcode{\sphinxupquote{io\_loop}} argument (deprecated since version 4.1) has been removed.
\index{set\_exit\_callback() (tornado.process.Subprocess method)@\spxentry{set\_exit\_callback()}\spxextra{tornado.process.Subprocess method}}

\begin{fulllineitems}
\phantomsection\label{\detokenize{process:tornado.process.Subprocess.set_exit_callback}}\pysiglinewithargsret{\sphinxbfcode{\sphinxupquote{set\_exit\_callback}}}{\emph{callback: Callable{[}{[}int{]}, None{]}}}{{ $\rightarrow$ None}}
Runs \sphinxcode{\sphinxupquote{callback}} when this process exits.

The callback takes one argument, the return code of the process.

This method uses a \sphinxcode{\sphinxupquote{SIGCHLD}} handler, which is a global setting
and may conflict if you have other libraries trying to handle the
same signal.  If you are using more than one \sphinxcode{\sphinxupquote{IOLoop}} it may
be necessary to call {\hyperref[\detokenize{process:tornado.process.Subprocess.initialize}]{\sphinxcrossref{\sphinxcode{\sphinxupquote{Subprocess.initialize}}}}} first to designate
one \sphinxcode{\sphinxupquote{IOLoop}} to run the signal handlers.

In many cases a close callback on the stdout or stderr streams
can be used as an alternative to an exit callback if the
signal handler is causing a problem.

\end{fulllineitems}

\index{wait\_for\_exit() (tornado.process.Subprocess method)@\spxentry{wait\_for\_exit()}\spxextra{tornado.process.Subprocess method}}

\begin{fulllineitems}
\phantomsection\label{\detokenize{process:tornado.process.Subprocess.wait_for_exit}}\pysiglinewithargsret{\sphinxbfcode{\sphinxupquote{wait\_for\_exit}}}{\emph{raise\_error: bool = True}}{{ $\rightarrow$ Future{[}int{]}}}
Returns a {\hyperref[\detokenize{concurrent:tornado.concurrent.Future}]{\sphinxcrossref{\sphinxcode{\sphinxupquote{Future}}}}} which resolves when the process exits.

Usage:

\begin{sphinxVerbatim}[commandchars=\\\{\}]
\PYG{n}{ret} \PYG{o}{=} \PYG{k}{yield} \PYG{n}{proc}\PYG{o}{.}\PYG{n}{wait\PYGZus{}for\PYGZus{}exit}\PYG{p}{(}\PYG{p}{)}
\end{sphinxVerbatim}

This is a coroutine-friendly alternative to {\hyperref[\detokenize{process:tornado.process.Subprocess.set_exit_callback}]{\sphinxcrossref{\sphinxcode{\sphinxupquote{set\_exit\_callback}}}}}
(and a replacement for the blocking \sphinxhref{https://docs.python.org/3.6/library/subprocess.html\#subprocess.Popen.wait}{\sphinxcode{\sphinxupquote{subprocess.Popen.wait}}}).

By default, raises \sphinxhref{https://docs.python.org/3.6/library/subprocess.html\#subprocess.CalledProcessError}{\sphinxcode{\sphinxupquote{subprocess.CalledProcessError}}} if the process
has a non-zero exit status. Use \sphinxcode{\sphinxupquote{wait\_for\_exit(raise\_error=False)}}
to suppress this behavior and return the exit status without raising.

\DUrole{versionmodified,added}{New in version 4.2.}

\end{fulllineitems}

\index{initialize() (tornado.process.Subprocess class method)@\spxentry{initialize()}\spxextra{tornado.process.Subprocess class method}}

\begin{fulllineitems}
\phantomsection\label{\detokenize{process:tornado.process.Subprocess.initialize}}\pysiglinewithargsret{\sphinxbfcode{\sphinxupquote{classmethod }}\sphinxbfcode{\sphinxupquote{initialize}}}{}{{ $\rightarrow$ None}}
Initializes the \sphinxcode{\sphinxupquote{SIGCHLD}} handler.

The signal handler is run on an {\hyperref[\detokenize{ioloop:tornado.ioloop.IOLoop}]{\sphinxcrossref{\sphinxcode{\sphinxupquote{IOLoop}}}}} to avoid locking issues.
Note that the {\hyperref[\detokenize{ioloop:tornado.ioloop.IOLoop}]{\sphinxcrossref{\sphinxcode{\sphinxupquote{IOLoop}}}}} used for signal handling need not be the
same one used by individual Subprocess objects (as long as the
\sphinxcode{\sphinxupquote{IOLoops}} are each running in separate threads).

\DUrole{versionmodified,changed}{Changed in version 5.0: }The \sphinxcode{\sphinxupquote{io\_loop}} argument (deprecated since version 4.1) has been
removed.

\end{fulllineitems}

\index{uninitialize() (tornado.process.Subprocess class method)@\spxentry{uninitialize()}\spxextra{tornado.process.Subprocess class method}}

\begin{fulllineitems}
\phantomsection\label{\detokenize{process:tornado.process.Subprocess.uninitialize}}\pysiglinewithargsret{\sphinxbfcode{\sphinxupquote{classmethod }}\sphinxbfcode{\sphinxupquote{uninitialize}}}{}{{ $\rightarrow$ None}}
Removes the \sphinxcode{\sphinxupquote{SIGCHLD}} handler.

\end{fulllineitems}


\end{fulllineitems}



\section{Integration with other services}
\label{\detokenize{integration:integration-with-other-services}}\label{\detokenize{integration::doc}}

\subsection{\sphinxstyleliteralintitle{\sphinxupquote{tornado.auth}} — Third-party login with OpenID and OAuth}
\label{\detokenize{auth:tornado-auth-third-party-login-with-openid-and-oauth}}\label{\detokenize{auth::doc}}\phantomsection\label{\detokenize{auth:module-tornado.auth}}\index{tornado.auth (module)@\spxentry{tornado.auth}\spxextra{module}}
This module contains implementations of various third-party
authentication schemes.

All the classes in this file are class mixins designed to be used with
the {\hyperref[\detokenize{web:tornado.web.RequestHandler}]{\sphinxcrossref{\sphinxcode{\sphinxupquote{tornado.web.RequestHandler}}}}} class.  They are used in two ways:
\begin{itemize}
\item {} 
On a login handler, use methods such as \sphinxcode{\sphinxupquote{authenticate\_redirect()}},
\sphinxcode{\sphinxupquote{authorize\_redirect()}}, and \sphinxcode{\sphinxupquote{get\_authenticated\_user()}} to
establish the user’s identity and store authentication tokens to your
database and/or cookies.

\item {} 
In non-login handlers, use methods such as \sphinxcode{\sphinxupquote{facebook\_request()}}
or \sphinxcode{\sphinxupquote{twitter\_request()}} to use the authentication tokens to make
requests to the respective services.

\end{itemize}

They all take slightly different arguments due to the fact all these
services implement authentication and authorization slightly differently.
See the individual service classes below for complete documentation.

Example usage for Google OAuth:

\begin{sphinxVerbatim}[commandchars=\\\{\}]
\PYG{k}{class} \PYG{n+nc}{GoogleOAuth2LoginHandler}\PYG{p}{(}\PYG{n}{tornado}\PYG{o}{.}\PYG{n}{web}\PYG{o}{.}\PYG{n}{RequestHandler}\PYG{p}{,}
                               \PYG{n}{tornado}\PYG{o}{.}\PYG{n}{auth}\PYG{o}{.}\PYG{n}{GoogleOAuth2Mixin}\PYG{p}{)}\PYG{p}{:}
    \PYG{k}{async} \PYG{k}{def} \PYG{n+nf}{get}\PYG{p}{(}\PYG{n+nb+bp}{self}\PYG{p}{)}\PYG{p}{:}
        \PYG{k}{if} \PYG{n+nb+bp}{self}\PYG{o}{.}\PYG{n}{get\PYGZus{}argument}\PYG{p}{(}\PYG{l+s+s1}{\PYGZsq{}}\PYG{l+s+s1}{code}\PYG{l+s+s1}{\PYGZsq{}}\PYG{p}{,} \PYG{k+kc}{False}\PYG{p}{)}\PYG{p}{:}
            \PYG{n}{user} \PYG{o}{=} \PYG{k}{await} \PYG{n+nb+bp}{self}\PYG{o}{.}\PYG{n}{get\PYGZus{}authenticated\PYGZus{}user}\PYG{p}{(}
                \PYG{n}{redirect\PYGZus{}uri}\PYG{o}{=}\PYG{l+s+s1}{\PYGZsq{}}\PYG{l+s+s1}{http://your.site.com/auth/google}\PYG{l+s+s1}{\PYGZsq{}}\PYG{p}{,}
                \PYG{n}{code}\PYG{o}{=}\PYG{n+nb+bp}{self}\PYG{o}{.}\PYG{n}{get\PYGZus{}argument}\PYG{p}{(}\PYG{l+s+s1}{\PYGZsq{}}\PYG{l+s+s1}{code}\PYG{l+s+s1}{\PYGZsq{}}\PYG{p}{)}\PYG{p}{)}
            \PYG{c+c1}{\PYGZsh{} Save the user with e.g. set\PYGZus{}secure\PYGZus{}cookie}
        \PYG{k}{else}\PYG{p}{:}
            \PYG{k}{await} \PYG{n+nb+bp}{self}\PYG{o}{.}\PYG{n}{authorize\PYGZus{}redirect}\PYG{p}{(}
                \PYG{n}{redirect\PYGZus{}uri}\PYG{o}{=}\PYG{l+s+s1}{\PYGZsq{}}\PYG{l+s+s1}{http://your.site.com/auth/google}\PYG{l+s+s1}{\PYGZsq{}}\PYG{p}{,}
                \PYG{n}{client\PYGZus{}id}\PYG{o}{=}\PYG{n+nb+bp}{self}\PYG{o}{.}\PYG{n}{settings}\PYG{p}{[}\PYG{l+s+s1}{\PYGZsq{}}\PYG{l+s+s1}{google\PYGZus{}oauth}\PYG{l+s+s1}{\PYGZsq{}}\PYG{p}{]}\PYG{p}{[}\PYG{l+s+s1}{\PYGZsq{}}\PYG{l+s+s1}{key}\PYG{l+s+s1}{\PYGZsq{}}\PYG{p}{]}\PYG{p}{,}
                \PYG{n}{scope}\PYG{o}{=}\PYG{p}{[}\PYG{l+s+s1}{\PYGZsq{}}\PYG{l+s+s1}{profile}\PYG{l+s+s1}{\PYGZsq{}}\PYG{p}{,} \PYG{l+s+s1}{\PYGZsq{}}\PYG{l+s+s1}{email}\PYG{l+s+s1}{\PYGZsq{}}\PYG{p}{]}\PYG{p}{,}
                \PYG{n}{response\PYGZus{}type}\PYG{o}{=}\PYG{l+s+s1}{\PYGZsq{}}\PYG{l+s+s1}{code}\PYG{l+s+s1}{\PYGZsq{}}\PYG{p}{,}
                \PYG{n}{extra\PYGZus{}params}\PYG{o}{=}\PYG{p}{\PYGZob{}}\PYG{l+s+s1}{\PYGZsq{}}\PYG{l+s+s1}{approval\PYGZus{}prompt}\PYG{l+s+s1}{\PYGZsq{}}\PYG{p}{:} \PYG{l+s+s1}{\PYGZsq{}}\PYG{l+s+s1}{auto}\PYG{l+s+s1}{\PYGZsq{}}\PYG{p}{\PYGZcb{}}\PYG{p}{)}
\end{sphinxVerbatim}


\subsubsection{Common protocols}
\label{\detokenize{auth:common-protocols}}
These classes implement the OpenID and OAuth standards.  They will
generally need to be subclassed to use them with any particular site.
The degree of customization required will vary, but in most cases
overriding the class attributes (which are named beginning with
underscores for historical reasons) should be sufficient.
\index{OpenIdMixin (class in tornado.auth)@\spxentry{OpenIdMixin}\spxextra{class in tornado.auth}}

\begin{fulllineitems}
\phantomsection\label{\detokenize{auth:tornado.auth.OpenIdMixin}}\pysigline{\sphinxbfcode{\sphinxupquote{class }}\sphinxcode{\sphinxupquote{tornado.auth.}}\sphinxbfcode{\sphinxupquote{OpenIdMixin}}}
Abstract implementation of OpenID and Attribute Exchange.

Class attributes:
\begin{itemize}
\item {} 
\sphinxcode{\sphinxupquote{\_OPENID\_ENDPOINT}}: the identity provider’s URI.

\end{itemize}
\index{authenticate\_redirect() (tornado.auth.OpenIdMixin method)@\spxentry{authenticate\_redirect()}\spxextra{tornado.auth.OpenIdMixin method}}

\begin{fulllineitems}
\phantomsection\label{\detokenize{auth:tornado.auth.OpenIdMixin.authenticate_redirect}}\pysiglinewithargsret{\sphinxbfcode{\sphinxupquote{authenticate\_redirect}}}{\emph{callback\_uri: str = None, ax\_attrs: List{[}str{]} = {[}'name', 'email', 'language', 'username'{]}}}{{ $\rightarrow$ None}}
Redirects to the authentication URL for this service.

After authentication, the service will redirect back to the given
callback URI with additional parameters including \sphinxcode{\sphinxupquote{openid.mode}}.

We request the given attributes for the authenticated user by
default (name, email, language, and username). If you don’t need
all those attributes for your app, you can request fewer with
the ax\_attrs keyword argument.

\DUrole{versionmodified,changed}{Changed in version 6.0: }The \sphinxcode{\sphinxupquote{callback}} argument was removed and this method no
longer returns an awaitable object. It is now an ordinary
synchronous function.

\end{fulllineitems}

\index{get\_auth\_http\_client() (tornado.auth.OpenIdMixin method)@\spxentry{get\_auth\_http\_client()}\spxextra{tornado.auth.OpenIdMixin method}}

\begin{fulllineitems}
\phantomsection\label{\detokenize{auth:tornado.auth.OpenIdMixin.get_auth_http_client}}\pysiglinewithargsret{\sphinxbfcode{\sphinxupquote{get\_auth\_http\_client}}}{}{{ $\rightarrow$ tornado.httpclient.AsyncHTTPClient}}
Returns the {\hyperref[\detokenize{httpclient:tornado.httpclient.AsyncHTTPClient}]{\sphinxcrossref{\sphinxcode{\sphinxupquote{AsyncHTTPClient}}}}} instance to be used for auth requests.

May be overridden by subclasses to use an HTTP client other than
the default.

\end{fulllineitems}

\index{get\_authenticated\_user() (tornado.auth.OpenIdMixin method)@\spxentry{get\_authenticated\_user()}\spxextra{tornado.auth.OpenIdMixin method}}

\begin{fulllineitems}
\phantomsection\label{\detokenize{auth:tornado.auth.OpenIdMixin.get_authenticated_user}}\pysiglinewithargsret{\sphinxbfcode{\sphinxupquote{coroutine }}\sphinxbfcode{\sphinxupquote{get\_authenticated\_user}}}{\emph{http\_client: tornado.httpclient.AsyncHTTPClient = None}}{{ $\rightarrow$ Dict{[}str, Any{]}}}
Fetches the authenticated user data upon redirect.

This method should be called by the handler that receives the
redirect from the {\hyperref[\detokenize{auth:tornado.auth.OpenIdMixin.authenticate_redirect}]{\sphinxcrossref{\sphinxcode{\sphinxupquote{authenticate\_redirect()}}}}} method (which is
often the same as the one that calls it; in that case you would
call {\hyperref[\detokenize{auth:tornado.auth.OpenIdMixin.get_authenticated_user}]{\sphinxcrossref{\sphinxcode{\sphinxupquote{get\_authenticated\_user}}}}} if the \sphinxcode{\sphinxupquote{openid.mode}} parameter
is present and {\hyperref[\detokenize{auth:tornado.auth.OpenIdMixin.authenticate_redirect}]{\sphinxcrossref{\sphinxcode{\sphinxupquote{authenticate\_redirect}}}}} if it is not).

The result of this method will generally be used to set a cookie.

\DUrole{versionmodified,changed}{Changed in version 6.0: }The \sphinxcode{\sphinxupquote{callback}} argument was removed. Use the returned
awaitable object instead.

\end{fulllineitems}


\end{fulllineitems}

\index{OAuthMixin (class in tornado.auth)@\spxentry{OAuthMixin}\spxextra{class in tornado.auth}}

\begin{fulllineitems}
\phantomsection\label{\detokenize{auth:tornado.auth.OAuthMixin}}\pysigline{\sphinxbfcode{\sphinxupquote{class }}\sphinxcode{\sphinxupquote{tornado.auth.}}\sphinxbfcode{\sphinxupquote{OAuthMixin}}}
Abstract implementation of OAuth 1.0 and 1.0a.

See {\hyperref[\detokenize{auth:tornado.auth.TwitterMixin}]{\sphinxcrossref{\sphinxcode{\sphinxupquote{TwitterMixin}}}}} below for an example implementation.

Class attributes:
\begin{itemize}
\item {} 
\sphinxcode{\sphinxupquote{\_OAUTH\_AUTHORIZE\_URL}}: The service’s OAuth authorization url.

\item {} 
\sphinxcode{\sphinxupquote{\_OAUTH\_ACCESS\_TOKEN\_URL}}: The service’s OAuth access token url.

\item {} 
\sphinxcode{\sphinxupquote{\_OAUTH\_VERSION}}: May be either “1.0” or “1.0a”.

\item {} 
\sphinxcode{\sphinxupquote{\_OAUTH\_NO\_CALLBACKS}}: Set this to True if the service requires
advance registration of callbacks.

\end{itemize}

Subclasses must also override the {\hyperref[\detokenize{auth:tornado.auth.OAuthMixin._oauth_get_user_future}]{\sphinxcrossref{\sphinxcode{\sphinxupquote{\_oauth\_get\_user\_future}}}}} and
{\hyperref[\detokenize{auth:tornado.auth.OAuthMixin._oauth_consumer_token}]{\sphinxcrossref{\sphinxcode{\sphinxupquote{\_oauth\_consumer\_token}}}}} methods.
\index{authorize\_redirect() (tornado.auth.OAuthMixin method)@\spxentry{authorize\_redirect()}\spxextra{tornado.auth.OAuthMixin method}}

\begin{fulllineitems}
\phantomsection\label{\detokenize{auth:tornado.auth.OAuthMixin.authorize_redirect}}\pysiglinewithargsret{\sphinxbfcode{\sphinxupquote{authorize\_redirect}}}{\emph{callback\_uri: str = None}, \emph{extra\_params: Dict{[}str}, \emph{Any{]} = None}, \emph{http\_client: tornado.httpclient.AsyncHTTPClient = None}}{{ $\rightarrow$ None}}
Redirects the user to obtain OAuth authorization for this service.

The \sphinxcode{\sphinxupquote{callback\_uri}} may be omitted if you have previously
registered a callback URI with the third-party service. For
some services, you must use a previously-registered callback
URI and cannot specify a callback via this method.

This method sets a cookie called \sphinxcode{\sphinxupquote{\_oauth\_request\_token}} which is
subsequently used (and cleared) in {\hyperref[\detokenize{auth:tornado.auth.OAuthMixin.get_authenticated_user}]{\sphinxcrossref{\sphinxcode{\sphinxupquote{get\_authenticated\_user}}}}} for
security purposes.

This method is asynchronous and must be called with \sphinxcode{\sphinxupquote{await}}
or \sphinxcode{\sphinxupquote{yield}} (This is different from other \sphinxcode{\sphinxupquote{auth*\_redirect}}
methods defined in this module). It calls
{\hyperref[\detokenize{web:tornado.web.RequestHandler.finish}]{\sphinxcrossref{\sphinxcode{\sphinxupquote{RequestHandler.finish}}}}} for you so you should not write any
other response after it returns.

\DUrole{versionmodified,changed}{Changed in version 3.1: }Now returns a {\hyperref[\detokenize{concurrent:tornado.concurrent.Future}]{\sphinxcrossref{\sphinxcode{\sphinxupquote{Future}}}}} and takes an optional callback, for
compatibility with {\hyperref[\detokenize{gen:tornado.gen.coroutine}]{\sphinxcrossref{\sphinxcode{\sphinxupquote{gen.coroutine}}}}}.

\DUrole{versionmodified,changed}{Changed in version 6.0: }The \sphinxcode{\sphinxupquote{callback}} argument was removed. Use the returned
awaitable object instead.

\end{fulllineitems}

\index{get\_authenticated\_user() (tornado.auth.OAuthMixin method)@\spxentry{get\_authenticated\_user()}\spxextra{tornado.auth.OAuthMixin method}}

\begin{fulllineitems}
\phantomsection\label{\detokenize{auth:tornado.auth.OAuthMixin.get_authenticated_user}}\pysiglinewithargsret{\sphinxbfcode{\sphinxupquote{get\_authenticated\_user}}}{\emph{http\_client: tornado.httpclient.AsyncHTTPClient = None}}{{ $\rightarrow$ Dict{[}str, Any{]}}}
Gets the OAuth authorized user and access token.

This method should be called from the handler for your
OAuth callback URL to complete the registration process. We run the
callback with the authenticated user dictionary.  This dictionary
will contain an \sphinxcode{\sphinxupquote{access\_key}} which can be used to make authorized
requests to this service on behalf of the user.  The dictionary will
also contain other fields such as \sphinxcode{\sphinxupquote{name}}, depending on the service
used.

\DUrole{versionmodified,changed}{Changed in version 6.0: }The \sphinxcode{\sphinxupquote{callback}} argument was removed. Use the returned
awaitable object instead.

\end{fulllineitems}

\index{\_oauth\_consumer\_token() (tornado.auth.OAuthMixin method)@\spxentry{\_oauth\_consumer\_token()}\spxextra{tornado.auth.OAuthMixin method}}

\begin{fulllineitems}
\phantomsection\label{\detokenize{auth:tornado.auth.OAuthMixin._oauth_consumer_token}}\pysiglinewithargsret{\sphinxbfcode{\sphinxupquote{\_oauth\_consumer\_token}}}{}{{ $\rightarrow$ Dict{[}str, Any{]}}}
Subclasses must override this to return their OAuth consumer keys.

The return value should be a \sphinxhref{https://docs.python.org/3.6/library/stdtypes.html\#dict}{\sphinxcode{\sphinxupquote{dict}}} with keys \sphinxcode{\sphinxupquote{key}} and \sphinxcode{\sphinxupquote{secret}}.

\end{fulllineitems}

\index{\_oauth\_get\_user\_future() (tornado.auth.OAuthMixin method)@\spxentry{\_oauth\_get\_user\_future()}\spxextra{tornado.auth.OAuthMixin method}}

\begin{fulllineitems}
\phantomsection\label{\detokenize{auth:tornado.auth.OAuthMixin._oauth_get_user_future}}\pysiglinewithargsret{\sphinxbfcode{\sphinxupquote{\_oauth\_get\_user\_future}}}{\emph{access\_token: Dict{[}str, Any{]}}}{{ $\rightarrow$ Dict{[}str, Any{]}}}
Subclasses must override this to get basic information about the
user.

Should be a coroutine whose result is a dictionary
containing information about the user, which may have been
retrieved by using \sphinxcode{\sphinxupquote{access\_token}} to make a request to the
service.

The access token will be added to the returned dictionary to make
the result of {\hyperref[\detokenize{auth:tornado.auth.OAuthMixin.get_authenticated_user}]{\sphinxcrossref{\sphinxcode{\sphinxupquote{get\_authenticated\_user}}}}}.

\DUrole{versionmodified,changed}{Changed in version 5.1: }Subclasses may also define this method with \sphinxcode{\sphinxupquote{async def}}.

\DUrole{versionmodified,changed}{Changed in version 6.0: }A synchronous fallback to \sphinxcode{\sphinxupquote{\_oauth\_get\_user}} was removed.

\end{fulllineitems}

\index{get\_auth\_http\_client() (tornado.auth.OAuthMixin method)@\spxentry{get\_auth\_http\_client()}\spxextra{tornado.auth.OAuthMixin method}}

\begin{fulllineitems}
\phantomsection\label{\detokenize{auth:tornado.auth.OAuthMixin.get_auth_http_client}}\pysiglinewithargsret{\sphinxbfcode{\sphinxupquote{get\_auth\_http\_client}}}{}{{ $\rightarrow$ tornado.httpclient.AsyncHTTPClient}}
Returns the {\hyperref[\detokenize{httpclient:tornado.httpclient.AsyncHTTPClient}]{\sphinxcrossref{\sphinxcode{\sphinxupquote{AsyncHTTPClient}}}}} instance to be used for auth requests.

May be overridden by subclasses to use an HTTP client other than
the default.

\end{fulllineitems}


\end{fulllineitems}

\index{OAuth2Mixin (class in tornado.auth)@\spxentry{OAuth2Mixin}\spxextra{class in tornado.auth}}

\begin{fulllineitems}
\phantomsection\label{\detokenize{auth:tornado.auth.OAuth2Mixin}}\pysigline{\sphinxbfcode{\sphinxupquote{class }}\sphinxcode{\sphinxupquote{tornado.auth.}}\sphinxbfcode{\sphinxupquote{OAuth2Mixin}}}
Abstract implementation of OAuth 2.0.

See {\hyperref[\detokenize{auth:tornado.auth.FacebookGraphMixin}]{\sphinxcrossref{\sphinxcode{\sphinxupquote{FacebookGraphMixin}}}}} or {\hyperref[\detokenize{auth:tornado.auth.GoogleOAuth2Mixin}]{\sphinxcrossref{\sphinxcode{\sphinxupquote{GoogleOAuth2Mixin}}}}} below for example
implementations.

Class attributes:
\begin{itemize}
\item {} 
\sphinxcode{\sphinxupquote{\_OAUTH\_AUTHORIZE\_URL}}: The service’s authorization url.

\item {} 
\sphinxcode{\sphinxupquote{\_OAUTH\_ACCESS\_TOKEN\_URL}}:  The service’s access token url.

\end{itemize}
\index{authorize\_redirect() (tornado.auth.OAuth2Mixin method)@\spxentry{authorize\_redirect()}\spxextra{tornado.auth.OAuth2Mixin method}}

\begin{fulllineitems}
\phantomsection\label{\detokenize{auth:tornado.auth.OAuth2Mixin.authorize_redirect}}\pysiglinewithargsret{\sphinxbfcode{\sphinxupquote{authorize\_redirect}}}{\emph{redirect\_uri: str = None}, \emph{client\_id: str = None}, \emph{client\_secret: str = None}, \emph{extra\_params: Dict{[}str}, \emph{Any{]} = None}, \emph{scope: str = None}, \emph{response\_type: str = 'code'}}{{ $\rightarrow$ None}}
Redirects the user to obtain OAuth authorization for this service.

Some providers require that you register a redirect URL with
your application instead of passing one via this method. You
should call this method to log the user in, and then call
\sphinxcode{\sphinxupquote{get\_authenticated\_user}} in the handler for your
redirect URL to complete the authorization process.

\DUrole{versionmodified,changed}{Changed in version 6.0: }The \sphinxcode{\sphinxupquote{callback}} argument and returned awaitable were removed;
this is now an ordinary synchronous function.

\end{fulllineitems}

\index{get\_auth\_http\_client() (tornado.auth.OAuth2Mixin method)@\spxentry{get\_auth\_http\_client()}\spxextra{tornado.auth.OAuth2Mixin method}}

\begin{fulllineitems}
\phantomsection\label{\detokenize{auth:tornado.auth.OAuth2Mixin.get_auth_http_client}}\pysiglinewithargsret{\sphinxbfcode{\sphinxupquote{get\_auth\_http\_client}}}{}{{ $\rightarrow$ tornado.httpclient.AsyncHTTPClient}}
Returns the {\hyperref[\detokenize{httpclient:tornado.httpclient.AsyncHTTPClient}]{\sphinxcrossref{\sphinxcode{\sphinxupquote{AsyncHTTPClient}}}}} instance to be used for auth requests.

May be overridden by subclasses to use an HTTP client other than
the default.

\DUrole{versionmodified,added}{New in version 4.3.}

\end{fulllineitems}

\index{oauth2\_request() (tornado.auth.OAuth2Mixin method)@\spxentry{oauth2\_request()}\spxextra{tornado.auth.OAuth2Mixin method}}

\begin{fulllineitems}
\phantomsection\label{\detokenize{auth:tornado.auth.OAuth2Mixin.oauth2_request}}\pysiglinewithargsret{\sphinxbfcode{\sphinxupquote{coroutine }}\sphinxbfcode{\sphinxupquote{oauth2\_request}}}{\emph{url: str}, \emph{access\_token: str = None}, \emph{post\_args: Dict{[}str}, \emph{Any{]} = None}, \emph{**args}}{{ $\rightarrow$ Any}}
Fetches the given URL auth an OAuth2 access token.

If the request is a POST, \sphinxcode{\sphinxupquote{post\_args}} should be provided. Query
string arguments should be given as keyword arguments.

Example usage:

..testcode:

\begin{sphinxVerbatim}[commandchars=\\\{\}]
\PYG{k}{class} \PYG{n+nc}{MainHandler}\PYG{p}{(}\PYG{n}{tornado}\PYG{o}{.}\PYG{n}{web}\PYG{o}{.}\PYG{n}{RequestHandler}\PYG{p}{,}
                  \PYG{n}{tornado}\PYG{o}{.}\PYG{n}{auth}\PYG{o}{.}\PYG{n}{FacebookGraphMixin}\PYG{p}{)}\PYG{p}{:}
    \PYG{n+nd}{@tornado}\PYG{o}{.}\PYG{n}{web}\PYG{o}{.}\PYG{n}{authenticated}
    \PYG{k}{async} \PYG{k}{def} \PYG{n+nf}{get}\PYG{p}{(}\PYG{n+nb+bp}{self}\PYG{p}{)}\PYG{p}{:}
        \PYG{n}{new\PYGZus{}entry} \PYG{o}{=} \PYG{k}{await} \PYG{n+nb+bp}{self}\PYG{o}{.}\PYG{n}{oauth2\PYGZus{}request}\PYG{p}{(}
            \PYG{l+s+s2}{\PYGZdq{}}\PYG{l+s+s2}{https://graph.facebook.com/me/feed}\PYG{l+s+s2}{\PYGZdq{}}\PYG{p}{,}
            \PYG{n}{post\PYGZus{}args}\PYG{o}{=}\PYG{p}{\PYGZob{}}\PYG{l+s+s2}{\PYGZdq{}}\PYG{l+s+s2}{message}\PYG{l+s+s2}{\PYGZdq{}}\PYG{p}{:} \PYG{l+s+s2}{\PYGZdq{}}\PYG{l+s+s2}{I am posting from my Tornado application!}\PYG{l+s+s2}{\PYGZdq{}}\PYG{p}{\PYGZcb{}}\PYG{p}{,}
            \PYG{n}{access\PYGZus{}token}\PYG{o}{=}\PYG{n+nb+bp}{self}\PYG{o}{.}\PYG{n}{current\PYGZus{}user}\PYG{p}{[}\PYG{l+s+s2}{\PYGZdq{}}\PYG{l+s+s2}{access\PYGZus{}token}\PYG{l+s+s2}{\PYGZdq{}}\PYG{p}{]}\PYG{p}{)}

        \PYG{k}{if} \PYG{o+ow}{not} \PYG{n}{new\PYGZus{}entry}\PYG{p}{:}
            \PYG{c+c1}{\PYGZsh{} Call failed; perhaps missing permission?}
            \PYG{k}{await} \PYG{n+nb+bp}{self}\PYG{o}{.}\PYG{n}{authorize\PYGZus{}redirect}\PYG{p}{(}\PYG{p}{)}
            \PYG{k}{return}
        \PYG{n+nb+bp}{self}\PYG{o}{.}\PYG{n}{finish}\PYG{p}{(}\PYG{l+s+s2}{\PYGZdq{}}\PYG{l+s+s2}{Posted a message!}\PYG{l+s+s2}{\PYGZdq{}}\PYG{p}{)}
\end{sphinxVerbatim}

\DUrole{versionmodified,added}{New in version 4.3.}

\end{fulllineitems}


\end{fulllineitems}



\subsubsection{Google}
\label{\detokenize{auth:google}}\index{GoogleOAuth2Mixin (class in tornado.auth)@\spxentry{GoogleOAuth2Mixin}\spxextra{class in tornado.auth}}

\begin{fulllineitems}
\phantomsection\label{\detokenize{auth:tornado.auth.GoogleOAuth2Mixin}}\pysigline{\sphinxbfcode{\sphinxupquote{class }}\sphinxcode{\sphinxupquote{tornado.auth.}}\sphinxbfcode{\sphinxupquote{GoogleOAuth2Mixin}}}
Google authentication using OAuth2.

In order to use, register your application with Google and copy the
relevant parameters to your application settings.
\begin{itemize}
\item {} 
Go to the Google Dev Console at \sphinxurl{http://console.developers.google.com}

\item {} 
Select a project, or create a new one.

\item {} 
In the sidebar on the left, select APIs \& Auth.

\item {} 
In the list of APIs, find the Google+ API service and set it to ON.

\item {} 
In the sidebar on the left, select Credentials.

\item {} 
In the OAuth section of the page, select Create New Client ID.

\item {} 
Set the Redirect URI to point to your auth handler

\item {} 
Copy the “Client secret” and “Client ID” to the application settings as
\sphinxcode{\sphinxupquote{\{"google\_oauth": \{"key": CLIENT\_ID, "secret": CLIENT\_SECRET\}\}}}

\end{itemize}

\DUrole{versionmodified,added}{New in version 3.2.}
\index{get\_authenticated\_user() (tornado.auth.GoogleOAuth2Mixin method)@\spxentry{get\_authenticated\_user()}\spxextra{tornado.auth.GoogleOAuth2Mixin method}}

\begin{fulllineitems}
\phantomsection\label{\detokenize{auth:tornado.auth.GoogleOAuth2Mixin.get_authenticated_user}}\pysiglinewithargsret{\sphinxbfcode{\sphinxupquote{coroutine }}\sphinxbfcode{\sphinxupquote{get\_authenticated\_user}}}{\emph{redirect\_uri: str}, \emph{code: str}}{{ $\rightarrow$ Dict{[}str, Any{]}}}
Handles the login for the Google user, returning an access token.

The result is a dictionary containing an \sphinxcode{\sphinxupquote{access\_token}} field
({[}among others{]}(\sphinxurl{https://developers.google.com/identity/protocols/OAuth2WebServer\#handlingtheresponse})).
Unlike other \sphinxcode{\sphinxupquote{get\_authenticated\_user}} methods in this package,
this method does not return any additional information about the user.
The returned access token can be used with {\hyperref[\detokenize{auth:tornado.auth.OAuth2Mixin.oauth2_request}]{\sphinxcrossref{\sphinxcode{\sphinxupquote{OAuth2Mixin.oauth2\_request}}}}}
to request additional information (perhaps from
\sphinxcode{\sphinxupquote{https://www.googleapis.com/oauth2/v2/userinfo}})

Example usage:

\begin{sphinxVerbatim}[commandchars=\\\{\}]
\PYG{k}{class} \PYG{n+nc}{GoogleOAuth2LoginHandler}\PYG{p}{(}\PYG{n}{tornado}\PYG{o}{.}\PYG{n}{web}\PYG{o}{.}\PYG{n}{RequestHandler}\PYG{p}{,}
                               \PYG{n}{tornado}\PYG{o}{.}\PYG{n}{auth}\PYG{o}{.}\PYG{n}{GoogleOAuth2Mixin}\PYG{p}{)}\PYG{p}{:}
    \PYG{k}{async} \PYG{k}{def} \PYG{n+nf}{get}\PYG{p}{(}\PYG{n+nb+bp}{self}\PYG{p}{)}\PYG{p}{:}
        \PYG{k}{if} \PYG{n+nb+bp}{self}\PYG{o}{.}\PYG{n}{get\PYGZus{}argument}\PYG{p}{(}\PYG{l+s+s1}{\PYGZsq{}}\PYG{l+s+s1}{code}\PYG{l+s+s1}{\PYGZsq{}}\PYG{p}{,} \PYG{k+kc}{False}\PYG{p}{)}\PYG{p}{:}
            \PYG{n}{access} \PYG{o}{=} \PYG{k}{await} \PYG{n+nb+bp}{self}\PYG{o}{.}\PYG{n}{get\PYGZus{}authenticated\PYGZus{}user}\PYG{p}{(}
                \PYG{n}{redirect\PYGZus{}uri}\PYG{o}{=}\PYG{l+s+s1}{\PYGZsq{}}\PYG{l+s+s1}{http://your.site.com/auth/google}\PYG{l+s+s1}{\PYGZsq{}}\PYG{p}{,}
                \PYG{n}{code}\PYG{o}{=}\PYG{n+nb+bp}{self}\PYG{o}{.}\PYG{n}{get\PYGZus{}argument}\PYG{p}{(}\PYG{l+s+s1}{\PYGZsq{}}\PYG{l+s+s1}{code}\PYG{l+s+s1}{\PYGZsq{}}\PYG{p}{)}\PYG{p}{)}
            \PYG{n}{user} \PYG{o}{=} \PYG{k}{await} \PYG{n+nb+bp}{self}\PYG{o}{.}\PYG{n}{oauth2\PYGZus{}request}\PYG{p}{(}
                \PYG{l+s+s2}{\PYGZdq{}}\PYG{l+s+s2}{https://www.googleapis.com/oauth2/v1/userinfo}\PYG{l+s+s2}{\PYGZdq{}}\PYG{p}{,}
                \PYG{n}{access\PYGZus{}token}\PYG{o}{=}\PYG{n}{access}\PYG{p}{[}\PYG{l+s+s2}{\PYGZdq{}}\PYG{l+s+s2}{access\PYGZus{}token}\PYG{l+s+s2}{\PYGZdq{}}\PYG{p}{]}\PYG{p}{)}
            \PYG{c+c1}{\PYGZsh{} Save the user and access token with}
            \PYG{c+c1}{\PYGZsh{} e.g. set\PYGZus{}secure\PYGZus{}cookie.}
        \PYG{k}{else}\PYG{p}{:}
            \PYG{k}{await} \PYG{n+nb+bp}{self}\PYG{o}{.}\PYG{n}{authorize\PYGZus{}redirect}\PYG{p}{(}
                \PYG{n}{redirect\PYGZus{}uri}\PYG{o}{=}\PYG{l+s+s1}{\PYGZsq{}}\PYG{l+s+s1}{http://your.site.com/auth/google}\PYG{l+s+s1}{\PYGZsq{}}\PYG{p}{,}
                \PYG{n}{client\PYGZus{}id}\PYG{o}{=}\PYG{n+nb+bp}{self}\PYG{o}{.}\PYG{n}{settings}\PYG{p}{[}\PYG{l+s+s1}{\PYGZsq{}}\PYG{l+s+s1}{google\PYGZus{}oauth}\PYG{l+s+s1}{\PYGZsq{}}\PYG{p}{]}\PYG{p}{[}\PYG{l+s+s1}{\PYGZsq{}}\PYG{l+s+s1}{key}\PYG{l+s+s1}{\PYGZsq{}}\PYG{p}{]}\PYG{p}{,}
                \PYG{n}{scope}\PYG{o}{=}\PYG{p}{[}\PYG{l+s+s1}{\PYGZsq{}}\PYG{l+s+s1}{profile}\PYG{l+s+s1}{\PYGZsq{}}\PYG{p}{,} \PYG{l+s+s1}{\PYGZsq{}}\PYG{l+s+s1}{email}\PYG{l+s+s1}{\PYGZsq{}}\PYG{p}{]}\PYG{p}{,}
                \PYG{n}{response\PYGZus{}type}\PYG{o}{=}\PYG{l+s+s1}{\PYGZsq{}}\PYG{l+s+s1}{code}\PYG{l+s+s1}{\PYGZsq{}}\PYG{p}{,}
                \PYG{n}{extra\PYGZus{}params}\PYG{o}{=}\PYG{p}{\PYGZob{}}\PYG{l+s+s1}{\PYGZsq{}}\PYG{l+s+s1}{approval\PYGZus{}prompt}\PYG{l+s+s1}{\PYGZsq{}}\PYG{p}{:} \PYG{l+s+s1}{\PYGZsq{}}\PYG{l+s+s1}{auto}\PYG{l+s+s1}{\PYGZsq{}}\PYG{p}{\PYGZcb{}}\PYG{p}{)}
\end{sphinxVerbatim}

\DUrole{versionmodified,changed}{Changed in version 6.0: }The \sphinxcode{\sphinxupquote{callback}} argument was removed. Use the returned awaitable object instead.

\end{fulllineitems}


\end{fulllineitems}



\subsubsection{Facebook}
\label{\detokenize{auth:facebook}}\index{FacebookGraphMixin (class in tornado.auth)@\spxentry{FacebookGraphMixin}\spxextra{class in tornado.auth}}

\begin{fulllineitems}
\phantomsection\label{\detokenize{auth:tornado.auth.FacebookGraphMixin}}\pysigline{\sphinxbfcode{\sphinxupquote{class }}\sphinxcode{\sphinxupquote{tornado.auth.}}\sphinxbfcode{\sphinxupquote{FacebookGraphMixin}}}
Facebook authentication using the new Graph API and OAuth2.
\index{facebook\_request() (tornado.auth.FacebookGraphMixin method)@\spxentry{facebook\_request()}\spxextra{tornado.auth.FacebookGraphMixin method}}

\begin{fulllineitems}
\phantomsection\label{\detokenize{auth:tornado.auth.FacebookGraphMixin.facebook_request}}\pysiglinewithargsret{\sphinxbfcode{\sphinxupquote{coroutine }}\sphinxbfcode{\sphinxupquote{facebook\_request}}}{\emph{path: str}, \emph{access\_token: str = None}, \emph{post\_args: Dict{[}str}, \emph{Any{]} = None}, \emph{**args}}{{ $\rightarrow$ Any}}
Fetches the given relative API path, e.g., “/btaylor/picture”

If the request is a POST, \sphinxcode{\sphinxupquote{post\_args}} should be provided. Query
string arguments should be given as keyword arguments.

An introduction to the Facebook Graph API can be found at
\sphinxurl{http://developers.facebook.com/docs/api}

Many methods require an OAuth access token which you can
obtain through {\hyperref[\detokenize{auth:tornado.auth.OAuth2Mixin.authorize_redirect}]{\sphinxcrossref{\sphinxcode{\sphinxupquote{authorize\_redirect}}}}} and
{\hyperref[\detokenize{auth:tornado.auth.FacebookGraphMixin.get_authenticated_user}]{\sphinxcrossref{\sphinxcode{\sphinxupquote{get\_authenticated\_user}}}}}. The user returned through that
process includes an \sphinxcode{\sphinxupquote{access\_token}} attribute that can be
used to make authenticated requests via this method.

Example usage:

\begin{sphinxVerbatim}[commandchars=\\\{\}]
\PYG{k}{class} \PYG{n+nc}{MainHandler}\PYG{p}{(}\PYG{n}{tornado}\PYG{o}{.}\PYG{n}{web}\PYG{o}{.}\PYG{n}{RequestHandler}\PYG{p}{,}
                  \PYG{n}{tornado}\PYG{o}{.}\PYG{n}{auth}\PYG{o}{.}\PYG{n}{FacebookGraphMixin}\PYG{p}{)}\PYG{p}{:}
    \PYG{n+nd}{@tornado}\PYG{o}{.}\PYG{n}{web}\PYG{o}{.}\PYG{n}{authenticated}
    \PYG{k}{async} \PYG{k}{def} \PYG{n+nf}{get}\PYG{p}{(}\PYG{n+nb+bp}{self}\PYG{p}{)}\PYG{p}{:}
        \PYG{n}{new\PYGZus{}entry} \PYG{o}{=} \PYG{k}{await} \PYG{n+nb+bp}{self}\PYG{o}{.}\PYG{n}{facebook\PYGZus{}request}\PYG{p}{(}
            \PYG{l+s+s2}{\PYGZdq{}}\PYG{l+s+s2}{/me/feed}\PYG{l+s+s2}{\PYGZdq{}}\PYG{p}{,}
            \PYG{n}{post\PYGZus{}args}\PYG{o}{=}\PYG{p}{\PYGZob{}}\PYG{l+s+s2}{\PYGZdq{}}\PYG{l+s+s2}{message}\PYG{l+s+s2}{\PYGZdq{}}\PYG{p}{:} \PYG{l+s+s2}{\PYGZdq{}}\PYG{l+s+s2}{I am posting from my Tornado application!}\PYG{l+s+s2}{\PYGZdq{}}\PYG{p}{\PYGZcb{}}\PYG{p}{,}
            \PYG{n}{access\PYGZus{}token}\PYG{o}{=}\PYG{n+nb+bp}{self}\PYG{o}{.}\PYG{n}{current\PYGZus{}user}\PYG{p}{[}\PYG{l+s+s2}{\PYGZdq{}}\PYG{l+s+s2}{access\PYGZus{}token}\PYG{l+s+s2}{\PYGZdq{}}\PYG{p}{]}\PYG{p}{)}

        \PYG{k}{if} \PYG{o+ow}{not} \PYG{n}{new\PYGZus{}entry}\PYG{p}{:}
            \PYG{c+c1}{\PYGZsh{} Call failed; perhaps missing permission?}
            \PYG{k}{yield} \PYG{n+nb+bp}{self}\PYG{o}{.}\PYG{n}{authorize\PYGZus{}redirect}\PYG{p}{(}\PYG{p}{)}
            \PYG{k}{return}
        \PYG{n+nb+bp}{self}\PYG{o}{.}\PYG{n}{finish}\PYG{p}{(}\PYG{l+s+s2}{\PYGZdq{}}\PYG{l+s+s2}{Posted a message!}\PYG{l+s+s2}{\PYGZdq{}}\PYG{p}{)}
\end{sphinxVerbatim}

The given path is relative to \sphinxcode{\sphinxupquote{self.\_FACEBOOK\_BASE\_URL}},
by default “\sphinxurl{https://graph.facebook.com}”.

This method is a wrapper around {\hyperref[\detokenize{auth:tornado.auth.OAuth2Mixin.oauth2_request}]{\sphinxcrossref{\sphinxcode{\sphinxupquote{OAuth2Mixin.oauth2\_request}}}}};
the only difference is that this method takes a relative path,
while \sphinxcode{\sphinxupquote{oauth2\_request}} takes a complete url.

\DUrole{versionmodified,changed}{Changed in version 3.1: }Added the ability to override \sphinxcode{\sphinxupquote{self.\_FACEBOOK\_BASE\_URL}}.

\DUrole{versionmodified,changed}{Changed in version 6.0: }The \sphinxcode{\sphinxupquote{callback}} argument was removed. Use the returned awaitable object instead.

\end{fulllineitems}

\index{get\_authenticated\_user() (tornado.auth.FacebookGraphMixin method)@\spxentry{get\_authenticated\_user()}\spxextra{tornado.auth.FacebookGraphMixin method}}

\begin{fulllineitems}
\phantomsection\label{\detokenize{auth:tornado.auth.FacebookGraphMixin.get_authenticated_user}}\pysiglinewithargsret{\sphinxbfcode{\sphinxupquote{coroutine }}\sphinxbfcode{\sphinxupquote{get\_authenticated\_user}}}{\emph{redirect\_uri: str}, \emph{client\_id: str}, \emph{client\_secret: str}, \emph{code: str}, \emph{extra\_fields: Dict{[}str}, \emph{Any{]} = None}}{{ $\rightarrow$ Optional{[}Dict{[}str, Any{]}{]}}}
Handles the login for the Facebook user, returning a user object.

Example usage:

\begin{sphinxVerbatim}[commandchars=\\\{\}]
\PYG{k}{class} \PYG{n+nc}{FacebookGraphLoginHandler}\PYG{p}{(}\PYG{n}{tornado}\PYG{o}{.}\PYG{n}{web}\PYG{o}{.}\PYG{n}{RequestHandler}\PYG{p}{,}
                                \PYG{n}{tornado}\PYG{o}{.}\PYG{n}{auth}\PYG{o}{.}\PYG{n}{FacebookGraphMixin}\PYG{p}{)}\PYG{p}{:}
  \PYG{k}{async} \PYG{k}{def} \PYG{n+nf}{get}\PYG{p}{(}\PYG{n+nb+bp}{self}\PYG{p}{)}\PYG{p}{:}
      \PYG{k}{if} \PYG{n+nb+bp}{self}\PYG{o}{.}\PYG{n}{get\PYGZus{}argument}\PYG{p}{(}\PYG{l+s+s2}{\PYGZdq{}}\PYG{l+s+s2}{code}\PYG{l+s+s2}{\PYGZdq{}}\PYG{p}{,} \PYG{k+kc}{False}\PYG{p}{)}\PYG{p}{:}
          \PYG{n}{user} \PYG{o}{=} \PYG{k}{await} \PYG{n+nb+bp}{self}\PYG{o}{.}\PYG{n}{get\PYGZus{}authenticated\PYGZus{}user}\PYG{p}{(}
              \PYG{n}{redirect\PYGZus{}uri}\PYG{o}{=}\PYG{l+s+s1}{\PYGZsq{}}\PYG{l+s+s1}{/auth/facebookgraph/}\PYG{l+s+s1}{\PYGZsq{}}\PYG{p}{,}
              \PYG{n}{client\PYGZus{}id}\PYG{o}{=}\PYG{n+nb+bp}{self}\PYG{o}{.}\PYG{n}{settings}\PYG{p}{[}\PYG{l+s+s2}{\PYGZdq{}}\PYG{l+s+s2}{facebook\PYGZus{}api\PYGZus{}key}\PYG{l+s+s2}{\PYGZdq{}}\PYG{p}{]}\PYG{p}{,}
              \PYG{n}{client\PYGZus{}secret}\PYG{o}{=}\PYG{n+nb+bp}{self}\PYG{o}{.}\PYG{n}{settings}\PYG{p}{[}\PYG{l+s+s2}{\PYGZdq{}}\PYG{l+s+s2}{facebook\PYGZus{}secret}\PYG{l+s+s2}{\PYGZdq{}}\PYG{p}{]}\PYG{p}{,}
              \PYG{n}{code}\PYG{o}{=}\PYG{n+nb+bp}{self}\PYG{o}{.}\PYG{n}{get\PYGZus{}argument}\PYG{p}{(}\PYG{l+s+s2}{\PYGZdq{}}\PYG{l+s+s2}{code}\PYG{l+s+s2}{\PYGZdq{}}\PYG{p}{)}\PYG{p}{)}
          \PYG{c+c1}{\PYGZsh{} Save the user with e.g. set\PYGZus{}secure\PYGZus{}cookie}
      \PYG{k}{else}\PYG{p}{:}
          \PYG{k}{await} \PYG{n+nb+bp}{self}\PYG{o}{.}\PYG{n}{authorize\PYGZus{}redirect}\PYG{p}{(}
              \PYG{n}{redirect\PYGZus{}uri}\PYG{o}{=}\PYG{l+s+s1}{\PYGZsq{}}\PYG{l+s+s1}{/auth/facebookgraph/}\PYG{l+s+s1}{\PYGZsq{}}\PYG{p}{,}
              \PYG{n}{client\PYGZus{}id}\PYG{o}{=}\PYG{n+nb+bp}{self}\PYG{o}{.}\PYG{n}{settings}\PYG{p}{[}\PYG{l+s+s2}{\PYGZdq{}}\PYG{l+s+s2}{facebook\PYGZus{}api\PYGZus{}key}\PYG{l+s+s2}{\PYGZdq{}}\PYG{p}{]}\PYG{p}{,}
              \PYG{n}{extra\PYGZus{}params}\PYG{o}{=}\PYG{p}{\PYGZob{}}\PYG{l+s+s2}{\PYGZdq{}}\PYG{l+s+s2}{scope}\PYG{l+s+s2}{\PYGZdq{}}\PYG{p}{:} \PYG{l+s+s2}{\PYGZdq{}}\PYG{l+s+s2}{read\PYGZus{}stream,offline\PYGZus{}access}\PYG{l+s+s2}{\PYGZdq{}}\PYG{p}{\PYGZcb{}}\PYG{p}{)}
\end{sphinxVerbatim}

This method returns a dictionary which may contain the following fields:
\begin{itemize}
\item {} 
\sphinxcode{\sphinxupquote{access\_token}}, a string which may be passed to {\hyperref[\detokenize{auth:tornado.auth.FacebookGraphMixin.facebook_request}]{\sphinxcrossref{\sphinxcode{\sphinxupquote{facebook\_request}}}}}

\item {} 
\sphinxcode{\sphinxupquote{session\_expires}}, an integer encoded as a string representing
the time until the access token expires in seconds. This field should
be used like \sphinxcode{\sphinxupquote{int(user{[}'session\_expires'{]})}}; in a future version of
Tornado it will change from a string to an integer.

\item {} 
\sphinxcode{\sphinxupquote{id}}, \sphinxcode{\sphinxupquote{name}}, \sphinxcode{\sphinxupquote{first\_name}}, \sphinxcode{\sphinxupquote{last\_name}}, \sphinxcode{\sphinxupquote{locale}}, \sphinxcode{\sphinxupquote{picture}},
\sphinxcode{\sphinxupquote{link}}, plus any fields named in the \sphinxcode{\sphinxupquote{extra\_fields}} argument. These
fields are copied from the Facebook graph API
\sphinxhref{https://developers.facebook.com/docs/graph-api/reference/user}{user object}

\end{itemize}

\DUrole{versionmodified,changed}{Changed in version 4.5: }The \sphinxcode{\sphinxupquote{session\_expires}} field was updated to support changes made to the
Facebook API in March 2017.

\DUrole{versionmodified,changed}{Changed in version 6.0: }The \sphinxcode{\sphinxupquote{callback}} argument was removed. Use the returned awaitable object instead.

\end{fulllineitems}


\end{fulllineitems}



\subsubsection{Twitter}
\label{\detokenize{auth:twitter}}\index{TwitterMixin (class in tornado.auth)@\spxentry{TwitterMixin}\spxextra{class in tornado.auth}}

\begin{fulllineitems}
\phantomsection\label{\detokenize{auth:tornado.auth.TwitterMixin}}\pysigline{\sphinxbfcode{\sphinxupquote{class }}\sphinxcode{\sphinxupquote{tornado.auth.}}\sphinxbfcode{\sphinxupquote{TwitterMixin}}}
Twitter OAuth authentication.

To authenticate with Twitter, register your application with
Twitter at \sphinxurl{http://twitter.com/apps}. Then copy your Consumer Key
and Consumer Secret to the application
{\hyperref[\detokenize{web:tornado.web.Application.settings}]{\sphinxcrossref{\sphinxcode{\sphinxupquote{settings}}}}} \sphinxcode{\sphinxupquote{twitter\_consumer\_key}} and
\sphinxcode{\sphinxupquote{twitter\_consumer\_secret}}. Use this mixin on the handler for the
URL you registered as your application’s callback URL.

When your application is set up, you can use this mixin like this
to authenticate the user with Twitter and get access to their stream:

\begin{sphinxVerbatim}[commandchars=\\\{\}]
\PYG{k}{class} \PYG{n+nc}{TwitterLoginHandler}\PYG{p}{(}\PYG{n}{tornado}\PYG{o}{.}\PYG{n}{web}\PYG{o}{.}\PYG{n}{RequestHandler}\PYG{p}{,}
                          \PYG{n}{tornado}\PYG{o}{.}\PYG{n}{auth}\PYG{o}{.}\PYG{n}{TwitterMixin}\PYG{p}{)}\PYG{p}{:}
    \PYG{k}{async} \PYG{k}{def} \PYG{n+nf}{get}\PYG{p}{(}\PYG{n+nb+bp}{self}\PYG{p}{)}\PYG{p}{:}
        \PYG{k}{if} \PYG{n+nb+bp}{self}\PYG{o}{.}\PYG{n}{get\PYGZus{}argument}\PYG{p}{(}\PYG{l+s+s2}{\PYGZdq{}}\PYG{l+s+s2}{oauth\PYGZus{}token}\PYG{l+s+s2}{\PYGZdq{}}\PYG{p}{,} \PYG{k+kc}{None}\PYG{p}{)}\PYG{p}{:}
            \PYG{n}{user} \PYG{o}{=} \PYG{k}{await} \PYG{n+nb+bp}{self}\PYG{o}{.}\PYG{n}{get\PYGZus{}authenticated\PYGZus{}user}\PYG{p}{(}\PYG{p}{)}
            \PYG{c+c1}{\PYGZsh{} Save the user using e.g. set\PYGZus{}secure\PYGZus{}cookie()}
        \PYG{k}{else}\PYG{p}{:}
            \PYG{k}{await} \PYG{n+nb+bp}{self}\PYG{o}{.}\PYG{n}{authorize\PYGZus{}redirect}\PYG{p}{(}\PYG{p}{)}
\end{sphinxVerbatim}

The user object returned by {\hyperref[\detokenize{auth:tornado.auth.OAuthMixin.get_authenticated_user}]{\sphinxcrossref{\sphinxcode{\sphinxupquote{get\_authenticated\_user}}}}}
includes the attributes \sphinxcode{\sphinxupquote{username}}, \sphinxcode{\sphinxupquote{name}}, \sphinxcode{\sphinxupquote{access\_token}},
and all of the custom Twitter user attributes described at
\sphinxurl{https://dev.twitter.com/docs/api/1.1/get/users/show}
\index{authenticate\_redirect() (tornado.auth.TwitterMixin method)@\spxentry{authenticate\_redirect()}\spxextra{tornado.auth.TwitterMixin method}}

\begin{fulllineitems}
\phantomsection\label{\detokenize{auth:tornado.auth.TwitterMixin.authenticate_redirect}}\pysiglinewithargsret{\sphinxbfcode{\sphinxupquote{coroutine }}\sphinxbfcode{\sphinxupquote{authenticate\_redirect}}}{\emph{callback\_uri: str = None}}{{ $\rightarrow$ None}}
Just like {\hyperref[\detokenize{auth:tornado.auth.OAuthMixin.authorize_redirect}]{\sphinxcrossref{\sphinxcode{\sphinxupquote{authorize\_redirect}}}}}, but
auto-redirects if authorized.

This is generally the right interface to use if you are using
Twitter for single-sign on.

\DUrole{versionmodified,changed}{Changed in version 3.1: }Now returns a {\hyperref[\detokenize{concurrent:tornado.concurrent.Future}]{\sphinxcrossref{\sphinxcode{\sphinxupquote{Future}}}}} and takes an optional callback, for
compatibility with {\hyperref[\detokenize{gen:tornado.gen.coroutine}]{\sphinxcrossref{\sphinxcode{\sphinxupquote{gen.coroutine}}}}}.

\DUrole{versionmodified,changed}{Changed in version 6.0: }The \sphinxcode{\sphinxupquote{callback}} argument was removed. Use the returned
awaitable object instead.

\end{fulllineitems}

\index{twitter\_request() (tornado.auth.TwitterMixin method)@\spxentry{twitter\_request()}\spxextra{tornado.auth.TwitterMixin method}}

\begin{fulllineitems}
\phantomsection\label{\detokenize{auth:tornado.auth.TwitterMixin.twitter_request}}\pysiglinewithargsret{\sphinxbfcode{\sphinxupquote{coroutine }}\sphinxbfcode{\sphinxupquote{twitter\_request}}}{\emph{path: str, access\_token: Dict{[}str, Any{]}, post\_args: Dict{[}str, Any{]} = None, **args}}{{ $\rightarrow$ Any}}
Fetches the given API path, e.g., \sphinxcode{\sphinxupquote{statuses/user\_timeline/btaylor}}

The path should not include the format or API version number.
(we automatically use JSON format and API version 1).

If the request is a POST, \sphinxcode{\sphinxupquote{post\_args}} should be provided. Query
string arguments should be given as keyword arguments.

All the Twitter methods are documented at \sphinxurl{http://dev.twitter.com/}

Many methods require an OAuth access token which you can
obtain through {\hyperref[\detokenize{auth:tornado.auth.OAuthMixin.authorize_redirect}]{\sphinxcrossref{\sphinxcode{\sphinxupquote{authorize\_redirect}}}}} and
{\hyperref[\detokenize{auth:tornado.auth.OAuthMixin.get_authenticated_user}]{\sphinxcrossref{\sphinxcode{\sphinxupquote{get\_authenticated\_user}}}}}. The user returned through that
process includes an ‘access\_token’ attribute that can be used
to make authenticated requests via this method. Example
usage:

\begin{sphinxVerbatim}[commandchars=\\\{\}]
\PYG{k}{class} \PYG{n+nc}{MainHandler}\PYG{p}{(}\PYG{n}{tornado}\PYG{o}{.}\PYG{n}{web}\PYG{o}{.}\PYG{n}{RequestHandler}\PYG{p}{,}
                  \PYG{n}{tornado}\PYG{o}{.}\PYG{n}{auth}\PYG{o}{.}\PYG{n}{TwitterMixin}\PYG{p}{)}\PYG{p}{:}
    \PYG{n+nd}{@tornado}\PYG{o}{.}\PYG{n}{web}\PYG{o}{.}\PYG{n}{authenticated}
    \PYG{k}{async} \PYG{k}{def} \PYG{n+nf}{get}\PYG{p}{(}\PYG{n+nb+bp}{self}\PYG{p}{)}\PYG{p}{:}
        \PYG{n}{new\PYGZus{}entry} \PYG{o}{=} \PYG{k}{await} \PYG{n+nb+bp}{self}\PYG{o}{.}\PYG{n}{twitter\PYGZus{}request}\PYG{p}{(}
            \PYG{l+s+s2}{\PYGZdq{}}\PYG{l+s+s2}{/statuses/update}\PYG{l+s+s2}{\PYGZdq{}}\PYG{p}{,}
            \PYG{n}{post\PYGZus{}args}\PYG{o}{=}\PYG{p}{\PYGZob{}}\PYG{l+s+s2}{\PYGZdq{}}\PYG{l+s+s2}{status}\PYG{l+s+s2}{\PYGZdq{}}\PYG{p}{:} \PYG{l+s+s2}{\PYGZdq{}}\PYG{l+s+s2}{Testing Tornado Web Server}\PYG{l+s+s2}{\PYGZdq{}}\PYG{p}{\PYGZcb{}}\PYG{p}{,}
            \PYG{n}{access\PYGZus{}token}\PYG{o}{=}\PYG{n+nb+bp}{self}\PYG{o}{.}\PYG{n}{current\PYGZus{}user}\PYG{p}{[}\PYG{l+s+s2}{\PYGZdq{}}\PYG{l+s+s2}{access\PYGZus{}token}\PYG{l+s+s2}{\PYGZdq{}}\PYG{p}{]}\PYG{p}{)}
        \PYG{k}{if} \PYG{o+ow}{not} \PYG{n}{new\PYGZus{}entry}\PYG{p}{:}
            \PYG{c+c1}{\PYGZsh{} Call failed; perhaps missing permission?}
            \PYG{k}{yield} \PYG{n+nb+bp}{self}\PYG{o}{.}\PYG{n}{authorize\PYGZus{}redirect}\PYG{p}{(}\PYG{p}{)}
            \PYG{k}{return}
        \PYG{n+nb+bp}{self}\PYG{o}{.}\PYG{n}{finish}\PYG{p}{(}\PYG{l+s+s2}{\PYGZdq{}}\PYG{l+s+s2}{Posted a message!}\PYG{l+s+s2}{\PYGZdq{}}\PYG{p}{)}
\end{sphinxVerbatim}

\DUrole{versionmodified,changed}{Changed in version 6.0: }The \sphinxcode{\sphinxupquote{callback}} argument was removed. Use the returned
awaitable object instead.

\end{fulllineitems}


\end{fulllineitems}



\subsection{\sphinxstyleliteralintitle{\sphinxupquote{tornado.wsgi}} — Interoperability with other Python frameworks and servers}
\label{\detokenize{wsgi:module-tornado.wsgi}}\label{\detokenize{wsgi:tornado-wsgi-interoperability-with-other-python-frameworks-and-servers}}\label{\detokenize{wsgi::doc}}\index{tornado.wsgi (module)@\spxentry{tornado.wsgi}\spxextra{module}}
WSGI support for the Tornado web framework.

WSGI is the Python standard for web servers, and allows for interoperability
between Tornado and other Python web frameworks and servers.

This module provides WSGI support via the {\hyperref[\detokenize{wsgi:tornado.wsgi.WSGIContainer}]{\sphinxcrossref{\sphinxcode{\sphinxupquote{WSGIContainer}}}}} class, which
makes it possible to run applications using other WSGI frameworks on
the Tornado HTTP server. The reverse is not supported; the Tornado
{\hyperref[\detokenize{web:tornado.web.Application}]{\sphinxcrossref{\sphinxcode{\sphinxupquote{Application}}}}} and {\hyperref[\detokenize{web:tornado.web.RequestHandler}]{\sphinxcrossref{\sphinxcode{\sphinxupquote{RequestHandler}}}}} classes are designed for use with
the Tornado {\hyperref[\detokenize{httpserver:tornado.httpserver.HTTPServer}]{\sphinxcrossref{\sphinxcode{\sphinxupquote{HTTPServer}}}}} and cannot be used in a generic WSGI
container.
\index{WSGIContainer (class in tornado.wsgi)@\spxentry{WSGIContainer}\spxextra{class in tornado.wsgi}}

\begin{fulllineitems}
\phantomsection\label{\detokenize{wsgi:tornado.wsgi.WSGIContainer}}\pysiglinewithargsret{\sphinxbfcode{\sphinxupquote{class }}\sphinxcode{\sphinxupquote{tornado.wsgi.}}\sphinxbfcode{\sphinxupquote{WSGIContainer}}}{\emph{wsgi\_application: WSGIAppType}}{}
Makes a WSGI-compatible function runnable on Tornado’s HTTP server.

\begin{sphinxadmonition}{warning}{Warning:}
WSGI is a \sphinxstyleemphasis{synchronous} interface, while Tornado’s concurrency model
is based on single-threaded asynchronous execution.  This means that
running a WSGI app with Tornado’s {\hyperref[\detokenize{wsgi:tornado.wsgi.WSGIContainer}]{\sphinxcrossref{\sphinxcode{\sphinxupquote{WSGIContainer}}}}} is \sphinxstyleemphasis{less scalable}
than running the same app in a multi-threaded WSGI server like
\sphinxcode{\sphinxupquote{gunicorn}} or \sphinxcode{\sphinxupquote{uwsgi}}.  Use {\hyperref[\detokenize{wsgi:tornado.wsgi.WSGIContainer}]{\sphinxcrossref{\sphinxcode{\sphinxupquote{WSGIContainer}}}}} only when there are
benefits to combining Tornado and WSGI in the same process that
outweigh the reduced scalability.
\end{sphinxadmonition}

Wrap a WSGI function in a {\hyperref[\detokenize{wsgi:tornado.wsgi.WSGIContainer}]{\sphinxcrossref{\sphinxcode{\sphinxupquote{WSGIContainer}}}}} and pass it to {\hyperref[\detokenize{httpserver:tornado.httpserver.HTTPServer}]{\sphinxcrossref{\sphinxcode{\sphinxupquote{HTTPServer}}}}} to
run it. For example:

\begin{sphinxVerbatim}[commandchars=\\\{\}]
\PYG{k}{def} \PYG{n+nf}{simple\PYGZus{}app}\PYG{p}{(}\PYG{n}{environ}\PYG{p}{,} \PYG{n}{start\PYGZus{}response}\PYG{p}{)}\PYG{p}{:}
    \PYG{n}{status} \PYG{o}{=} \PYG{l+s+s2}{\PYGZdq{}}\PYG{l+s+s2}{200 OK}\PYG{l+s+s2}{\PYGZdq{}}
    \PYG{n}{response\PYGZus{}headers} \PYG{o}{=} \PYG{p}{[}\PYG{p}{(}\PYG{l+s+s2}{\PYGZdq{}}\PYG{l+s+s2}{Content\PYGZhy{}type}\PYG{l+s+s2}{\PYGZdq{}}\PYG{p}{,} \PYG{l+s+s2}{\PYGZdq{}}\PYG{l+s+s2}{text/plain}\PYG{l+s+s2}{\PYGZdq{}}\PYG{p}{)}\PYG{p}{]}
    \PYG{n}{start\PYGZus{}response}\PYG{p}{(}\PYG{n}{status}\PYG{p}{,} \PYG{n}{response\PYGZus{}headers}\PYG{p}{)}
    \PYG{k}{return} \PYG{p}{[}\PYG{l+s+s2}{\PYGZdq{}}\PYG{l+s+s2}{Hello world!}\PYG{l+s+se}{\PYGZbs{}n}\PYG{l+s+s2}{\PYGZdq{}}\PYG{p}{]}

\PYG{n}{container} \PYG{o}{=} \PYG{n}{tornado}\PYG{o}{.}\PYG{n}{wsgi}\PYG{o}{.}\PYG{n}{WSGIContainer}\PYG{p}{(}\PYG{n}{simple\PYGZus{}app}\PYG{p}{)}
\PYG{n}{http\PYGZus{}server} \PYG{o}{=} \PYG{n}{tornado}\PYG{o}{.}\PYG{n}{httpserver}\PYG{o}{.}\PYG{n}{HTTPServer}\PYG{p}{(}\PYG{n}{container}\PYG{p}{)}
\PYG{n}{http\PYGZus{}server}\PYG{o}{.}\PYG{n}{listen}\PYG{p}{(}\PYG{l+m+mi}{8888}\PYG{p}{)}
\PYG{n}{tornado}\PYG{o}{.}\PYG{n}{ioloop}\PYG{o}{.}\PYG{n}{IOLoop}\PYG{o}{.}\PYG{n}{current}\PYG{p}{(}\PYG{p}{)}\PYG{o}{.}\PYG{n}{start}\PYG{p}{(}\PYG{p}{)}
\end{sphinxVerbatim}

This class is intended to let other frameworks (Django, web.py, etc)
run on the Tornado HTTP server and I/O loop.

The {\hyperref[\detokenize{web:tornado.web.FallbackHandler}]{\sphinxcrossref{\sphinxcode{\sphinxupquote{tornado.web.FallbackHandler}}}}} class is often useful for mixing
Tornado and WSGI apps in the same server.  See
\sphinxurl{https://github.com/bdarnell/django-tornado-demo} for a complete example.
\index{environ() (tornado.wsgi.WSGIContainer static method)@\spxentry{environ()}\spxextra{tornado.wsgi.WSGIContainer static method}}

\begin{fulllineitems}
\phantomsection\label{\detokenize{wsgi:tornado.wsgi.WSGIContainer.environ}}\pysiglinewithargsret{\sphinxbfcode{\sphinxupquote{static }}\sphinxbfcode{\sphinxupquote{environ}}}{\emph{request: tornado.httputil.HTTPServerRequest}}{{ $\rightarrow$ Dict{[}str, Any{]}}}
Converts a {\hyperref[\detokenize{httputil:tornado.httputil.HTTPServerRequest}]{\sphinxcrossref{\sphinxcode{\sphinxupquote{tornado.httputil.HTTPServerRequest}}}}} to a WSGI environment.

\end{fulllineitems}


\end{fulllineitems}



\subsection{\sphinxstyleliteralintitle{\sphinxupquote{tornado.platform.caresresolver}} — Asynchronous DNS Resolver using C-Ares}
\label{\detokenize{caresresolver:module-tornado.platform.caresresolver}}\label{\detokenize{caresresolver:tornado-platform-caresresolver-asynchronous-dns-resolver-using-c-ares}}\label{\detokenize{caresresolver::doc}}\index{tornado.platform.caresresolver (module)@\spxentry{tornado.platform.caresresolver}\spxextra{module}}
This module contains a DNS resolver using the c-ares library (and its
wrapper \sphinxcode{\sphinxupquote{pycares}}).
\index{CaresResolver (class in tornado.platform.caresresolver)@\spxentry{CaresResolver}\spxextra{class in tornado.platform.caresresolver}}

\begin{fulllineitems}
\phantomsection\label{\detokenize{caresresolver:tornado.platform.caresresolver.CaresResolver}}\pysigline{\sphinxbfcode{\sphinxupquote{class }}\sphinxcode{\sphinxupquote{tornado.platform.caresresolver.}}\sphinxbfcode{\sphinxupquote{CaresResolver}}}
Name resolver based on the c-ares library.

This is a non-blocking and non-threaded resolver.  It may not produce
the same results as the system resolver, but can be used for non-blocking
resolution when threads cannot be used.

c-ares fails to resolve some names when \sphinxcode{\sphinxupquote{family}} is \sphinxcode{\sphinxupquote{AF\_UNSPEC}},
so it is only recommended for use in \sphinxcode{\sphinxupquote{AF\_INET}} (i.e. IPv4).  This is
the default for \sphinxcode{\sphinxupquote{tornado.simple\_httpclient}}, but other libraries
may default to \sphinxcode{\sphinxupquote{AF\_UNSPEC}}.

\end{fulllineitems}



\subsection{\sphinxstyleliteralintitle{\sphinxupquote{tornado.platform.twisted}} — Bridges between Twisted and Tornado}
\label{\detokenize{twisted:module-tornado.platform.twisted}}\label{\detokenize{twisted:tornado-platform-twisted-bridges-between-twisted-and-tornado}}\label{\detokenize{twisted::doc}}\index{tornado.platform.twisted (module)@\spxentry{tornado.platform.twisted}\spxextra{module}}
Bridges between the Twisted reactor and Tornado IOLoop.

This module lets you run applications and libraries written for
Twisted in a Tornado application.  It can be used in two modes,
depending on which library’s underlying event loop you want to use.


\subsubsection{Twisted DNS resolver}
\label{\detokenize{twisted:twisted-dns-resolver}}\index{TwistedResolver (class in tornado.platform.twisted)@\spxentry{TwistedResolver}\spxextra{class in tornado.platform.twisted}}

\begin{fulllineitems}
\phantomsection\label{\detokenize{twisted:tornado.platform.twisted.TwistedResolver}}\pysigline{\sphinxbfcode{\sphinxupquote{class }}\sphinxcode{\sphinxupquote{tornado.platform.twisted.}}\sphinxbfcode{\sphinxupquote{TwistedResolver}}}
Twisted-based asynchronous resolver.

This is a non-blocking and non-threaded resolver.  It is
recommended only when threads cannot be used, since it has
limitations compared to the standard \sphinxcode{\sphinxupquote{getaddrinfo}}-based
{\hyperref[\detokenize{netutil:tornado.netutil.Resolver}]{\sphinxcrossref{\sphinxcode{\sphinxupquote{Resolver}}}}} and
{\hyperref[\detokenize{netutil:tornado.netutil.DefaultExecutorResolver}]{\sphinxcrossref{\sphinxcode{\sphinxupquote{DefaultExecutorResolver}}}}}.  Specifically, it returns at
most one result, and arguments other than \sphinxcode{\sphinxupquote{host}} and \sphinxcode{\sphinxupquote{family}}
are ignored.  It may fail to resolve when \sphinxcode{\sphinxupquote{family}} is not
\sphinxcode{\sphinxupquote{socket.AF\_UNSPEC}}.

Requires Twisted 12.1 or newer.

\DUrole{versionmodified,changed}{Changed in version 5.0: }The \sphinxcode{\sphinxupquote{io\_loop}} argument (deprecated since version 4.1) has been removed.

\end{fulllineitems}



\subsection{\sphinxstyleliteralintitle{\sphinxupquote{tornado.platform.asyncio}} — Bridge between \sphinxstyleliteralintitle{\sphinxupquote{asyncio}} and Tornado}
\label{\detokenize{asyncio:module-tornado.platform.asyncio}}\label{\detokenize{asyncio:tornado-platform-asyncio-bridge-between-asyncio-and-tornado}}\label{\detokenize{asyncio::doc}}\index{tornado.platform.asyncio (module)@\spxentry{tornado.platform.asyncio}\spxextra{module}}
Bridges between the \sphinxhref{https://docs.python.org/3.6/library/asyncio.html\#module-asyncio}{\sphinxcode{\sphinxupquote{asyncio}}} module and Tornado IOLoop.

\DUrole{versionmodified,added}{New in version 3.2.}

This module integrates Tornado with the \sphinxcode{\sphinxupquote{asyncio}} module introduced
in Python 3.4. This makes it possible to combine the two libraries on
the same event loop.

\DUrole{versionmodified,deprecated}{Deprecated since version 5.0: }While the code in this module is still used, it is now enabled
automatically when \sphinxhref{https://docs.python.org/3.6/library/asyncio.html\#module-asyncio}{\sphinxcode{\sphinxupquote{asyncio}}} is available, so applications should
no longer need to refer to this module directly.

\begin{sphinxadmonition}{note}{Note:}
Tornado requires the \sphinxhref{https://docs.python.org/3.6/library/asyncio-eventloop.html\#asyncio.AbstractEventLoop.add\_reader}{\sphinxcode{\sphinxupquote{add\_reader}}} family of
methods, so it is not compatible with the \sphinxhref{https://docs.python.org/3.6/library/asyncio-eventloops.html\#asyncio.ProactorEventLoop}{\sphinxcode{\sphinxupquote{ProactorEventLoop}}} on
Windows. Use the \sphinxhref{https://docs.python.org/3.6/library/asyncio-eventloops.html\#asyncio.SelectorEventLoop}{\sphinxcode{\sphinxupquote{SelectorEventLoop}}} instead.
\end{sphinxadmonition}
\index{AsyncIOMainLoop (class in tornado.platform.asyncio)@\spxentry{AsyncIOMainLoop}\spxextra{class in tornado.platform.asyncio}}

\begin{fulllineitems}
\phantomsection\label{\detokenize{asyncio:tornado.platform.asyncio.AsyncIOMainLoop}}\pysigline{\sphinxbfcode{\sphinxupquote{class }}\sphinxcode{\sphinxupquote{tornado.platform.asyncio.}}\sphinxbfcode{\sphinxupquote{AsyncIOMainLoop}}}
\sphinxcode{\sphinxupquote{AsyncIOMainLoop}} creates an {\hyperref[\detokenize{ioloop:tornado.ioloop.IOLoop}]{\sphinxcrossref{\sphinxcode{\sphinxupquote{IOLoop}}}}} that corresponds to the
current \sphinxcode{\sphinxupquote{asyncio}} event loop (i.e. the one returned by
\sphinxcode{\sphinxupquote{asyncio.get\_event\_loop()}}).

\DUrole{versionmodified,deprecated}{Deprecated since version 5.0: }Now used automatically when appropriate; it is no longer necessary
to refer to this class directly.

\DUrole{versionmodified,changed}{Changed in version 5.0: }Closing an {\hyperref[\detokenize{asyncio:tornado.platform.asyncio.AsyncIOMainLoop}]{\sphinxcrossref{\sphinxcode{\sphinxupquote{AsyncIOMainLoop}}}}} now closes the underlying asyncio loop.

\end{fulllineitems}

\index{AsyncIOLoop (class in tornado.platform.asyncio)@\spxentry{AsyncIOLoop}\spxextra{class in tornado.platform.asyncio}}

\begin{fulllineitems}
\phantomsection\label{\detokenize{asyncio:tornado.platform.asyncio.AsyncIOLoop}}\pysigline{\sphinxbfcode{\sphinxupquote{class }}\sphinxcode{\sphinxupquote{tornado.platform.asyncio.}}\sphinxbfcode{\sphinxupquote{AsyncIOLoop}}}
\sphinxcode{\sphinxupquote{AsyncIOLoop}} is an {\hyperref[\detokenize{ioloop:tornado.ioloop.IOLoop}]{\sphinxcrossref{\sphinxcode{\sphinxupquote{IOLoop}}}}} that runs on an \sphinxcode{\sphinxupquote{asyncio}} event loop.
This class follows the usual Tornado semantics for creating new
\sphinxcode{\sphinxupquote{IOLoops}}; these loops are not necessarily related to the
\sphinxcode{\sphinxupquote{asyncio}} default event loop.

Each \sphinxcode{\sphinxupquote{AsyncIOLoop}} creates a new \sphinxcode{\sphinxupquote{asyncio.EventLoop}}; this object
can be accessed with the \sphinxcode{\sphinxupquote{asyncio\_loop}} attribute.

\DUrole{versionmodified,changed}{Changed in version 5.0: }When an \sphinxcode{\sphinxupquote{AsyncIOLoop}} becomes the current {\hyperref[\detokenize{ioloop:tornado.ioloop.IOLoop}]{\sphinxcrossref{\sphinxcode{\sphinxupquote{IOLoop}}}}}, it also sets
the current \sphinxhref{https://docs.python.org/3.6/library/asyncio.html\#module-asyncio}{\sphinxcode{\sphinxupquote{asyncio}}} event loop.

\DUrole{versionmodified,deprecated}{Deprecated since version 5.0: }Now used automatically when appropriate; it is no longer necessary
to refer to this class directly.

\end{fulllineitems}

\index{to\_tornado\_future() (in module tornado.platform.asyncio)@\spxentry{to\_tornado\_future()}\spxextra{in module tornado.platform.asyncio}}

\begin{fulllineitems}
\phantomsection\label{\detokenize{asyncio:tornado.platform.asyncio.to_tornado_future}}\pysiglinewithargsret{\sphinxcode{\sphinxupquote{tornado.platform.asyncio.}}\sphinxbfcode{\sphinxupquote{to\_tornado\_future}}}{\emph{asyncio\_future: \_asyncio.Future}}{{ $\rightarrow$ \_asyncio.Future}}
Convert an \sphinxhref{https://docs.python.org/3.6/library/asyncio-task.html\#asyncio.Future}{\sphinxcode{\sphinxupquote{asyncio.Future}}} to a {\hyperref[\detokenize{concurrent:tornado.concurrent.Future}]{\sphinxcrossref{\sphinxcode{\sphinxupquote{tornado.concurrent.Future}}}}}.

\DUrole{versionmodified,added}{New in version 4.1.}

\DUrole{versionmodified,deprecated}{Deprecated since version 5.0: }Tornado \sphinxcode{\sphinxupquote{Futures}} have been merged with \sphinxhref{https://docs.python.org/3.6/library/asyncio-task.html\#asyncio.Future}{\sphinxcode{\sphinxupquote{asyncio.Future}}},
so this method is now a no-op.

\end{fulllineitems}

\index{to\_asyncio\_future() (in module tornado.platform.asyncio)@\spxentry{to\_asyncio\_future()}\spxextra{in module tornado.platform.asyncio}}

\begin{fulllineitems}
\phantomsection\label{\detokenize{asyncio:tornado.platform.asyncio.to_asyncio_future}}\pysiglinewithargsret{\sphinxcode{\sphinxupquote{tornado.platform.asyncio.}}\sphinxbfcode{\sphinxupquote{to\_asyncio\_future}}}{\emph{tornado\_future: \_asyncio.Future}}{{ $\rightarrow$ \_asyncio.Future}}
Convert a Tornado yieldable object to an \sphinxhref{https://docs.python.org/3.6/library/asyncio-task.html\#asyncio.Future}{\sphinxcode{\sphinxupquote{asyncio.Future}}}.

\DUrole{versionmodified,added}{New in version 4.1.}

\DUrole{versionmodified,changed}{Changed in version 4.3: }Now accepts any yieldable object, not just
{\hyperref[\detokenize{concurrent:tornado.concurrent.Future}]{\sphinxcrossref{\sphinxcode{\sphinxupquote{tornado.concurrent.Future}}}}}.

\DUrole{versionmodified,deprecated}{Deprecated since version 5.0: }Tornado \sphinxcode{\sphinxupquote{Futures}} have been merged with \sphinxhref{https://docs.python.org/3.6/library/asyncio-task.html\#asyncio.Future}{\sphinxcode{\sphinxupquote{asyncio.Future}}},
so this method is now equivalent to {\hyperref[\detokenize{gen:tornado.gen.convert_yielded}]{\sphinxcrossref{\sphinxcode{\sphinxupquote{tornado.gen.convert\_yielded}}}}}.

\end{fulllineitems}

\index{AnyThreadEventLoopPolicy (class in tornado.platform.asyncio)@\spxentry{AnyThreadEventLoopPolicy}\spxextra{class in tornado.platform.asyncio}}

\begin{fulllineitems}
\phantomsection\label{\detokenize{asyncio:tornado.platform.asyncio.AnyThreadEventLoopPolicy}}\pysigline{\sphinxbfcode{\sphinxupquote{class }}\sphinxcode{\sphinxupquote{tornado.platform.asyncio.}}\sphinxbfcode{\sphinxupquote{AnyThreadEventLoopPolicy}}}
Event loop policy that allows loop creation on any thread.

The default \sphinxhref{https://docs.python.org/3.6/library/asyncio.html\#module-asyncio}{\sphinxcode{\sphinxupquote{asyncio}}} event loop policy only automatically creates
event loops in the main threads. Other threads must create event
loops explicitly or \sphinxhref{https://docs.python.org/3.6/library/asyncio-eventloops.html\#asyncio.get\_event\_loop}{\sphinxcode{\sphinxupquote{asyncio.get\_event\_loop}}} (and therefore
{\hyperref[\detokenize{ioloop:tornado.ioloop.IOLoop.current}]{\sphinxcrossref{\sphinxcode{\sphinxupquote{IOLoop.current}}}}}) will fail. Installing this policy allows event
loops to be created automatically on any thread, matching the
behavior of Tornado versions prior to 5.0 (or 5.0 on Python 2).

Usage:

\begin{sphinxVerbatim}[commandchars=\\\{\}]
\PYG{n}{asyncio}\PYG{o}{.}\PYG{n}{set\PYGZus{}event\PYGZus{}loop\PYGZus{}policy}\PYG{p}{(}\PYG{n}{AnyThreadEventLoopPolicy}\PYG{p}{(}\PYG{p}{)}\PYG{p}{)}
\end{sphinxVerbatim}

\DUrole{versionmodified,added}{New in version 5.0.}

\end{fulllineitems}



\section{Utilities}
\label{\detokenize{utilities:utilities}}\label{\detokenize{utilities::doc}}

\subsection{\sphinxstyleliteralintitle{\sphinxupquote{tornado.autoreload}} — Automatically detect code changes in development}
\label{\detokenize{autoreload:module-tornado.autoreload}}\label{\detokenize{autoreload:tornado-autoreload-automatically-detect-code-changes-in-development}}\label{\detokenize{autoreload::doc}}\index{tornado.autoreload (module)@\spxentry{tornado.autoreload}\spxextra{module}}
Automatically restart the server when a source file is modified.

Most applications should not access this module directly.  Instead,
pass the keyword argument \sphinxcode{\sphinxupquote{autoreload=True}} to the
{\hyperref[\detokenize{web:tornado.web.Application}]{\sphinxcrossref{\sphinxcode{\sphinxupquote{tornado.web.Application}}}}} constructor (or \sphinxcode{\sphinxupquote{debug=True}}, which
enables this setting and several others).  This will enable autoreload
mode as well as checking for changes to templates and static
resources.  Note that restarting is a destructive operation and any
requests in progress will be aborted when the process restarts.  (If
you want to disable autoreload while using other debug-mode features,
pass both \sphinxcode{\sphinxupquote{debug=True}} and \sphinxcode{\sphinxupquote{autoreload=False}}).

This module can also be used as a command-line wrapper around scripts
such as unit test runners.  See the {\hyperref[\detokenize{autoreload:tornado.autoreload.main}]{\sphinxcrossref{\sphinxcode{\sphinxupquote{main}}}}} method for details.

The command-line wrapper and Application debug modes can be used together.
This combination is encouraged as the wrapper catches syntax errors and
other import-time failures, while debug mode catches changes once
the server has started.

This module will not work correctly when {\hyperref[\detokenize{httpserver:tornado.httpserver.HTTPServer}]{\sphinxcrossref{\sphinxcode{\sphinxupquote{HTTPServer}}}}}’s multi-process
mode is used.

Reloading loses any Python interpreter command-line arguments (e.g. \sphinxcode{\sphinxupquote{-u}})
because it re-executes Python using \sphinxcode{\sphinxupquote{sys.executable}} and \sphinxcode{\sphinxupquote{sys.argv}}.
Additionally, modifying these variables will cause reloading to behave
incorrectly.
\index{start() (in module tornado.autoreload)@\spxentry{start()}\spxextra{in module tornado.autoreload}}

\begin{fulllineitems}
\phantomsection\label{\detokenize{autoreload:tornado.autoreload.start}}\pysiglinewithargsret{\sphinxcode{\sphinxupquote{tornado.autoreload.}}\sphinxbfcode{\sphinxupquote{start}}}{\emph{check\_time: int = 500}}{{ $\rightarrow$ None}}
Begins watching source files for changes.

\DUrole{versionmodified,changed}{Changed in version 5.0: }The \sphinxcode{\sphinxupquote{io\_loop}} argument (deprecated since version 4.1) has been removed.

\end{fulllineitems}

\index{wait() (in module tornado.autoreload)@\spxentry{wait()}\spxextra{in module tornado.autoreload}}

\begin{fulllineitems}
\phantomsection\label{\detokenize{autoreload:tornado.autoreload.wait}}\pysiglinewithargsret{\sphinxcode{\sphinxupquote{tornado.autoreload.}}\sphinxbfcode{\sphinxupquote{wait}}}{}{{ $\rightarrow$ None}}
Wait for a watched file to change, then restart the process.

Intended to be used at the end of scripts like unit test runners,
to run the tests again after any source file changes (but see also
the command-line interface in {\hyperref[\detokenize{autoreload:tornado.autoreload.main}]{\sphinxcrossref{\sphinxcode{\sphinxupquote{main}}}}})

\end{fulllineitems}

\index{watch() (in module tornado.autoreload)@\spxentry{watch()}\spxextra{in module tornado.autoreload}}

\begin{fulllineitems}
\phantomsection\label{\detokenize{autoreload:tornado.autoreload.watch}}\pysiglinewithargsret{\sphinxcode{\sphinxupquote{tornado.autoreload.}}\sphinxbfcode{\sphinxupquote{watch}}}{\emph{filename: str}}{{ $\rightarrow$ None}}
Add a file to the watch list.

All imported modules are watched by default.

\end{fulllineitems}

\index{add\_reload\_hook() (in module tornado.autoreload)@\spxentry{add\_reload\_hook()}\spxextra{in module tornado.autoreload}}

\begin{fulllineitems}
\phantomsection\label{\detokenize{autoreload:tornado.autoreload.add_reload_hook}}\pysiglinewithargsret{\sphinxcode{\sphinxupquote{tornado.autoreload.}}\sphinxbfcode{\sphinxupquote{add\_reload\_hook}}}{\emph{fn: Callable{[}{[}{]}, None{]}}}{{ $\rightarrow$ None}}
Add a function to be called before reloading the process.

Note that for open file and socket handles it is generally
preferable to set the \sphinxcode{\sphinxupquote{FD\_CLOEXEC}} flag (using \sphinxhref{https://docs.python.org/3.6/library/fcntl.html\#module-fcntl}{\sphinxcode{\sphinxupquote{fcntl}}} or
\sphinxcode{\sphinxupquote{tornado.platform.auto.set\_close\_exec}}) instead
of using a reload hook to close them.

\end{fulllineitems}

\index{main() (in module tornado.autoreload)@\spxentry{main()}\spxextra{in module tornado.autoreload}}

\begin{fulllineitems}
\phantomsection\label{\detokenize{autoreload:tornado.autoreload.main}}\pysiglinewithargsret{\sphinxcode{\sphinxupquote{tornado.autoreload.}}\sphinxbfcode{\sphinxupquote{main}}}{}{{ $\rightarrow$ None}}
Command-line wrapper to re-run a script whenever its source changes.

Scripts may be specified by filename or module name:

\begin{sphinxVerbatim}[commandchars=\\\{\}]
\PYG{n}{python} \PYG{o}{\PYGZhy{}}\PYG{n}{m} \PYG{n}{tornado}\PYG{o}{.}\PYG{n}{autoreload} \PYG{o}{\PYGZhy{}}\PYG{n}{m} \PYG{n}{tornado}\PYG{o}{.}\PYG{n}{test}\PYG{o}{.}\PYG{n}{runtests}
\PYG{n}{python} \PYG{o}{\PYGZhy{}}\PYG{n}{m} \PYG{n}{tornado}\PYG{o}{.}\PYG{n}{autoreload} \PYG{n}{tornado}\PYG{o}{/}\PYG{n}{test}\PYG{o}{/}\PYG{n}{runtests}\PYG{o}{.}\PYG{n}{py}
\end{sphinxVerbatim}

Running a script with this wrapper is similar to calling
{\hyperref[\detokenize{autoreload:tornado.autoreload.wait}]{\sphinxcrossref{\sphinxcode{\sphinxupquote{tornado.autoreload.wait}}}}} at the end of the script, but this wrapper
can catch import-time problems like syntax errors that would otherwise
prevent the script from reaching its call to {\hyperref[\detokenize{autoreload:tornado.autoreload.wait}]{\sphinxcrossref{\sphinxcode{\sphinxupquote{wait}}}}}.

\end{fulllineitems}



\subsection{\sphinxstyleliteralintitle{\sphinxupquote{tornado.concurrent}} — Work with \sphinxstyleliteralintitle{\sphinxupquote{Future}} objects}
\label{\detokenize{concurrent:tornado-concurrent-work-with-future-objects}}\label{\detokenize{concurrent::doc}}\phantomsection\label{\detokenize{concurrent:module-tornado.concurrent}}\index{tornado.concurrent (module)@\spxentry{tornado.concurrent}\spxextra{module}}
Utilities for working with \sphinxcode{\sphinxupquote{Future}} objects.

Tornado previously provided its own \sphinxcode{\sphinxupquote{Future}} class, but now uses
\sphinxhref{https://docs.python.org/3.6/library/asyncio-task.html\#asyncio.Future}{\sphinxcode{\sphinxupquote{asyncio.Future}}}. This module contains utility functions for working
with \sphinxhref{https://docs.python.org/3.6/library/asyncio-task.html\#asyncio.Future}{\sphinxcode{\sphinxupquote{asyncio.Future}}} in a way that is backwards-compatible with
Tornado’s old \sphinxcode{\sphinxupquote{Future}} implementation.

While this module is an important part of Tornado’s internal
implementation, applications rarely need to interact with it
directly.
\begin{quote}
\index{Future (class in tornado.concurrent)@\spxentry{Future}\spxextra{class in tornado.concurrent}}

\begin{fulllineitems}
\phantomsection\label{\detokenize{concurrent:tornado.concurrent.Future}}\pysigline{\sphinxbfcode{\sphinxupquote{class }}\sphinxcode{\sphinxupquote{tornado.concurrent.}}\sphinxbfcode{\sphinxupquote{Future}}}
\sphinxcode{\sphinxupquote{tornado.concurrent.Future}} is an alias for \sphinxhref{https://docs.python.org/3.6/library/asyncio-task.html\#asyncio.Future}{\sphinxcode{\sphinxupquote{asyncio.Future}}}.

In Tornado, the main way in which applications interact with
\sphinxcode{\sphinxupquote{Future}} objects is by \sphinxcode{\sphinxupquote{awaiting}} or \sphinxcode{\sphinxupquote{yielding}} them in
coroutines, instead of calling methods on the \sphinxcode{\sphinxupquote{Future}} objects
themselves. For more information on the available methods, see
the \sphinxhref{https://docs.python.org/3.6/library/asyncio-task.html\#asyncio.Future}{\sphinxcode{\sphinxupquote{asyncio.Future}}} docs.

\DUrole{versionmodified,changed}{Changed in version 5.0: }Tornado’s implementation of \sphinxcode{\sphinxupquote{Future}} has been replaced by
the version from \sphinxhref{https://docs.python.org/3.6/library/asyncio.html\#module-asyncio}{\sphinxcode{\sphinxupquote{asyncio}}} when available.
\begin{itemize}
\item {} 
\sphinxcode{\sphinxupquote{Future}} objects can only be created while there is a
current {\hyperref[\detokenize{ioloop:tornado.ioloop.IOLoop}]{\sphinxcrossref{\sphinxcode{\sphinxupquote{IOLoop}}}}}

\item {} 
The timing of callbacks scheduled with
\sphinxcode{\sphinxupquote{Future.add\_done\_callback}} has changed.

\item {} 
Cancellation is now partially supported (only on Python 3)

\item {} 
The \sphinxcode{\sphinxupquote{exc\_info}} and \sphinxcode{\sphinxupquote{set\_exc\_info}} methods are no longer
available on Python 3.

\end{itemize}

\end{fulllineitems}

\end{quote}
\index{run\_on\_executor() (in module tornado.concurrent)@\spxentry{run\_on\_executor()}\spxextra{in module tornado.concurrent}}

\begin{fulllineitems}
\phantomsection\label{\detokenize{concurrent:tornado.concurrent.run_on_executor}}\pysiglinewithargsret{\sphinxcode{\sphinxupquote{tornado.concurrent.}}\sphinxbfcode{\sphinxupquote{run\_on\_executor}}}{\emph{*args}, \emph{**kwargs}}{{ $\rightarrow$ Callable}}
Decorator to run a synchronous method asynchronously on an executor.

The decorated method may be called with a \sphinxcode{\sphinxupquote{callback}} keyword
argument and returns a future.

The executor to be used is determined by the \sphinxcode{\sphinxupquote{executor}}
attributes of \sphinxcode{\sphinxupquote{self}}. To use a different attribute name, pass a
keyword argument to the decorator:

\begin{sphinxVerbatim}[commandchars=\\\{\}]
\PYG{n+nd}{@run\PYGZus{}on\PYGZus{}executor}\PYG{p}{(}\PYG{n}{executor}\PYG{o}{=}\PYG{l+s+s1}{\PYGZsq{}}\PYG{l+s+s1}{\PYGZus{}thread\PYGZus{}pool}\PYG{l+s+s1}{\PYGZsq{}}\PYG{p}{)}
\PYG{k}{def} \PYG{n+nf}{foo}\PYG{p}{(}\PYG{n+nb+bp}{self}\PYG{p}{)}\PYG{p}{:}
    \PYG{k}{pass}
\end{sphinxVerbatim}

This decorator should not be confused with the similarly-named
{\hyperref[\detokenize{ioloop:tornado.ioloop.IOLoop.run_in_executor}]{\sphinxcrossref{\sphinxcode{\sphinxupquote{IOLoop.run\_in\_executor}}}}}. In general, using \sphinxcode{\sphinxupquote{run\_in\_executor}}
when \sphinxstyleemphasis{calling} a blocking method is recommended instead of using
this decorator when \sphinxstyleemphasis{defining} a method. If compatibility with older
versions of Tornado is required, consider defining an executor
and using \sphinxcode{\sphinxupquote{executor.submit()}} at the call site.

\DUrole{versionmodified,changed}{Changed in version 4.2: }Added keyword arguments to use alternative attributes.

\DUrole{versionmodified,changed}{Changed in version 5.0: }Always uses the current IOLoop instead of \sphinxcode{\sphinxupquote{self.io\_loop}}.

\DUrole{versionmodified,changed}{Changed in version 5.1: }Returns a {\hyperref[\detokenize{concurrent:tornado.concurrent.Future}]{\sphinxcrossref{\sphinxcode{\sphinxupquote{Future}}}}} compatible with \sphinxcode{\sphinxupquote{await}} instead of a
\sphinxhref{https://docs.python.org/3.6/library/concurrent.futures.html\#concurrent.futures.Future}{\sphinxcode{\sphinxupquote{concurrent.futures.Future}}}.

\DUrole{versionmodified,deprecated}{Deprecated since version 5.1: }The \sphinxcode{\sphinxupquote{callback}} argument is deprecated and will be removed in
6.0. The decorator itself is discouraged in new code but will
not be removed in 6.0.

\DUrole{versionmodified,changed}{Changed in version 6.0: }The \sphinxcode{\sphinxupquote{callback}} argument was removed.

\end{fulllineitems}

\index{chain\_future() (in module tornado.concurrent)@\spxentry{chain\_future()}\spxextra{in module tornado.concurrent}}

\begin{fulllineitems}
\phantomsection\label{\detokenize{concurrent:tornado.concurrent.chain_future}}\pysiglinewithargsret{\sphinxcode{\sphinxupquote{tornado.concurrent.}}\sphinxbfcode{\sphinxupquote{chain\_future}}}{\emph{a: Future{[}\_T{]}, b: Future{[}\_T{]}}}{{ $\rightarrow$ None}}
Chain two futures together so that when one completes, so does the other.

The result (success or failure) of \sphinxcode{\sphinxupquote{a}} will be copied to \sphinxcode{\sphinxupquote{b}}, unless
\sphinxcode{\sphinxupquote{b}} has already been completed or cancelled by the time \sphinxcode{\sphinxupquote{a}} finishes.

\DUrole{versionmodified,changed}{Changed in version 5.0: }Now accepts both Tornado/asyncio {\hyperref[\detokenize{concurrent:tornado.concurrent.Future}]{\sphinxcrossref{\sphinxcode{\sphinxupquote{Future}}}}} objects and
\sphinxhref{https://docs.python.org/3.6/library/concurrent.futures.html\#concurrent.futures.Future}{\sphinxcode{\sphinxupquote{concurrent.futures.Future}}}.

\end{fulllineitems}

\index{future\_set\_result\_unless\_cancelled() (in module tornado.concurrent)@\spxentry{future\_set\_result\_unless\_cancelled()}\spxextra{in module tornado.concurrent}}

\begin{fulllineitems}
\phantomsection\label{\detokenize{concurrent:tornado.concurrent.future_set_result_unless_cancelled}}\pysiglinewithargsret{\sphinxcode{\sphinxupquote{tornado.concurrent.}}\sphinxbfcode{\sphinxupquote{future\_set\_result\_unless\_cancelled}}}{\emph{future: Union{[}futures.Future{[}\_T{]}, Future{[}\_T{]}{]}, value: \_T}}{{ $\rightarrow$ None}}
Set the given \sphinxcode{\sphinxupquote{value}} as the {\hyperref[\detokenize{concurrent:tornado.concurrent.Future}]{\sphinxcrossref{\sphinxcode{\sphinxupquote{Future}}}}}’s result, if not cancelled.

Avoids \sphinxcode{\sphinxupquote{asyncio.InvalidStateError}} when calling \sphinxcode{\sphinxupquote{set\_result()}} on
a cancelled \sphinxhref{https://docs.python.org/3.6/library/asyncio-task.html\#asyncio.Future}{\sphinxcode{\sphinxupquote{asyncio.Future}}}.

\DUrole{versionmodified,added}{New in version 5.0.}

\end{fulllineitems}

\index{future\_set\_exception\_unless\_cancelled() (in module tornado.concurrent)@\spxentry{future\_set\_exception\_unless\_cancelled()}\spxextra{in module tornado.concurrent}}

\begin{fulllineitems}
\phantomsection\label{\detokenize{concurrent:tornado.concurrent.future_set_exception_unless_cancelled}}\pysiglinewithargsret{\sphinxcode{\sphinxupquote{tornado.concurrent.}}\sphinxbfcode{\sphinxupquote{future\_set\_exception\_unless\_cancelled}}}{\emph{future: Union{[}futures.Future{[}\_T{]}, Future{[}\_T{]}{]}, exc: BaseException}}{{ $\rightarrow$ None}}
Set the given \sphinxcode{\sphinxupquote{exc}} as the {\hyperref[\detokenize{concurrent:tornado.concurrent.Future}]{\sphinxcrossref{\sphinxcode{\sphinxupquote{Future}}}}}’s exception.

If the Future is already canceled, logs the exception instead. If
this logging is not desired, the caller should explicitly check
the state of the Future and call \sphinxcode{\sphinxupquote{Future.set\_exception}} instead of
this wrapper.

Avoids \sphinxcode{\sphinxupquote{asyncio.InvalidStateError}} when calling \sphinxcode{\sphinxupquote{set\_exception()}} on
a cancelled \sphinxhref{https://docs.python.org/3.6/library/asyncio-task.html\#asyncio.Future}{\sphinxcode{\sphinxupquote{asyncio.Future}}}.

\DUrole{versionmodified,added}{New in version 6.0.}

\end{fulllineitems}

\index{future\_set\_exc\_info() (in module tornado.concurrent)@\spxentry{future\_set\_exc\_info()}\spxextra{in module tornado.concurrent}}

\begin{fulllineitems}
\phantomsection\label{\detokenize{concurrent:tornado.concurrent.future_set_exc_info}}\pysiglinewithargsret{\sphinxcode{\sphinxupquote{tornado.concurrent.}}\sphinxbfcode{\sphinxupquote{future\_set\_exc\_info}}}{\emph{future: Union{[}futures.Future{[}\_T{]}, Future{[}\_T{]}{]}, exc\_info: Tuple{[}Optional{[}type{]}, Optional{[}BaseException{]}, Optional{[}traceback{]}{]}}}{{ $\rightarrow$ None}}
Set the given \sphinxcode{\sphinxupquote{exc\_info}} as the {\hyperref[\detokenize{concurrent:tornado.concurrent.Future}]{\sphinxcrossref{\sphinxcode{\sphinxupquote{Future}}}}}’s exception.

Understands both \sphinxhref{https://docs.python.org/3.6/library/asyncio-task.html\#asyncio.Future}{\sphinxcode{\sphinxupquote{asyncio.Future}}} and the extensions in older
versions of Tornado to enable better tracebacks on Python 2.

\DUrole{versionmodified,added}{New in version 5.0.}

\DUrole{versionmodified,changed}{Changed in version 6.0: }If the future is already cancelled, this function is a no-op.
(previously \sphinxcode{\sphinxupquote{asyncio.InvalidStateError}} would be raised)

\end{fulllineitems}

\index{future\_add\_done\_callback() (in module tornado.concurrent)@\spxentry{future\_add\_done\_callback()}\spxextra{in module tornado.concurrent}}

\begin{fulllineitems}
\phantomsection\label{\detokenize{concurrent:tornado.concurrent.future_add_done_callback}}\pysiglinewithargsret{\sphinxcode{\sphinxupquote{tornado.concurrent.}}\sphinxbfcode{\sphinxupquote{future\_add\_done\_callback}}}{\emph{future: Union{[}futures.Future{[}\_T{]}, Future{[}\_T{]}{]}, callback: Callable{[}{[}...{]}, None{]}}}{{ $\rightarrow$ None}}
Arrange to call \sphinxcode{\sphinxupquote{callback}} when \sphinxcode{\sphinxupquote{future}} is complete.

\sphinxcode{\sphinxupquote{callback}} is invoked with one argument, the \sphinxcode{\sphinxupquote{future}}.

If \sphinxcode{\sphinxupquote{future}} is already done, \sphinxcode{\sphinxupquote{callback}} is invoked immediately.
This may differ from the behavior of \sphinxcode{\sphinxupquote{Future.add\_done\_callback}},
which makes no such guarantee.

\DUrole{versionmodified,added}{New in version 5.0.}

\end{fulllineitems}



\subsection{\sphinxstyleliteralintitle{\sphinxupquote{tornado.log}} — Logging support}
\label{\detokenize{log:module-tornado.log}}\label{\detokenize{log:tornado-log-logging-support}}\label{\detokenize{log::doc}}\index{tornado.log (module)@\spxentry{tornado.log}\spxextra{module}}
Logging support for Tornado.

Tornado uses three logger streams:
\begin{itemize}
\item {} 
\sphinxcode{\sphinxupquote{tornado.access}}: Per-request logging for Tornado’s HTTP servers (and
potentially other servers in the future)

\item {} 
\sphinxcode{\sphinxupquote{tornado.application}}: Logging of errors from application code (i.e.
uncaught exceptions from callbacks)

\item {} 
\sphinxcode{\sphinxupquote{tornado.general}}: General-purpose logging, including any errors
or warnings from Tornado itself.

\end{itemize}

These streams may be configured independently using the standard library’s
\sphinxhref{https://docs.python.org/3.6/library/logging.html\#module-logging}{\sphinxcode{\sphinxupquote{logging}}} module.  For example, you may wish to send \sphinxcode{\sphinxupquote{tornado.access}} logs
to a separate file for analysis.
\index{LogFormatter (class in tornado.log)@\spxentry{LogFormatter}\spxextra{class in tornado.log}}

\begin{fulllineitems}
\phantomsection\label{\detokenize{log:tornado.log.LogFormatter}}\pysiglinewithargsret{\sphinxbfcode{\sphinxupquote{class }}\sphinxcode{\sphinxupquote{tornado.log.}}\sphinxbfcode{\sphinxupquote{LogFormatter}}}{\emph{fmt: str = '\%(color)s{[}\%(levelname)1.1s \%(asctime)s \%(module)s:\%(lineno)d{]}\%(end\_color)s \%(message)s'}, \emph{datefmt: str = '\%y\%m\%d \%H:\%M:\%S'}, \emph{style: str = '\%'}, \emph{color: bool = True}, \emph{colors: Dict{[}int}, \emph{int{]} = \{10: 4}, \emph{20: 2}, \emph{30: 3}, \emph{40: 1\}}}{}
Log formatter used in Tornado.

Key features of this formatter are:
\begin{itemize}
\item {} 
Color support when logging to a terminal that supports it.

\item {} 
Timestamps on every log line.

\item {} 
Robust against str/bytes encoding problems.

\end{itemize}

This formatter is enabled automatically by
{\hyperref[\detokenize{options:tornado.options.parse_command_line}]{\sphinxcrossref{\sphinxcode{\sphinxupquote{tornado.options.parse\_command\_line}}}}} or {\hyperref[\detokenize{options:tornado.options.parse_config_file}]{\sphinxcrossref{\sphinxcode{\sphinxupquote{tornado.options.parse\_config\_file}}}}}
(unless \sphinxcode{\sphinxupquote{-{-}logging=none}} is used).

Color support on Windows versions that do not support ANSI color codes is
enabled by use of the \sphinxhref{https://pypi.python.org/pypi/colorama}{colorama} library. Applications that wish to use
this must first initialize colorama with a call to \sphinxcode{\sphinxupquote{colorama.init}}.
See the colorama documentation for details.

\DUrole{versionmodified,changed}{Changed in version 4.5: }Added support for \sphinxcode{\sphinxupquote{colorama}}. Changed the constructor
signature to be compatible with \sphinxhref{https://docs.python.org/3.6/library/logging.config.html\#logging.config.dictConfig}{\sphinxcode{\sphinxupquote{logging.config.dictConfig}}}.
\begin{quote}\begin{description}
\item[{Parameters}] \leavevmode\begin{itemize}
\item {} 
\sphinxstyleliteralstrong{\sphinxupquote{color}} (\sphinxhref{https://docs.python.org/3.6/library/functions.html\#bool}{\sphinxstyleliteralemphasis{\sphinxupquote{bool}}}) \textendash{} Enables color support.

\item {} 
\sphinxstyleliteralstrong{\sphinxupquote{fmt}} (\sphinxhref{https://docs.python.org/3.6/library/stdtypes.html\#str}{\sphinxstyleliteralemphasis{\sphinxupquote{str}}}) \textendash{} Log message format.
It will be applied to the attributes dict of log records. The
text between \sphinxcode{\sphinxupquote{\%(color)s}} and \sphinxcode{\sphinxupquote{\%(end\_color)s}} will be colored
depending on the level if color support is on.

\item {} 
\sphinxstyleliteralstrong{\sphinxupquote{colors}} (\sphinxhref{https://docs.python.org/3.6/library/stdtypes.html\#dict}{\sphinxstyleliteralemphasis{\sphinxupquote{dict}}}) \textendash{} color mappings from logging level to terminal color
code

\item {} 
\sphinxstyleliteralstrong{\sphinxupquote{datefmt}} (\sphinxhref{https://docs.python.org/3.6/library/stdtypes.html\#str}{\sphinxstyleliteralemphasis{\sphinxupquote{str}}}) \textendash{} Datetime format.
Used for formatting \sphinxcode{\sphinxupquote{(asctime)}} placeholder in \sphinxcode{\sphinxupquote{prefix\_fmt}}.

\end{itemize}

\end{description}\end{quote}

\DUrole{versionmodified,changed}{Changed in version 3.2: }Added \sphinxcode{\sphinxupquote{fmt}} and \sphinxcode{\sphinxupquote{datefmt}} arguments.

\end{fulllineitems}

\index{enable\_pretty\_logging() (in module tornado.log)@\spxentry{enable\_pretty\_logging()}\spxextra{in module tornado.log}}

\begin{fulllineitems}
\phantomsection\label{\detokenize{log:tornado.log.enable_pretty_logging}}\pysiglinewithargsret{\sphinxcode{\sphinxupquote{tornado.log.}}\sphinxbfcode{\sphinxupquote{enable\_pretty\_logging}}}{\emph{options: Any = None}, \emph{logger: logging.Logger = None}}{{ $\rightarrow$ None}}
Turns on formatted logging output as configured.

This is called automatically by {\hyperref[\detokenize{options:tornado.options.parse_command_line}]{\sphinxcrossref{\sphinxcode{\sphinxupquote{tornado.options.parse\_command\_line}}}}}
and {\hyperref[\detokenize{options:tornado.options.parse_config_file}]{\sphinxcrossref{\sphinxcode{\sphinxupquote{tornado.options.parse\_config\_file}}}}}.

\end{fulllineitems}

\index{define\_logging\_options() (in module tornado.log)@\spxentry{define\_logging\_options()}\spxextra{in module tornado.log}}

\begin{fulllineitems}
\phantomsection\label{\detokenize{log:tornado.log.define_logging_options}}\pysiglinewithargsret{\sphinxcode{\sphinxupquote{tornado.log.}}\sphinxbfcode{\sphinxupquote{define\_logging\_options}}}{\emph{options: Any = None}}{{ $\rightarrow$ None}}
Add logging-related flags to \sphinxcode{\sphinxupquote{options}}.

These options are present automatically on the default options instance;
this method is only necessary if you have created your own {\hyperref[\detokenize{options:tornado.options.OptionParser}]{\sphinxcrossref{\sphinxcode{\sphinxupquote{OptionParser}}}}}.

\DUrole{versionmodified,added}{New in version 4.2: }This function existed in prior versions but was broken and undocumented until 4.2.

\end{fulllineitems}



\subsection{\sphinxstyleliteralintitle{\sphinxupquote{tornado.options}} — Command-line parsing}
\label{\detokenize{options:module-tornado.options}}\label{\detokenize{options:tornado-options-command-line-parsing}}\label{\detokenize{options::doc}}\index{tornado.options (module)@\spxentry{tornado.options}\spxextra{module}}
A command line parsing module that lets modules define their own options.

This module is inspired by Google’s \sphinxhref{https://github.com/google/python-gflags}{gflags}. The primary difference
with libraries such as \sphinxhref{https://docs.python.org/3.6/library/argparse.html\#module-argparse}{\sphinxcode{\sphinxupquote{argparse}}} is that a global registry is used so
that options may be defined in any module (it also enables
{\hyperref[\detokenize{log:module-tornado.log}]{\sphinxcrossref{\sphinxcode{\sphinxupquote{tornado.log}}}}} by default). The rest of Tornado does not depend on this
module, so feel free to use \sphinxhref{https://docs.python.org/3.6/library/argparse.html\#module-argparse}{\sphinxcode{\sphinxupquote{argparse}}} or other configuration
libraries if you prefer them.

Options must be defined with {\hyperref[\detokenize{options:tornado.options.define}]{\sphinxcrossref{\sphinxcode{\sphinxupquote{tornado.options.define}}}}} before use,
generally at the top level of a module. The options are then
accessible as attributes of {\hyperref[\detokenize{options:tornado.options.options}]{\sphinxcrossref{\sphinxcode{\sphinxupquote{tornado.options.options}}}}}:

\begin{sphinxVerbatim}[commandchars=\\\{\}]
\PYG{c+c1}{\PYGZsh{} myapp/db.py}
\PYG{k+kn}{from} \PYG{n+nn}{tornado}\PYG{n+nn}{.}\PYG{n+nn}{options} \PYG{k}{import} \PYG{n}{define}\PYG{p}{,} \PYG{n}{options}

\PYG{n}{define}\PYG{p}{(}\PYG{l+s+s2}{\PYGZdq{}}\PYG{l+s+s2}{mysql\PYGZus{}host}\PYG{l+s+s2}{\PYGZdq{}}\PYG{p}{,} \PYG{n}{default}\PYG{o}{=}\PYG{l+s+s2}{\PYGZdq{}}\PYG{l+s+s2}{127.0.0.1:3306}\PYG{l+s+s2}{\PYGZdq{}}\PYG{p}{,} \PYG{n}{help}\PYG{o}{=}\PYG{l+s+s2}{\PYGZdq{}}\PYG{l+s+s2}{Main user DB}\PYG{l+s+s2}{\PYGZdq{}}\PYG{p}{)}
\PYG{n}{define}\PYG{p}{(}\PYG{l+s+s2}{\PYGZdq{}}\PYG{l+s+s2}{memcache\PYGZus{}hosts}\PYG{l+s+s2}{\PYGZdq{}}\PYG{p}{,} \PYG{n}{default}\PYG{o}{=}\PYG{l+s+s2}{\PYGZdq{}}\PYG{l+s+s2}{127.0.0.1:11011}\PYG{l+s+s2}{\PYGZdq{}}\PYG{p}{,} \PYG{n}{multiple}\PYG{o}{=}\PYG{k+kc}{True}\PYG{p}{,}
       \PYG{n}{help}\PYG{o}{=}\PYG{l+s+s2}{\PYGZdq{}}\PYG{l+s+s2}{Main user memcache servers}\PYG{l+s+s2}{\PYGZdq{}}\PYG{p}{)}

\PYG{k}{def} \PYG{n+nf}{connect}\PYG{p}{(}\PYG{p}{)}\PYG{p}{:}
    \PYG{n}{db} \PYG{o}{=} \PYG{n}{database}\PYG{o}{.}\PYG{n}{Connection}\PYG{p}{(}\PYG{n}{options}\PYG{o}{.}\PYG{n}{mysql\PYGZus{}host}\PYG{p}{)}
    \PYG{o}{.}\PYG{o}{.}\PYG{o}{.}

\PYG{c+c1}{\PYGZsh{} myapp/server.py}
\PYG{k+kn}{from} \PYG{n+nn}{tornado}\PYG{n+nn}{.}\PYG{n+nn}{options} \PYG{k}{import} \PYG{n}{define}\PYG{p}{,} \PYG{n}{options}

\PYG{n}{define}\PYG{p}{(}\PYG{l+s+s2}{\PYGZdq{}}\PYG{l+s+s2}{port}\PYG{l+s+s2}{\PYGZdq{}}\PYG{p}{,} \PYG{n}{default}\PYG{o}{=}\PYG{l+m+mi}{8080}\PYG{p}{,} \PYG{n}{help}\PYG{o}{=}\PYG{l+s+s2}{\PYGZdq{}}\PYG{l+s+s2}{port to listen on}\PYG{l+s+s2}{\PYGZdq{}}\PYG{p}{)}

\PYG{k}{def} \PYG{n+nf}{start\PYGZus{}server}\PYG{p}{(}\PYG{p}{)}\PYG{p}{:}
    \PYG{n}{app} \PYG{o}{=} \PYG{n}{make\PYGZus{}app}\PYG{p}{(}\PYG{p}{)}
    \PYG{n}{app}\PYG{o}{.}\PYG{n}{listen}\PYG{p}{(}\PYG{n}{options}\PYG{o}{.}\PYG{n}{port}\PYG{p}{)}
\end{sphinxVerbatim}

The \sphinxcode{\sphinxupquote{main()}} method of your application does not need to be aware of all of
the options used throughout your program; they are all automatically loaded
when the modules are loaded.  However, all modules that define options
must have been imported before the command line is parsed.

Your \sphinxcode{\sphinxupquote{main()}} method can parse the command line or parse a config file with
either {\hyperref[\detokenize{options:tornado.options.parse_command_line}]{\sphinxcrossref{\sphinxcode{\sphinxupquote{parse\_command\_line}}}}} or {\hyperref[\detokenize{options:tornado.options.parse_config_file}]{\sphinxcrossref{\sphinxcode{\sphinxupquote{parse\_config\_file}}}}}:

\begin{sphinxVerbatim}[commandchars=\\\{\}]
\PYG{k+kn}{import} \PYG{n+nn}{myapp}\PYG{n+nn}{.}\PYG{n+nn}{db}\PYG{o}{,} \PYG{n+nn}{myapp}\PYG{n+nn}{.}\PYG{n+nn}{server}
\PYG{k+kn}{import} \PYG{n+nn}{tornado}\PYG{n+nn}{.}\PYG{n+nn}{options}

\PYG{k}{if} \PYG{n+nv+vm}{\PYGZus{}\PYGZus{}name\PYGZus{}\PYGZus{}} \PYG{o}{==} \PYG{l+s+s1}{\PYGZsq{}}\PYG{l+s+s1}{\PYGZus{}\PYGZus{}main\PYGZus{}\PYGZus{}}\PYG{l+s+s1}{\PYGZsq{}}\PYG{p}{:}
    \PYG{n}{tornado}\PYG{o}{.}\PYG{n}{options}\PYG{o}{.}\PYG{n}{parse\PYGZus{}command\PYGZus{}line}\PYG{p}{(}\PYG{p}{)}
    \PYG{c+c1}{\PYGZsh{} or}
    \PYG{n}{tornado}\PYG{o}{.}\PYG{n}{options}\PYG{o}{.}\PYG{n}{parse\PYGZus{}config\PYGZus{}file}\PYG{p}{(}\PYG{l+s+s2}{\PYGZdq{}}\PYG{l+s+s2}{/etc/server.conf}\PYG{l+s+s2}{\PYGZdq{}}\PYG{p}{)}
\end{sphinxVerbatim}

\begin{sphinxadmonition}{note}{Note:}
When using multiple \sphinxcode{\sphinxupquote{parse\_*}} functions, pass \sphinxcode{\sphinxupquote{final=False}} to all
but the last one, or side effects may occur twice (in particular,
this can result in log messages being doubled).
\end{sphinxadmonition}

{\hyperref[\detokenize{options:tornado.options.options}]{\sphinxcrossref{\sphinxcode{\sphinxupquote{tornado.options.options}}}}} is a singleton instance of {\hyperref[\detokenize{options:tornado.options.OptionParser}]{\sphinxcrossref{\sphinxcode{\sphinxupquote{OptionParser}}}}}, and
the top-level functions in this module ({\hyperref[\detokenize{options:tornado.options.define}]{\sphinxcrossref{\sphinxcode{\sphinxupquote{define}}}}}, {\hyperref[\detokenize{options:tornado.options.parse_command_line}]{\sphinxcrossref{\sphinxcode{\sphinxupquote{parse\_command\_line}}}}}, etc)
simply call methods on it.  You may create additional {\hyperref[\detokenize{options:tornado.options.OptionParser}]{\sphinxcrossref{\sphinxcode{\sphinxupquote{OptionParser}}}}}
instances to define isolated sets of options, such as for subcommands.

\begin{sphinxadmonition}{note}{Note:}
By default, several options are defined that will configure the
standard \sphinxhref{https://docs.python.org/3.6/library/logging.html\#module-logging}{\sphinxcode{\sphinxupquote{logging}}} module when {\hyperref[\detokenize{options:tornado.options.parse_command_line}]{\sphinxcrossref{\sphinxcode{\sphinxupquote{parse\_command\_line}}}}} or {\hyperref[\detokenize{options:tornado.options.parse_config_file}]{\sphinxcrossref{\sphinxcode{\sphinxupquote{parse\_config\_file}}}}}
are called.  If you want Tornado to leave the logging configuration
alone so you can manage it yourself, either pass \sphinxcode{\sphinxupquote{-{-}logging=none}}
on the command line or do the following to disable it in code:

\begin{sphinxVerbatim}[commandchars=\\\{\}]
\PYG{k+kn}{from} \PYG{n+nn}{tornado}\PYG{n+nn}{.}\PYG{n+nn}{options} \PYG{k}{import} \PYG{n}{options}\PYG{p}{,} \PYG{n}{parse\PYGZus{}command\PYGZus{}line}
\PYG{n}{options}\PYG{o}{.}\PYG{n}{logging} \PYG{o}{=} \PYG{k+kc}{None}
\PYG{n}{parse\PYGZus{}command\PYGZus{}line}\PYG{p}{(}\PYG{p}{)}
\end{sphinxVerbatim}
\end{sphinxadmonition}

\DUrole{versionmodified,changed}{Changed in version 4.3: }Dashes and underscores are fully interchangeable in option names;
options can be defined, set, and read with any mix of the two.
Dashes are typical for command-line usage while config files require
underscores.


\subsubsection{Global functions}
\label{\detokenize{options:global-functions}}\index{define() (in module tornado.options)@\spxentry{define()}\spxextra{in module tornado.options}}

\begin{fulllineitems}
\phantomsection\label{\detokenize{options:tornado.options.define}}\pysiglinewithargsret{\sphinxcode{\sphinxupquote{tornado.options.}}\sphinxbfcode{\sphinxupquote{define}}}{\emph{name: str, default: Any = None, type: type = None, help: str = None, metavar: str = None, multiple: bool = False, group: str = None, callback: Callable{[}{[}Any{]}, None{]} = None}}{{ $\rightarrow$ None}}
Defines an option in the global namespace.

See {\hyperref[\detokenize{options:tornado.options.OptionParser.define}]{\sphinxcrossref{\sphinxcode{\sphinxupquote{OptionParser.define}}}}}.

\end{fulllineitems}

\index{options (in module tornado.options)@\spxentry{options}\spxextra{in module tornado.options}}

\begin{fulllineitems}
\phantomsection\label{\detokenize{options:tornado.options.options}}\pysigline{\sphinxcode{\sphinxupquote{tornado.options.}}\sphinxbfcode{\sphinxupquote{options}}}
Global options object.  All defined options are available as attributes
on this object.

\end{fulllineitems}

\index{parse\_command\_line() (in module tornado.options)@\spxentry{parse\_command\_line()}\spxextra{in module tornado.options}}

\begin{fulllineitems}
\phantomsection\label{\detokenize{options:tornado.options.parse_command_line}}\pysiglinewithargsret{\sphinxcode{\sphinxupquote{tornado.options.}}\sphinxbfcode{\sphinxupquote{parse\_command\_line}}}{\emph{args: List{[}str{]} = None}, \emph{final: bool = True}}{{ $\rightarrow$ List{[}str{]}}}
Parses global options from the command line.

See {\hyperref[\detokenize{options:tornado.options.OptionParser.parse_command_line}]{\sphinxcrossref{\sphinxcode{\sphinxupquote{OptionParser.parse\_command\_line}}}}}.

\end{fulllineitems}

\index{parse\_config\_file() (in module tornado.options)@\spxentry{parse\_config\_file()}\spxextra{in module tornado.options}}

\begin{fulllineitems}
\phantomsection\label{\detokenize{options:tornado.options.parse_config_file}}\pysiglinewithargsret{\sphinxcode{\sphinxupquote{tornado.options.}}\sphinxbfcode{\sphinxupquote{parse\_config\_file}}}{\emph{path: str}, \emph{final: bool = True}}{{ $\rightarrow$ None}}
Parses global options from a config file.

See {\hyperref[\detokenize{options:tornado.options.OptionParser.parse_config_file}]{\sphinxcrossref{\sphinxcode{\sphinxupquote{OptionParser.parse\_config\_file}}}}}.

\end{fulllineitems}

\index{print\_help() (in module tornado.options)@\spxentry{print\_help()}\spxextra{in module tornado.options}}

\begin{fulllineitems}
\phantomsection\label{\detokenize{options:tornado.options.print_help}}\pysiglinewithargsret{\sphinxcode{\sphinxupquote{tornado.options.}}\sphinxbfcode{\sphinxupquote{print\_help}}}{\emph{file=sys.stderr}}{}
Prints all the command line options to stderr (or another file).

See {\hyperref[\detokenize{options:tornado.options.OptionParser.print_help}]{\sphinxcrossref{\sphinxcode{\sphinxupquote{OptionParser.print\_help}}}}}.

\end{fulllineitems}

\index{add\_parse\_callback() (in module tornado.options)@\spxentry{add\_parse\_callback()}\spxextra{in module tornado.options}}

\begin{fulllineitems}
\phantomsection\label{\detokenize{options:tornado.options.add_parse_callback}}\pysiglinewithargsret{\sphinxcode{\sphinxupquote{tornado.options.}}\sphinxbfcode{\sphinxupquote{add\_parse\_callback}}}{\emph{callback: Callable{[}{[}{]}, None{]}}}{{ $\rightarrow$ None}}
Adds a parse callback, to be invoked when option parsing is done.

See {\hyperref[\detokenize{options:tornado.options.OptionParser.add_parse_callback}]{\sphinxcrossref{\sphinxcode{\sphinxupquote{OptionParser.add\_parse\_callback}}}}}

\end{fulllineitems}

\index{Error@\spxentry{Error}}

\begin{fulllineitems}
\phantomsection\label{\detokenize{options:tornado.options.Error}}\pysigline{\sphinxbfcode{\sphinxupquote{exception }}\sphinxcode{\sphinxupquote{tornado.options.}}\sphinxbfcode{\sphinxupquote{Error}}}
Exception raised by errors in the options module.

\end{fulllineitems}



\subsubsection{OptionParser class}
\label{\detokenize{options:optionparser-class}}\index{OptionParser (class in tornado.options)@\spxentry{OptionParser}\spxextra{class in tornado.options}}

\begin{fulllineitems}
\phantomsection\label{\detokenize{options:tornado.options.OptionParser}}\pysigline{\sphinxbfcode{\sphinxupquote{class }}\sphinxcode{\sphinxupquote{tornado.options.}}\sphinxbfcode{\sphinxupquote{OptionParser}}}
A collection of options, a dictionary with object-like access.

Normally accessed via static functions in the {\hyperref[\detokenize{options:module-tornado.options}]{\sphinxcrossref{\sphinxcode{\sphinxupquote{tornado.options}}}}} module,
which reference a global instance.

\end{fulllineitems}

\index{define() (tornado.options.OptionParser method)@\spxentry{define()}\spxextra{tornado.options.OptionParser method}}

\begin{fulllineitems}
\phantomsection\label{\detokenize{options:tornado.options.OptionParser.define}}\pysiglinewithargsret{\sphinxcode{\sphinxupquote{OptionParser.}}\sphinxbfcode{\sphinxupquote{define}}}{\emph{name: str, default: Any = None, type: type = None, help: str = None, metavar: str = None, multiple: bool = False, group: str = None, callback: Callable{[}{[}Any{]}, None{]} = None}}{{ $\rightarrow$ None}}
Defines a new command line option.

\sphinxcode{\sphinxupquote{type}} can be any of \sphinxhref{https://docs.python.org/3.6/library/stdtypes.html\#str}{\sphinxcode{\sphinxupquote{str}}}, \sphinxhref{https://docs.python.org/3.6/library/functions.html\#int}{\sphinxcode{\sphinxupquote{int}}}, \sphinxhref{https://docs.python.org/3.6/library/functions.html\#float}{\sphinxcode{\sphinxupquote{float}}}, \sphinxhref{https://docs.python.org/3.6/library/functions.html\#bool}{\sphinxcode{\sphinxupquote{bool}}},
\sphinxhref{https://docs.python.org/3.6/library/datetime.html\#datetime.datetime}{\sphinxcode{\sphinxupquote{datetime}}}, or \sphinxhref{https://docs.python.org/3.6/library/datetime.html\#datetime.timedelta}{\sphinxcode{\sphinxupquote{timedelta}}}. If no \sphinxcode{\sphinxupquote{type}}
is given but a \sphinxcode{\sphinxupquote{default}} is, \sphinxcode{\sphinxupquote{type}} is the type of
\sphinxcode{\sphinxupquote{default}}. Otherwise, \sphinxcode{\sphinxupquote{type}} defaults to \sphinxhref{https://docs.python.org/3.6/library/stdtypes.html\#str}{\sphinxcode{\sphinxupquote{str}}}.

If \sphinxcode{\sphinxupquote{multiple}} is True, the option value is a list of \sphinxcode{\sphinxupquote{type}}
instead of an instance of \sphinxcode{\sphinxupquote{type}}.

\sphinxcode{\sphinxupquote{help}} and \sphinxcode{\sphinxupquote{metavar}} are used to construct the
automatically generated command line help string. The help
message is formatted like:

\begin{sphinxVerbatim}[commandchars=\\\{\}]
\PYG{o}{\PYGZhy{}}\PYG{o}{\PYGZhy{}}\PYG{n}{name}\PYG{o}{=}\PYG{n}{METAVAR}      \PYG{n}{help} \PYG{n}{string}
\end{sphinxVerbatim}

\sphinxcode{\sphinxupquote{group}} is used to group the defined options in logical
groups. By default, command line options are grouped by the
file in which they are defined.

Command line option names must be unique globally.

If a \sphinxcode{\sphinxupquote{callback}} is given, it will be run with the new value whenever
the option is changed.  This can be used to combine command-line
and file-based options:

\begin{sphinxVerbatim}[commandchars=\\\{\}]
\PYG{n}{define}\PYG{p}{(}\PYG{l+s+s2}{\PYGZdq{}}\PYG{l+s+s2}{config}\PYG{l+s+s2}{\PYGZdq{}}\PYG{p}{,} \PYG{n+nb}{type}\PYG{o}{=}\PYG{n+nb}{str}\PYG{p}{,} \PYG{n}{help}\PYG{o}{=}\PYG{l+s+s2}{\PYGZdq{}}\PYG{l+s+s2}{path to config file}\PYG{l+s+s2}{\PYGZdq{}}\PYG{p}{,}
       \PYG{n}{callback}\PYG{o}{=}\PYG{k}{lambda} \PYG{n}{path}\PYG{p}{:} \PYG{n}{parse\PYGZus{}config\PYGZus{}file}\PYG{p}{(}\PYG{n}{path}\PYG{p}{,} \PYG{n}{final}\PYG{o}{=}\PYG{k+kc}{False}\PYG{p}{)}\PYG{p}{)}
\end{sphinxVerbatim}

With this definition, options in the file specified by \sphinxcode{\sphinxupquote{-{-}config}} will
override options set earlier on the command line, but can be overridden
by later flags.

\end{fulllineitems}

\index{parse\_command\_line() (tornado.options.OptionParser method)@\spxentry{parse\_command\_line()}\spxextra{tornado.options.OptionParser method}}

\begin{fulllineitems}
\phantomsection\label{\detokenize{options:tornado.options.OptionParser.parse_command_line}}\pysiglinewithargsret{\sphinxcode{\sphinxupquote{OptionParser.}}\sphinxbfcode{\sphinxupquote{parse\_command\_line}}}{\emph{args: List{[}str{]} = None}, \emph{final: bool = True}}{{ $\rightarrow$ List{[}str{]}}}
Parses all options given on the command line (defaults to
\sphinxhref{https://docs.python.org/3.6/library/sys.html\#sys.argv}{\sphinxcode{\sphinxupquote{sys.argv}}}).

Options look like \sphinxcode{\sphinxupquote{-{-}option=value}} and are parsed according
to their \sphinxcode{\sphinxupquote{type}}. For boolean options, \sphinxcode{\sphinxupquote{-{-}option}} is
equivalent to \sphinxcode{\sphinxupquote{-{-}option=true}}

If the option has \sphinxcode{\sphinxupquote{multiple=True}}, comma-separated values
are accepted. For multi-value integer options, the syntax
\sphinxcode{\sphinxupquote{x:y}} is also accepted and equivalent to \sphinxcode{\sphinxupquote{range(x, y)}}.

Note that \sphinxcode{\sphinxupquote{args{[}0{]}}} is ignored since it is the program name
in \sphinxhref{https://docs.python.org/3.6/library/sys.html\#sys.argv}{\sphinxcode{\sphinxupquote{sys.argv}}}.

We return a list of all arguments that are not parsed as options.

If \sphinxcode{\sphinxupquote{final}} is \sphinxcode{\sphinxupquote{False}}, parse callbacks will not be run.
This is useful for applications that wish to combine configurations
from multiple sources.

\end{fulllineitems}

\index{parse\_config\_file() (tornado.options.OptionParser method)@\spxentry{parse\_config\_file()}\spxextra{tornado.options.OptionParser method}}

\begin{fulllineitems}
\phantomsection\label{\detokenize{options:tornado.options.OptionParser.parse_config_file}}\pysiglinewithargsret{\sphinxcode{\sphinxupquote{OptionParser.}}\sphinxbfcode{\sphinxupquote{parse\_config\_file}}}{\emph{path: str}, \emph{final: bool = True}}{{ $\rightarrow$ None}}
Parses and loads the config file at the given path.

The config file contains Python code that will be executed (so
it is \sphinxstylestrong{not safe} to use untrusted config files). Anything in
the global namespace that matches a defined option will be
used to set that option’s value.

Options may either be the specified type for the option or
strings (in which case they will be parsed the same way as in
{\hyperref[\detokenize{options:tornado.options.OptionParser.parse_command_line}]{\sphinxcrossref{\sphinxcode{\sphinxupquote{parse\_command\_line}}}}})

Example (using the options defined in the top-level docs of
this module):

\begin{sphinxVerbatim}[commandchars=\\\{\}]
\PYG{n}{port} \PYG{o}{=} \PYG{l+m+mi}{80}
\PYG{n}{mysql\PYGZus{}host} \PYG{o}{=} \PYG{l+s+s1}{\PYGZsq{}}\PYG{l+s+s1}{mydb.example.com:3306}\PYG{l+s+s1}{\PYGZsq{}}
\PYG{c+c1}{\PYGZsh{} Both lists and comma\PYGZhy{}separated strings are allowed for}
\PYG{c+c1}{\PYGZsh{} multiple=True.}
\PYG{n}{memcache\PYGZus{}hosts} \PYG{o}{=} \PYG{p}{[}\PYG{l+s+s1}{\PYGZsq{}}\PYG{l+s+s1}{cache1.example.com:11011}\PYG{l+s+s1}{\PYGZsq{}}\PYG{p}{,}
                  \PYG{l+s+s1}{\PYGZsq{}}\PYG{l+s+s1}{cache2.example.com:11011}\PYG{l+s+s1}{\PYGZsq{}}\PYG{p}{]}
\PYG{n}{memcache\PYGZus{}hosts} \PYG{o}{=} \PYG{l+s+s1}{\PYGZsq{}}\PYG{l+s+s1}{cache1.example.com:11011,cache2.example.com:11011}\PYG{l+s+s1}{\PYGZsq{}}
\end{sphinxVerbatim}

If \sphinxcode{\sphinxupquote{final}} is \sphinxcode{\sphinxupquote{False}}, parse callbacks will not be run.
This is useful for applications that wish to combine configurations
from multiple sources.

\begin{sphinxadmonition}{note}{Note:}
{\hyperref[\detokenize{options:module-tornado.options}]{\sphinxcrossref{\sphinxcode{\sphinxupquote{tornado.options}}}}} is primarily a command-line library.
Config file support is provided for applications that wish
to use it, but applications that prefer config files may
wish to look at other libraries instead.
\end{sphinxadmonition}

\DUrole{versionmodified,changed}{Changed in version 4.1: }Config files are now always interpreted as utf-8 instead of
the system default encoding.

\DUrole{versionmodified,changed}{Changed in version 4.4: }The special variable \sphinxcode{\sphinxupquote{\_\_file\_\_}} is available inside config
files, specifying the absolute path to the config file itself.

\DUrole{versionmodified,changed}{Changed in version 5.1: }Added the ability to set options via strings in config files.

\end{fulllineitems}

\index{print\_help() (tornado.options.OptionParser method)@\spxentry{print\_help()}\spxextra{tornado.options.OptionParser method}}

\begin{fulllineitems}
\phantomsection\label{\detokenize{options:tornado.options.OptionParser.print_help}}\pysiglinewithargsret{\sphinxcode{\sphinxupquote{OptionParser.}}\sphinxbfcode{\sphinxupquote{print\_help}}}{\emph{file: TextIO = None}}{{ $\rightarrow$ None}}
Prints all the command line options to stderr (or another file).

\end{fulllineitems}

\index{add\_parse\_callback() (tornado.options.OptionParser method)@\spxentry{add\_parse\_callback()}\spxextra{tornado.options.OptionParser method}}

\begin{fulllineitems}
\phantomsection\label{\detokenize{options:tornado.options.OptionParser.add_parse_callback}}\pysiglinewithargsret{\sphinxcode{\sphinxupquote{OptionParser.}}\sphinxbfcode{\sphinxupquote{add\_parse\_callback}}}{\emph{callback: Callable{[}{[}{]}, None{]}}}{{ $\rightarrow$ None}}
Adds a parse callback, to be invoked when option parsing is done.

\end{fulllineitems}

\index{mockable() (tornado.options.OptionParser method)@\spxentry{mockable()}\spxextra{tornado.options.OptionParser method}}

\begin{fulllineitems}
\phantomsection\label{\detokenize{options:tornado.options.OptionParser.mockable}}\pysiglinewithargsret{\sphinxcode{\sphinxupquote{OptionParser.}}\sphinxbfcode{\sphinxupquote{mockable}}}{}{{ $\rightarrow$ tornado.options.\_Mockable}}
Returns a wrapper around self that is compatible with
\sphinxhref{https://docs.python.org/3.6/library/unittest.mock.html\#unittest.mock.patch}{\sphinxcode{\sphinxupquote{mock.patch}}}.

The \sphinxhref{https://docs.python.org/3.6/library/unittest.mock.html\#unittest.mock.patch}{\sphinxcode{\sphinxupquote{mock.patch}}} function (included in
the standard library \sphinxhref{https://docs.python.org/3.6/library/unittest.mock.html\#module-unittest.mock}{\sphinxcode{\sphinxupquote{unittest.mock}}} package since Python 3.3,
or in the third-party \sphinxcode{\sphinxupquote{mock}} package for older versions of
Python) is incompatible with objects like \sphinxcode{\sphinxupquote{options}} that
override \sphinxcode{\sphinxupquote{\_\_getattr\_\_}} and \sphinxcode{\sphinxupquote{\_\_setattr\_\_}}.  This function
returns an object that can be used with \sphinxhref{https://docs.python.org/3.6/library/unittest.mock.html\#unittest.mock.patch.object}{\sphinxcode{\sphinxupquote{mock.patch.object}}} to modify option values:

\begin{sphinxVerbatim}[commandchars=\\\{\}]
\PYG{k}{with} \PYG{n}{mock}\PYG{o}{.}\PYG{n}{patch}\PYG{o}{.}\PYG{n}{object}\PYG{p}{(}\PYG{n}{options}\PYG{o}{.}\PYG{n}{mockable}\PYG{p}{(}\PYG{p}{)}\PYG{p}{,} \PYG{l+s+s1}{\PYGZsq{}}\PYG{l+s+s1}{name}\PYG{l+s+s1}{\PYGZsq{}}\PYG{p}{,} \PYG{n}{value}\PYG{p}{)}\PYG{p}{:}
    \PYG{k}{assert} \PYG{n}{options}\PYG{o}{.}\PYG{n}{name} \PYG{o}{==} \PYG{n}{value}
\end{sphinxVerbatim}

\end{fulllineitems}

\index{items() (tornado.options.OptionParser method)@\spxentry{items()}\spxextra{tornado.options.OptionParser method}}

\begin{fulllineitems}
\phantomsection\label{\detokenize{options:tornado.options.OptionParser.items}}\pysiglinewithargsret{\sphinxcode{\sphinxupquote{OptionParser.}}\sphinxbfcode{\sphinxupquote{items}}}{}{{ $\rightarrow$ Iterable{[}Tuple{[}str, Any{]}{]}}}
An iterable of (name, value) pairs.

\DUrole{versionmodified,added}{New in version 3.1.}

\end{fulllineitems}

\index{as\_dict() (tornado.options.OptionParser method)@\spxentry{as\_dict()}\spxextra{tornado.options.OptionParser method}}

\begin{fulllineitems}
\phantomsection\label{\detokenize{options:tornado.options.OptionParser.as_dict}}\pysiglinewithargsret{\sphinxcode{\sphinxupquote{OptionParser.}}\sphinxbfcode{\sphinxupquote{as\_dict}}}{}{{ $\rightarrow$ Dict{[}str, Any{]}}}
The names and values of all options.

\DUrole{versionmodified,added}{New in version 3.1.}

\end{fulllineitems}

\index{groups() (tornado.options.OptionParser method)@\spxentry{groups()}\spxextra{tornado.options.OptionParser method}}

\begin{fulllineitems}
\phantomsection\label{\detokenize{options:tornado.options.OptionParser.groups}}\pysiglinewithargsret{\sphinxcode{\sphinxupquote{OptionParser.}}\sphinxbfcode{\sphinxupquote{groups}}}{}{{ $\rightarrow$ Set{[}str{]}}}
The set of option-groups created by \sphinxcode{\sphinxupquote{define}}.

\DUrole{versionmodified,added}{New in version 3.1.}

\end{fulllineitems}

\index{group\_dict() (tornado.options.OptionParser method)@\spxentry{group\_dict()}\spxextra{tornado.options.OptionParser method}}

\begin{fulllineitems}
\phantomsection\label{\detokenize{options:tornado.options.OptionParser.group_dict}}\pysiglinewithargsret{\sphinxcode{\sphinxupquote{OptionParser.}}\sphinxbfcode{\sphinxupquote{group\_dict}}}{\emph{group: str}}{{ $\rightarrow$ Dict{[}str, Any{]}}}
The names and values of options in a group.

Useful for copying options into Application settings:

\begin{sphinxVerbatim}[commandchars=\\\{\}]
\PYG{k+kn}{from} \PYG{n+nn}{tornado}\PYG{n+nn}{.}\PYG{n+nn}{options} \PYG{k}{import} \PYG{n}{define}\PYG{p}{,} \PYG{n}{parse\PYGZus{}command\PYGZus{}line}\PYG{p}{,} \PYG{n}{options}

\PYG{n}{define}\PYG{p}{(}\PYG{l+s+s1}{\PYGZsq{}}\PYG{l+s+s1}{template\PYGZus{}path}\PYG{l+s+s1}{\PYGZsq{}}\PYG{p}{,} \PYG{n}{group}\PYG{o}{=}\PYG{l+s+s1}{\PYGZsq{}}\PYG{l+s+s1}{application}\PYG{l+s+s1}{\PYGZsq{}}\PYG{p}{)}
\PYG{n}{define}\PYG{p}{(}\PYG{l+s+s1}{\PYGZsq{}}\PYG{l+s+s1}{static\PYGZus{}path}\PYG{l+s+s1}{\PYGZsq{}}\PYG{p}{,} \PYG{n}{group}\PYG{o}{=}\PYG{l+s+s1}{\PYGZsq{}}\PYG{l+s+s1}{application}\PYG{l+s+s1}{\PYGZsq{}}\PYG{p}{)}

\PYG{n}{parse\PYGZus{}command\PYGZus{}line}\PYG{p}{(}\PYG{p}{)}

\PYG{n}{application} \PYG{o}{=} \PYG{n}{Application}\PYG{p}{(}
    \PYG{n}{handlers}\PYG{p}{,} \PYG{o}{*}\PYG{o}{*}\PYG{n}{options}\PYG{o}{.}\PYG{n}{group\PYGZus{}dict}\PYG{p}{(}\PYG{l+s+s1}{\PYGZsq{}}\PYG{l+s+s1}{application}\PYG{l+s+s1}{\PYGZsq{}}\PYG{p}{)}\PYG{p}{)}
\end{sphinxVerbatim}

\DUrole{versionmodified,added}{New in version 3.1.}

\end{fulllineitems}



\subsection{\sphinxstyleliteralintitle{\sphinxupquote{tornado.testing}} — Unit testing support for asynchronous code}
\label{\detokenize{testing:module-tornado.testing}}\label{\detokenize{testing:tornado-testing-unit-testing-support-for-asynchronous-code}}\label{\detokenize{testing::doc}}\index{tornado.testing (module)@\spxentry{tornado.testing}\spxextra{module}}
Support classes for automated testing.
\begin{itemize}
\item {} 
{\hyperref[\detokenize{testing:tornado.testing.AsyncTestCase}]{\sphinxcrossref{\sphinxcode{\sphinxupquote{AsyncTestCase}}}}} and {\hyperref[\detokenize{testing:tornado.testing.AsyncHTTPTestCase}]{\sphinxcrossref{\sphinxcode{\sphinxupquote{AsyncHTTPTestCase}}}}}:  Subclasses of unittest.TestCase
with additional support for testing asynchronous ({\hyperref[\detokenize{ioloop:tornado.ioloop.IOLoop}]{\sphinxcrossref{\sphinxcode{\sphinxupquote{IOLoop}}}}}-based) code.

\item {} 
{\hyperref[\detokenize{testing:tornado.testing.ExpectLog}]{\sphinxcrossref{\sphinxcode{\sphinxupquote{ExpectLog}}}}}: Make test logs less spammy.

\item {} 
{\hyperref[\detokenize{testing:tornado.testing.main}]{\sphinxcrossref{\sphinxcode{\sphinxupquote{main()}}}}}: A simple test runner (wrapper around unittest.main()) with support
for the tornado.autoreload module to rerun the tests when code changes.

\end{itemize}


\subsubsection{Asynchronous test cases}
\label{\detokenize{testing:asynchronous-test-cases}}\index{AsyncTestCase (class in tornado.testing)@\spxentry{AsyncTestCase}\spxextra{class in tornado.testing}}

\begin{fulllineitems}
\phantomsection\label{\detokenize{testing:tornado.testing.AsyncTestCase}}\pysiglinewithargsret{\sphinxbfcode{\sphinxupquote{class }}\sphinxcode{\sphinxupquote{tornado.testing.}}\sphinxbfcode{\sphinxupquote{AsyncTestCase}}}{\emph{methodName: str = 'runTest'}}{}
\sphinxhref{https://docs.python.org/3.6/library/unittest.html\#unittest.TestCase}{\sphinxcode{\sphinxupquote{TestCase}}} subclass for testing {\hyperref[\detokenize{ioloop:tornado.ioloop.IOLoop}]{\sphinxcrossref{\sphinxcode{\sphinxupquote{IOLoop}}}}}-based
asynchronous code.

The unittest framework is synchronous, so the test must be
complete by the time the test method returns. This means that
asynchronous code cannot be used in quite the same way as usual
and must be adapted to fit. To write your tests with coroutines,
decorate your test methods with {\hyperref[\detokenize{testing:tornado.testing.gen_test}]{\sphinxcrossref{\sphinxcode{\sphinxupquote{tornado.testing.gen\_test}}}}} instead
of {\hyperref[\detokenize{gen:tornado.gen.coroutine}]{\sphinxcrossref{\sphinxcode{\sphinxupquote{tornado.gen.coroutine}}}}}.

This class also provides the (deprecated) {\hyperref[\detokenize{testing:tornado.testing.AsyncTestCase.stop}]{\sphinxcrossref{\sphinxcode{\sphinxupquote{stop()}}}}} and {\hyperref[\detokenize{testing:tornado.testing.AsyncTestCase.wait}]{\sphinxcrossref{\sphinxcode{\sphinxupquote{wait()}}}}}
methods for a more manual style of testing. The test method itself
must call \sphinxcode{\sphinxupquote{self.wait()}}, and asynchronous callbacks should call
\sphinxcode{\sphinxupquote{self.stop()}} to signal completion.

By default, a new {\hyperref[\detokenize{ioloop:tornado.ioloop.IOLoop}]{\sphinxcrossref{\sphinxcode{\sphinxupquote{IOLoop}}}}} is constructed for each test and is available
as \sphinxcode{\sphinxupquote{self.io\_loop}}.  If the code being tested requires a
global {\hyperref[\detokenize{ioloop:tornado.ioloop.IOLoop}]{\sphinxcrossref{\sphinxcode{\sphinxupquote{IOLoop}}}}}, subclasses should override {\hyperref[\detokenize{testing:tornado.testing.AsyncTestCase.get_new_ioloop}]{\sphinxcrossref{\sphinxcode{\sphinxupquote{get\_new\_ioloop}}}}} to return it.

The {\hyperref[\detokenize{ioloop:tornado.ioloop.IOLoop}]{\sphinxcrossref{\sphinxcode{\sphinxupquote{IOLoop}}}}}’s \sphinxcode{\sphinxupquote{start}} and \sphinxcode{\sphinxupquote{stop}} methods should not be
called directly.  Instead, use {\hyperref[\detokenize{testing:tornado.testing.AsyncTestCase.stop}]{\sphinxcrossref{\sphinxcode{\sphinxupquote{self.stop}}}}} and {\hyperref[\detokenize{testing:tornado.testing.AsyncTestCase.wait}]{\sphinxcrossref{\sphinxcode{\sphinxupquote{self.wait}}}}}.  Arguments passed to \sphinxcode{\sphinxupquote{self.stop}} are returned from
\sphinxcode{\sphinxupquote{self.wait}}.  It is possible to have multiple \sphinxcode{\sphinxupquote{wait}}/\sphinxcode{\sphinxupquote{stop}}
cycles in the same test.

Example:

\begin{sphinxVerbatim}[commandchars=\\\{\}]
\PYG{c+c1}{\PYGZsh{} This test uses coroutine style.}
\PYG{k}{class} \PYG{n+nc}{MyTestCase}\PYG{p}{(}\PYG{n}{AsyncTestCase}\PYG{p}{)}\PYG{p}{:}
    \PYG{n+nd}{@tornado}\PYG{o}{.}\PYG{n}{testing}\PYG{o}{.}\PYG{n}{gen\PYGZus{}test}
    \PYG{k}{def} \PYG{n+nf}{test\PYGZus{}http\PYGZus{}fetch}\PYG{p}{(}\PYG{n+nb+bp}{self}\PYG{p}{)}\PYG{p}{:}
        \PYG{n}{client} \PYG{o}{=} \PYG{n}{AsyncHTTPClient}\PYG{p}{(}\PYG{p}{)}
        \PYG{n}{response} \PYG{o}{=} \PYG{k}{yield} \PYG{n}{client}\PYG{o}{.}\PYG{n}{fetch}\PYG{p}{(}\PYG{l+s+s2}{\PYGZdq{}}\PYG{l+s+s2}{http://www.tornadoweb.org}\PYG{l+s+s2}{\PYGZdq{}}\PYG{p}{)}
        \PYG{c+c1}{\PYGZsh{} Test contents of response}
        \PYG{n+nb+bp}{self}\PYG{o}{.}\PYG{n}{assertIn}\PYG{p}{(}\PYG{l+s+s2}{\PYGZdq{}}\PYG{l+s+s2}{FriendFeed}\PYG{l+s+s2}{\PYGZdq{}}\PYG{p}{,} \PYG{n}{response}\PYG{o}{.}\PYG{n}{body}\PYG{p}{)}

\PYG{c+c1}{\PYGZsh{} This test uses argument passing between self.stop and self.wait.}
\PYG{k}{class} \PYG{n+nc}{MyTestCase2}\PYG{p}{(}\PYG{n}{AsyncTestCase}\PYG{p}{)}\PYG{p}{:}
    \PYG{k}{def} \PYG{n+nf}{test\PYGZus{}http\PYGZus{}fetch}\PYG{p}{(}\PYG{n+nb+bp}{self}\PYG{p}{)}\PYG{p}{:}
        \PYG{n}{client} \PYG{o}{=} \PYG{n}{AsyncHTTPClient}\PYG{p}{(}\PYG{p}{)}
        \PYG{n}{client}\PYG{o}{.}\PYG{n}{fetch}\PYG{p}{(}\PYG{l+s+s2}{\PYGZdq{}}\PYG{l+s+s2}{http://www.tornadoweb.org/}\PYG{l+s+s2}{\PYGZdq{}}\PYG{p}{,} \PYG{n+nb+bp}{self}\PYG{o}{.}\PYG{n}{stop}\PYG{p}{)}
        \PYG{n}{response} \PYG{o}{=} \PYG{n+nb+bp}{self}\PYG{o}{.}\PYG{n}{wait}\PYG{p}{(}\PYG{p}{)}
        \PYG{c+c1}{\PYGZsh{} Test contents of response}
        \PYG{n+nb+bp}{self}\PYG{o}{.}\PYG{n}{assertIn}\PYG{p}{(}\PYG{l+s+s2}{\PYGZdq{}}\PYG{l+s+s2}{FriendFeed}\PYG{l+s+s2}{\PYGZdq{}}\PYG{p}{,} \PYG{n}{response}\PYG{o}{.}\PYG{n}{body}\PYG{p}{)}
\end{sphinxVerbatim}
\index{get\_new\_ioloop() (tornado.testing.AsyncTestCase method)@\spxentry{get\_new\_ioloop()}\spxextra{tornado.testing.AsyncTestCase method}}

\begin{fulllineitems}
\phantomsection\label{\detokenize{testing:tornado.testing.AsyncTestCase.get_new_ioloop}}\pysiglinewithargsret{\sphinxbfcode{\sphinxupquote{get\_new\_ioloop}}}{}{{ $\rightarrow$ tornado.ioloop.IOLoop}}
Returns the {\hyperref[\detokenize{ioloop:tornado.ioloop.IOLoop}]{\sphinxcrossref{\sphinxcode{\sphinxupquote{IOLoop}}}}} to use for this test.

By default, a new {\hyperref[\detokenize{ioloop:tornado.ioloop.IOLoop}]{\sphinxcrossref{\sphinxcode{\sphinxupquote{IOLoop}}}}} is created for each test.
Subclasses may override this method to return
{\hyperref[\detokenize{ioloop:tornado.ioloop.IOLoop.current}]{\sphinxcrossref{\sphinxcode{\sphinxupquote{IOLoop.current()}}}}} if it is not appropriate to use a new
{\hyperref[\detokenize{ioloop:tornado.ioloop.IOLoop}]{\sphinxcrossref{\sphinxcode{\sphinxupquote{IOLoop}}}}} in each tests (for example, if there are global
singletons using the default {\hyperref[\detokenize{ioloop:tornado.ioloop.IOLoop}]{\sphinxcrossref{\sphinxcode{\sphinxupquote{IOLoop}}}}}) or if a per-test event
loop is being provided by another system (such as
\sphinxcode{\sphinxupquote{pytest-asyncio}}).

\end{fulllineitems}

\index{stop() (tornado.testing.AsyncTestCase method)@\spxentry{stop()}\spxextra{tornado.testing.AsyncTestCase method}}

\begin{fulllineitems}
\phantomsection\label{\detokenize{testing:tornado.testing.AsyncTestCase.stop}}\pysiglinewithargsret{\sphinxbfcode{\sphinxupquote{stop}}}{\emph{\_arg: Any = None}, \emph{**kwargs}}{{ $\rightarrow$ None}}
Stops the {\hyperref[\detokenize{ioloop:tornado.ioloop.IOLoop}]{\sphinxcrossref{\sphinxcode{\sphinxupquote{IOLoop}}}}}, causing one pending (or future) call to {\hyperref[\detokenize{testing:tornado.testing.AsyncTestCase.wait}]{\sphinxcrossref{\sphinxcode{\sphinxupquote{wait()}}}}}
to return.

Keyword arguments or a single positional argument passed to {\hyperref[\detokenize{testing:tornado.testing.AsyncTestCase.stop}]{\sphinxcrossref{\sphinxcode{\sphinxupquote{stop()}}}}} are
saved and will be returned by {\hyperref[\detokenize{testing:tornado.testing.AsyncTestCase.wait}]{\sphinxcrossref{\sphinxcode{\sphinxupquote{wait()}}}}}.

\DUrole{versionmodified,deprecated}{Deprecated since version 5.1: }{\hyperref[\detokenize{testing:tornado.testing.AsyncTestCase.stop}]{\sphinxcrossref{\sphinxcode{\sphinxupquote{stop}}}}} and {\hyperref[\detokenize{testing:tornado.testing.AsyncTestCase.wait}]{\sphinxcrossref{\sphinxcode{\sphinxupquote{wait}}}}} are deprecated; use \sphinxcode{\sphinxupquote{@gen\_test}} instead.

\end{fulllineitems}

\index{wait() (tornado.testing.AsyncTestCase method)@\spxentry{wait()}\spxextra{tornado.testing.AsyncTestCase method}}

\begin{fulllineitems}
\phantomsection\label{\detokenize{testing:tornado.testing.AsyncTestCase.wait}}\pysiglinewithargsret{\sphinxbfcode{\sphinxupquote{wait}}}{\emph{condition: Callable{[}{[}...{]}, bool{]} = None, timeout: float = None}}{{ $\rightarrow$ None}}
Runs the {\hyperref[\detokenize{ioloop:tornado.ioloop.IOLoop}]{\sphinxcrossref{\sphinxcode{\sphinxupquote{IOLoop}}}}} until stop is called or timeout has passed.

In the event of a timeout, an exception will be thrown. The
default timeout is 5 seconds; it may be overridden with a
\sphinxcode{\sphinxupquote{timeout}} keyword argument or globally with the
\sphinxcode{\sphinxupquote{ASYNC\_TEST\_TIMEOUT}} environment variable.

If \sphinxcode{\sphinxupquote{condition}} is not \sphinxcode{\sphinxupquote{None}}, the {\hyperref[\detokenize{ioloop:tornado.ioloop.IOLoop}]{\sphinxcrossref{\sphinxcode{\sphinxupquote{IOLoop}}}}} will be restarted
after {\hyperref[\detokenize{testing:tornado.testing.AsyncTestCase.stop}]{\sphinxcrossref{\sphinxcode{\sphinxupquote{stop()}}}}} until \sphinxcode{\sphinxupquote{condition()}} returns \sphinxcode{\sphinxupquote{True}}.

\DUrole{versionmodified,changed}{Changed in version 3.1: }Added the \sphinxcode{\sphinxupquote{ASYNC\_TEST\_TIMEOUT}} environment variable.

\DUrole{versionmodified,deprecated}{Deprecated since version 5.1: }{\hyperref[\detokenize{testing:tornado.testing.AsyncTestCase.stop}]{\sphinxcrossref{\sphinxcode{\sphinxupquote{stop}}}}} and {\hyperref[\detokenize{testing:tornado.testing.AsyncTestCase.wait}]{\sphinxcrossref{\sphinxcode{\sphinxupquote{wait}}}}} are deprecated; use \sphinxcode{\sphinxupquote{@gen\_test}} instead.

\end{fulllineitems}


\end{fulllineitems}

\index{AsyncHTTPTestCase (class in tornado.testing)@\spxentry{AsyncHTTPTestCase}\spxextra{class in tornado.testing}}

\begin{fulllineitems}
\phantomsection\label{\detokenize{testing:tornado.testing.AsyncHTTPTestCase}}\pysiglinewithargsret{\sphinxbfcode{\sphinxupquote{class }}\sphinxcode{\sphinxupquote{tornado.testing.}}\sphinxbfcode{\sphinxupquote{AsyncHTTPTestCase}}}{\emph{methodName: str = 'runTest'}}{}
A test case that starts up an HTTP server.

Subclasses must override {\hyperref[\detokenize{testing:tornado.testing.AsyncHTTPTestCase.get_app}]{\sphinxcrossref{\sphinxcode{\sphinxupquote{get\_app()}}}}}, which returns the
{\hyperref[\detokenize{web:tornado.web.Application}]{\sphinxcrossref{\sphinxcode{\sphinxupquote{tornado.web.Application}}}}} (or other {\hyperref[\detokenize{httpserver:tornado.httpserver.HTTPServer}]{\sphinxcrossref{\sphinxcode{\sphinxupquote{HTTPServer}}}}} callback) to be tested.
Tests will typically use the provided \sphinxcode{\sphinxupquote{self.http\_client}} to fetch
URLs from this server.

Example, assuming the “Hello, world” example from the user guide is in
\sphinxcode{\sphinxupquote{hello.py}}:

\begin{sphinxVerbatim}[commandchars=\\\{\}]
\PYG{k+kn}{import} \PYG{n+nn}{hello}

\PYG{k}{class} \PYG{n+nc}{TestHelloApp}\PYG{p}{(}\PYG{n}{AsyncHTTPTestCase}\PYG{p}{)}\PYG{p}{:}
    \PYG{k}{def} \PYG{n+nf}{get\PYGZus{}app}\PYG{p}{(}\PYG{n+nb+bp}{self}\PYG{p}{)}\PYG{p}{:}
        \PYG{k}{return} \PYG{n}{hello}\PYG{o}{.}\PYG{n}{make\PYGZus{}app}\PYG{p}{(}\PYG{p}{)}

    \PYG{k}{def} \PYG{n+nf}{test\PYGZus{}homepage}\PYG{p}{(}\PYG{n+nb+bp}{self}\PYG{p}{)}\PYG{p}{:}
        \PYG{n}{response} \PYG{o}{=} \PYG{n+nb+bp}{self}\PYG{o}{.}\PYG{n}{fetch}\PYG{p}{(}\PYG{l+s+s1}{\PYGZsq{}}\PYG{l+s+s1}{/}\PYG{l+s+s1}{\PYGZsq{}}\PYG{p}{)}
        \PYG{n+nb+bp}{self}\PYG{o}{.}\PYG{n}{assertEqual}\PYG{p}{(}\PYG{n}{response}\PYG{o}{.}\PYG{n}{code}\PYG{p}{,} \PYG{l+m+mi}{200}\PYG{p}{)}
        \PYG{n+nb+bp}{self}\PYG{o}{.}\PYG{n}{assertEqual}\PYG{p}{(}\PYG{n}{response}\PYG{o}{.}\PYG{n}{body}\PYG{p}{,} \PYG{l+s+s1}{\PYGZsq{}}\PYG{l+s+s1}{Hello, world}\PYG{l+s+s1}{\PYGZsq{}}\PYG{p}{)}
\end{sphinxVerbatim}

That call to \sphinxcode{\sphinxupquote{self.fetch()}} is equivalent to

\begin{sphinxVerbatim}[commandchars=\\\{\}]
\PYG{n+nb+bp}{self}\PYG{o}{.}\PYG{n}{http\PYGZus{}client}\PYG{o}{.}\PYG{n}{fetch}\PYG{p}{(}\PYG{n+nb+bp}{self}\PYG{o}{.}\PYG{n}{get\PYGZus{}url}\PYG{p}{(}\PYG{l+s+s1}{\PYGZsq{}}\PYG{l+s+s1}{/}\PYG{l+s+s1}{\PYGZsq{}}\PYG{p}{)}\PYG{p}{,} \PYG{n+nb+bp}{self}\PYG{o}{.}\PYG{n}{stop}\PYG{p}{)}
\PYG{n}{response} \PYG{o}{=} \PYG{n+nb+bp}{self}\PYG{o}{.}\PYG{n}{wait}\PYG{p}{(}\PYG{p}{)}
\end{sphinxVerbatim}

which illustrates how AsyncTestCase can turn an asynchronous operation,
like \sphinxcode{\sphinxupquote{http\_client.fetch()}}, into a synchronous operation. If you need
to do other asynchronous operations in tests, you’ll probably need to use
\sphinxcode{\sphinxupquote{stop()}} and \sphinxcode{\sphinxupquote{wait()}} yourself.
\index{get\_app() (tornado.testing.AsyncHTTPTestCase method)@\spxentry{get\_app()}\spxextra{tornado.testing.AsyncHTTPTestCase method}}

\begin{fulllineitems}
\phantomsection\label{\detokenize{testing:tornado.testing.AsyncHTTPTestCase.get_app}}\pysiglinewithargsret{\sphinxbfcode{\sphinxupquote{get\_app}}}{}{{ $\rightarrow$ tornado.web.Application}}
Should be overridden by subclasses to return a
{\hyperref[\detokenize{web:tornado.web.Application}]{\sphinxcrossref{\sphinxcode{\sphinxupquote{tornado.web.Application}}}}} or other {\hyperref[\detokenize{httpserver:tornado.httpserver.HTTPServer}]{\sphinxcrossref{\sphinxcode{\sphinxupquote{HTTPServer}}}}} callback.

\end{fulllineitems}

\index{fetch() (tornado.testing.AsyncHTTPTestCase method)@\spxentry{fetch()}\spxextra{tornado.testing.AsyncHTTPTestCase method}}

\begin{fulllineitems}
\phantomsection\label{\detokenize{testing:tornado.testing.AsyncHTTPTestCase.fetch}}\pysiglinewithargsret{\sphinxbfcode{\sphinxupquote{fetch}}}{\emph{path: str}, \emph{raise\_error: bool = False}, \emph{**kwargs}}{{ $\rightarrow$ tornado.httpclient.HTTPResponse}}
Convenience method to synchronously fetch a URL.

The given path will be appended to the local server’s host and
port.  Any additional keyword arguments will be passed directly to
{\hyperref[\detokenize{httpclient:tornado.httpclient.AsyncHTTPClient.fetch}]{\sphinxcrossref{\sphinxcode{\sphinxupquote{AsyncHTTPClient.fetch}}}}} (and so could be used to pass
\sphinxcode{\sphinxupquote{method="POST"}}, \sphinxcode{\sphinxupquote{body="..."}}, etc).

If the path begins with \sphinxurl{http://} or \sphinxurl{https://}, it will be treated as a
full URL and will be fetched as-is.

If \sphinxcode{\sphinxupquote{raise\_error}} is \sphinxcode{\sphinxupquote{True}}, a {\hyperref[\detokenize{httpclient:tornado.httpclient.HTTPError}]{\sphinxcrossref{\sphinxcode{\sphinxupquote{tornado.httpclient.HTTPError}}}}} will
be raised if the response code is not 200. This is the same behavior
as the \sphinxcode{\sphinxupquote{raise\_error}} argument to {\hyperref[\detokenize{httpclient:tornado.httpclient.AsyncHTTPClient.fetch}]{\sphinxcrossref{\sphinxcode{\sphinxupquote{AsyncHTTPClient.fetch}}}}}, but
the default is \sphinxcode{\sphinxupquote{False}} here (it’s \sphinxcode{\sphinxupquote{True}} in {\hyperref[\detokenize{httpclient:tornado.httpclient.AsyncHTTPClient}]{\sphinxcrossref{\sphinxcode{\sphinxupquote{AsyncHTTPClient}}}}})
because tests often need to deal with non-200 response codes.

\DUrole{versionmodified,changed}{Changed in version 5.0: }Added support for absolute URLs.

\DUrole{versionmodified,changed}{Changed in version 5.1: }Added the \sphinxcode{\sphinxupquote{raise\_error}} argument.

\DUrole{versionmodified,deprecated}{Deprecated since version 5.1: }This method currently turns any exception into an
{\hyperref[\detokenize{httpclient:tornado.httpclient.HTTPResponse}]{\sphinxcrossref{\sphinxcode{\sphinxupquote{HTTPResponse}}}}} with status code 599. In Tornado 6.0,
errors other than {\hyperref[\detokenize{httpclient:tornado.httpclient.HTTPError}]{\sphinxcrossref{\sphinxcode{\sphinxupquote{tornado.httpclient.HTTPError}}}}} will be
passed through, and \sphinxcode{\sphinxupquote{raise\_error=False}} will only
suppress errors that would be raised due to non-200
response codes.

\end{fulllineitems}

\index{get\_httpserver\_options() (tornado.testing.AsyncHTTPTestCase method)@\spxentry{get\_httpserver\_options()}\spxextra{tornado.testing.AsyncHTTPTestCase method}}

\begin{fulllineitems}
\phantomsection\label{\detokenize{testing:tornado.testing.AsyncHTTPTestCase.get_httpserver_options}}\pysiglinewithargsret{\sphinxbfcode{\sphinxupquote{get\_httpserver\_options}}}{}{{ $\rightarrow$ Dict{[}str, Any{]}}}
May be overridden by subclasses to return additional
keyword arguments for the server.

\end{fulllineitems}

\index{get\_http\_port() (tornado.testing.AsyncHTTPTestCase method)@\spxentry{get\_http\_port()}\spxextra{tornado.testing.AsyncHTTPTestCase method}}

\begin{fulllineitems}
\phantomsection\label{\detokenize{testing:tornado.testing.AsyncHTTPTestCase.get_http_port}}\pysiglinewithargsret{\sphinxbfcode{\sphinxupquote{get\_http\_port}}}{}{{ $\rightarrow$ int}}
Returns the port used by the server.

A new port is chosen for each test.

\end{fulllineitems}

\index{get\_url() (tornado.testing.AsyncHTTPTestCase method)@\spxentry{get\_url()}\spxextra{tornado.testing.AsyncHTTPTestCase method}}

\begin{fulllineitems}
\phantomsection\label{\detokenize{testing:tornado.testing.AsyncHTTPTestCase.get_url}}\pysiglinewithargsret{\sphinxbfcode{\sphinxupquote{get\_url}}}{\emph{path: str}}{{ $\rightarrow$ str}}
Returns an absolute url for the given path on the test server.

\end{fulllineitems}


\end{fulllineitems}

\index{AsyncHTTPSTestCase (class in tornado.testing)@\spxentry{AsyncHTTPSTestCase}\spxextra{class in tornado.testing}}

\begin{fulllineitems}
\phantomsection\label{\detokenize{testing:tornado.testing.AsyncHTTPSTestCase}}\pysiglinewithargsret{\sphinxbfcode{\sphinxupquote{class }}\sphinxcode{\sphinxupquote{tornado.testing.}}\sphinxbfcode{\sphinxupquote{AsyncHTTPSTestCase}}}{\emph{methodName: str = 'runTest'}}{}
A test case that starts an HTTPS server.

Interface is generally the same as {\hyperref[\detokenize{testing:tornado.testing.AsyncHTTPTestCase}]{\sphinxcrossref{\sphinxcode{\sphinxupquote{AsyncHTTPTestCase}}}}}.
\index{get\_ssl\_options() (tornado.testing.AsyncHTTPSTestCase method)@\spxentry{get\_ssl\_options()}\spxextra{tornado.testing.AsyncHTTPSTestCase method}}

\begin{fulllineitems}
\phantomsection\label{\detokenize{testing:tornado.testing.AsyncHTTPSTestCase.get_ssl_options}}\pysiglinewithargsret{\sphinxbfcode{\sphinxupquote{get\_ssl\_options}}}{}{{ $\rightarrow$ Dict{[}str, Any{]}}}
May be overridden by subclasses to select SSL options.

By default includes a self-signed testing certificate.

\end{fulllineitems}


\end{fulllineitems}

\index{gen\_test() (in module tornado.testing)@\spxentry{gen\_test()}\spxextra{in module tornado.testing}}

\begin{fulllineitems}
\phantomsection\label{\detokenize{testing:tornado.testing.gen_test}}\pysiglinewithargsret{\sphinxcode{\sphinxupquote{tornado.testing.}}\sphinxbfcode{\sphinxupquote{gen\_test}}}{\emph{func: Callable{[}{[}...{]}, Union{[}collections.abc.Generator, Coroutine{]}{]} = None, timeout: float = None}}{{ $\rightarrow$ Union{[}Callable{[}{[}...{]}, None{]}, Callable{[}{[}Callable{[}{[}...{]}, Union{[}collections.abc.Generator, Coroutine{]}{]}{]}, Callable{[}{[}...{]}, None{]}{]}{]}}}
Testing equivalent of \sphinxcode{\sphinxupquote{@gen.coroutine}}, to be applied to test methods.

\sphinxcode{\sphinxupquote{@gen.coroutine}} cannot be used on tests because the {\hyperref[\detokenize{ioloop:tornado.ioloop.IOLoop}]{\sphinxcrossref{\sphinxcode{\sphinxupquote{IOLoop}}}}} is not
already running.  \sphinxcode{\sphinxupquote{@gen\_test}} should be applied to test methods
on subclasses of {\hyperref[\detokenize{testing:tornado.testing.AsyncTestCase}]{\sphinxcrossref{\sphinxcode{\sphinxupquote{AsyncTestCase}}}}}.

Example:

\begin{sphinxVerbatim}[commandchars=\\\{\}]
\PYG{k}{class} \PYG{n+nc}{MyTest}\PYG{p}{(}\PYG{n}{AsyncHTTPTestCase}\PYG{p}{)}\PYG{p}{:}
    \PYG{n+nd}{@gen\PYGZus{}test}
    \PYG{k}{def} \PYG{n+nf}{test\PYGZus{}something}\PYG{p}{(}\PYG{n+nb+bp}{self}\PYG{p}{)}\PYG{p}{:}
        \PYG{n}{response} \PYG{o}{=} \PYG{k}{yield} \PYG{n+nb+bp}{self}\PYG{o}{.}\PYG{n}{http\PYGZus{}client}\PYG{o}{.}\PYG{n}{fetch}\PYG{p}{(}\PYG{n+nb+bp}{self}\PYG{o}{.}\PYG{n}{get\PYGZus{}url}\PYG{p}{(}\PYG{l+s+s1}{\PYGZsq{}}\PYG{l+s+s1}{/}\PYG{l+s+s1}{\PYGZsq{}}\PYG{p}{)}\PYG{p}{)}
\end{sphinxVerbatim}

By default, \sphinxcode{\sphinxupquote{@gen\_test}} times out after 5 seconds. The timeout may be
overridden globally with the \sphinxcode{\sphinxupquote{ASYNC\_TEST\_TIMEOUT}} environment variable,
or for each test with the \sphinxcode{\sphinxupquote{timeout}} keyword argument:

\begin{sphinxVerbatim}[commandchars=\\\{\}]
\PYG{k}{class} \PYG{n+nc}{MyTest}\PYG{p}{(}\PYG{n}{AsyncHTTPTestCase}\PYG{p}{)}\PYG{p}{:}
    \PYG{n+nd}{@gen\PYGZus{}test}\PYG{p}{(}\PYG{n}{timeout}\PYG{o}{=}\PYG{l+m+mi}{10}\PYG{p}{)}
    \PYG{k}{def} \PYG{n+nf}{test\PYGZus{}something\PYGZus{}slow}\PYG{p}{(}\PYG{n+nb+bp}{self}\PYG{p}{)}\PYG{p}{:}
        \PYG{n}{response} \PYG{o}{=} \PYG{k}{yield} \PYG{n+nb+bp}{self}\PYG{o}{.}\PYG{n}{http\PYGZus{}client}\PYG{o}{.}\PYG{n}{fetch}\PYG{p}{(}\PYG{n+nb+bp}{self}\PYG{o}{.}\PYG{n}{get\PYGZus{}url}\PYG{p}{(}\PYG{l+s+s1}{\PYGZsq{}}\PYG{l+s+s1}{/}\PYG{l+s+s1}{\PYGZsq{}}\PYG{p}{)}\PYG{p}{)}
\end{sphinxVerbatim}

Note that \sphinxcode{\sphinxupquote{@gen\_test}} is incompatible with {\hyperref[\detokenize{testing:tornado.testing.AsyncTestCase.stop}]{\sphinxcrossref{\sphinxcode{\sphinxupquote{AsyncTestCase.stop}}}}},
{\hyperref[\detokenize{testing:tornado.testing.AsyncTestCase.wait}]{\sphinxcrossref{\sphinxcode{\sphinxupquote{AsyncTestCase.wait}}}}}, and {\hyperref[\detokenize{testing:tornado.testing.AsyncHTTPTestCase.fetch}]{\sphinxcrossref{\sphinxcode{\sphinxupquote{AsyncHTTPTestCase.fetch}}}}}. Use \sphinxcode{\sphinxupquote{yield
self.http\_client.fetch(self.get\_url())}} as shown above instead.

\DUrole{versionmodified,added}{New in version 3.1: }The \sphinxcode{\sphinxupquote{timeout}} argument and \sphinxcode{\sphinxupquote{ASYNC\_TEST\_TIMEOUT}} environment
variable.

\DUrole{versionmodified,changed}{Changed in version 4.0: }The wrapper now passes along \sphinxcode{\sphinxupquote{*args, **kwargs}} so it can be used
on functions with arguments.

\end{fulllineitems}



\subsubsection{Controlling log output}
\label{\detokenize{testing:controlling-log-output}}\index{ExpectLog (class in tornado.testing)@\spxentry{ExpectLog}\spxextra{class in tornado.testing}}

\begin{fulllineitems}
\phantomsection\label{\detokenize{testing:tornado.testing.ExpectLog}}\pysiglinewithargsret{\sphinxbfcode{\sphinxupquote{class }}\sphinxcode{\sphinxupquote{tornado.testing.}}\sphinxbfcode{\sphinxupquote{ExpectLog}}}{\emph{logger: Union{[}logging.Logger, str{]}, regex: str, required: bool = True}}{}
Context manager to capture and suppress expected log output.

Useful to make tests of error conditions less noisy, while still
leaving unexpected log entries visible.  \sphinxstyleemphasis{Not thread safe.}

The attribute \sphinxcode{\sphinxupquote{logged\_stack}} is set to \sphinxcode{\sphinxupquote{True}} if any exception
stack trace was logged.

Usage:

\begin{sphinxVerbatim}[commandchars=\\\{\}]
\PYG{k}{with} \PYG{n}{ExpectLog}\PYG{p}{(}\PYG{l+s+s1}{\PYGZsq{}}\PYG{l+s+s1}{tornado.application}\PYG{l+s+s1}{\PYGZsq{}}\PYG{p}{,} \PYG{l+s+s2}{\PYGZdq{}}\PYG{l+s+s2}{Uncaught exception}\PYG{l+s+s2}{\PYGZdq{}}\PYG{p}{)}\PYG{p}{:}
    \PYG{n}{error\PYGZus{}response} \PYG{o}{=} \PYG{n+nb+bp}{self}\PYG{o}{.}\PYG{n}{fetch}\PYG{p}{(}\PYG{l+s+s2}{\PYGZdq{}}\PYG{l+s+s2}{/some\PYGZus{}page}\PYG{l+s+s2}{\PYGZdq{}}\PYG{p}{)}
\end{sphinxVerbatim}

\DUrole{versionmodified,changed}{Changed in version 4.3: }Added the \sphinxcode{\sphinxupquote{logged\_stack}} attribute.

Constructs an ExpectLog context manager.
\begin{quote}\begin{description}
\item[{Parameters}] \leavevmode\begin{itemize}
\item {} 
\sphinxstyleliteralstrong{\sphinxupquote{logger}} \textendash{} Logger object (or name of logger) to watch.  Pass
an empty string to watch the root logger.

\item {} 
\sphinxstyleliteralstrong{\sphinxupquote{regex}} \textendash{} Regular expression to match.  Any log entries on
the specified logger that match this regex will be suppressed.

\item {} 
\sphinxstyleliteralstrong{\sphinxupquote{required}} \textendash{} If true, an exception will be raised if the end of
the \sphinxcode{\sphinxupquote{with}} statement is reached without matching any log entries.

\end{itemize}

\end{description}\end{quote}

\end{fulllineitems}



\subsubsection{Test runner}
\label{\detokenize{testing:test-runner}}\index{main() (in module tornado.testing)@\spxentry{main()}\spxextra{in module tornado.testing}}

\begin{fulllineitems}
\phantomsection\label{\detokenize{testing:tornado.testing.main}}\pysiglinewithargsret{\sphinxcode{\sphinxupquote{tornado.testing.}}\sphinxbfcode{\sphinxupquote{main}}}{\emph{**kwargs}}{{ $\rightarrow$ None}}
A simple test runner.

This test runner is essentially equivalent to \sphinxhref{https://docs.python.org/3.6/library/unittest.html\#unittest.main}{\sphinxcode{\sphinxupquote{unittest.main}}} from
the standard library, but adds support for Tornado-style option
parsing and log formatting. It is \sphinxstyleemphasis{not} necessary to use this
{\hyperref[\detokenize{testing:tornado.testing.main}]{\sphinxcrossref{\sphinxcode{\sphinxupquote{main}}}}} function to run tests using {\hyperref[\detokenize{testing:tornado.testing.AsyncTestCase}]{\sphinxcrossref{\sphinxcode{\sphinxupquote{AsyncTestCase}}}}}; these tests
are self-contained and can run with any test runner.

The easiest way to run a test is via the command line:

\begin{sphinxVerbatim}[commandchars=\\\{\}]
\PYG{n}{python} \PYG{o}{\PYGZhy{}}\PYG{n}{m} \PYG{n}{tornado}\PYG{o}{.}\PYG{n}{testing} \PYG{n}{tornado}\PYG{o}{.}\PYG{n}{test}\PYG{o}{.}\PYG{n}{web\PYGZus{}test}
\end{sphinxVerbatim}

See the standard library \sphinxcode{\sphinxupquote{unittest}} module for ways in which
tests can be specified.

Projects with many tests may wish to define a test script like
\sphinxcode{\sphinxupquote{tornado/test/runtests.py}}.  This script should define a method
\sphinxcode{\sphinxupquote{all()}} which returns a test suite and then call
{\hyperref[\detokenize{testing:tornado.testing.main}]{\sphinxcrossref{\sphinxcode{\sphinxupquote{tornado.testing.main()}}}}}.  Note that even when a test script is
used, the \sphinxcode{\sphinxupquote{all()}} test suite may be overridden by naming a
single test on the command line:

\begin{sphinxVerbatim}[commandchars=\\\{\}]
\PYG{c+c1}{\PYGZsh{} Runs all tests}
\PYG{n}{python} \PYG{o}{\PYGZhy{}}\PYG{n}{m} \PYG{n}{tornado}\PYG{o}{.}\PYG{n}{test}\PYG{o}{.}\PYG{n}{runtests}
\PYG{c+c1}{\PYGZsh{} Runs one test}
\PYG{n}{python} \PYG{o}{\PYGZhy{}}\PYG{n}{m} \PYG{n}{tornado}\PYG{o}{.}\PYG{n}{test}\PYG{o}{.}\PYG{n}{runtests} \PYG{n}{tornado}\PYG{o}{.}\PYG{n}{test}\PYG{o}{.}\PYG{n}{web\PYGZus{}test}
\end{sphinxVerbatim}

Additional keyword arguments passed through to \sphinxcode{\sphinxupquote{unittest.main()}}.
For example, use \sphinxcode{\sphinxupquote{tornado.testing.main(verbosity=2)}}
to show many test details as they are run.
See \sphinxurl{http://docs.python.org/library/unittest.html\#unittest.main}
for full argument list.

\DUrole{versionmodified,changed}{Changed in version 5.0: }This function produces no output of its own; only that produced
by the \sphinxhref{https://docs.python.org/3.6/library/unittest.html\#module-unittest}{\sphinxcode{\sphinxupquote{unittest}}} module (previously it would add a PASS or FAIL
log message).

\end{fulllineitems}



\subsubsection{Helper functions}
\label{\detokenize{testing:helper-functions}}\index{bind\_unused\_port() (in module tornado.testing)@\spxentry{bind\_unused\_port()}\spxextra{in module tornado.testing}}

\begin{fulllineitems}
\phantomsection\label{\detokenize{testing:tornado.testing.bind_unused_port}}\pysiglinewithargsret{\sphinxcode{\sphinxupquote{tornado.testing.}}\sphinxbfcode{\sphinxupquote{bind\_unused\_port}}}{\emph{reuse\_port: bool = False}}{{ $\rightarrow$ Tuple{[}socket.socket, int{]}}}
Binds a server socket to an available port on localhost.

Returns a tuple (socket, port).

\DUrole{versionmodified,changed}{Changed in version 4.4: }Always binds to \sphinxcode{\sphinxupquote{127.0.0.1}} without resolving the name
\sphinxcode{\sphinxupquote{localhost}}.

\end{fulllineitems}

\index{get\_async\_test\_timeout() (in module tornado.testing)@\spxentry{get\_async\_test\_timeout()}\spxextra{in module tornado.testing}}

\begin{fulllineitems}
\phantomsection\label{\detokenize{testing:tornado.testing.get_async_test_timeout}}\pysiglinewithargsret{\sphinxcode{\sphinxupquote{tornado.testing.}}\sphinxbfcode{\sphinxupquote{get\_async\_test\_timeout}}}{}{{ $\rightarrow$ float}}
Get the global timeout setting for async tests.

Returns a float, the timeout in seconds.

\DUrole{versionmodified,added}{New in version 3.1.}

\end{fulllineitems}



\subsection{\sphinxstyleliteralintitle{\sphinxupquote{tornado.util}} — General-purpose utilities}
\label{\detokenize{util:tornado-util-general-purpose-utilities}}\label{\detokenize{util::doc}}\phantomsection\label{\detokenize{util:module-tornado.util}}\index{tornado.util (module)@\spxentry{tornado.util}\spxextra{module}}
Miscellaneous utility functions and classes.

This module is used internally by Tornado.  It is not necessarily expected
that the functions and classes defined here will be useful to other
applications, but they are documented here in case they are.

The one public-facing part of this module is the {\hyperref[\detokenize{util:tornado.util.Configurable}]{\sphinxcrossref{\sphinxcode{\sphinxupquote{Configurable}}}}} class
and its {\hyperref[\detokenize{util:tornado.util.Configurable.configure}]{\sphinxcrossref{\sphinxcode{\sphinxupquote{configure}}}}} method, which becomes a part of the
interface of its subclasses, including {\hyperref[\detokenize{httpclient:tornado.httpclient.AsyncHTTPClient}]{\sphinxcrossref{\sphinxcode{\sphinxupquote{AsyncHTTPClient}}}}}, {\hyperref[\detokenize{ioloop:tornado.ioloop.IOLoop}]{\sphinxcrossref{\sphinxcode{\sphinxupquote{IOLoop}}}}},
and {\hyperref[\detokenize{netutil:tornado.netutil.Resolver}]{\sphinxcrossref{\sphinxcode{\sphinxupquote{Resolver}}}}}.
\index{TimeoutError@\spxentry{TimeoutError}}

\begin{fulllineitems}
\phantomsection\label{\detokenize{util:tornado.util.TimeoutError}}\pysigline{\sphinxbfcode{\sphinxupquote{exception }}\sphinxcode{\sphinxupquote{tornado.util.}}\sphinxbfcode{\sphinxupquote{TimeoutError}}}
Exception raised by {\hyperref[\detokenize{gen:tornado.gen.with_timeout}]{\sphinxcrossref{\sphinxcode{\sphinxupquote{with\_timeout}}}}} and {\hyperref[\detokenize{ioloop:tornado.ioloop.IOLoop.run_sync}]{\sphinxcrossref{\sphinxcode{\sphinxupquote{IOLoop.run\_sync}}}}}.

\DUrole{versionmodified,changed}{Changed in version 5.0:: }Unified \sphinxcode{\sphinxupquote{tornado.gen.TimeoutError}} and
\sphinxcode{\sphinxupquote{tornado.ioloop.TimeoutError}} as \sphinxcode{\sphinxupquote{tornado.util.TimeoutError}}.
Both former names remain as aliases.

\end{fulllineitems}

\index{ObjectDict (class in tornado.util)@\spxentry{ObjectDict}\spxextra{class in tornado.util}}

\begin{fulllineitems}
\phantomsection\label{\detokenize{util:tornado.util.ObjectDict}}\pysigline{\sphinxbfcode{\sphinxupquote{class }}\sphinxcode{\sphinxupquote{tornado.util.}}\sphinxbfcode{\sphinxupquote{ObjectDict}}}
Makes a dictionary behave like an object, with attribute-style access.

\end{fulllineitems}

\index{GzipDecompressor (class in tornado.util)@\spxentry{GzipDecompressor}\spxextra{class in tornado.util}}

\begin{fulllineitems}
\phantomsection\label{\detokenize{util:tornado.util.GzipDecompressor}}\pysigline{\sphinxbfcode{\sphinxupquote{class }}\sphinxcode{\sphinxupquote{tornado.util.}}\sphinxbfcode{\sphinxupquote{GzipDecompressor}}}
Streaming gzip decompressor.

The interface is like that of \sphinxhref{https://docs.python.org/3.6/library/zlib.html\#zlib.decompressobj}{\sphinxcode{\sphinxupquote{zlib.decompressobj}}} (without some of the
optional arguments, but it understands gzip headers and checksums.
\index{decompress() (tornado.util.GzipDecompressor method)@\spxentry{decompress()}\spxextra{tornado.util.GzipDecompressor method}}

\begin{fulllineitems}
\phantomsection\label{\detokenize{util:tornado.util.GzipDecompressor.decompress}}\pysiglinewithargsret{\sphinxbfcode{\sphinxupquote{decompress}}}{\emph{value: bytes}, \emph{max\_length: int = 0}}{{ $\rightarrow$ bytes}}
Decompress a chunk, returning newly-available data.

Some data may be buffered for later processing; {\hyperref[\detokenize{util:tornado.util.GzipDecompressor.flush}]{\sphinxcrossref{\sphinxcode{\sphinxupquote{flush}}}}} must
be called when there is no more input data to ensure that
all data was processed.

If \sphinxcode{\sphinxupquote{max\_length}} is given, some input data may be left over
in \sphinxcode{\sphinxupquote{unconsumed\_tail}}; you must retrieve this value and pass
it back to a future call to {\hyperref[\detokenize{util:tornado.util.GzipDecompressor.decompress}]{\sphinxcrossref{\sphinxcode{\sphinxupquote{decompress}}}}} if it is not empty.

\end{fulllineitems}

\index{unconsumed\_tail (tornado.util.GzipDecompressor attribute)@\spxentry{unconsumed\_tail}\spxextra{tornado.util.GzipDecompressor attribute}}

\begin{fulllineitems}
\phantomsection\label{\detokenize{util:tornado.util.GzipDecompressor.unconsumed_tail}}\pysigline{\sphinxbfcode{\sphinxupquote{unconsumed\_tail}}}
Returns the unconsumed portion left over

\end{fulllineitems}

\index{flush() (tornado.util.GzipDecompressor method)@\spxentry{flush()}\spxextra{tornado.util.GzipDecompressor method}}

\begin{fulllineitems}
\phantomsection\label{\detokenize{util:tornado.util.GzipDecompressor.flush}}\pysiglinewithargsret{\sphinxbfcode{\sphinxupquote{flush}}}{}{{ $\rightarrow$ bytes}}
Return any remaining buffered data not yet returned by decompress.

Also checks for errors such as truncated input.
No other methods may be called on this object after {\hyperref[\detokenize{util:tornado.util.GzipDecompressor.flush}]{\sphinxcrossref{\sphinxcode{\sphinxupquote{flush}}}}}.

\end{fulllineitems}


\end{fulllineitems}

\index{import\_object() (in module tornado.util)@\spxentry{import\_object()}\spxextra{in module tornado.util}}

\begin{fulllineitems}
\phantomsection\label{\detokenize{util:tornado.util.import_object}}\pysiglinewithargsret{\sphinxcode{\sphinxupquote{tornado.util.}}\sphinxbfcode{\sphinxupquote{import\_object}}}{\emph{name: str}}{{ $\rightarrow$ Any}}
Imports an object by name.

\sphinxcode{\sphinxupquote{import\_object('x')}} is equivalent to \sphinxcode{\sphinxupquote{import x}}.
\sphinxcode{\sphinxupquote{import\_object('x.y.z')}} is equivalent to \sphinxcode{\sphinxupquote{from x.y import z}}.

\begin{sphinxVerbatim}[commandchars=\\\{\}]
\PYG{g+gp}{\PYGZgt{}\PYGZgt{}\PYGZgt{} }\PYG{k+kn}{import} \PYG{n+nn}{tornado}\PYG{n+nn}{.}\PYG{n+nn}{escape}
\PYG{g+gp}{\PYGZgt{}\PYGZgt{}\PYGZgt{} }\PYG{n}{import\PYGZus{}object}\PYG{p}{(}\PYG{l+s+s1}{\PYGZsq{}}\PYG{l+s+s1}{tornado.escape}\PYG{l+s+s1}{\PYGZsq{}}\PYG{p}{)} \PYG{o+ow}{is} \PYG{n}{tornado}\PYG{o}{.}\PYG{n}{escape}
\PYG{g+go}{True}
\PYG{g+gp}{\PYGZgt{}\PYGZgt{}\PYGZgt{} }\PYG{n}{import\PYGZus{}object}\PYG{p}{(}\PYG{l+s+s1}{\PYGZsq{}}\PYG{l+s+s1}{tornado.escape.utf8}\PYG{l+s+s1}{\PYGZsq{}}\PYG{p}{)} \PYG{o+ow}{is} \PYG{n}{tornado}\PYG{o}{.}\PYG{n}{escape}\PYG{o}{.}\PYG{n}{utf8}
\PYG{g+go}{True}
\PYG{g+gp}{\PYGZgt{}\PYGZgt{}\PYGZgt{} }\PYG{n}{import\PYGZus{}object}\PYG{p}{(}\PYG{l+s+s1}{\PYGZsq{}}\PYG{l+s+s1}{tornado}\PYG{l+s+s1}{\PYGZsq{}}\PYG{p}{)} \PYG{o+ow}{is} \PYG{n}{tornado}
\PYG{g+go}{True}
\PYG{g+gp}{\PYGZgt{}\PYGZgt{}\PYGZgt{} }\PYG{n}{import\PYGZus{}object}\PYG{p}{(}\PYG{l+s+s1}{\PYGZsq{}}\PYG{l+s+s1}{tornado.missing\PYGZus{}module}\PYG{l+s+s1}{\PYGZsq{}}\PYG{p}{)}
\PYG{g+gt}{Traceback (most recent call last):}
    \PYG{o}{.}\PYG{o}{.}\PYG{o}{.}
\PYG{g+gr}{ImportError}: \PYG{n}{No module named missing\PYGZus{}module}
\end{sphinxVerbatim}

\end{fulllineitems}

\index{errno\_from\_exception() (in module tornado.util)@\spxentry{errno\_from\_exception()}\spxextra{in module tornado.util}}

\begin{fulllineitems}
\phantomsection\label{\detokenize{util:tornado.util.errno_from_exception}}\pysiglinewithargsret{\sphinxcode{\sphinxupquote{tornado.util.}}\sphinxbfcode{\sphinxupquote{errno\_from\_exception}}}{\emph{e: BaseException}}{{ $\rightarrow$ Optional{[}int{]}}}
Provides the errno from an Exception object.

There are cases that the errno attribute was not set so we pull
the errno out of the args but if someone instantiates an Exception
without any args you will get a tuple error. So this function
abstracts all that behavior to give you a safe way to get the
errno.

\end{fulllineitems}

\index{re\_unescape() (in module tornado.util)@\spxentry{re\_unescape()}\spxextra{in module tornado.util}}

\begin{fulllineitems}
\phantomsection\label{\detokenize{util:tornado.util.re_unescape}}\pysiglinewithargsret{\sphinxcode{\sphinxupquote{tornado.util.}}\sphinxbfcode{\sphinxupquote{re\_unescape}}}{\emph{s: str}}{{ $\rightarrow$ str}}
Unescape a string escaped by \sphinxhref{https://docs.python.org/3.6/library/re.html\#re.escape}{\sphinxcode{\sphinxupquote{re.escape}}}.

May raise \sphinxcode{\sphinxupquote{ValueError}} for regular expressions which could not
have been produced by \sphinxhref{https://docs.python.org/3.6/library/re.html\#re.escape}{\sphinxcode{\sphinxupquote{re.escape}}} (for example, strings containing
\sphinxcode{\sphinxupquote{\textbackslash{}d}} cannot be unescaped).

\DUrole{versionmodified,added}{New in version 4.4.}

\end{fulllineitems}

\index{Configurable (class in tornado.util)@\spxentry{Configurable}\spxextra{class in tornado.util}}

\begin{fulllineitems}
\phantomsection\label{\detokenize{util:tornado.util.Configurable}}\pysigline{\sphinxbfcode{\sphinxupquote{class }}\sphinxcode{\sphinxupquote{tornado.util.}}\sphinxbfcode{\sphinxupquote{Configurable}}}
Base class for configurable interfaces.

A configurable interface is an (abstract) class whose constructor
acts as a factory function for one of its implementation subclasses.
The implementation subclass as well as optional keyword arguments to
its initializer can be set globally at runtime with {\hyperref[\detokenize{util:tornado.util.Configurable.configure}]{\sphinxcrossref{\sphinxcode{\sphinxupquote{configure}}}}}.

By using the constructor as the factory method, the interface
looks like a normal class, \sphinxhref{https://docs.python.org/3.6/library/functions.html\#isinstance}{\sphinxcode{\sphinxupquote{isinstance}}} works as usual, etc.  This
pattern is most useful when the choice of implementation is likely
to be a global decision (e.g. when \sphinxhref{https://docs.python.org/3.6/library/select.html\#select.epoll}{\sphinxcode{\sphinxupquote{epoll}}} is available,
always use it instead of \sphinxhref{https://docs.python.org/3.6/library/select.html\#select.select}{\sphinxcode{\sphinxupquote{select}}}), or when a
previously-monolithic class has been split into specialized
subclasses.

Configurable subclasses must define the class methods
{\hyperref[\detokenize{util:tornado.util.Configurable.configurable_base}]{\sphinxcrossref{\sphinxcode{\sphinxupquote{configurable\_base}}}}} and {\hyperref[\detokenize{util:tornado.util.Configurable.configurable_default}]{\sphinxcrossref{\sphinxcode{\sphinxupquote{configurable\_default}}}}}, and use the instance
method {\hyperref[\detokenize{util:tornado.util.Configurable.initialize}]{\sphinxcrossref{\sphinxcode{\sphinxupquote{initialize}}}}} instead of \sphinxcode{\sphinxupquote{\_\_init\_\_}}.

\DUrole{versionmodified,changed}{Changed in version 5.0: }It is now possible for configuration to be specified at
multiple levels of a class hierarchy.
\index{configurable\_base() (tornado.util.Configurable class method)@\spxentry{configurable\_base()}\spxextra{tornado.util.Configurable class method}}

\begin{fulllineitems}
\phantomsection\label{\detokenize{util:tornado.util.Configurable.configurable_base}}\pysiglinewithargsret{\sphinxbfcode{\sphinxupquote{classmethod }}\sphinxbfcode{\sphinxupquote{configurable\_base}}}{}{}
Returns the base class of a configurable hierarchy.

This will normally return the class in which it is defined.
(which is \sphinxstyleemphasis{not} necessarily the same as the \sphinxcode{\sphinxupquote{cls}} classmethod
parameter).

\end{fulllineitems}

\index{configurable\_default() (tornado.util.Configurable class method)@\spxentry{configurable\_default()}\spxextra{tornado.util.Configurable class method}}

\begin{fulllineitems}
\phantomsection\label{\detokenize{util:tornado.util.Configurable.configurable_default}}\pysiglinewithargsret{\sphinxbfcode{\sphinxupquote{classmethod }}\sphinxbfcode{\sphinxupquote{configurable\_default}}}{}{}
Returns the implementation class to be used if none is configured.

\end{fulllineitems}

\index{initialize() (tornado.util.Configurable method)@\spxentry{initialize()}\spxextra{tornado.util.Configurable method}}

\begin{fulllineitems}
\phantomsection\label{\detokenize{util:tornado.util.Configurable.initialize}}\pysiglinewithargsret{\sphinxbfcode{\sphinxupquote{initialize}}}{}{{ $\rightarrow$ None}}
Initialize a {\hyperref[\detokenize{util:tornado.util.Configurable}]{\sphinxcrossref{\sphinxcode{\sphinxupquote{Configurable}}}}} subclass instance.

Configurable classes should use {\hyperref[\detokenize{util:tornado.util.Configurable.initialize}]{\sphinxcrossref{\sphinxcode{\sphinxupquote{initialize}}}}} instead of \sphinxcode{\sphinxupquote{\_\_init\_\_}}.

\DUrole{versionmodified,changed}{Changed in version 4.2: }Now accepts positional arguments in addition to keyword arguments.

\end{fulllineitems}

\index{configure() (tornado.util.Configurable class method)@\spxentry{configure()}\spxextra{tornado.util.Configurable class method}}

\begin{fulllineitems}
\phantomsection\label{\detokenize{util:tornado.util.Configurable.configure}}\pysiglinewithargsret{\sphinxbfcode{\sphinxupquote{classmethod }}\sphinxbfcode{\sphinxupquote{configure}}}{\emph{impl}, \emph{**kwargs}}{}
Sets the class to use when the base class is instantiated.

Keyword arguments will be saved and added to the arguments passed
to the constructor.  This can be used to set global defaults for
some parameters.

\end{fulllineitems}

\index{configured\_class() (tornado.util.Configurable class method)@\spxentry{configured\_class()}\spxextra{tornado.util.Configurable class method}}

\begin{fulllineitems}
\phantomsection\label{\detokenize{util:tornado.util.Configurable.configured_class}}\pysiglinewithargsret{\sphinxbfcode{\sphinxupquote{classmethod }}\sphinxbfcode{\sphinxupquote{configured\_class}}}{}{}
Returns the currently configured class.

\end{fulllineitems}


\end{fulllineitems}

\index{ArgReplacer (class in tornado.util)@\spxentry{ArgReplacer}\spxextra{class in tornado.util}}

\begin{fulllineitems}
\phantomsection\label{\detokenize{util:tornado.util.ArgReplacer}}\pysiglinewithargsret{\sphinxbfcode{\sphinxupquote{class }}\sphinxcode{\sphinxupquote{tornado.util.}}\sphinxbfcode{\sphinxupquote{ArgReplacer}}}{\emph{func: Callable}, \emph{name: str}}{}
Replaces one value in an \sphinxcode{\sphinxupquote{args, kwargs}} pair.

Inspects the function signature to find an argument by name
whether it is passed by position or keyword.  For use in decorators
and similar wrappers.
\index{get\_old\_value() (tornado.util.ArgReplacer method)@\spxentry{get\_old\_value()}\spxextra{tornado.util.ArgReplacer method}}

\begin{fulllineitems}
\phantomsection\label{\detokenize{util:tornado.util.ArgReplacer.get_old_value}}\pysiglinewithargsret{\sphinxbfcode{\sphinxupquote{get\_old\_value}}}{\emph{args: Sequence{[}Any{]}, kwargs: Dict{[}str, Any{]}, default: Any = None}}{{ $\rightarrow$ Any}}
Returns the old value of the named argument without replacing it.

Returns \sphinxcode{\sphinxupquote{default}} if the argument is not present.

\end{fulllineitems}

\index{replace() (tornado.util.ArgReplacer method)@\spxentry{replace()}\spxextra{tornado.util.ArgReplacer method}}

\begin{fulllineitems}
\phantomsection\label{\detokenize{util:tornado.util.ArgReplacer.replace}}\pysiglinewithargsret{\sphinxbfcode{\sphinxupquote{replace}}}{\emph{new\_value: Any, args: Sequence{[}Any{]}, kwargs: Dict{[}str, Any{]}}}{{ $\rightarrow$ Tuple{[}Any, Sequence{[}Any{]}, Dict{[}str, Any{]}{]}}}
Replace the named argument in \sphinxcode{\sphinxupquote{args, kwargs}} with \sphinxcode{\sphinxupquote{new\_value}}.

Returns \sphinxcode{\sphinxupquote{(old\_value, args, kwargs)}}.  The returned \sphinxcode{\sphinxupquote{args}} and
\sphinxcode{\sphinxupquote{kwargs}} objects may not be the same as the input objects, or
the input objects may be mutated.

If the named argument was not found, \sphinxcode{\sphinxupquote{new\_value}} will be added
to \sphinxcode{\sphinxupquote{kwargs}} and None will be returned as \sphinxcode{\sphinxupquote{old\_value}}.

\end{fulllineitems}


\end{fulllineitems}

\index{timedelta\_to\_seconds() (in module tornado.util)@\spxentry{timedelta\_to\_seconds()}\spxextra{in module tornado.util}}

\begin{fulllineitems}
\phantomsection\label{\detokenize{util:tornado.util.timedelta_to_seconds}}\pysiglinewithargsret{\sphinxcode{\sphinxupquote{tornado.util.}}\sphinxbfcode{\sphinxupquote{timedelta\_to\_seconds}}}{\emph{td}}{}
Equivalent to \sphinxcode{\sphinxupquote{td.total\_seconds()}} (introduced in Python 2.7).

\end{fulllineitems}



\section{Frequently Asked Questions}
\label{\detokenize{faq:frequently-asked-questions}}\label{\detokenize{faq::doc}}
\begin{sphinxShadowBox}
\begin{itemize}
\item {} 
\phantomsection\label{\detokenize{faq:id2}}{\hyperref[\detokenize{faq:why-isn-t-this-example-with-time-sleep-running-in-parallel}]{\sphinxcrossref{Why isn’t this example with \sphinxcode{\sphinxupquote{time.sleep()}} running in parallel?}}}

\item {} 
\phantomsection\label{\detokenize{faq:id3}}{\hyperref[\detokenize{faq:my-code-is-asynchronous-why-is-it-not-running-in-parallel-in-two-browser-tabs}]{\sphinxcrossref{My code is asynchronous. Why is it not running in parallel in two browser tabs?}}}

\end{itemize}
\end{sphinxShadowBox}


\subsection{Why isn’t this example with \sphinxstyleliteralintitle{\sphinxupquote{time.sleep()}} running in parallel?}
\label{\detokenize{faq:why-isn-t-this-example-with-time-sleep-running-in-parallel}}
Many people’s first foray into Tornado’s concurrency looks something like
this:

\begin{sphinxVerbatim}[commandchars=\\\{\}]
\PYG{k}{class} \PYG{n+nc}{BadExampleHandler}\PYG{p}{(}\PYG{n}{RequestHandler}\PYG{p}{)}\PYG{p}{:}
    \PYG{k}{def} \PYG{n+nf}{get}\PYG{p}{(}\PYG{n+nb+bp}{self}\PYG{p}{)}\PYG{p}{:}
        \PYG{k}{for} \PYG{n}{i} \PYG{o+ow}{in} \PYG{n+nb}{range}\PYG{p}{(}\PYG{l+m+mi}{5}\PYG{p}{)}\PYG{p}{:}
            \PYG{n+nb}{print}\PYG{p}{(}\PYG{n}{i}\PYG{p}{)}
            \PYG{n}{time}\PYG{o}{.}\PYG{n}{sleep}\PYG{p}{(}\PYG{l+m+mi}{1}\PYG{p}{)}
\end{sphinxVerbatim}

Fetch this handler twice at the same time and you’ll see that the second
five-second countdown doesn’t start until the first one has completely
finished. The reason for this is that \sphinxhref{https://docs.python.org/3.6/library/time.html\#time.sleep}{\sphinxcode{\sphinxupquote{time.sleep}}} is a \sphinxstylestrong{blocking}
function: it doesn’t allow control to return to the {\hyperref[\detokenize{ioloop:tornado.ioloop.IOLoop}]{\sphinxcrossref{\sphinxcode{\sphinxupquote{IOLoop}}}}} so that other
handlers can be run.

Of course, \sphinxhref{https://docs.python.org/3.6/library/time.html\#time.sleep}{\sphinxcode{\sphinxupquote{time.sleep}}} is really just a placeholder in these examples,
the point is to show what happens when something in a handler gets slow.
No matter what the real code is doing, to achieve concurrency blocking
code must be replaced with non-blocking equivalents. This means one of three things:
\begin{enumerate}
\def\theenumi{\arabic{enumi}}
\def\labelenumi{\theenumi .}
\makeatletter\def\p@enumii{\p@enumi \theenumi .}\makeatother
\item {} 
\sphinxstyleemphasis{Find a coroutine-friendly equivalent.} For \sphinxhref{https://docs.python.org/3.6/library/time.html\#time.sleep}{\sphinxcode{\sphinxupquote{time.sleep}}}, use
{\hyperref[\detokenize{gen:tornado.gen.sleep}]{\sphinxcrossref{\sphinxcode{\sphinxupquote{tornado.gen.sleep}}}}} (or \sphinxhref{https://docs.python.org/3.6/library/asyncio-task.html\#asyncio.sleep}{\sphinxcode{\sphinxupquote{asyncio.sleep}}}) instead:

\begin{sphinxVerbatim}[commandchars=\\\{\}]
\PYG{k}{class} \PYG{n+nc}{CoroutineSleepHandler}\PYG{p}{(}\PYG{n}{RequestHandler}\PYG{p}{)}\PYG{p}{:}
    \PYG{k}{async} \PYG{k}{def} \PYG{n+nf}{get}\PYG{p}{(}\PYG{n+nb+bp}{self}\PYG{p}{)}\PYG{p}{:}
        \PYG{k}{for} \PYG{n}{i} \PYG{o+ow}{in} \PYG{n+nb}{range}\PYG{p}{(}\PYG{l+m+mi}{5}\PYG{p}{)}\PYG{p}{:}
            \PYG{n+nb}{print}\PYG{p}{(}\PYG{n}{i}\PYG{p}{)}
            \PYG{k}{await} \PYG{n}{gen}\PYG{o}{.}\PYG{n}{sleep}\PYG{p}{(}\PYG{l+m+mi}{1}\PYG{p}{)}
\end{sphinxVerbatim}

When this option is available, it is usually the best approach.
See the \sphinxhref{https://github.com/tornadoweb/tornado/wiki/Links}{Tornado wiki}
for links to asynchronous libraries that may be useful.

\item {} 
\sphinxstyleemphasis{Find a callback-based equivalent.} Similar to the first option,
callback-based libraries are available for many tasks, although they
are slightly more complicated to use than a library designed for
coroutines. Adapt the callback-based function into a future:

\begin{sphinxVerbatim}[commandchars=\\\{\}]
\PYG{k}{class} \PYG{n+nc}{CoroutineTimeoutHandler}\PYG{p}{(}\PYG{n}{RequestHandler}\PYG{p}{)}\PYG{p}{:}
    \PYG{k}{async} \PYG{k}{def} \PYG{n+nf}{get}\PYG{p}{(}\PYG{n+nb+bp}{self}\PYG{p}{)}\PYG{p}{:}
        \PYG{n}{io\PYGZus{}loop} \PYG{o}{=} \PYG{n}{IOLoop}\PYG{o}{.}\PYG{n}{current}\PYG{p}{(}\PYG{p}{)}
        \PYG{k}{for} \PYG{n}{i} \PYG{o+ow}{in} \PYG{n+nb}{range}\PYG{p}{(}\PYG{l+m+mi}{5}\PYG{p}{)}\PYG{p}{:}
            \PYG{n+nb}{print}\PYG{p}{(}\PYG{n}{i}\PYG{p}{)}
            \PYG{n}{f} \PYG{o}{=} \PYG{n}{tornado}\PYG{o}{.}\PYG{n}{concurrent}\PYG{o}{.}\PYG{n}{Future}\PYG{p}{(}\PYG{p}{)}
            \PYG{n}{do\PYGZus{}something\PYGZus{}with\PYGZus{}callback}\PYG{p}{(}\PYG{n}{f}\PYG{o}{.}\PYG{n}{set\PYGZus{}result}\PYG{p}{)}
            \PYG{n}{result} \PYG{o}{=} \PYG{k}{await} \PYG{n}{f}
\end{sphinxVerbatim}

Again, the
\sphinxhref{https://github.com/tornadoweb/tornado/wiki/Links}{Tornado wiki}
can be useful to find suitable libraries.

\item {} 
\sphinxstyleemphasis{Run the blocking code on another thread.} When asynchronous libraries
are not available, \sphinxhref{https://docs.python.org/3.6/library/concurrent.futures.html\#concurrent.futures.ThreadPoolExecutor}{\sphinxcode{\sphinxupquote{concurrent.futures.ThreadPoolExecutor}}} can be used
to run any blocking code on another thread. This is a universal solution
that can be used for any blocking function whether an asynchronous
counterpart exists or not:

\begin{sphinxVerbatim}[commandchars=\\\{\}]
\PYG{k}{class} \PYG{n+nc}{ThreadPoolHandler}\PYG{p}{(}\PYG{n}{RequestHandler}\PYG{p}{)}\PYG{p}{:}
    \PYG{k}{async} \PYG{k}{def} \PYG{n+nf}{get}\PYG{p}{(}\PYG{n+nb+bp}{self}\PYG{p}{)}\PYG{p}{:}
        \PYG{k}{for} \PYG{n}{i} \PYG{o+ow}{in} \PYG{n+nb}{range}\PYG{p}{(}\PYG{l+m+mi}{5}\PYG{p}{)}\PYG{p}{:}
            \PYG{n+nb}{print}\PYG{p}{(}\PYG{n}{i}\PYG{p}{)}
            \PYG{k}{await} \PYG{n}{IOLoop}\PYG{o}{.}\PYG{n}{current}\PYG{p}{(}\PYG{p}{)}\PYG{o}{.}\PYG{n}{run\PYGZus{}in\PYGZus{}executor}\PYG{p}{(}\PYG{k+kc}{None}\PYG{p}{,} \PYG{n}{time}\PYG{o}{.}\PYG{n}{sleep}\PYG{p}{,} \PYG{l+m+mi}{1}\PYG{p}{)}
\end{sphinxVerbatim}

\end{enumerate}

See the {\hyperref[\detokenize{guide/async::doc}]{\sphinxcrossref{\DUrole{doc}{Asynchronous I/O}}}} chapter of the Tornado
user’s guide for more on blocking and asynchronous functions.


\subsection{My code is asynchronous. Why is it not running in parallel in two browser tabs?}
\label{\detokenize{faq:my-code-is-asynchronous-why-is-it-not-running-in-parallel-in-two-browser-tabs}}
Even when a handler is asynchronous and non-blocking, it can be surprisingly
tricky to verify this. Browsers will recognize that you are trying to
load the same page in two different tabs and delay the second request
until the first has finished. To work around this and see that the server
is in fact working in parallel, do one of two things:
\begin{itemize}
\item {} 
Add something to your urls to make them unique. Instead of
\sphinxcode{\sphinxupquote{http://localhost:8888}} in both tabs, load
\sphinxcode{\sphinxupquote{http://localhost:8888/?x=1}} in one and
\sphinxcode{\sphinxupquote{http://localhost:8888/?x=2}} in the other.

\item {} 
Use two different browsers. For example, Firefox will be able to load
a url even while that same url is being loaded in a Chrome tab.

\end{itemize}


\section{Release notes}
\label{\detokenize{releases:release-notes}}\label{\detokenize{releases::doc}}

\subsection{What’s new in Tornado 6.0}
\label{\detokenize{releases/v6.0.0:what-s-new-in-tornado-6-0}}\label{\detokenize{releases/v6.0.0::doc}}

\subsubsection{Mar 1, 2019}
\label{\detokenize{releases/v6.0.0:mar-1-2019}}

\paragraph{Backwards-incompatible changes}
\label{\detokenize{releases/v6.0.0:backwards-incompatible-changes}}\begin{itemize}
\item {} 
Python 2.7 and 3.4 are no longer supported; the minimum supported
Python version is 3.5.2.

\item {} 
APIs deprecated in Tornado 5.1 have been removed. This includes the
\sphinxcode{\sphinxupquote{tornado.stack\_context}} module and most \sphinxcode{\sphinxupquote{callback}} arguments
throughout the package. All removed APIs emitted
\sphinxhref{https://docs.python.org/3.6/library/exceptions.html\#DeprecationWarning}{\sphinxcode{\sphinxupquote{DeprecationWarning}}} when used in Tornado 5.1, so running your
application with the \sphinxcode{\sphinxupquote{-Wd}} Python command-line flag or the
environment variable \sphinxcode{\sphinxupquote{PYTHONWARNINGS=d}} should tell you whether
your application is ready to move to Tornado 6.0.

\item {} 
\sphinxcode{\sphinxupquote{.WebSocketHandler.get}} is now a coroutine and must be called
accordingly in any subclasses that override this method (but note
that overriding \sphinxcode{\sphinxupquote{get}} is not recommended; either \sphinxcode{\sphinxupquote{prepare}} or
\sphinxcode{\sphinxupquote{open}} should be used instead).

\end{itemize}


\paragraph{General changes}
\label{\detokenize{releases/v6.0.0:general-changes}}\begin{itemize}
\item {} 
Tornado now includes type annotations compatible with \sphinxcode{\sphinxupquote{mypy}}.
These annotations will be used when type-checking your application
with \sphinxcode{\sphinxupquote{mypy}}, and may be usable in editors and other tools.

\item {} 
Tornado now uses native coroutines internally, improving performance.

\end{itemize}


\paragraph{\sphinxstyleliteralintitle{\sphinxupquote{tornado.auth}}}
\label{\detokenize{releases/v6.0.0:tornado-auth}}\begin{itemize}
\item {} 
All \sphinxcode{\sphinxupquote{callback}} arguments in this package have been removed. Use
the coroutine interfaces instead.

\item {} 
The \sphinxcode{\sphinxupquote{OAuthMixin.\_oauth\_get\_user}} method has been removed.
Override {\hyperref[\detokenize{auth:tornado.auth.OAuthMixin._oauth_get_user_future}]{\sphinxcrossref{\sphinxcode{\sphinxupquote{\_oauth\_get\_user\_future}}}}} instead.

\end{itemize}


\paragraph{\sphinxstyleliteralintitle{\sphinxupquote{tornado.concurrent}}}
\label{\detokenize{releases/v6.0.0:tornado-concurrent}}\begin{itemize}
\item {} 
The \sphinxcode{\sphinxupquote{callback}} argument to {\hyperref[\detokenize{concurrent:tornado.concurrent.run_on_executor}]{\sphinxcrossref{\sphinxcode{\sphinxupquote{run\_on\_executor}}}}} has been removed.

\item {} 
\sphinxcode{\sphinxupquote{return\_future}} has been removed.

\end{itemize}


\paragraph{\sphinxstyleliteralintitle{\sphinxupquote{tornado.gen}}}
\label{\detokenize{releases/v6.0.0:tornado-gen}}\begin{itemize}
\item {} 
Some older portions of this module have been removed. This includes
\sphinxcode{\sphinxupquote{engine}}, \sphinxcode{\sphinxupquote{YieldPoint}}, \sphinxcode{\sphinxupquote{Callback}}, \sphinxcode{\sphinxupquote{Wait}}, \sphinxcode{\sphinxupquote{WaitAll}},
\sphinxcode{\sphinxupquote{MultiYieldPoint}}, and \sphinxcode{\sphinxupquote{Task}}.

\item {} 
Functions decorated with \sphinxcode{\sphinxupquote{@gen.coroutine}} no longer accept
\sphinxcode{\sphinxupquote{callback}} arguments.

\end{itemize}


\paragraph{\sphinxstyleliteralintitle{\sphinxupquote{tornado.httpclient}}}
\label{\detokenize{releases/v6.0.0:tornado-httpclient}}\begin{itemize}
\item {} 
The behavior of \sphinxcode{\sphinxupquote{raise\_error=False}} has changed. Now only
suppresses the errors raised due to completed responses with non-200
status codes (previously it suppressed all errors).

\item {} 
The \sphinxcode{\sphinxupquote{callback}} argument to {\hyperref[\detokenize{httpclient:tornado.httpclient.AsyncHTTPClient.fetch}]{\sphinxcrossref{\sphinxcode{\sphinxupquote{AsyncHTTPClient.fetch}}}}} has been removed.

\end{itemize}


\paragraph{\sphinxstyleliteralintitle{\sphinxupquote{tornado.httputil}}}
\label{\detokenize{releases/v6.0.0:tornado-httputil}}\begin{itemize}
\item {} 
\sphinxcode{\sphinxupquote{HTTPServerRequest.write}} has been removed. Use the methods of
\sphinxcode{\sphinxupquote{request.connection}} instead.

\item {} 
Unrecognized \sphinxcode{\sphinxupquote{Content-Encoding}} values now log warnings only for
content types that we would otherwise attempt to parse.

\end{itemize}


\paragraph{\sphinxstyleliteralintitle{\sphinxupquote{tornado.ioloop}}}
\label{\detokenize{releases/v6.0.0:tornado-ioloop}}\begin{itemize}
\item {} 
\sphinxcode{\sphinxupquote{IOLoop.set\_blocking\_signal\_threshold}},
\sphinxcode{\sphinxupquote{IOLoop.set\_blocking\_log\_threshold}}, \sphinxcode{\sphinxupquote{IOLoop.log\_stack}},
and \sphinxcode{\sphinxupquote{IOLoop.handle\_callback\_exception}} have been removed.

\item {} 
Improved performance of {\hyperref[\detokenize{ioloop:tornado.ioloop.IOLoop.add_callback}]{\sphinxcrossref{\sphinxcode{\sphinxupquote{IOLoop.add\_callback}}}}}.

\end{itemize}


\paragraph{\sphinxstyleliteralintitle{\sphinxupquote{tornado.iostream}}}
\label{\detokenize{releases/v6.0.0:tornado-iostream}}\begin{itemize}
\item {} 
All \sphinxcode{\sphinxupquote{callback}} arguments in this module have been removed except
for {\hyperref[\detokenize{iostream:tornado.iostream.BaseIOStream.set_close_callback}]{\sphinxcrossref{\sphinxcode{\sphinxupquote{BaseIOStream.set\_close\_callback}}}}}.

\item {} 
\sphinxcode{\sphinxupquote{streaming\_callback}} arguments to {\hyperref[\detokenize{iostream:tornado.iostream.BaseIOStream.read_bytes}]{\sphinxcrossref{\sphinxcode{\sphinxupquote{BaseIOStream.read\_bytes}}}}} and
{\hyperref[\detokenize{iostream:tornado.iostream.BaseIOStream.read_until_close}]{\sphinxcrossref{\sphinxcode{\sphinxupquote{BaseIOStream.read\_until\_close}}}}} have been removed.

\item {} 
Eliminated unnecessary logging of “Errno 0”.

\end{itemize}


\paragraph{\sphinxstyleliteralintitle{\sphinxupquote{tornado.log}}}
\label{\detokenize{releases/v6.0.0:tornado-log}}\begin{itemize}
\item {} 
Log files opened by this module are now explicitly set to UTF-8 encoding.

\end{itemize}


\paragraph{\sphinxstyleliteralintitle{\sphinxupquote{tornado.netutil}}}
\label{\detokenize{releases/v6.0.0:tornado-netutil}}\begin{itemize}
\item {} 
The results of \sphinxcode{\sphinxupquote{getaddrinfo}} are now sorted by address family to
avoid partial failures and deadlocks.

\end{itemize}


\paragraph{\sphinxstyleliteralintitle{\sphinxupquote{tornado.platform.twisted}}}
\label{\detokenize{releases/v6.0.0:tornado-platform-twisted}}\begin{itemize}
\item {} 
\sphinxcode{\sphinxupquote{TornadoReactor}} and \sphinxcode{\sphinxupquote{TwistedIOLoop}} have been removed.

\end{itemize}


\paragraph{\sphinxstyleliteralintitle{\sphinxupquote{tornado.simple\_httpclient}}}
\label{\detokenize{releases/v6.0.0:tornado-simple-httpclient}}\begin{itemize}
\item {} 
The default HTTP client now supports the \sphinxcode{\sphinxupquote{network\_interface}}
request argument to specify the source IP for the connection.

\item {} 
If a server returns a 3xx response code without a \sphinxcode{\sphinxupquote{Location}}
header, the response is raised or returned directly instead of
trying and failing to follow the redirect.

\item {} 
When following redirects, methods other than \sphinxcode{\sphinxupquote{POST}} will no longer
be transformed into \sphinxcode{\sphinxupquote{GET}} requests. 301 (permanent) redirects are
now treated the same way as 302 (temporary) and 303 (see other)
redirects in this respect.

\item {} 
Following redirects now works with \sphinxcode{\sphinxupquote{body\_producer}}.

\end{itemize}


\paragraph{\sphinxstyleliteralintitle{\sphinxupquote{tornado.stack\_context}}}
\label{\detokenize{releases/v6.0.0:tornado-stack-context}}\begin{itemize}
\item {} 
The \sphinxcode{\sphinxupquote{tornado.stack\_context}} module has been removed.

\end{itemize}


\paragraph{\sphinxstyleliteralintitle{\sphinxupquote{tornado.tcpserver}}}
\label{\detokenize{releases/v6.0.0:tornado-tcpserver}}\begin{itemize}
\item {} 
{\hyperref[\detokenize{tcpserver:tornado.tcpserver.TCPServer.start}]{\sphinxcrossref{\sphinxcode{\sphinxupquote{TCPServer.start}}}}} now supports a \sphinxcode{\sphinxupquote{max\_restarts}} argument (same as
{\hyperref[\detokenize{process:tornado.process.fork_processes}]{\sphinxcrossref{\sphinxcode{\sphinxupquote{fork\_processes}}}}}).

\end{itemize}


\paragraph{\sphinxstyleliteralintitle{\sphinxupquote{tornado.testing}}}
\label{\detokenize{releases/v6.0.0:tornado-testing}}\begin{itemize}
\item {} 
{\hyperref[\detokenize{testing:tornado.testing.AsyncHTTPTestCase}]{\sphinxcrossref{\sphinxcode{\sphinxupquote{AsyncHTTPTestCase}}}}} now drops all references to the {\hyperref[\detokenize{web:tornado.web.Application}]{\sphinxcrossref{\sphinxcode{\sphinxupquote{Application}}}}}
during \sphinxcode{\sphinxupquote{tearDown}}, allowing its memory to be reclaimed sooner.

\item {} 
{\hyperref[\detokenize{testing:tornado.testing.AsyncTestCase}]{\sphinxcrossref{\sphinxcode{\sphinxupquote{AsyncTestCase}}}}} now cancels all pending coroutines in \sphinxcode{\sphinxupquote{tearDown}},
in an effort to reduce warnings from the python runtime about
coroutines that were not awaited. Note that this may cause
\sphinxcode{\sphinxupquote{asyncio.CancelledError}} to be logged in other places. Coroutines
that expect to be running at test shutdown may need to catch this
exception.

\end{itemize}


\paragraph{\sphinxstyleliteralintitle{\sphinxupquote{tornado.web}}}
\label{\detokenize{releases/v6.0.0:tornado-web}}\begin{itemize}
\item {} 
The \sphinxcode{\sphinxupquote{asynchronous}} decorator has been removed.

\item {} 
The \sphinxcode{\sphinxupquote{callback}} argument to {\hyperref[\detokenize{web:tornado.web.RequestHandler.flush}]{\sphinxcrossref{\sphinxcode{\sphinxupquote{RequestHandler.flush}}}}} has been removed.

\item {} 
{\hyperref[\detokenize{web:tornado.web.StaticFileHandler}]{\sphinxcrossref{\sphinxcode{\sphinxupquote{StaticFileHandler}}}}} now supports large negative values for the
\sphinxcode{\sphinxupquote{Range}} header and returns an appropriate error for \sphinxcode{\sphinxupquote{end \textgreater{}
start}}.

\item {} 
It is now possible to set \sphinxcode{\sphinxupquote{expires\_days}} in \sphinxcode{\sphinxupquote{xsrf\_cookie\_kwargs}}.

\end{itemize}


\paragraph{\sphinxstyleliteralintitle{\sphinxupquote{tornado.websocket}}}
\label{\detokenize{releases/v6.0.0:tornado-websocket}}\begin{itemize}
\item {} 
Pings and other messages sent while the connection is closing are
now silently dropped instead of logging exceptions.

\item {} 
Errors raised by \sphinxcode{\sphinxupquote{open()}} are now caught correctly when this method
is a coroutine.

\end{itemize}


\paragraph{\sphinxstyleliteralintitle{\sphinxupquote{tornado.wsgi}}}
\label{\detokenize{releases/v6.0.0:tornado-wsgi}}\begin{itemize}
\item {} 
\sphinxcode{\sphinxupquote{WSGIApplication}} and \sphinxcode{\sphinxupquote{WSGIAdapter}} have been removed.

\end{itemize}


\subsection{What’s new in Tornado 5.1.1}
\label{\detokenize{releases/v5.1.1:what-s-new-in-tornado-5-1-1}}\label{\detokenize{releases/v5.1.1::doc}}

\subsubsection{Sep 16, 2018}
\label{\detokenize{releases/v5.1.1:sep-16-2018}}

\paragraph{Bug fixes}
\label{\detokenize{releases/v5.1.1:bug-fixes}}\begin{itemize}
\item {} 
Fixed an case in which the {\hyperref[\detokenize{concurrent:tornado.concurrent.Future}]{\sphinxcrossref{\sphinxcode{\sphinxupquote{Future}}}}} returned by
{\hyperref[\detokenize{web:tornado.web.RequestHandler.finish}]{\sphinxcrossref{\sphinxcode{\sphinxupquote{RequestHandler.finish}}}}} could fail to resolve.

\item {} 
The {\hyperref[\detokenize{auth:tornado.auth.TwitterMixin.authenticate_redirect}]{\sphinxcrossref{\sphinxcode{\sphinxupquote{TwitterMixin.authenticate\_redirect}}}}} method works again.

\item {} 
Improved error handling in the {\hyperref[\detokenize{auth:module-tornado.auth}]{\sphinxcrossref{\sphinxcode{\sphinxupquote{tornado.auth}}}}} module, fixing hanging
requests when a network or other error occurs.

\end{itemize}


\subsection{What’s new in Tornado 5.1}
\label{\detokenize{releases/v5.1.0:what-s-new-in-tornado-5-1}}\label{\detokenize{releases/v5.1.0::doc}}

\subsubsection{July 12, 2018}
\label{\detokenize{releases/v5.1.0:july-12-2018}}

\paragraph{Deprecation notice}
\label{\detokenize{releases/v5.1.0:deprecation-notice}}\begin{itemize}
\item {} 
Tornado 6.0 will drop support for Python 2.7 and 3.4. The minimum
supported Python version will be 3.5.2.

\item {} 
The \sphinxcode{\sphinxupquote{tornado.stack\_context}} module is deprecated and will be removed
in Tornado 6.0. The reason for this is that it is not feasible to
provide this module’s semantics in the presence of \sphinxcode{\sphinxupquote{async def}}
native coroutines. \sphinxcode{\sphinxupquote{ExceptionStackContext}} is mainly obsolete
thanks to coroutines. \sphinxcode{\sphinxupquote{StackContext}} lacks a direct replacement
although the new \sphinxcode{\sphinxupquote{contextvars}} package (in the Python standard
library beginning in Python 3.7) may be an alternative.

\item {} 
Callback-oriented code often relies on \sphinxcode{\sphinxupquote{ExceptionStackContext}} to
handle errors and prevent leaked connections. In order to avoid the
risk of silently introducing subtle leaks (and to consolidate all of
Tornado’s interfaces behind the coroutine pattern), \sphinxcode{\sphinxupquote{callback}}
arguments throughout the package are deprecated and will be removed
in version 6.0. All functions that had a \sphinxcode{\sphinxupquote{callback}} argument
removed now return a {\hyperref[\detokenize{concurrent:tornado.concurrent.Future}]{\sphinxcrossref{\sphinxcode{\sphinxupquote{Future}}}}} which should be used instead.

\item {} 
Where possible, deprecation warnings are emitted when any of these
deprecated interfaces is used. However, Python does not display
deprecation warnings by default. To prepare your application for
Tornado 6.0, run Python with the \sphinxcode{\sphinxupquote{-Wd}} argument or set the
environment variable \sphinxcode{\sphinxupquote{PYTHONWARNINGS}} to \sphinxcode{\sphinxupquote{d}}. If your
application runs on Python 3 without deprecation warnings, it should
be able to move to Tornado 6.0 without disruption.

\end{itemize}


\paragraph{\sphinxstyleliteralintitle{\sphinxupquote{tornado.auth}}}
\label{\detokenize{releases/v5.1.0:tornado-auth}}\begin{itemize}
\item {} 
{\hyperref[\detokenize{auth:tornado.auth.OAuthMixin._oauth_get_user_future}]{\sphinxcrossref{\sphinxcode{\sphinxupquote{OAuthMixin.\_oauth\_get\_user\_future}}}}} may now be a native coroutine.

\item {} 
All \sphinxcode{\sphinxupquote{callback}} arguments in this package are deprecated and will
be removed in 6.0. Use the coroutine interfaces instead.

\item {} 
The \sphinxcode{\sphinxupquote{OAuthMixin.\_oauth\_get\_user}} method is deprecated and will be removed in
6.0. Override {\hyperref[\detokenize{auth:tornado.auth.OAuthMixin._oauth_get_user_future}]{\sphinxcrossref{\sphinxcode{\sphinxupquote{\_oauth\_get\_user\_future}}}}} instead.

\end{itemize}


\paragraph{\sphinxstyleliteralintitle{\sphinxupquote{tornado.autoreload}}}
\label{\detokenize{releases/v5.1.0:tornado-autoreload}}\begin{itemize}
\item {} 
The command-line autoreload wrapper is now preserved if an internal
autoreload fires.

\item {} 
The command-line wrapper no longer starts duplicated processes on windows
when combined with internal autoreload.

\end{itemize}


\paragraph{\sphinxstyleliteralintitle{\sphinxupquote{tornado.concurrent}}}
\label{\detokenize{releases/v5.1.0:tornado-concurrent}}\begin{itemize}
\item {} 
{\hyperref[\detokenize{concurrent:tornado.concurrent.run_on_executor}]{\sphinxcrossref{\sphinxcode{\sphinxupquote{run\_on\_executor}}}}} now returns {\hyperref[\detokenize{concurrent:tornado.concurrent.Future}]{\sphinxcrossref{\sphinxcode{\sphinxupquote{Future}}}}} objects that are compatible
with \sphinxcode{\sphinxupquote{await}}.

\item {} 
The \sphinxcode{\sphinxupquote{callback}} argument to {\hyperref[\detokenize{concurrent:tornado.concurrent.run_on_executor}]{\sphinxcrossref{\sphinxcode{\sphinxupquote{run\_on\_executor}}}}} is deprecated and will
be removed in 6.0.

\item {} 
\sphinxcode{\sphinxupquote{return\_future}} is deprecated and will be removed in 6.0.

\end{itemize}


\paragraph{\sphinxstyleliteralintitle{\sphinxupquote{tornado.gen}}}
\label{\detokenize{releases/v5.1.0:tornado-gen}}\begin{itemize}
\item {} 
Some older portions of this module are deprecated and will be removed
in 6.0. This includes \sphinxcode{\sphinxupquote{engine}}, \sphinxcode{\sphinxupquote{YieldPoint}}, \sphinxcode{\sphinxupquote{Callback}},
\sphinxcode{\sphinxupquote{Wait}}, \sphinxcode{\sphinxupquote{WaitAll}}, \sphinxcode{\sphinxupquote{MultiYieldPoint}}, and \sphinxcode{\sphinxupquote{Task}}.

\item {} 
Functions decorated with \sphinxcode{\sphinxupquote{@gen.coroutine}} will no longer accept
\sphinxcode{\sphinxupquote{callback}} arguments in 6.0.

\end{itemize}


\paragraph{\sphinxstyleliteralintitle{\sphinxupquote{tornado.httpclient}}}
\label{\detokenize{releases/v5.1.0:tornado-httpclient}}\begin{itemize}
\item {} 
The behavior of \sphinxcode{\sphinxupquote{raise\_error=False}} is changing in 6.0. Currently
it suppresses all errors; in 6.0 it will only suppress the errors
raised due to completed responses with non-200 status codes.

\item {} 
The \sphinxcode{\sphinxupquote{callback}} argument to {\hyperref[\detokenize{httpclient:tornado.httpclient.AsyncHTTPClient.fetch}]{\sphinxcrossref{\sphinxcode{\sphinxupquote{AsyncHTTPClient.fetch}}}}} is deprecated
and will be removed in 6.0.

\item {} 
{\hyperref[\detokenize{httpclient:tornado.httpclient.HTTPError}]{\sphinxcrossref{\sphinxcode{\sphinxupquote{tornado.httpclient.HTTPError}}}}} has been renamed to
{\hyperref[\detokenize{httpclient:tornado.httpclient.HTTPClientError}]{\sphinxcrossref{\sphinxcode{\sphinxupquote{HTTPClientError}}}}} to avoid ambiguity in code that also has to deal
with {\hyperref[\detokenize{web:tornado.web.HTTPError}]{\sphinxcrossref{\sphinxcode{\sphinxupquote{tornado.web.HTTPError}}}}}. The old name remains as an alias.

\item {} 
\sphinxcode{\sphinxupquote{tornado.curl\_httpclient}} now supports non-ASCII characters in
username and password arguments.

\item {} 
\sphinxcode{\sphinxupquote{.HTTPResponse.request\_time}} now behaves consistently across
\sphinxcode{\sphinxupquote{simple\_httpclient}} and \sphinxcode{\sphinxupquote{curl\_httpclient}}, excluding time spent
in the \sphinxcode{\sphinxupquote{max\_clients}} queue in both cases (previously this time was
included in \sphinxcode{\sphinxupquote{simple\_httpclient}} but excluded in
\sphinxcode{\sphinxupquote{curl\_httpclient}}). In both cases the time is now computed using
a monotonic clock where available.

\item {} 
{\hyperref[\detokenize{httpclient:tornado.httpclient.HTTPResponse}]{\sphinxcrossref{\sphinxcode{\sphinxupquote{HTTPResponse}}}}} now has a \sphinxcode{\sphinxupquote{start\_time}} attribute recording a
wall-clock (\sphinxhref{https://docs.python.org/3.6/library/time.html\#time.time}{\sphinxcode{\sphinxupquote{time.time}}}) timestamp at which the request started
(after leaving the \sphinxcode{\sphinxupquote{max\_clients}} queue if applicable).

\end{itemize}


\paragraph{\sphinxstyleliteralintitle{\sphinxupquote{tornado.httputil}}}
\label{\detokenize{releases/v5.1.0:tornado-httputil}}\begin{itemize}
\item {} 
{\hyperref[\detokenize{httputil:tornado.httputil.parse_multipart_form_data}]{\sphinxcrossref{\sphinxcode{\sphinxupquote{parse\_multipart\_form\_data}}}}} now recognizes non-ASCII filenames in
RFC 2231/5987 (\sphinxcode{\sphinxupquote{filename*=}}) format.

\item {} 
\sphinxcode{\sphinxupquote{HTTPServerRequest.write}} is deprecated and will be removed in 6.0. Use
the methods of \sphinxcode{\sphinxupquote{request.connection}} instead.

\item {} 
Malformed HTTP headers are now logged less noisily.

\end{itemize}


\paragraph{\sphinxstyleliteralintitle{\sphinxupquote{tornado.ioloop}}}
\label{\detokenize{releases/v5.1.0:tornado-ioloop}}\begin{itemize}
\item {} 
{\hyperref[\detokenize{ioloop:tornado.ioloop.PeriodicCallback}]{\sphinxcrossref{\sphinxcode{\sphinxupquote{PeriodicCallback}}}}} now supports a \sphinxcode{\sphinxupquote{jitter}} argument to randomly
vary the timeout.

\item {} 
\sphinxcode{\sphinxupquote{IOLoop.set\_blocking\_signal\_threshold}},
\sphinxcode{\sphinxupquote{IOLoop.set\_blocking\_log\_threshold}}, \sphinxcode{\sphinxupquote{IOLoop.log\_stack}},
and \sphinxcode{\sphinxupquote{IOLoop.handle\_callback\_exception}} are deprecated and will
be removed in 6.0.

\item {} 
Fixed a \sphinxhref{https://docs.python.org/3.6/library/exceptions.html\#KeyError}{\sphinxcode{\sphinxupquote{KeyError}}} in {\hyperref[\detokenize{ioloop:tornado.ioloop.IOLoop.close}]{\sphinxcrossref{\sphinxcode{\sphinxupquote{IOLoop.close}}}}} when {\hyperref[\detokenize{ioloop:tornado.ioloop.IOLoop}]{\sphinxcrossref{\sphinxcode{\sphinxupquote{IOLoop}}}}} objects are
being opened and closed in multiple threads.

\end{itemize}


\paragraph{\sphinxstyleliteralintitle{\sphinxupquote{tornado.iostream}}}
\label{\detokenize{releases/v5.1.0:tornado-iostream}}\begin{itemize}
\item {} 
All \sphinxcode{\sphinxupquote{callback}} arguments in this module are deprecated except for
{\hyperref[\detokenize{iostream:tornado.iostream.BaseIOStream.set_close_callback}]{\sphinxcrossref{\sphinxcode{\sphinxupquote{BaseIOStream.set\_close\_callback}}}}}. They will be removed in 6.0.

\item {} 
\sphinxcode{\sphinxupquote{streaming\_callback}} arguments to {\hyperref[\detokenize{iostream:tornado.iostream.BaseIOStream.read_bytes}]{\sphinxcrossref{\sphinxcode{\sphinxupquote{BaseIOStream.read\_bytes}}}}} and
{\hyperref[\detokenize{iostream:tornado.iostream.BaseIOStream.read_until_close}]{\sphinxcrossref{\sphinxcode{\sphinxupquote{BaseIOStream.read\_until\_close}}}}} are deprecated and will be removed
in 6.0.

\end{itemize}


\paragraph{\sphinxstyleliteralintitle{\sphinxupquote{tornado.netutil}}}
\label{\detokenize{releases/v5.1.0:tornado-netutil}}\begin{itemize}
\item {} 
Improved compatibility with GNU Hurd.

\end{itemize}


\paragraph{\sphinxstyleliteralintitle{\sphinxupquote{tornado.options}}}
\label{\detokenize{releases/v5.1.0:tornado-options}}\begin{itemize}
\item {} 
{\hyperref[\detokenize{options:tornado.options.parse_config_file}]{\sphinxcrossref{\sphinxcode{\sphinxupquote{tornado.options.parse\_config\_file}}}}} now allows setting options to
strings (which will be parsed the same way as
{\hyperref[\detokenize{options:tornado.options.parse_command_line}]{\sphinxcrossref{\sphinxcode{\sphinxupquote{tornado.options.parse\_command\_line}}}}}) in addition to the specified
type for the option.

\end{itemize}


\paragraph{\sphinxstyleliteralintitle{\sphinxupquote{tornado.platform.twisted}}}
\label{\detokenize{releases/v5.1.0:tornado-platform-twisted}}\begin{itemize}
\item {} 
\sphinxcode{\sphinxupquote{TornadoReactor}} and \sphinxcode{\sphinxupquote{TwistedIOLoop}} are deprecated and will be
removed in 6.0. Instead, Tornado will always use the asyncio event loop
and twisted can be configured to do so as well.

\end{itemize}


\paragraph{\sphinxstyleliteralintitle{\sphinxupquote{tornado.stack\_context}}}
\label{\detokenize{releases/v5.1.0:tornado-stack-context}}\begin{itemize}
\item {} 
The \sphinxcode{\sphinxupquote{tornado.stack\_context}} module is deprecated and will be removed
in 6.0.

\end{itemize}


\paragraph{\sphinxstyleliteralintitle{\sphinxupquote{tornado.testing}}}
\label{\detokenize{releases/v5.1.0:tornado-testing}}\begin{itemize}
\item {} 
{\hyperref[\detokenize{testing:tornado.testing.AsyncHTTPTestCase.fetch}]{\sphinxcrossref{\sphinxcode{\sphinxupquote{AsyncHTTPTestCase.fetch}}}}} now takes a \sphinxcode{\sphinxupquote{raise\_error}} argument.
This argument has the same semantics as {\hyperref[\detokenize{httpclient:tornado.httpclient.AsyncHTTPClient.fetch}]{\sphinxcrossref{\sphinxcode{\sphinxupquote{AsyncHTTPClient.fetch}}}}},
but defaults to false because tests often need to deal with non-200
responses (and for backwards-compatibility).

\item {} 
The {\hyperref[\detokenize{testing:tornado.testing.AsyncTestCase.stop}]{\sphinxcrossref{\sphinxcode{\sphinxupquote{AsyncTestCase.stop}}}}} and {\hyperref[\detokenize{testing:tornado.testing.AsyncTestCase.wait}]{\sphinxcrossref{\sphinxcode{\sphinxupquote{AsyncTestCase.wait}}}}} methods are
deprecated.

\end{itemize}


\paragraph{\sphinxstyleliteralintitle{\sphinxupquote{tornado.web}}}
\label{\detokenize{releases/v5.1.0:tornado-web}}\begin{itemize}
\item {} 
New method {\hyperref[\detokenize{web:tornado.web.RequestHandler.detach}]{\sphinxcrossref{\sphinxcode{\sphinxupquote{RequestHandler.detach}}}}} can be used from methods
that are not decorated with \sphinxcode{\sphinxupquote{@asynchronous}} (the decorator
was required to use \sphinxcode{\sphinxupquote{self.request.connection.detach()}}.

\item {} 
{\hyperref[\detokenize{web:tornado.web.RequestHandler.finish}]{\sphinxcrossref{\sphinxcode{\sphinxupquote{RequestHandler.finish}}}}} and {\hyperref[\detokenize{web:tornado.web.RequestHandler.render}]{\sphinxcrossref{\sphinxcode{\sphinxupquote{RequestHandler.render}}}}} now return
\sphinxcode{\sphinxupquote{Futures}} that can be used to wait for the last part of the
response to be sent to the client.

\item {} 
{\hyperref[\detokenize{web:tornado.web.FallbackHandler}]{\sphinxcrossref{\sphinxcode{\sphinxupquote{FallbackHandler}}}}} now calls \sphinxcode{\sphinxupquote{on\_finish}} for the benefit of
subclasses that may have overridden it.

\item {} 
The \sphinxcode{\sphinxupquote{asynchronous}} decorator is deprecated and will be removed in 6.0.

\item {} 
The \sphinxcode{\sphinxupquote{callback}} argument to {\hyperref[\detokenize{web:tornado.web.RequestHandler.flush}]{\sphinxcrossref{\sphinxcode{\sphinxupquote{RequestHandler.flush}}}}} is deprecated
and will be removed in 6.0.

\end{itemize}


\paragraph{\sphinxstyleliteralintitle{\sphinxupquote{tornado.websocket}}}
\label{\detokenize{releases/v5.1.0:tornado-websocket}}\begin{itemize}
\item {} 
When compression is enabled, memory limits now apply to the
post-decompression size of the data, protecting against DoS attacks.

\item {} 
{\hyperref[\detokenize{websocket:tornado.websocket.websocket_connect}]{\sphinxcrossref{\sphinxcode{\sphinxupquote{websocket\_connect}}}}} now supports subprotocols.

\item {} 
{\hyperref[\detokenize{websocket:tornado.websocket.WebSocketHandler}]{\sphinxcrossref{\sphinxcode{\sphinxupquote{WebSocketHandler}}}}} and {\hyperref[\detokenize{websocket:tornado.websocket.WebSocketClientConnection}]{\sphinxcrossref{\sphinxcode{\sphinxupquote{WebSocketClientConnection}}}}} now have
\sphinxcode{\sphinxupquote{selected\_subprotocol}} attributes to see the subprotocol in use.

\item {} 
The {\hyperref[\detokenize{websocket:tornado.websocket.WebSocketHandler.select_subprotocol}]{\sphinxcrossref{\sphinxcode{\sphinxupquote{WebSocketHandler.select\_subprotocol}}}}} method is now called with
an empty list instead of a list containing an empty string if no
subprotocols were requested by the client.

\item {} 
{\hyperref[\detokenize{websocket:tornado.websocket.WebSocketHandler.open}]{\sphinxcrossref{\sphinxcode{\sphinxupquote{WebSocketHandler.open}}}}} may now be a coroutine.

\item {} 
The \sphinxcode{\sphinxupquote{data}} argument to {\hyperref[\detokenize{websocket:tornado.websocket.WebSocketHandler.ping}]{\sphinxcrossref{\sphinxcode{\sphinxupquote{WebSocketHandler.ping}}}}} is now optional.

\item {} 
Client-side websocket connections no longer buffer more than one
message in memory at a time.

\item {} 
Exception logging now uses {\hyperref[\detokenize{web:tornado.web.RequestHandler.log_exception}]{\sphinxcrossref{\sphinxcode{\sphinxupquote{RequestHandler.log\_exception}}}}}.

\end{itemize}


\paragraph{\sphinxstyleliteralintitle{\sphinxupquote{tornado.wsgi}}}
\label{\detokenize{releases/v5.1.0:tornado-wsgi}}\begin{itemize}
\item {} 
\sphinxcode{\sphinxupquote{WSGIApplication}} and \sphinxcode{\sphinxupquote{WSGIAdapter}} are deprecated and will be removed
in Tornado 6.0.

\end{itemize}


\subsection{What’s new in Tornado 5.0.2}
\label{\detokenize{releases/v5.0.2:what-s-new-in-tornado-5-0-2}}\label{\detokenize{releases/v5.0.2::doc}}

\subsubsection{Apr 7, 2018}
\label{\detokenize{releases/v5.0.2:apr-7-2018}}

\paragraph{Bug fixes}
\label{\detokenize{releases/v5.0.2:bug-fixes}}\begin{itemize}
\item {} 
Fixed a memory leak when {\hyperref[\detokenize{ioloop:tornado.ioloop.IOLoop}]{\sphinxcrossref{\sphinxcode{\sphinxupquote{IOLoop}}}}} objects are created and destroyed.

\item {} 
If {\hyperref[\detokenize{testing:tornado.testing.AsyncTestCase.get_new_ioloop}]{\sphinxcrossref{\sphinxcode{\sphinxupquote{AsyncTestCase.get\_new\_ioloop}}}}} returns a reference to a
preexisting event loop (typically when it has been overridden to
return {\hyperref[\detokenize{ioloop:tornado.ioloop.IOLoop.current}]{\sphinxcrossref{\sphinxcode{\sphinxupquote{IOLoop.current()}}}}}), the test’s \sphinxcode{\sphinxupquote{tearDown}} method will not
close this loop.

\item {} 
Fixed a confusing error message when the synchronous {\hyperref[\detokenize{httpclient:tornado.httpclient.HTTPClient}]{\sphinxcrossref{\sphinxcode{\sphinxupquote{HTTPClient}}}}}
fails to initialize because an event loop is already running.

\item {} 
{\hyperref[\detokenize{ioloop:tornado.ioloop.PeriodicCallback}]{\sphinxcrossref{\sphinxcode{\sphinxupquote{PeriodicCallback}}}}} no longer executes twice in a row due to
backwards clock adjustments.

\end{itemize}


\subsection{What’s new in Tornado 5.0.1}
\label{\detokenize{releases/v5.0.1:what-s-new-in-tornado-5-0-1}}\label{\detokenize{releases/v5.0.1::doc}}

\subsubsection{Mar 18, 2018}
\label{\detokenize{releases/v5.0.1:mar-18-2018}}

\paragraph{Bug fix}
\label{\detokenize{releases/v5.0.1:bug-fix}}\begin{itemize}
\item {} 
This release restores support for versions of Python 3.4 prior to
3.4.4. This is important for compatibility with Debian Jessie which
has 3.4.2 as its version of Python 3.

\end{itemize}


\subsection{What’s new in Tornado 5.0}
\label{\detokenize{releases/v5.0.0:what-s-new-in-tornado-5-0}}\label{\detokenize{releases/v5.0.0::doc}}

\subsubsection{Mar 5, 2018}
\label{\detokenize{releases/v5.0.0:mar-5-2018}}

\paragraph{Highlights}
\label{\detokenize{releases/v5.0.0:highlights}}\begin{itemize}
\item {} 
The focus of this release is improving integration with \sphinxhref{https://docs.python.org/3.6/library/asyncio.html\#module-asyncio}{\sphinxcode{\sphinxupquote{asyncio}}}.
On Python 3, the {\hyperref[\detokenize{ioloop:tornado.ioloop.IOLoop}]{\sphinxcrossref{\sphinxcode{\sphinxupquote{IOLoop}}}}} is always a wrapper around the \sphinxhref{https://docs.python.org/3.6/library/asyncio.html\#module-asyncio}{\sphinxcode{\sphinxupquote{asyncio}}}
event loop, and \sphinxhref{https://docs.python.org/3.6/library/asyncio-task.html\#asyncio.Future}{\sphinxcode{\sphinxupquote{asyncio.Future}}} and \sphinxhref{https://docs.python.org/3.6/library/asyncio-task.html\#asyncio.Task}{\sphinxcode{\sphinxupquote{asyncio.Task}}} are used instead
of their Tornado counterparts. This means that libraries based on
\sphinxhref{https://docs.python.org/3.6/library/asyncio.html\#module-asyncio}{\sphinxcode{\sphinxupquote{asyncio}}} can be mixed relatively seamlessly with those using
Tornado. While care has been taken to minimize the disruption from
this change, code changes may be required for compatibility with
Tornado 5.0, as detailed in the following section.

\item {} 
Tornado 5.0 supports Python 2.7.9+ and 3.4+. Python 2.7 and 3.4 are
deprecated and support for them will be removed in Tornado 6.0,
which will require Python 3.5+.

\end{itemize}


\paragraph{Backwards-compatibility notes}
\label{\detokenize{releases/v5.0.0:backwards-compatibility-notes}}\begin{itemize}
\item {} 
Python 3.3 is no longer supported.

\item {} 
Versions of Python 2.7 that predate the \sphinxhref{https://docs.python.org/3.6/library/ssl.html\#module-ssl}{\sphinxcode{\sphinxupquote{ssl}}} module update are no
longer supported. (The \sphinxhref{https://docs.python.org/3.6/library/ssl.html\#module-ssl}{\sphinxcode{\sphinxupquote{ssl}}} module was updated in version 2.7.9,
although in some distributions the updates are present in builds
with a lower version number. Tornado requires \sphinxhref{https://docs.python.org/3.6/library/ssl.html\#ssl.SSLContext}{\sphinxcode{\sphinxupquote{ssl.SSLContext}}},
\sphinxhref{https://docs.python.org/3.6/library/ssl.html\#ssl.create\_default\_context}{\sphinxcode{\sphinxupquote{ssl.create\_default\_context}}}, and \sphinxhref{https://docs.python.org/3.6/library/ssl.html\#ssl.match\_hostname}{\sphinxcode{\sphinxupquote{ssl.match\_hostname}}})

\item {} 
Versions of Python 3.5 prior to 3.5.2 are no longer supported due to
a change in the async iterator protocol in that version.

\item {} 
The \sphinxcode{\sphinxupquote{trollius}} project (\sphinxhref{https://docs.python.org/3.6/library/asyncio.html\#module-asyncio}{\sphinxcode{\sphinxupquote{asyncio}}} backported to Python 2) is no
longer supported.

\item {} 
{\hyperref[\detokenize{concurrent:tornado.concurrent.Future}]{\sphinxcrossref{\sphinxcode{\sphinxupquote{tornado.concurrent.Future}}}}} is now an alias for \sphinxhref{https://docs.python.org/3.6/library/asyncio-task.html\#asyncio.Future}{\sphinxcode{\sphinxupquote{asyncio.Future}}}
when running on Python 3. This results in a number of minor
behavioral changes:
\begin{itemize}
\item {} 
{\hyperref[\detokenize{concurrent:tornado.concurrent.Future}]{\sphinxcrossref{\sphinxcode{\sphinxupquote{Future}}}}} objects can only be created while there is a current
{\hyperref[\detokenize{ioloop:tornado.ioloop.IOLoop}]{\sphinxcrossref{\sphinxcode{\sphinxupquote{IOLoop}}}}}

\item {} 
The timing of callbacks scheduled with
\sphinxcode{\sphinxupquote{Future.add\_done\_callback}} has changed.
{\hyperref[\detokenize{concurrent:tornado.concurrent.future_add_done_callback}]{\sphinxcrossref{\sphinxcode{\sphinxupquote{tornado.concurrent.future\_add\_done\_callback}}}}} can be used to
make the behavior more like older versions of Tornado (but not
identical). Some of these changes are also present in the Python
2 version of {\hyperref[\detokenize{concurrent:tornado.concurrent.Future}]{\sphinxcrossref{\sphinxcode{\sphinxupquote{tornado.concurrent.Future}}}}} to minimize the
difference between Python 2 and 3.

\item {} 
Cancellation is now partially supported, via
\sphinxhref{https://docs.python.org/3.6/library/asyncio-task.html\#asyncio.Future.cancel}{\sphinxcode{\sphinxupquote{asyncio.Future.cancel}}}. A canceled {\hyperref[\detokenize{concurrent:tornado.concurrent.Future}]{\sphinxcrossref{\sphinxcode{\sphinxupquote{Future}}}}} can no longer have
its result set. Applications that handle \sphinxhref{https://docs.python.org/3.6/library/asyncio-task.html\#asyncio.Future}{\sphinxcode{\sphinxupquote{Future}}}
objects directly may want to use
{\hyperref[\detokenize{concurrent:tornado.concurrent.future_set_result_unless_cancelled}]{\sphinxcrossref{\sphinxcode{\sphinxupquote{tornado.concurrent.future\_set\_result\_unless\_cancelled}}}}}. In
native coroutines, cancellation will cause an exception to be
raised in the coroutine.

\item {} 
The \sphinxcode{\sphinxupquote{exc\_info}} and \sphinxcode{\sphinxupquote{set\_exc\_info}} methods are no longer
present. Use {\hyperref[\detokenize{concurrent:tornado.concurrent.future_set_exc_info}]{\sphinxcrossref{\sphinxcode{\sphinxupquote{tornado.concurrent.future\_set\_exc\_info}}}}} to replace
the latter, and raise the exception with
\sphinxhref{https://docs.python.org/3.6/library/asyncio-task.html\#asyncio.Future.result}{\sphinxcode{\sphinxupquote{result}}} to replace the former.

\end{itemize}

\item {} 
\sphinxcode{\sphinxupquote{io\_loop}} arguments to many Tornado functions have been removed.
Use {\hyperref[\detokenize{ioloop:tornado.ioloop.IOLoop.current}]{\sphinxcrossref{\sphinxcode{\sphinxupquote{IOLoop.current()}}}}} instead of passing {\hyperref[\detokenize{ioloop:tornado.ioloop.IOLoop}]{\sphinxcrossref{\sphinxcode{\sphinxupquote{IOLoop}}}}} objects
explicitly.

\item {} 
On Python 3, {\hyperref[\detokenize{ioloop:tornado.ioloop.IOLoop}]{\sphinxcrossref{\sphinxcode{\sphinxupquote{IOLoop}}}}} is always a wrapper around the \sphinxhref{https://docs.python.org/3.6/library/asyncio.html\#module-asyncio}{\sphinxcode{\sphinxupquote{asyncio}}}
event loop. \sphinxcode{\sphinxupquote{IOLoop.configure}} is effectively removed on Python 3
(for compatibility, it may be called to redundantly specify the
\sphinxhref{https://docs.python.org/3.6/library/asyncio.html\#module-asyncio}{\sphinxcode{\sphinxupquote{asyncio}}}-backed {\hyperref[\detokenize{ioloop:tornado.ioloop.IOLoop}]{\sphinxcrossref{\sphinxcode{\sphinxupquote{IOLoop}}}}})

\item {} 
{\hyperref[\detokenize{ioloop:tornado.ioloop.IOLoop.instance}]{\sphinxcrossref{\sphinxcode{\sphinxupquote{IOLoop.instance}}}}} is now a deprecated alias for {\hyperref[\detokenize{ioloop:tornado.ioloop.IOLoop.current}]{\sphinxcrossref{\sphinxcode{\sphinxupquote{IOLoop.current}}}}}.
Applications that need the cross-thread communication behavior
facilitated by {\hyperref[\detokenize{ioloop:tornado.ioloop.IOLoop.instance}]{\sphinxcrossref{\sphinxcode{\sphinxupquote{IOLoop.instance}}}}} should use their own global variable
instead.

\end{itemize}


\paragraph{Other notes}
\label{\detokenize{releases/v5.0.0:other-notes}}\begin{itemize}
\item {} 
The \sphinxcode{\sphinxupquote{futures}} (\sphinxhref{https://docs.python.org/3.6/library/concurrent.futures.html\#module-concurrent.futures}{\sphinxcode{\sphinxupquote{concurrent.futures}}} backport) package is now required
on Python 2.7.

\item {} 
The \sphinxcode{\sphinxupquote{certifi}} and \sphinxcode{\sphinxupquote{backports.ssl-match-hostname}} packages are no
longer required on Python 2.7.

\item {} 
Python 3.6 or higher is recommended, because it features more
efficient garbage collection of \sphinxhref{https://docs.python.org/3.6/library/asyncio-task.html\#asyncio.Future}{\sphinxcode{\sphinxupquote{asyncio.Future}}} objects.

\end{itemize}


\paragraph{\sphinxstyleliteralintitle{\sphinxupquote{tornado.auth}}}
\label{\detokenize{releases/v5.0.0:tornado-auth}}\begin{itemize}
\item {} 
{\hyperref[\detokenize{auth:tornado.auth.GoogleOAuth2Mixin}]{\sphinxcrossref{\sphinxcode{\sphinxupquote{GoogleOAuth2Mixin}}}}} now uses a newer set of URLs.

\end{itemize}


\paragraph{\sphinxstyleliteralintitle{\sphinxupquote{tornado.autoreload}}}
\label{\detokenize{releases/v5.0.0:tornado-autoreload}}\begin{itemize}
\item {} 
On Python 3, uses \sphinxcode{\sphinxupquote{\_\_main\_\_.\_\_spec}} to more reliably reconstruct
the original command line and avoid modifying \sphinxcode{\sphinxupquote{PYTHONPATH}}.

\item {} 
The \sphinxcode{\sphinxupquote{io\_loop}} argument to {\hyperref[\detokenize{autoreload:tornado.autoreload.start}]{\sphinxcrossref{\sphinxcode{\sphinxupquote{tornado.autoreload.start}}}}} has been removed.

\end{itemize}


\paragraph{\sphinxstyleliteralintitle{\sphinxupquote{tornado.concurrent}}}
\label{\detokenize{releases/v5.0.0:tornado-concurrent}}\begin{itemize}
\item {} 
{\hyperref[\detokenize{concurrent:tornado.concurrent.Future}]{\sphinxcrossref{\sphinxcode{\sphinxupquote{tornado.concurrent.Future}}}}} is now an alias for \sphinxhref{https://docs.python.org/3.6/library/asyncio-task.html\#asyncio.Future}{\sphinxcode{\sphinxupquote{asyncio.Future}}}
when running on Python 3. See “Backwards-compatibility notes” for
more.

\item {} 
Setting the result of a \sphinxcode{\sphinxupquote{Future}} no longer blocks while callbacks
are being run. Instead, the callbacks are scheduled on the next
{\hyperref[\detokenize{ioloop:tornado.ioloop.IOLoop}]{\sphinxcrossref{\sphinxcode{\sphinxupquote{IOLoop}}}}} iteration.

\item {} 
The deprecated alias \sphinxcode{\sphinxupquote{tornado.concurrent.TracebackFuture}} has been
removed.

\item {} 
{\hyperref[\detokenize{concurrent:tornado.concurrent.chain_future}]{\sphinxcrossref{\sphinxcode{\sphinxupquote{tornado.concurrent.chain\_future}}}}} now works with all three kinds of
\sphinxcode{\sphinxupquote{Futures}} (Tornado, \sphinxhref{https://docs.python.org/3.6/library/asyncio.html\#module-asyncio}{\sphinxcode{\sphinxupquote{asyncio}}}, and \sphinxhref{https://docs.python.org/3.6/library/concurrent.futures.html\#module-concurrent.futures}{\sphinxcode{\sphinxupquote{concurrent.futures}}})

\item {} 
The \sphinxcode{\sphinxupquote{io\_loop}} argument to {\hyperref[\detokenize{concurrent:tornado.concurrent.run_on_executor}]{\sphinxcrossref{\sphinxcode{\sphinxupquote{tornado.concurrent.run\_on\_executor}}}}} has
been removed.

\item {} 
New functions {\hyperref[\detokenize{concurrent:tornado.concurrent.future_set_result_unless_cancelled}]{\sphinxcrossref{\sphinxcode{\sphinxupquote{future\_set\_result\_unless\_cancelled}}}}},
{\hyperref[\detokenize{concurrent:tornado.concurrent.future_set_exc_info}]{\sphinxcrossref{\sphinxcode{\sphinxupquote{future\_set\_exc\_info}}}}}, and {\hyperref[\detokenize{concurrent:tornado.concurrent.future_add_done_callback}]{\sphinxcrossref{\sphinxcode{\sphinxupquote{future\_add\_done\_callback}}}}} help mask
the difference between \sphinxhref{https://docs.python.org/3.6/library/asyncio-task.html\#asyncio.Future}{\sphinxcode{\sphinxupquote{asyncio.Future}}} and Tornado’s previous
\sphinxcode{\sphinxupquote{Future}} implementation.

\end{itemize}


\paragraph{\sphinxstyleliteralintitle{\sphinxupquote{tornado.curl\_httpclient}}}
\label{\detokenize{releases/v5.0.0:tornado-curl-httpclient}}\begin{itemize}
\item {} 
Improved debug logging on Python 3.

\item {} 
The \sphinxcode{\sphinxupquote{time\_info}} response attribute now includes \sphinxcode{\sphinxupquote{appconnect}} in
addition to other measurements.

\item {} 
Closing a {\hyperref[\detokenize{httpclient:tornado.curl_httpclient.CurlAsyncHTTPClient}]{\sphinxcrossref{\sphinxcode{\sphinxupquote{CurlAsyncHTTPClient}}}}} now breaks circular references that
could delay garbage collection.

\item {} 
The \sphinxcode{\sphinxupquote{io\_loop}} argument to the {\hyperref[\detokenize{httpclient:tornado.curl_httpclient.CurlAsyncHTTPClient}]{\sphinxcrossref{\sphinxcode{\sphinxupquote{CurlAsyncHTTPClient}}}}} constructor
has been removed.

\end{itemize}


\paragraph{\sphinxstyleliteralintitle{\sphinxupquote{tornado.gen}}}
\label{\detokenize{releases/v5.0.0:tornado-gen}}\begin{itemize}
\item {} 
\sphinxcode{\sphinxupquote{tornado.gen.TimeoutError}} is now an alias for
{\hyperref[\detokenize{util:tornado.util.TimeoutError}]{\sphinxcrossref{\sphinxcode{\sphinxupquote{tornado.util.TimeoutError}}}}}.

\item {} 
Leak detection for \sphinxcode{\sphinxupquote{Futures}} created by this module now attributes
them to their proper caller instead of the coroutine machinery.

\item {} 
Several circular references that could delay garbage collection have
been broken up.

\item {} 
On Python 3, \sphinxhref{https://docs.python.org/3.6/library/asyncio-task.html\#asyncio.Task}{\sphinxcode{\sphinxupquote{asyncio.Task}}} is used instead of the Tornado coroutine
runner. This improves compatibility with some \sphinxhref{https://docs.python.org/3.6/library/asyncio.html\#module-asyncio}{\sphinxcode{\sphinxupquote{asyncio}}} libraries
and adds support for cancellation.

\item {} 
The \sphinxcode{\sphinxupquote{io\_loop}} arguments to \sphinxcode{\sphinxupquote{YieldFuture}} and {\hyperref[\detokenize{gen:tornado.gen.with_timeout}]{\sphinxcrossref{\sphinxcode{\sphinxupquote{with\_timeout}}}}} have
been removed.

\end{itemize}


\paragraph{\sphinxstyleliteralintitle{\sphinxupquote{tornado.httpclient}}}
\label{\detokenize{releases/v5.0.0:tornado-httpclient}}\begin{itemize}
\item {} 
The \sphinxcode{\sphinxupquote{io\_loop}} argument to all {\hyperref[\detokenize{httpclient:tornado.httpclient.AsyncHTTPClient}]{\sphinxcrossref{\sphinxcode{\sphinxupquote{AsyncHTTPClient}}}}} constructors has
been removed.

\end{itemize}


\paragraph{\sphinxstyleliteralintitle{\sphinxupquote{tornado.httpserver}}}
\label{\detokenize{releases/v5.0.0:tornado-httpserver}}\begin{itemize}
\item {} 
It is now possible for a client to reuse a connection after sending
a chunked request.

\item {} 
If a client sends a malformed request, the server now responds with
a 400 error instead of simply closing the connection.

\item {} 
\sphinxcode{\sphinxupquote{Content-Length}} and \sphinxcode{\sphinxupquote{Transfer-Encoding}} headers are no longer
sent with 1xx or 204 responses (this was already true of 304
responses).

\item {} 
When closing a connection to a HTTP/1.1 client, the \sphinxcode{\sphinxupquote{Connection:
close}} header is sent with the response.

\item {} 
The \sphinxcode{\sphinxupquote{io\_loop}} argument to the {\hyperref[\detokenize{httpserver:tornado.httpserver.HTTPServer}]{\sphinxcrossref{\sphinxcode{\sphinxupquote{HTTPServer}}}}} constructor has been
removed.

\item {} 
If more than one \sphinxcode{\sphinxupquote{X-Scheme}} or \sphinxcode{\sphinxupquote{X-Forwarded-Proto}} header is
present, only the last is used.

\end{itemize}


\paragraph{\sphinxstyleliteralintitle{\sphinxupquote{tornado.httputil}}}
\label{\detokenize{releases/v5.0.0:tornado-httputil}}\begin{itemize}
\item {} 
The string representation of {\hyperref[\detokenize{httputil:tornado.httputil.HTTPServerRequest}]{\sphinxcrossref{\sphinxcode{\sphinxupquote{HTTPServerRequest}}}}} objects (which are
sometimes used in log messages) no longer includes the request
headers.

\item {} 
New function {\hyperref[\detokenize{httputil:tornado.httputil.qs_to_qsl}]{\sphinxcrossref{\sphinxcode{\sphinxupquote{qs\_to\_qsl}}}}} converts the result of
\sphinxhref{https://docs.python.org/3.6/library/urllib.parse.html\#urllib.parse.parse\_qs}{\sphinxcode{\sphinxupquote{urllib.parse.parse\_qs}}} to name-value pairs.

\end{itemize}


\paragraph{\sphinxstyleliteralintitle{\sphinxupquote{tornado.ioloop}}}
\label{\detokenize{releases/v5.0.0:tornado-ioloop}}\begin{itemize}
\item {} 
\sphinxcode{\sphinxupquote{tornado.ioloop.TimeoutError}} is now an alias for
{\hyperref[\detokenize{util:tornado.util.TimeoutError}]{\sphinxcrossref{\sphinxcode{\sphinxupquote{tornado.util.TimeoutError}}}}}.

\item {} 
{\hyperref[\detokenize{ioloop:tornado.ioloop.IOLoop.instance}]{\sphinxcrossref{\sphinxcode{\sphinxupquote{IOLoop.instance}}}}} is now a deprecated alias for {\hyperref[\detokenize{ioloop:tornado.ioloop.IOLoop.current}]{\sphinxcrossref{\sphinxcode{\sphinxupquote{IOLoop.current}}}}}.

\item {} 
{\hyperref[\detokenize{ioloop:tornado.ioloop.IOLoop.install}]{\sphinxcrossref{\sphinxcode{\sphinxupquote{IOLoop.install}}}}} and {\hyperref[\detokenize{ioloop:tornado.ioloop.IOLoop.clear_instance}]{\sphinxcrossref{\sphinxcode{\sphinxupquote{IOLoop.clear\_instance}}}}} are deprecated.

\item {} 
The \sphinxcode{\sphinxupquote{IOLoop.initialized}} method has been removed.

\item {} 
On Python 3, the \sphinxhref{https://docs.python.org/3.6/library/asyncio.html\#module-asyncio}{\sphinxcode{\sphinxupquote{asyncio}}}-backed {\hyperref[\detokenize{ioloop:tornado.ioloop.IOLoop}]{\sphinxcrossref{\sphinxcode{\sphinxupquote{IOLoop}}}}} is always used and
alternative {\hyperref[\detokenize{ioloop:tornado.ioloop.IOLoop}]{\sphinxcrossref{\sphinxcode{\sphinxupquote{IOLoop}}}}} implementations cannot be configured.
{\hyperref[\detokenize{ioloop:tornado.ioloop.IOLoop.current}]{\sphinxcrossref{\sphinxcode{\sphinxupquote{IOLoop.current}}}}} and related methods pass through to
\sphinxhref{https://docs.python.org/3.6/library/asyncio-eventloops.html\#asyncio.get\_event\_loop}{\sphinxcode{\sphinxupquote{asyncio.get\_event\_loop}}}.

\item {} 
{\hyperref[\detokenize{ioloop:tornado.ioloop.IOLoop.run_sync}]{\sphinxcrossref{\sphinxcode{\sphinxupquote{run\_sync}}}}} cancels its argument on a timeout. This
results in better stack traces (and avoids log messages about leaks)
in native coroutines.

\item {} 
New methods {\hyperref[\detokenize{ioloop:tornado.ioloop.IOLoop.run_in_executor}]{\sphinxcrossref{\sphinxcode{\sphinxupquote{IOLoop.run\_in\_executor}}}}} and
{\hyperref[\detokenize{ioloop:tornado.ioloop.IOLoop.set_default_executor}]{\sphinxcrossref{\sphinxcode{\sphinxupquote{IOLoop.set\_default\_executor}}}}} make it easier to run functions in
other threads from native coroutines (since
\sphinxhref{https://docs.python.org/3.6/library/concurrent.futures.html\#concurrent.futures.Future}{\sphinxcode{\sphinxupquote{concurrent.futures.Future}}} does not support \sphinxcode{\sphinxupquote{await}}).

\item {} 
\sphinxcode{\sphinxupquote{PollIOLoop}} (the default on Python 2) attempts to detect misuse
of {\hyperref[\detokenize{ioloop:tornado.ioloop.IOLoop}]{\sphinxcrossref{\sphinxcode{\sphinxupquote{IOLoop}}}}} instances across \sphinxhref{https://docs.python.org/3.6/library/os.html\#os.fork}{\sphinxcode{\sphinxupquote{os.fork}}}.

\item {} 
The \sphinxcode{\sphinxupquote{io\_loop}} argument to {\hyperref[\detokenize{ioloop:tornado.ioloop.PeriodicCallback}]{\sphinxcrossref{\sphinxcode{\sphinxupquote{PeriodicCallback}}}}} has been removed.

\item {} 
It is now possible to create a {\hyperref[\detokenize{ioloop:tornado.ioloop.PeriodicCallback}]{\sphinxcrossref{\sphinxcode{\sphinxupquote{PeriodicCallback}}}}} in one thread
and start it in another without passing an explicit event loop.

\item {} 
The \sphinxcode{\sphinxupquote{IOLoop.set\_blocking\_signal\_threshold}} and
\sphinxcode{\sphinxupquote{IOLoop.set\_blocking\_log\_threshold}} methods are deprecated because
they are not implemented for the \sphinxhref{https://docs.python.org/3.6/library/asyncio.html\#module-asyncio}{\sphinxcode{\sphinxupquote{asyncio}}} event loop{}`. Use the
\sphinxcode{\sphinxupquote{PYTHONASYNCIODEBUG=1}} environment variable instead.

\item {} 
{\hyperref[\detokenize{ioloop:tornado.ioloop.IOLoop.clear_current}]{\sphinxcrossref{\sphinxcode{\sphinxupquote{IOLoop.clear\_current}}}}} now works if it is called before any
current loop is established.

\end{itemize}


\paragraph{\sphinxstyleliteralintitle{\sphinxupquote{tornado.iostream}}}
\label{\detokenize{releases/v5.0.0:tornado-iostream}}\begin{itemize}
\item {} 
The \sphinxcode{\sphinxupquote{io\_loop}} argument to the {\hyperref[\detokenize{iostream:tornado.iostream.IOStream}]{\sphinxcrossref{\sphinxcode{\sphinxupquote{IOStream}}}}} constructor has been removed.

\item {} 
New method {\hyperref[\detokenize{iostream:tornado.iostream.BaseIOStream.read_into}]{\sphinxcrossref{\sphinxcode{\sphinxupquote{BaseIOStream.read\_into}}}}} provides a minimal-copy alternative to
{\hyperref[\detokenize{iostream:tornado.iostream.BaseIOStream.read_bytes}]{\sphinxcrossref{\sphinxcode{\sphinxupquote{BaseIOStream.read\_bytes}}}}}.

\item {} 
{\hyperref[\detokenize{iostream:tornado.iostream.BaseIOStream.write}]{\sphinxcrossref{\sphinxcode{\sphinxupquote{BaseIOStream.write}}}}} is now much more efficient for very large amounts of data.

\item {} 
Fixed some cases in which \sphinxcode{\sphinxupquote{IOStream.error}} could be inaccurate.

\item {} 
Writing a \sphinxhref{https://docs.python.org/3.6/library/stdtypes.html\#memoryview}{\sphinxcode{\sphinxupquote{memoryview}}} can no longer result in “BufferError:
Existing exports of data: object cannot be re-sized”.

\end{itemize}


\paragraph{\sphinxstyleliteralintitle{\sphinxupquote{tornado.locks}}}
\label{\detokenize{releases/v5.0.0:tornado-locks}}\begin{itemize}
\item {} 
As a side effect of the \sphinxcode{\sphinxupquote{Future}} changes, waiters are always
notified asynchronously with respect to {\hyperref[\detokenize{locks:tornado.locks.Condition.notify}]{\sphinxcrossref{\sphinxcode{\sphinxupquote{Condition.notify}}}}}.

\end{itemize}


\paragraph{\sphinxstyleliteralintitle{\sphinxupquote{tornado.netutil}}}
\label{\detokenize{releases/v5.0.0:tornado-netutil}}\begin{itemize}
\item {} 
The default {\hyperref[\detokenize{netutil:tornado.netutil.Resolver}]{\sphinxcrossref{\sphinxcode{\sphinxupquote{Resolver}}}}} now uses {\hyperref[\detokenize{ioloop:tornado.ioloop.IOLoop.run_in_executor}]{\sphinxcrossref{\sphinxcode{\sphinxupquote{IOLoop.run\_in\_executor}}}}}.
{\hyperref[\detokenize{netutil:tornado.netutil.ExecutorResolver}]{\sphinxcrossref{\sphinxcode{\sphinxupquote{ExecutorResolver}}}}}, {\hyperref[\detokenize{netutil:tornado.netutil.BlockingResolver}]{\sphinxcrossref{\sphinxcode{\sphinxupquote{BlockingResolver}}}}}, and {\hyperref[\detokenize{netutil:tornado.netutil.ThreadedResolver}]{\sphinxcrossref{\sphinxcode{\sphinxupquote{ThreadedResolver}}}}} are
deprecated.

\item {} 
The \sphinxcode{\sphinxupquote{io\_loop}} arguments to {\hyperref[\detokenize{netutil:tornado.netutil.add_accept_handler}]{\sphinxcrossref{\sphinxcode{\sphinxupquote{add\_accept\_handler}}}}},
{\hyperref[\detokenize{netutil:tornado.netutil.ExecutorResolver}]{\sphinxcrossref{\sphinxcode{\sphinxupquote{ExecutorResolver}}}}}, and {\hyperref[\detokenize{netutil:tornado.netutil.ThreadedResolver}]{\sphinxcrossref{\sphinxcode{\sphinxupquote{ThreadedResolver}}}}} have been removed.

\item {} 
{\hyperref[\detokenize{netutil:tornado.netutil.add_accept_handler}]{\sphinxcrossref{\sphinxcode{\sphinxupquote{add\_accept\_handler}}}}} returns a callable which can be used to remove
all handlers that were added.

\item {} 
{\hyperref[\detokenize{netutil:tornado.netutil.OverrideResolver}]{\sphinxcrossref{\sphinxcode{\sphinxupquote{OverrideResolver}}}}} now accepts per-family overrides.

\end{itemize}


\paragraph{\sphinxstyleliteralintitle{\sphinxupquote{tornado.options}}}
\label{\detokenize{releases/v5.0.0:tornado-options}}\begin{itemize}
\item {} 
Duplicate option names are now detected properly whether they use
hyphens or underscores.

\end{itemize}


\paragraph{\sphinxstyleliteralintitle{\sphinxupquote{tornado.platform.asyncio}}}
\label{\detokenize{releases/v5.0.0:tornado-platform-asyncio}}\begin{itemize}
\item {} 
{\hyperref[\detokenize{asyncio:tornado.platform.asyncio.AsyncIOLoop}]{\sphinxcrossref{\sphinxcode{\sphinxupquote{AsyncIOLoop}}}}} and {\hyperref[\detokenize{asyncio:tornado.platform.asyncio.AsyncIOMainLoop}]{\sphinxcrossref{\sphinxcode{\sphinxupquote{AsyncIOMainLoop}}}}} are now used automatically
when appropriate; referencing them explicitly is no longer
recommended.

\item {} 
Starting an {\hyperref[\detokenize{ioloop:tornado.ioloop.IOLoop}]{\sphinxcrossref{\sphinxcode{\sphinxupquote{IOLoop}}}}} or making it current now also sets the
\sphinxhref{https://docs.python.org/3.6/library/asyncio.html\#module-asyncio}{\sphinxcode{\sphinxupquote{asyncio}}} event loop for the current thread. Closing an {\hyperref[\detokenize{ioloop:tornado.ioloop.IOLoop}]{\sphinxcrossref{\sphinxcode{\sphinxupquote{IOLoop}}}}}
closes the corresponding \sphinxhref{https://docs.python.org/3.6/library/asyncio.html\#module-asyncio}{\sphinxcode{\sphinxupquote{asyncio}}} event loop.

\item {} 
{\hyperref[\detokenize{asyncio:tornado.platform.asyncio.to_tornado_future}]{\sphinxcrossref{\sphinxcode{\sphinxupquote{to\_tornado\_future}}}}} and {\hyperref[\detokenize{asyncio:tornado.platform.asyncio.to_asyncio_future}]{\sphinxcrossref{\sphinxcode{\sphinxupquote{to\_asyncio\_future}}}}} are deprecated since
they are now no-ops.

\item {} 
{\hyperref[\detokenize{asyncio:tornado.platform.asyncio.AnyThreadEventLoopPolicy}]{\sphinxcrossref{\sphinxcode{\sphinxupquote{AnyThreadEventLoopPolicy}}}}} can now be used to easily allow the creation
of event loops on any thread (similar to Tornado’s prior policy).

\end{itemize}


\paragraph{\sphinxstyleliteralintitle{\sphinxupquote{tornado.platform.caresresolver}}}
\label{\detokenize{releases/v5.0.0:tornado-platform-caresresolver}}\begin{itemize}
\item {} 
The \sphinxcode{\sphinxupquote{io\_loop}} argument to {\hyperref[\detokenize{caresresolver:tornado.platform.caresresolver.CaresResolver}]{\sphinxcrossref{\sphinxcode{\sphinxupquote{CaresResolver}}}}} has been removed.

\end{itemize}


\paragraph{\sphinxstyleliteralintitle{\sphinxupquote{tornado.platform.twisted}}}
\label{\detokenize{releases/v5.0.0:tornado-platform-twisted}}\begin{itemize}
\item {} 
The \sphinxcode{\sphinxupquote{io\_loop}} arguments to \sphinxcode{\sphinxupquote{TornadoReactor}}, {\hyperref[\detokenize{twisted:tornado.platform.twisted.TwistedResolver}]{\sphinxcrossref{\sphinxcode{\sphinxupquote{TwistedResolver}}}}},
and \sphinxcode{\sphinxupquote{tornado.platform.twisted.install}} have been removed.

\end{itemize}


\paragraph{\sphinxstyleliteralintitle{\sphinxupquote{tornado.process}}}
\label{\detokenize{releases/v5.0.0:tornado-process}}\begin{itemize}
\item {} 
The \sphinxcode{\sphinxupquote{io\_loop}} argument to the {\hyperref[\detokenize{process:tornado.process.Subprocess}]{\sphinxcrossref{\sphinxcode{\sphinxupquote{Subprocess}}}}} constructor and
{\hyperref[\detokenize{process:tornado.process.Subprocess.initialize}]{\sphinxcrossref{\sphinxcode{\sphinxupquote{Subprocess.initialize}}}}} has been removed.

\end{itemize}


\paragraph{\sphinxstyleliteralintitle{\sphinxupquote{tornado.routing}}}
\label{\detokenize{releases/v5.0.0:tornado-routing}}\begin{itemize}
\item {} 
A default 404 response is now generated if no delegate is found for
a request.

\end{itemize}


\paragraph{\sphinxstyleliteralintitle{\sphinxupquote{tornado.simple\_httpclient}}}
\label{\detokenize{releases/v5.0.0:tornado-simple-httpclient}}\begin{itemize}
\item {} 
The \sphinxcode{\sphinxupquote{io\_loop}} argument to {\hyperref[\detokenize{httpclient:tornado.simple_httpclient.SimpleAsyncHTTPClient}]{\sphinxcrossref{\sphinxcode{\sphinxupquote{SimpleAsyncHTTPClient}}}}} has been removed.

\item {} 
TLS is now configured according to \sphinxhref{https://docs.python.org/3.6/library/ssl.html\#ssl.create\_default\_context}{\sphinxcode{\sphinxupquote{ssl.create\_default\_context}}} by
default.

\end{itemize}


\paragraph{\sphinxstyleliteralintitle{\sphinxupquote{tornado.tcpclient}}}
\label{\detokenize{releases/v5.0.0:tornado-tcpclient}}\begin{itemize}
\item {} 
The \sphinxcode{\sphinxupquote{io\_loop}} argument to the {\hyperref[\detokenize{tcpclient:tornado.tcpclient.TCPClient}]{\sphinxcrossref{\sphinxcode{\sphinxupquote{TCPClient}}}}} constructor has been
removed.

\item {} 
{\hyperref[\detokenize{tcpclient:tornado.tcpclient.TCPClient.connect}]{\sphinxcrossref{\sphinxcode{\sphinxupquote{TCPClient.connect}}}}} has a new \sphinxcode{\sphinxupquote{timeout}} argument.

\end{itemize}


\paragraph{\sphinxstyleliteralintitle{\sphinxupquote{tornado.tcpserver}}}
\label{\detokenize{releases/v5.0.0:tornado-tcpserver}}\begin{itemize}
\item {} 
The \sphinxcode{\sphinxupquote{io\_loop}} argument to the {\hyperref[\detokenize{tcpserver:tornado.tcpserver.TCPServer}]{\sphinxcrossref{\sphinxcode{\sphinxupquote{TCPServer}}}}} constructor has been
removed.

\item {} 
{\hyperref[\detokenize{tcpserver:tornado.tcpserver.TCPServer}]{\sphinxcrossref{\sphinxcode{\sphinxupquote{TCPServer}}}}} no longer logs \sphinxcode{\sphinxupquote{EBADF}} errors during shutdown.

\end{itemize}


\paragraph{\sphinxstyleliteralintitle{\sphinxupquote{tornado.testing}}}
\label{\detokenize{releases/v5.0.0:tornado-testing}}\begin{itemize}
\item {} 
The deprecated \sphinxcode{\sphinxupquote{tornado.testing.get\_unused\_port}} and
\sphinxcode{\sphinxupquote{tornado.testing.LogTrapTestCase}} have been removed.

\item {} 
{\hyperref[\detokenize{testing:tornado.testing.AsyncHTTPTestCase.fetch}]{\sphinxcrossref{\sphinxcode{\sphinxupquote{AsyncHTTPTestCase.fetch}}}}} now supports absolute URLs.

\item {} 
{\hyperref[\detokenize{testing:tornado.testing.AsyncHTTPTestCase.fetch}]{\sphinxcrossref{\sphinxcode{\sphinxupquote{AsyncHTTPTestCase.fetch}}}}} now connects to \sphinxcode{\sphinxupquote{127.0.0.1}}
instead of \sphinxcode{\sphinxupquote{localhost}} to be more robust against faulty
ipv6 configurations.

\end{itemize}


\paragraph{\sphinxstyleliteralintitle{\sphinxupquote{tornado.util}}}
\label{\detokenize{releases/v5.0.0:tornado-util}}\begin{itemize}
\item {} 
{\hyperref[\detokenize{util:tornado.util.TimeoutError}]{\sphinxcrossref{\sphinxcode{\sphinxupquote{tornado.util.TimeoutError}}}}} replaces \sphinxcode{\sphinxupquote{tornado.gen.TimeoutError}}
and \sphinxcode{\sphinxupquote{tornado.ioloop.TimeoutError}}.

\item {} 
{\hyperref[\detokenize{util:tornado.util.Configurable}]{\sphinxcrossref{\sphinxcode{\sphinxupquote{Configurable}}}}} now supports configuration at multiple levels of an
inheritance hierarchy.

\end{itemize}


\paragraph{\sphinxstyleliteralintitle{\sphinxupquote{tornado.web}}}
\label{\detokenize{releases/v5.0.0:tornado-web}}\begin{itemize}
\item {} 
{\hyperref[\detokenize{web:tornado.web.RequestHandler.set_status}]{\sphinxcrossref{\sphinxcode{\sphinxupquote{RequestHandler.set\_status}}}}} no longer requires that the given
status code appear in \sphinxhref{https://docs.python.org/3.6/library/http.client.html\#http.client.responses}{\sphinxcode{\sphinxupquote{http.client.responses}}}.

\item {} 
It is no longer allowed to send a body with 1xx or 204 responses.

\item {} 
Exception handling now breaks up reference cycles that could delay
garbage collection.

\item {} 
{\hyperref[\detokenize{web:tornado.web.RedirectHandler}]{\sphinxcrossref{\sphinxcode{\sphinxupquote{RedirectHandler}}}}} now copies any query arguments from the request
to the redirect location.

\item {} 
If both \sphinxcode{\sphinxupquote{If-None-Match}} and \sphinxcode{\sphinxupquote{If-Modified-Since}} headers are present
in a request to {\hyperref[\detokenize{web:tornado.web.StaticFileHandler}]{\sphinxcrossref{\sphinxcode{\sphinxupquote{StaticFileHandler}}}}}, the latter is now ignored.

\end{itemize}


\paragraph{\sphinxstyleliteralintitle{\sphinxupquote{tornado.websocket}}}
\label{\detokenize{releases/v5.0.0:tornado-websocket}}\begin{itemize}
\item {} 
The C accelerator now operates on multiple bytes at a time to
improve performance.

\item {} 
Requests with invalid websocket headers now get a response with
status code 400 instead of a closed connection.

\item {} 
{\hyperref[\detokenize{websocket:tornado.websocket.WebSocketHandler.write_message}]{\sphinxcrossref{\sphinxcode{\sphinxupquote{WebSocketHandler.write\_message}}}}} now raises {\hyperref[\detokenize{websocket:tornado.websocket.WebSocketClosedError}]{\sphinxcrossref{\sphinxcode{\sphinxupquote{WebSocketClosedError}}}}} if
the connection closes while the write is in progress.

\item {} 
The \sphinxcode{\sphinxupquote{io\_loop}} argument to {\hyperref[\detokenize{websocket:tornado.websocket.websocket_connect}]{\sphinxcrossref{\sphinxcode{\sphinxupquote{websocket\_connect}}}}} has been removed.

\end{itemize}


\subsection{What’s new in Tornado 4.5.3}
\label{\detokenize{releases/v4.5.3:what-s-new-in-tornado-4-5-3}}\label{\detokenize{releases/v4.5.3::doc}}

\subsubsection{Jan 6, 2018}
\label{\detokenize{releases/v4.5.3:jan-6-2018}}

\paragraph{\sphinxstyleliteralintitle{\sphinxupquote{tornado.curl\_httpclient}}}
\label{\detokenize{releases/v4.5.3:tornado-curl-httpclient}}\begin{itemize}
\item {} 
Improved debug logging on Python 3.

\end{itemize}


\paragraph{\sphinxstyleliteralintitle{\sphinxupquote{tornado.httpserver}}}
\label{\detokenize{releases/v4.5.3:tornado-httpserver}}\begin{itemize}
\item {} 
\sphinxcode{\sphinxupquote{Content-Length}} and \sphinxcode{\sphinxupquote{Transfer-Encoding}} headers are no longer
sent with 1xx or 204 responses (this was already true of 304
responses).

\item {} 
Reading chunked requests no longer leaves the connection in a broken
state.

\end{itemize}


\paragraph{\sphinxstyleliteralintitle{\sphinxupquote{tornado.iostream}}}
\label{\detokenize{releases/v4.5.3:tornado-iostream}}\begin{itemize}
\item {} 
Writing a \sphinxhref{https://docs.python.org/3.6/library/stdtypes.html\#memoryview}{\sphinxcode{\sphinxupquote{memoryview}}} can no longer result in “BufferError:
Existing exports of data: object cannot be re-sized”.

\end{itemize}


\paragraph{\sphinxstyleliteralintitle{\sphinxupquote{tornado.options}}}
\label{\detokenize{releases/v4.5.3:tornado-options}}\begin{itemize}
\item {} 
Duplicate option names are now detected properly whether they use
hyphens or underscores.

\end{itemize}


\paragraph{\sphinxstyleliteralintitle{\sphinxupquote{tornado.testing}}}
\label{\detokenize{releases/v4.5.3:tornado-testing}}\begin{itemize}
\item {} 
{\hyperref[\detokenize{testing:tornado.testing.AsyncHTTPTestCase.fetch}]{\sphinxcrossref{\sphinxcode{\sphinxupquote{AsyncHTTPTestCase.fetch}}}}} now uses \sphinxcode{\sphinxupquote{127.0.0.1}} instead of
\sphinxcode{\sphinxupquote{localhost}}, improving compatibility with systems that have
partially-working ipv6 stacks.

\end{itemize}


\paragraph{\sphinxstyleliteralintitle{\sphinxupquote{tornado.web}}}
\label{\detokenize{releases/v4.5.3:tornado-web}}\begin{itemize}
\item {} 
It is no longer allowed to send a body with 1xx or 204 responses.

\end{itemize}


\paragraph{\sphinxstyleliteralintitle{\sphinxupquote{tornado.websocket}}}
\label{\detokenize{releases/v4.5.3:tornado-websocket}}\begin{itemize}
\item {} 
Requests with invalid websocket headers now get a response with
status code 400 instead of a closed connection.

\end{itemize}


\subsection{What’s new in Tornado 4.5.2}
\label{\detokenize{releases/v4.5.2:what-s-new-in-tornado-4-5-2}}\label{\detokenize{releases/v4.5.2::doc}}

\subsubsection{Aug 27, 2017}
\label{\detokenize{releases/v4.5.2:aug-27-2017}}

\paragraph{Bug Fixes}
\label{\detokenize{releases/v4.5.2:bug-fixes}}\begin{itemize}
\item {} 
Tornado now sets the \sphinxcode{\sphinxupquote{FD\_CLOEXEC}} flag on all file descriptors it creates. This prevents hanging client connections and resource leaks when the {\hyperref[\detokenize{autoreload:module-tornado.autoreload}]{\sphinxcrossref{\sphinxcode{\sphinxupquote{tornado.autoreload}}}}} module (or \sphinxcode{\sphinxupquote{Application(debug=True)}}) is used.

\end{itemize}


\subsection{What’s new in Tornado 4.5.1}
\label{\detokenize{releases/v4.5.1:what-s-new-in-tornado-4-5-1}}\label{\detokenize{releases/v4.5.1::doc}}

\subsubsection{Apr 20, 2017}
\label{\detokenize{releases/v4.5.1:apr-20-2017}}

\paragraph{\sphinxstyleliteralintitle{\sphinxupquote{tornado.log}}}
\label{\detokenize{releases/v4.5.1:tornado-log}}\begin{itemize}
\item {} 
Improved detection of libraries for colorized logging.

\end{itemize}


\paragraph{\sphinxstyleliteralintitle{\sphinxupquote{tornado.httputil}}}
\label{\detokenize{releases/v4.5.1:tornado-httputil}}\begin{itemize}
\item {} 
{\hyperref[\detokenize{httputil:tornado.httputil.url_concat}]{\sphinxcrossref{\sphinxcode{\sphinxupquote{url\_concat}}}}} once again treats None as equivalent to an empty sequence.

\end{itemize}


\subsection{What’s new in Tornado 4.5}
\label{\detokenize{releases/v4.5.0:what-s-new-in-tornado-4-5}}\label{\detokenize{releases/v4.5.0::doc}}

\subsubsection{Apr 16, 2017}
\label{\detokenize{releases/v4.5.0:apr-16-2017}}

\paragraph{Backwards-compatibility warning}
\label{\detokenize{releases/v4.5.0:backwards-compatibility-warning}}\begin{itemize}
\item {} 
The {\hyperref[\detokenize{websocket:module-tornado.websocket}]{\sphinxcrossref{\sphinxcode{\sphinxupquote{tornado.websocket}}}}} module now imposes a limit on the size of incoming
messages, which defaults to 10MiB.

\end{itemize}


\paragraph{New module}
\label{\detokenize{releases/v4.5.0:new-module}}\begin{itemize}
\item {} 
{\hyperref[\detokenize{routing:module-tornado.routing}]{\sphinxcrossref{\sphinxcode{\sphinxupquote{tornado.routing}}}}} provides a more flexible routing system than the one built in
to {\hyperref[\detokenize{web:tornado.web.Application}]{\sphinxcrossref{\sphinxcode{\sphinxupquote{Application}}}}}.

\end{itemize}


\paragraph{General changes}
\label{\detokenize{releases/v4.5.0:general-changes}}\begin{itemize}
\item {} 
Reduced the number of circular references, reducing memory usage and
improving performance.

\end{itemize}


\paragraph{\sphinxstyleliteralintitle{\sphinxupquote{tornado.auth}}}
\label{\detokenize{releases/v4.5.0:tornado-auth}}\begin{itemize}
\item {} 
The {\hyperref[\detokenize{auth:module-tornado.auth}]{\sphinxcrossref{\sphinxcode{\sphinxupquote{tornado.auth}}}}} module has been updated for compatibility with \sphinxhref{https://github.com/tornadoweb/tornado/pull/1977}{a
change to Facebook’s access\_token endpoint}. This includes both
the changes initially released in Tornado 4.4.3 and an additional change
to support the \sphinxcode{\sphinxupquote{{}`session\_expires}} field in the new format.
The \sphinxcode{\sphinxupquote{session\_expires}} field is currently a string; it should be accessed
as \sphinxcode{\sphinxupquote{int(user{[}'session\_expires'{]})}} because it will change from a string to
an int in Tornado 5.0.

\end{itemize}


\paragraph{\sphinxstyleliteralintitle{\sphinxupquote{tornado.autoreload}}}
\label{\detokenize{releases/v4.5.0:tornado-autoreload}}\begin{itemize}
\item {} 
Autoreload is now compatible with the \sphinxhref{https://docs.python.org/3.6/library/asyncio.html\#module-asyncio}{\sphinxcode{\sphinxupquote{asyncio}}} event loop.

\item {} 
Autoreload no longer attempts to close the {\hyperref[\detokenize{ioloop:tornado.ioloop.IOLoop}]{\sphinxcrossref{\sphinxcode{\sphinxupquote{IOLoop}}}}} and all registered
file descriptors before restarting; it relies on the \sphinxcode{\sphinxupquote{CLOEXEC}} flag
being set instead.

\end{itemize}


\paragraph{\sphinxstyleliteralintitle{\sphinxupquote{tornado.concurrent}}}
\label{\detokenize{releases/v4.5.0:tornado-concurrent}}\begin{itemize}
\item {} 
Suppressed some “‘NoneType’ object not callback” messages that could
be logged at shutdown.

\end{itemize}


\paragraph{\sphinxstyleliteralintitle{\sphinxupquote{tornado.gen}}}
\label{\detokenize{releases/v4.5.0:tornado-gen}}\begin{itemize}
\item {} 
\sphinxcode{\sphinxupquote{yield None}} is now equivalent to \sphinxcode{\sphinxupquote{yield gen.moment}}.
{\hyperref[\detokenize{gen:tornado.gen.moment}]{\sphinxcrossref{\sphinxcode{\sphinxupquote{moment}}}}} is deprecated. This improves compatibility with
\sphinxhref{https://docs.python.org/3.6/library/asyncio.html\#module-asyncio}{\sphinxcode{\sphinxupquote{asyncio}}}.

\item {} 
Fixed an issue in which a generator object could be garbage
collected prematurely (most often when weak references are used.

\item {} 
New function {\hyperref[\detokenize{gen:tornado.gen.is_coroutine_function}]{\sphinxcrossref{\sphinxcode{\sphinxupquote{is\_coroutine\_function}}}}} identifies functions wrapped
by {\hyperref[\detokenize{gen:tornado.gen.coroutine}]{\sphinxcrossref{\sphinxcode{\sphinxupquote{coroutine}}}}} or \sphinxcode{\sphinxupquote{engine}}.

\end{itemize}


\paragraph{\sphinxstyleliteralintitle{\sphinxupquote{tornado.http1connection}}}
\label{\detokenize{releases/v4.5.0:tornado-http1connection}}\begin{itemize}
\item {} 
The \sphinxcode{\sphinxupquote{Transfer-Encoding}} header is now parsed case-insensitively.

\end{itemize}


\paragraph{\sphinxstyleliteralintitle{\sphinxupquote{tornado.httpclient}}}
\label{\detokenize{releases/v4.5.0:tornado-httpclient}}\begin{itemize}
\item {} 
\sphinxcode{\sphinxupquote{SimpleAsyncHTTPClient}} now follows 308 redirects.

\item {} 
\sphinxcode{\sphinxupquote{CurlAsyncHTTPClient}} will no longer accept protocols other than
\sphinxcode{\sphinxupquote{http}} and \sphinxcode{\sphinxupquote{https}}. To override this, set \sphinxcode{\sphinxupquote{pycurl.PROTOCOLS}}
and \sphinxcode{\sphinxupquote{pycurl.REDIR\_PROTOCOLS}} in a \sphinxcode{\sphinxupquote{prepare\_curl\_callback}}.

\item {} 
\sphinxcode{\sphinxupquote{CurlAsyncHTTPClient}} now supports digest authentication for proxies
(in addition to basic auth) via the new \sphinxcode{\sphinxupquote{proxy\_auth\_mode}} argument.

\item {} 
The minimum supported version of \sphinxcode{\sphinxupquote{libcurl}} is now \sphinxcode{\sphinxupquote{7.22.0}}.

\end{itemize}


\paragraph{\sphinxstyleliteralintitle{\sphinxupquote{tornado.httpserver}}}
\label{\detokenize{releases/v4.5.0:tornado-httpserver}}\begin{itemize}
\item {} 
{\hyperref[\detokenize{httpserver:tornado.httpserver.HTTPServer}]{\sphinxcrossref{\sphinxcode{\sphinxupquote{HTTPServer}}}}} now accepts the keyword argument
\sphinxcode{\sphinxupquote{trusted\_downstream}} which controls the parsing of
\sphinxcode{\sphinxupquote{X-Forwarded-For}} headers. This header may be a list or set of IP
addresses of trusted proxies which will be skipped in the
\sphinxcode{\sphinxupquote{X-Forwarded-For}} list.

\item {} 
The \sphinxcode{\sphinxupquote{no\_keep\_alive}} argument works again.

\end{itemize}


\paragraph{\sphinxstyleliteralintitle{\sphinxupquote{tornado.httputil}}}
\label{\detokenize{releases/v4.5.0:tornado-httputil}}\begin{itemize}
\item {} 
{\hyperref[\detokenize{httputil:tornado.httputil.url_concat}]{\sphinxcrossref{\sphinxcode{\sphinxupquote{url\_concat}}}}} correctly handles fragments and existing query arguments.

\end{itemize}


\paragraph{\sphinxstyleliteralintitle{\sphinxupquote{tornado.ioloop}}}
\label{\detokenize{releases/v4.5.0:tornado-ioloop}}\begin{itemize}
\item {} 
Fixed 100\% CPU usage after a callback returns an empty list or dict.

\item {} 
{\hyperref[\detokenize{ioloop:tornado.ioloop.IOLoop.add_callback}]{\sphinxcrossref{\sphinxcode{\sphinxupquote{IOLoop.add\_callback}}}}} now uses a lockless implementation which
makes it safe for use from \sphinxcode{\sphinxupquote{\_\_del\_\_}} methods. This improves
performance of calls to {\hyperref[\detokenize{ioloop:tornado.ioloop.IOLoop.add_callback}]{\sphinxcrossref{\sphinxcode{\sphinxupquote{add\_callback}}}}} from the {\hyperref[\detokenize{ioloop:tornado.ioloop.IOLoop}]{\sphinxcrossref{\sphinxcode{\sphinxupquote{IOLoop}}}}}
thread, and slightly decreases it for calls from other threads.

\end{itemize}


\paragraph{\sphinxstyleliteralintitle{\sphinxupquote{tornado.iostream}}}
\label{\detokenize{releases/v4.5.0:tornado-iostream}}\begin{itemize}
\item {} 
\sphinxhref{https://docs.python.org/3.6/library/stdtypes.html\#memoryview}{\sphinxcode{\sphinxupquote{memoryview}}} objects are now permitted as arguments to {\hyperref[\detokenize{iostream:tornado.iostream.BaseIOStream.write}]{\sphinxcrossref{\sphinxcode{\sphinxupquote{write}}}}}.

\item {} 
The internal memory buffers used by {\hyperref[\detokenize{iostream:tornado.iostream.IOStream}]{\sphinxcrossref{\sphinxcode{\sphinxupquote{IOStream}}}}} now use \sphinxhref{https://docs.python.org/3.6/library/stdtypes.html\#bytearray}{\sphinxcode{\sphinxupquote{bytearray}}}
instead of a list of \sphinxhref{https://docs.python.org/3.6/library/stdtypes.html\#bytes}{\sphinxcode{\sphinxupquote{bytes}}}, improving performance.

\item {} 
Futures returned by {\hyperref[\detokenize{iostream:tornado.iostream.BaseIOStream.write}]{\sphinxcrossref{\sphinxcode{\sphinxupquote{write}}}}} are no longer orphaned if a second
call to \sphinxcode{\sphinxupquote{write}} occurs before the previous one is finished.

\end{itemize}


\paragraph{\sphinxstyleliteralintitle{\sphinxupquote{tornado.log}}}
\label{\detokenize{releases/v4.5.0:tornado-log}}\begin{itemize}
\item {} 
Colored log output is now supported on Windows if the
\sphinxhref{https://pypi.python.org/pypi/colorama}{colorama} library
is installed  and the application calls \sphinxcode{\sphinxupquote{colorama.init()}} at
startup.

\item {} 
The signature of the {\hyperref[\detokenize{log:tornado.log.LogFormatter}]{\sphinxcrossref{\sphinxcode{\sphinxupquote{LogFormatter}}}}} constructor has been changed to
make it compatible with \sphinxhref{https://docs.python.org/3.6/library/logging.config.html\#logging.config.dictConfig}{\sphinxcode{\sphinxupquote{logging.config.dictConfig}}}.

\end{itemize}


\paragraph{\sphinxstyleliteralintitle{\sphinxupquote{tornado.netutil}}}
\label{\detokenize{releases/v4.5.0:tornado-netutil}}\begin{itemize}
\item {} 
Worked around an issue that caused “LookupError: unknown encoding:
latin1” errors on Solaris.

\end{itemize}


\paragraph{\sphinxstyleliteralintitle{\sphinxupquote{tornado.process}}}
\label{\detokenize{releases/v4.5.0:tornado-process}}\begin{itemize}
\item {} 
{\hyperref[\detokenize{process:tornado.process.Subprocess}]{\sphinxcrossref{\sphinxcode{\sphinxupquote{Subprocess}}}}} no longer causes “subprocess still running” warnings on Python 3.6.

\item {} 
Improved error handling in {\hyperref[\detokenize{process:tornado.process.cpu_count}]{\sphinxcrossref{\sphinxcode{\sphinxupquote{cpu\_count}}}}}.

\end{itemize}


\paragraph{\sphinxstyleliteralintitle{\sphinxupquote{tornado.tcpclient}}}
\label{\detokenize{releases/v4.5.0:tornado-tcpclient}}\begin{itemize}
\item {} 
{\hyperref[\detokenize{tcpclient:tornado.tcpclient.TCPClient}]{\sphinxcrossref{\sphinxcode{\sphinxupquote{TCPClient}}}}} now supports a \sphinxcode{\sphinxupquote{source\_ip}} and \sphinxcode{\sphinxupquote{source\_port}} argument.

\item {} 
Improved error handling for environments where IPv6 support is incomplete.

\end{itemize}


\paragraph{\sphinxstyleliteralintitle{\sphinxupquote{tornado.tcpserver}}}
\label{\detokenize{releases/v4.5.0:tornado-tcpserver}}\begin{itemize}
\item {} 
{\hyperref[\detokenize{tcpserver:tornado.tcpserver.TCPServer.handle_stream}]{\sphinxcrossref{\sphinxcode{\sphinxupquote{TCPServer.handle\_stream}}}}} implementations may now be native coroutines.

\item {} 
Stopping a {\hyperref[\detokenize{tcpserver:tornado.tcpserver.TCPServer}]{\sphinxcrossref{\sphinxcode{\sphinxupquote{TCPServer}}}}} twice no longer raises an exception.

\end{itemize}


\paragraph{\sphinxstyleliteralintitle{\sphinxupquote{tornado.web}}}
\label{\detokenize{releases/v4.5.0:tornado-web}}\begin{itemize}
\item {} 
{\hyperref[\detokenize{web:tornado.web.RedirectHandler}]{\sphinxcrossref{\sphinxcode{\sphinxupquote{RedirectHandler}}}}} now supports substituting parts of the matched
URL into the redirect location using \sphinxhref{https://docs.python.org/3.6/library/stdtypes.html\#str.format}{\sphinxcode{\sphinxupquote{str.format}}} syntax.

\item {} 
New methods {\hyperref[\detokenize{web:tornado.web.RequestHandler.render_linked_js}]{\sphinxcrossref{\sphinxcode{\sphinxupquote{RequestHandler.render\_linked\_js}}}}},
{\hyperref[\detokenize{web:tornado.web.RequestHandler.render_embed_js}]{\sphinxcrossref{\sphinxcode{\sphinxupquote{RequestHandler.render\_embed\_js}}}}},
{\hyperref[\detokenize{web:tornado.web.RequestHandler.render_linked_css}]{\sphinxcrossref{\sphinxcode{\sphinxupquote{RequestHandler.render\_linked\_css}}}}}, and
{\hyperref[\detokenize{web:tornado.web.RequestHandler.render_embed_css}]{\sphinxcrossref{\sphinxcode{\sphinxupquote{RequestHandler.render\_embed\_css}}}}} can be overridden to customize
the output of {\hyperref[\detokenize{web:tornado.web.UIModule}]{\sphinxcrossref{\sphinxcode{\sphinxupquote{UIModule}}}}}.

\end{itemize}


\paragraph{\sphinxstyleliteralintitle{\sphinxupquote{tornado.websocket}}}
\label{\detokenize{releases/v4.5.0:tornado-websocket}}\begin{itemize}
\item {} 
{\hyperref[\detokenize{websocket:tornado.websocket.WebSocketHandler.on_message}]{\sphinxcrossref{\sphinxcode{\sphinxupquote{WebSocketHandler.on\_message}}}}} implementations may now be
coroutines. New messages will not be processed until the previous
\sphinxcode{\sphinxupquote{on\_message}} coroutine has finished.

\item {} 
The \sphinxcode{\sphinxupquote{websocket\_ping\_interval}} and \sphinxcode{\sphinxupquote{websocket\_ping\_timeout}}
application settings can now be used to enable a periodic ping of
the websocket connection, allowing dropped connections to be
detected and closed.

\item {} 
The new \sphinxcode{\sphinxupquote{websocket\_max\_message\_size}} setting defaults to 10MiB.
The connection will be closed if messages larger than this are received.

\item {} 
Headers set by {\hyperref[\detokenize{web:tornado.web.RequestHandler.prepare}]{\sphinxcrossref{\sphinxcode{\sphinxupquote{RequestHandler.prepare}}}}} or
{\hyperref[\detokenize{web:tornado.web.RequestHandler.set_default_headers}]{\sphinxcrossref{\sphinxcode{\sphinxupquote{RequestHandler.set\_default\_headers}}}}} are now sent as a part of the
websocket handshake.

\item {} 
Return values from {\hyperref[\detokenize{websocket:tornado.websocket.WebSocketHandler.get_compression_options}]{\sphinxcrossref{\sphinxcode{\sphinxupquote{WebSocketHandler.get\_compression\_options}}}}} may now include
the keys \sphinxcode{\sphinxupquote{compression\_level}} and \sphinxcode{\sphinxupquote{mem\_level}} to set gzip parameters.
The default compression level is now 6 instead of 9.

\end{itemize}


\paragraph{Demos}
\label{\detokenize{releases/v4.5.0:demos}}\begin{itemize}
\item {} 
A new file upload demo is available in the \sphinxhref{https://github.com/tornadoweb/tornado/tree/master/demos/file\_upload}{file\_upload}
directory.

\item {} 
A new {\hyperref[\detokenize{tcpclient:tornado.tcpclient.TCPClient}]{\sphinxcrossref{\sphinxcode{\sphinxupquote{TCPClient}}}}} and {\hyperref[\detokenize{tcpserver:tornado.tcpserver.TCPServer}]{\sphinxcrossref{\sphinxcode{\sphinxupquote{TCPServer}}}}} demo is available in the
\sphinxhref{https://github.com/tornadoweb/tornado/tree/master/demos/tcpecho}{tcpecho} directory.

\item {} 
Minor updates have been made to several existing demos, including
updates to more recent versions of jquery.

\end{itemize}


\paragraph{Credits}
\label{\detokenize{releases/v4.5.0:credits}}
The following people contributed commits to this release:
\begin{itemize}
\item {} 
A. Jesse Jiryu Davis

\item {} 
Aaron Opfer

\item {} 
Akihiro Yamazaki

\item {} 
Alexander

\item {} 
Andreas Røsdal

\item {} 
Andrew Rabert

\item {} 
Andrew Sumin

\item {} 
Antoine Pietri

\item {} 
Antoine Pitrou

\item {} 
Artur Stawiarski

\item {} 
Ben Darnell

\item {} 
Brian Mego

\item {} 
Dario

\item {} 
Doug Vargas

\item {} 
Eugene Dubovoy

\item {} 
Iver Jordal

\item {} 
JZQT

\item {} 
James Maier

\item {} 
Jeff Hunter

\item {} 
Leynos

\item {} 
Mark Henderson

\item {} 
Michael V. DePalatis

\item {} 
Min RK

\item {} 
Mircea Ulinic

\item {} 
Ping

\item {} 
Ping Yang

\item {} 
Riccardo Magliocchetti

\item {} 
Samuel Chen

\item {} 
Samuel Dion-Girardeau

\item {} 
Scott Meisburger

\item {} 
Shawn Ding

\item {} 
TaoBeier

\item {} 
Thomas Kluyver

\item {} 
Vadim Semenov

\item {} 
matee

\item {} 
mike820324

\item {} 
stiletto

\item {} 
zhimin

\item {} 
依云

\end{itemize}


\subsection{What’s new in Tornado 4.4.3}
\label{\detokenize{releases/v4.4.3:what-s-new-in-tornado-4-4-3}}\label{\detokenize{releases/v4.4.3::doc}}

\subsubsection{Mar 30, 2017}
\label{\detokenize{releases/v4.4.3:mar-30-2017}}

\paragraph{Bug fixes}
\label{\detokenize{releases/v4.4.3:bug-fixes}}\begin{itemize}
\item {} 
The {\hyperref[\detokenize{auth:module-tornado.auth}]{\sphinxcrossref{\sphinxcode{\sphinxupquote{tornado.auth}}}}} module has been updated for compatibility with \sphinxhref{https://github.com/tornadoweb/tornado/pull/1977}{a
change to Facebook’s access\_token endpoint.}

\end{itemize}


\subsection{What’s new in Tornado 4.4.2}
\label{\detokenize{releases/v4.4.2:what-s-new-in-tornado-4-4-2}}\label{\detokenize{releases/v4.4.2::doc}}

\subsubsection{Oct 1, 2016}
\label{\detokenize{releases/v4.4.2:oct-1-2016}}

\paragraph{Security fixes}
\label{\detokenize{releases/v4.4.2:security-fixes}}\begin{itemize}
\item {} 
A difference in cookie parsing between Tornado and web browsers
(especially when combined with Google Analytics) could allow an
attacker to set arbitrary cookies and bypass XSRF protection. The
cookie parser has been rewritten to fix this attack.

\end{itemize}


\paragraph{Backwards-compatibility notes}
\label{\detokenize{releases/v4.4.2:backwards-compatibility-notes}}\begin{itemize}
\item {} 
Cookies containing certain special characters (in particular semicolon
and square brackets) are now parsed differently.

\item {} 
If the cookie header contains a combination of valid and invalid cookies,
the valid ones will be returned (older versions of Tornado would reject the
entire header for a single invalid cookie).

\end{itemize}


\subsection{What’s new in Tornado 4.4.1}
\label{\detokenize{releases/v4.4.1:what-s-new-in-tornado-4-4-1}}\label{\detokenize{releases/v4.4.1::doc}}

\subsubsection{Jul 23, 2016}
\label{\detokenize{releases/v4.4.1:jul-23-2016}}

\paragraph{\sphinxstyleliteralintitle{\sphinxupquote{tornado.web}}}
\label{\detokenize{releases/v4.4.1:tornado-web}}\begin{itemize}
\item {} 
Fixed a regression in Tornado 4.4 which caused URL regexes
containing backslash escapes outside capturing groups to be
rejected.

\end{itemize}


\subsection{What’s new in Tornado 4.4}
\label{\detokenize{releases/v4.4.0:what-s-new-in-tornado-4-4}}\label{\detokenize{releases/v4.4.0::doc}}

\subsubsection{Jul 15, 2016}
\label{\detokenize{releases/v4.4.0:jul-15-2016}}

\paragraph{General}
\label{\detokenize{releases/v4.4.0:general}}\begin{itemize}
\item {} 
Tornado now requires Python 2.7 or 3.3+; versions 2.6 and 3.2 are no
longer supported. Pypy3 is still supported even though its latest
release is mainly based on Python 3.2.

\item {} 
The \sphinxhref{https://pypi.python.org/pypi/monotonic}{monotonic} package is
now supported as an alternative to \sphinxhref{https://pypi.python.org/pypi/Monotime}{Monotime} for monotonic clock support
on Python 2.

\end{itemize}


\paragraph{\sphinxstyleliteralintitle{\sphinxupquote{tornado.curl\_httpclient}}}
\label{\detokenize{releases/v4.4.0:tornado-curl-httpclient}}\begin{itemize}
\item {} 
Failures in \sphinxcode{\sphinxupquote{\_curl\_setup\_request}} no longer cause the
\sphinxcode{\sphinxupquote{max\_clients}} pool to be exhausted.

\item {} 
Non-ascii header values are now handled correctly.

\end{itemize}


\paragraph{\sphinxstyleliteralintitle{\sphinxupquote{tornado.gen}}}
\label{\detokenize{releases/v4.4.0:tornado-gen}}\begin{itemize}
\item {} 
{\hyperref[\detokenize{gen:tornado.gen.with_timeout}]{\sphinxcrossref{\sphinxcode{\sphinxupquote{with\_timeout}}}}} now accepts any yieldable object (except
\sphinxcode{\sphinxupquote{YieldPoint}}), not just {\hyperref[\detokenize{concurrent:tornado.concurrent.Future}]{\sphinxcrossref{\sphinxcode{\sphinxupquote{tornado.concurrent.Future}}}}}.

\end{itemize}


\paragraph{\sphinxstyleliteralintitle{\sphinxupquote{tornado.httpclient}}}
\label{\detokenize{releases/v4.4.0:tornado-httpclient}}\begin{itemize}
\item {} 
The errors raised by timeouts now indicate what state the request
was in; the error message is no longer simply “599 Timeout”.

\item {} 
Calling \sphinxhref{https://docs.python.org/3.6/library/functions.html\#repr}{\sphinxcode{\sphinxupquote{repr}}} on a {\hyperref[\detokenize{httpclient:tornado.httpclient.HTTPError}]{\sphinxcrossref{\sphinxcode{\sphinxupquote{tornado.httpclient.HTTPError}}}}} no longer raises
an error.

\end{itemize}


\paragraph{\sphinxstyleliteralintitle{\sphinxupquote{tornado.httpserver}}}
\label{\detokenize{releases/v4.4.0:tornado-httpserver}}\begin{itemize}
\item {} 
Int-like enums (including \sphinxhref{https://docs.python.org/3.6/library/http.html\#http.HTTPStatus}{\sphinxcode{\sphinxupquote{http.HTTPStatus}}}) can now be used as
status codes.

\item {} 
Responses with status code \sphinxcode{\sphinxupquote{204 No Content}} no longer emit a
\sphinxcode{\sphinxupquote{Content-Length: 0}} header.

\end{itemize}


\paragraph{\sphinxstyleliteralintitle{\sphinxupquote{tornado.ioloop}}}
\label{\detokenize{releases/v4.4.0:tornado-ioloop}}\begin{itemize}
\item {} 
Improved performance when there are large numbers of active timeouts.

\end{itemize}


\paragraph{\sphinxstyleliteralintitle{\sphinxupquote{tornado.netutil}}}
\label{\detokenize{releases/v4.4.0:tornado-netutil}}\begin{itemize}
\item {} 
All included {\hyperref[\detokenize{netutil:tornado.netutil.Resolver}]{\sphinxcrossref{\sphinxcode{\sphinxupquote{Resolver}}}}} implementations raise \sphinxhref{https://docs.python.org/3.6/library/exceptions.html\#IOError}{\sphinxcode{\sphinxupquote{IOError}}} (or a
subclass) for any resolution failure.

\end{itemize}


\paragraph{\sphinxstyleliteralintitle{\sphinxupquote{tornado.options}}}
\label{\detokenize{releases/v4.4.0:tornado-options}}\begin{itemize}
\item {} 
Options can now be modified with subscript syntax in addition to
attribute syntax.

\item {} 
The special variable \sphinxcode{\sphinxupquote{\_\_file\_\_}} is now available inside config files.

\end{itemize}


\paragraph{\sphinxstyleliteralintitle{\sphinxupquote{tornado.simple\_httpclient}}}
\label{\detokenize{releases/v4.4.0:tornado-simple-httpclient}}\begin{itemize}
\item {} 
HTTP/1.0 (not 1.1) responses without a \sphinxcode{\sphinxupquote{Content-Length}} header now
work correctly.

\end{itemize}


\paragraph{\sphinxstyleliteralintitle{\sphinxupquote{tornado.tcpserver}}}
\label{\detokenize{releases/v4.4.0:tornado-tcpserver}}\begin{itemize}
\item {} 
{\hyperref[\detokenize{tcpserver:tornado.tcpserver.TCPServer.bind}]{\sphinxcrossref{\sphinxcode{\sphinxupquote{TCPServer.bind}}}}} now accepts a \sphinxcode{\sphinxupquote{reuse\_port}} argument.

\end{itemize}


\paragraph{\sphinxstyleliteralintitle{\sphinxupquote{tornado.testing}}}
\label{\detokenize{releases/v4.4.0:tornado-testing}}\begin{itemize}
\item {} 
Test sockets now always use \sphinxcode{\sphinxupquote{127.0.0.1}} instead of \sphinxcode{\sphinxupquote{localhost}}.
This avoids conflicts when the automatically-assigned port is
available on IPv4 but not IPv6, or in unusual network configurations
when \sphinxcode{\sphinxupquote{localhost}} has multiple IP addresses.

\end{itemize}


\paragraph{\sphinxstyleliteralintitle{\sphinxupquote{tornado.web}}}
\label{\detokenize{releases/v4.4.0:tornado-web}}\begin{itemize}
\item {} 
\sphinxcode{\sphinxupquote{image/svg+xml}} is now on the list of compressible mime types.

\item {} 
Fixed an error on Python 3 when compression is used with multiple
\sphinxcode{\sphinxupquote{Vary}} headers.

\end{itemize}


\paragraph{\sphinxstyleliteralintitle{\sphinxupquote{tornado.websocket}}}
\label{\detokenize{releases/v4.4.0:tornado-websocket}}\begin{itemize}
\item {} 
\sphinxcode{\sphinxupquote{WebSocketHandler.\_\_init\_\_}} now uses \sphinxhref{https://docs.python.org/3.6/library/functions.html\#super}{\sphinxcode{\sphinxupquote{super}}}, which improves
support for multiple inheritance.

\end{itemize}


\subsection{What’s new in Tornado 4.3}
\label{\detokenize{releases/v4.3.0:what-s-new-in-tornado-4-3}}\label{\detokenize{releases/v4.3.0::doc}}

\subsubsection{Nov 6, 2015}
\label{\detokenize{releases/v4.3.0:nov-6-2015}}

\paragraph{Highlights}
\label{\detokenize{releases/v4.3.0:highlights}}\begin{itemize}
\item {} 
The new async/await keywords in Python 3.5 are supported. In most cases,
\sphinxcode{\sphinxupquote{async def}} can be used in place of the \sphinxcode{\sphinxupquote{@gen.coroutine}} decorator.
Inside a function defined with \sphinxcode{\sphinxupquote{async def}}, use \sphinxcode{\sphinxupquote{await}} instead of
\sphinxcode{\sphinxupquote{yield}} to wait on an asynchronous operation. Coroutines defined with
async/await will be faster than those defined with \sphinxcode{\sphinxupquote{@gen.coroutine}} and
\sphinxcode{\sphinxupquote{yield}}, but do not support some features including \sphinxcode{\sphinxupquote{Callback}}/\sphinxcode{\sphinxupquote{Wait}} or
the ability to yield a Twisted \sphinxcode{\sphinxupquote{Deferred}}. See {\hyperref[\detokenize{guide/coroutines:native-coroutines}]{\sphinxcrossref{\DUrole{std,std-ref}{the users’
guide}}}} for more.

\item {} 
The async/await keywords are also available when compiling with Cython in
older versions of Python.

\end{itemize}


\paragraph{Deprecation notice}
\label{\detokenize{releases/v4.3.0:deprecation-notice}}\begin{itemize}
\item {} 
This will be the last release of Tornado to support Python 2.6 or 3.2.
Note that PyPy3 will continue to be supported even though it implements
a mix of Python 3.2 and 3.3 features.

\end{itemize}


\paragraph{Installation}
\label{\detokenize{releases/v4.3.0:installation}}\begin{itemize}
\item {} 
Tornado has several new dependencies: \sphinxcode{\sphinxupquote{ordereddict}} on Python 2.6,
\sphinxcode{\sphinxupquote{singledispatch}} on all Python versions prior to 3.4 (This was an
optional dependency in prior versions of Tornado, and is now
mandatory), and \sphinxcode{\sphinxupquote{backports\_abc\textgreater{}=0.4}} on all versions prior to
3.5. These dependencies will be installed automatically when installing
with \sphinxcode{\sphinxupquote{pip}} or \sphinxcode{\sphinxupquote{setup.py install}}. These dependencies will not
be required when running on Google App Engine.

\item {} 
Binary wheels are provided for Python 3.5 on Windows (32 and 64 bit).

\end{itemize}


\paragraph{\sphinxstyleliteralintitle{\sphinxupquote{tornado.auth}}}
\label{\detokenize{releases/v4.3.0:tornado-auth}}\begin{itemize}
\item {} 
New method {\hyperref[\detokenize{auth:tornado.auth.OAuth2Mixin.oauth2_request}]{\sphinxcrossref{\sphinxcode{\sphinxupquote{OAuth2Mixin.oauth2\_request}}}}} can be used to make authenticated
requests with an access token.

\item {} 
Now compatible with callbacks that have been compiled with Cython.

\end{itemize}


\paragraph{\sphinxstyleliteralintitle{\sphinxupquote{tornado.autoreload}}}
\label{\detokenize{releases/v4.3.0:tornado-autoreload}}\begin{itemize}
\item {} 
Fixed an issue with the autoreload command-line wrapper in which
imports would be incorrectly interpreted as relative.

\end{itemize}


\paragraph{\sphinxstyleliteralintitle{\sphinxupquote{tornado.curl\_httpclient}}}
\label{\detokenize{releases/v4.3.0:tornado-curl-httpclient}}\begin{itemize}
\item {} 
Fixed parsing of multi-line headers.

\item {} 
\sphinxcode{\sphinxupquote{allow\_nonstandard\_methods=True}} now bypasses body sanity checks,
in the same way as in \sphinxcode{\sphinxupquote{simple\_httpclient}}.

\item {} 
The \sphinxcode{\sphinxupquote{PATCH}} method now allows a body without
\sphinxcode{\sphinxupquote{allow\_nonstandard\_methods=True}}.

\end{itemize}


\paragraph{\sphinxstyleliteralintitle{\sphinxupquote{tornado.gen}}}
\label{\detokenize{releases/v4.3.0:tornado-gen}}\begin{itemize}
\item {} 
{\hyperref[\detokenize{gen:tornado.gen.WaitIterator}]{\sphinxcrossref{\sphinxcode{\sphinxupquote{WaitIterator}}}}} now supports the \sphinxcode{\sphinxupquote{async for}} statement on Python 3.5.

\item {} 
\sphinxcode{\sphinxupquote{@gen.coroutine}} can be applied to functions compiled with Cython.
On python versions prior to 3.5, the \sphinxcode{\sphinxupquote{backports\_abc}} package must
be installed for this functionality.

\item {} 
\sphinxcode{\sphinxupquote{Multi}} and {\hyperref[\detokenize{gen:tornado.gen.multi_future}]{\sphinxcrossref{\sphinxcode{\sphinxupquote{multi\_future}}}}} are deprecated and replaced by
a unified function {\hyperref[\detokenize{gen:tornado.gen.multi}]{\sphinxcrossref{\sphinxcode{\sphinxupquote{multi}}}}}.

\end{itemize}


\paragraph{\sphinxstyleliteralintitle{\sphinxupquote{tornado.httpclient}}}
\label{\detokenize{releases/v4.3.0:tornado-httpclient}}\begin{itemize}
\item {} 
{\hyperref[\detokenize{httpclient:tornado.httpclient.HTTPError}]{\sphinxcrossref{\sphinxcode{\sphinxupquote{tornado.httpclient.HTTPError}}}}} is now copyable with the \sphinxhref{https://docs.python.org/3.6/library/copy.html\#module-copy}{\sphinxcode{\sphinxupquote{copy}}} module.

\end{itemize}


\paragraph{\sphinxstyleliteralintitle{\sphinxupquote{tornado.httpserver}}}
\label{\detokenize{releases/v4.3.0:tornado-httpserver}}\begin{itemize}
\item {} 
Requests containing both \sphinxcode{\sphinxupquote{Content-Length}} and \sphinxcode{\sphinxupquote{Transfer-Encoding}}
will be treated as an error.

\end{itemize}


\paragraph{\sphinxstyleliteralintitle{\sphinxupquote{tornado.httputil}}}
\label{\detokenize{releases/v4.3.0:tornado-httputil}}\begin{itemize}
\item {} 
{\hyperref[\detokenize{httputil:tornado.httputil.HTTPHeaders}]{\sphinxcrossref{\sphinxcode{\sphinxupquote{HTTPHeaders}}}}} can now be pickled and unpickled.

\end{itemize}


\paragraph{\sphinxstyleliteralintitle{\sphinxupquote{tornado.ioloop}}}
\label{\detokenize{releases/v4.3.0:tornado-ioloop}}\begin{itemize}
\item {} 
\sphinxcode{\sphinxupquote{IOLoop(make\_current=True)}} now works as intended instead
of raising an exception.

\item {} 
The Twisted and asyncio IOLoop implementations now clear
\sphinxcode{\sphinxupquote{current()}} when they exit, like the standard IOLoops.

\item {} 
{\hyperref[\detokenize{ioloop:tornado.ioloop.IOLoop.add_callback}]{\sphinxcrossref{\sphinxcode{\sphinxupquote{IOLoop.add\_callback}}}}} is faster in the single-threaded case.

\item {} 
{\hyperref[\detokenize{ioloop:tornado.ioloop.IOLoop.add_callback}]{\sphinxcrossref{\sphinxcode{\sphinxupquote{IOLoop.add\_callback}}}}} no longer raises an error when called on
a closed IOLoop, but the callback will not be invoked.

\end{itemize}


\paragraph{\sphinxstyleliteralintitle{\sphinxupquote{tornado.iostream}}}
\label{\detokenize{releases/v4.3.0:tornado-iostream}}\begin{itemize}
\item {} 
Coroutine-style usage of {\hyperref[\detokenize{iostream:tornado.iostream.IOStream}]{\sphinxcrossref{\sphinxcode{\sphinxupquote{IOStream}}}}} now converts most errors into
{\hyperref[\detokenize{iostream:tornado.iostream.StreamClosedError}]{\sphinxcrossref{\sphinxcode{\sphinxupquote{StreamClosedError}}}}}, which has the effect of reducing log noise from
exceptions that are outside the application’s control (especially
SSL errors).

\item {} 
{\hyperref[\detokenize{iostream:tornado.iostream.StreamClosedError}]{\sphinxcrossref{\sphinxcode{\sphinxupquote{StreamClosedError}}}}} now has a \sphinxcode{\sphinxupquote{real\_error}} attribute which indicates
why the stream was closed. It is the same as the \sphinxcode{\sphinxupquote{error}} attribute of
{\hyperref[\detokenize{iostream:tornado.iostream.IOStream}]{\sphinxcrossref{\sphinxcode{\sphinxupquote{IOStream}}}}} but may be more easily accessible than the {\hyperref[\detokenize{iostream:tornado.iostream.IOStream}]{\sphinxcrossref{\sphinxcode{\sphinxupquote{IOStream}}}}} itself.

\item {} 
Improved error handling in {\hyperref[\detokenize{iostream:tornado.iostream.BaseIOStream.read_until_close}]{\sphinxcrossref{\sphinxcode{\sphinxupquote{read\_until\_close}}}}}.

\item {} 
Logging is less noisy when an SSL server is port scanned.

\item {} 
\sphinxcode{\sphinxupquote{EINTR}} is now handled on all reads.

\end{itemize}


\paragraph{\sphinxstyleliteralintitle{\sphinxupquote{tornado.locale}}}
\label{\detokenize{releases/v4.3.0:tornado-locale}}\begin{itemize}
\item {} 
{\hyperref[\detokenize{locale:tornado.locale.load_translations}]{\sphinxcrossref{\sphinxcode{\sphinxupquote{tornado.locale.load\_translations}}}}} now accepts encodings other than
UTF-8. UTF-16 and UTF-8 will be detected automatically if a BOM is
present; for other encodings {\hyperref[\detokenize{locale:tornado.locale.load_translations}]{\sphinxcrossref{\sphinxcode{\sphinxupquote{load\_translations}}}}} has an \sphinxcode{\sphinxupquote{encoding}}
parameter.

\end{itemize}


\paragraph{\sphinxstyleliteralintitle{\sphinxupquote{tornado.locks}}}
\label{\detokenize{releases/v4.3.0:tornado-locks}}\begin{itemize}
\item {} 
{\hyperref[\detokenize{locks:tornado.locks.Lock}]{\sphinxcrossref{\sphinxcode{\sphinxupquote{Lock}}}}} and {\hyperref[\detokenize{locks:tornado.locks.Semaphore}]{\sphinxcrossref{\sphinxcode{\sphinxupquote{Semaphore}}}}} now support the \sphinxcode{\sphinxupquote{async with}} statement on
Python 3.5.

\end{itemize}


\paragraph{\sphinxstyleliteralintitle{\sphinxupquote{tornado.log}}}
\label{\detokenize{releases/v4.3.0:tornado-log}}\begin{itemize}
\item {} 
A new time-based log rotation mode is available with
\sphinxcode{\sphinxupquote{-{-}log\_rotate\_mode=time}}, \sphinxcode{\sphinxupquote{-{-}log-rotate-when}}, and
\sphinxcode{\sphinxupquote{log-rotate-interval}}.

\end{itemize}


\paragraph{\sphinxstyleliteralintitle{\sphinxupquote{tornado.netutil}}}
\label{\detokenize{releases/v4.3.0:tornado-netutil}}\begin{itemize}
\item {} 
{\hyperref[\detokenize{netutil:tornado.netutil.bind_sockets}]{\sphinxcrossref{\sphinxcode{\sphinxupquote{bind\_sockets}}}}} now supports \sphinxcode{\sphinxupquote{SO\_REUSEPORT}} with the \sphinxcode{\sphinxupquote{reuse\_port=True}}
argument.

\end{itemize}


\paragraph{\sphinxstyleliteralintitle{\sphinxupquote{tornado.options}}}
\label{\detokenize{releases/v4.3.0:tornado-options}}\begin{itemize}
\item {} 
Dashes and underscores are now fully interchangeable in option names.

\end{itemize}


\paragraph{\sphinxstyleliteralintitle{\sphinxupquote{tornado.queues}}}
\label{\detokenize{releases/v4.3.0:tornado-queues}}\begin{itemize}
\item {} 
{\hyperref[\detokenize{queues:tornado.queues.Queue}]{\sphinxcrossref{\sphinxcode{\sphinxupquote{Queue}}}}} now supports the \sphinxcode{\sphinxupquote{async for}} statement on Python 3.5.

\end{itemize}


\paragraph{\sphinxstyleliteralintitle{\sphinxupquote{tornado.simple\_httpclient}}}
\label{\detokenize{releases/v4.3.0:tornado-simple-httpclient}}\begin{itemize}
\item {} 
When following redirects, \sphinxcode{\sphinxupquote{streaming\_callback}} and
\sphinxcode{\sphinxupquote{header\_callback}} will no longer be run on the redirect responses
(only the final non-redirect).

\item {} 
Responses containing both \sphinxcode{\sphinxupquote{Content-Length}} and \sphinxcode{\sphinxupquote{Transfer-Encoding}}
will be treated as an error.

\end{itemize}


\paragraph{\sphinxstyleliteralintitle{\sphinxupquote{tornado.template}}}
\label{\detokenize{releases/v4.3.0:tornado-template}}\begin{itemize}
\item {} 
{\hyperref[\detokenize{template:tornado.template.ParseError}]{\sphinxcrossref{\sphinxcode{\sphinxupquote{tornado.template.ParseError}}}}} now includes the filename in addition to
line number.

\item {} 
Whitespace handling has become more configurable. The {\hyperref[\detokenize{template:tornado.template.Loader}]{\sphinxcrossref{\sphinxcode{\sphinxupquote{Loader}}}}}
constructor now has a \sphinxcode{\sphinxupquote{whitespace}} argument, there is a new
\sphinxcode{\sphinxupquote{template\_whitespace}} {\hyperref[\detokenize{web:tornado.web.Application}]{\sphinxcrossref{\sphinxcode{\sphinxupquote{Application}}}}} setting, and there is a new
\sphinxcode{\sphinxupquote{\{\% whitespace \%\}}} template directive. All of these options take
a mode name defined in the {\hyperref[\detokenize{template:tornado.template.filter_whitespace}]{\sphinxcrossref{\sphinxcode{\sphinxupquote{tornado.template.filter\_whitespace}}}}} function.
The default mode is \sphinxcode{\sphinxupquote{single}}, which is the same behavior as prior
versions of Tornado.

\item {} 
Non-ASCII filenames are now supported.

\end{itemize}


\paragraph{\sphinxstyleliteralintitle{\sphinxupquote{tornado.testing}}}
\label{\detokenize{releases/v4.3.0:tornado-testing}}\begin{itemize}
\item {} 
{\hyperref[\detokenize{testing:tornado.testing.ExpectLog}]{\sphinxcrossref{\sphinxcode{\sphinxupquote{ExpectLog}}}}} objects now have a boolean \sphinxcode{\sphinxupquote{logged\_stack}} attribute to
make it easier to test whether an exception stack trace was logged.

\end{itemize}


\paragraph{\sphinxstyleliteralintitle{\sphinxupquote{tornado.web}}}
\label{\detokenize{releases/v4.3.0:tornado-web}}\begin{itemize}
\item {} 
The hard limit of 4000 bytes per outgoing header has been removed.

\item {} 
{\hyperref[\detokenize{web:tornado.web.StaticFileHandler}]{\sphinxcrossref{\sphinxcode{\sphinxupquote{StaticFileHandler}}}}} returns the correct \sphinxcode{\sphinxupquote{Content-Type}} for files
with \sphinxcode{\sphinxupquote{.gz}}, \sphinxcode{\sphinxupquote{.bz2}}, and \sphinxcode{\sphinxupquote{.xz}} extensions.

\item {} 
Responses smaller than 1000 bytes will no longer be compressed.

\item {} 
The default gzip compression level is now 6 (was 9).

\item {} 
Fixed a regression in Tornado 4.2.1 that broke {\hyperref[\detokenize{web:tornado.web.StaticFileHandler}]{\sphinxcrossref{\sphinxcode{\sphinxupquote{StaticFileHandler}}}}}
with a \sphinxcode{\sphinxupquote{path}} of \sphinxcode{\sphinxupquote{/}}.

\item {} 
{\hyperref[\detokenize{web:tornado.web.HTTPError}]{\sphinxcrossref{\sphinxcode{\sphinxupquote{tornado.web.HTTPError}}}}} is now copyable with the \sphinxhref{https://docs.python.org/3.6/library/copy.html\#module-copy}{\sphinxcode{\sphinxupquote{copy}}} module.

\item {} 
The exception {\hyperref[\detokenize{web:tornado.web.Finish}]{\sphinxcrossref{\sphinxcode{\sphinxupquote{Finish}}}}} now accepts an argument which will be passed to
the method {\hyperref[\detokenize{web:tornado.web.RequestHandler.finish}]{\sphinxcrossref{\sphinxcode{\sphinxupquote{RequestHandler.finish}}}}}.

\item {} 
New {\hyperref[\detokenize{web:tornado.web.Application}]{\sphinxcrossref{\sphinxcode{\sphinxupquote{Application}}}}} setting \sphinxcode{\sphinxupquote{xsrf\_cookie\_kwargs}} can be used to set
additional attributes such as \sphinxcode{\sphinxupquote{secure}} or \sphinxcode{\sphinxupquote{httponly}} on the
XSRF cookie.

\item {} 
{\hyperref[\detokenize{web:tornado.web.Application.listen}]{\sphinxcrossref{\sphinxcode{\sphinxupquote{Application.listen}}}}} now returns the {\hyperref[\detokenize{httpserver:tornado.httpserver.HTTPServer}]{\sphinxcrossref{\sphinxcode{\sphinxupquote{HTTPServer}}}}} it created.

\end{itemize}


\paragraph{\sphinxstyleliteralintitle{\sphinxupquote{tornado.websocket}}}
\label{\detokenize{releases/v4.3.0:tornado-websocket}}\begin{itemize}
\item {} 
Fixed handling of continuation frames when compression is enabled.

\end{itemize}


\subsection{What’s new in Tornado 4.2.1}
\label{\detokenize{releases/v4.2.1:what-s-new-in-tornado-4-2-1}}\label{\detokenize{releases/v4.2.1::doc}}

\subsubsection{Jul 17, 2015}
\label{\detokenize{releases/v4.2.1:jul-17-2015}}

\paragraph{Security fix}
\label{\detokenize{releases/v4.2.1:security-fix}}\begin{itemize}
\item {} 
This release fixes a path traversal vulnerability in {\hyperref[\detokenize{web:tornado.web.StaticFileHandler}]{\sphinxcrossref{\sphinxcode{\sphinxupquote{StaticFileHandler}}}}},
in which files whose names \sphinxstyleemphasis{started with} the \sphinxcode{\sphinxupquote{static\_path}} directory
but were not actually \sphinxstyleemphasis{in} that directory could be accessed.

\end{itemize}


\subsection{What’s new in Tornado 4.2}
\label{\detokenize{releases/v4.2.0:what-s-new-in-tornado-4-2}}\label{\detokenize{releases/v4.2.0::doc}}

\subsubsection{May 26, 2015}
\label{\detokenize{releases/v4.2.0:may-26-2015}}

\paragraph{Backwards-compatibility notes}
\label{\detokenize{releases/v4.2.0:backwards-compatibility-notes}}\begin{itemize}
\item {} 
\sphinxcode{\sphinxupquote{SSLIOStream.connect}} and {\hyperref[\detokenize{iostream:tornado.iostream.IOStream.start_tls}]{\sphinxcrossref{\sphinxcode{\sphinxupquote{IOStream.start\_tls}}}}} now validate certificates
by default.

\item {} 
Certificate validation will now use the system CA root certificates instead
of \sphinxcode{\sphinxupquote{certifi}} when possible (i.e. Python 2.7.9+ or 3.4+). This includes
{\hyperref[\detokenize{iostream:tornado.iostream.IOStream}]{\sphinxcrossref{\sphinxcode{\sphinxupquote{IOStream}}}}} and \sphinxcode{\sphinxupquote{simple\_httpclient}}, but not \sphinxcode{\sphinxupquote{curl\_httpclient}}.

\item {} 
The default SSL configuration has become stricter, using
\sphinxhref{https://docs.python.org/3.6/library/ssl.html\#ssl.create\_default\_context}{\sphinxcode{\sphinxupquote{ssl.create\_default\_context}}} where available on the client side.
(On the server side, applications are encouraged to migrate from the
\sphinxcode{\sphinxupquote{ssl\_options}} dict-based API to pass an \sphinxhref{https://docs.python.org/3.6/library/ssl.html\#ssl.SSLContext}{\sphinxcode{\sphinxupquote{ssl.SSLContext}}} instead).

\item {} 
The deprecated classes in the {\hyperref[\detokenize{auth:module-tornado.auth}]{\sphinxcrossref{\sphinxcode{\sphinxupquote{tornado.auth}}}}} module, \sphinxcode{\sphinxupquote{GoogleMixin}},
\sphinxcode{\sphinxupquote{FacebookMixin}}, and \sphinxcode{\sphinxupquote{FriendFeedMixin}} have been removed.

\end{itemize}


\paragraph{New modules: \sphinxstyleliteralintitle{\sphinxupquote{tornado.locks}} and \sphinxstyleliteralintitle{\sphinxupquote{tornado.queues}}}
\label{\detokenize{releases/v4.2.0:new-modules-tornado-locks-and-tornado-queues}}
These modules provide classes for coordinating coroutines, merged from
\sphinxhref{https://toro.readthedocs.io}{Toro}.

To port your code from Toro’s queues to Tornado 4.2, import {\hyperref[\detokenize{queues:tornado.queues.Queue}]{\sphinxcrossref{\sphinxcode{\sphinxupquote{Queue}}}}},
{\hyperref[\detokenize{queues:tornado.queues.PriorityQueue}]{\sphinxcrossref{\sphinxcode{\sphinxupquote{PriorityQueue}}}}}, or {\hyperref[\detokenize{queues:tornado.queues.LifoQueue}]{\sphinxcrossref{\sphinxcode{\sphinxupquote{LifoQueue}}}}} from {\hyperref[\detokenize{queues:module-tornado.queues}]{\sphinxcrossref{\sphinxcode{\sphinxupquote{tornado.queues}}}}} instead of from
\sphinxcode{\sphinxupquote{toro}}.

Use {\hyperref[\detokenize{queues:tornado.queues.Queue}]{\sphinxcrossref{\sphinxcode{\sphinxupquote{Queue}}}}} instead of Toro’s \sphinxcode{\sphinxupquote{JoinableQueue}}. In Tornado the methods
{\hyperref[\detokenize{queues:tornado.queues.Queue.join}]{\sphinxcrossref{\sphinxcode{\sphinxupquote{join}}}}} and {\hyperref[\detokenize{queues:tornado.queues.Queue.task_done}]{\sphinxcrossref{\sphinxcode{\sphinxupquote{task\_done}}}}} are available on all queues, not on a
special \sphinxcode{\sphinxupquote{JoinableQueue}}.

Tornado queues raise exceptions specific to Tornado instead of reusing
exceptions from the Python standard library.
Therefore instead of catching the standard \sphinxhref{https://docs.python.org/3.6/library/queue.html\#queue.Empty}{\sphinxcode{\sphinxupquote{queue.Empty}}} exception from
{\hyperref[\detokenize{queues:tornado.queues.Queue.get_nowait}]{\sphinxcrossref{\sphinxcode{\sphinxupquote{Queue.get\_nowait}}}}}, catch the special {\hyperref[\detokenize{queues:tornado.queues.QueueEmpty}]{\sphinxcrossref{\sphinxcode{\sphinxupquote{tornado.queues.QueueEmpty}}}}} exception,
and instead of catching the standard \sphinxhref{https://docs.python.org/3.6/library/queue.html\#queue.Full}{\sphinxcode{\sphinxupquote{queue.Full}}} from {\hyperref[\detokenize{queues:tornado.queues.Queue.get_nowait}]{\sphinxcrossref{\sphinxcode{\sphinxupquote{Queue.get\_nowait}}}}},
catch {\hyperref[\detokenize{queues:tornado.queues.QueueFull}]{\sphinxcrossref{\sphinxcode{\sphinxupquote{tornado.queues.QueueFull}}}}}.

To port from Toro’s locks to Tornado 4.2, import {\hyperref[\detokenize{locks:tornado.locks.Condition}]{\sphinxcrossref{\sphinxcode{\sphinxupquote{Condition}}}}}, {\hyperref[\detokenize{locks:tornado.locks.Event}]{\sphinxcrossref{\sphinxcode{\sphinxupquote{Event}}}}},
{\hyperref[\detokenize{locks:tornado.locks.Semaphore}]{\sphinxcrossref{\sphinxcode{\sphinxupquote{Semaphore}}}}}, {\hyperref[\detokenize{locks:tornado.locks.BoundedSemaphore}]{\sphinxcrossref{\sphinxcode{\sphinxupquote{BoundedSemaphore}}}}}, or {\hyperref[\detokenize{locks:tornado.locks.Lock}]{\sphinxcrossref{\sphinxcode{\sphinxupquote{Lock}}}}} from {\hyperref[\detokenize{locks:module-tornado.locks}]{\sphinxcrossref{\sphinxcode{\sphinxupquote{tornado.locks}}}}}
instead of from \sphinxcode{\sphinxupquote{toro}}.

Toro’s \sphinxcode{\sphinxupquote{Semaphore.wait}} allowed a coroutine to wait for the semaphore to
be unlocked \sphinxstyleemphasis{without} acquiring it. This encouraged unorthodox patterns; in
Tornado, just use {\hyperref[\detokenize{locks:tornado.locks.Semaphore.acquire}]{\sphinxcrossref{\sphinxcode{\sphinxupquote{acquire}}}}}.

Toro’s \sphinxcode{\sphinxupquote{Event.wait}} raised a \sphinxcode{\sphinxupquote{Timeout}} exception after a timeout. In
Tornado, {\hyperref[\detokenize{locks:tornado.locks.Event.wait}]{\sphinxcrossref{\sphinxcode{\sphinxupquote{Event.wait}}}}} raises \sphinxcode{\sphinxupquote{tornado.gen.TimeoutError}}.

Toro’s \sphinxcode{\sphinxupquote{Condition.wait}} also raised \sphinxcode{\sphinxupquote{Timeout}}, but in Tornado, the {\hyperref[\detokenize{concurrent:tornado.concurrent.Future}]{\sphinxcrossref{\sphinxcode{\sphinxupquote{Future}}}}}
returned by {\hyperref[\detokenize{locks:tornado.locks.Condition.wait}]{\sphinxcrossref{\sphinxcode{\sphinxupquote{Condition.wait}}}}} resolves to False after a timeout:

\begin{sphinxVerbatim}[commandchars=\\\{\}]
\PYG{n+nd}{@gen}\PYG{o}{.}\PYG{n}{coroutine}
\PYG{k}{def} \PYG{n+nf}{await\PYGZus{}notification}\PYG{p}{(}\PYG{p}{)}\PYG{p}{:}
    \PYG{k}{if} \PYG{o+ow}{not} \PYG{p}{(}\PYG{k}{yield} \PYG{n}{condition}\PYG{o}{.}\PYG{n}{wait}\PYG{p}{(}\PYG{n}{timeout}\PYG{o}{=}\PYG{n}{timedelta}\PYG{p}{(}\PYG{n}{seconds}\PYG{o}{=}\PYG{l+m+mi}{1}\PYG{p}{)}\PYG{p}{)}\PYG{p}{)}\PYG{p}{:}
        \PYG{n+nb}{print}\PYG{p}{(}\PYG{l+s+s1}{\PYGZsq{}}\PYG{l+s+s1}{timed out}\PYG{l+s+s1}{\PYGZsq{}}\PYG{p}{)}
    \PYG{k}{else}\PYG{p}{:}
        \PYG{n+nb}{print}\PYG{p}{(}\PYG{l+s+s1}{\PYGZsq{}}\PYG{l+s+s1}{condition is true}\PYG{l+s+s1}{\PYGZsq{}}\PYG{p}{)}
\end{sphinxVerbatim}

In lock and queue methods, wherever Toro accepted \sphinxcode{\sphinxupquote{deadline}} as a keyword
argument, Tornado names the argument \sphinxcode{\sphinxupquote{timeout}} instead.

Toro’s \sphinxcode{\sphinxupquote{AsyncResult}} is not merged into Tornado, nor its exceptions
\sphinxcode{\sphinxupquote{NotReady}} and \sphinxcode{\sphinxupquote{AlreadySet}}. Use a {\hyperref[\detokenize{concurrent:tornado.concurrent.Future}]{\sphinxcrossref{\sphinxcode{\sphinxupquote{Future}}}}} instead. If you wrote code like
this:

\begin{sphinxVerbatim}[commandchars=\\\{\}]
\PYG{k+kn}{from} \PYG{n+nn}{tornado} \PYG{k}{import} \PYG{n}{gen}
\PYG{k+kn}{import} \PYG{n+nn}{toro}

\PYG{n}{result} \PYG{o}{=} \PYG{n}{toro}\PYG{o}{.}\PYG{n}{AsyncResult}\PYG{p}{(}\PYG{p}{)}

\PYG{n+nd}{@gen}\PYG{o}{.}\PYG{n}{coroutine}
\PYG{k}{def} \PYG{n+nf}{setter}\PYG{p}{(}\PYG{p}{)}\PYG{p}{:}
    \PYG{n}{result}\PYG{o}{.}\PYG{n}{set}\PYG{p}{(}\PYG{l+m+mi}{1}\PYG{p}{)}

\PYG{n+nd}{@gen}\PYG{o}{.}\PYG{n}{coroutine}
\PYG{k}{def} \PYG{n+nf}{getter}\PYG{p}{(}\PYG{p}{)}\PYG{p}{:}
    \PYG{n}{value} \PYG{o}{=} \PYG{k}{yield} \PYG{n}{result}\PYG{o}{.}\PYG{n}{get}\PYG{p}{(}\PYG{p}{)}
    \PYG{n+nb}{print}\PYG{p}{(}\PYG{n}{value}\PYG{p}{)}  \PYG{c+c1}{\PYGZsh{} Prints \PYGZdq{}1\PYGZdq{}.}
\end{sphinxVerbatim}

Then the Tornado equivalent is:

\begin{sphinxVerbatim}[commandchars=\\\{\}]
\PYG{k+kn}{from} \PYG{n+nn}{tornado} \PYG{k}{import} \PYG{n}{gen}
\PYG{k+kn}{from} \PYG{n+nn}{tornado}\PYG{n+nn}{.}\PYG{n+nn}{concurrent} \PYG{k}{import} \PYG{n}{Future}

\PYG{n}{result} \PYG{o}{=} \PYG{n}{Future}\PYG{p}{(}\PYG{p}{)}

\PYG{n+nd}{@gen}\PYG{o}{.}\PYG{n}{coroutine}
\PYG{k}{def} \PYG{n+nf}{setter}\PYG{p}{(}\PYG{p}{)}\PYG{p}{:}
    \PYG{n}{result}\PYG{o}{.}\PYG{n}{set\PYGZus{}result}\PYG{p}{(}\PYG{l+m+mi}{1}\PYG{p}{)}

\PYG{n+nd}{@gen}\PYG{o}{.}\PYG{n}{coroutine}
\PYG{k}{def} \PYG{n+nf}{getter}\PYG{p}{(}\PYG{p}{)}\PYG{p}{:}
    \PYG{n}{value} \PYG{o}{=} \PYG{k}{yield} \PYG{n}{result}
    \PYG{n+nb}{print}\PYG{p}{(}\PYG{n}{value}\PYG{p}{)}  \PYG{c+c1}{\PYGZsh{} Prints \PYGZdq{}1\PYGZdq{}.}
\end{sphinxVerbatim}


\paragraph{\sphinxstyleliteralintitle{\sphinxupquote{tornado.autoreload}}}
\label{\detokenize{releases/v4.2.0:tornado-autoreload}}\begin{itemize}
\item {} 
Improved compatibility with Windows.

\item {} 
Fixed a bug in Python 3 if a module was imported during a reload check.

\end{itemize}


\paragraph{\sphinxstyleliteralintitle{\sphinxupquote{tornado.concurrent}}}
\label{\detokenize{releases/v4.2.0:tornado-concurrent}}\begin{itemize}
\item {} 
{\hyperref[\detokenize{concurrent:tornado.concurrent.run_on_executor}]{\sphinxcrossref{\sphinxcode{\sphinxupquote{run\_on\_executor}}}}} now accepts arguments to control which attributes
it uses to find the {\hyperref[\detokenize{ioloop:tornado.ioloop.IOLoop}]{\sphinxcrossref{\sphinxcode{\sphinxupquote{IOLoop}}}}} and executor.

\end{itemize}


\paragraph{\sphinxstyleliteralintitle{\sphinxupquote{tornado.curl\_httpclient}}}
\label{\detokenize{releases/v4.2.0:tornado-curl-httpclient}}\begin{itemize}
\item {} 
Fixed a bug that would cause the client to stop processing requests
if an exception occurred in certain places while there is a queue.

\end{itemize}


\paragraph{\sphinxstyleliteralintitle{\sphinxupquote{tornado.escape}}}
\label{\detokenize{releases/v4.2.0:tornado-escape}}\begin{itemize}
\item {} 
{\hyperref[\detokenize{escape:tornado.escape.xhtml_escape}]{\sphinxcrossref{\sphinxcode{\sphinxupquote{xhtml\_escape}}}}} now supports numeric character references in hex
format (\sphinxcode{\sphinxupquote{\&\#x20;}})

\end{itemize}


\paragraph{\sphinxstyleliteralintitle{\sphinxupquote{tornado.gen}}}
\label{\detokenize{releases/v4.2.0:tornado-gen}}\begin{itemize}
\item {} 
{\hyperref[\detokenize{gen:tornado.gen.WaitIterator}]{\sphinxcrossref{\sphinxcode{\sphinxupquote{WaitIterator}}}}} no longer uses weak references, which fixes several
garbage-collection-related bugs.

\item {} 
\sphinxcode{\sphinxupquote{tornado.gen.Multi}} and {\hyperref[\detokenize{gen:tornado.gen.multi_future}]{\sphinxcrossref{\sphinxcode{\sphinxupquote{tornado.gen.multi\_future}}}}} (which are used when
yielding a list or dict in a coroutine) now log any exceptions after the
first if more than one {\hyperref[\detokenize{concurrent:tornado.concurrent.Future}]{\sphinxcrossref{\sphinxcode{\sphinxupquote{Future}}}}} fails (previously they would be logged
when the {\hyperref[\detokenize{concurrent:tornado.concurrent.Future}]{\sphinxcrossref{\sphinxcode{\sphinxupquote{Future}}}}} was garbage-collected, but this is more reliable).
Both have a new keyword argument \sphinxcode{\sphinxupquote{quiet\_exceptions}} to suppress
logging of certain exception types; to use this argument you must
call \sphinxcode{\sphinxupquote{Multi}} or \sphinxcode{\sphinxupquote{multi\_future}} directly instead of simply yielding
a list.

\item {} 
{\hyperref[\detokenize{gen:tornado.gen.multi_future}]{\sphinxcrossref{\sphinxcode{\sphinxupquote{multi\_future}}}}} now works when given multiple copies of the same {\hyperref[\detokenize{concurrent:tornado.concurrent.Future}]{\sphinxcrossref{\sphinxcode{\sphinxupquote{Future}}}}}.

\item {} 
On Python 3, catching an exception in a coroutine no longer leads to
leaks via \sphinxcode{\sphinxupquote{Exception.\_\_context\_\_}}.

\end{itemize}


\paragraph{\sphinxstyleliteralintitle{\sphinxupquote{tornado.httpclient}}}
\label{\detokenize{releases/v4.2.0:tornado-httpclient}}\begin{itemize}
\item {} 
The \sphinxcode{\sphinxupquote{raise\_error}} argument now works correctly with the synchronous
{\hyperref[\detokenize{httpclient:tornado.httpclient.HTTPClient}]{\sphinxcrossref{\sphinxcode{\sphinxupquote{HTTPClient}}}}}.

\item {} 
The synchronous {\hyperref[\detokenize{httpclient:tornado.httpclient.HTTPClient}]{\sphinxcrossref{\sphinxcode{\sphinxupquote{HTTPClient}}}}} no longer interferes with {\hyperref[\detokenize{ioloop:tornado.ioloop.IOLoop.current}]{\sphinxcrossref{\sphinxcode{\sphinxupquote{IOLoop.current()}}}}}.

\end{itemize}


\paragraph{\sphinxstyleliteralintitle{\sphinxupquote{tornado.httpserver}}}
\label{\detokenize{releases/v4.2.0:tornado-httpserver}}\begin{itemize}
\item {} 
{\hyperref[\detokenize{httpserver:tornado.httpserver.HTTPServer}]{\sphinxcrossref{\sphinxcode{\sphinxupquote{HTTPServer}}}}} is now a subclass of {\hyperref[\detokenize{util:tornado.util.Configurable}]{\sphinxcrossref{\sphinxcode{\sphinxupquote{tornado.util.Configurable}}}}}.

\end{itemize}


\paragraph{\sphinxstyleliteralintitle{\sphinxupquote{tornado.httputil}}}
\label{\detokenize{releases/v4.2.0:tornado-httputil}}\begin{itemize}
\item {} 
{\hyperref[\detokenize{httputil:tornado.httputil.HTTPHeaders}]{\sphinxcrossref{\sphinxcode{\sphinxupquote{HTTPHeaders}}}}} can now be copied with \sphinxhref{https://docs.python.org/3.6/library/copy.html\#copy.copy}{\sphinxcode{\sphinxupquote{copy.copy}}} and \sphinxhref{https://docs.python.org/3.6/library/copy.html\#copy.deepcopy}{\sphinxcode{\sphinxupquote{copy.deepcopy}}}.

\end{itemize}


\paragraph{\sphinxstyleliteralintitle{\sphinxupquote{tornado.ioloop}}}
\label{\detokenize{releases/v4.2.0:tornado-ioloop}}\begin{itemize}
\item {} 
The {\hyperref[\detokenize{ioloop:tornado.ioloop.IOLoop}]{\sphinxcrossref{\sphinxcode{\sphinxupquote{IOLoop}}}}} constructor now has a \sphinxcode{\sphinxupquote{make\_current}} keyword argument
to control whether the new {\hyperref[\detokenize{ioloop:tornado.ioloop.IOLoop}]{\sphinxcrossref{\sphinxcode{\sphinxupquote{IOLoop}}}}} becomes {\hyperref[\detokenize{ioloop:tornado.ioloop.IOLoop.current}]{\sphinxcrossref{\sphinxcode{\sphinxupquote{IOLoop.current()}}}}}.

\item {} 
Third-party implementations of {\hyperref[\detokenize{ioloop:tornado.ioloop.IOLoop}]{\sphinxcrossref{\sphinxcode{\sphinxupquote{IOLoop}}}}} should accept \sphinxcode{\sphinxupquote{**kwargs}}
in their \sphinxcode{\sphinxupquote{IOLoop.initialize}} methods and pass them to the superclass
implementation.

\item {} 
{\hyperref[\detokenize{ioloop:tornado.ioloop.PeriodicCallback}]{\sphinxcrossref{\sphinxcode{\sphinxupquote{PeriodicCallback}}}}} is now more efficient when the clock jumps forward
by a large amount.

\end{itemize}


\paragraph{\sphinxstyleliteralintitle{\sphinxupquote{tornado.iostream}}}
\label{\detokenize{releases/v4.2.0:tornado-iostream}}\begin{itemize}
\item {} 
\sphinxcode{\sphinxupquote{SSLIOStream.connect}} and {\hyperref[\detokenize{iostream:tornado.iostream.IOStream.start_tls}]{\sphinxcrossref{\sphinxcode{\sphinxupquote{IOStream.start\_tls}}}}} now validate certificates
by default.

\item {} 
New method {\hyperref[\detokenize{iostream:tornado.iostream.SSLIOStream.wait_for_handshake}]{\sphinxcrossref{\sphinxcode{\sphinxupquote{SSLIOStream.wait\_for\_handshake}}}}} allows server-side applications
to wait for the handshake to complete in order to verify client certificates
or use NPN/ALPN.

\item {} 
The {\hyperref[\detokenize{concurrent:tornado.concurrent.Future}]{\sphinxcrossref{\sphinxcode{\sphinxupquote{Future}}}}} returned by \sphinxcode{\sphinxupquote{SSLIOStream.connect}} now resolves after the
handshake is complete instead of as soon as the TCP connection is
established.

\item {} 
Reduced logging of SSL errors.

\item {} 
{\hyperref[\detokenize{iostream:tornado.iostream.BaseIOStream.read_until_close}]{\sphinxcrossref{\sphinxcode{\sphinxupquote{BaseIOStream.read\_until\_close}}}}} now works correctly when a
\sphinxcode{\sphinxupquote{streaming\_callback}} is given but \sphinxcode{\sphinxupquote{callback}} is None (i.e. when
it returns a {\hyperref[\detokenize{concurrent:tornado.concurrent.Future}]{\sphinxcrossref{\sphinxcode{\sphinxupquote{Future}}}}})

\end{itemize}


\paragraph{\sphinxstyleliteralintitle{\sphinxupquote{tornado.locale}}}
\label{\detokenize{releases/v4.2.0:tornado-locale}}\begin{itemize}
\item {} 
New method {\hyperref[\detokenize{locale:tornado.locale.GettextLocale.pgettext}]{\sphinxcrossref{\sphinxcode{\sphinxupquote{GettextLocale.pgettext}}}}} allows additional context to be
supplied for gettext translations.

\end{itemize}


\paragraph{\sphinxstyleliteralintitle{\sphinxupquote{tornado.log}}}
\label{\detokenize{releases/v4.2.0:tornado-log}}\begin{itemize}
\item {} 
{\hyperref[\detokenize{log:tornado.log.define_logging_options}]{\sphinxcrossref{\sphinxcode{\sphinxupquote{define\_logging\_options}}}}} now works correctly when given a non-default
\sphinxcode{\sphinxupquote{options}} object.

\end{itemize}


\paragraph{\sphinxstyleliteralintitle{\sphinxupquote{tornado.process}}}
\label{\detokenize{releases/v4.2.0:tornado-process}}\begin{itemize}
\item {} 
New method {\hyperref[\detokenize{process:tornado.process.Subprocess.wait_for_exit}]{\sphinxcrossref{\sphinxcode{\sphinxupquote{Subprocess.wait\_for\_exit}}}}} is a coroutine-friendly
version of {\hyperref[\detokenize{process:tornado.process.Subprocess.set_exit_callback}]{\sphinxcrossref{\sphinxcode{\sphinxupquote{Subprocess.set\_exit\_callback}}}}}.

\end{itemize}


\paragraph{\sphinxstyleliteralintitle{\sphinxupquote{tornado.simple\_httpclient}}}
\label{\detokenize{releases/v4.2.0:tornado-simple-httpclient}}\begin{itemize}
\item {} 
Improved performance on Python 3 by reusing a single \sphinxhref{https://docs.python.org/3.6/library/ssl.html\#ssl.SSLContext}{\sphinxcode{\sphinxupquote{ssl.SSLContext}}}.

\item {} 
New constructor argument \sphinxcode{\sphinxupquote{max\_body\_size}} controls the maximum response
size the client is willing to accept. It may be bigger than
\sphinxcode{\sphinxupquote{max\_buffer\_size}} if \sphinxcode{\sphinxupquote{streaming\_callback}} is used.

\end{itemize}


\paragraph{\sphinxstyleliteralintitle{\sphinxupquote{tornado.tcpserver}}}
\label{\detokenize{releases/v4.2.0:tornado-tcpserver}}\begin{itemize}
\item {} 
{\hyperref[\detokenize{tcpserver:tornado.tcpserver.TCPServer.handle_stream}]{\sphinxcrossref{\sphinxcode{\sphinxupquote{TCPServer.handle\_stream}}}}} may be a coroutine (so that any exceptions
it raises will be logged).

\end{itemize}


\paragraph{\sphinxstyleliteralintitle{\sphinxupquote{tornado.util}}}
\label{\detokenize{releases/v4.2.0:tornado-util}}\begin{itemize}
\item {} 
{\hyperref[\detokenize{util:tornado.util.import_object}]{\sphinxcrossref{\sphinxcode{\sphinxupquote{import\_object}}}}} now supports unicode strings on Python 2.

\item {} 
{\hyperref[\detokenize{util:tornado.util.Configurable.initialize}]{\sphinxcrossref{\sphinxcode{\sphinxupquote{Configurable.initialize}}}}} now supports positional arguments.

\end{itemize}


\paragraph{\sphinxstyleliteralintitle{\sphinxupquote{tornado.web}}}
\label{\detokenize{releases/v4.2.0:tornado-web}}\begin{itemize}
\item {} 
Key versioning support for cookie signing. \sphinxcode{\sphinxupquote{cookie\_secret}} application
setting can now contain a dict of valid keys with version as key. The
current signing key then must be specified via \sphinxcode{\sphinxupquote{key\_version}} setting.

\item {} 
Parsing of the \sphinxcode{\sphinxupquote{If-None-Match}} header now follows the RFC and supports
weak validators.

\item {} 
Passing \sphinxcode{\sphinxupquote{secure=False}} or \sphinxcode{\sphinxupquote{httponly=False}} to
{\hyperref[\detokenize{web:tornado.web.RequestHandler.set_cookie}]{\sphinxcrossref{\sphinxcode{\sphinxupquote{RequestHandler.set\_cookie}}}}} now works as expected (previously only the
presence of the argument was considered and its value was ignored).

\item {} 
{\hyperref[\detokenize{web:tornado.web.RequestHandler.get_arguments}]{\sphinxcrossref{\sphinxcode{\sphinxupquote{RequestHandler.get\_arguments}}}}} now requires that its \sphinxcode{\sphinxupquote{strip}} argument
be of type bool. This helps prevent errors caused by the slightly dissimilar
interfaces between the singular and plural methods.

\item {} 
Errors raised in \sphinxcode{\sphinxupquote{\_handle\_request\_exception}} are now logged more reliably.

\item {} 
{\hyperref[\detokenize{web:tornado.web.RequestHandler.redirect}]{\sphinxcrossref{\sphinxcode{\sphinxupquote{RequestHandler.redirect}}}}} now works correctly when called from a handler
whose path begins with two slashes.

\item {} 
Passing messages containing \sphinxcode{\sphinxupquote{\%}} characters to {\hyperref[\detokenize{web:tornado.web.HTTPError}]{\sphinxcrossref{\sphinxcode{\sphinxupquote{tornado.web.HTTPError}}}}}
no longer causes broken error messages.

\end{itemize}


\paragraph{\sphinxstyleliteralintitle{\sphinxupquote{tornado.websocket}}}
\label{\detokenize{releases/v4.2.0:tornado-websocket}}\begin{itemize}
\item {} 
The \sphinxcode{\sphinxupquote{on\_close}} method will no longer be called more than once.

\item {} 
When the other side closes a connection, we now echo the received close
code back instead of sending an empty close frame.

\end{itemize}


\subsection{What’s new in Tornado 4.1}
\label{\detokenize{releases/v4.1.0:what-s-new-in-tornado-4-1}}\label{\detokenize{releases/v4.1.0::doc}}

\subsubsection{Feb 7, 2015}
\label{\detokenize{releases/v4.1.0:feb-7-2015}}

\paragraph{Highlights}
\label{\detokenize{releases/v4.1.0:highlights}}\begin{itemize}
\item {} 
If a {\hyperref[\detokenize{concurrent:tornado.concurrent.Future}]{\sphinxcrossref{\sphinxcode{\sphinxupquote{Future}}}}} contains an exception but that exception is never
examined or re-raised (e.g. by yielding the {\hyperref[\detokenize{concurrent:tornado.concurrent.Future}]{\sphinxcrossref{\sphinxcode{\sphinxupquote{Future}}}}}), a stack
trace will be logged when the {\hyperref[\detokenize{concurrent:tornado.concurrent.Future}]{\sphinxcrossref{\sphinxcode{\sphinxupquote{Future}}}}} is garbage-collected.

\item {} 
New class {\hyperref[\detokenize{gen:tornado.gen.WaitIterator}]{\sphinxcrossref{\sphinxcode{\sphinxupquote{tornado.gen.WaitIterator}}}}} provides a way to iterate
over \sphinxcode{\sphinxupquote{Futures}} in the order they resolve.

\item {} 
The {\hyperref[\detokenize{websocket:module-tornado.websocket}]{\sphinxcrossref{\sphinxcode{\sphinxupquote{tornado.websocket}}}}} module now supports compression via the
“permessage-deflate” extension.  Override
{\hyperref[\detokenize{websocket:tornado.websocket.WebSocketHandler.get_compression_options}]{\sphinxcrossref{\sphinxcode{\sphinxupquote{WebSocketHandler.get\_compression\_options}}}}} to enable on the server
side, and use the \sphinxcode{\sphinxupquote{compression\_options}} keyword argument to
{\hyperref[\detokenize{websocket:tornado.websocket.websocket_connect}]{\sphinxcrossref{\sphinxcode{\sphinxupquote{websocket\_connect}}}}} on the client side.

\item {} 
When the appropriate packages are installed, it is possible to yield
\sphinxhref{https://docs.python.org/3.6/library/asyncio-task.html\#asyncio.Future}{\sphinxcode{\sphinxupquote{asyncio.Future}}} or Twisted \sphinxcode{\sphinxupquote{Defered}} objects in Tornado coroutines.

\end{itemize}


\paragraph{Backwards-compatibility notes}
\label{\detokenize{releases/v4.1.0:backwards-compatibility-notes}}\begin{itemize}
\item {} 
{\hyperref[\detokenize{httpserver:tornado.httpserver.HTTPServer}]{\sphinxcrossref{\sphinxcode{\sphinxupquote{HTTPServer}}}}} now calls \sphinxcode{\sphinxupquote{start\_request}} with the correct
arguments.  This change is backwards-incompatible, affecting any
application which implemented {\hyperref[\detokenize{httputil:tornado.httputil.HTTPServerConnectionDelegate}]{\sphinxcrossref{\sphinxcode{\sphinxupquote{HTTPServerConnectionDelegate}}}}} by
following the example of {\hyperref[\detokenize{web:tornado.web.Application}]{\sphinxcrossref{\sphinxcode{\sphinxupquote{Application}}}}} instead of the documented
method signatures.

\end{itemize}


\paragraph{\sphinxstyleliteralintitle{\sphinxupquote{tornado.concurrent}}}
\label{\detokenize{releases/v4.1.0:tornado-concurrent}}\begin{itemize}
\item {} 
If a {\hyperref[\detokenize{concurrent:tornado.concurrent.Future}]{\sphinxcrossref{\sphinxcode{\sphinxupquote{Future}}}}} contains an exception but that exception is never
examined or re-raised (e.g. by yielding the {\hyperref[\detokenize{concurrent:tornado.concurrent.Future}]{\sphinxcrossref{\sphinxcode{\sphinxupquote{Future}}}}}), a stack
trace will be logged when the {\hyperref[\detokenize{concurrent:tornado.concurrent.Future}]{\sphinxcrossref{\sphinxcode{\sphinxupquote{Future}}}}} is garbage-collected.

\item {} 
{\hyperref[\detokenize{concurrent:tornado.concurrent.Future}]{\sphinxcrossref{\sphinxcode{\sphinxupquote{Future}}}}} now catches and logs exceptions in its callbacks.

\end{itemize}


\paragraph{\sphinxstyleliteralintitle{\sphinxupquote{tornado.curl\_httpclient}}}
\label{\detokenize{releases/v4.1.0:tornado-curl-httpclient}}\begin{itemize}
\item {} 
\sphinxcode{\sphinxupquote{tornado.curl\_httpclient}} now supports request bodies for \sphinxcode{\sphinxupquote{PATCH}}
and custom methods.

\item {} 
\sphinxcode{\sphinxupquote{tornado.curl\_httpclient}} now supports resubmitting bodies after
following redirects for methods other than \sphinxcode{\sphinxupquote{POST}}.

\item {} 
\sphinxcode{\sphinxupquote{curl\_httpclient}} now runs the streaming and header callbacks on
the IOLoop.

\item {} 
\sphinxcode{\sphinxupquote{tornado.curl\_httpclient}} now uses its own logger for debug output
so it can be filtered more easily.

\end{itemize}


\paragraph{\sphinxstyleliteralintitle{\sphinxupquote{tornado.gen}}}
\label{\detokenize{releases/v4.1.0:tornado-gen}}\begin{itemize}
\item {} 
New class {\hyperref[\detokenize{gen:tornado.gen.WaitIterator}]{\sphinxcrossref{\sphinxcode{\sphinxupquote{tornado.gen.WaitIterator}}}}} provides a way to iterate
over \sphinxcode{\sphinxupquote{Futures}} in the order they resolve.

\item {} 
When the \sphinxhref{https://docs.python.org/3.6/library/functools.html\#functools.singledispatch}{\sphinxcode{\sphinxupquote{singledispatch}}} library is available (standard on
Python 3.4, available via \sphinxcode{\sphinxupquote{pip install singledispatch}} on older versions),
the {\hyperref[\detokenize{gen:tornado.gen.convert_yielded}]{\sphinxcrossref{\sphinxcode{\sphinxupquote{convert\_yielded}}}}} function can be used to make other kinds of objects
yieldable in coroutines.

\item {} 
New function {\hyperref[\detokenize{gen:tornado.gen.sleep}]{\sphinxcrossref{\sphinxcode{\sphinxupquote{tornado.gen.sleep}}}}} is a coroutine-friendly
analogue to \sphinxhref{https://docs.python.org/3.6/library/time.html\#time.sleep}{\sphinxcode{\sphinxupquote{time.sleep}}}.

\item {} 
\sphinxcode{\sphinxupquote{gen.engine}} now correctly captures the stack context for its callbacks.

\end{itemize}


\paragraph{\sphinxstyleliteralintitle{\sphinxupquote{tornado.httpclient}}}
\label{\detokenize{releases/v4.1.0:tornado-httpclient}}\begin{itemize}
\item {} 
{\hyperref[\detokenize{httpclient:tornado.httpclient.HTTPRequest}]{\sphinxcrossref{\sphinxcode{\sphinxupquote{tornado.httpclient.HTTPRequest}}}}} accepts a new argument
\sphinxcode{\sphinxupquote{raise\_error=False}} to suppress the default behavior of raising an
error for non-200 response codes.

\end{itemize}


\paragraph{\sphinxstyleliteralintitle{\sphinxupquote{tornado.httpserver}}}
\label{\detokenize{releases/v4.1.0:tornado-httpserver}}\begin{itemize}
\item {} 
{\hyperref[\detokenize{httpserver:tornado.httpserver.HTTPServer}]{\sphinxcrossref{\sphinxcode{\sphinxupquote{HTTPServer}}}}} now calls \sphinxcode{\sphinxupquote{start\_request}} with the correct
arguments.  This change is backwards-incompatible, afffecting any
application which implemented {\hyperref[\detokenize{httputil:tornado.httputil.HTTPServerConnectionDelegate}]{\sphinxcrossref{\sphinxcode{\sphinxupquote{HTTPServerConnectionDelegate}}}}} by
following the example of {\hyperref[\detokenize{web:tornado.web.Application}]{\sphinxcrossref{\sphinxcode{\sphinxupquote{Application}}}}} instead of the documented
method signatures.

\item {} 
{\hyperref[\detokenize{httpserver:tornado.httpserver.HTTPServer}]{\sphinxcrossref{\sphinxcode{\sphinxupquote{HTTPServer}}}}} now tolerates extra newlines which are sometimes inserted
between requests on keep-alive connections.

\item {} 
{\hyperref[\detokenize{httpserver:tornado.httpserver.HTTPServer}]{\sphinxcrossref{\sphinxcode{\sphinxupquote{HTTPServer}}}}} can now use keep-alive connections after a request
with a chunked body.

\item {} 
{\hyperref[\detokenize{httpserver:tornado.httpserver.HTTPServer}]{\sphinxcrossref{\sphinxcode{\sphinxupquote{HTTPServer}}}}} now always reports \sphinxcode{\sphinxupquote{HTTP/1.1}} instead of echoing
the request version.

\end{itemize}


\paragraph{\sphinxstyleliteralintitle{\sphinxupquote{tornado.httputil}}}
\label{\detokenize{releases/v4.1.0:tornado-httputil}}\begin{itemize}
\item {} 
New function {\hyperref[\detokenize{httputil:tornado.httputil.split_host_and_port}]{\sphinxcrossref{\sphinxcode{\sphinxupquote{tornado.httputil.split\_host\_and\_port}}}}} for parsing
the \sphinxcode{\sphinxupquote{netloc}} portion of URLs.

\item {} 
The \sphinxcode{\sphinxupquote{context}} argument to {\hyperref[\detokenize{httputil:tornado.httputil.HTTPServerRequest}]{\sphinxcrossref{\sphinxcode{\sphinxupquote{HTTPServerRequest}}}}} is now optional,
and if a context is supplied the \sphinxcode{\sphinxupquote{remote\_ip}} attribute is also optional.

\item {} 
{\hyperref[\detokenize{httputil:tornado.httputil.HTTPServerRequest.body}]{\sphinxcrossref{\sphinxcode{\sphinxupquote{HTTPServerRequest.body}}}}} is now always a byte string (previously the default
empty body would be a unicode string on python 3).

\item {} 
Header parsing now works correctly when newline-like unicode characters
are present.

\item {} 
Header parsing again supports both CRLF and bare LF line separators.

\item {} 
Malformed \sphinxcode{\sphinxupquote{multipart/form-data}} bodies will always be logged
quietly instead of raising an unhandled exception; previously
the behavior was inconsistent depending on the exact error.

\end{itemize}


\paragraph{\sphinxstyleliteralintitle{\sphinxupquote{tornado.ioloop}}}
\label{\detokenize{releases/v4.1.0:tornado-ioloop}}\begin{itemize}
\item {} 
The \sphinxcode{\sphinxupquote{kqueue}} and \sphinxcode{\sphinxupquote{select}} IOLoop implementations now report
writeability correctly, fixing flow control in IOStream.

\item {} 
When a new {\hyperref[\detokenize{ioloop:tornado.ioloop.IOLoop}]{\sphinxcrossref{\sphinxcode{\sphinxupquote{IOLoop}}}}} is created, it automatically becomes “current”
for the thread if there is not already a current instance.

\item {} 
New method {\hyperref[\detokenize{ioloop:tornado.ioloop.PeriodicCallback.is_running}]{\sphinxcrossref{\sphinxcode{\sphinxupquote{PeriodicCallback.is\_running}}}}} can be used to see
whether the {\hyperref[\detokenize{ioloop:tornado.ioloop.PeriodicCallback}]{\sphinxcrossref{\sphinxcode{\sphinxupquote{PeriodicCallback}}}}} has been started.

\end{itemize}


\paragraph{\sphinxstyleliteralintitle{\sphinxupquote{tornado.iostream}}}
\label{\detokenize{releases/v4.1.0:tornado-iostream}}\begin{itemize}
\item {} 
{\hyperref[\detokenize{iostream:tornado.iostream.IOStream.start_tls}]{\sphinxcrossref{\sphinxcode{\sphinxupquote{IOStream.start\_tls}}}}} now uses the \sphinxcode{\sphinxupquote{server\_hostname}} parameter
for certificate validation.

\item {} 
{\hyperref[\detokenize{iostream:tornado.iostream.SSLIOStream}]{\sphinxcrossref{\sphinxcode{\sphinxupquote{SSLIOStream}}}}} will no longer consume 100\% CPU after certain error conditions.

\item {} 
{\hyperref[\detokenize{iostream:tornado.iostream.SSLIOStream}]{\sphinxcrossref{\sphinxcode{\sphinxupquote{SSLIOStream}}}}} no longer logs \sphinxcode{\sphinxupquote{EBADF}} errors during the handshake as they
can result from nmap scans in certain modes.

\end{itemize}


\paragraph{\sphinxstyleliteralintitle{\sphinxupquote{tornado.options}}}
\label{\detokenize{releases/v4.1.0:tornado-options}}\begin{itemize}
\item {} 
{\hyperref[\detokenize{options:tornado.options.parse_config_file}]{\sphinxcrossref{\sphinxcode{\sphinxupquote{parse\_config\_file}}}}} now always decodes the config
file as utf8 on Python 3.

\item {} 
{\hyperref[\detokenize{options:tornado.options.define}]{\sphinxcrossref{\sphinxcode{\sphinxupquote{tornado.options.define}}}}} more accurately finds the module defining the
option.

\end{itemize}


\paragraph{\sphinxstyleliteralintitle{\sphinxupquote{tornado.platform.asyncio}}}
\label{\detokenize{releases/v4.1.0:tornado-platform-asyncio}}\begin{itemize}
\item {} 
It is now possible to yield \sphinxcode{\sphinxupquote{asyncio.Future}} objects in coroutines
when the \sphinxhref{https://docs.python.org/3.6/library/functools.html\#functools.singledispatch}{\sphinxcode{\sphinxupquote{singledispatch}}} library is available and
\sphinxcode{\sphinxupquote{tornado.platform.asyncio}} has been imported.

\item {} 
New methods {\hyperref[\detokenize{asyncio:tornado.platform.asyncio.to_tornado_future}]{\sphinxcrossref{\sphinxcode{\sphinxupquote{tornado.platform.asyncio.to\_tornado\_future}}}}} and
{\hyperref[\detokenize{asyncio:tornado.platform.asyncio.to_asyncio_future}]{\sphinxcrossref{\sphinxcode{\sphinxupquote{to\_asyncio\_future}}}}} convert between
the two libraries’ {\hyperref[\detokenize{concurrent:tornado.concurrent.Future}]{\sphinxcrossref{\sphinxcode{\sphinxupquote{Future}}}}} classes.

\end{itemize}


\paragraph{\sphinxstyleliteralintitle{\sphinxupquote{tornado.platform.twisted}}}
\label{\detokenize{releases/v4.1.0:tornado-platform-twisted}}\begin{itemize}
\item {} 
It is now possible to yield \sphinxcode{\sphinxupquote{Deferred}} objects in coroutines
when the \sphinxhref{https://docs.python.org/3.6/library/functools.html\#functools.singledispatch}{\sphinxcode{\sphinxupquote{singledispatch}}} library is available and
\sphinxcode{\sphinxupquote{tornado.platform.twisted}} has been imported.

\end{itemize}


\paragraph{\sphinxstyleliteralintitle{\sphinxupquote{tornado.tcpclient}}}
\label{\detokenize{releases/v4.1.0:tornado-tcpclient}}\begin{itemize}
\item {} 
{\hyperref[\detokenize{tcpclient:tornado.tcpclient.TCPClient}]{\sphinxcrossref{\sphinxcode{\sphinxupquote{TCPClient}}}}} will no longer raise an exception due to an ill-timed
timeout.

\end{itemize}


\paragraph{\sphinxstyleliteralintitle{\sphinxupquote{tornado.tcpserver}}}
\label{\detokenize{releases/v4.1.0:tornado-tcpserver}}\begin{itemize}
\item {} 
{\hyperref[\detokenize{tcpserver:tornado.tcpserver.TCPServer}]{\sphinxcrossref{\sphinxcode{\sphinxupquote{TCPServer}}}}} no longer ignores its \sphinxcode{\sphinxupquote{read\_chunk\_size}} argument.

\end{itemize}


\paragraph{\sphinxstyleliteralintitle{\sphinxupquote{tornado.testing}}}
\label{\detokenize{releases/v4.1.0:tornado-testing}}\begin{itemize}
\item {} 
{\hyperref[\detokenize{testing:tornado.testing.AsyncTestCase}]{\sphinxcrossref{\sphinxcode{\sphinxupquote{AsyncTestCase}}}}} has better support for multiple exceptions. Previously
it would silently swallow all but the last; now it raises the first
and logs all the rest.

\item {} 
{\hyperref[\detokenize{testing:tornado.testing.AsyncTestCase}]{\sphinxcrossref{\sphinxcode{\sphinxupquote{AsyncTestCase}}}}} now cleans up {\hyperref[\detokenize{process:tornado.process.Subprocess}]{\sphinxcrossref{\sphinxcode{\sphinxupquote{Subprocess}}}}} state on \sphinxcode{\sphinxupquote{tearDown}} when
necessary.

\end{itemize}


\paragraph{\sphinxstyleliteralintitle{\sphinxupquote{tornado.web}}}
\label{\detokenize{releases/v4.1.0:tornado-web}}\begin{itemize}
\item {} 
The \sphinxcode{\sphinxupquote{asynchronous}} decorator now understands \sphinxhref{https://docs.python.org/3.6/library/concurrent.futures.html\#concurrent.futures.Future}{\sphinxcode{\sphinxupquote{concurrent.futures.Future}}}
in addition to {\hyperref[\detokenize{concurrent:tornado.concurrent.Future}]{\sphinxcrossref{\sphinxcode{\sphinxupquote{tornado.concurrent.Future}}}}}.

\item {} 
{\hyperref[\detokenize{web:tornado.web.StaticFileHandler}]{\sphinxcrossref{\sphinxcode{\sphinxupquote{StaticFileHandler}}}}} no longer logs a stack trace if the connection is
closed while sending the file.

\item {} 
{\hyperref[\detokenize{web:tornado.web.RequestHandler.send_error}]{\sphinxcrossref{\sphinxcode{\sphinxupquote{RequestHandler.send\_error}}}}} now supports a \sphinxcode{\sphinxupquote{reason}} keyword
argument, similar to {\hyperref[\detokenize{web:tornado.web.HTTPError}]{\sphinxcrossref{\sphinxcode{\sphinxupquote{tornado.web.HTTPError}}}}}.

\item {} 
{\hyperref[\detokenize{web:tornado.web.RequestHandler.locale}]{\sphinxcrossref{\sphinxcode{\sphinxupquote{RequestHandler.locale}}}}} now has a property setter.

\item {} 
{\hyperref[\detokenize{web:tornado.web.Application.add_handlers}]{\sphinxcrossref{\sphinxcode{\sphinxupquote{Application.add\_handlers}}}}} hostname matching now works correctly with
IPv6 literals.

\item {} 
Redirects for the {\hyperref[\detokenize{web:tornado.web.Application}]{\sphinxcrossref{\sphinxcode{\sphinxupquote{Application}}}}} \sphinxcode{\sphinxupquote{default\_host}} setting now match
the request protocol instead of redirecting HTTPS to HTTP.

\item {} 
Malformed \sphinxcode{\sphinxupquote{\_xsrf}} cookies are now ignored instead of causing
uncaught exceptions.

\item {} 
\sphinxcode{\sphinxupquote{Application.start\_request}} now has the same signature as
{\hyperref[\detokenize{httputil:tornado.httputil.HTTPServerConnectionDelegate.start_request}]{\sphinxcrossref{\sphinxcode{\sphinxupquote{HTTPServerConnectionDelegate.start\_request}}}}}.

\end{itemize}


\paragraph{\sphinxstyleliteralintitle{\sphinxupquote{tornado.websocket}}}
\label{\detokenize{releases/v4.1.0:tornado-websocket}}\begin{itemize}
\item {} 
The {\hyperref[\detokenize{websocket:module-tornado.websocket}]{\sphinxcrossref{\sphinxcode{\sphinxupquote{tornado.websocket}}}}} module now supports compression via the
“permessage-deflate” extension.  Override
{\hyperref[\detokenize{websocket:tornado.websocket.WebSocketHandler.get_compression_options}]{\sphinxcrossref{\sphinxcode{\sphinxupquote{WebSocketHandler.get\_compression\_options}}}}} to enable on the server
side, and use the \sphinxcode{\sphinxupquote{compression\_options}} keyword argument to
{\hyperref[\detokenize{websocket:tornado.websocket.websocket_connect}]{\sphinxcrossref{\sphinxcode{\sphinxupquote{websocket\_connect}}}}} on the client side.

\item {} 
{\hyperref[\detokenize{websocket:tornado.websocket.WebSocketHandler}]{\sphinxcrossref{\sphinxcode{\sphinxupquote{WebSocketHandler}}}}} no longer logs stack traces when the connection
is closed.

\item {} 
{\hyperref[\detokenize{websocket:tornado.websocket.WebSocketHandler.open}]{\sphinxcrossref{\sphinxcode{\sphinxupquote{WebSocketHandler.open}}}}} now accepts \sphinxcode{\sphinxupquote{*args, **kw}} for consistency
with \sphinxcode{\sphinxupquote{RequestHandler.get}} and related methods.

\item {} 
The \sphinxcode{\sphinxupquote{Sec-WebSocket-Version}} header now includes all supported versions.

\item {} 
{\hyperref[\detokenize{websocket:tornado.websocket.websocket_connect}]{\sphinxcrossref{\sphinxcode{\sphinxupquote{websocket\_connect}}}}} now has a \sphinxcode{\sphinxupquote{on\_message\_callback}} keyword argument
for callback-style use without \sphinxcode{\sphinxupquote{read\_message()}}.

\end{itemize}


\subsection{What’s new in Tornado 4.0.2}
\label{\detokenize{releases/v4.0.2:what-s-new-in-tornado-4-0-2}}\label{\detokenize{releases/v4.0.2::doc}}

\subsubsection{Sept 10, 2014}
\label{\detokenize{releases/v4.0.2:sept-10-2014}}

\paragraph{Bug fixes}
\label{\detokenize{releases/v4.0.2:bug-fixes}}\begin{itemize}
\item {} 
Fixed a bug that could sometimes cause a timeout to fire after being
cancelled.

\item {} 
{\hyperref[\detokenize{testing:tornado.testing.AsyncTestCase}]{\sphinxcrossref{\sphinxcode{\sphinxupquote{AsyncTestCase}}}}} once again passes along arguments to test methods,
making it compatible with extensions such as Nose’s test generators.

\item {} 
{\hyperref[\detokenize{web:tornado.web.StaticFileHandler}]{\sphinxcrossref{\sphinxcode{\sphinxupquote{StaticFileHandler}}}}} can again compress its responses when gzip is enabled.

\item {} 
\sphinxcode{\sphinxupquote{simple\_httpclient}} passes its \sphinxcode{\sphinxupquote{max\_buffer\_size}} argument to the
underlying stream.

\item {} 
Fixed a reference cycle that can lead to increased memory consumption.

\item {} 
{\hyperref[\detokenize{netutil:tornado.netutil.add_accept_handler}]{\sphinxcrossref{\sphinxcode{\sphinxupquote{add\_accept\_handler}}}}} will now limit the number of times it will call
\sphinxhref{https://docs.python.org/3.6/library/socket.html\#socket.socket.accept}{\sphinxcode{\sphinxupquote{accept}}} per {\hyperref[\detokenize{ioloop:tornado.ioloop.IOLoop}]{\sphinxcrossref{\sphinxcode{\sphinxupquote{IOLoop}}}}} iteration, addressing a potential
starvation issue.

\item {} 
Improved error handling in {\hyperref[\detokenize{iostream:tornado.iostream.IOStream.connect}]{\sphinxcrossref{\sphinxcode{\sphinxupquote{IOStream.connect}}}}} (primarily for FreeBSD
systems)

\end{itemize}


\subsection{What’s new in Tornado 4.0.1}
\label{\detokenize{releases/v4.0.1:what-s-new-in-tornado-4-0-1}}\label{\detokenize{releases/v4.0.1::doc}}

\subsubsection{Aug 12, 2014}
\label{\detokenize{releases/v4.0.1:aug-12-2014}}\begin{itemize}
\item {} 
The build will now fall back to pure-python mode if the C extension
fails to build for any reason (previously it would fall back for some
errors but not others).

\item {} 
{\hyperref[\detokenize{ioloop:tornado.ioloop.IOLoop.call_at}]{\sphinxcrossref{\sphinxcode{\sphinxupquote{IOLoop.call\_at}}}}} and {\hyperref[\detokenize{ioloop:tornado.ioloop.IOLoop.call_later}]{\sphinxcrossref{\sphinxcode{\sphinxupquote{IOLoop.call\_later}}}}} now always return
a timeout handle for use with {\hyperref[\detokenize{ioloop:tornado.ioloop.IOLoop.remove_timeout}]{\sphinxcrossref{\sphinxcode{\sphinxupquote{IOLoop.remove\_timeout}}}}}.

\item {} 
If any callback of a {\hyperref[\detokenize{ioloop:tornado.ioloop.PeriodicCallback}]{\sphinxcrossref{\sphinxcode{\sphinxupquote{PeriodicCallback}}}}} or {\hyperref[\detokenize{iostream:tornado.iostream.IOStream}]{\sphinxcrossref{\sphinxcode{\sphinxupquote{IOStream}}}}} returns a
{\hyperref[\detokenize{concurrent:tornado.concurrent.Future}]{\sphinxcrossref{\sphinxcode{\sphinxupquote{Future}}}}}, any error raised in that future will now be logged
(similar to the behavior of {\hyperref[\detokenize{ioloop:tornado.ioloop.IOLoop.add_callback}]{\sphinxcrossref{\sphinxcode{\sphinxupquote{IOLoop.add\_callback}}}}}).

\item {} 
Fixed an exception in client-side websocket connections when the
connection is closed.

\item {} 
\sphinxcode{\sphinxupquote{simple\_httpclient}} once again correctly handles 204 status
codes with no content-length header.

\item {} 
Fixed a regression in \sphinxcode{\sphinxupquote{simple\_httpclient}} that would result in
timeouts for certain kinds of errors.

\end{itemize}


\subsection{What’s new in Tornado 4.0}
\label{\detokenize{releases/v4.0.0:what-s-new-in-tornado-4-0}}\label{\detokenize{releases/v4.0.0::doc}}

\subsubsection{July 15, 2014}
\label{\detokenize{releases/v4.0.0:july-15-2014}}

\paragraph{Highlights}
\label{\detokenize{releases/v4.0.0:highlights}}\begin{itemize}
\item {} 
The {\hyperref[\detokenize{web:tornado.web.stream_request_body}]{\sphinxcrossref{\sphinxcode{\sphinxupquote{tornado.web.stream\_request\_body}}}}} decorator allows large files to be
uploaded with limited memory usage.

\item {} 
Coroutines are now faster and are used extensively throughout Tornado itself.
More methods now return {\hyperref[\detokenize{concurrent:tornado.concurrent.Future}]{\sphinxcrossref{\sphinxcode{\sphinxupquote{Futures}}}}}, including most {\hyperref[\detokenize{iostream:tornado.iostream.IOStream}]{\sphinxcrossref{\sphinxcode{\sphinxupquote{IOStream}}}}}
methods and {\hyperref[\detokenize{web:tornado.web.RequestHandler.flush}]{\sphinxcrossref{\sphinxcode{\sphinxupquote{RequestHandler.flush}}}}}.

\item {} 
Many user-overridden methods are now allowed to return a {\hyperref[\detokenize{concurrent:tornado.concurrent.Future}]{\sphinxcrossref{\sphinxcode{\sphinxupquote{Future}}}}}
for flow control.

\item {} 
HTTP-related code is now shared between the {\hyperref[\detokenize{httpserver:module-tornado.httpserver}]{\sphinxcrossref{\sphinxcode{\sphinxupquote{tornado.httpserver}}}}},
\sphinxcode{\sphinxupquote{tornado.simple\_httpclient}} and {\hyperref[\detokenize{wsgi:module-tornado.wsgi}]{\sphinxcrossref{\sphinxcode{\sphinxupquote{tornado.wsgi}}}}} modules, making support
for features such as chunked and gzip encoding more consistent.
{\hyperref[\detokenize{httpserver:tornado.httpserver.HTTPServer}]{\sphinxcrossref{\sphinxcode{\sphinxupquote{HTTPServer}}}}} now uses new delegate interfaces defined in {\hyperref[\detokenize{httputil:module-tornado.httputil}]{\sphinxcrossref{\sphinxcode{\sphinxupquote{tornado.httputil}}}}}
in addition to its old single-callback interface.

\item {} 
New module {\hyperref[\detokenize{tcpclient:module-tornado.tcpclient}]{\sphinxcrossref{\sphinxcode{\sphinxupquote{tornado.tcpclient}}}}} creates TCP connections with non-blocking
DNS, SSL handshaking, and support for IPv6.

\end{itemize}


\paragraph{Backwards-compatibility notes}
\label{\detokenize{releases/v4.0.0:backwards-compatibility-notes}}\begin{itemize}
\item {} 
{\hyperref[\detokenize{concurrent:tornado.concurrent.Future}]{\sphinxcrossref{\sphinxcode{\sphinxupquote{tornado.concurrent.Future}}}}} is no longer thread-safe; use
\sphinxhref{https://docs.python.org/3.6/library/concurrent.futures.html\#concurrent.futures.Future}{\sphinxcode{\sphinxupquote{concurrent.futures.Future}}} when thread-safety is needed.

\item {} 
Tornado now depends on the \sphinxhref{https://pypi.python.org/pypi/certifi}{certifi}
package instead of bundling its own copy of the Mozilla CA list. This will
be installed automatically when using \sphinxcode{\sphinxupquote{pip}} or \sphinxcode{\sphinxupquote{easy\_install}}.

\item {} 
This version includes the changes to the secure cookie format first
introduced in version {\hyperref[\detokenize{releases/v3.2.1::doc}]{\sphinxcrossref{\DUrole{doc}{3.2.1}}}}, and the xsrf token change
in version {\hyperref[\detokenize{releases/v3.2.2::doc}]{\sphinxcrossref{\DUrole{doc}{3.2.2}}}}.  If you are upgrading from an earlier
version, see those versions’ release notes.

\item {} 
WebSocket connections from other origin sites are now rejected by default.
To accept cross-origin websocket connections, override
the new method {\hyperref[\detokenize{websocket:tornado.websocket.WebSocketHandler.check_origin}]{\sphinxcrossref{\sphinxcode{\sphinxupquote{WebSocketHandler.check\_origin}}}}}.

\item {} 
{\hyperref[\detokenize{websocket:tornado.websocket.WebSocketHandler}]{\sphinxcrossref{\sphinxcode{\sphinxupquote{WebSocketHandler}}}}} no longer supports the old \sphinxcode{\sphinxupquote{draft 76}} protocol
(this mainly affects Safari 5.x browsers).  Applications should use
non-websocket workarounds for these browsers.

\item {} 
Authors of alternative {\hyperref[\detokenize{ioloop:tornado.ioloop.IOLoop}]{\sphinxcrossref{\sphinxcode{\sphinxupquote{IOLoop}}}}} implementations should see the changes
to {\hyperref[\detokenize{ioloop:tornado.ioloop.IOLoop.add_handler}]{\sphinxcrossref{\sphinxcode{\sphinxupquote{IOLoop.add\_handler}}}}} in this release.

\item {} 
The \sphinxcode{\sphinxupquote{RequestHandler.async\_callback}} and \sphinxcode{\sphinxupquote{WebSocketHandler.async\_callback}}
wrapper functions have been removed; they have been obsolete for a long
time due to stack contexts (and more recently coroutines).

\item {} 
\sphinxcode{\sphinxupquote{curl\_httpclient}} now requires a minimum of libcurl version 7.21.1 and
pycurl 7.18.2.

\item {} 
Support for \sphinxcode{\sphinxupquote{RequestHandler.get\_error\_html}} has been removed;
override {\hyperref[\detokenize{web:tornado.web.RequestHandler.write_error}]{\sphinxcrossref{\sphinxcode{\sphinxupquote{RequestHandler.write\_error}}}}} instead.

\end{itemize}


\paragraph{Other notes}
\label{\detokenize{releases/v4.0.0:other-notes}}\begin{itemize}
\item {} 
The git repository has moved to \sphinxurl{https://github.com/tornadoweb/tornado}.
All old links should be redirected to the new location.

\item {} 
An \sphinxhref{http://groups.google.com/group/python-tornado-announce}{announcement mailing list} is now available.

\item {} 
All Tornado modules are now importable on Google App Engine (although
the App Engine environment does not allow the system calls used
by {\hyperref[\detokenize{ioloop:tornado.ioloop.IOLoop}]{\sphinxcrossref{\sphinxcode{\sphinxupquote{IOLoop}}}}} so many modules are still unusable).

\end{itemize}


\paragraph{\sphinxstyleliteralintitle{\sphinxupquote{tornado.auth}}}
\label{\detokenize{releases/v4.0.0:tornado-auth}}\begin{itemize}
\item {} 
Fixed a bug in \sphinxcode{\sphinxupquote{.FacebookMixin}} on Python 3.

\item {} 
When using the {\hyperref[\detokenize{concurrent:tornado.concurrent.Future}]{\sphinxcrossref{\sphinxcode{\sphinxupquote{Future}}}}} interface, exceptions are more reliably delivered
to the caller.

\end{itemize}


\paragraph{\sphinxstyleliteralintitle{\sphinxupquote{tornado.concurrent}}}
\label{\detokenize{releases/v4.0.0:tornado-concurrent}}\begin{itemize}
\item {} 
{\hyperref[\detokenize{concurrent:tornado.concurrent.Future}]{\sphinxcrossref{\sphinxcode{\sphinxupquote{tornado.concurrent.Future}}}}} is now always thread-unsafe (previously
it would be thread-safe if the \sphinxhref{https://docs.python.org/3.6/library/concurrent.futures.html\#module-concurrent.futures}{\sphinxcode{\sphinxupquote{concurrent.futures}}} package was available).
This improves performance and provides more consistent semantics.
The parts of Tornado that accept Futures will accept both Tornado’s
thread-unsafe Futures and the thread-safe \sphinxhref{https://docs.python.org/3.6/library/concurrent.futures.html\#concurrent.futures.Future}{\sphinxcode{\sphinxupquote{concurrent.futures.Future}}}.

\item {} 
{\hyperref[\detokenize{concurrent:tornado.concurrent.Future}]{\sphinxcrossref{\sphinxcode{\sphinxupquote{tornado.concurrent.Future}}}}} now includes all the functionality
of the old \sphinxcode{\sphinxupquote{TracebackFuture}} class.  \sphinxcode{\sphinxupquote{TracebackFuture}} is now
simply an alias for \sphinxcode{\sphinxupquote{Future}}.

\end{itemize}


\paragraph{\sphinxstyleliteralintitle{\sphinxupquote{tornado.curl\_httpclient}}}
\label{\detokenize{releases/v4.0.0:tornado-curl-httpclient}}\begin{itemize}
\item {} 
\sphinxcode{\sphinxupquote{curl\_httpclient}} now passes along the HTTP “reason” string
in \sphinxcode{\sphinxupquote{response.reason}}.

\end{itemize}


\paragraph{\sphinxstyleliteralintitle{\sphinxupquote{tornado.gen}}}
\label{\detokenize{releases/v4.0.0:tornado-gen}}\begin{itemize}
\item {} 
Performance of coroutines has been improved.

\item {} 
Coroutines no longer generate \sphinxcode{\sphinxupquote{StackContexts}} by default, but they
will be created on demand when needed.

\item {} 
The internals of the {\hyperref[\detokenize{gen:module-tornado.gen}]{\sphinxcrossref{\sphinxcode{\sphinxupquote{tornado.gen}}}}} module have been rewritten to
improve performance when using \sphinxcode{\sphinxupquote{Futures}}, at the expense of some
performance degradation for the older \sphinxcode{\sphinxupquote{YieldPoint}} interfaces.

\item {} 
New function {\hyperref[\detokenize{gen:tornado.gen.with_timeout}]{\sphinxcrossref{\sphinxcode{\sphinxupquote{with\_timeout}}}}} wraps a {\hyperref[\detokenize{concurrent:tornado.concurrent.Future}]{\sphinxcrossref{\sphinxcode{\sphinxupquote{Future}}}}} and raises an exception
if it doesn’t complete in a given amount of time.

\item {} 
New object {\hyperref[\detokenize{gen:tornado.gen.moment}]{\sphinxcrossref{\sphinxcode{\sphinxupquote{moment}}}}} can be yielded to allow the IOLoop to run for
one iteration before resuming.

\item {} 
\sphinxcode{\sphinxupquote{Task}} is now a function returning a {\hyperref[\detokenize{concurrent:tornado.concurrent.Future}]{\sphinxcrossref{\sphinxcode{\sphinxupquote{Future}}}}} instead of a \sphinxcode{\sphinxupquote{YieldPoint}}
subclass.  This change should be transparent to application code, but
allows \sphinxcode{\sphinxupquote{Task}} to take advantage of the newly-optimized {\hyperref[\detokenize{concurrent:tornado.concurrent.Future}]{\sphinxcrossref{\sphinxcode{\sphinxupquote{Future}}}}}
handling.

\end{itemize}


\paragraph{\sphinxstyleliteralintitle{\sphinxupquote{tornado.http1connection}}}
\label{\detokenize{releases/v4.0.0:tornado-http1connection}}\begin{itemize}
\item {} 
New module contains the HTTP implementation shared by {\hyperref[\detokenize{httpserver:module-tornado.httpserver}]{\sphinxcrossref{\sphinxcode{\sphinxupquote{tornado.httpserver}}}}}
and \sphinxcode{\sphinxupquote{tornado.simple\_httpclient}}.

\end{itemize}


\paragraph{\sphinxstyleliteralintitle{\sphinxupquote{tornado.httpclient}}}
\label{\detokenize{releases/v4.0.0:tornado-httpclient}}\begin{itemize}
\item {} 
The command-line HTTP client (\sphinxcode{\sphinxupquote{python -m tornado.httpclient \$URL}})
now works on Python 3.

\item {} 
Fixed a memory leak in {\hyperref[\detokenize{httpclient:tornado.httpclient.AsyncHTTPClient}]{\sphinxcrossref{\sphinxcode{\sphinxupquote{AsyncHTTPClient}}}}} shutdown that affected
applications that created many HTTP clients and IOLoops.

\item {} 
New client request parameter \sphinxcode{\sphinxupquote{decompress\_response}} replaces
the existing \sphinxcode{\sphinxupquote{use\_gzip}} parameter; both names are accepted.

\end{itemize}


\paragraph{\sphinxstyleliteralintitle{\sphinxupquote{tornado.httpserver}}}
\label{\detokenize{releases/v4.0.0:tornado-httpserver}}\begin{itemize}
\item {} 
\sphinxcode{\sphinxupquote{tornado.httpserver.HTTPRequest}} has moved to
{\hyperref[\detokenize{httputil:tornado.httputil.HTTPServerRequest}]{\sphinxcrossref{\sphinxcode{\sphinxupquote{tornado.httputil.HTTPServerRequest}}}}}.

\item {} 
HTTP implementation has been unified with \sphinxcode{\sphinxupquote{tornado.simple\_httpclient}}
in {\hyperref[\detokenize{http1connection:module-tornado.http1connection}]{\sphinxcrossref{\sphinxcode{\sphinxupquote{tornado.http1connection}}}}}.

\item {} 
Now supports \sphinxcode{\sphinxupquote{Transfer-Encoding: chunked}} for request bodies.

\item {} 
Now supports \sphinxcode{\sphinxupquote{Content-Encoding: gzip}} for request bodies if
\sphinxcode{\sphinxupquote{decompress\_request=True}} is passed to the {\hyperref[\detokenize{httpserver:tornado.httpserver.HTTPServer}]{\sphinxcrossref{\sphinxcode{\sphinxupquote{HTTPServer}}}}} constructor.

\item {} 
The \sphinxcode{\sphinxupquote{connection}} attribute of {\hyperref[\detokenize{httputil:tornado.httputil.HTTPServerRequest}]{\sphinxcrossref{\sphinxcode{\sphinxupquote{HTTPServerRequest}}}}} is now documented
for public use; applications are expected to write their responses
via the {\hyperref[\detokenize{httputil:tornado.httputil.HTTPConnection}]{\sphinxcrossref{\sphinxcode{\sphinxupquote{HTTPConnection}}}}} interface.

\item {} 
The \sphinxcode{\sphinxupquote{HTTPServerRequest.write}} and \sphinxcode{\sphinxupquote{HTTPServerRequest.finish}} methods
are now deprecated.  ({\hyperref[\detokenize{web:tornado.web.RequestHandler.write}]{\sphinxcrossref{\sphinxcode{\sphinxupquote{RequestHandler.write}}}}} and {\hyperref[\detokenize{web:tornado.web.RequestHandler.finish}]{\sphinxcrossref{\sphinxcode{\sphinxupquote{RequestHandler.finish}}}}}
are \sphinxstyleemphasis{not} deprecated; this only applies to the methods on
{\hyperref[\detokenize{httputil:tornado.httputil.HTTPServerRequest}]{\sphinxcrossref{\sphinxcode{\sphinxupquote{HTTPServerRequest}}}}})

\item {} 
{\hyperref[\detokenize{httpserver:tornado.httpserver.HTTPServer}]{\sphinxcrossref{\sphinxcode{\sphinxupquote{HTTPServer}}}}} now supports {\hyperref[\detokenize{httputil:tornado.httputil.HTTPServerConnectionDelegate}]{\sphinxcrossref{\sphinxcode{\sphinxupquote{HTTPServerConnectionDelegate}}}}} in addition to
the old \sphinxcode{\sphinxupquote{request\_callback}} interface.  The delegate interface supports
streaming of request bodies.

\item {} 
{\hyperref[\detokenize{httpserver:tornado.httpserver.HTTPServer}]{\sphinxcrossref{\sphinxcode{\sphinxupquote{HTTPServer}}}}} now detects the error of an application sending a
\sphinxcode{\sphinxupquote{Content-Length}} error that is inconsistent with the actual content.

\item {} 
New constructor arguments \sphinxcode{\sphinxupquote{max\_header\_size}} and \sphinxcode{\sphinxupquote{max\_body\_size}}
allow separate limits to be set for different parts of the request.
\sphinxcode{\sphinxupquote{max\_body\_size}} is applied even in streaming mode.

\item {} 
New constructor argument \sphinxcode{\sphinxupquote{chunk\_size}} can be used to limit the amount
of data read into memory at one time per request.

\item {} 
New constructor arguments \sphinxcode{\sphinxupquote{idle\_connection\_timeout}} and \sphinxcode{\sphinxupquote{body\_timeout}}
allow time limits to be placed on the reading of requests.

\item {} 
Form-encoded message bodies are now parsed for all HTTP methods, not just
\sphinxcode{\sphinxupquote{POST}}, \sphinxcode{\sphinxupquote{PUT}}, and \sphinxcode{\sphinxupquote{PATCH}}.

\end{itemize}


\paragraph{\sphinxstyleliteralintitle{\sphinxupquote{tornado.httputil}}}
\label{\detokenize{releases/v4.0.0:tornado-httputil}}\begin{itemize}
\item {} 
{\hyperref[\detokenize{httputil:tornado.httputil.HTTPServerRequest}]{\sphinxcrossref{\sphinxcode{\sphinxupquote{HTTPServerRequest}}}}} was moved to this module from {\hyperref[\detokenize{httpserver:module-tornado.httpserver}]{\sphinxcrossref{\sphinxcode{\sphinxupquote{tornado.httpserver}}}}}.

\item {} 
New base classes {\hyperref[\detokenize{httputil:tornado.httputil.HTTPConnection}]{\sphinxcrossref{\sphinxcode{\sphinxupquote{HTTPConnection}}}}}, {\hyperref[\detokenize{httputil:tornado.httputil.HTTPServerConnectionDelegate}]{\sphinxcrossref{\sphinxcode{\sphinxupquote{HTTPServerConnectionDelegate}}}}},
and {\hyperref[\detokenize{httputil:tornado.httputil.HTTPMessageDelegate}]{\sphinxcrossref{\sphinxcode{\sphinxupquote{HTTPMessageDelegate}}}}} define the interaction between applications
and the HTTP implementation.

\end{itemize}


\paragraph{\sphinxstyleliteralintitle{\sphinxupquote{tornado.ioloop}}}
\label{\detokenize{releases/v4.0.0:tornado-ioloop}}\begin{itemize}
\item {} 
{\hyperref[\detokenize{ioloop:tornado.ioloop.IOLoop.add_handler}]{\sphinxcrossref{\sphinxcode{\sphinxupquote{IOLoop.add\_handler}}}}} and related methods now accept file-like objects
in addition to raw file descriptors.  Passing the objects is recommended
(when possible) to avoid a garbage-collection-related problem in unit tests.

\item {} 
New method {\hyperref[\detokenize{ioloop:tornado.ioloop.IOLoop.clear_instance}]{\sphinxcrossref{\sphinxcode{\sphinxupquote{IOLoop.clear\_instance}}}}} makes it possible to uninstall the
singleton instance.

\item {} 
Timeout scheduling is now more robust against slow callbacks.

\item {} 
{\hyperref[\detokenize{ioloop:tornado.ioloop.IOLoop.add_timeout}]{\sphinxcrossref{\sphinxcode{\sphinxupquote{IOLoop.add\_timeout}}}}} is now a bit more efficient.

\item {} 
When a function run by the {\hyperref[\detokenize{ioloop:tornado.ioloop.IOLoop}]{\sphinxcrossref{\sphinxcode{\sphinxupquote{IOLoop}}}}} returns a {\hyperref[\detokenize{concurrent:tornado.concurrent.Future}]{\sphinxcrossref{\sphinxcode{\sphinxupquote{Future}}}}} and that {\hyperref[\detokenize{concurrent:tornado.concurrent.Future}]{\sphinxcrossref{\sphinxcode{\sphinxupquote{Future}}}}}
has an exception, the {\hyperref[\detokenize{ioloop:tornado.ioloop.IOLoop}]{\sphinxcrossref{\sphinxcode{\sphinxupquote{IOLoop}}}}} will log the exception.

\item {} 
New method {\hyperref[\detokenize{ioloop:tornado.ioloop.IOLoop.spawn_callback}]{\sphinxcrossref{\sphinxcode{\sphinxupquote{IOLoop.spawn\_callback}}}}} simplifies the process of launching
a fire-and-forget callback that is separated from the caller’s stack context.

\item {} 
New methods {\hyperref[\detokenize{ioloop:tornado.ioloop.IOLoop.call_later}]{\sphinxcrossref{\sphinxcode{\sphinxupquote{IOLoop.call\_later}}}}} and {\hyperref[\detokenize{ioloop:tornado.ioloop.IOLoop.call_at}]{\sphinxcrossref{\sphinxcode{\sphinxupquote{IOLoop.call\_at}}}}} simplify the
specification of relative or absolute timeouts (as opposed to
{\hyperref[\detokenize{ioloop:tornado.ioloop.IOLoop.add_timeout}]{\sphinxcrossref{\sphinxcode{\sphinxupquote{add\_timeout}}}}}, which used the type of its argument).

\end{itemize}


\paragraph{\sphinxstyleliteralintitle{\sphinxupquote{tornado.iostream}}}
\label{\detokenize{releases/v4.0.0:tornado-iostream}}\begin{itemize}
\item {} 
The \sphinxcode{\sphinxupquote{callback}} argument to most {\hyperref[\detokenize{iostream:tornado.iostream.IOStream}]{\sphinxcrossref{\sphinxcode{\sphinxupquote{IOStream}}}}} methods is now optional.
When called without a callback the method will return a {\hyperref[\detokenize{concurrent:tornado.concurrent.Future}]{\sphinxcrossref{\sphinxcode{\sphinxupquote{Future}}}}}
for use with coroutines.

\item {} 
New method {\hyperref[\detokenize{iostream:tornado.iostream.IOStream.start_tls}]{\sphinxcrossref{\sphinxcode{\sphinxupquote{IOStream.start\_tls}}}}} converts an {\hyperref[\detokenize{iostream:tornado.iostream.IOStream}]{\sphinxcrossref{\sphinxcode{\sphinxupquote{IOStream}}}}} to an
{\hyperref[\detokenize{iostream:tornado.iostream.SSLIOStream}]{\sphinxcrossref{\sphinxcode{\sphinxupquote{SSLIOStream}}}}}.

\item {} 
No longer gets confused when an \sphinxcode{\sphinxupquote{IOError}} or \sphinxcode{\sphinxupquote{OSError}} without
an \sphinxcode{\sphinxupquote{errno}} attribute is raised.

\item {} 
{\hyperref[\detokenize{iostream:tornado.iostream.BaseIOStream.read_bytes}]{\sphinxcrossref{\sphinxcode{\sphinxupquote{BaseIOStream.read\_bytes}}}}} now accepts a \sphinxcode{\sphinxupquote{partial}} keyword argument,
which can be used to return before the full amount has been read.
This is a more coroutine-friendly alternative to \sphinxcode{\sphinxupquote{streaming\_callback}}.

\item {} 
{\hyperref[\detokenize{iostream:tornado.iostream.BaseIOStream.read_until}]{\sphinxcrossref{\sphinxcode{\sphinxupquote{BaseIOStream.read\_until}}}}} and \sphinxcode{\sphinxupquote{read\_until\_regex}} now acept a
\sphinxcode{\sphinxupquote{max\_bytes}} keyword argument which will cause the request to fail if
it cannot be satisfied from the given number of bytes.

\item {} 
{\hyperref[\detokenize{iostream:tornado.iostream.IOStream}]{\sphinxcrossref{\sphinxcode{\sphinxupquote{IOStream}}}}} no longer reads from the socket into memory if it does not
need data to satisfy a pending read.  As a side effect, the close callback
will not be run immediately if the other side closes the connection
while there is unconsumed data in the buffer.

\item {} 
The default \sphinxcode{\sphinxupquote{chunk\_size}} has been increased to 64KB (from 4KB)

\item {} 
The {\hyperref[\detokenize{iostream:tornado.iostream.IOStream}]{\sphinxcrossref{\sphinxcode{\sphinxupquote{IOStream}}}}} constructor takes a new keyword argument
\sphinxcode{\sphinxupquote{max\_write\_buffer\_size}} (defaults to unlimited).  Calls to
{\hyperref[\detokenize{iostream:tornado.iostream.BaseIOStream.write}]{\sphinxcrossref{\sphinxcode{\sphinxupquote{BaseIOStream.write}}}}} will raise {\hyperref[\detokenize{iostream:tornado.iostream.StreamBufferFullError}]{\sphinxcrossref{\sphinxcode{\sphinxupquote{StreamBufferFullError}}}}} if the amount
of unsent buffered data exceeds this limit.

\item {} 
\sphinxcode{\sphinxupquote{ETIMEDOUT}} errors are no longer logged.  If you need to distinguish
timeouts from other forms of closed connections, examine \sphinxcode{\sphinxupquote{stream.error}}
from a close callback.

\end{itemize}


\paragraph{\sphinxstyleliteralintitle{\sphinxupquote{tornado.netutil}}}
\label{\detokenize{releases/v4.0.0:tornado-netutil}}\begin{itemize}
\item {} 
When {\hyperref[\detokenize{netutil:tornado.netutil.bind_sockets}]{\sphinxcrossref{\sphinxcode{\sphinxupquote{bind\_sockets}}}}} chooses a port automatically, it will now use
the same port for IPv4 and IPv6.

\item {} 
TLS compression is now disabled by default on Python 3.3 and higher
(it is not possible to change this option in older versions).

\end{itemize}


\paragraph{\sphinxstyleliteralintitle{\sphinxupquote{tornado.options}}}
\label{\detokenize{releases/v4.0.0:tornado-options}}\begin{itemize}
\item {} 
It is now possible to disable the default logging configuration
by setting \sphinxcode{\sphinxupquote{options.logging}} to \sphinxcode{\sphinxupquote{None}} instead of the string \sphinxcode{\sphinxupquote{"none"}}.

\end{itemize}


\paragraph{\sphinxstyleliteralintitle{\sphinxupquote{tornado.platform.asyncio}}}
\label{\detokenize{releases/v4.0.0:tornado-platform-asyncio}}\begin{itemize}
\item {} 
Now works on Python 2.6.

\item {} 
Now works with Trollius version 0.3.

\end{itemize}


\paragraph{\sphinxstyleliteralintitle{\sphinxupquote{tornado.platform.twisted}}}
\label{\detokenize{releases/v4.0.0:tornado-platform-twisted}}\begin{itemize}
\item {} 
\sphinxcode{\sphinxupquote{TwistedIOLoop}} now works on Python 3.3+ (with Twisted 14.0.0+).

\end{itemize}


\paragraph{\sphinxstyleliteralintitle{\sphinxupquote{tornado.simple\_httpclient}}}
\label{\detokenize{releases/v4.0.0:tornado-simple-httpclient}}\begin{itemize}
\item {} 
\sphinxcode{\sphinxupquote{simple\_httpclient}} has better support for IPv6, which is now enabled
by default.

\item {} 
Improved default cipher suite selection (Python 2.7+).

\item {} 
HTTP implementation has been unified with \sphinxcode{\sphinxupquote{tornado.httpserver}}
in {\hyperref[\detokenize{http1connection:module-tornado.http1connection}]{\sphinxcrossref{\sphinxcode{\sphinxupquote{tornado.http1connection}}}}}

\item {} 
Streaming request bodies are now supported via the \sphinxcode{\sphinxupquote{body\_producer}}
keyword argument to {\hyperref[\detokenize{httpclient:tornado.httpclient.HTTPRequest}]{\sphinxcrossref{\sphinxcode{\sphinxupquote{tornado.httpclient.HTTPRequest}}}}}.

\item {} 
The \sphinxcode{\sphinxupquote{expect\_100\_continue}} keyword argument to
{\hyperref[\detokenize{httpclient:tornado.httpclient.HTTPRequest}]{\sphinxcrossref{\sphinxcode{\sphinxupquote{tornado.httpclient.HTTPRequest}}}}} allows the use of the HTTP \sphinxcode{\sphinxupquote{Expect:
100-continue}} feature.

\item {} 
\sphinxcode{\sphinxupquote{simple\_httpclient}} now raises the original exception (e.g. an \sphinxhref{https://docs.python.org/3.6/library/exceptions.html\#IOError}{\sphinxcode{\sphinxupquote{IOError}}})
in more cases, instead of converting everything to \sphinxcode{\sphinxupquote{HTTPError}}.

\end{itemize}


\paragraph{\sphinxstyleliteralintitle{\sphinxupquote{tornado.stack\_context}}}
\label{\detokenize{releases/v4.0.0:tornado-stack-context}}\begin{itemize}
\item {} 
The stack context system now has less performance overhead when no
stack contexts are active.

\end{itemize}


\paragraph{\sphinxstyleliteralintitle{\sphinxupquote{tornado.tcpclient}}}
\label{\detokenize{releases/v4.0.0:tornado-tcpclient}}\begin{itemize}
\item {} 
New module which creates TCP connections and IOStreams, including
name resolution, connecting, and SSL handshakes.

\end{itemize}


\paragraph{\sphinxstyleliteralintitle{\sphinxupquote{tornado.testing}}}
\label{\detokenize{releases/v4.0.0:tornado-testing}}\begin{itemize}
\item {} 
{\hyperref[\detokenize{testing:tornado.testing.AsyncTestCase}]{\sphinxcrossref{\sphinxcode{\sphinxupquote{AsyncTestCase}}}}} now attempts to detect test methods that are generators
but were not run with \sphinxcode{\sphinxupquote{@gen\_test}} or any similar decorator (this would
previously result in the test silently being skipped).

\item {} 
Better stack traces are now displayed when a test times out.

\item {} 
The \sphinxcode{\sphinxupquote{@gen\_test}} decorator now passes along \sphinxcode{\sphinxupquote{*args, **kwargs}} so it
can be used on functions with arguments.

\item {} 
Fixed the test suite when \sphinxcode{\sphinxupquote{unittest2}} is installed on Python 3.

\end{itemize}


\paragraph{\sphinxstyleliteralintitle{\sphinxupquote{tornado.web}}}
\label{\detokenize{releases/v4.0.0:tornado-web}}\begin{itemize}
\item {} 
It is now possible to support streaming request bodies with the
{\hyperref[\detokenize{web:tornado.web.stream_request_body}]{\sphinxcrossref{\sphinxcode{\sphinxupquote{stream\_request\_body}}}}} decorator and the new {\hyperref[\detokenize{web:tornado.web.RequestHandler.data_received}]{\sphinxcrossref{\sphinxcode{\sphinxupquote{RequestHandler.data\_received}}}}}
method.

\item {} 
{\hyperref[\detokenize{web:tornado.web.RequestHandler.flush}]{\sphinxcrossref{\sphinxcode{\sphinxupquote{RequestHandler.flush}}}}} now returns a {\hyperref[\detokenize{concurrent:tornado.concurrent.Future}]{\sphinxcrossref{\sphinxcode{\sphinxupquote{Future}}}}} if no callback is given.

\item {} 
New exception {\hyperref[\detokenize{web:tornado.web.Finish}]{\sphinxcrossref{\sphinxcode{\sphinxupquote{Finish}}}}} may be raised to finish a request without
triggering error handling.

\item {} 
When gzip support is enabled, all \sphinxcode{\sphinxupquote{text/*}} mime types will be compressed,
not just those on a whitelist.

\item {} 
{\hyperref[\detokenize{web:tornado.web.Application}]{\sphinxcrossref{\sphinxcode{\sphinxupquote{Application}}}}} now implements the {\hyperref[\detokenize{httputil:tornado.httputil.HTTPMessageDelegate}]{\sphinxcrossref{\sphinxcode{\sphinxupquote{HTTPMessageDelegate}}}}} interface.

\item {} 
\sphinxcode{\sphinxupquote{HEAD}} requests in {\hyperref[\detokenize{web:tornado.web.StaticFileHandler}]{\sphinxcrossref{\sphinxcode{\sphinxupquote{StaticFileHandler}}}}} no longer read the entire file.

\item {} 
{\hyperref[\detokenize{web:tornado.web.StaticFileHandler}]{\sphinxcrossref{\sphinxcode{\sphinxupquote{StaticFileHandler}}}}} now streams response bodies to the client.

\item {} 
New setting \sphinxcode{\sphinxupquote{compress\_response}} replaces the existing \sphinxcode{\sphinxupquote{gzip}}
setting; both names are accepted.

\item {} 
XSRF cookies that were not generated by this module (i.e. strings without
any particular formatting) are once again accepted (as long as the
cookie and body/header match).  This pattern was common for
testing and non-browser clients but was broken by the changes in
Tornado 3.2.2.

\end{itemize}


\paragraph{\sphinxstyleliteralintitle{\sphinxupquote{tornado.websocket}}}
\label{\detokenize{releases/v4.0.0:tornado-websocket}}\begin{itemize}
\item {} 
WebSocket connections from other origin sites are now rejected by default.
Browsers do not use the same-origin policy for WebSocket connections as they
do for most other browser-initiated communications.  This can be surprising
and a security risk, so we disallow these connections on the server side
by default.  To accept cross-origin websocket connections, override
the new method {\hyperref[\detokenize{websocket:tornado.websocket.WebSocketHandler.check_origin}]{\sphinxcrossref{\sphinxcode{\sphinxupquote{WebSocketHandler.check\_origin}}}}}.

\item {} 
{\hyperref[\detokenize{websocket:tornado.websocket.WebSocketHandler.close}]{\sphinxcrossref{\sphinxcode{\sphinxupquote{WebSocketHandler.close}}}}} and {\hyperref[\detokenize{websocket:tornado.websocket.WebSocketClientConnection.close}]{\sphinxcrossref{\sphinxcode{\sphinxupquote{WebSocketClientConnection.close}}}}} now
support \sphinxcode{\sphinxupquote{code}} and \sphinxcode{\sphinxupquote{reason}} arguments to send a status code and
message to the other side of the connection when closing.  Both classes
also have \sphinxcode{\sphinxupquote{close\_code}} and \sphinxcode{\sphinxupquote{close\_reason}} attributes to receive these
values when the other side closes.

\item {} 
The C speedup module now builds correctly with MSVC, and can support
messages larger than 2GB on 64-bit systems.

\item {} 
The fallback mechanism for detecting a missing C compiler now
works correctly on Mac OS X.

\item {} 
Arguments to {\hyperref[\detokenize{websocket:tornado.websocket.WebSocketHandler.open}]{\sphinxcrossref{\sphinxcode{\sphinxupquote{WebSocketHandler.open}}}}} are now decoded in the same way
as arguments to {\hyperref[\detokenize{web:tornado.web.RequestHandler.get}]{\sphinxcrossref{\sphinxcode{\sphinxupquote{RequestHandler.get}}}}} and similar methods.

\item {} 
It is now allowed to override \sphinxcode{\sphinxupquote{prepare}} in a {\hyperref[\detokenize{websocket:tornado.websocket.WebSocketHandler}]{\sphinxcrossref{\sphinxcode{\sphinxupquote{WebSocketHandler}}}}},
and this method may generate HTTP responses (error pages) in the usual
way.  The HTTP response methods are still not allowed once the
WebSocket handshake has completed.

\end{itemize}


\paragraph{\sphinxstyleliteralintitle{\sphinxupquote{tornado.wsgi}}}
\label{\detokenize{releases/v4.0.0:tornado-wsgi}}\begin{itemize}
\item {} 
New class \sphinxcode{\sphinxupquote{WSGIAdapter}} supports running a Tornado {\hyperref[\detokenize{web:tornado.web.Application}]{\sphinxcrossref{\sphinxcode{\sphinxupquote{Application}}}}} on
a WSGI server in a way that is more compatible with Tornado’s non-WSGI
{\hyperref[\detokenize{httpserver:tornado.httpserver.HTTPServer}]{\sphinxcrossref{\sphinxcode{\sphinxupquote{HTTPServer}}}}}.  \sphinxcode{\sphinxupquote{WSGIApplication}} is deprecated in favor of using
\sphinxcode{\sphinxupquote{WSGIAdapter}} with a regular {\hyperref[\detokenize{web:tornado.web.Application}]{\sphinxcrossref{\sphinxcode{\sphinxupquote{Application}}}}}.

\item {} 
\sphinxcode{\sphinxupquote{WSGIAdapter}} now supports gzipped output.

\end{itemize}


\subsection{What’s new in Tornado 3.2.2}
\label{\detokenize{releases/v3.2.2:what-s-new-in-tornado-3-2-2}}\label{\detokenize{releases/v3.2.2::doc}}

\subsubsection{June 3, 2014}
\label{\detokenize{releases/v3.2.2:june-3-2014}}

\paragraph{Security fixes}
\label{\detokenize{releases/v3.2.2:security-fixes}}\begin{itemize}
\item {} 
The XSRF token is now encoded with a random mask on each request.
This makes it safe to include in compressed pages without being
vulnerable to the \sphinxhref{http://breachattack.com}{BREACH attack}.
This applies to most applications that use both the \sphinxcode{\sphinxupquote{xsrf\_cookies}}
and \sphinxcode{\sphinxupquote{gzip}} options (or have gzip applied by a proxy).

\end{itemize}


\paragraph{Backwards-compatibility notes}
\label{\detokenize{releases/v3.2.2:backwards-compatibility-notes}}\begin{itemize}
\item {} 
If Tornado 3.2.2 is run at the same time as older versions on the same
domain, there is some potential for issues with the differing cookie
versions.  The {\hyperref[\detokenize{web:tornado.web.Application}]{\sphinxcrossref{\sphinxcode{\sphinxupquote{Application}}}}} setting \sphinxcode{\sphinxupquote{xsrf\_cookie\_version=1}} can
be used for a transitional period to generate the older cookie format
on newer servers.

\end{itemize}


\paragraph{Other changes}
\label{\detokenize{releases/v3.2.2:other-changes}}\begin{itemize}
\item {} 
\sphinxcode{\sphinxupquote{tornado.platform.asyncio}} is now compatible with \sphinxcode{\sphinxupquote{trollius}} version 0.3.

\end{itemize}


\subsection{What’s new in Tornado 3.2.1}
\label{\detokenize{releases/v3.2.1:what-s-new-in-tornado-3-2-1}}\label{\detokenize{releases/v3.2.1::doc}}

\subsubsection{May 5, 2014}
\label{\detokenize{releases/v3.2.1:may-5-2014}}

\paragraph{Security fixes}
\label{\detokenize{releases/v3.2.1:security-fixes}}\begin{itemize}
\item {} 
The signed-value format used by {\hyperref[\detokenize{web:tornado.web.RequestHandler.set_secure_cookie}]{\sphinxcrossref{\sphinxcode{\sphinxupquote{RequestHandler.set\_secure\_cookie}}}}}
and {\hyperref[\detokenize{web:tornado.web.RequestHandler.get_secure_cookie}]{\sphinxcrossref{\sphinxcode{\sphinxupquote{RequestHandler.get\_secure\_cookie}}}}} has changed to be more secure.
\sphinxstylestrong{This is a disruptive change}.  The \sphinxcode{\sphinxupquote{secure\_cookie}} functions
take new \sphinxcode{\sphinxupquote{version}} parameters to support transitions between cookie
formats.

\item {} 
The new cookie format fixes a vulnerability that may be present in
applications that use multiple cookies where the name of one cookie
is a prefix of the name of another.

\item {} 
To minimize disruption, cookies in the older format will be accepted
by default until they expire.  Applications that may be vulnerable
can reject all cookies in the older format by passing \sphinxcode{\sphinxupquote{min\_version=2}}
to {\hyperref[\detokenize{web:tornado.web.RequestHandler.get_secure_cookie}]{\sphinxcrossref{\sphinxcode{\sphinxupquote{RequestHandler.get\_secure\_cookie}}}}}.

\item {} 
Thanks to Joost Pol of \sphinxhref{https://www.certifiedsecure.com}{Certified Secure}
for reporting this issue.

\end{itemize}


\paragraph{Backwards-compatibility notes}
\label{\detokenize{releases/v3.2.1:backwards-compatibility-notes}}\begin{itemize}
\item {} 
Signed cookies issued by {\hyperref[\detokenize{web:tornado.web.RequestHandler.set_secure_cookie}]{\sphinxcrossref{\sphinxcode{\sphinxupquote{RequestHandler.set\_secure\_cookie}}}}} in Tornado
3.2.1 cannot be read by older releases.  If you need to run 3.2.1
in parallel with older releases, you can pass \sphinxcode{\sphinxupquote{version=1}} to
{\hyperref[\detokenize{web:tornado.web.RequestHandler.set_secure_cookie}]{\sphinxcrossref{\sphinxcode{\sphinxupquote{RequestHandler.set\_secure\_cookie}}}}} to issue cookies that are
backwards-compatible (but have a known weakness, so this option
should only be used for a transitional period).

\end{itemize}


\paragraph{Other changes}
\label{\detokenize{releases/v3.2.1:other-changes}}\begin{itemize}
\item {} 
The C extension used to speed up the websocket module now compiles
correctly on Windows with MSVC and 64-bit mode.  The fallback to
the pure-Python alternative now works correctly on Mac OS X machines
with no C compiler installed.

\end{itemize}


\subsection{What’s new in Tornado 3.2}
\label{\detokenize{releases/v3.2.0:what-s-new-in-tornado-3-2}}\label{\detokenize{releases/v3.2.0::doc}}

\subsubsection{Jan 14, 2014}
\label{\detokenize{releases/v3.2.0:jan-14-2014}}

\paragraph{Installation}
\label{\detokenize{releases/v3.2.0:installation}}\begin{itemize}
\item {} 
Tornado now depends on the \sphinxhref{https://pypi.python.org/pypi/backports.ssl\_match\_hostname}{backports.ssl\_match\_hostname} when
running on Python 2.  This will be installed automatically when using \sphinxcode{\sphinxupquote{pip}}
or \sphinxcode{\sphinxupquote{easy\_install}}

\item {} 
Tornado now includes an optional C extension module, which greatly improves
performance of websockets.  This extension will be built automatically
if a C compiler is found at install time.

\end{itemize}


\paragraph{New modules}
\label{\detokenize{releases/v3.2.0:new-modules}}\begin{itemize}
\item {} 
The {\hyperref[\detokenize{asyncio:module-tornado.platform.asyncio}]{\sphinxcrossref{\sphinxcode{\sphinxupquote{tornado.platform.asyncio}}}}} module provides integration with the
\sphinxcode{\sphinxupquote{asyncio}} module introduced in Python 3.4 (also available for Python
3.3 with \sphinxcode{\sphinxupquote{pip install asyncio}}).

\end{itemize}


\paragraph{\sphinxstyleliteralintitle{\sphinxupquote{tornado.auth}}}
\label{\detokenize{releases/v3.2.0:tornado-auth}}\begin{itemize}
\item {} 
Added {\hyperref[\detokenize{auth:tornado.auth.GoogleOAuth2Mixin}]{\sphinxcrossref{\sphinxcode{\sphinxupquote{GoogleOAuth2Mixin}}}}} support authentication to Google services
with OAuth 2 instead of OpenID and OAuth 1.

\item {} 
{\hyperref[\detokenize{auth:tornado.auth.FacebookGraphMixin}]{\sphinxcrossref{\sphinxcode{\sphinxupquote{FacebookGraphMixin}}}}} has been updated to use the current Facebook login
URL, which saves a redirect.

\end{itemize}


\paragraph{\sphinxstyleliteralintitle{\sphinxupquote{tornado.concurrent}}}
\label{\detokenize{releases/v3.2.0:tornado-concurrent}}\begin{itemize}
\item {} 
\sphinxcode{\sphinxupquote{TracebackFuture}} now accepts a \sphinxcode{\sphinxupquote{timeout}} keyword argument (although
it is still incorrect to use a non-zero timeout in non-blocking code).

\end{itemize}


\paragraph{\sphinxstyleliteralintitle{\sphinxupquote{tornado.curl\_httpclient}}}
\label{\detokenize{releases/v3.2.0:tornado-curl-httpclient}}\begin{itemize}
\item {} 
\sphinxcode{\sphinxupquote{tornado.curl\_httpclient}} now works on Python 3 with the
soon-to-be-released pycurl 7.19.3, which will officially support
Python 3 for the first time.  Note that there are some unofficial
Python 3 ports of pycurl (Ubuntu has included one for its past
several releases); these are not supported for use with Tornado.

\end{itemize}


\paragraph{\sphinxstyleliteralintitle{\sphinxupquote{tornado.escape}}}
\label{\detokenize{releases/v3.2.0:tornado-escape}}\begin{itemize}
\item {} 
{\hyperref[\detokenize{escape:tornado.escape.xhtml_escape}]{\sphinxcrossref{\sphinxcode{\sphinxupquote{xhtml\_escape}}}}} now escapes apostrophes as well.

\item {} 
{\hyperref[\detokenize{escape:tornado.escape.utf8}]{\sphinxcrossref{\sphinxcode{\sphinxupquote{tornado.escape.utf8}}}}}, {\hyperref[\detokenize{escape:tornado.escape.to_unicode}]{\sphinxcrossref{\sphinxcode{\sphinxupquote{to\_unicode}}}}}, and {\hyperref[\detokenize{escape:tornado.escape.native_str}]{\sphinxcrossref{\sphinxcode{\sphinxupquote{native\_str}}}}} now raise
\sphinxhref{https://docs.python.org/3.6/library/exceptions.html\#TypeError}{\sphinxcode{\sphinxupquote{TypeError}}} instead of \sphinxhref{https://docs.python.org/3.6/library/exceptions.html\#AssertionError}{\sphinxcode{\sphinxupquote{AssertionError}}} when given an invalid value.

\end{itemize}


\paragraph{\sphinxstyleliteralintitle{\sphinxupquote{tornado.gen}}}
\label{\detokenize{releases/v3.2.0:tornado-gen}}\begin{itemize}
\item {} 
Coroutines may now yield dicts in addition to lists to wait for
multiple tasks in parallel.

\item {} 
Improved performance of {\hyperref[\detokenize{gen:module-tornado.gen}]{\sphinxcrossref{\sphinxcode{\sphinxupquote{tornado.gen}}}}} when yielding a {\hyperref[\detokenize{concurrent:tornado.concurrent.Future}]{\sphinxcrossref{\sphinxcode{\sphinxupquote{Future}}}}} that is
already done.

\end{itemize}


\paragraph{\sphinxstyleliteralintitle{\sphinxupquote{tornado.httpclient}}}
\label{\detokenize{releases/v3.2.0:tornado-httpclient}}\begin{itemize}
\item {} 
{\hyperref[\detokenize{httpclient:tornado.httpclient.HTTPRequest}]{\sphinxcrossref{\sphinxcode{\sphinxupquote{tornado.httpclient.HTTPRequest}}}}} now uses property setters so that
setting attributes after construction applies the same conversions
as \sphinxcode{\sphinxupquote{\_\_init\_\_}} (e.g. converting the body attribute to bytes).

\end{itemize}


\paragraph{\sphinxstyleliteralintitle{\sphinxupquote{tornado.httpserver}}}
\label{\detokenize{releases/v3.2.0:tornado-httpserver}}\begin{itemize}
\item {} 
Malformed \sphinxcode{\sphinxupquote{x-www-form-urlencoded}} request bodies will now log a warning
and continue instead of causing the request to fail (similar to the existing
handling of malformed \sphinxcode{\sphinxupquote{multipart/form-data}} bodies.  This is done mainly
because some libraries send this content type by default even when the data
is not form-encoded.

\item {} 
Fix some error messages for unix sockets (and other non-IP sockets)

\end{itemize}


\paragraph{\sphinxstyleliteralintitle{\sphinxupquote{tornado.ioloop}}}
\label{\detokenize{releases/v3.2.0:tornado-ioloop}}\begin{itemize}
\item {} 
{\hyperref[\detokenize{ioloop:tornado.ioloop.IOLoop}]{\sphinxcrossref{\sphinxcode{\sphinxupquote{IOLoop}}}}} now uses \sphinxcode{\sphinxupquote{IOLoop.handle\_callback\_exception}} consistently for
error logging.

\item {} 
{\hyperref[\detokenize{ioloop:tornado.ioloop.IOLoop}]{\sphinxcrossref{\sphinxcode{\sphinxupquote{IOLoop}}}}} now frees callback objects earlier, reducing memory usage
while idle.

\item {} 
{\hyperref[\detokenize{ioloop:tornado.ioloop.IOLoop}]{\sphinxcrossref{\sphinxcode{\sphinxupquote{IOLoop}}}}} will no longer call \sphinxhref{https://docs.python.org/3.6/library/logging.html\#logging.basicConfig}{\sphinxcode{\sphinxupquote{logging.basicConfig}}} if there is a handler
defined for the root logger or for the \sphinxcode{\sphinxupquote{tornado}} or \sphinxcode{\sphinxupquote{tornado.application}}
loggers (previously it only looked at the root logger).

\end{itemize}


\paragraph{\sphinxstyleliteralintitle{\sphinxupquote{tornado.iostream}}}
\label{\detokenize{releases/v3.2.0:tornado-iostream}}\begin{itemize}
\item {} 
{\hyperref[\detokenize{iostream:tornado.iostream.IOStream}]{\sphinxcrossref{\sphinxcode{\sphinxupquote{IOStream}}}}} now recognizes \sphinxcode{\sphinxupquote{ECONNABORTED}} error codes in more places
(which was mainly an issue on Windows).

\item {} 
{\hyperref[\detokenize{iostream:tornado.iostream.IOStream}]{\sphinxcrossref{\sphinxcode{\sphinxupquote{IOStream}}}}} now frees memory earlier if a connection is closed while
there is data in the write buffer.

\item {} 
{\hyperref[\detokenize{iostream:tornado.iostream.PipeIOStream}]{\sphinxcrossref{\sphinxcode{\sphinxupquote{PipeIOStream}}}}} now handles \sphinxcode{\sphinxupquote{EAGAIN}} error codes correctly.

\item {} 
{\hyperref[\detokenize{iostream:tornado.iostream.SSLIOStream}]{\sphinxcrossref{\sphinxcode{\sphinxupquote{SSLIOStream}}}}} now initiates the SSL handshake automatically without
waiting for the application to try and read or write to the connection.

\item {} 
Swallow a spurious exception from \sphinxcode{\sphinxupquote{set\_nodelay}} when a connection
has been reset.

\end{itemize}


\paragraph{\sphinxstyleliteralintitle{\sphinxupquote{tornado.locale}}}
\label{\detokenize{releases/v3.2.0:tornado-locale}}\begin{itemize}
\item {} 
{\hyperref[\detokenize{locale:tornado.locale.Locale.format_date}]{\sphinxcrossref{\sphinxcode{\sphinxupquote{Locale.format\_date}}}}} no longer forces the use of absolute
dates in Russian.

\end{itemize}


\paragraph{\sphinxstyleliteralintitle{\sphinxupquote{tornado.log}}}
\label{\detokenize{releases/v3.2.0:tornado-log}}\begin{itemize}
\item {} 
Fix an error from {\hyperref[\detokenize{log:tornado.log.enable_pretty_logging}]{\sphinxcrossref{\sphinxcode{\sphinxupquote{tornado.log.enable\_pretty\_logging}}}}} when
\sphinxhref{https://docs.python.org/3.6/library/sys.html\#sys.stderr}{\sphinxcode{\sphinxupquote{sys.stderr}}} does not have an \sphinxcode{\sphinxupquote{isatty}} method.

\item {} 
{\hyperref[\detokenize{log:tornado.log.LogFormatter}]{\sphinxcrossref{\sphinxcode{\sphinxupquote{tornado.log.LogFormatter}}}}} now accepts keyword arguments \sphinxcode{\sphinxupquote{fmt}}
and \sphinxcode{\sphinxupquote{datefmt}}.

\end{itemize}


\paragraph{\sphinxstyleliteralintitle{\sphinxupquote{tornado.netutil}}}
\label{\detokenize{releases/v3.2.0:tornado-netutil}}\begin{itemize}
\item {} 
{\hyperref[\detokenize{netutil:tornado.netutil.is_valid_ip}]{\sphinxcrossref{\sphinxcode{\sphinxupquote{is\_valid\_ip}}}}} (and therefore \sphinxcode{\sphinxupquote{HTTPRequest.remote\_ip}}) now rejects
empty strings.

\item {} 
Synchronously using {\hyperref[\detokenize{netutil:tornado.netutil.ThreadedResolver}]{\sphinxcrossref{\sphinxcode{\sphinxupquote{ThreadedResolver}}}}} at import time to resolve
a unicode hostname no longer deadlocks.

\end{itemize}


\paragraph{\sphinxstyleliteralintitle{\sphinxupquote{tornado.platform.twisted}}}
\label{\detokenize{releases/v3.2.0:tornado-platform-twisted}}\begin{itemize}
\item {} 
{\hyperref[\detokenize{twisted:tornado.platform.twisted.TwistedResolver}]{\sphinxcrossref{\sphinxcode{\sphinxupquote{TwistedResolver}}}}} now has better error handling.

\end{itemize}


\paragraph{\sphinxstyleliteralintitle{\sphinxupquote{tornado.process}}}
\label{\detokenize{releases/v3.2.0:tornado-process}}\begin{itemize}
\item {} 
{\hyperref[\detokenize{process:tornado.process.Subprocess}]{\sphinxcrossref{\sphinxcode{\sphinxupquote{Subprocess}}}}} no longer leaks file descriptors if \sphinxhref{https://docs.python.org/3.6/library/subprocess.html\#subprocess.Popen}{\sphinxcode{\sphinxupquote{subprocess.Popen}}} fails.

\end{itemize}


\paragraph{\sphinxstyleliteralintitle{\sphinxupquote{tornado.simple\_httpclient}}}
\label{\detokenize{releases/v3.2.0:tornado-simple-httpclient}}\begin{itemize}
\item {} 
\sphinxcode{\sphinxupquote{simple\_httpclient}} now applies the \sphinxcode{\sphinxupquote{connect\_timeout}} to requests
that are queued and have not yet started.

\item {} 
On Python 2.6, \sphinxcode{\sphinxupquote{simple\_httpclient}} now uses TLSv1 instead of SSLv3.

\item {} 
\sphinxcode{\sphinxupquote{simple\_httpclient}} now enforces the connect timeout during DNS resolution.

\item {} 
The embedded \sphinxcode{\sphinxupquote{ca-certificates.crt}} file has been updated with the current
Mozilla CA list.

\end{itemize}


\paragraph{\sphinxstyleliteralintitle{\sphinxupquote{tornado.web}}}
\label{\detokenize{releases/v3.2.0:tornado-web}}\begin{itemize}
\item {} 
{\hyperref[\detokenize{web:tornado.web.StaticFileHandler}]{\sphinxcrossref{\sphinxcode{\sphinxupquote{StaticFileHandler}}}}} no longer fails if the client requests a \sphinxcode{\sphinxupquote{Range}} that
is larger than the entire file (Facebook has a crawler that does this).

\item {} 
{\hyperref[\detokenize{web:tornado.web.RequestHandler.on_connection_close}]{\sphinxcrossref{\sphinxcode{\sphinxupquote{RequestHandler.on\_connection\_close}}}}} now works correctly on subsequent
requests of a keep-alive connection.

\item {} 
New application setting \sphinxcode{\sphinxupquote{default\_handler\_class}} can be used to easily
set up custom 404 pages.

\item {} 
New application settings \sphinxcode{\sphinxupquote{autoreload}}, \sphinxcode{\sphinxupquote{compiled\_template\_cache}},
\sphinxcode{\sphinxupquote{static\_hash\_cache}}, and \sphinxcode{\sphinxupquote{serve\_traceback}} can be used to control
individual aspects of debug mode.

\item {} 
New methods {\hyperref[\detokenize{web:tornado.web.RequestHandler.get_query_argument}]{\sphinxcrossref{\sphinxcode{\sphinxupquote{RequestHandler.get\_query\_argument}}}}} and
{\hyperref[\detokenize{web:tornado.web.RequestHandler.get_body_argument}]{\sphinxcrossref{\sphinxcode{\sphinxupquote{RequestHandler.get\_body\_argument}}}}} and new attributes
\sphinxcode{\sphinxupquote{HTTPRequest.query\_arguments}} and \sphinxcode{\sphinxupquote{HTTPRequest.body\_arguments}} allow access
to arguments without intermingling those from the query string with those
from the request body.

\item {} 
{\hyperref[\detokenize{web:tornado.web.RequestHandler.decode_argument}]{\sphinxcrossref{\sphinxcode{\sphinxupquote{RequestHandler.decode\_argument}}}}} and related methods now raise
an \sphinxcode{\sphinxupquote{HTTPError(400)}} instead of \sphinxhref{https://docs.python.org/3.6/library/exceptions.html\#UnicodeDecodeError}{\sphinxcode{\sphinxupquote{UnicodeDecodeError}}} when the
argument could not be decoded.

\item {} 
{\hyperref[\detokenize{web:tornado.web.RequestHandler.clear_all_cookies}]{\sphinxcrossref{\sphinxcode{\sphinxupquote{RequestHandler.clear\_all\_cookies}}}}} now accepts \sphinxcode{\sphinxupquote{domain}} and \sphinxcode{\sphinxupquote{path}}
arguments, just like {\hyperref[\detokenize{web:tornado.web.RequestHandler.clear_cookie}]{\sphinxcrossref{\sphinxcode{\sphinxupquote{clear\_cookie}}}}}.

\item {} 
It is now possible to specify handlers by name when using the
{\hyperref[\detokenize{web:tornado.web.URLSpec}]{\sphinxcrossref{\sphinxcode{\sphinxupquote{tornado.web.URLSpec}}}}} class.

\item {} 
{\hyperref[\detokenize{web:tornado.web.Application}]{\sphinxcrossref{\sphinxcode{\sphinxupquote{Application}}}}} now accepts 4-tuples to specify the \sphinxcode{\sphinxupquote{name}} parameter
(which previously required constructing a {\hyperref[\detokenize{web:tornado.web.URLSpec}]{\sphinxcrossref{\sphinxcode{\sphinxupquote{tornado.web.URLSpec}}}}} object
instead of a tuple).

\item {} 
Fixed an incorrect error message when handler methods return a value
other than None or a Future.

\item {} 
Exceptions will no longer be logged twice when using both \sphinxcode{\sphinxupquote{@asynchronous}}
and \sphinxcode{\sphinxupquote{@gen.coroutine}}

\end{itemize}


\paragraph{\sphinxstyleliteralintitle{\sphinxupquote{tornado.websocket}}}
\label{\detokenize{releases/v3.2.0:tornado-websocket}}\begin{itemize}
\item {} 
{\hyperref[\detokenize{websocket:tornado.websocket.WebSocketHandler.write_message}]{\sphinxcrossref{\sphinxcode{\sphinxupquote{WebSocketHandler.write\_message}}}}} now raises {\hyperref[\detokenize{websocket:tornado.websocket.WebSocketClosedError}]{\sphinxcrossref{\sphinxcode{\sphinxupquote{WebSocketClosedError}}}}} instead
of \sphinxhref{https://docs.python.org/3.6/library/exceptions.html\#AttributeError}{\sphinxcode{\sphinxupquote{AttributeError}}} when the connection has been closed.

\item {} 
{\hyperref[\detokenize{websocket:tornado.websocket.websocket_connect}]{\sphinxcrossref{\sphinxcode{\sphinxupquote{websocket\_connect}}}}} now accepts preconstructed \sphinxcode{\sphinxupquote{HTTPRequest}} objects.

\item {} 
Fix a bug with {\hyperref[\detokenize{websocket:tornado.websocket.WebSocketHandler}]{\sphinxcrossref{\sphinxcode{\sphinxupquote{WebSocketHandler}}}}} when used with some proxies that
unconditionally modify the \sphinxcode{\sphinxupquote{Connection}} header.

\item {} 
{\hyperref[\detokenize{websocket:tornado.websocket.websocket_connect}]{\sphinxcrossref{\sphinxcode{\sphinxupquote{websocket\_connect}}}}} now returns an error immediately for refused connections
instead of waiting for the timeout.

\item {} 
{\hyperref[\detokenize{websocket:tornado.websocket.WebSocketClientConnection}]{\sphinxcrossref{\sphinxcode{\sphinxupquote{WebSocketClientConnection}}}}} now has a \sphinxcode{\sphinxupquote{close}} method.

\end{itemize}


\paragraph{\sphinxstyleliteralintitle{\sphinxupquote{tornado.wsgi}}}
\label{\detokenize{releases/v3.2.0:tornado-wsgi}}\begin{itemize}
\item {} 
{\hyperref[\detokenize{wsgi:tornado.wsgi.WSGIContainer}]{\sphinxcrossref{\sphinxcode{\sphinxupquote{WSGIContainer}}}}} now calls the iterable’s \sphinxcode{\sphinxupquote{close()}} method even if
an error is raised, in compliance with the spec.

\end{itemize}


\subsection{What’s new in Tornado 3.1.1}
\label{\detokenize{releases/v3.1.1:what-s-new-in-tornado-3-1-1}}\label{\detokenize{releases/v3.1.1::doc}}

\subsubsection{Sep 1, 2013}
\label{\detokenize{releases/v3.1.1:sep-1-2013}}\begin{itemize}
\item {} 
{\hyperref[\detokenize{web:tornado.web.StaticFileHandler}]{\sphinxcrossref{\sphinxcode{\sphinxupquote{StaticFileHandler}}}}} no longer fails if the client requests a \sphinxcode{\sphinxupquote{Range}} that
is larger than the entire file (Facebook has a crawler that does this).

\item {} 
{\hyperref[\detokenize{web:tornado.web.RequestHandler.on_connection_close}]{\sphinxcrossref{\sphinxcode{\sphinxupquote{RequestHandler.on\_connection\_close}}}}} now works correctly on subsequent
requests of a keep-alive connection.

\end{itemize}


\subsection{What’s new in Tornado 3.1}
\label{\detokenize{releases/v3.1.0:what-s-new-in-tornado-3-1}}\label{\detokenize{releases/v3.1.0::doc}}

\subsubsection{Jun 15, 2013}
\label{\detokenize{releases/v3.1.0:jun-15-2013}}

\paragraph{Multiple modules}
\label{\detokenize{releases/v3.1.0:multiple-modules}}\begin{itemize}
\item {} 
Many reference cycles have been broken up throughout the package,
allowing for more efficient garbage collection on CPython.

\item {} 
Silenced some log messages when connections are opened and immediately
closed (i.e. port scans), or other situations related to closed
connections.

\item {} 
Various small speedups: {\hyperref[\detokenize{httputil:tornado.httputil.HTTPHeaders}]{\sphinxcrossref{\sphinxcode{\sphinxupquote{HTTPHeaders}}}}} case normalization, {\hyperref[\detokenize{web:tornado.web.UIModule}]{\sphinxcrossref{\sphinxcode{\sphinxupquote{UIModule}}}}}
proxy objects, precompile some regexes.

\end{itemize}


\paragraph{\sphinxstyleliteralintitle{\sphinxupquote{tornado.auth}}}
\label{\detokenize{releases/v3.1.0:tornado-auth}}\begin{itemize}
\item {} 
{\hyperref[\detokenize{auth:tornado.auth.OAuthMixin}]{\sphinxcrossref{\sphinxcode{\sphinxupquote{OAuthMixin}}}}} always sends \sphinxcode{\sphinxupquote{oauth\_version=1.0}} in its
request as required by the spec.

\item {} 
{\hyperref[\detokenize{auth:tornado.auth.FacebookGraphMixin}]{\sphinxcrossref{\sphinxcode{\sphinxupquote{FacebookGraphMixin}}}}} now uses \sphinxcode{\sphinxupquote{self.\_FACEBOOK\_BASE\_URL}}
in {\hyperref[\detokenize{auth:tornado.auth.FacebookGraphMixin.facebook_request}]{\sphinxcrossref{\sphinxcode{\sphinxupquote{facebook\_request}}}}} to allow the base url to be
overridden.

\item {} 
The \sphinxcode{\sphinxupquote{authenticate\_redirect}} and \sphinxcode{\sphinxupquote{authorize\_redirect}} methods in the
{\hyperref[\detokenize{auth:module-tornado.auth}]{\sphinxcrossref{\sphinxcode{\sphinxupquote{tornado.auth}}}}} mixin classes all now return Futures.  These methods
are asynchronous in {\hyperref[\detokenize{auth:tornado.auth.OAuthMixin}]{\sphinxcrossref{\sphinxcode{\sphinxupquote{OAuthMixin}}}}} and derived classes, although they
do not take a callback.  The {\hyperref[\detokenize{concurrent:tornado.concurrent.Future}]{\sphinxcrossref{\sphinxcode{\sphinxupquote{Future}}}}} these methods return must be
yielded if they are called from a function decorated with {\hyperref[\detokenize{gen:tornado.gen.coroutine}]{\sphinxcrossref{\sphinxcode{\sphinxupquote{gen.coroutine}}}}}
(but not \sphinxcode{\sphinxupquote{gen.engine}}).

\item {} 
{\hyperref[\detokenize{auth:tornado.auth.TwitterMixin}]{\sphinxcrossref{\sphinxcode{\sphinxupquote{TwitterMixin}}}}} now uses \sphinxcode{\sphinxupquote{/account/verify\_credentials}} to get information
about the logged-in user, which is more robust against changing screen
names.

\item {} 
The \sphinxcode{\sphinxupquote{demos}} directory (in the source distribution) has a new
\sphinxcode{\sphinxupquote{twitter}} demo using {\hyperref[\detokenize{auth:tornado.auth.TwitterMixin}]{\sphinxcrossref{\sphinxcode{\sphinxupquote{TwitterMixin}}}}}.

\end{itemize}


\paragraph{\sphinxstyleliteralintitle{\sphinxupquote{tornado.escape}}}
\label{\detokenize{releases/v3.1.0:tornado-escape}}\begin{itemize}
\item {} 
{\hyperref[\detokenize{escape:tornado.escape.url_escape}]{\sphinxcrossref{\sphinxcode{\sphinxupquote{url\_escape}}}}} and {\hyperref[\detokenize{escape:tornado.escape.url_unescape}]{\sphinxcrossref{\sphinxcode{\sphinxupquote{url\_unescape}}}}} have a new \sphinxcode{\sphinxupquote{plus}} argument (defaulting
to True for consistency with the previous behavior) which specifies
whether they work like \sphinxhref{https://docs.python.org/3.6/library/urllib.parse.html\#urllib.parse.unquote}{\sphinxcode{\sphinxupquote{urllib.parse.unquote}}} or \sphinxhref{https://docs.python.org/3.6/library/urllib.parse.html\#urllib.parse.unquote\_plus}{\sphinxcode{\sphinxupquote{urllib.parse.unquote\_plus}}}.

\end{itemize}


\paragraph{\sphinxstyleliteralintitle{\sphinxupquote{tornado.gen}}}
\label{\detokenize{releases/v3.1.0:tornado-gen}}\begin{itemize}
\item {} 
Fixed a potential memory leak with long chains of {\hyperref[\detokenize{gen:module-tornado.gen}]{\sphinxcrossref{\sphinxcode{\sphinxupquote{tornado.gen}}}}} coroutines.

\end{itemize}


\paragraph{\sphinxstyleliteralintitle{\sphinxupquote{tornado.httpclient}}}
\label{\detokenize{releases/v3.1.0:tornado-httpclient}}\begin{itemize}
\item {} 
{\hyperref[\detokenize{httpclient:tornado.httpclient.HTTPRequest}]{\sphinxcrossref{\sphinxcode{\sphinxupquote{tornado.httpclient.HTTPRequest}}}}} takes a new argument \sphinxcode{\sphinxupquote{auth\_mode}},
which can be either \sphinxcode{\sphinxupquote{basic}} or \sphinxcode{\sphinxupquote{digest}}.  Digest authentication
is only supported with \sphinxcode{\sphinxupquote{tornado.curl\_httpclient}}.

\item {} 
\sphinxcode{\sphinxupquote{tornado.curl\_httpclient}} no longer goes into an infinite loop when
pycurl returns a negative timeout.

\item {} 
\sphinxcode{\sphinxupquote{curl\_httpclient}} now supports the \sphinxcode{\sphinxupquote{PATCH}} and \sphinxcode{\sphinxupquote{OPTIONS}} methods
without the use of \sphinxcode{\sphinxupquote{allow\_nonstandard\_methods=True}}.

\item {} 
Worked around a class of bugs in libcurl that would result in
errors from {\hyperref[\detokenize{ioloop:tornado.ioloop.IOLoop.update_handler}]{\sphinxcrossref{\sphinxcode{\sphinxupquote{IOLoop.update\_handler}}}}} in various scenarios including
digest authentication and socks proxies.

\item {} 
The \sphinxcode{\sphinxupquote{TCP\_NODELAY}} flag is now set when appropriate in \sphinxcode{\sphinxupquote{simple\_httpclient}}.

\item {} 
\sphinxcode{\sphinxupquote{simple\_httpclient}} no longer logs exceptions, since those exceptions
are made available to the caller as \sphinxcode{\sphinxupquote{HTTPResponse.error}}.

\end{itemize}


\paragraph{\sphinxstyleliteralintitle{\sphinxupquote{tornado.httpserver}}}
\label{\detokenize{releases/v3.1.0:tornado-httpserver}}\begin{itemize}
\item {} 
{\hyperref[\detokenize{httpserver:tornado.httpserver.HTTPServer}]{\sphinxcrossref{\sphinxcode{\sphinxupquote{tornado.httpserver.HTTPServer}}}}} handles malformed HTTP headers more
gracefully.

\item {} 
{\hyperref[\detokenize{httpserver:tornado.httpserver.HTTPServer}]{\sphinxcrossref{\sphinxcode{\sphinxupquote{HTTPServer}}}}} now supports lists of IPs in \sphinxcode{\sphinxupquote{X-Forwarded-For}}
(it chooses the last, i.e. nearest one).

\item {} 
Memory is now reclaimed promptly on CPython when an HTTP request
fails because it exceeded the maximum upload size.

\item {} 
The \sphinxcode{\sphinxupquote{TCP\_NODELAY}} flag is now set when appropriate in {\hyperref[\detokenize{httpserver:tornado.httpserver.HTTPServer}]{\sphinxcrossref{\sphinxcode{\sphinxupquote{HTTPServer}}}}}.

\item {} 
The {\hyperref[\detokenize{httpserver:tornado.httpserver.HTTPServer}]{\sphinxcrossref{\sphinxcode{\sphinxupquote{HTTPServer}}}}} \sphinxcode{\sphinxupquote{no\_keep\_alive}} option is now respected with
HTTP 1.0 connections that explicitly pass \sphinxcode{\sphinxupquote{Connection: keep-alive}}.

\item {} 
The \sphinxcode{\sphinxupquote{Connection: keep-alive}} check for HTTP 1.0 connections is now
case-insensitive.

\item {} 
The \sphinxhref{https://docs.python.org/3.6/library/stdtypes.html\#str}{\sphinxcode{\sphinxupquote{str}}} and \sphinxhref{https://docs.python.org/3.6/library/functions.html\#repr}{\sphinxcode{\sphinxupquote{repr}}} of \sphinxcode{\sphinxupquote{tornado.httpserver.HTTPRequest}} no longer
include the request body, reducing log spam on errors (and potential
exposure/retention of private data).

\end{itemize}


\paragraph{\sphinxstyleliteralintitle{\sphinxupquote{tornado.httputil}}}
\label{\detokenize{releases/v3.1.0:tornado-httputil}}\begin{itemize}
\item {} 
The cache used in {\hyperref[\detokenize{httputil:tornado.httputil.HTTPHeaders}]{\sphinxcrossref{\sphinxcode{\sphinxupquote{HTTPHeaders}}}}} will no longer grow without bound.

\end{itemize}


\paragraph{\sphinxstyleliteralintitle{\sphinxupquote{tornado.ioloop}}}
\label{\detokenize{releases/v3.1.0:tornado-ioloop}}\begin{itemize}
\item {} 
Some {\hyperref[\detokenize{ioloop:tornado.ioloop.IOLoop}]{\sphinxcrossref{\sphinxcode{\sphinxupquote{IOLoop}}}}} implementations (such as \sphinxcode{\sphinxupquote{pyzmq}}) accept objects
other than integer file descriptors; these objects will now have
their \sphinxcode{\sphinxupquote{.close()}} method called when the \sphinxcode{\sphinxupquote{IOLoop{}` is closed with
{}`{}`all\_fds=True}}.

\item {} 
The stub handles left behind by {\hyperref[\detokenize{ioloop:tornado.ioloop.IOLoop.remove_timeout}]{\sphinxcrossref{\sphinxcode{\sphinxupquote{IOLoop.remove\_timeout}}}}} will now get
cleaned up instead of waiting to expire.

\end{itemize}


\paragraph{\sphinxstyleliteralintitle{\sphinxupquote{tornado.iostream}}}
\label{\detokenize{releases/v3.1.0:tornado-iostream}}\begin{itemize}
\item {} 
Fixed a bug in {\hyperref[\detokenize{iostream:tornado.iostream.BaseIOStream.read_until_close}]{\sphinxcrossref{\sphinxcode{\sphinxupquote{BaseIOStream.read\_until\_close}}}}} that would sometimes
cause data to be passed to the final callback instead of the streaming
callback.

\item {} 
The {\hyperref[\detokenize{iostream:tornado.iostream.IOStream}]{\sphinxcrossref{\sphinxcode{\sphinxupquote{IOStream}}}}} close callback is now run more reliably if there is
an exception in \sphinxcode{\sphinxupquote{\_try\_inline\_read}}.

\item {} 
New method {\hyperref[\detokenize{iostream:tornado.iostream.BaseIOStream.set_nodelay}]{\sphinxcrossref{\sphinxcode{\sphinxupquote{BaseIOStream.set\_nodelay}}}}} can be used to set the
\sphinxcode{\sphinxupquote{TCP\_NODELAY}} flag.

\item {} 
Fixed a case where errors in \sphinxcode{\sphinxupquote{SSLIOStream.connect}} (and
\sphinxcode{\sphinxupquote{SimpleAsyncHTTPClient}}) were not being reported correctly.

\end{itemize}


\paragraph{\sphinxstyleliteralintitle{\sphinxupquote{tornado.locale}}}
\label{\detokenize{releases/v3.1.0:tornado-locale}}\begin{itemize}
\item {} 
{\hyperref[\detokenize{locale:tornado.locale.Locale.format_date}]{\sphinxcrossref{\sphinxcode{\sphinxupquote{Locale.format\_date}}}}} now works on Python 3.

\end{itemize}


\paragraph{\sphinxstyleliteralintitle{\sphinxupquote{tornado.netutil}}}
\label{\detokenize{releases/v3.1.0:tornado-netutil}}\begin{itemize}
\item {} 
The default {\hyperref[\detokenize{netutil:tornado.netutil.Resolver}]{\sphinxcrossref{\sphinxcode{\sphinxupquote{Resolver}}}}} implementation now works on Solaris.

\item {} 
{\hyperref[\detokenize{netutil:tornado.netutil.Resolver}]{\sphinxcrossref{\sphinxcode{\sphinxupquote{Resolver}}}}} now has a {\hyperref[\detokenize{netutil:tornado.netutil.Resolver.close}]{\sphinxcrossref{\sphinxcode{\sphinxupquote{close}}}}} method.

\item {} 
Fixed a potential CPU DoS when \sphinxcode{\sphinxupquote{tornado.netutil.ssl\_match\_hostname}}
is used on certificates with an abusive wildcard pattern.

\item {} 
All instances of {\hyperref[\detokenize{netutil:tornado.netutil.ThreadedResolver}]{\sphinxcrossref{\sphinxcode{\sphinxupquote{ThreadedResolver}}}}} now share a single thread pool,
whose size is set by the first one to be created (or the static
\sphinxcode{\sphinxupquote{Resolver.configure}} method).

\item {} 
{\hyperref[\detokenize{netutil:tornado.netutil.ExecutorResolver}]{\sphinxcrossref{\sphinxcode{\sphinxupquote{ExecutorResolver}}}}} is now documented for public use.

\item {} 
{\hyperref[\detokenize{netutil:tornado.netutil.bind_sockets}]{\sphinxcrossref{\sphinxcode{\sphinxupquote{bind\_sockets}}}}} now works in configurations with incomplete IPv6 support.

\end{itemize}


\paragraph{\sphinxstyleliteralintitle{\sphinxupquote{tornado.options}}}
\label{\detokenize{releases/v3.1.0:tornado-options}}\begin{itemize}
\item {} 
{\hyperref[\detokenize{options:tornado.options.define}]{\sphinxcrossref{\sphinxcode{\sphinxupquote{tornado.options.define}}}}} with \sphinxcode{\sphinxupquote{multiple=True}} now works on Python 3.

\item {} 
{\hyperref[\detokenize{options:tornado.options.options}]{\sphinxcrossref{\sphinxcode{\sphinxupquote{tornado.options.options}}}}} and other {\hyperref[\detokenize{options:tornado.options.OptionParser}]{\sphinxcrossref{\sphinxcode{\sphinxupquote{OptionParser}}}}} instances support some
new dict-like methods: {\hyperref[\detokenize{options:tornado.options.OptionParser.items}]{\sphinxcrossref{\sphinxcode{\sphinxupquote{items()}}}}}, iteration over keys,
and (read-only) access to options with square braket syntax.
{\hyperref[\detokenize{options:tornado.options.OptionParser.group_dict}]{\sphinxcrossref{\sphinxcode{\sphinxupquote{OptionParser.group\_dict}}}}} returns all options with a given group
name, and {\hyperref[\detokenize{options:tornado.options.OptionParser.as_dict}]{\sphinxcrossref{\sphinxcode{\sphinxupquote{OptionParser.as\_dict}}}}} returns all options.

\end{itemize}


\paragraph{\sphinxstyleliteralintitle{\sphinxupquote{tornado.process}}}
\label{\detokenize{releases/v3.1.0:tornado-process}}\begin{itemize}
\item {} 
{\hyperref[\detokenize{process:tornado.process.Subprocess}]{\sphinxcrossref{\sphinxcode{\sphinxupquote{tornado.process.Subprocess}}}}} no longer leaks file descriptors into
the child process, which fixes a problem in which the child could not
detect that the parent process had closed its stdin pipe.

\item {} 
{\hyperref[\detokenize{process:tornado.process.Subprocess.set_exit_callback}]{\sphinxcrossref{\sphinxcode{\sphinxupquote{Subprocess.set\_exit\_callback}}}}} now works for subprocesses created
without an explicit \sphinxcode{\sphinxupquote{io\_loop}} parameter.

\end{itemize}


\paragraph{\sphinxstyleliteralintitle{\sphinxupquote{tornado.stack\_context}}}
\label{\detokenize{releases/v3.1.0:tornado-stack-context}}\begin{itemize}
\item {} 
\sphinxcode{\sphinxupquote{tornado.stack\_context}} has been rewritten and is now much faster.

\item {} 
New function \sphinxcode{\sphinxupquote{run\_with\_stack\_context}} facilitates the use of stack
contexts with coroutines.

\end{itemize}


\paragraph{\sphinxstyleliteralintitle{\sphinxupquote{tornado.tcpserver}}}
\label{\detokenize{releases/v3.1.0:tornado-tcpserver}}\begin{itemize}
\item {} 
The constructors of {\hyperref[\detokenize{tcpserver:tornado.tcpserver.TCPServer}]{\sphinxcrossref{\sphinxcode{\sphinxupquote{TCPServer}}}}} and {\hyperref[\detokenize{httpserver:tornado.httpserver.HTTPServer}]{\sphinxcrossref{\sphinxcode{\sphinxupquote{HTTPServer}}}}} now take a
\sphinxcode{\sphinxupquote{max\_buffer\_size}} keyword argument.

\end{itemize}


\paragraph{\sphinxstyleliteralintitle{\sphinxupquote{tornado.template}}}
\label{\detokenize{releases/v3.1.0:tornado-template}}\begin{itemize}
\item {} 
Some internal names used by the template system have been changed;
now all “reserved” names in templates start with \sphinxcode{\sphinxupquote{\_tt\_}}.

\end{itemize}


\paragraph{\sphinxstyleliteralintitle{\sphinxupquote{tornado.testing}}}
\label{\detokenize{releases/v3.1.0:tornado-testing}}\begin{itemize}
\item {} 
{\hyperref[\detokenize{testing:tornado.testing.AsyncTestCase.wait}]{\sphinxcrossref{\sphinxcode{\sphinxupquote{tornado.testing.AsyncTestCase.wait}}}}} now raises the correct exception
when it has been modified by \sphinxcode{\sphinxupquote{tornado.stack\_context}}.

\item {} 
{\hyperref[\detokenize{testing:tornado.testing.gen_test}]{\sphinxcrossref{\sphinxcode{\sphinxupquote{tornado.testing.gen\_test}}}}} can now be called as \sphinxcode{\sphinxupquote{@gen\_test(timeout=60)}}
to give some tests a longer timeout than others.

\item {} 
The environment variable \sphinxcode{\sphinxupquote{ASYNC\_TEST\_TIMEOUT}} can now be set to
override the default timeout for {\hyperref[\detokenize{testing:tornado.testing.AsyncTestCase.wait}]{\sphinxcrossref{\sphinxcode{\sphinxupquote{AsyncTestCase.wait}}}}} and {\hyperref[\detokenize{testing:tornado.testing.gen_test}]{\sphinxcrossref{\sphinxcode{\sphinxupquote{gen\_test}}}}}.

\item {} 
{\hyperref[\detokenize{testing:tornado.testing.bind_unused_port}]{\sphinxcrossref{\sphinxcode{\sphinxupquote{bind\_unused\_port}}}}} now passes \sphinxcode{\sphinxupquote{None}} instead of \sphinxcode{\sphinxupquote{0}} as the port
to \sphinxcode{\sphinxupquote{getaddrinfo}}, which works better with some unusual network
configurations.

\end{itemize}


\paragraph{\sphinxstyleliteralintitle{\sphinxupquote{tornado.util}}}
\label{\detokenize{releases/v3.1.0:tornado-util}}\begin{itemize}
\item {} 
{\hyperref[\detokenize{util:tornado.util.import_object}]{\sphinxcrossref{\sphinxcode{\sphinxupquote{tornado.util.import\_object}}}}} now works with top-level module names that
do not contain a dot.

\item {} 
{\hyperref[\detokenize{util:tornado.util.import_object}]{\sphinxcrossref{\sphinxcode{\sphinxupquote{tornado.util.import\_object}}}}} now consistently raises \sphinxhref{https://docs.python.org/3.6/library/exceptions.html\#ImportError}{\sphinxcode{\sphinxupquote{ImportError}}}
instead of \sphinxhref{https://docs.python.org/3.6/library/exceptions.html\#AttributeError}{\sphinxcode{\sphinxupquote{AttributeError}}} when it fails.

\end{itemize}


\paragraph{\sphinxstyleliteralintitle{\sphinxupquote{tornado.web}}}
\label{\detokenize{releases/v3.1.0:tornado-web}}\begin{itemize}
\item {} 
The \sphinxcode{\sphinxupquote{handlers}} list passed to the {\hyperref[\detokenize{web:tornado.web.Application}]{\sphinxcrossref{\sphinxcode{\sphinxupquote{tornado.web.Application}}}}} constructor
and {\hyperref[\detokenize{web:tornado.web.Application.add_handlers}]{\sphinxcrossref{\sphinxcode{\sphinxupquote{add\_handlers}}}}} methods can now contain
lists in addition to tuples and {\hyperref[\detokenize{web:tornado.web.URLSpec}]{\sphinxcrossref{\sphinxcode{\sphinxupquote{URLSpec}}}}} objects.

\item {} 
{\hyperref[\detokenize{web:tornado.web.StaticFileHandler}]{\sphinxcrossref{\sphinxcode{\sphinxupquote{tornado.web.StaticFileHandler}}}}} now works on Windows when the client
passes an \sphinxcode{\sphinxupquote{If-Modified-Since}} timestamp before 1970.

\item {} 
New method {\hyperref[\detokenize{web:tornado.web.RequestHandler.log_exception}]{\sphinxcrossref{\sphinxcode{\sphinxupquote{RequestHandler.log\_exception}}}}} can be overridden to
customize the logging behavior when an exception is uncaught.  Most
apps that currently override \sphinxcode{\sphinxupquote{\_handle\_request\_exception}} can now
use a combination of {\hyperref[\detokenize{web:tornado.web.RequestHandler.log_exception}]{\sphinxcrossref{\sphinxcode{\sphinxupquote{RequestHandler.log\_exception}}}}} and
{\hyperref[\detokenize{web:tornado.web.RequestHandler.write_error}]{\sphinxcrossref{\sphinxcode{\sphinxupquote{write\_error}}}}}.

\item {} 
{\hyperref[\detokenize{web:tornado.web.RequestHandler.get_argument}]{\sphinxcrossref{\sphinxcode{\sphinxupquote{RequestHandler.get\_argument}}}}} now raises {\hyperref[\detokenize{web:tornado.web.MissingArgumentError}]{\sphinxcrossref{\sphinxcode{\sphinxupquote{MissingArgumentError}}}}}
(a subclass of {\hyperref[\detokenize{web:tornado.web.HTTPError}]{\sphinxcrossref{\sphinxcode{\sphinxupquote{tornado.web.HTTPError}}}}}, which is what it raised previously)
if the argument cannot be found.

\item {} 
{\hyperref[\detokenize{web:tornado.web.Application.reverse_url}]{\sphinxcrossref{\sphinxcode{\sphinxupquote{Application.reverse\_url}}}}} now uses {\hyperref[\detokenize{escape:tornado.escape.url_escape}]{\sphinxcrossref{\sphinxcode{\sphinxupquote{url\_escape}}}}} with \sphinxcode{\sphinxupquote{plus=False}},
i.e. spaces are encoded as \sphinxcode{\sphinxupquote{\%20}} instead of \sphinxcode{\sphinxupquote{+}}.

\item {} 
Arguments extracted from the url path are now decoded with
{\hyperref[\detokenize{escape:tornado.escape.url_unescape}]{\sphinxcrossref{\sphinxcode{\sphinxupquote{url\_unescape}}}}} with \sphinxcode{\sphinxupquote{plus=False}}, so plus signs are left as-is
instead of being turned into spaces.

\item {} 
{\hyperref[\detokenize{web:tornado.web.RequestHandler.send_error}]{\sphinxcrossref{\sphinxcode{\sphinxupquote{RequestHandler.send\_error}}}}} will now only be called once per request,
even if multiple exceptions are caught by the stack context.

\item {} 
The \sphinxcode{\sphinxupquote{tornado.web.asynchronous}} decorator is no longer necessary for
methods that return a {\hyperref[\detokenize{concurrent:tornado.concurrent.Future}]{\sphinxcrossref{\sphinxcode{\sphinxupquote{Future}}}}} (i.e. those that use the {\hyperref[\detokenize{gen:tornado.gen.coroutine}]{\sphinxcrossref{\sphinxcode{\sphinxupquote{gen.coroutine}}}}}
or \sphinxcode{\sphinxupquote{return\_future}} decorators)

\item {} 
{\hyperref[\detokenize{web:tornado.web.RequestHandler.prepare}]{\sphinxcrossref{\sphinxcode{\sphinxupquote{RequestHandler.prepare}}}}} may now be asynchronous if it returns a
{\hyperref[\detokenize{concurrent:tornado.concurrent.Future}]{\sphinxcrossref{\sphinxcode{\sphinxupquote{Future}}}}}.  The \sphinxcode{\sphinxupquote{tornado.web.asynchronous}} decorator is not used with
\sphinxcode{\sphinxupquote{prepare}}; one of the {\hyperref[\detokenize{concurrent:tornado.concurrent.Future}]{\sphinxcrossref{\sphinxcode{\sphinxupquote{Future}}}}}-related decorators should be used instead.

\item {} 
\sphinxcode{\sphinxupquote{RequestHandler.current\_user}} may now be assigned to normally.

\item {} 
{\hyperref[\detokenize{web:tornado.web.RequestHandler.redirect}]{\sphinxcrossref{\sphinxcode{\sphinxupquote{RequestHandler.redirect}}}}} no longer silently strips control characters
and whitespace.  It is now an error to pass control characters, newlines
or tabs.

\item {} 
{\hyperref[\detokenize{web:tornado.web.StaticFileHandler}]{\sphinxcrossref{\sphinxcode{\sphinxupquote{StaticFileHandler}}}}} has been reorganized internally and now has additional
extension points that can be overridden in subclasses.

\item {} 
{\hyperref[\detokenize{web:tornado.web.StaticFileHandler}]{\sphinxcrossref{\sphinxcode{\sphinxupquote{StaticFileHandler}}}}} now supports HTTP \sphinxcode{\sphinxupquote{Range}} requests.
{\hyperref[\detokenize{web:tornado.web.StaticFileHandler}]{\sphinxcrossref{\sphinxcode{\sphinxupquote{StaticFileHandler}}}}} is still not suitable for files too large to
comfortably fit in memory, but \sphinxcode{\sphinxupquote{Range}} support is necessary in some
browsers to enable seeking of HTML5 audio and video.

\item {} 
{\hyperref[\detokenize{web:tornado.web.StaticFileHandler}]{\sphinxcrossref{\sphinxcode{\sphinxupquote{StaticFileHandler}}}}} now uses longer hashes by default, and uses the same
hashes for \sphinxcode{\sphinxupquote{Etag}} as it does for versioned urls.

\item {} 
{\hyperref[\detokenize{web:tornado.web.StaticFileHandler.make_static_url}]{\sphinxcrossref{\sphinxcode{\sphinxupquote{StaticFileHandler.make\_static\_url}}}}} and {\hyperref[\detokenize{web:tornado.web.RequestHandler.static_url}]{\sphinxcrossref{\sphinxcode{\sphinxupquote{RequestHandler.static\_url}}}}}
now have an additional keyword argument \sphinxcode{\sphinxupquote{include\_version}} to suppress
the url versioning.

\item {} 
{\hyperref[\detokenize{web:tornado.web.StaticFileHandler}]{\sphinxcrossref{\sphinxcode{\sphinxupquote{StaticFileHandler}}}}} now reads its file in chunks, which will reduce
memory fragmentation.

\item {} 
Fixed a problem with the \sphinxcode{\sphinxupquote{Date}} header and cookie expiration dates
when the system locale is set to a non-english configuration.

\end{itemize}


\paragraph{\sphinxstyleliteralintitle{\sphinxupquote{tornado.websocket}}}
\label{\detokenize{releases/v3.1.0:tornado-websocket}}\begin{itemize}
\item {} 
{\hyperref[\detokenize{websocket:tornado.websocket.WebSocketHandler}]{\sphinxcrossref{\sphinxcode{\sphinxupquote{WebSocketHandler}}}}} now catches {\hyperref[\detokenize{iostream:tornado.iostream.StreamClosedError}]{\sphinxcrossref{\sphinxcode{\sphinxupquote{StreamClosedError}}}}} and runs
{\hyperref[\detokenize{websocket:tornado.websocket.WebSocketHandler.on_close}]{\sphinxcrossref{\sphinxcode{\sphinxupquote{on\_close}}}}} immediately instead of logging a
stack trace.

\item {} 
New method {\hyperref[\detokenize{websocket:tornado.websocket.WebSocketHandler.set_nodelay}]{\sphinxcrossref{\sphinxcode{\sphinxupquote{WebSocketHandler.set\_nodelay}}}}} can be used to set the
\sphinxcode{\sphinxupquote{TCP\_NODELAY}} flag.

\end{itemize}


\paragraph{\sphinxstyleliteralintitle{\sphinxupquote{tornado.wsgi}}}
\label{\detokenize{releases/v3.1.0:tornado-wsgi}}\begin{itemize}
\item {} 
Fixed an exception in {\hyperref[\detokenize{wsgi:tornado.wsgi.WSGIContainer}]{\sphinxcrossref{\sphinxcode{\sphinxupquote{WSGIContainer}}}}} when the connection is closed
while output is being written.

\end{itemize}


\subsection{What’s new in Tornado 3.0.2}
\label{\detokenize{releases/v3.0.2:what-s-new-in-tornado-3-0-2}}\label{\detokenize{releases/v3.0.2::doc}}

\subsubsection{Jun 2, 2013}
\label{\detokenize{releases/v3.0.2:jun-2-2013}}\begin{itemize}
\item {} 
{\hyperref[\detokenize{auth:tornado.auth.TwitterMixin}]{\sphinxcrossref{\sphinxcode{\sphinxupquote{tornado.auth.TwitterMixin}}}}} now defaults to version 1.1 of the
Twitter API, instead of version 1.0 which is being \sphinxhref{https://dev.twitter.com/calendar}{discontinued on
June 11}.  It also now uses HTTPS
when talking to Twitter.

\item {} 
Fixed a potential memory leak with a long chain of {\hyperref[\detokenize{gen:tornado.gen.coroutine}]{\sphinxcrossref{\sphinxcode{\sphinxupquote{gen.coroutine}}}}}
or \sphinxcode{\sphinxupquote{gen.engine}} functions.

\end{itemize}


\subsection{What’s new in Tornado 3.0.1}
\label{\detokenize{releases/v3.0.1:what-s-new-in-tornado-3-0-1}}\label{\detokenize{releases/v3.0.1::doc}}

\subsubsection{Apr 8, 2013}
\label{\detokenize{releases/v3.0.1:apr-8-2013}}\begin{itemize}
\item {} 
The interface of {\hyperref[\detokenize{auth:tornado.auth.FacebookGraphMixin}]{\sphinxcrossref{\sphinxcode{\sphinxupquote{tornado.auth.FacebookGraphMixin}}}}} is now consistent
with its documentation and the rest of the module.  The
\sphinxcode{\sphinxupquote{get\_authenticated\_user}} and \sphinxcode{\sphinxupquote{facebook\_request}} methods return a
\sphinxcode{\sphinxupquote{Future}} and the \sphinxcode{\sphinxupquote{callback}} argument is optional.

\item {} 
The {\hyperref[\detokenize{testing:tornado.testing.gen_test}]{\sphinxcrossref{\sphinxcode{\sphinxupquote{tornado.testing.gen\_test}}}}} decorator will no longer be recognized
as a (broken) test by \sphinxcode{\sphinxupquote{nose}}.

\item {} 
Work around a bug in Ubuntu 13.04 betas involving an incomplete backport
of the \sphinxhref{https://docs.python.org/3.6/library/ssl.html\#ssl.match\_hostname}{\sphinxcode{\sphinxupquote{ssl.match\_hostname}}} function.

\item {} 
{\hyperref[\detokenize{websocket:tornado.websocket.websocket_connect}]{\sphinxcrossref{\sphinxcode{\sphinxupquote{tornado.websocket.websocket\_connect}}}}} now fails cleanly when it attempts
to connect to a non-websocket url.

\item {} 
\sphinxcode{\sphinxupquote{tornado.testing.LogTrapTestCase}} once again works with byte strings
on Python 2.

\item {} 
The \sphinxcode{\sphinxupquote{request}} attribute of {\hyperref[\detokenize{httpclient:tornado.httpclient.HTTPResponse}]{\sphinxcrossref{\sphinxcode{\sphinxupquote{tornado.httpclient.HTTPResponse}}}}} is
now always an {\hyperref[\detokenize{httpclient:tornado.httpclient.HTTPRequest}]{\sphinxcrossref{\sphinxcode{\sphinxupquote{HTTPRequest}}}}}, never a \sphinxcode{\sphinxupquote{\_RequestProxy}}.

\item {} 
Exceptions raised by the {\hyperref[\detokenize{gen:module-tornado.gen}]{\sphinxcrossref{\sphinxcode{\sphinxupquote{tornado.gen}}}}} module now have better messages
when tuples are used as callback keys.

\end{itemize}


\subsection{What’s new in Tornado 3.0}
\label{\detokenize{releases/v3.0.0:what-s-new-in-tornado-3-0}}\label{\detokenize{releases/v3.0.0::doc}}

\subsubsection{Mar 29, 2013}
\label{\detokenize{releases/v3.0.0:mar-29-2013}}

\paragraph{Highlights}
\label{\detokenize{releases/v3.0.0:highlights}}\begin{itemize}
\item {} 
The \sphinxcode{\sphinxupquote{callback}} argument to many asynchronous methods is now
optional, and these methods return a {\hyperref[\detokenize{concurrent:tornado.concurrent.Future}]{\sphinxcrossref{\sphinxcode{\sphinxupquote{Future}}}}}.  The {\hyperref[\detokenize{gen:module-tornado.gen}]{\sphinxcrossref{\sphinxcode{\sphinxupquote{tornado.gen}}}}}
module now understands \sphinxcode{\sphinxupquote{Futures}}, and these methods can be used
directly without a \sphinxcode{\sphinxupquote{gen.Task}} wrapper.

\item {} 
New function {\hyperref[\detokenize{ioloop:tornado.ioloop.IOLoop.current}]{\sphinxcrossref{\sphinxcode{\sphinxupquote{IOLoop.current}}}}} returns the {\hyperref[\detokenize{ioloop:tornado.ioloop.IOLoop}]{\sphinxcrossref{\sphinxcode{\sphinxupquote{IOLoop}}}}} that is running
on the current thread (as opposed to {\hyperref[\detokenize{ioloop:tornado.ioloop.IOLoop.instance}]{\sphinxcrossref{\sphinxcode{\sphinxupquote{IOLoop.instance}}}}}, which
returns a specific thread’s (usually the main thread’s) IOLoop.

\item {} 
New class {\hyperref[\detokenize{netutil:tornado.netutil.Resolver}]{\sphinxcrossref{\sphinxcode{\sphinxupquote{tornado.netutil.Resolver}}}}} provides an asynchronous
interface to DNS resolution.  The default implementation is still
blocking, but non-blocking implementations are available using one
of three optional dependencies: {\hyperref[\detokenize{netutil:tornado.netutil.ThreadedResolver}]{\sphinxcrossref{\sphinxcode{\sphinxupquote{ThreadedResolver}}}}}
using the \sphinxhref{https://docs.python.org/3.6/library/concurrent.futures.html\#module-concurrent.futures}{\sphinxcode{\sphinxupquote{concurrent.futures}}} thread pool,
\sphinxcode{\sphinxupquote{tornado.platform.caresresolver.CaresResolver}} using the \sphinxcode{\sphinxupquote{pycares}}
library, or \sphinxcode{\sphinxupquote{tornado.platform.twisted.TwistedResolver}} using \sphinxcode{\sphinxupquote{twisted}}

\item {} 
Tornado’s logging is now less noisy, and it no longer goes directly
to the root logger, allowing for finer-grained configuration.

\item {} 
New class {\hyperref[\detokenize{process:tornado.process.Subprocess}]{\sphinxcrossref{\sphinxcode{\sphinxupquote{tornado.process.Subprocess}}}}} wraps \sphinxhref{https://docs.python.org/3.6/library/subprocess.html\#subprocess.Popen}{\sphinxcode{\sphinxupquote{subprocess.Popen}}} with
{\hyperref[\detokenize{iostream:tornado.iostream.PipeIOStream}]{\sphinxcrossref{\sphinxcode{\sphinxupquote{PipeIOStream}}}}} access to the child’s file descriptors.

\item {} 
{\hyperref[\detokenize{ioloop:tornado.ioloop.IOLoop}]{\sphinxcrossref{\sphinxcode{\sphinxupquote{IOLoop}}}}} now has a static {\hyperref[\detokenize{util:tornado.util.Configurable.configure}]{\sphinxcrossref{\sphinxcode{\sphinxupquote{configure}}}}}
method like the one on {\hyperref[\detokenize{httpclient:tornado.httpclient.AsyncHTTPClient}]{\sphinxcrossref{\sphinxcode{\sphinxupquote{AsyncHTTPClient}}}}}, which can be used to
select an {\hyperref[\detokenize{ioloop:tornado.ioloop.IOLoop}]{\sphinxcrossref{\sphinxcode{\sphinxupquote{IOLoop}}}}} implementation other than the default.

\item {} 
{\hyperref[\detokenize{ioloop:tornado.ioloop.IOLoop}]{\sphinxcrossref{\sphinxcode{\sphinxupquote{IOLoop}}}}} can now optionally use a monotonic clock if available
(see below for more details).

\end{itemize}


\paragraph{Backwards-incompatible changes}
\label{\detokenize{releases/v3.0.0:backwards-incompatible-changes}}\begin{itemize}
\item {} 
Python 2.5 is no longer supported.  Python 3 is now supported in a single
codebase instead of using \sphinxcode{\sphinxupquote{2to3}}

\item {} 
The \sphinxcode{\sphinxupquote{tornado.database}} module has been removed.  It is now available
as a separate package, \sphinxhref{https://github.com/bdarnell/torndb}{torndb}

\item {} 
Functions that take an \sphinxcode{\sphinxupquote{io\_loop}} parameter now default to
{\hyperref[\detokenize{ioloop:tornado.ioloop.IOLoop.current}]{\sphinxcrossref{\sphinxcode{\sphinxupquote{IOLoop.current()}}}}} instead of {\hyperref[\detokenize{ioloop:tornado.ioloop.IOLoop.instance}]{\sphinxcrossref{\sphinxcode{\sphinxupquote{IOLoop.instance()}}}}}.

\item {} 
Empty HTTP request arguments are no longer ignored.  This applies to
\sphinxcode{\sphinxupquote{HTTPRequest.arguments}} and \sphinxcode{\sphinxupquote{RequestHandler.get\_argument{[}s{]}}}
in WSGI and non-WSGI modes.

\item {} 
On Python 3, {\hyperref[\detokenize{escape:tornado.escape.json_encode}]{\sphinxcrossref{\sphinxcode{\sphinxupquote{tornado.escape.json\_encode}}}}} no longer accepts byte strings.

\item {} 
On Python 3, the \sphinxcode{\sphinxupquote{get\_authenticated\_user}} methods in {\hyperref[\detokenize{auth:module-tornado.auth}]{\sphinxcrossref{\sphinxcode{\sphinxupquote{tornado.auth}}}}}
now return character strings instead of byte strings.

\item {} 
\sphinxcode{\sphinxupquote{tornado.netutil.TCPServer}} has moved to its own module,
{\hyperref[\detokenize{tcpserver:module-tornado.tcpserver}]{\sphinxcrossref{\sphinxcode{\sphinxupquote{tornado.tcpserver}}}}}.

\item {} 
The Tornado test suite now requires \sphinxcode{\sphinxupquote{unittest2}} when run on Python 2.6.

\item {} 
{\hyperref[\detokenize{options:tornado.options.options}]{\sphinxcrossref{\sphinxcode{\sphinxupquote{tornado.options.options}}}}} is no longer a subclass of \sphinxhref{https://docs.python.org/3.6/library/stdtypes.html\#dict}{\sphinxcode{\sphinxupquote{dict}}}; attribute-style
access is now required.

\end{itemize}


\paragraph{Detailed changes by module}
\label{\detokenize{releases/v3.0.0:detailed-changes-by-module}}

\subparagraph{Multiple modules}
\label{\detokenize{releases/v3.0.0:multiple-modules}}\begin{itemize}
\item {} 
Tornado no longer logs to the root logger.  Details on the new logging
scheme can be found under the {\hyperref[\detokenize{log:module-tornado.log}]{\sphinxcrossref{\sphinxcode{\sphinxupquote{tornado.log}}}}} module.  Note that in some
cases this will require that you add an explicit logging configuration
in order to see any output (perhaps just calling \sphinxcode{\sphinxupquote{logging.basicConfig()}}),
although both {\hyperref[\detokenize{ioloop:tornado.ioloop.IOLoop.start}]{\sphinxcrossref{\sphinxcode{\sphinxupquote{IOLoop.start()}}}}} and {\hyperref[\detokenize{options:tornado.options.parse_command_line}]{\sphinxcrossref{\sphinxcode{\sphinxupquote{tornado.options.parse\_command\_line}}}}}
will do this for you.

\item {} 
On python 3.2+, methods that take an \sphinxcode{\sphinxupquote{ssl\_options}} argument (on
{\hyperref[\detokenize{iostream:tornado.iostream.SSLIOStream}]{\sphinxcrossref{\sphinxcode{\sphinxupquote{SSLIOStream}}}}}, {\hyperref[\detokenize{tcpserver:tornado.tcpserver.TCPServer}]{\sphinxcrossref{\sphinxcode{\sphinxupquote{TCPServer}}}}}, and {\hyperref[\detokenize{httpserver:tornado.httpserver.HTTPServer}]{\sphinxcrossref{\sphinxcode{\sphinxupquote{HTTPServer}}}}}) now accept either a
dictionary of options or an \sphinxhref{https://docs.python.org/3.6/library/ssl.html\#ssl.SSLContext}{\sphinxcode{\sphinxupquote{ssl.SSLContext}}} object.

\item {} 
New optional dependency on \sphinxhref{https://docs.python.org/3.6/library/concurrent.futures.html\#module-concurrent.futures}{\sphinxcode{\sphinxupquote{concurrent.futures}}} to provide better support
for working with threads.  \sphinxhref{https://docs.python.org/3.6/library/concurrent.futures.html\#module-concurrent.futures}{\sphinxcode{\sphinxupquote{concurrent.futures}}} is in the standard library
for Python 3.2+, and can be installed on older versions with
\sphinxcode{\sphinxupquote{pip install futures}}.

\end{itemize}


\subparagraph{\sphinxstyleliteralintitle{\sphinxupquote{tornado.autoreload}}}
\label{\detokenize{releases/v3.0.0:tornado-autoreload}}\begin{itemize}
\item {} 
{\hyperref[\detokenize{autoreload:module-tornado.autoreload}]{\sphinxcrossref{\sphinxcode{\sphinxupquote{tornado.autoreload}}}}} is now more reliable when there are errors at import
time.

\item {} 
Calling {\hyperref[\detokenize{autoreload:tornado.autoreload.start}]{\sphinxcrossref{\sphinxcode{\sphinxupquote{tornado.autoreload.start}}}}} (or creating an {\hyperref[\detokenize{web:tornado.web.Application}]{\sphinxcrossref{\sphinxcode{\sphinxupquote{Application}}}}} with
\sphinxcode{\sphinxupquote{debug=True}}) twice on the same {\hyperref[\detokenize{ioloop:tornado.ioloop.IOLoop}]{\sphinxcrossref{\sphinxcode{\sphinxupquote{IOLoop}}}}} now does nothing (instead of
creating multiple periodic callbacks).  Starting autoreload on
more than one {\hyperref[\detokenize{ioloop:tornado.ioloop.IOLoop}]{\sphinxcrossref{\sphinxcode{\sphinxupquote{IOLoop}}}}} in the same process now logs a warning.

\item {} 
Scripts run by autoreload no longer inherit \sphinxcode{\sphinxupquote{\_\_future\_\_}} imports
used by Tornado.

\end{itemize}


\subparagraph{\sphinxstyleliteralintitle{\sphinxupquote{tornado.auth}}}
\label{\detokenize{releases/v3.0.0:tornado-auth}}\begin{itemize}
\item {} 
On Python 3, the \sphinxcode{\sphinxupquote{get\_authenticated\_user}} method family now returns
character strings instead of byte strings.

\item {} 
Asynchronous methods defined in {\hyperref[\detokenize{auth:module-tornado.auth}]{\sphinxcrossref{\sphinxcode{\sphinxupquote{tornado.auth}}}}} now return a
{\hyperref[\detokenize{concurrent:tornado.concurrent.Future}]{\sphinxcrossref{\sphinxcode{\sphinxupquote{Future}}}}}, and their \sphinxcode{\sphinxupquote{callback}} argument is optional.  The
\sphinxcode{\sphinxupquote{Future}} interface is preferred as it offers better error handling
(the previous interface just logged a warning and returned None).

\item {} 
The {\hyperref[\detokenize{auth:module-tornado.auth}]{\sphinxcrossref{\sphinxcode{\sphinxupquote{tornado.auth}}}}} mixin classes now define a method
\sphinxcode{\sphinxupquote{get\_auth\_http\_client}}, which can be overridden to use a non-default
{\hyperref[\detokenize{httpclient:tornado.httpclient.AsyncHTTPClient}]{\sphinxcrossref{\sphinxcode{\sphinxupquote{AsyncHTTPClient}}}}} instance (e.g. to use a different {\hyperref[\detokenize{ioloop:tornado.ioloop.IOLoop}]{\sphinxcrossref{\sphinxcode{\sphinxupquote{IOLoop}}}}})

\item {} 
Subclasses of {\hyperref[\detokenize{auth:tornado.auth.OAuthMixin}]{\sphinxcrossref{\sphinxcode{\sphinxupquote{OAuthMixin}}}}} are encouraged to override
{\hyperref[\detokenize{auth:tornado.auth.OAuthMixin._oauth_get_user_future}]{\sphinxcrossref{\sphinxcode{\sphinxupquote{OAuthMixin.\_oauth\_get\_user\_future}}}}} instead of \sphinxcode{\sphinxupquote{\_oauth\_get\_user}},
although both methods are still supported.

\end{itemize}


\subparagraph{\sphinxstyleliteralintitle{\sphinxupquote{tornado.concurrent}}}
\label{\detokenize{releases/v3.0.0:tornado-concurrent}}\begin{itemize}
\item {} 
New module {\hyperref[\detokenize{concurrent:module-tornado.concurrent}]{\sphinxcrossref{\sphinxcode{\sphinxupquote{tornado.concurrent}}}}} contains code to support working with
\sphinxhref{https://docs.python.org/3.6/library/concurrent.futures.html\#module-concurrent.futures}{\sphinxcode{\sphinxupquote{concurrent.futures}}}, or to emulate future-based interface when that module
is not available.

\end{itemize}


\subparagraph{\sphinxstyleliteralintitle{\sphinxupquote{tornado.curl\_httpclient}}}
\label{\detokenize{releases/v3.0.0:tornado-curl-httpclient}}\begin{itemize}
\item {} 
Preliminary support for \sphinxcode{\sphinxupquote{tornado.curl\_httpclient}} on Python 3.  The latest
official release of pycurl only supports Python 2, but Ubuntu has a
port available in 12.10 (\sphinxcode{\sphinxupquote{apt-get install python3-pycurl}}).  This port
currently has bugs that prevent it from handling arbitrary binary data
but it should work for textual (utf8) resources.

\item {} 
Fix a crash with libcurl 7.29.0 if a curl object is created and closed
without being used.

\end{itemize}


\subparagraph{\sphinxstyleliteralintitle{\sphinxupquote{tornado.escape}}}
\label{\detokenize{releases/v3.0.0:tornado-escape}}\begin{itemize}
\item {} 
On Python 3, {\hyperref[\detokenize{escape:tornado.escape.json_encode}]{\sphinxcrossref{\sphinxcode{\sphinxupquote{json\_encode}}}}} no longer accepts byte strings.
This mirrors the behavior of the underlying json module.  Python 2 behavior
is unchanged but should be faster.

\end{itemize}


\subparagraph{\sphinxstyleliteralintitle{\sphinxupquote{tornado.gen}}}
\label{\detokenize{releases/v3.0.0:tornado-gen}}\begin{itemize}
\item {} 
New decorator \sphinxcode{\sphinxupquote{@gen.coroutine}} is available as an alternative to
\sphinxcode{\sphinxupquote{@gen.engine}}.  It automatically returns a
{\hyperref[\detokenize{concurrent:tornado.concurrent.Future}]{\sphinxcrossref{\sphinxcode{\sphinxupquote{Future}}}}}, and within the function instead of
calling a callback you return a value with \sphinxcode{\sphinxupquote{raise
gen.Return(value)}} (or simply \sphinxcode{\sphinxupquote{return value}} in Python 3.3).

\item {} 
Generators may now yield {\hyperref[\detokenize{concurrent:tornado.concurrent.Future}]{\sphinxcrossref{\sphinxcode{\sphinxupquote{Future}}}}} objects.

\item {} 
Callbacks produced by \sphinxcode{\sphinxupquote{gen.Callback}} and \sphinxcode{\sphinxupquote{gen.Task}} are now automatically
stack-context-wrapped, to minimize the risk of context leaks when used
with asynchronous functions that don’t do their own wrapping.

\item {} 
Fixed a memory leak involving generators, {\hyperref[\detokenize{web:tornado.web.RequestHandler.flush}]{\sphinxcrossref{\sphinxcode{\sphinxupquote{RequestHandler.flush}}}}},
and clients closing connections while output is being written.

\item {} 
Yielding a large list no longer has quadratic performance.

\end{itemize}


\subparagraph{\sphinxstyleliteralintitle{\sphinxupquote{tornado.httpclient}}}
\label{\detokenize{releases/v3.0.0:tornado-httpclient}}\begin{itemize}
\item {} 
{\hyperref[\detokenize{httpclient:tornado.httpclient.AsyncHTTPClient.fetch}]{\sphinxcrossref{\sphinxcode{\sphinxupquote{AsyncHTTPClient.fetch}}}}} now returns a {\hyperref[\detokenize{concurrent:tornado.concurrent.Future}]{\sphinxcrossref{\sphinxcode{\sphinxupquote{Future}}}}} and its callback argument
is optional.  When the future interface is used, any error will be raised
automatically, as if {\hyperref[\detokenize{httpclient:tornado.httpclient.HTTPResponse.rethrow}]{\sphinxcrossref{\sphinxcode{\sphinxupquote{HTTPResponse.rethrow}}}}} was called.

\item {} 
{\hyperref[\detokenize{httpclient:tornado.httpclient.AsyncHTTPClient.configure}]{\sphinxcrossref{\sphinxcode{\sphinxupquote{AsyncHTTPClient.configure}}}}} and all {\hyperref[\detokenize{httpclient:tornado.httpclient.AsyncHTTPClient}]{\sphinxcrossref{\sphinxcode{\sphinxupquote{AsyncHTTPClient}}}}} constructors
now take a \sphinxcode{\sphinxupquote{defaults}} keyword argument.  This argument should be a
dictionary, and its values will be used in place of corresponding
attributes of {\hyperref[\detokenize{httpclient:tornado.httpclient.HTTPRequest}]{\sphinxcrossref{\sphinxcode{\sphinxupquote{HTTPRequest}}}}} that are not set.

\item {} 
All unset attributes of {\hyperref[\detokenize{httpclient:tornado.httpclient.HTTPRequest}]{\sphinxcrossref{\sphinxcode{\sphinxupquote{tornado.httpclient.HTTPRequest}}}}} are now
\sphinxcode{\sphinxupquote{None}}.  The default values of some attributes
(\sphinxcode{\sphinxupquote{connect\_timeout}}, \sphinxcode{\sphinxupquote{request\_timeout}}, \sphinxcode{\sphinxupquote{follow\_redirects}},
\sphinxcode{\sphinxupquote{max\_redirects}}, \sphinxcode{\sphinxupquote{use\_gzip}}, \sphinxcode{\sphinxupquote{proxy\_password}},
\sphinxcode{\sphinxupquote{allow\_nonstandard\_methods}}, and \sphinxcode{\sphinxupquote{validate\_cert}} have been moved
from {\hyperref[\detokenize{httpclient:tornado.httpclient.HTTPRequest}]{\sphinxcrossref{\sphinxcode{\sphinxupquote{HTTPRequest}}}}} to the client
implementations.

\item {} 
The \sphinxcode{\sphinxupquote{max\_clients}} argument to {\hyperref[\detokenize{httpclient:tornado.httpclient.AsyncHTTPClient}]{\sphinxcrossref{\sphinxcode{\sphinxupquote{AsyncHTTPClient}}}}} is now a keyword-only
argument.

\item {} 
Keyword arguments to {\hyperref[\detokenize{httpclient:tornado.httpclient.AsyncHTTPClient.configure}]{\sphinxcrossref{\sphinxcode{\sphinxupquote{AsyncHTTPClient.configure}}}}} are no longer used
when instantiating an implementation subclass directly.

\item {} 
Secondary {\hyperref[\detokenize{httpclient:tornado.httpclient.AsyncHTTPClient}]{\sphinxcrossref{\sphinxcode{\sphinxupquote{AsyncHTTPClient}}}}} callbacks (\sphinxcode{\sphinxupquote{streaming\_callback}},
\sphinxcode{\sphinxupquote{header\_callback}}, and \sphinxcode{\sphinxupquote{prepare\_curl\_callback}}) now respect
\sphinxcode{\sphinxupquote{StackContext}}.

\end{itemize}


\subparagraph{\sphinxstyleliteralintitle{\sphinxupquote{tornado.httpserver}}}
\label{\detokenize{releases/v3.0.0:tornado-httpserver}}\begin{itemize}
\item {} 
{\hyperref[\detokenize{httpserver:tornado.httpserver.HTTPServer}]{\sphinxcrossref{\sphinxcode{\sphinxupquote{HTTPServer}}}}} no longer logs an error when it is unable to read a second
request from an HTTP 1.1 keep-alive connection.

\item {} 
{\hyperref[\detokenize{httpserver:tornado.httpserver.HTTPServer}]{\sphinxcrossref{\sphinxcode{\sphinxupquote{HTTPServer}}}}} now takes a \sphinxcode{\sphinxupquote{protocol}} keyword argument which can be set
to \sphinxcode{\sphinxupquote{https}} if the server is behind an SSL-decoding proxy that does not
set any supported X-headers.

\item {} 
\sphinxcode{\sphinxupquote{tornado.httpserver.HTTPConnection}} now has a \sphinxcode{\sphinxupquote{set\_close\_callback}}
method that should be used instead of reaching into its \sphinxcode{\sphinxupquote{stream}}
attribute.

\item {} 
Empty HTTP request arguments are no longer ignored.  This applies to
\sphinxcode{\sphinxupquote{HTTPRequest.arguments}} and \sphinxcode{\sphinxupquote{RequestHandler.get\_argument{[}s{]}}}
in WSGI and non-WSGI modes.

\end{itemize}


\subparagraph{\sphinxstyleliteralintitle{\sphinxupquote{tornado.ioloop}}}
\label{\detokenize{releases/v3.0.0:tornado-ioloop}}\begin{itemize}
\item {} 
New function {\hyperref[\detokenize{ioloop:tornado.ioloop.IOLoop.current}]{\sphinxcrossref{\sphinxcode{\sphinxupquote{IOLoop.current}}}}} returns the \sphinxcode{\sphinxupquote{IOLoop}} that is running
on the current thread (as opposed to {\hyperref[\detokenize{ioloop:tornado.ioloop.IOLoop.instance}]{\sphinxcrossref{\sphinxcode{\sphinxupquote{IOLoop.instance}}}}}, which returns a
specific thread’s (usually the main thread’s) IOLoop).

\item {} 
New method {\hyperref[\detokenize{ioloop:tornado.ioloop.IOLoop.add_future}]{\sphinxcrossref{\sphinxcode{\sphinxupquote{IOLoop.add\_future}}}}} to run a callback on the IOLoop when
an asynchronous {\hyperref[\detokenize{concurrent:tornado.concurrent.Future}]{\sphinxcrossref{\sphinxcode{\sphinxupquote{Future}}}}} finishes.

\item {} 
{\hyperref[\detokenize{ioloop:tornado.ioloop.IOLoop}]{\sphinxcrossref{\sphinxcode{\sphinxupquote{IOLoop}}}}} now has a static {\hyperref[\detokenize{util:tornado.util.Configurable.configure}]{\sphinxcrossref{\sphinxcode{\sphinxupquote{configure}}}}}
method like the one on {\hyperref[\detokenize{httpclient:tornado.httpclient.AsyncHTTPClient}]{\sphinxcrossref{\sphinxcode{\sphinxupquote{AsyncHTTPClient}}}}}, which can be used to
select an {\hyperref[\detokenize{ioloop:tornado.ioloop.IOLoop}]{\sphinxcrossref{\sphinxcode{\sphinxupquote{IOLoop}}}}} implementation other than the default.

\item {} 
The {\hyperref[\detokenize{ioloop:tornado.ioloop.IOLoop}]{\sphinxcrossref{\sphinxcode{\sphinxupquote{IOLoop}}}}} poller implementations (\sphinxcode{\sphinxupquote{select}}, \sphinxcode{\sphinxupquote{epoll}}, \sphinxcode{\sphinxupquote{kqueue}})
are now available as distinct subclasses of {\hyperref[\detokenize{ioloop:tornado.ioloop.IOLoop}]{\sphinxcrossref{\sphinxcode{\sphinxupquote{IOLoop}}}}}.  Instantiating
{\hyperref[\detokenize{ioloop:tornado.ioloop.IOLoop}]{\sphinxcrossref{\sphinxcode{\sphinxupquote{IOLoop}}}}} will continue to automatically choose the best available
implementation.

\item {} 
The {\hyperref[\detokenize{ioloop:tornado.ioloop.IOLoop}]{\sphinxcrossref{\sphinxcode{\sphinxupquote{IOLoop}}}}} constructor has a new keyword argument \sphinxcode{\sphinxupquote{time\_func}},
which can be used to set the time function used when scheduling callbacks.
This is most useful with the \sphinxhref{https://docs.python.org/3.6/library/time.html\#time.monotonic}{\sphinxcode{\sphinxupquote{time.monotonic}}} function, introduced
in Python 3.3 and backported to older versions via the \sphinxcode{\sphinxupquote{monotime}}
module.  Using a monotonic clock here avoids problems when the system
clock is changed.

\item {} 
New function {\hyperref[\detokenize{ioloop:tornado.ioloop.IOLoop.time}]{\sphinxcrossref{\sphinxcode{\sphinxupquote{IOLoop.time}}}}} returns the current time according to the
IOLoop.  To use the new monotonic clock functionality, all calls to
{\hyperref[\detokenize{ioloop:tornado.ioloop.IOLoop.add_timeout}]{\sphinxcrossref{\sphinxcode{\sphinxupquote{IOLoop.add\_timeout}}}}} must be either pass a \sphinxhref{https://docs.python.org/3.6/library/datetime.html\#datetime.timedelta}{\sphinxcode{\sphinxupquote{datetime.timedelta}}} or
a time relative to {\hyperref[\detokenize{ioloop:tornado.ioloop.IOLoop.time}]{\sphinxcrossref{\sphinxcode{\sphinxupquote{IOLoop.time}}}}}, not \sphinxhref{https://docs.python.org/3.6/library/time.html\#time.time}{\sphinxcode{\sphinxupquote{time.time}}}.  (\sphinxhref{https://docs.python.org/3.6/library/time.html\#time.time}{\sphinxcode{\sphinxupquote{time.time}}} will
continue to work only as long as the IOLoop’s \sphinxcode{\sphinxupquote{time\_func}} argument
is not used).

\item {} 
New convenience method {\hyperref[\detokenize{ioloop:tornado.ioloop.IOLoop.run_sync}]{\sphinxcrossref{\sphinxcode{\sphinxupquote{IOLoop.run\_sync}}}}} can be used to start an IOLoop
just long enough to run a single coroutine.

\item {} 
New method {\hyperref[\detokenize{ioloop:tornado.ioloop.IOLoop.add_callback_from_signal}]{\sphinxcrossref{\sphinxcode{\sphinxupquote{IOLoop.add\_callback\_from\_signal}}}}} is safe to use in a signal
handler (the regular {\hyperref[\detokenize{ioloop:tornado.ioloop.IOLoop.add_callback}]{\sphinxcrossref{\sphinxcode{\sphinxupquote{add\_callback}}}}} method may deadlock).

\item {} 
{\hyperref[\detokenize{ioloop:tornado.ioloop.IOLoop}]{\sphinxcrossref{\sphinxcode{\sphinxupquote{IOLoop}}}}} now uses \sphinxhref{https://docs.python.org/3.6/library/signal.html\#signal.set\_wakeup\_fd}{\sphinxcode{\sphinxupquote{signal.set\_wakeup\_fd}}} where available (Python 2.6+
on Unix) to avoid a race condition that could result in Python signal
handlers being delayed.

\item {} 
Method \sphinxcode{\sphinxupquote{IOLoop.running()}} has been removed.

\item {} 
{\hyperref[\detokenize{ioloop:tornado.ioloop.IOLoop}]{\sphinxcrossref{\sphinxcode{\sphinxupquote{IOLoop}}}}} has been refactored to better support subclassing.

\item {} 
{\hyperref[\detokenize{ioloop:tornado.ioloop.IOLoop.add_callback}]{\sphinxcrossref{\sphinxcode{\sphinxupquote{IOLoop.add\_callback}}}}} and {\hyperref[\detokenize{ioloop:tornado.ioloop.IOLoop.add_callback_from_signal}]{\sphinxcrossref{\sphinxcode{\sphinxupquote{add\_callback\_from\_signal}}}}} now take
\sphinxcode{\sphinxupquote{*args, **kwargs}} to pass along to the callback.

\end{itemize}


\subparagraph{\sphinxstyleliteralintitle{\sphinxupquote{tornado.iostream}}}
\label{\detokenize{releases/v3.0.0:tornado-iostream}}\begin{itemize}
\item {} 
{\hyperref[\detokenize{iostream:tornado.iostream.IOStream.connect}]{\sphinxcrossref{\sphinxcode{\sphinxupquote{IOStream.connect}}}}} now has an optional \sphinxcode{\sphinxupquote{server\_hostname}} argument
which will be used for SSL certificate validation when applicable.
Additionally, when supported (on Python 3.2+), this hostname
will be sent via SNI (and this is supported by \sphinxcode{\sphinxupquote{tornado.simple\_httpclient}})

\item {} 
Much of {\hyperref[\detokenize{iostream:tornado.iostream.IOStream}]{\sphinxcrossref{\sphinxcode{\sphinxupquote{IOStream}}}}} has been refactored into a separate class
{\hyperref[\detokenize{iostream:tornado.iostream.BaseIOStream}]{\sphinxcrossref{\sphinxcode{\sphinxupquote{BaseIOStream}}}}}.

\item {} 
New class {\hyperref[\detokenize{iostream:tornado.iostream.PipeIOStream}]{\sphinxcrossref{\sphinxcode{\sphinxupquote{tornado.iostream.PipeIOStream}}}}} provides the IOStream
interface on pipe file descriptors.

\item {} 
{\hyperref[\detokenize{iostream:tornado.iostream.IOStream}]{\sphinxcrossref{\sphinxcode{\sphinxupquote{IOStream}}}}} now raises a new exception
\sphinxcode{\sphinxupquote{tornado.iostream.StreamClosedError}} when you attempt to read or
write after the stream has been closed (by either side).

\item {} 
{\hyperref[\detokenize{iostream:tornado.iostream.IOStream}]{\sphinxcrossref{\sphinxcode{\sphinxupquote{IOStream}}}}} now simply closes the connection when it gets an
\sphinxcode{\sphinxupquote{ECONNRESET}} error, rather than logging it as an error.

\item {} 
\sphinxcode{\sphinxupquote{IOStream.error}} no longer picks up unrelated exceptions.

\item {} 
{\hyperref[\detokenize{iostream:tornado.iostream.BaseIOStream.close}]{\sphinxcrossref{\sphinxcode{\sphinxupquote{BaseIOStream.close}}}}} now has an \sphinxcode{\sphinxupquote{exc\_info}} argument (similar to the
one used in the \sphinxhref{https://docs.python.org/3.6/library/logging.html\#module-logging}{\sphinxcode{\sphinxupquote{logging}}} module) that can be used to set the stream’s
\sphinxcode{\sphinxupquote{error}} attribute when closing it.

\item {} 
{\hyperref[\detokenize{iostream:tornado.iostream.BaseIOStream.read_until_close}]{\sphinxcrossref{\sphinxcode{\sphinxupquote{BaseIOStream.read\_until\_close}}}}} now works correctly when it is called
while there is buffered data.

\item {} 
Fixed a major performance regression when run on PyPy (introduced in
Tornado 2.3).

\end{itemize}


\subparagraph{\sphinxstyleliteralintitle{\sphinxupquote{tornado.log}}}
\label{\detokenize{releases/v3.0.0:tornado-log}}\begin{itemize}
\item {} 
New module containing {\hyperref[\detokenize{log:tornado.log.enable_pretty_logging}]{\sphinxcrossref{\sphinxcode{\sphinxupquote{enable\_pretty\_logging}}}}} and {\hyperref[\detokenize{log:tornado.log.LogFormatter}]{\sphinxcrossref{\sphinxcode{\sphinxupquote{LogFormatter}}}}},
moved from the options module.

\item {} 
{\hyperref[\detokenize{log:tornado.log.LogFormatter}]{\sphinxcrossref{\sphinxcode{\sphinxupquote{LogFormatter}}}}} now handles non-ascii data in messages and tracebacks better.

\end{itemize}


\subparagraph{\sphinxstyleliteralintitle{\sphinxupquote{tornado.netutil}}}
\label{\detokenize{releases/v3.0.0:tornado-netutil}}\begin{itemize}
\item {} 
New class {\hyperref[\detokenize{netutil:tornado.netutil.Resolver}]{\sphinxcrossref{\sphinxcode{\sphinxupquote{tornado.netutil.Resolver}}}}} provides an asynchronous
interface to DNS resolution.  The default implementation is still
blocking, but non-blocking implementations are available using one
of three optional dependencies: {\hyperref[\detokenize{netutil:tornado.netutil.ThreadedResolver}]{\sphinxcrossref{\sphinxcode{\sphinxupquote{ThreadedResolver}}}}}
using the \sphinxhref{https://docs.python.org/3.6/library/concurrent.futures.html\#module-concurrent.futures}{\sphinxcode{\sphinxupquote{concurrent.futures}}} thread pool,
{\hyperref[\detokenize{caresresolver:tornado.platform.caresresolver.CaresResolver}]{\sphinxcrossref{\sphinxcode{\sphinxupquote{tornado.platform.caresresolver.CaresResolver}}}}} using the \sphinxcode{\sphinxupquote{pycares}}
library, or {\hyperref[\detokenize{twisted:tornado.platform.twisted.TwistedResolver}]{\sphinxcrossref{\sphinxcode{\sphinxupquote{tornado.platform.twisted.TwistedResolver}}}}} using \sphinxcode{\sphinxupquote{twisted}}

\item {} 
New function {\hyperref[\detokenize{netutil:tornado.netutil.is_valid_ip}]{\sphinxcrossref{\sphinxcode{\sphinxupquote{tornado.netutil.is\_valid\_ip}}}}} returns true if a given string
is a valid IP (v4 or v6) address.

\item {} 
{\hyperref[\detokenize{netutil:tornado.netutil.bind_sockets}]{\sphinxcrossref{\sphinxcode{\sphinxupquote{tornado.netutil.bind\_sockets}}}}} has a new \sphinxcode{\sphinxupquote{flags}} argument that can
be used to pass additional flags to \sphinxcode{\sphinxupquote{getaddrinfo}}.

\item {} 
{\hyperref[\detokenize{netutil:tornado.netutil.bind_sockets}]{\sphinxcrossref{\sphinxcode{\sphinxupquote{tornado.netutil.bind\_sockets}}}}} no longer sets \sphinxcode{\sphinxupquote{AI\_ADDRCONFIG}}; this will
cause it to bind to both ipv4 and ipv6 more often than before.

\item {} 
{\hyperref[\detokenize{netutil:tornado.netutil.bind_sockets}]{\sphinxcrossref{\sphinxcode{\sphinxupquote{tornado.netutil.bind\_sockets}}}}} now works when Python was compiled
with \sphinxcode{\sphinxupquote{-{-}disable-ipv6}} but IPv6 DNS resolution is available on the
system.

\item {} 
\sphinxcode{\sphinxupquote{tornado.netutil.TCPServer}} has moved to its own module, {\hyperref[\detokenize{tcpserver:module-tornado.tcpserver}]{\sphinxcrossref{\sphinxcode{\sphinxupquote{tornado.tcpserver}}}}}.

\end{itemize}


\subparagraph{\sphinxstyleliteralintitle{\sphinxupquote{tornado.options}}}
\label{\detokenize{releases/v3.0.0:tornado-options}}\begin{itemize}
\item {} 
The class underlying the functions in {\hyperref[\detokenize{options:module-tornado.options}]{\sphinxcrossref{\sphinxcode{\sphinxupquote{tornado.options}}}}} is now public
({\hyperref[\detokenize{options:tornado.options.OptionParser}]{\sphinxcrossref{\sphinxcode{\sphinxupquote{tornado.options.OptionParser}}}}}).  This can be used to create multiple
independent option sets, such as for subcommands.

\item {} 
{\hyperref[\detokenize{options:tornado.options.parse_config_file}]{\sphinxcrossref{\sphinxcode{\sphinxupquote{tornado.options.parse\_config\_file}}}}} now configures logging automatically
by default, in the same way that {\hyperref[\detokenize{options:tornado.options.parse_command_line}]{\sphinxcrossref{\sphinxcode{\sphinxupquote{parse\_command\_line}}}}} does.

\item {} 
New function {\hyperref[\detokenize{options:tornado.options.add_parse_callback}]{\sphinxcrossref{\sphinxcode{\sphinxupquote{tornado.options.add\_parse\_callback}}}}} schedules a callback
to be run after the command line or config file has been parsed.  The
keyword argument \sphinxcode{\sphinxupquote{final=False}} can be used on either parsing function
to suppress these callbacks.

\item {} 
{\hyperref[\detokenize{options:tornado.options.define}]{\sphinxcrossref{\sphinxcode{\sphinxupquote{tornado.options.define}}}}} now takes a \sphinxcode{\sphinxupquote{callback}} argument.  This callback
will be run with the new value whenever the option is changed.  This is
especially useful for options that set other options, such as by reading
from a config file.

\item {} 
{\hyperref[\detokenize{options:tornado.options.parse_command_line}]{\sphinxcrossref{\sphinxcode{\sphinxupquote{tornado.options.parse\_command\_line}}}}} \sphinxcode{\sphinxupquote{-{-}help}} output now goes to \sphinxcode{\sphinxupquote{stderr}}
rather than \sphinxcode{\sphinxupquote{stdout}}.

\item {} 
{\hyperref[\detokenize{options:tornado.options.options}]{\sphinxcrossref{\sphinxcode{\sphinxupquote{tornado.options.options}}}}} is no longer a subclass of \sphinxhref{https://docs.python.org/3.6/library/stdtypes.html\#dict}{\sphinxcode{\sphinxupquote{dict}}}; attribute-style
access is now required.

\item {} 
{\hyperref[\detokenize{options:tornado.options.options}]{\sphinxcrossref{\sphinxcode{\sphinxupquote{tornado.options.options}}}}} (and {\hyperref[\detokenize{options:tornado.options.OptionParser}]{\sphinxcrossref{\sphinxcode{\sphinxupquote{OptionParser}}}}} instances generally) now
have a {\hyperref[\detokenize{options:tornado.options.OptionParser.mockable}]{\sphinxcrossref{\sphinxcode{\sphinxupquote{mockable()}}}}} method that returns a wrapper object compatible with
\sphinxhref{https://docs.python.org/3.6/library/unittest.mock.html\#unittest.mock.patch}{\sphinxcode{\sphinxupquote{mock.patch}}}.

\item {} 
Function \sphinxcode{\sphinxupquote{tornado.options.enable\_pretty\_logging}} has been moved to the
{\hyperref[\detokenize{log:module-tornado.log}]{\sphinxcrossref{\sphinxcode{\sphinxupquote{tornado.log}}}}} module.

\end{itemize}


\subparagraph{\sphinxstyleliteralintitle{\sphinxupquote{tornado.platform.caresresolver}}}
\label{\detokenize{releases/v3.0.0:tornado-platform-caresresolver}}\begin{itemize}
\item {} 
New module containing an asynchronous implementation of the {\hyperref[\detokenize{netutil:tornado.netutil.Resolver}]{\sphinxcrossref{\sphinxcode{\sphinxupquote{Resolver}}}}}
interface, using the \sphinxcode{\sphinxupquote{pycares}} library.

\end{itemize}


\subparagraph{\sphinxstyleliteralintitle{\sphinxupquote{tornado.platform.twisted}}}
\label{\detokenize{releases/v3.0.0:tornado-platform-twisted}}\begin{itemize}
\item {} 
New class \sphinxcode{\sphinxupquote{tornado.platform.twisted.TwistedIOLoop}} allows Tornado
code to be run on the Twisted reactor (as opposed to the existing
\sphinxcode{\sphinxupquote{TornadoReactor}}, which bridges the gap in the other direction).

\item {} 
New class {\hyperref[\detokenize{twisted:tornado.platform.twisted.TwistedResolver}]{\sphinxcrossref{\sphinxcode{\sphinxupquote{tornado.platform.twisted.TwistedResolver}}}}} is an asynchronous
implementation of the {\hyperref[\detokenize{netutil:tornado.netutil.Resolver}]{\sphinxcrossref{\sphinxcode{\sphinxupquote{Resolver}}}}} interface.

\end{itemize}


\subparagraph{\sphinxstyleliteralintitle{\sphinxupquote{tornado.process}}}
\label{\detokenize{releases/v3.0.0:tornado-process}}\begin{itemize}
\item {} 
New class {\hyperref[\detokenize{process:tornado.process.Subprocess}]{\sphinxcrossref{\sphinxcode{\sphinxupquote{tornado.process.Subprocess}}}}} wraps \sphinxhref{https://docs.python.org/3.6/library/subprocess.html\#subprocess.Popen}{\sphinxcode{\sphinxupquote{subprocess.Popen}}} with
{\hyperref[\detokenize{iostream:tornado.iostream.PipeIOStream}]{\sphinxcrossref{\sphinxcode{\sphinxupquote{PipeIOStream}}}}} access to the child’s file descriptors.

\end{itemize}


\subparagraph{\sphinxstyleliteralintitle{\sphinxupquote{tornado.simple\_httpclient}}}
\label{\detokenize{releases/v3.0.0:tornado-simple-httpclient}}\begin{itemize}
\item {} 
\sphinxcode{\sphinxupquote{SimpleAsyncHTTPClient}} now takes a \sphinxcode{\sphinxupquote{resolver}} keyword argument
(which may be passed to either the constructor or {\hyperref[\detokenize{util:tornado.util.Configurable.configure}]{\sphinxcrossref{\sphinxcode{\sphinxupquote{configure}}}}}), to allow it to use the new non-blocking
{\hyperref[\detokenize{netutil:tornado.netutil.Resolver}]{\sphinxcrossref{\sphinxcode{\sphinxupquote{tornado.netutil.Resolver}}}}}.

\item {} 
When following redirects, \sphinxcode{\sphinxupquote{SimpleAsyncHTTPClient}} now treats a 302
response code the same as a 303.  This is contrary to the HTTP spec
but consistent with all browsers and other major HTTP clients
(including \sphinxcode{\sphinxupquote{CurlAsyncHTTPClient}}).

\item {} 
The behavior of \sphinxcode{\sphinxupquote{header\_callback}} with \sphinxcode{\sphinxupquote{SimpleAsyncHTTPClient}} has
changed and is now the same as that of \sphinxcode{\sphinxupquote{CurlAsyncHTTPClient}}.  The
header callback now receives the first line of the response (e.g.
\sphinxcode{\sphinxupquote{HTTP/1.0 200 OK}}) and the final empty line.

\item {} 
\sphinxcode{\sphinxupquote{tornado.simple\_httpclient}} now accepts responses with a 304
status code that include a \sphinxcode{\sphinxupquote{Content-Length}} header.

\item {} 
Fixed a bug in which \sphinxcode{\sphinxupquote{SimpleAsyncHTTPClient}} callbacks were being run in the
client’s \sphinxcode{\sphinxupquote{stack\_context}}.

\end{itemize}


\subparagraph{\sphinxstyleliteralintitle{\sphinxupquote{tornado.stack\_context}}}
\label{\detokenize{releases/v3.0.0:tornado-stack-context}}\begin{itemize}
\item {} 
\sphinxcode{\sphinxupquote{stack\_context.wrap}} now runs the wrapped callback in a more consistent
environment by recreating contexts even if they already exist on the
stack.

\item {} 
Fixed a bug in which stack contexts could leak from one callback
chain to another.

\item {} 
Yield statements inside a \sphinxcode{\sphinxupquote{with}} statement can cause stack
contexts to become inconsistent; an exception will now be raised
when this case is detected.

\end{itemize}


\subparagraph{\sphinxstyleliteralintitle{\sphinxupquote{tornado.template}}}
\label{\detokenize{releases/v3.0.0:tornado-template}}\begin{itemize}
\item {} 
Errors while rendering templates no longer log the generated code,
since the enhanced stack traces (from version 2.1) should make this
unnecessary.

\item {} 
The \sphinxcode{\sphinxupquote{\{\% apply \%\}}} directive now works properly with functions that return
both unicode strings and byte strings (previously only byte strings were
supported).

\item {} 
Code in templates is no longer affected by Tornado’s \sphinxcode{\sphinxupquote{\_\_future\_\_}} imports
(which previously included \sphinxcode{\sphinxupquote{absolute\_import}} and \sphinxcode{\sphinxupquote{division}}).

\end{itemize}


\subparagraph{\sphinxstyleliteralintitle{\sphinxupquote{tornado.testing}}}
\label{\detokenize{releases/v3.0.0:tornado-testing}}\begin{itemize}
\item {} 
New function {\hyperref[\detokenize{testing:tornado.testing.bind_unused_port}]{\sphinxcrossref{\sphinxcode{\sphinxupquote{tornado.testing.bind\_unused\_port}}}}} both chooses a port
and binds a socket to it, so there is no risk of another process
using the same port.  \sphinxcode{\sphinxupquote{get\_unused\_port}} is now deprecated.

\item {} 
New decorator {\hyperref[\detokenize{testing:tornado.testing.gen_test}]{\sphinxcrossref{\sphinxcode{\sphinxupquote{tornado.testing.gen\_test}}}}} can be used to allow for
yielding {\hyperref[\detokenize{gen:module-tornado.gen}]{\sphinxcrossref{\sphinxcode{\sphinxupquote{tornado.gen}}}}} objects in tests, as an alternative to the
\sphinxcode{\sphinxupquote{stop}} and \sphinxcode{\sphinxupquote{wait}} methods of {\hyperref[\detokenize{testing:tornado.testing.AsyncTestCase}]{\sphinxcrossref{\sphinxcode{\sphinxupquote{AsyncTestCase}}}}}.

\item {} 
{\hyperref[\detokenize{testing:tornado.testing.AsyncTestCase}]{\sphinxcrossref{\sphinxcode{\sphinxupquote{tornado.testing.AsyncTestCase}}}}} and friends now extend \sphinxcode{\sphinxupquote{unittest2.TestCase}}
when it is available (and continue to use the standard \sphinxcode{\sphinxupquote{unittest}} module
when \sphinxcode{\sphinxupquote{unittest2}} is not available)

\item {} 
{\hyperref[\detokenize{testing:tornado.testing.ExpectLog}]{\sphinxcrossref{\sphinxcode{\sphinxupquote{tornado.testing.ExpectLog}}}}} can be used as a finer-grained alternative
to \sphinxcode{\sphinxupquote{tornado.testing.LogTrapTestCase}}

\item {} 
The command-line interface to {\hyperref[\detokenize{testing:tornado.testing.main}]{\sphinxcrossref{\sphinxcode{\sphinxupquote{tornado.testing.main}}}}} now supports
additional arguments from the underlying \sphinxhref{https://docs.python.org/3.6/library/unittest.html\#module-unittest}{\sphinxcode{\sphinxupquote{unittest}}} module:
\sphinxcode{\sphinxupquote{verbose}}, \sphinxcode{\sphinxupquote{quiet}}, \sphinxcode{\sphinxupquote{failfast}}, \sphinxcode{\sphinxupquote{catch}}, \sphinxcode{\sphinxupquote{buffer}}.

\item {} 
The deprecated \sphinxcode{\sphinxupquote{-{-}autoreload}} option of {\hyperref[\detokenize{testing:tornado.testing.main}]{\sphinxcrossref{\sphinxcode{\sphinxupquote{tornado.testing.main}}}}} has
been removed.  Use \sphinxcode{\sphinxupquote{python -m tornado.autoreload}} as a prefix command
instead.

\item {} 
The \sphinxcode{\sphinxupquote{-{-}httpclient}} option of {\hyperref[\detokenize{testing:tornado.testing.main}]{\sphinxcrossref{\sphinxcode{\sphinxupquote{tornado.testing.main}}}}} has been moved
to \sphinxcode{\sphinxupquote{tornado.test.runtests}} so as not to pollute the application
option namespace.  The {\hyperref[\detokenize{options:module-tornado.options}]{\sphinxcrossref{\sphinxcode{\sphinxupquote{tornado.options}}}}} module’s new callback
support now makes it easy to add options from a wrapper script
instead of putting all possible options in {\hyperref[\detokenize{testing:tornado.testing.main}]{\sphinxcrossref{\sphinxcode{\sphinxupquote{tornado.testing.main}}}}}.

\item {} 
{\hyperref[\detokenize{testing:tornado.testing.AsyncHTTPTestCase}]{\sphinxcrossref{\sphinxcode{\sphinxupquote{AsyncHTTPTestCase}}}}} no longer calls {\hyperref[\detokenize{httpclient:tornado.httpclient.AsyncHTTPClient.close}]{\sphinxcrossref{\sphinxcode{\sphinxupquote{AsyncHTTPClient.close}}}}} for tests
that use the singleton {\hyperref[\detokenize{ioloop:tornado.ioloop.IOLoop.instance}]{\sphinxcrossref{\sphinxcode{\sphinxupquote{IOLoop.instance}}}}}.

\item {} 
\sphinxcode{\sphinxupquote{LogTrapTestCase}} no longer fails when run in unknown logging
configurations.  This allows tests to be run under nose, which does its
own log buffering (\sphinxcode{\sphinxupquote{LogTrapTestCase}} doesn’t do anything useful in this
case, but at least it doesn’t break things any more).

\end{itemize}


\subparagraph{\sphinxstyleliteralintitle{\sphinxupquote{tornado.util}}}
\label{\detokenize{releases/v3.0.0:tornado-util}}\begin{itemize}
\item {} 
\sphinxcode{\sphinxupquote{tornado.util.b}} (which was only intended for internal use) is gone.

\end{itemize}


\subparagraph{\sphinxstyleliteralintitle{\sphinxupquote{tornado.web}}}
\label{\detokenize{releases/v3.0.0:tornado-web}}\begin{itemize}
\item {} 
{\hyperref[\detokenize{web:tornado.web.RequestHandler.set_header}]{\sphinxcrossref{\sphinxcode{\sphinxupquote{RequestHandler.set\_header}}}}} now overwrites previous header values
case-insensitively.

\item {} 
{\hyperref[\detokenize{web:tornado.web.RequestHandler}]{\sphinxcrossref{\sphinxcode{\sphinxupquote{tornado.web.RequestHandler}}}}} has new attributes \sphinxcode{\sphinxupquote{path\_args}} and
\sphinxcode{\sphinxupquote{path\_kwargs}}, which contain the positional and keyword arguments
that are passed to the \sphinxcode{\sphinxupquote{get}}/\sphinxcode{\sphinxupquote{post}}/etc method.  These attributes
are set before those methods are called, so they are available during
\sphinxcode{\sphinxupquote{prepare()}}

\item {} 
{\hyperref[\detokenize{web:tornado.web.ErrorHandler}]{\sphinxcrossref{\sphinxcode{\sphinxupquote{tornado.web.ErrorHandler}}}}} no longer requires XSRF tokens on \sphinxcode{\sphinxupquote{POST}}
requests, so posts to an unknown url will always return 404 instead of
complaining about XSRF tokens.

\item {} 
Several methods related to HTTP status codes now take a \sphinxcode{\sphinxupquote{reason}} keyword
argument to specify an alternate “reason” string (i.e. the “Not Found” in
“HTTP/1.1 404 Not Found”).  It is now possible to set status codes other
than those defined in the spec, as long as a reason string is given.

\item {} 
The \sphinxcode{\sphinxupquote{Date}} HTTP header is now set by default on all responses.

\item {} 
\sphinxcode{\sphinxupquote{Etag}}/\sphinxcode{\sphinxupquote{If-None-Match}} requests now work with {\hyperref[\detokenize{web:tornado.web.StaticFileHandler}]{\sphinxcrossref{\sphinxcode{\sphinxupquote{StaticFileHandler}}}}}.

\item {} 
{\hyperref[\detokenize{web:tornado.web.StaticFileHandler}]{\sphinxcrossref{\sphinxcode{\sphinxupquote{StaticFileHandler}}}}} no longer sets \sphinxcode{\sphinxupquote{Cache-Control: public}} unnecessarily.

\item {} 
When gzip is enabled in a {\hyperref[\detokenize{web:tornado.web.Application}]{\sphinxcrossref{\sphinxcode{\sphinxupquote{tornado.web.Application}}}}}, appropriate
\sphinxcode{\sphinxupquote{Vary: Accept-Encoding}} headers are now sent.

\item {} 
It is no longer necessary to pass all handlers for a host in a single
{\hyperref[\detokenize{web:tornado.web.Application.add_handlers}]{\sphinxcrossref{\sphinxcode{\sphinxupquote{Application.add\_handlers}}}}} call.  Now the request will be matched
against the handlers for any \sphinxcode{\sphinxupquote{host\_pattern}} that includes the request’s
\sphinxcode{\sphinxupquote{Host}} header.

\end{itemize}


\subparagraph{\sphinxstyleliteralintitle{\sphinxupquote{tornado.websocket}}}
\label{\detokenize{releases/v3.0.0:tornado-websocket}}\begin{itemize}
\item {} 
Client-side WebSocket support is now available:
{\hyperref[\detokenize{websocket:tornado.websocket.websocket_connect}]{\sphinxcrossref{\sphinxcode{\sphinxupquote{tornado.websocket.websocket\_connect}}}}}

\item {} 
{\hyperref[\detokenize{websocket:tornado.websocket.WebSocketHandler}]{\sphinxcrossref{\sphinxcode{\sphinxupquote{WebSocketHandler}}}}} has new methods {\hyperref[\detokenize{websocket:tornado.websocket.WebSocketHandler.ping}]{\sphinxcrossref{\sphinxcode{\sphinxupquote{ping}}}}} and
{\hyperref[\detokenize{websocket:tornado.websocket.WebSocketHandler.on_pong}]{\sphinxcrossref{\sphinxcode{\sphinxupquote{on\_pong}}}}} to send pings to the browser (not
supported on the \sphinxcode{\sphinxupquote{draft76}} protocol)

\end{itemize}


\subsection{What’s new in Tornado 2.4.1}
\label{\detokenize{releases/v2.4.1:what-s-new-in-tornado-2-4-1}}\label{\detokenize{releases/v2.4.1::doc}}

\subsubsection{Nov 24, 2012}
\label{\detokenize{releases/v2.4.1:nov-24-2012}}

\paragraph{Bug fixes}
\label{\detokenize{releases/v2.4.1:bug-fixes}}\begin{itemize}
\item {} 
Fixed a memory leak in \sphinxcode{\sphinxupquote{tornado.stack\_context}} that was especially likely
with long-running \sphinxcode{\sphinxupquote{@gen.engine}} functions.

\item {} 
{\hyperref[\detokenize{auth:tornado.auth.TwitterMixin}]{\sphinxcrossref{\sphinxcode{\sphinxupquote{tornado.auth.TwitterMixin}}}}} now works on Python 3.

\item {} 
Fixed a bug in which \sphinxcode{\sphinxupquote{IOStream.read\_until\_close}} with a streaming callback
would sometimes pass the last chunk of data to the final callback instead
of the streaming callback.

\end{itemize}


\subsection{What’s new in Tornado 2.4}
\label{\detokenize{releases/v2.4.0:what-s-new-in-tornado-2-4}}\label{\detokenize{releases/v2.4.0::doc}}

\subsubsection{Sep 4, 2012}
\label{\detokenize{releases/v2.4.0:sep-4-2012}}

\paragraph{General}
\label{\detokenize{releases/v2.4.0:general}}\begin{itemize}
\item {} 
Fixed Python 3 bugs in {\hyperref[\detokenize{auth:module-tornado.auth}]{\sphinxcrossref{\sphinxcode{\sphinxupquote{tornado.auth}}}}}, {\hyperref[\detokenize{locale:module-tornado.locale}]{\sphinxcrossref{\sphinxcode{\sphinxupquote{tornado.locale}}}}}, and {\hyperref[\detokenize{wsgi:module-tornado.wsgi}]{\sphinxcrossref{\sphinxcode{\sphinxupquote{tornado.wsgi}}}}}.

\end{itemize}


\paragraph{HTTP clients}
\label{\detokenize{releases/v2.4.0:http-clients}}\begin{itemize}
\item {} 
Removed \sphinxcode{\sphinxupquote{max\_simultaneous\_connections}} argument from {\hyperref[\detokenize{httpclient:module-tornado.httpclient}]{\sphinxcrossref{\sphinxcode{\sphinxupquote{tornado.httpclient}}}}}
(both implementations).  This argument hasn’t been useful for some time
(if you were using it you probably want \sphinxcode{\sphinxupquote{max\_clients}} instead)

\item {} 
\sphinxcode{\sphinxupquote{tornado.simple\_httpclient}} now accepts and ignores HTTP 1xx status
responses.

\end{itemize}


\paragraph{\sphinxstyleliteralintitle{\sphinxupquote{tornado.ioloop}} and \sphinxstyleliteralintitle{\sphinxupquote{tornado.iostream}}}
\label{\detokenize{releases/v2.4.0:tornado-ioloop-and-tornado-iostream}}\begin{itemize}
\item {} 
Fixed a bug introduced in 2.3 that would cause {\hyperref[\detokenize{iostream:tornado.iostream.IOStream}]{\sphinxcrossref{\sphinxcode{\sphinxupquote{IOStream}}}}} close callbacks
to not run if there were pending reads.

\item {} 
Improved error handling in {\hyperref[\detokenize{iostream:tornado.iostream.SSLIOStream}]{\sphinxcrossref{\sphinxcode{\sphinxupquote{SSLIOStream}}}}} and SSL-enabled {\hyperref[\detokenize{tcpserver:tornado.tcpserver.TCPServer}]{\sphinxcrossref{\sphinxcode{\sphinxupquote{TCPServer}}}}}.

\item {} 
\sphinxcode{\sphinxupquote{SSLIOStream.get\_ssl\_certificate}} now has a \sphinxcode{\sphinxupquote{binary\_form}} argument
which is passed to \sphinxcode{\sphinxupquote{SSLSocket.getpeercert}}.

\item {} 
\sphinxcode{\sphinxupquote{SSLIOStream.write}} can now be called while the connection is in progress,
same as non-SSL {\hyperref[\detokenize{iostream:tornado.iostream.IOStream}]{\sphinxcrossref{\sphinxcode{\sphinxupquote{IOStream}}}}} (but be careful not to send sensitive data until
the connection has completed and the certificate has been verified).

\item {} 
{\hyperref[\detokenize{ioloop:tornado.ioloop.IOLoop.add_handler}]{\sphinxcrossref{\sphinxcode{\sphinxupquote{IOLoop.add\_handler}}}}} cannot be called more than once with the same file
descriptor.  This was always true for \sphinxcode{\sphinxupquote{epoll}}, but now the other
implementations enforce it too.

\item {} 
On Windows, {\hyperref[\detokenize{tcpserver:tornado.tcpserver.TCPServer}]{\sphinxcrossref{\sphinxcode{\sphinxupquote{TCPServer}}}}} uses \sphinxcode{\sphinxupquote{SO\_EXCLUSIVEADDRUSER}} instead of \sphinxcode{\sphinxupquote{SO\_REUSEADDR}}.

\end{itemize}


\paragraph{\sphinxstyleliteralintitle{\sphinxupquote{tornado.template}}}
\label{\detokenize{releases/v2.4.0:tornado-template}}\begin{itemize}
\item {} 
\sphinxcode{\sphinxupquote{\{\% break \%\}}} and \sphinxcode{\sphinxupquote{\{\% continue \%\}}} can now be used looping constructs
in templates.

\item {} 
It is no longer an error for an if/else/for/etc block in a template to
have an empty body.

\end{itemize}


\paragraph{\sphinxstyleliteralintitle{\sphinxupquote{tornado.testing}}}
\label{\detokenize{releases/v2.4.0:tornado-testing}}\begin{itemize}
\item {} 
New class {\hyperref[\detokenize{testing:tornado.testing.AsyncHTTPSTestCase}]{\sphinxcrossref{\sphinxcode{\sphinxupquote{tornado.testing.AsyncHTTPSTestCase}}}}} is like {\hyperref[\detokenize{testing:tornado.testing.AsyncHTTPTestCase}]{\sphinxcrossref{\sphinxcode{\sphinxupquote{AsyncHTTPTestCase}}}}}.
but enables SSL for the testing server (by default using a self-signed
testing certificate).

\item {} 
{\hyperref[\detokenize{testing:tornado.testing.main}]{\sphinxcrossref{\sphinxcode{\sphinxupquote{tornado.testing.main}}}}} now accepts additional keyword arguments and forwards
them to \sphinxhref{https://docs.python.org/3.6/library/unittest.html\#unittest.main}{\sphinxcode{\sphinxupquote{unittest.main}}}.

\end{itemize}


\paragraph{\sphinxstyleliteralintitle{\sphinxupquote{tornado.web}}}
\label{\detokenize{releases/v2.4.0:tornado-web}}\begin{itemize}
\item {} 
New method {\hyperref[\detokenize{web:tornado.web.RequestHandler.get_template_namespace}]{\sphinxcrossref{\sphinxcode{\sphinxupquote{RequestHandler.get\_template\_namespace}}}}} can be overridden to
add additional variables without modifying keyword arguments to
\sphinxcode{\sphinxupquote{render\_string}}.

\item {} 
{\hyperref[\detokenize{web:tornado.web.RequestHandler.add_header}]{\sphinxcrossref{\sphinxcode{\sphinxupquote{RequestHandler.add\_header}}}}} now works with \sphinxcode{\sphinxupquote{WSGIApplication}}.

\item {} 
{\hyperref[\detokenize{web:tornado.web.RequestHandler.get_secure_cookie}]{\sphinxcrossref{\sphinxcode{\sphinxupquote{RequestHandler.get\_secure\_cookie}}}}} now handles a potential error case.

\item {} 
\sphinxcode{\sphinxupquote{RequestHandler.\_\_init\_\_}} now calls \sphinxcode{\sphinxupquote{super().\_\_init\_\_}} to ensure that
all constructors are called when multiple inheritance is used.

\item {} 
Docs have been updated with a description of all available
{\hyperref[\detokenize{web:tornado.web.Application.settings}]{\sphinxcrossref{\sphinxcode{\sphinxupquote{Application settings}}}}}

\end{itemize}


\paragraph{Other modules}
\label{\detokenize{releases/v2.4.0:other-modules}}\begin{itemize}
\item {} 
{\hyperref[\detokenize{auth:tornado.auth.OAuthMixin}]{\sphinxcrossref{\sphinxcode{\sphinxupquote{OAuthMixin}}}}} now accepts \sphinxcode{\sphinxupquote{"oob"}} as a \sphinxcode{\sphinxupquote{callback\_uri}}.

\item {} 
{\hyperref[\detokenize{auth:tornado.auth.OpenIdMixin}]{\sphinxcrossref{\sphinxcode{\sphinxupquote{OpenIdMixin}}}}} now also returns the \sphinxcode{\sphinxupquote{claimed\_id}} field for the user.

\item {} 
{\hyperref[\detokenize{twisted:module-tornado.platform.twisted}]{\sphinxcrossref{\sphinxcode{\sphinxupquote{tornado.platform.twisted}}}}} shutdown sequence is now more compatible.

\item {} 
The logging configuration used in {\hyperref[\detokenize{options:module-tornado.options}]{\sphinxcrossref{\sphinxcode{\sphinxupquote{tornado.options}}}}} is now more tolerant
of non-ascii byte strings.

\end{itemize}


\subsection{What’s new in Tornado 2.3}
\label{\detokenize{releases/v2.3.0:what-s-new-in-tornado-2-3}}\label{\detokenize{releases/v2.3.0::doc}}

\subsubsection{May 31, 2012}
\label{\detokenize{releases/v2.3.0:may-31-2012}}

\paragraph{HTTP clients}
\label{\detokenize{releases/v2.3.0:http-clients}}\begin{itemize}
\item {} 
{\hyperref[\detokenize{httpclient:tornado.httpclient.HTTPClient}]{\sphinxcrossref{\sphinxcode{\sphinxupquote{tornado.httpclient.HTTPClient}}}}} now supports the same constructor
keyword arguments as {\hyperref[\detokenize{httpclient:tornado.httpclient.AsyncHTTPClient}]{\sphinxcrossref{\sphinxcode{\sphinxupquote{AsyncHTTPClient}}}}}.

\item {} 
The \sphinxcode{\sphinxupquote{max\_clients}} keyword argument to {\hyperref[\detokenize{httpclient:tornado.httpclient.AsyncHTTPClient.configure}]{\sphinxcrossref{\sphinxcode{\sphinxupquote{AsyncHTTPClient.configure}}}}} now works.

\item {} 
\sphinxcode{\sphinxupquote{tornado.simple\_httpclient}} now supports the \sphinxcode{\sphinxupquote{OPTIONS}} and \sphinxcode{\sphinxupquote{PATCH}}
HTTP methods.

\item {} 
\sphinxcode{\sphinxupquote{tornado.simple\_httpclient}} is better about closing its sockets
instead of leaving them for garbage collection.

\item {} 
\sphinxcode{\sphinxupquote{tornado.simple\_httpclient}} correctly verifies SSL certificates for
URLs containing IPv6 literals (This bug affected Python 2.5 and 2.6).

\item {} 
\sphinxcode{\sphinxupquote{tornado.simple\_httpclient}} no longer includes basic auth credentials
in the \sphinxcode{\sphinxupquote{Host}} header when those credentials are extracted from the URL.

\item {} 
\sphinxcode{\sphinxupquote{tornado.simple\_httpclient}} no longer modifies the caller-supplied header
dictionary, which caused problems when following redirects.

\item {} 
\sphinxcode{\sphinxupquote{tornado.curl\_httpclient}} now supports client SSL certificates (using
the same \sphinxcode{\sphinxupquote{client\_cert}} and \sphinxcode{\sphinxupquote{client\_key}} arguments as
\sphinxcode{\sphinxupquote{tornado.simple\_httpclient}})

\end{itemize}


\paragraph{HTTP Server}
\label{\detokenize{releases/v2.3.0:http-server}}\begin{itemize}
\item {} 
{\hyperref[\detokenize{httpserver:tornado.httpserver.HTTPServer}]{\sphinxcrossref{\sphinxcode{\sphinxupquote{HTTPServer}}}}} now works correctly with paths starting with \sphinxcode{\sphinxupquote{//}}

\item {} 
\sphinxcode{\sphinxupquote{HTTPHeaders.copy}} (inherited from \sphinxhref{https://docs.python.org/3.6/library/stdtypes.html\#dict.copy}{\sphinxcode{\sphinxupquote{dict.copy}}}) now works correctly.

\item {} 
\sphinxcode{\sphinxupquote{HTTPConnection.address}} is now always the socket address, even for non-IP
sockets.  \sphinxcode{\sphinxupquote{HTTPRequest.remote\_ip}} is still always an IP-style address
(fake data is used for non-IP sockets)

\item {} 
Extra data at the end of multipart form bodies is now ignored, which fixes
a compatibility problem with an iOS HTTP client library.

\end{itemize}


\paragraph{\sphinxstyleliteralintitle{\sphinxupquote{IOLoop}} and \sphinxstyleliteralintitle{\sphinxupquote{IOStream}}}
\label{\detokenize{releases/v2.3.0:ioloop-and-iostream}}\begin{itemize}
\item {} 
{\hyperref[\detokenize{iostream:tornado.iostream.IOStream}]{\sphinxcrossref{\sphinxcode{\sphinxupquote{IOStream}}}}} now has an \sphinxcode{\sphinxupquote{error}} attribute that can be used to determine
why a socket was closed.

\item {} 
\sphinxcode{\sphinxupquote{tornado.iostream.IOStream.read\_until}} and \sphinxcode{\sphinxupquote{read\_until\_regex}} are much
faster with large input.

\item {} 
\sphinxcode{\sphinxupquote{IOStream.write}} performs better when given very large strings.

\item {} 
{\hyperref[\detokenize{ioloop:tornado.ioloop.IOLoop.instance}]{\sphinxcrossref{\sphinxcode{\sphinxupquote{IOLoop.instance()}}}}} is now thread-safe.

\end{itemize}


\paragraph{\sphinxstyleliteralintitle{\sphinxupquote{tornado.options}}}
\label{\detokenize{releases/v2.3.0:tornado-options}}\begin{itemize}
\item {} 
{\hyperref[\detokenize{options:module-tornado.options}]{\sphinxcrossref{\sphinxcode{\sphinxupquote{tornado.options}}}}} options with \sphinxcode{\sphinxupquote{multiple=True}} that are set more than
once now overwrite rather than append.  This makes it possible to override
values set in \sphinxcode{\sphinxupquote{parse\_config\_file}} with \sphinxcode{\sphinxupquote{parse\_command\_line}}.

\item {} 
{\hyperref[\detokenize{options:module-tornado.options}]{\sphinxcrossref{\sphinxcode{\sphinxupquote{tornado.options}}}}} \sphinxcode{\sphinxupquote{-{-}help}} output is now prettier.

\item {} 
{\hyperref[\detokenize{options:tornado.options.options}]{\sphinxcrossref{\sphinxcode{\sphinxupquote{tornado.options.options}}}}} now supports attribute assignment.

\end{itemize}


\paragraph{\sphinxstyleliteralintitle{\sphinxupquote{tornado.template}}}
\label{\detokenize{releases/v2.3.0:tornado-template}}\begin{itemize}
\item {} 
Template files containing non-ASCII (utf8) characters now work on Python 3
regardless of the locale environment variables.

\item {} 
Templates now support \sphinxcode{\sphinxupquote{else}} clauses in
\sphinxcode{\sphinxupquote{try}}/\sphinxcode{\sphinxupquote{except}}/\sphinxcode{\sphinxupquote{finally}}/\sphinxcode{\sphinxupquote{else}} blocks.

\end{itemize}


\paragraph{\sphinxstyleliteralintitle{\sphinxupquote{tornado.web}}}
\label{\detokenize{releases/v2.3.0:tornado-web}}\begin{itemize}
\item {} 
{\hyperref[\detokenize{web:tornado.web.RequestHandler}]{\sphinxcrossref{\sphinxcode{\sphinxupquote{tornado.web.RequestHandler}}}}} now supports the \sphinxcode{\sphinxupquote{PATCH}} HTTP method.
Note that this means any existing methods named \sphinxcode{\sphinxupquote{patch}} in
\sphinxcode{\sphinxupquote{RequestHandler}} subclasses will need to be renamed.

\item {} 
{\hyperref[\detokenize{web:tornado.web.addslash}]{\sphinxcrossref{\sphinxcode{\sphinxupquote{tornado.web.addslash}}}}} and \sphinxcode{\sphinxupquote{removeslash}} decorators now send permanent
redirects (301) instead of temporary (302).

\item {} 
{\hyperref[\detokenize{web:tornado.web.RequestHandler.flush}]{\sphinxcrossref{\sphinxcode{\sphinxupquote{RequestHandler.flush}}}}} now invokes its callback whether there was any data
to flush or not.

\item {} 
Repeated calls to {\hyperref[\detokenize{web:tornado.web.RequestHandler.set_cookie}]{\sphinxcrossref{\sphinxcode{\sphinxupquote{RequestHandler.set\_cookie}}}}} with the same name now
overwrite the previous cookie instead of producing additional copies.

\item {} 
\sphinxcode{\sphinxupquote{tornado.web.OutputTransform.transform\_first\_chunk}} now takes and returns
a status code in addition to the headers and chunk.  This is a
backwards-incompatible change to an interface that was never technically
private, but was not included in the documentation and does not appear
to have been used outside Tornado itself.

\item {} 
Fixed a bug on python versions before 2.6.5 when {\hyperref[\detokenize{web:tornado.web.URLSpec}]{\sphinxcrossref{\sphinxcode{\sphinxupquote{tornado.web.URLSpec}}}}} regexes
are constructed from unicode strings and keyword arguments are extracted.

\item {} 
The \sphinxcode{\sphinxupquote{reverse\_url}} function in the template namespace now comes from
the {\hyperref[\detokenize{web:tornado.web.RequestHandler}]{\sphinxcrossref{\sphinxcode{\sphinxupquote{RequestHandler}}}}} rather than the {\hyperref[\detokenize{web:tornado.web.Application}]{\sphinxcrossref{\sphinxcode{\sphinxupquote{Application}}}}}.  (Unless overridden,
{\hyperref[\detokenize{web:tornado.web.RequestHandler.reverse_url}]{\sphinxcrossref{\sphinxcode{\sphinxupquote{RequestHandler.reverse\_url}}}}} is just an alias for the {\hyperref[\detokenize{web:tornado.web.Application}]{\sphinxcrossref{\sphinxcode{\sphinxupquote{Application}}}}}
method).

\item {} 
The \sphinxcode{\sphinxupquote{Etag}} header is now returned on 304 responses to an \sphinxcode{\sphinxupquote{If-None-Match}}
request, improving compatibility with some caches.

\item {} 
{\hyperref[\detokenize{web:module-tornado.web}]{\sphinxcrossref{\sphinxcode{\sphinxupquote{tornado.web}}}}} will no longer produce responses with status code 304
that also have entity headers such as \sphinxcode{\sphinxupquote{Content-Length}}.

\end{itemize}


\paragraph{Other modules}
\label{\detokenize{releases/v2.3.0:other-modules}}\begin{itemize}
\item {} 
{\hyperref[\detokenize{auth:tornado.auth.FacebookGraphMixin}]{\sphinxcrossref{\sphinxcode{\sphinxupquote{tornado.auth.FacebookGraphMixin}}}}} no longer sends \sphinxcode{\sphinxupquote{post\_args}} redundantly
in the url.

\item {} 
The \sphinxcode{\sphinxupquote{extra\_params}} argument to {\hyperref[\detokenize{escape:tornado.escape.linkify}]{\sphinxcrossref{\sphinxcode{\sphinxupquote{tornado.escape.linkify}}}}} may now be
a callable, to allow parameters to be chosen separately for each link.

\item {} 
{\hyperref[\detokenize{gen:module-tornado.gen}]{\sphinxcrossref{\sphinxcode{\sphinxupquote{tornado.gen}}}}} no longer leaks \sphinxcode{\sphinxupquote{StackContexts}} when a \sphinxcode{\sphinxupquote{@gen.engine}} wrapped
function is called repeatedly.

\item {} 
{\hyperref[\detokenize{locale:tornado.locale.get_supported_locales}]{\sphinxcrossref{\sphinxcode{\sphinxupquote{tornado.locale.get\_supported\_locales}}}}} no longer takes a meaningless
\sphinxcode{\sphinxupquote{cls}} argument.

\item {} 
\sphinxcode{\sphinxupquote{StackContext}} instances now have a deactivation callback that can be
used to prevent further propagation.

\item {} 
{\hyperref[\detokenize{testing:tornado.testing.AsyncTestCase.wait}]{\sphinxcrossref{\sphinxcode{\sphinxupquote{tornado.testing.AsyncTestCase.wait}}}}} now resets its timeout on each call.

\item {} 
\sphinxcode{\sphinxupquote{tornado.wsgi.WSGIApplication}} now parses arguments correctly on Python 3.

\item {} 
Exception handling on Python 3 has been improved; previously some exceptions
such as \sphinxhref{https://docs.python.org/3.6/library/exceptions.html\#UnicodeDecodeError}{\sphinxcode{\sphinxupquote{UnicodeDecodeError}}} would generate \sphinxcode{\sphinxupquote{TypeErrors}}

\end{itemize}


\subsection{What’s new in Tornado 2.2.1}
\label{\detokenize{releases/v2.2.1:what-s-new-in-tornado-2-2-1}}\label{\detokenize{releases/v2.2.1::doc}}

\subsubsection{Apr 23, 2012}
\label{\detokenize{releases/v2.2.1:apr-23-2012}}

\paragraph{Security fixes}
\label{\detokenize{releases/v2.2.1:security-fixes}}\begin{itemize}
\item {} 
{\hyperref[\detokenize{web:tornado.web.RequestHandler.set_header}]{\sphinxcrossref{\sphinxcode{\sphinxupquote{tornado.web.RequestHandler.set\_header}}}}} now properly sanitizes input
values to protect against header injection, response splitting, etc.
(it has always attempted to do this, but the check was incorrect).
Note that redirects, the most likely source of such bugs, are protected
by a separate check in {\hyperref[\detokenize{web:tornado.web.RequestHandler.redirect}]{\sphinxcrossref{\sphinxcode{\sphinxupquote{RequestHandler.redirect}}}}}.

\end{itemize}


\paragraph{Bug fixes}
\label{\detokenize{releases/v2.2.1:bug-fixes}}\begin{itemize}
\item {} 
Colored logging configuration in {\hyperref[\detokenize{options:module-tornado.options}]{\sphinxcrossref{\sphinxcode{\sphinxupquote{tornado.options}}}}} is compatible with
Python 3.2.3 (and 3.3).

\end{itemize}


\subsection{What’s new in Tornado 2.2}
\label{\detokenize{releases/v2.2.0:what-s-new-in-tornado-2-2}}\label{\detokenize{releases/v2.2.0::doc}}

\subsubsection{Jan 30, 2012}
\label{\detokenize{releases/v2.2.0:jan-30-2012}}

\paragraph{Highlights}
\label{\detokenize{releases/v2.2.0:highlights}}\begin{itemize}
\item {} 
Updated and expanded WebSocket support.

\item {} 
Improved compatibility in the Twisted/Tornado bridge.

\item {} 
Template errors now generate better stack traces.

\item {} 
Better exception handling in {\hyperref[\detokenize{gen:module-tornado.gen}]{\sphinxcrossref{\sphinxcode{\sphinxupquote{tornado.gen}}}}}.

\end{itemize}


\paragraph{Security fixes}
\label{\detokenize{releases/v2.2.0:security-fixes}}\begin{itemize}
\item {} 
\sphinxcode{\sphinxupquote{tornado.simple\_httpclient}} now disables SSLv2 in all cases.  Previously
SSLv2 would be allowed if the Python interpreter was linked against a
pre-1.0 version of OpenSSL.

\end{itemize}


\paragraph{Backwards-incompatible changes}
\label{\detokenize{releases/v2.2.0:backwards-incompatible-changes}}\begin{itemize}
\item {} 
{\hyperref[\detokenize{process:tornado.process.fork_processes}]{\sphinxcrossref{\sphinxcode{\sphinxupquote{tornado.process.fork\_processes}}}}} now raises \sphinxhref{https://docs.python.org/3.6/library/exceptions.html\#SystemExit}{\sphinxcode{\sphinxupquote{SystemExit}}} if all child
processes exit cleanly rather than returning \sphinxcode{\sphinxupquote{None}}.  The old behavior
was surprising and inconsistent with most of the documented examples
of this function (which did not check the return value).

\item {} 
On Python 2.6, \sphinxcode{\sphinxupquote{tornado.simple\_httpclient}} only supports SSLv3.  This
is because Python 2.6 does not expose a way to support both SSLv3 and TLSv1
without also supporting the insecure SSLv2.

\item {} 
{\hyperref[\detokenize{websocket:module-tornado.websocket}]{\sphinxcrossref{\sphinxcode{\sphinxupquote{tornado.websocket}}}}} no longer supports the older “draft 76” version
of the websocket protocol by default, although this version can
be enabled by overriding \sphinxcode{\sphinxupquote{tornado.websocket.WebSocketHandler.allow\_draft76}}.

\end{itemize}


\paragraph{\sphinxstyleliteralintitle{\sphinxupquote{tornado.httpclient}}}
\label{\detokenize{releases/v2.2.0:tornado-httpclient}}\begin{itemize}
\item {} 
\sphinxcode{\sphinxupquote{SimpleAsyncHTTPClient}} no longer hangs on \sphinxcode{\sphinxupquote{HEAD}} requests,
responses with no content, or empty \sphinxcode{\sphinxupquote{POST}}/\sphinxcode{\sphinxupquote{PUT}} response bodies.

\item {} 
\sphinxcode{\sphinxupquote{SimpleAsyncHTTPClient}} now supports 303 and 307 redirect codes.

\item {} 
\sphinxcode{\sphinxupquote{tornado.curl\_httpclient}} now accepts non-integer timeouts.

\item {} 
\sphinxcode{\sphinxupquote{tornado.curl\_httpclient}} now supports basic authentication with an
empty password.

\end{itemize}


\paragraph{\sphinxstyleliteralintitle{\sphinxupquote{tornado.httpserver}}}
\label{\detokenize{releases/v2.2.0:tornado-httpserver}}\begin{itemize}
\item {} 
{\hyperref[\detokenize{httpserver:tornado.httpserver.HTTPServer}]{\sphinxcrossref{\sphinxcode{\sphinxupquote{HTTPServer}}}}} with \sphinxcode{\sphinxupquote{xheaders=True}} will no longer accept
\sphinxcode{\sphinxupquote{X-Real-IP}} headers that don’t look like valid IP addresses.

\item {} 
{\hyperref[\detokenize{httpserver:tornado.httpserver.HTTPServer}]{\sphinxcrossref{\sphinxcode{\sphinxupquote{HTTPServer}}}}} now treats the \sphinxcode{\sphinxupquote{Connection}} request header as
case-insensitive.

\end{itemize}


\paragraph{\sphinxstyleliteralintitle{\sphinxupquote{tornado.ioloop}} and \sphinxstyleliteralintitle{\sphinxupquote{tornado.iostream}}}
\label{\detokenize{releases/v2.2.0:tornado-ioloop-and-tornado-iostream}}\begin{itemize}
\item {} 
\sphinxcode{\sphinxupquote{IOStream.write}} now works correctly when given an empty string.

\item {} 
\sphinxcode{\sphinxupquote{IOStream.read\_until}} (and \sphinxcode{\sphinxupquote{read\_until\_regex}}) now perform better
when there is a lot of buffered data, which improves peformance of
\sphinxcode{\sphinxupquote{SimpleAsyncHTTPClient}} when downloading files with lots of
chunks.

\item {} 
{\hyperref[\detokenize{iostream:tornado.iostream.SSLIOStream}]{\sphinxcrossref{\sphinxcode{\sphinxupquote{SSLIOStream}}}}} now works correctly when \sphinxcode{\sphinxupquote{ssl\_version}} is set to
a value other than \sphinxcode{\sphinxupquote{SSLv23}}.

\item {} 
Idle \sphinxcode{\sphinxupquote{IOLoops}} no longer wake up several times a second.

\item {} 
{\hyperref[\detokenize{ioloop:tornado.ioloop.PeriodicCallback}]{\sphinxcrossref{\sphinxcode{\sphinxupquote{tornado.ioloop.PeriodicCallback}}}}} no longer triggers duplicate callbacks
when stopped and started repeatedly.

\end{itemize}


\paragraph{\sphinxstyleliteralintitle{\sphinxupquote{tornado.template}}}
\label{\detokenize{releases/v2.2.0:tornado-template}}\begin{itemize}
\item {} 
Exceptions in template code will now show better stack traces that
reference lines from the original template file.

\item {} 
\sphinxcode{\sphinxupquote{\{\#}} and \sphinxcode{\sphinxupquote{\#\}}} can now be used for comments (and unlike the old
\sphinxcode{\sphinxupquote{\{\% comment \%\}}} directive, these can wrap other template directives).

\item {} 
Template directives may now span multiple lines.

\end{itemize}


\paragraph{\sphinxstyleliteralintitle{\sphinxupquote{tornado.web}}}
\label{\detokenize{releases/v2.2.0:tornado-web}}\begin{itemize}
\item {} 
Now behaves better when given malformed \sphinxcode{\sphinxupquote{Cookie}} headers

\item {} 
{\hyperref[\detokenize{web:tornado.web.RequestHandler.redirect}]{\sphinxcrossref{\sphinxcode{\sphinxupquote{RequestHandler.redirect}}}}} now has a \sphinxcode{\sphinxupquote{status}} argument to send
status codes other than 301 and 302.

\item {} 
New method {\hyperref[\detokenize{web:tornado.web.RequestHandler.on_finish}]{\sphinxcrossref{\sphinxcode{\sphinxupquote{RequestHandler.on\_finish}}}}} may be overridden for post-request
processing (as a counterpart to {\hyperref[\detokenize{web:tornado.web.RequestHandler.prepare}]{\sphinxcrossref{\sphinxcode{\sphinxupquote{RequestHandler.prepare}}}}})

\item {} 
{\hyperref[\detokenize{web:tornado.web.StaticFileHandler}]{\sphinxcrossref{\sphinxcode{\sphinxupquote{StaticFileHandler}}}}} now outputs \sphinxcode{\sphinxupquote{Content-Length}} and \sphinxcode{\sphinxupquote{Etag}} headers
on \sphinxcode{\sphinxupquote{HEAD}} requests.

\item {} 
{\hyperref[\detokenize{web:tornado.web.StaticFileHandler}]{\sphinxcrossref{\sphinxcode{\sphinxupquote{StaticFileHandler}}}}} now has overridable \sphinxcode{\sphinxupquote{get\_version}} and
\sphinxcode{\sphinxupquote{parse\_url\_path}} methods for use in subclasses.

\item {} 
{\hyperref[\detokenize{web:tornado.web.RequestHandler.static_url}]{\sphinxcrossref{\sphinxcode{\sphinxupquote{RequestHandler.static\_url}}}}} now takes an \sphinxcode{\sphinxupquote{include\_host}} parameter
(in addition to the old support for the \sphinxcode{\sphinxupquote{RequestHandler.include\_host}}
attribute).

\end{itemize}


\paragraph{\sphinxstyleliteralintitle{\sphinxupquote{tornado.websocket}}}
\label{\detokenize{releases/v2.2.0:tornado-websocket}}\begin{itemize}
\item {} 
Updated to support the latest version of the protocol, as finalized
in RFC 6455.

\item {} 
Many bugs were fixed in all supported protocol versions.

\item {} 
{\hyperref[\detokenize{websocket:module-tornado.websocket}]{\sphinxcrossref{\sphinxcode{\sphinxupquote{tornado.websocket}}}}} no longer supports the older “draft 76” version
of the websocket protocol by default, although this version can
be enabled by overriding \sphinxcode{\sphinxupquote{tornado.websocket.WebSocketHandler.allow\_draft76}}.

\item {} 
{\hyperref[\detokenize{websocket:tornado.websocket.WebSocketHandler.write_message}]{\sphinxcrossref{\sphinxcode{\sphinxupquote{WebSocketHandler.write\_message}}}}} now accepts a \sphinxcode{\sphinxupquote{binary}} argument
to send binary messages.

\item {} 
Subprotocols (i.e. the \sphinxcode{\sphinxupquote{Sec-WebSocket-Protocol}} header) are now supported;
see the {\hyperref[\detokenize{websocket:tornado.websocket.WebSocketHandler.select_subprotocol}]{\sphinxcrossref{\sphinxcode{\sphinxupquote{WebSocketHandler.select\_subprotocol}}}}} method for details.

\item {} 
\sphinxcode{\sphinxupquote{.WebSocketHandler.get\_websocket\_scheme}} can be used to select the
appropriate url scheme (\sphinxcode{\sphinxupquote{ws://}} or \sphinxcode{\sphinxupquote{wss://}}) in cases where
\sphinxcode{\sphinxupquote{HTTPRequest.protocol}} is not set correctly.

\end{itemize}


\paragraph{Other modules}
\label{\detokenize{releases/v2.2.0:other-modules}}\begin{itemize}
\item {} 
{\hyperref[\detokenize{auth:tornado.auth.TwitterMixin.authenticate_redirect}]{\sphinxcrossref{\sphinxcode{\sphinxupquote{tornado.auth.TwitterMixin.authenticate\_redirect}}}}} now takes a
\sphinxcode{\sphinxupquote{callback\_uri}} parameter.

\item {} 
{\hyperref[\detokenize{auth:tornado.auth.TwitterMixin.twitter_request}]{\sphinxcrossref{\sphinxcode{\sphinxupquote{tornado.auth.TwitterMixin.twitter\_request}}}}} now accepts both URLs and
partial paths (complete URLs are useful for the search API which follows
different patterns).

\item {} 
Exception handling in {\hyperref[\detokenize{gen:module-tornado.gen}]{\sphinxcrossref{\sphinxcode{\sphinxupquote{tornado.gen}}}}} has been improved.  It is now possible
to catch exceptions thrown by a \sphinxcode{\sphinxupquote{Task}}.

\item {} 
{\hyperref[\detokenize{netutil:tornado.netutil.bind_sockets}]{\sphinxcrossref{\sphinxcode{\sphinxupquote{tornado.netutil.bind\_sockets}}}}} now works when \sphinxcode{\sphinxupquote{getaddrinfo}} returns
duplicate addresses.

\item {} 
{\hyperref[\detokenize{twisted:module-tornado.platform.twisted}]{\sphinxcrossref{\sphinxcode{\sphinxupquote{tornado.platform.twisted}}}}} compatibility has been significantly improved.
Twisted version 11.1.0 is now supported in addition to 11.0.0.

\item {} 
{\hyperref[\detokenize{process:tornado.process.fork_processes}]{\sphinxcrossref{\sphinxcode{\sphinxupquote{tornado.process.fork\_processes}}}}} correctly reseeds the \sphinxhref{https://docs.python.org/3.6/library/random.html\#module-random}{\sphinxcode{\sphinxupquote{random}}} module
even when \sphinxhref{https://docs.python.org/3.6/library/os.html\#os.urandom}{\sphinxcode{\sphinxupquote{os.urandom}}} is not implemented.

\item {} 
{\hyperref[\detokenize{testing:tornado.testing.main}]{\sphinxcrossref{\sphinxcode{\sphinxupquote{tornado.testing.main}}}}} supports a new flag \sphinxcode{\sphinxupquote{-{-}exception\_on\_interrupt}},
which can be set to false to make \sphinxcode{\sphinxupquote{Ctrl-C}} kill the process more
reliably (at the expense of stack traces when it does so).

\item {} 
\sphinxcode{\sphinxupquote{tornado.version\_info}} is now a four-tuple so official releases can be
distinguished from development branches.

\end{itemize}


\subsection{What’s new in Tornado 2.1.1}
\label{\detokenize{releases/v2.1.1:what-s-new-in-tornado-2-1-1}}\label{\detokenize{releases/v2.1.1::doc}}

\subsubsection{Oct 4, 2011}
\label{\detokenize{releases/v2.1.1:oct-4-2011}}

\paragraph{Bug fixes}
\label{\detokenize{releases/v2.1.1:bug-fixes}}\begin{itemize}
\item {} 
Fixed handling of closed connections with the \sphinxcode{\sphinxupquote{epoll}} (i.e. Linux)
\sphinxcode{\sphinxupquote{IOLoop}}.  Previously, closed connections could be shut down too early,
which most often manifested as “Stream is closed” exceptions in
\sphinxcode{\sphinxupquote{SimpleAsyncHTTPClient}}.

\item {} 
Fixed a case in which chunked responses could be closed prematurely,
leading to truncated output.

\item {} 
\sphinxcode{\sphinxupquote{IOStream.connect}} now reports errors more consistently via logging
and the close callback (this affects e.g. connections to localhost
on FreeBSD).

\item {} 
\sphinxcode{\sphinxupquote{IOStream.read\_bytes}} again accepts both \sphinxcode{\sphinxupquote{int}} and \sphinxcode{\sphinxupquote{long}} arguments.

\item {} 
\sphinxcode{\sphinxupquote{PeriodicCallback}} no longer runs repeatedly when \sphinxcode{\sphinxupquote{IOLoop}} iterations
complete faster than the resolution of \sphinxcode{\sphinxupquote{time.time()}} (mainly a problem
on Windows).

\end{itemize}


\paragraph{Backwards-compatibility note}
\label{\detokenize{releases/v2.1.1:backwards-compatibility-note}}\begin{itemize}
\item {} 
Listening for \sphinxcode{\sphinxupquote{IOLoop.ERROR}} alone is no longer sufficient for detecting
closed connections on an otherwise unused socket.  \sphinxcode{\sphinxupquote{IOLoop.ERROR}} must
always be used in combination with \sphinxcode{\sphinxupquote{READ}} or \sphinxcode{\sphinxupquote{WRITE}}.

\end{itemize}


\subsection{What’s new in Tornado 2.1}
\label{\detokenize{releases/v2.1.0:what-s-new-in-tornado-2-1}}\label{\detokenize{releases/v2.1.0::doc}}

\subsubsection{Sep 20, 2011}
\label{\detokenize{releases/v2.1.0:sep-20-2011}}

\paragraph{Backwards-incompatible changes}
\label{\detokenize{releases/v2.1.0:backwards-incompatible-changes}}\begin{itemize}
\item {} 
Support for secure cookies written by pre-1.0 releases of Tornado has
been removed.  The {\hyperref[\detokenize{web:tornado.web.RequestHandler.get_secure_cookie}]{\sphinxcrossref{\sphinxcode{\sphinxupquote{RequestHandler.get\_secure\_cookie}}}}} method no longer
takes an \sphinxcode{\sphinxupquote{include\_name}} parameter.

\item {} 
The \sphinxcode{\sphinxupquote{debug}} application setting now causes stack traces to be displayed
in the browser on uncaught exceptions.  Since this may leak sensitive
information, debug mode is not recommended for public-facing servers.

\end{itemize}


\paragraph{Security fixes}
\label{\detokenize{releases/v2.1.0:security-fixes}}\begin{itemize}
\item {} 
Diginotar has been removed from the default CA certificates file used
by \sphinxcode{\sphinxupquote{SimpleAsyncHTTPClient}}.

\end{itemize}


\paragraph{New modules}
\label{\detokenize{releases/v2.1.0:new-modules}}\begin{itemize}
\item {} 
{\hyperref[\detokenize{gen:module-tornado.gen}]{\sphinxcrossref{\sphinxcode{\sphinxupquote{tornado.gen}}}}}:  A generator-based interface to simplify writing
asynchronous functions.

\item {} 
{\hyperref[\detokenize{netutil:module-tornado.netutil}]{\sphinxcrossref{\sphinxcode{\sphinxupquote{tornado.netutil}}}}}:  Parts of {\hyperref[\detokenize{httpserver:module-tornado.httpserver}]{\sphinxcrossref{\sphinxcode{\sphinxupquote{tornado.httpserver}}}}} have been extracted into
a new module for use with non-HTTP protocols.

\item {} 
{\hyperref[\detokenize{twisted:module-tornado.platform.twisted}]{\sphinxcrossref{\sphinxcode{\sphinxupquote{tornado.platform.twisted}}}}}:  A bridge between the Tornado IOLoop and the
Twisted Reactor, allowing code written for Twisted to be run on Tornado.

\item {} 
{\hyperref[\detokenize{process:module-tornado.process}]{\sphinxcrossref{\sphinxcode{\sphinxupquote{tornado.process}}}}}:  Multi-process mode has been improved, and can now restart
crashed child processes.  A new entry point has been added at
{\hyperref[\detokenize{process:tornado.process.fork_processes}]{\sphinxcrossref{\sphinxcode{\sphinxupquote{tornado.process.fork\_processes}}}}}, although
\sphinxcode{\sphinxupquote{tornado.httpserver.HTTPServer.start}} is still supported.

\end{itemize}


\paragraph{\sphinxstyleliteralintitle{\sphinxupquote{tornado.web}}}
\label{\detokenize{releases/v2.1.0:tornado-web}}\begin{itemize}
\item {} 
{\hyperref[\detokenize{web:tornado.web.RequestHandler.write_error}]{\sphinxcrossref{\sphinxcode{\sphinxupquote{tornado.web.RequestHandler.write\_error}}}}} replaces \sphinxcode{\sphinxupquote{get\_error\_html}} as the
preferred way to generate custom error pages (\sphinxcode{\sphinxupquote{get\_error\_html}} is still
supported, but deprecated)

\item {} 
In {\hyperref[\detokenize{web:tornado.web.Application}]{\sphinxcrossref{\sphinxcode{\sphinxupquote{tornado.web.Application}}}}}, handlers may be specified by
(fully-qualified) name instead of importing and passing the class object
itself.

\item {} 
It is now possible to use a custom subclass of \sphinxcode{\sphinxupquote{StaticFileHandler}}
with the \sphinxcode{\sphinxupquote{static\_handler\_class}} application setting, and this subclass
can override the behavior of the \sphinxcode{\sphinxupquote{static\_url}} method.

\item {} 
{\hyperref[\detokenize{web:tornado.web.StaticFileHandler}]{\sphinxcrossref{\sphinxcode{\sphinxupquote{StaticFileHandler}}}}} subclasses can now override
\sphinxcode{\sphinxupquote{get\_cache\_time}} to customize cache control behavior.

\item {} 
{\hyperref[\detokenize{web:tornado.web.RequestHandler.get_secure_cookie}]{\sphinxcrossref{\sphinxcode{\sphinxupquote{tornado.web.RequestHandler.get\_secure\_cookie}}}}} now has a \sphinxcode{\sphinxupquote{max\_age\_days}}
parameter to allow applications to override the default one-month expiration.

\item {} 
{\hyperref[\detokenize{web:tornado.web.RequestHandler.set_cookie}]{\sphinxcrossref{\sphinxcode{\sphinxupquote{set\_cookie}}}}} now accepts a \sphinxcode{\sphinxupquote{max\_age}} keyword
argument to set the \sphinxcode{\sphinxupquote{max-age}} cookie attribute (note underscore vs dash)

\item {} 
{\hyperref[\detokenize{web:tornado.web.RequestHandler.set_default_headers}]{\sphinxcrossref{\sphinxcode{\sphinxupquote{tornado.web.RequestHandler.set\_default\_headers}}}}} may be overridden to set
headers in a way that does not get reset during error handling.

\item {} 
{\hyperref[\detokenize{web:tornado.web.RequestHandler.add_header}]{\sphinxcrossref{\sphinxcode{\sphinxupquote{RequestHandler.add\_header}}}}} can now be used to set a header that can
appear multiple times in the response.

\item {} 
{\hyperref[\detokenize{web:tornado.web.RequestHandler.flush}]{\sphinxcrossref{\sphinxcode{\sphinxupquote{RequestHandler.flush}}}}} can now take a callback for flow control.

\item {} 
The \sphinxcode{\sphinxupquote{application/json}} content type can now be gzipped.

\item {} 
The cookie-signing functions are now accessible as static functions
\sphinxcode{\sphinxupquote{tornado.web.create\_signed\_value}} and \sphinxcode{\sphinxupquote{tornado.web.decode\_signed\_value}}.

\end{itemize}


\paragraph{\sphinxstyleliteralintitle{\sphinxupquote{tornado.httpserver}}}
\label{\detokenize{releases/v2.1.0:tornado-httpserver}}\begin{itemize}
\item {} 
To facilitate some advanced multi-process scenarios, \sphinxcode{\sphinxupquote{HTTPServer}}
has a new method \sphinxcode{\sphinxupquote{add\_sockets}}, and socket-opening code is
available separately as {\hyperref[\detokenize{netutil:tornado.netutil.bind_sockets}]{\sphinxcrossref{\sphinxcode{\sphinxupquote{tornado.netutil.bind\_sockets}}}}}.

\item {} 
The \sphinxcode{\sphinxupquote{cookies}} property is now available on \sphinxcode{\sphinxupquote{tornado.httpserver.HTTPRequest}}
(it is also available in its old location as a property of
{\hyperref[\detokenize{web:tornado.web.RequestHandler}]{\sphinxcrossref{\sphinxcode{\sphinxupquote{RequestHandler}}}}})

\item {} 
\sphinxcode{\sphinxupquote{tornado.httpserver.HTTPServer.bind}} now takes a backlog argument with the
same meaning as \sphinxcode{\sphinxupquote{socket.listen}}.

\item {} 
{\hyperref[\detokenize{httpserver:tornado.httpserver.HTTPServer}]{\sphinxcrossref{\sphinxcode{\sphinxupquote{HTTPServer}}}}} can now be run on a unix socket as well
as TCP.

\item {} 
Fixed exception at startup when \sphinxcode{\sphinxupquote{socket.AI\_ADDRCONFIG}} is not available,
as on Windows XP

\end{itemize}


\paragraph{\sphinxstyleliteralintitle{\sphinxupquote{IOLoop}} and \sphinxstyleliteralintitle{\sphinxupquote{IOStream}}}
\label{\detokenize{releases/v2.1.0:ioloop-and-iostream}}\begin{itemize}
\item {} 
{\hyperref[\detokenize{iostream:tornado.iostream.IOStream}]{\sphinxcrossref{\sphinxcode{\sphinxupquote{IOStream}}}}} performance has been improved, especially for
small synchronous requests.

\item {} 
New methods \sphinxcode{\sphinxupquote{tornado.iostream.IOStream.read\_until\_close}} and
\sphinxcode{\sphinxupquote{tornado.iostream.IOStream.read\_until\_regex}}.

\item {} 
\sphinxcode{\sphinxupquote{IOStream.read\_bytes}} and \sphinxcode{\sphinxupquote{IOStream.read\_until\_close}} now take a
\sphinxcode{\sphinxupquote{streaming\_callback}} argument to return data as it is received rather
than all at once.

\item {} 
{\hyperref[\detokenize{ioloop:tornado.ioloop.IOLoop.add_timeout}]{\sphinxcrossref{\sphinxcode{\sphinxupquote{IOLoop.add\_timeout}}}}} now accepts \sphinxhref{https://docs.python.org/3.6/library/datetime.html\#datetime.timedelta}{\sphinxcode{\sphinxupquote{datetime.timedelta}}} objects in addition
to absolute timestamps.

\item {} 
{\hyperref[\detokenize{ioloop:tornado.ioloop.PeriodicCallback}]{\sphinxcrossref{\sphinxcode{\sphinxupquote{PeriodicCallback}}}}} now sticks to the specified period
instead of creeping later due to accumulated errors.

\item {} 
{\hyperref[\detokenize{ioloop:tornado.ioloop.IOLoop}]{\sphinxcrossref{\sphinxcode{\sphinxupquote{tornado.ioloop.IOLoop}}}}} and {\hyperref[\detokenize{httpclient:tornado.httpclient.HTTPClient}]{\sphinxcrossref{\sphinxcode{\sphinxupquote{tornado.httpclient.HTTPClient}}}}} now have
\sphinxcode{\sphinxupquote{close()}} methods that should be used in applications that create
and destroy many of these objects.

\item {} 
{\hyperref[\detokenize{ioloop:tornado.ioloop.IOLoop.install}]{\sphinxcrossref{\sphinxcode{\sphinxupquote{IOLoop.install}}}}} can now be used to use a custom subclass of IOLoop
as the singleton without monkey-patching.

\item {} 
{\hyperref[\detokenize{iostream:tornado.iostream.IOStream}]{\sphinxcrossref{\sphinxcode{\sphinxupquote{IOStream}}}}} should now always call the close callback
instead of the connect callback on a connection error.

\item {} 
The {\hyperref[\detokenize{iostream:tornado.iostream.IOStream}]{\sphinxcrossref{\sphinxcode{\sphinxupquote{IOStream}}}}} close callback will no longer be called while there
are pending read callbacks that can be satisfied with buffered data.

\end{itemize}


\paragraph{\sphinxstyleliteralintitle{\sphinxupquote{tornado.simple\_httpclient}}}
\label{\detokenize{releases/v2.1.0:tornado-simple-httpclient}}\begin{itemize}
\item {} 
Now supports client SSL certificates with the \sphinxcode{\sphinxupquote{client\_key}} and
\sphinxcode{\sphinxupquote{client\_cert}} parameters to {\hyperref[\detokenize{httpclient:tornado.httpclient.HTTPRequest}]{\sphinxcrossref{\sphinxcode{\sphinxupquote{tornado.httpclient.HTTPRequest}}}}}

\item {} 
Now takes a maximum buffer size, to allow reading files larger than 100MB

\item {} 
Now works with HTTP 1.0 servers that don’t send a Content-Length header

\item {} 
The \sphinxcode{\sphinxupquote{allow\_nonstandard\_methods}} flag on HTTP client requests now
permits methods other than \sphinxcode{\sphinxupquote{POST}} and \sphinxcode{\sphinxupquote{PUT}} to contain bodies.

\item {} 
Fixed file descriptor leaks and multiple callback invocations in
\sphinxcode{\sphinxupquote{SimpleAsyncHTTPClient}}

\item {} 
No longer consumes extra connection resources when following redirects.

\item {} 
Now works with buggy web servers that separate headers with \sphinxcode{\sphinxupquote{\textbackslash{}n}} instead
of \sphinxcode{\sphinxupquote{\textbackslash{}r\textbackslash{}n\textbackslash{}r\textbackslash{}n}}.

\item {} 
Now sets \sphinxcode{\sphinxupquote{response.request\_time}} correctly.

\item {} 
Connect timeouts now work correctly.

\end{itemize}


\paragraph{Other modules}
\label{\detokenize{releases/v2.1.0:other-modules}}\begin{itemize}
\item {} 
{\hyperref[\detokenize{auth:tornado.auth.OpenIdMixin}]{\sphinxcrossref{\sphinxcode{\sphinxupquote{tornado.auth.OpenIdMixin}}}}} now uses the correct realm when the
callback URI is on a different domain.

\item {} 
{\hyperref[\detokenize{autoreload:module-tornado.autoreload}]{\sphinxcrossref{\sphinxcode{\sphinxupquote{tornado.autoreload}}}}} has a new command-line interface which can be used
to wrap any script.  This replaces the \sphinxcode{\sphinxupquote{-{-}autoreload}} argument to
{\hyperref[\detokenize{testing:tornado.testing.main}]{\sphinxcrossref{\sphinxcode{\sphinxupquote{tornado.testing.main}}}}} and is more robust against syntax errors.

\item {} 
{\hyperref[\detokenize{autoreload:tornado.autoreload.watch}]{\sphinxcrossref{\sphinxcode{\sphinxupquote{tornado.autoreload.watch}}}}} can be used to watch files other than
the sources of imported modules.

\item {} 
\sphinxcode{\sphinxupquote{tornado.database.Connection}} has new variants of \sphinxcode{\sphinxupquote{execute}} and
\sphinxcode{\sphinxupquote{executemany}} that return the number of rows affected instead of
the last inserted row id.

\item {} 
{\hyperref[\detokenize{locale:tornado.locale.load_translations}]{\sphinxcrossref{\sphinxcode{\sphinxupquote{tornado.locale.load\_translations}}}}} now accepts any properly-formatted
locale name, not just those in the predefined \sphinxcode{\sphinxupquote{LOCALE\_NAMES}} list.

\item {} 
{\hyperref[\detokenize{options:tornado.options.define}]{\sphinxcrossref{\sphinxcode{\sphinxupquote{tornado.options.define}}}}} now takes a \sphinxcode{\sphinxupquote{group}} parameter to group options
in \sphinxcode{\sphinxupquote{-{-}help}} output.

\item {} 
Template loaders now take a \sphinxcode{\sphinxupquote{namespace}} constructor argument to add
entries to the template namespace.

\item {} 
{\hyperref[\detokenize{websocket:module-tornado.websocket}]{\sphinxcrossref{\sphinxcode{\sphinxupquote{tornado.websocket}}}}} now supports the latest (“hybi-10”) version of the
protocol (the old version, “hixie-76” is still supported; the correct
version is detected automatically).

\item {} 
{\hyperref[\detokenize{websocket:module-tornado.websocket}]{\sphinxcrossref{\sphinxcode{\sphinxupquote{tornado.websocket}}}}} now works on Python 3

\end{itemize}


\paragraph{Bug fixes}
\label{\detokenize{releases/v2.1.0:bug-fixes}}\begin{itemize}
\item {} 
Windows support has been improved.  Windows is still not an officially
supported platform, but the test suite now passes and
{\hyperref[\detokenize{autoreload:module-tornado.autoreload}]{\sphinxcrossref{\sphinxcode{\sphinxupquote{tornado.autoreload}}}}} works.

\item {} 
Uploading files whose names contain special characters will now work.

\item {} 
Cookie values containing special characters are now properly quoted
and unquoted.

\item {} 
Multi-line headers are now supported.

\item {} 
Repeated Content-Length headers (which may be added by certain proxies)
are now supported in {\hyperref[\detokenize{httpserver:tornado.httpserver.HTTPServer}]{\sphinxcrossref{\sphinxcode{\sphinxupquote{HTTPServer}}}}}.

\item {} 
Unicode string literals now work in template expressions.

\item {} 
The template \sphinxcode{\sphinxupquote{\{\% module \%\}}} directive now works even if applications
use a template variable named \sphinxcode{\sphinxupquote{modules}}.

\item {} 
Requests with “Expect: 100-continue” now work on python 3

\end{itemize}


\subsection{What’s new in Tornado 2.0}
\label{\detokenize{releases/v2.0.0:what-s-new-in-tornado-2-0}}\label{\detokenize{releases/v2.0.0::doc}}

\subsubsection{Jun 21, 2011}
\label{\detokenize{releases/v2.0.0:jun-21-2011}}
\begin{sphinxVerbatim}[commandchars=\\\{\}]
\PYG{n}{Major} \PYG{n}{changes}\PYG{p}{:}
\PYG{o}{*} \PYG{n}{Template} \PYG{n}{output} \PYG{o+ow}{is} \PYG{n}{automatically} \PYG{n}{escaped} \PYG{n}{by} \PYG{n}{default}\PYG{p}{;} \PYG{n}{see} \PYG{n}{backwards}
  \PYG{n}{compatibility} \PYG{n}{note} \PYG{n}{below}\PYG{o}{.}
\PYG{o}{*} \PYG{n}{The} \PYG{n}{default} \PYG{n}{AsyncHTTPClient} \PYG{n}{implementation} \PYG{o+ow}{is} \PYG{n}{now} \PYG{n}{simple\PYGZus{}httpclient}\PYG{o}{.}
\PYG{o}{*} \PYG{n}{Python} \PYG{l+m+mf}{3.2} \PYG{o+ow}{is} \PYG{n}{now} \PYG{n}{supported}\PYG{o}{.}

\PYG{n}{Backwards} \PYG{n}{compatibility}\PYG{p}{:}
\PYG{o}{*} \PYG{n}{Template} \PYG{n}{autoescaping} \PYG{o+ow}{is} \PYG{n}{enabled} \PYG{n}{by} \PYG{n}{default}\PYG{o}{.}  \PYG{n}{Applications} \PYG{n}{upgrading} \PYG{k+kn}{from}
  \PYG{n+nn}{a} \PYG{n}{previous} \PYG{n}{release} \PYG{n}{of} \PYG{n}{Tornado} \PYG{n}{must} \PYG{n}{either} \PYG{n}{disable} \PYG{n}{autoescaping} \PYG{o+ow}{or} \PYG{n}{adapt}
  \PYG{n}{their} \PYG{n}{templates} \PYG{n}{to} \PYG{n}{work} \PYG{k}{with} \PYG{n}{it}\PYG{o}{.}  \PYG{n}{For} \PYG{n}{most} \PYG{n}{applications}\PYG{p}{,} \PYG{n}{the} \PYG{n}{simplest}
  \PYG{n}{way} \PYG{n}{to} \PYG{n}{do} \PYG{n}{this} \PYG{o+ow}{is} \PYG{n}{to} \PYG{k}{pass} \PYG{n}{autoescape}\PYG{o}{=}\PYG{k+kc}{None} \PYG{n}{to} \PYG{n}{the} \PYG{n}{Application} \PYG{n}{constructor}\PYG{o}{.}
  \PYG{n}{Note} \PYG{n}{that} \PYG{n}{this} \PYG{n}{affects} \PYG{n}{certain} \PYG{n}{built}\PYG{o}{\PYGZhy{}}\PYG{o+ow}{in} \PYG{n}{methods}\PYG{p}{,} \PYG{n}{e}\PYG{o}{.}\PYG{n}{g}\PYG{o}{.} \PYG{n}{xsrf\PYGZus{}form\PYGZus{}html}
  \PYG{o+ow}{and} \PYG{n}{linkify}\PYG{p}{,} \PYG{n}{which} \PYG{n}{must} \PYG{n}{now} \PYG{n}{be} \PYG{n}{called} \PYG{k}{with} \PYG{p}{\PYGZob{}}\PYG{o}{\PYGZpc{}} \PYG{n}{raw} \PYG{o}{\PYGZpc{}}\PYG{p}{\PYGZcb{}} \PYG{n}{instead} \PYG{n}{of} \PYG{p}{\PYGZob{}}\PYG{p}{\PYGZcb{}}
\PYG{o}{*} \PYG{n}{Applications} \PYG{n}{that} \PYG{n}{wish} \PYG{n}{to} \PYG{k}{continue} \PYG{n}{using} \PYG{n}{curl\PYGZus{}httpclient} \PYG{n}{instead} \PYG{n}{of}
  \PYG{n}{simple\PYGZus{}httpclient} \PYG{n}{may} \PYG{n}{do} \PYG{n}{so} \PYG{n}{by} \PYG{n}{calling}
    \PYG{n}{AsyncHTTPClient}\PYG{o}{.}\PYG{n}{configure}\PYG{p}{(}\PYG{l+s+s2}{\PYGZdq{}}\PYG{l+s+s2}{tornado.curl\PYGZus{}httpclient.CurlAsyncHTTPClient}\PYG{l+s+s2}{\PYGZdq{}}\PYG{p}{)}
  \PYG{n}{at} \PYG{n}{the} \PYG{n}{beginning} \PYG{n}{of} \PYG{n}{the} \PYG{n}{process}\PYG{o}{.}  \PYG{n}{Users} \PYG{n}{of} \PYG{n}{Python} \PYG{l+m+mf}{2.5} \PYG{n}{will} \PYG{n}{probably} \PYG{n}{want}
  \PYG{n}{to} \PYG{n}{use} \PYG{n}{curl\PYGZus{}httpclient} \PYG{k}{as} \PYG{n}{simple\PYGZus{}httpclient} \PYG{n}{only} \PYG{n}{supports} \PYG{n}{ssl} \PYG{n}{on} \PYG{n}{Python} \PYG{l+m+mf}{2.6}\PYG{o}{+}\PYG{o}{.}
\PYG{o}{*} \PYG{n}{Python} \PYG{l+m+mi}{3} \PYG{n}{compatibility} \PYG{n}{involved} \PYG{n}{many} \PYG{n}{changes} \PYG{n}{throughout} \PYG{n}{the} \PYG{n}{codebase}\PYG{p}{,}
  \PYG{n}{so} \PYG{n}{users} \PYG{n}{are} \PYG{n}{encouraged} \PYG{n}{to} \PYG{n}{test} \PYG{n}{their} \PYG{n}{applications} \PYG{n}{more} \PYG{n}{thoroughly} \PYG{n}{than}
  \PYG{n}{usual} \PYG{n}{when} \PYG{n}{upgrading} \PYG{n}{to} \PYG{n}{this} \PYG{n}{release}\PYG{o}{.}

\PYG{n}{Other} \PYG{n}{changes} \PYG{o+ow}{in} \PYG{n}{this} \PYG{n}{release}\PYG{p}{:}
\PYG{o}{*} \PYG{n}{Templates} \PYG{n}{support} \PYG{n}{several} \PYG{n}{new} \PYG{n}{directives}\PYG{p}{:}
  \PYG{o}{\PYGZhy{}} \PYG{p}{\PYGZob{}}\PYG{o}{\PYGZpc{}} \PYG{n}{autoescape} \PYG{o}{.}\PYG{o}{.}\PYG{o}{.}\PYG{o}{\PYGZpc{}}\PYG{p}{\PYGZcb{}} \PYG{n}{to} \PYG{n}{control} \PYG{n}{escaping} \PYG{n}{behavior}
  \PYG{o}{\PYGZhy{}} \PYG{p}{\PYGZob{}}\PYG{o}{\PYGZpc{}} \PYG{n}{raw} \PYG{o}{.}\PYG{o}{.}\PYG{o}{.} \PYG{o}{\PYGZpc{}}\PYG{p}{\PYGZcb{}} \PYG{k}{for} \PYG{n}{unescaped} \PYG{n}{output}
  \PYG{o}{\PYGZhy{}} \PYG{p}{\PYGZob{}}\PYG{o}{\PYGZpc{}} \PYG{n}{module} \PYG{o}{.}\PYG{o}{.}\PYG{o}{.} \PYG{o}{\PYGZpc{}}\PYG{p}{\PYGZcb{}} \PYG{k}{for} \PYG{n}{calling} \PYG{n}{UIModules}
\PYG{o}{*} \PYG{p}{\PYGZob{}}\PYG{o}{\PYGZpc{}} \PYG{n}{module} \PYG{n}{Template}\PYG{p}{(}\PYG{n}{path}\PYG{p}{,} \PYG{o}{*}\PYG{o}{*}\PYG{n}{kwargs}\PYG{p}{)} \PYG{o}{\PYGZpc{}}\PYG{p}{\PYGZcb{}} \PYG{n}{may} \PYG{n}{now} \PYG{n}{be} \PYG{n}{used} \PYG{n}{to} \PYG{n}{call} \PYG{n}{another}
  \PYG{n}{template} \PYG{k}{with} \PYG{n}{an} \PYG{n}{independent} \PYG{n}{namespace}
\PYG{o}{*} \PYG{n}{All} \PYG{n}{IOStream} \PYG{n}{callbacks} \PYG{n}{are} \PYG{n}{now} \PYG{n}{run} \PYG{n}{directly} \PYG{n}{on} \PYG{n}{the} \PYG{n}{IOLoop} \PYG{n}{via} \PYG{n}{add\PYGZus{}callback}\PYG{o}{.}
\PYG{o}{*} \PYG{n}{HTTPServer} \PYG{n}{now} \PYG{n}{supports} \PYG{n}{IPv6} \PYG{n}{where} \PYG{n}{available}\PYG{o}{.}  \PYG{n}{To} \PYG{n}{disable}\PYG{p}{,} \PYG{k}{pass}
  \PYG{n}{family}\PYG{o}{=}\PYG{n}{socket}\PYG{o}{.}\PYG{n}{AF\PYGZus{}INET} \PYG{n}{to} \PYG{n}{HTTPServer}\PYG{o}{.}\PYG{n}{bind}\PYG{p}{(}\PYG{p}{)}\PYG{o}{.}
\PYG{o}{*} \PYG{n}{HTTPClient} \PYG{n}{now} \PYG{n}{supports} \PYG{n}{IPv6}\PYG{p}{,} \PYG{n}{configurable} \PYG{n}{via} \PYG{n}{allow\PYGZus{}ipv6}\PYG{o}{=}\PYG{n+nb}{bool} \PYG{n}{on} \PYG{n}{the}
  \PYG{n}{HTTPRequest}\PYG{o}{.}  \PYG{n}{allow\PYGZus{}ipv6} \PYG{n}{defaults} \PYG{n}{to} \PYG{n}{false} \PYG{n}{on} \PYG{n}{simple\PYGZus{}httpclient} \PYG{o+ow}{and} \PYG{n}{true}
  \PYG{n}{on} \PYG{n}{curl\PYGZus{}httpclient}\PYG{o}{.}
\PYG{o}{*} \PYG{n}{RequestHandlers} \PYG{n}{can} \PYG{n}{use} \PYG{n}{an} \PYG{n}{encoding} \PYG{n}{other} \PYG{n}{than} \PYG{n}{utf}\PYG{o}{\PYGZhy{}}\PYG{l+m+mi}{8} \PYG{k}{for} \PYG{n}{query} \PYG{n}{parameters}
  \PYG{n}{by} \PYG{n}{overriding} \PYG{n}{decode\PYGZus{}argument}\PYG{p}{(}\PYG{p}{)}
\PYG{o}{*} \PYG{n}{Performance} \PYG{n}{improvements}\PYG{p}{,} \PYG{n}{especially} \PYG{k}{for} \PYG{n}{applications} \PYG{n}{that} \PYG{n}{use} \PYG{n}{a} \PYG{n}{lot} \PYG{n}{of}
  \PYG{n}{IOLoop} \PYG{n}{timeouts}
\PYG{o}{*} \PYG{n}{HTTP} \PYG{n}{OPTIONS} \PYG{n}{method} \PYG{n}{no} \PYG{n}{longer} \PYG{n}{requires} \PYG{n}{an} \PYG{n}{XSRF} \PYG{n}{token}\PYG{o}{.}
\PYG{o}{*} \PYG{n}{JSON} \PYG{n}{output} \PYG{p}{(}\PYG{n}{RequestHandler}\PYG{o}{.}\PYG{n}{write}\PYG{p}{(}\PYG{n+nb}{dict}\PYG{p}{)}\PYG{p}{)} \PYG{n}{now} \PYG{n}{sets} \PYG{n}{Content}\PYG{o}{\PYGZhy{}}\PYG{n}{Type} \PYG{n}{to}
  \PYG{n}{application}\PYG{o}{/}\PYG{n}{json}
\PYG{o}{*} \PYG{n}{Etag} \PYG{n}{computation} \PYG{n}{can} \PYG{n}{now} \PYG{n}{be} \PYG{n}{customized} \PYG{o+ow}{or} \PYG{n}{disabled} \PYG{n}{by} \PYG{n}{overriding}
  \PYG{n}{RequestHandler}\PYG{o}{.}\PYG{n}{compute\PYGZus{}etag}
\PYG{o}{*} \PYG{n}{USE\PYGZus{}SIMPLE\PYGZus{}HTTPCLIENT} \PYG{n}{environment} \PYG{n}{variable} \PYG{o+ow}{is} \PYG{n}{no} \PYG{n}{longer} \PYG{n}{supported}\PYG{o}{.}
  \PYG{n}{Use} \PYG{n}{AsyncHTTPClient}\PYG{o}{.}\PYG{n}{configure} \PYG{n}{instead}\PYG{o}{.}
\end{sphinxVerbatim}


\subsection{What’s new in Tornado 1.2.1}
\label{\detokenize{releases/v1.2.1:what-s-new-in-tornado-1-2-1}}\label{\detokenize{releases/v1.2.1::doc}}

\subsubsection{Mar 3, 2011}
\label{\detokenize{releases/v1.2.1:mar-3-2011}}
\begin{sphinxVerbatim}[commandchars=\\\{\}]
\PYG{n}{We} \PYG{n}{are} \PYG{n}{pleased} \PYG{n}{to} \PYG{n}{announce} \PYG{n}{the} \PYG{n}{release} \PYG{n}{of} \PYG{n}{Tornado} \PYG{l+m+mf}{1.2}\PYG{o}{.}\PYG{l+m+mi}{1}\PYG{p}{,} \PYG{n}{available} \PYG{k+kn}{from}
\PYG{n+nn}{https}\PYG{p}{:}\PYG{o}{/}\PYG{o}{/}\PYG{n}{github}\PYG{o}{.}\PYG{n}{com}\PYG{o}{/}\PYG{n}{downloads}\PYG{o}{/}\PYG{n}{facebook}\PYG{o}{/}\PYG{n}{tornado}\PYG{o}{/}\PYG{n}{tornado}\PYG{o}{\PYGZhy{}}\PYG{l+m+mf}{1.2}\PYG{o}{.}\PYG{l+m+mf}{1.}\PYG{n}{tar}\PYG{o}{.}\PYG{n}{gz}

\PYG{n}{This} \PYG{n}{release} \PYG{n}{contains} \PYG{n}{only} \PYG{n}{two} \PYG{n}{small} \PYG{n}{changes} \PYG{n}{relative} \PYG{n}{to} \PYG{n}{version} \PYG{l+m+mf}{1.2}\PYG{p}{:}
\PYG{o}{*} \PYG{n}{FacebookGraphMixin} \PYG{n}{has} \PYG{n}{been} \PYG{n}{updated} \PYG{n}{to} \PYG{n}{work} \PYG{k}{with} \PYG{n}{a} \PYG{n}{recent} \PYG{n}{change} \PYG{n}{to} \PYG{n}{the}
  \PYG{n}{Facebook} \PYG{n}{API}\PYG{o}{.}
\PYG{o}{*} \PYG{n}{Running} \PYG{l+s+s2}{\PYGZdq{}}\PYG{l+s+s2}{setup.py install}\PYG{l+s+s2}{\PYGZdq{}} \PYG{n}{will} \PYG{n}{no} \PYG{n}{longer} \PYG{n}{attempt} \PYG{n}{to} \PYG{n}{automatically}
  \PYG{n}{install} \PYG{n}{pycurl}\PYG{o}{.}  \PYG{n}{This} \PYG{n}{wasn}\PYG{l+s+s1}{\PYGZsq{}}\PYG{l+s+s1}{t working well on platforms where the best way}
  \PYG{n}{to} \PYG{n}{install} \PYG{n}{pycurl} \PYG{o+ow}{is} \PYG{n}{via} \PYG{n}{something} \PYG{n}{like} \PYG{n}{apt}\PYG{o}{\PYGZhy{}}\PYG{n}{get} \PYG{n}{instead} \PYG{n}{of} \PYG{n}{easy\PYGZus{}install}\PYG{o}{.}

\PYG{n}{This} \PYG{o+ow}{is} \PYG{n}{an} \PYG{n}{important} \PYG{n}{upgrade} \PYG{k}{if} \PYG{n}{you} \PYG{n}{are} \PYG{n}{using} \PYG{n}{FacebookGraphMixin}\PYG{p}{,} \PYG{n}{but}
\PYG{n}{otherwise} \PYG{n}{it} \PYG{n}{can} \PYG{n}{be} \PYG{n}{safely} \PYG{n}{ignored}\PYG{o}{.}
\end{sphinxVerbatim}


\subsection{What’s new in Tornado 1.2}
\label{\detokenize{releases/v1.2.0:what-s-new-in-tornado-1-2}}\label{\detokenize{releases/v1.2.0::doc}}

\subsubsection{Feb 20, 2011}
\label{\detokenize{releases/v1.2.0:feb-20-2011}}
\begin{sphinxVerbatim}[commandchars=\\\{\}]
We are pleased to announce the release of Tornado 1.2, available from
https://github.com/downloads/facebook/tornado/tornado\PYGZhy{}1.2.tar.gz

Backwards compatibility notes:
* This release includes the backwards\PYGZhy{}incompatible security change from
  version 1.1.1.  Users upgrading from 1.1 or earlier should read the
  release notes from that release:
  http://groups.google.com/group/python\PYGZhy{}tornado/browse\PYGZus{}thread/thread/b36191c781580cde
* StackContexts that do something other than catch exceptions may need to
  be modified to be reentrant.
  https://github.com/tornadoweb/tornado/commit/7a7e24143e77481d140fb5579bc67e4c45cbcfad
* When XSRF tokens are used, the token must also be present on PUT and
  DELETE requests (anything but GET and HEAD)

New features:
* A new HTTP client implementation is available in the module
  tornado.simple\PYGZus{}httpclient.  This HTTP client does not depend on pycurl.
  It has not yet been tested extensively in production, but is intended
  to eventually replace the pycurl\PYGZhy{}based HTTP client in a future release of
  Tornado.  To transparently replace tornado.httpclient.AsyncHTTPClient with
  this new implementation, you can set the environment variable
  USE\PYGZus{}SIMPLE\PYGZus{}HTTPCLIENT=1 (note that the next release of Tornado will
  likely include a different way to select HTTP client implementations)
* Request logging is now done by the Application rather than the
  RequestHandler.  Logging behavior may be customized by either overriding
  Application.log\PYGZus{}request in a subclass or by passing log\PYGZus{}function
  as an Application setting
* Application.listen(port): Convenience method as an alternative to
  explicitly creating an HTTPServer
* tornado.escape.linkify(): Wrap urls in \PYGZlt{}a\PYGZgt{} tags
* RequestHandler.create\PYGZus{}signed\PYGZus{}value(): Create signatures like the
  secure\PYGZus{}cookie methods without setting cookies.
* tornado.testing.get\PYGZus{}unused\PYGZus{}port(): Returns a port selected in the same
  way as inAsyncHTTPTestCase
* AsyncHTTPTestCase.fetch(): Convenience method for synchronous fetches
* IOLoop.set\PYGZus{}blocking\PYGZus{}signal\PYGZus{}threshold(): Set a callback to be run when
  the IOLoop is blocked.
* IOStream.connect(): Asynchronously connect a client socket
* AsyncHTTPClient.handle\PYGZus{}callback\PYGZus{}exception(): May be overridden
  in subclass for custom error handling
* httpclient.HTTPRequest has two new keyword arguments, validate\PYGZus{}cert and
  ca\PYGZus{}certs. Setting validate\PYGZus{}cert=False will disable all certificate checks
  when fetching https urls.  ca\PYGZus{}certs may be set to a filename containing
  trusted certificate authorities (defaults will be used if this is
  unspecified)
* HTTPRequest.get\PYGZus{}ssl\PYGZus{}certificate(): Returns the client\PYGZsq{}s SSL certificate
  (if client certificates were requested in the server\PYGZsq{}s ssl\PYGZus{}options
* StaticFileHandler can be configured to return a default file (e.g.
  index.html) when a directory is requested
* Template directives of the form \PYGZdq{}\PYGZob{}\PYGZpc{} from x import y \PYGZpc{}\PYGZcb{}\PYGZdq{} are now
  supported (in addition to the existing support for \PYGZdq{}\PYGZob{}\PYGZpc{} import x
  \PYGZpc{}\PYGZcb{}\PYGZdq{}
* FacebookGraphMixin.get\PYGZus{}authenticated\PYGZus{}user now accepts a new
  parameter \PYGZsq{}extra\PYGZus{}fields\PYGZsq{} which may be used to request additional
  information about the user

Bug fixes:
* auth: Fixed KeyError with Facebook offline\PYGZus{}access
* auth: Uses request.uri instead of request.path as the default redirect
  so that parameters are preserved.
* escape: xhtml\PYGZus{}escape() now returns a unicode string, not
  utf8\PYGZhy{}encoded bytes
* ioloop: Callbacks added with add\PYGZus{}callback are now run in the order they
  were added
* ioloop: PeriodicCallback.stop can now be called from inside the callback.
* iostream: Fixed several bugs in SSLIOStream
* iostream: Detect when the other side has closed the connection even with
  the select()\PYGZhy{}based IOLoop
* iostream: read\PYGZus{}bytes(0) now works as expected
* iostream: Fixed bug when writing large amounts of data on windows
* iostream: Fixed infinite loop that could occur with unhandled exceptions
* httpclient: Fix bugs when some requests use proxies and others don\PYGZsq{}t
* httpserver: HTTPRequest.protocol is now set correctly when using the
  built\PYGZhy{}in SSL support
* httpserver: When using multiple processes, the standard library\PYGZsq{}s
  random number generator is re\PYGZhy{}seeded in each child process
* httpserver: With xheaders enabled, X\PYGZhy{}Forwarded\PYGZhy{}Proto is supported as an
  alternative to X\PYGZhy{}Scheme
* httpserver: Fixed bugs in multipart/form\PYGZhy{}data parsing
* locale: format\PYGZus{}date() now behaves sanely with dates in the future
* locale: Updates to the language list
* stack\PYGZus{}context: Fixed bug with contexts leaking through reused IOStreams
* stack\PYGZus{}context: Simplified semantics and improved performance
* web: The order of css\PYGZus{}files from UIModules is now preserved
* web: Fixed error with default\PYGZus{}host redirect
* web: StaticFileHandler works when os.path.sep != \PYGZsq{}/\PYGZsq{} (i.e. on Windows)
* web: Fixed a caching\PYGZhy{}related bug in StaticFileHandler when a file\PYGZsq{}s
  timestamp has changed but its contents have not.
* web: Fixed bugs with HEAD requests and e.g. Etag headers
* web: Fix bugs when different handlers have different static\PYGZus{}paths
* web: @removeslash will no longer cause a redirect loop when applied to the
  root path
* websocket: Now works over SSL
* websocket: Improved compatibility with proxies

Many thanks to everyone who contributed patches, bug reports, and feedback
that went into this release!

\PYGZhy{}Ben
\end{sphinxVerbatim}


\subsection{What’s new in Tornado 1.1.1}
\label{\detokenize{releases/v1.1.1:what-s-new-in-tornado-1-1-1}}\label{\detokenize{releases/v1.1.1::doc}}

\subsubsection{Feb 8, 2011}
\label{\detokenize{releases/v1.1.1:feb-8-2011}}
\begin{sphinxVerbatim}[commandchars=\\\{\}]
\PYG{n}{Tornado} \PYG{l+m+mf}{1.1}\PYG{o}{.}\PYG{l+m+mi}{1} \PYG{o+ow}{is} \PYG{n}{a} \PYG{n}{BACKWARDS}\PYG{o}{\PYGZhy{}}\PYG{n}{INCOMPATIBLE} \PYG{n}{security} \PYG{n}{update} \PYG{n}{that} \PYG{n}{fixes} \PYG{n}{an}
\PYG{n}{XSRF} \PYG{n}{vulnerability}\PYG{o}{.}  \PYG{n}{It} \PYG{o+ow}{is} \PYG{n}{available} \PYG{n}{at}
\PYG{n}{https}\PYG{p}{:}\PYG{o}{/}\PYG{o}{/}\PYG{n}{github}\PYG{o}{.}\PYG{n}{com}\PYG{o}{/}\PYG{n}{downloads}\PYG{o}{/}\PYG{n}{facebook}\PYG{o}{/}\PYG{n}{tornado}\PYG{o}{/}\PYG{n}{tornado}\PYG{o}{\PYGZhy{}}\PYG{l+m+mf}{1.1}\PYG{o}{.}\PYG{l+m+mf}{1.}\PYG{n}{tar}\PYG{o}{.}\PYG{n}{gz}

\PYG{n}{This} \PYG{o+ow}{is} \PYG{n}{a} \PYG{n}{backwards}\PYG{o}{\PYGZhy{}}\PYG{n}{incompatible} \PYG{n}{change}\PYG{o}{.}  \PYG{n}{Applications} \PYG{n}{that} \PYG{n}{previously}
\PYG{n}{relied} \PYG{n}{on} \PYG{n}{a} \PYG{n}{blanket} \PYG{n}{exception} \PYG{k}{for} \PYG{n}{XMLHTTPRequest} \PYG{n}{may} \PYG{n}{need} \PYG{n}{to} \PYG{n}{be} \PYG{n}{modified}
\PYG{n}{to} \PYG{n}{explicitly} \PYG{n}{include} \PYG{n}{the} \PYG{n}{XSRF} \PYG{n}{token} \PYG{n}{when} \PYG{n}{making} \PYG{n}{ajax} \PYG{n}{requests}\PYG{o}{.}

\PYG{n}{The} \PYG{n}{tornado} \PYG{n}{chat} \PYG{n}{demo} \PYG{n}{application} \PYG{n}{demonstrates} \PYG{n}{one} \PYG{n}{way} \PYG{n}{of} \PYG{n}{adding} \PYG{n}{this}
\PYG{n}{token} \PYG{p}{(}\PYG{n}{specifically} \PYG{n}{the} \PYG{n}{function} \PYG{n}{postJSON} \PYG{o+ow}{in} \PYG{n}{demos}\PYG{o}{/}\PYG{n}{chat}\PYG{o}{/}\PYG{n}{static}\PYG{o}{/}\PYG{n}{chat}\PYG{o}{.}\PYG{n}{js}\PYG{p}{)}\PYG{o}{.}

\PYG{n}{More} \PYG{n}{information} \PYG{n}{about} \PYG{n}{this} \PYG{n}{change} \PYG{o+ow}{and} \PYG{n}{its} \PYG{n}{justification} \PYG{n}{can} \PYG{n}{be} \PYG{n}{found} \PYG{n}{at}
\PYG{n}{http}\PYG{p}{:}\PYG{o}{/}\PYG{o}{/}\PYG{n}{www}\PYG{o}{.}\PYG{n}{djangoproject}\PYG{o}{.}\PYG{n}{com}\PYG{o}{/}\PYG{n}{weblog}\PYG{o}{/}\PYG{l+m+mi}{2011}\PYG{o}{/}\PYG{n}{feb}\PYG{o}{/}\PYG{l+m+mi}{08}\PYG{o}{/}\PYG{n}{security}\PYG{o}{/}
\PYG{n}{http}\PYG{p}{:}\PYG{o}{/}\PYG{o}{/}\PYG{n}{weblog}\PYG{o}{.}\PYG{n}{rubyonrails}\PYG{o}{.}\PYG{n}{org}\PYG{o}{/}\PYG{l+m+mi}{2011}\PYG{o}{/}\PYG{l+m+mi}{2}\PYG{o}{/}\PYG{l+m+mi}{8}\PYG{o}{/}\PYG{n}{csrf}\PYG{o}{\PYGZhy{}}\PYG{n}{protection}\PYG{o}{\PYGZhy{}}\PYG{n}{bypass}\PYG{o}{\PYGZhy{}}\PYG{o+ow}{in}\PYG{o}{\PYGZhy{}}\PYG{n}{ruby}\PYG{o}{\PYGZhy{}}\PYG{n}{on}\PYG{o}{\PYGZhy{}}\PYG{n}{rails}
\end{sphinxVerbatim}


\subsection{What’s new in Tornado 1.1}
\label{\detokenize{releases/v1.1.0:what-s-new-in-tornado-1-1}}\label{\detokenize{releases/v1.1.0::doc}}

\subsubsection{Sep 7, 2010}
\label{\detokenize{releases/v1.1.0:sep-7-2010}}
\begin{sphinxVerbatim}[commandchars=\\\{\}]
We are pleased to announce the release of Tornado 1.1, available from
https://github.com/downloads/facebook/tornado/tornado\PYGZhy{}1.1.tar.gz

Changes in this release:
* RequestHandler.async\PYGZus{}callback and related functions in other classes
  are no longer needed in most cases (although it\PYGZsq{}s harmless to continue
  using them).  Uncaught exceptions will now cause the request to be closed
  even in a callback.  If you\PYGZsq{}re curious how this works, see the new
  tornado.stack\PYGZus{}context module.
* The new tornado.testing module contains support for unit testing
  asynchronous IOLoop\PYGZhy{}based code.
* AsyncHTTPClient has been rewritten (the new implementation was
  available as AsyncHTTPClient2 in Tornado 1.0; both names are
  supported for backwards compatibility).
* The tornado.auth module has had a number of updates, including support
  for OAuth 2.0 and the Facebook Graph API, and upgrading Twitter and
  Google support to OAuth 1.0a.
* The websocket module is back and supports the latest version (76) of the
  websocket protocol.  Note that this module\PYGZsq{}s interface is different
  from the websocket module that appeared in pre\PYGZhy{}1.0 versions of Tornado.
* New method RequestHandler.initialize() can be overridden in subclasses
  to simplify handling arguments from URLSpecs.  The sequence of methods
  called during initialization is documented at
  http://tornadoweb.org/documentation\PYGZsh{}overriding\PYGZhy{}requesthandler\PYGZhy{}methods
* get\PYGZus{}argument() and related methods now work on PUT requests in addition
  to POST.
* The httpclient module now supports HTTP proxies.
* When HTTPServer is run in SSL mode, the SSL handshake is now non\PYGZhy{}blocking.
* Many smaller bug fixes and documentation updates

Backwards\PYGZhy{}compatibility notes:
* While most users of Tornado should not have to deal with the stack\PYGZus{}context
  module directly, users of worker thread pools and similar constructs may
  need to use stack\PYGZus{}context.wrap and/or NullContext to avoid memory leaks.
* The new AsyncHTTPClient still works with libcurl version 7.16.x, but it
  performs better when both libcurl and pycurl are at least version 7.18.2.
* OAuth transactions started under previous versions of the auth module
  cannot be completed under the new module.  This applies only to the
  initial authorization process; once an authorized token is issued that
  token works with either version.

Many thanks to everyone who contributed patches, bug reports, and feedback
that went into this release!

\PYGZhy{}Ben
\end{sphinxVerbatim}


\subsection{What’s new in Tornado 1.0.1}
\label{\detokenize{releases/v1.0.1:what-s-new-in-tornado-1-0-1}}\label{\detokenize{releases/v1.0.1::doc}}

\subsubsection{Aug 13, 2010}
\label{\detokenize{releases/v1.0.1:aug-13-2010}}
\begin{sphinxVerbatim}[commandchars=\\\{\}]
\PYG{n}{This} \PYG{n}{release} \PYG{n}{fixes} \PYG{n}{a} \PYG{n}{bug} \PYG{k}{with} \PYG{n}{RequestHandler}\PYG{o}{.}\PYG{n}{get\PYGZus{}secure\PYGZus{}cookie}\PYG{p}{,} \PYG{n}{which} \PYG{n}{would}
\PYG{o+ow}{in} \PYG{n}{some} \PYG{n}{circumstances} \PYG{n}{allow} \PYG{n}{an} \PYG{n}{attacker} \PYG{n}{to} \PYG{n}{tamper} \PYG{k}{with} \PYG{n}{data} \PYG{n}{stored} \PYG{o+ow}{in} \PYG{n}{the}
\PYG{n}{cookie}\PYG{o}{.}
\end{sphinxVerbatim}


\subsection{What’s new in Tornado 1.0}
\label{\detokenize{releases/v1.0.0:what-s-new-in-tornado-1-0}}\label{\detokenize{releases/v1.0.0::doc}}

\subsubsection{July 22, 2010}
\label{\detokenize{releases/v1.0.0:july-22-2010}}
\begin{sphinxVerbatim}[commandchars=\\\{\}]
We are pleased to announce the release of Tornado 1.0, available
from
https://github.com/downloads/facebook/tornado/tornado\PYGZhy{}1.0.tar.gz.
There have been many changes since version 0.2; here are some of
the highlights:

New features:
* Improved support for running other WSGI applications in a
  Tornado server (tested with Django and CherryPy)
* Improved performance on Mac OS X and BSD (kqueue\PYGZhy{}based IOLoop),
  and experimental support for win32
* Rewritten AsyncHTTPClient available as
  tornado.httpclient.AsyncHTTPClient2 (this will become the
  default in a future release)
* Support for standard .mo files in addition to .csv in the locale
  module
* Pre\PYGZhy{}forking support for running multiple Tornado processes at
  once (see HTTPServer.start())
* SSL and gzip support in HTTPServer
* reverse\PYGZus{}url() function refers to urls from the Application
  config by name from templates and RequestHandlers
* RequestHandler.on\PYGZus{}connection\PYGZus{}close() callback is called when the
  client has closed the connection (subject to limitations of the
  underlying network stack, any proxies, etc)
* Static files can now be served somewhere other than /static/ via
  the static\PYGZus{}url\PYGZus{}prefix application setting
* URL regexes can now use named groups (\PYGZdq{}(?P\PYGZlt{}name\PYGZgt{})\PYGZdq{}) to pass
  arguments to get()/post() via keyword instead of position
* HTTP header dictionary\PYGZhy{}like objects now support multiple values
  for the same header via the get\PYGZus{}all() and add() methods.
* Several new options in the httpclient module, including
  prepare\PYGZus{}curl\PYGZus{}callback and header\PYGZus{}callback
* Improved logging configuration in tornado.options.
* UIModule.html\PYGZus{}body() can be used to return html to be inserted
  at the end of the document body.

Backwards\PYGZhy{}incompatible changes:
* RequestHandler.get\PYGZus{}error\PYGZus{}html() now receives the exception
  object as a keyword argument if the error was caused by an
  uncaught exception.
* Secure cookies are now more secure, but incompatible with
  cookies set by Tornado 0.2.  To read cookies set by older
  versions of Tornado, pass include\PYGZus{}name=False to
  RequestHandler.get\PYGZus{}secure\PYGZus{}cookie()
* Parameters passed to RequestHandler.get/post() by extraction
  from the path now have \PYGZpc{}\PYGZhy{}escapes decoded, for consistency with
  the processing that was already done with other query
  parameters.

Many thanks to everyone who contributed patches, bug reports, and
feedback that went into this release!

\PYGZhy{}Ben
\end{sphinxVerbatim}
\begin{itemize}
\item {} 
\DUrole{xref,std,std-ref}{genindex}

\item {} 
\DUrole{xref,std,std-ref}{modindex}

\item {} 
\DUrole{xref,std,std-ref}{search}

\end{itemize}


\chapter{Discussion and support}
\label{\detokenize{index:discussion-and-support}}
You can discuss Tornado on \sphinxhref{http://groups.google.com/group/python-tornado}{the Tornado developer mailing list}, and report bugs on
the \sphinxhref{https://github.com/tornadoweb/tornado/issues}{GitHub issue tracker}.  Links to additional
resources can be found on the \sphinxhref{https://github.com/tornadoweb/tornado/wiki/Links}{Tornado wiki}.  New releases are
announced on the \sphinxhref{http://groups.google.com/group/python-tornado-announce}{announcements mailing list}.

Tornado is available under
the \sphinxhref{http://www.apache.org/licenses/LICENSE-2.0.html}{Apache License, Version 2.0}.

This web site and all documentation is licensed under \sphinxhref{http://creativecommons.org/licenses/by/3.0/}{Creative
Commons 3.0}.


\renewcommand{\indexname}{Python Module Index}
\begin{sphinxtheindex}
\let\bigletter\sphinxstyleindexlettergroup
\bigletter{t}
\item\relax\sphinxstyleindexentry{tornado.auth}\sphinxstyleindexpageref{auth:\detokenize{module-tornado.auth}}
\item\relax\sphinxstyleindexentry{tornado.autoreload}\sphinxstyleindexpageref{autoreload:\detokenize{module-tornado.autoreload}}
\item\relax\sphinxstyleindexentry{tornado.concurrent}\sphinxstyleindexpageref{concurrent:\detokenize{module-tornado.concurrent}}
\item\relax\sphinxstyleindexentry{tornado.curl\_httpclient}\sphinxstyleindexpageref{httpclient:\detokenize{module-tornado.curl_httpclient}}
\item\relax\sphinxstyleindexentry{tornado.escape}\sphinxstyleindexpageref{escape:\detokenize{module-tornado.escape}}
\item\relax\sphinxstyleindexentry{tornado.gen}\sphinxstyleindexpageref{gen:\detokenize{module-tornado.gen}}
\item\relax\sphinxstyleindexentry{tornado.http1connection}\sphinxstyleindexpageref{http1connection:\detokenize{module-tornado.http1connection}}
\item\relax\sphinxstyleindexentry{tornado.httpclient}\sphinxstyleindexpageref{httpclient:\detokenize{module-tornado.httpclient}}
\item\relax\sphinxstyleindexentry{tornado.httpserver}\sphinxstyleindexpageref{httpserver:\detokenize{module-tornado.httpserver}}
\item\relax\sphinxstyleindexentry{tornado.httputil}\sphinxstyleindexpageref{httputil:\detokenize{module-tornado.httputil}}
\item\relax\sphinxstyleindexentry{tornado.ioloop}\sphinxstyleindexpageref{ioloop:\detokenize{module-tornado.ioloop}}
\item\relax\sphinxstyleindexentry{tornado.iostream}\sphinxstyleindexpageref{iostream:\detokenize{module-tornado.iostream}}
\item\relax\sphinxstyleindexentry{tornado.locale}\sphinxstyleindexpageref{locale:\detokenize{module-tornado.locale}}
\item\relax\sphinxstyleindexentry{tornado.locks}\sphinxstyleindexpageref{locks:\detokenize{module-tornado.locks}}
\item\relax\sphinxstyleindexentry{tornado.log}\sphinxstyleindexpageref{log:\detokenize{module-tornado.log}}
\item\relax\sphinxstyleindexentry{tornado.netutil}\sphinxstyleindexpageref{netutil:\detokenize{module-tornado.netutil}}
\item\relax\sphinxstyleindexentry{tornado.options}\sphinxstyleindexpageref{options:\detokenize{module-tornado.options}}
\item\relax\sphinxstyleindexentry{tornado.platform.asyncio}\sphinxstyleindexpageref{asyncio:\detokenize{module-tornado.platform.asyncio}}
\item\relax\sphinxstyleindexentry{tornado.platform.caresresolver}\sphinxstyleindexpageref{caresresolver:\detokenize{module-tornado.platform.caresresolver}}
\item\relax\sphinxstyleindexentry{tornado.platform.twisted}\sphinxstyleindexpageref{twisted:\detokenize{module-tornado.platform.twisted}}
\item\relax\sphinxstyleindexentry{tornado.process}\sphinxstyleindexpageref{process:\detokenize{module-tornado.process}}
\item\relax\sphinxstyleindexentry{tornado.queues}\sphinxstyleindexpageref{queues:\detokenize{module-tornado.queues}}
\item\relax\sphinxstyleindexentry{tornado.routing}\sphinxstyleindexpageref{routing:\detokenize{module-tornado.routing}}
\item\relax\sphinxstyleindexentry{tornado.simple\_httpclient}\sphinxstyleindexpageref{httpclient:\detokenize{module-tornado.simple_httpclient}}
\item\relax\sphinxstyleindexentry{tornado.tcpclient}\sphinxstyleindexpageref{tcpclient:\detokenize{module-tornado.tcpclient}}
\item\relax\sphinxstyleindexentry{tornado.tcpserver}\sphinxstyleindexpageref{tcpserver:\detokenize{module-tornado.tcpserver}}
\item\relax\sphinxstyleindexentry{tornado.template}\sphinxstyleindexpageref{template:\detokenize{module-tornado.template}}
\item\relax\sphinxstyleindexentry{tornado.testing}\sphinxstyleindexpageref{testing:\detokenize{module-tornado.testing}}
\item\relax\sphinxstyleindexentry{tornado.util}\sphinxstyleindexpageref{util:\detokenize{module-tornado.util}}
\item\relax\sphinxstyleindexentry{tornado.web}\sphinxstyleindexpageref{web:\detokenize{module-tornado.web}}
\item\relax\sphinxstyleindexentry{tornado.websocket}\sphinxstyleindexpageref{websocket:\detokenize{module-tornado.websocket}}
\item\relax\sphinxstyleindexentry{tornado.wsgi}\sphinxstyleindexpageref{wsgi:\detokenize{module-tornado.wsgi}}
\end{sphinxtheindex}

\renewcommand{\indexname}{Index}
\printindex
\end{document}